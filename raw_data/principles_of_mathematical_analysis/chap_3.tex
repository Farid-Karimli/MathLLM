3 NUMERICAL SEQUENCES AND SERIES As the title indicates, this chapter will deal primarily with sequences and series of complex numbers. The basic facts about convergence,however, are just as easily explained in a more general setting. The first three sections will therefore be concerned with sequences in euclidean spaces, or even in metric spaces. CONVERGENT SEQUENCES 3.1 Definition A sequence $\{p_{n}\}$ in a metric space X is said to converge if there N such that n ≥ N implies that is a point pe X with te following property: For every 8> 0 there is an intege (Here d denotes the distance in X. $d(p_{n},p)<s.$ In this case we also say that {p $\{p_{n}\}$ n} converges to $\textstyle{\mathcal{P}}$ 。or that $\textstyle{\mathcal{D}}$ is the limit of {p}[see Theorem 3.2(6)], and we write $p_{n}\to p,$ or $$ \operatorname*{lim}_{n\to\infty}p_{n}=p. $$ If{p,} does not converge, it is said to diverge.48 PRINCIPLES OF MATHEMATICAL ANALYSIs It might be well to point out that our definition of“convergent sequence' depends not only on{p} but also on X;for instance, the sequence {1/n} con- verges in $\textstyle{\mathcal{R}}^{1}$ (to 0),but fails to converge in the set of all positive real numbers [with $d(x,y)=\left|x-y\right|\}$ In cases of possible ambiguity,we can be more precise and specify “"convergent in X”rather than““convergent. We recall that the set of all points pn(n = 1,2,3,...)is the range of {p) The range of a sequence may bea finite set, or it may be infinite. The sequence $\{p_{n}\}$ is said to be bounded if its range is bounded. As examples, consider the following sequences of complex numbers (that is, X = R-): (a)If s, = 1/n, then lim-c。 S,= 0; the range is infinite, and the sequence is bounded (b) lf s, = n-, the sequence {s,) is unbounded,is divergent, and has infinite range. (c) lf s,= 1 +【(- 1)"/n], the sequence $\{s_{n}\}$ converges to 1,is bounded, and has infinite range (d) If s,= i", the sequence {s,} is divergent,is bounded, and has finite range. (e) If s,= 1 (n = 1,2,3,...),then {s,} converges to 1,is bounded, and has finite range. We now summarize some important properties of convergent sequences in metric spaces. 3.2 Theorem Let {p,} be a sequence in a metric space ${\mathcal{N}}$ (a){p} converges to p∈X if and only if every neighborhood of p contains PAfor all ut fniely many n (b) If p ∈ X, p’∈ X,and if $\{p_{n}\}$ converges to p and to p',then $p^{\prime}=p.$ (c)If{PA} converges, then {PA} is bounded. (d)If E c Xand if p is a limit point of $E,$ then there is a sequence $\{p_{n}\}$ in E such that p lim p Proof(a)Suppose p。→p and let V be a neighborhood of p. For someé>0,the conditions d(q,p)< E,q∈X imply q ∈ V. Correspond ing to this e, there exists $\mathcal{N}$ such that n≥ N implies d(p,P)<8. Thus ${\mathcal{N}}$ ≥ N implies p,∈ V. Conversely、 suppose every ncighborhood of p contains all but finitely many of the p,、Fix > 0, and let V be the set of all q∈ X such that $d(p,q)$ <8. By assumption,there exists $\mathcal{N}$ (corresponding to this V) such that $\textstyle{p_{n}}$ e V if n ≥ N.Thus $d(p_{n},p)<\varepsilon$ if n≥N; hence p。→P·NUMERICAL SEQUENCES AND SERIES 49 (b) Let é> 0 be given. There exist integers $N,\,N^{\prime}$ such that $$ \begin{array}{r l}{n\geq N}&{{}{\mathrm{implies}}\quad d(p_{n},p)<{\frac{\varepsilon}{2}},}\\ {n\geq N^{\prime}}&{{\mathrm{implies}}\quad d(p_{n},p^{\prime})<{\frac{\varepsilon}{2}}.}\end{array} $$ Hence if n ≥ max (V, $\left.N^{\prime}\right\}$ J’),we have $$ d(p,p^{\prime})\leq d(p,p_{n})+d(p_{n},p^{\prime})<\varepsilon. $$ Since s was arbitrary, we conclude that $d(p,\,p^{\prime})=0$ (c)Suppose p。→P. There is an integer $\mathcal{N}$ such that n > $\mathcal{N}$ implies $d(p_{n},p)<1.$ Put $$ r=\operatorname*{max}\left\{1,d(p_{1},p),\ldots,d(p_{N},p)\right\}\!_{\sim} $$ Then $d(p_{n},\,p)$ ≤r for $n=1,2,3,\ldots.$ (d)For each positive integer n,there is a point $\textstyle{\mathcal{P}}_{n}$ ∈ ${\widetilde{F}}^{\nu}$ such that $d(p_{n},p)<1/n.$ Givenc>0,choose N so that Ne > 1. If n > N,it follows that d(pm,p)<8. Hence p→p This completes the proof For sequences in $\textstyle{\mathcal{R}}^{k}$ we can study the relation between convergence,on the one hand,and the algebraic operations on the other. We first consider sequences of complex numbers. lim-。1, = 1. Then 3.3 Theorem Suppose {s,},1,} are complex sequences,and lim,-o。 5。= 8 (a)lim(S。 ,,))一s +1: (b) lim cs,= cs, lim (Gc + s,) = + s, for any number c; (c)lim s,1, = s*; (d) lim oS , provided s, 0 (n = 1,2,3,...), and s ≠ 0 Proof (a) Given B>0, there exist integers $\textstyle\bigwedge_{1},\,\bigwedge_{2}$ such that $$ \begin{array}{r l}{n\geq N_{1}\quad{\mathrm{implies}}\quad\vert s_{n}-s\vert<{\frac{s}{2}},}\\ {n\geq N_{2}\quad{\mathrm{implies}}\quad\vert t_{n}-t\vert<{\frac{s}{2}}.}\end{array} $$50 PRINCIPLES OF MATHEMATICAL ANALYsIs If N = max(N,,N,), then n ≥ $\mathcal{N}$ implies $$ \left|(s_{n}+t_{n})-(s+t)\right|\leq\left|s_{n}-s\right|\,+\,\left|\,t_{n}-t\right|\,<\varepsilon. $$ This proves (a). The proof of(b) is trivial (c)We use the identity (1) $$ s_{n}t_{n}-s t=(s_{n}-s)(t_{n}-t)+s(t_{n}-t)+t(s_{n}-s). $$ Given e> 0, there are integers $N_{1},$ Nz such that $$ \begin{array}{l l}{{n\geq N_{1}}}&{{\mathrm{Implies}}}&{{|s_{n}-s|<\backslash{\varepsilon},}}\\ {{n\geq N_{2}}}&{{\mathrm{implies}}}&{{|t_{n}-t|<\sqrt{\varepsilon}.}}\end{array} $$ lf we take N = max (N,N),n≥ N implies $$ \left|(s_{n}-s)(t_{n}-t)\right|<s, $$ so that $$ \operatorname*{lim}_{n\to\infty}(s_{n}-s)(t_{n}-t)=0. $$ We now apply (a) and (b) to (1), and conclude that $$ \operatorname*{lim}_{x\to\infty}(s_{n}t_{n}-s t)=0. $$ (d)Choosing m such that |s。一s<士is| if n≥ m,we see that $$ |s_{n}|>\frac{1}{2}|s|\qquad(n\ge m). $$ Given B> 0, there is an integer $N>m$ such that n≥ $\mathcal{N}$ implies $$ \left|s_{n}-s\right|<\textstyle{\frac{1}{2}}\left|s\right|^{2}s. $$ Hence,for n ≥ N $$ \left|\frac{1}{s_{n}}-\frac{1}{s}\right|=\left|\frac{s_{n}-s}{s_{n}s}\right|<\frac{2}{|s|^{2}}\left|s_{n}-s\right|<s. $$ 3.4 Theorem (a)Suppose $\mathbf{X}_{n}$ e R' (n =1,2,3,...)and $$ {\bf x}_{n}=(\mathcal{A}_{1,n},\star,\alpha_{k,n}). $$ Then {x} converges to x=(α,.…xA)if and only i (2) $$ \operatorname*{lim}_{n\to\infty}\alpha_{J,n}=\alpha_{J}\qquad(1\leq j\leq k). $$NUMERICAL sEOUENCEs AND SERIES51 (b)Suppose {x,)},{y,} are sequences in R*,{B,} is a sequence of real numbers and x,→X,Yn→y,β。→β.Then lim (x。 + y,)=X+ y, D→C limx。》。=x·y, lim βAx,= Bx 1→文 Proof (a)If x。→x, the inequalities $$ |\alpha_{J,n}-\alpha_{J}|\leq|{\bf x}_{n}-{\bf x}|\,, $$ which follow immediately from the definition of the norm in R', show that (2) holds Conversely, if(2) holds,then to each &>0 there corresponds an integer $\mathcal{N}$ such that n ≥ $\mathcal{N}$ V implies $$ |\alpha_{j,n}-\alpha_{j}|<\frac{s}{\sqrt{k}}\qquad(1\leq j\leq k). $$ Hence n ≥ $\mathcal{N}$ implies $$ \left|{\bf x}_{n}-{\bf x}\right|=\left\{\sum_{j=1}^{k}\left|\alpha_{j,n}-\alpha_{j}\right|^{2}\right>^{1/2}<\varepsilon $$ so that $\mathbf{X}_{n}$ →x、This proves (a) Part (b) follows from (a) and Theorem 3.3. SUBSEQUENCES 3.5 Definition Given a scquenc fp,) cosider a sequence n,} of positiv of fP} integers,such that $n_{1}<n_{2}<n_{3}$ <….Then the sequence {pm} is called a subsequence of {pA}. If {p} converges, its limit is called a subsequential limit t is clear that {p,} converges to $\textstyle{\mathcal{D}}$ if and only if every subsequence of {p} converges to p.We leave the details of the proof to the reader. 3.6 Theorem (a)f{Pm}is a sequence in a compact metric space $X_{\mathrm{{J}}}$ then some sub- sequence of{p,} converges to a point of X. (b)Every bounded sequence in $\textstyle{\mathcal{H}}^{k}$ contains a convergent subsequence.52 PRINCIPLEs Or MATHEMATICAL ANALYs Proof (a)Let E be the range of {ph}. If Eis fnite then there is a pe E and a sequence {n;} with n,<n,<ng <…,such that $$ p_{n_{i}}=p_{n_{2}}=\cdot\cdot\cdot=p. $$ If The subsequence fpm} so obtained converges evidently to p. Choose ${\widehat{P}}^{\nu}$ is infinite, Theorem 2.37 shows that E has a limit point pe X ${\mathcal{n}}_{1}$ so that d(p,P,)<1. Having chosen n,.…,1-., we see from Then (Pn} converges to p. Theorem 2.20 that there is an integer n,>n- Such that d(p,P,)< 1/1 (の)This follows from ao), since Theorem 2.41 implies that every bounded subset of $\textstyle{\mathcal{K}}$ lies in a compact subset of Rh. form a closed subset of X. 3. Theorem The subsequential limits of a sequence {pm} in a metric space X Proof Let E* be the set of all subsequential limits of {p,}and let g be a limit point of $E^{\bullet}$ We have to show that qe E*. Choose n, so that pm q.(If no such n exists, then $E^{\bullet}$ has only one point, and there is nothing to prove.)Put $\delta=d(q,p_{n_{1}}).$ Suppose n,...,n;-1 are chosen. Since q is a limit point of E*,there is an xe E· with dx,4)<2-'8.Since xé E*、there is an n,> n-1such that d(x, pn)< 2*18. Thus $$ d(q,p_{n})\leq2^{1-i}\delta $$ for i= 1,2,3, .…… This says that {p,} converges to q、Hence q e E*. CAUCHY SEQUENCES 3.8 Defnitin A sequence {p,} in a metric space X is said to be a Cauchy such that d(PA,P><sifn ≥ $\textstyle{\mathcal{N}}$ sequence if for every e > O there is an integer $\textstyle{\mathcal{N}}$ and m ≥ N. In our discussion of Cauchy sequences,as well as in other situations which will arise later, the following geometric concept willbe useful of Sis called the diameter of E. 3.9 Defnition Let E be a nonempty subset of a metric space X, and let S be and qe E. The sup the set o al real numbers of the form dp.q) with pe $\textstyle{\overline{{f}}}^{-1}$NUMERICAL sEOUENCES AND SERIES 53 if ad only i If{phis a sequence in $\lambda$ and if $E_{N}$ consists of the points px,Px+1,Pv+2, it is clear from the two preceding definitions that (pn,} is a Cauchy sequence N→ lim diam E. = 0. 3.10 Theorem (a)If E is the closure of a set $\widetilde{{\cal H}^{\nu}}$ in a metric space X, then $$ \mathrm{diam}\ \bar{E}=\mathrm{diam}\ E. $$ (b)If $K_{n}$ is a sequence of compact sets in X such that $K_{n}\supset K_{n+1}$ (n = 1,2,3,...) and ij $$ \operatorname*{lim}_{n\to\infty}\dim K_{n}=0, $$ then [YK, consists of exactly one point Proof (a)Since $E\subset E,$ ,it is clear that $$ \dim E\leq\dim E. $$ Fix $\varepsilon>0.$ and choose $p\in{\overrightarrow{E}},\,q\in{\overrightarrow{E}}.$ 、By the definition of E,there are points p',o',in $\textstyle{\mathcal{F}}$ such that $d(p,p^{\prime})<\varepsilon,\,d(q,q^{\prime})<\varepsilon$ ,Hence $$ d(p,q)\leq d(p,p^{\prime})+d(p^{\prime}\,q^{\prime})+d(q^{\prime},q) $$ < 26 + d(p',q)≤ 28 + diam E lt follows that dian $$ \mathrm{\boldmath~\nabla~}\bar{E}\leq2\varepsilon+\mathrm{diam}\;E, $$ and since e was arbitrary,(a is proved. (b)Put K = 0°K,By Theorem 2.36,K is not empty. If $K$ contains more than one point, then diam K>0. But for each n, $K_{n}\supseteq K,$ so that diam $K_{n}$ ≥ diam ${\bar{\mathcal{F}}}$ This contradicts the assumption that diam $K_{n} arrow0.$ 3.11 Theorem (a) In any metric space X, every convergent sequence is a Cauchy sequence (b) If X is a compact metric space and $i f\left\{p_{n}\right\}$ is a Cauchy sequence in X then {p} converges to some point of $X.$ (c) In $R^{k}{}_{;}$ ,every Cauchy sequence converges Note: The difference between the definition of convergence and the definition of a Cauchy sequence is that the limit is explicitly involved in the former, but not in the latter. Thus Theorem 3.11(b) may enable us54 PRINCIPLES OF MATHEMATICAL ANALYsIS to decide whether or not a given sequence converges without knowledgc of the imit to which it may converge. The fact (contained in Theorem 3.11) that a sequence converges in R' if and only if it is a Cauchy sequence is usually called the Cauchy criterion for convergence. Proof (a)f p,→p and if e > 0, there is an integer ${\mathcal{N}}$ such that d(p,Pp)<8 for all n ≥ N. Hence $$ d(p_{n},p_{m})\leq d(p_{n},p)+d(p,p_{m})<2s $$ as soon as n > $\mathcal{N}$ and m ≥ N. Thus{phis a Cauchy sequence. (b) Let {p,} be a Cauchy sequence in the compact space X. For $N=1$ 2,3,.…., let $E_{N}$ be the set consisting of pv,Pv+1, PN+2, Then (3) $$ \operatorname*{lim}_{N arrow\infty}\mathrm{diam}\ \tilde{E}_{N}=0, $$ by Definition 3.9 and Theorem 3.10(a). Being a closed subset of the compact space X, each $E_{N}$ is compact (Theorem 2.35).Also $E_{N}\supset E_{N+1},$ so that ${\tilde{E}}_{N}\supset{\tilde{E}}_{N+1}$ Theorem 3.10(b) shows now that there is a unique pe X which lies in every ${\tilde{E}}_{N}$ Let > 0 be given. By(3)there is an integer $N_{0}$ such that diam I ${\bar{E}}_{N}$ <e if N≥ No. Since p E $:{\vec{E}}_{N}$ ,it follows that $d(p,q)<$ B for n ≥ N。. This says precisely that every qe E,, hence for every qe EN. In other words, d(P, Pn)<eif $p_{n}\to p.$ (c)Let {x,}) be a Cauchy sequence in R". Define ${\mathcal{E}}_{N}$ as in(b),with x of in place of $\boldsymbol{P_{i}}\,.$ For some N, diam $E_{n}<1$ The range of {x,} is the union $E_{N}$ every bounded subset of $\textstyle{\mathcal{R}}^{k}$ and the finite set {x,………Xx-n}. Hence ({x,} is bounded.Sinc (Theorem 2.41) has compact closure in $\textstyle{\mathcal{R}}^{k}$ (c) follows from (b). 3.12 Definition A metric space in which every Cauchy sequence converges is said to be complete. Thus Theorem 3.11 says that all compact metric spaces and all Euclidean spaces are complete. Theorem 3.11 implies also that every closed subset E of a complete metric space X is complete.(Every Cauchy sequence in $\widehat{\cal H}^{\nu}$ is a Cauchy sequence in X,hence it converges to some $\textstyle{\mathcal{D}}$ e X, and actually p e E since ${\mathcal{F}}^{\dagger}$ is closed.)An example of a metric space which is not complete is the space of al rational numbers, with $d(x,y)=\ |x-y|$NUMERICAL SEOUENCEs AND SERIEs S5 Theorem 3.2(c) and example(d) of Definition 3.1 show that convergen sequences are bounded, but that bounded sequences in $\textstyle{\mathcal{R}}^{k}$ need not converge. However, there is one important case in which convergence is equivalent to boundedness; this happens for monotonic sequences in $R^{1}$ 3.13 Definition A sequence {sA} of real numbers is said to be (o)monotonically increasing if ${\mathfrak{P}}_{j i}$ ≤ S%+1(n = 1,2,3,...) (b)monotonically decreasing if $s_{n}\geq s_{n+1}$ (n = 1,2,3,.….) The class of monotonic sequences consists of the increasing and the decreasing sequences. 3.14 Theorem Suppose Ss,} is monotonic. Then {s;)} conerges if and only ifi is bounded. Let Proot Suppose s,≤ SA+1(theproof is analogous in the other case 上 be the range of {s,).、I {s,} is bounded,let s be the least upper ${\mathcal{H}}^{\vee}$ bound of E. Then $$ s_{n}\leq s\qquad(n=1,2,\,3,\,\ldots). $$ For every 8 > 0, there is an integer $\mathcal{N}$ such that $$ s\,-\,\varepsilon\,<s_{N}\leq s_{\ \/} $$ for otherwise s-e would be an upper bound of E. Since {s,} increases, ${\mathcal{N}}$ ≥ N therefore implies $$ s-s<s_{n}\leq s, $$ which shows that {s,} converges (to s) The converse follows from Theorem 3.2(c) UPPER AND LOWER LIMITS 3.15 Defnitio Let Ss,} be a secquence of real numbers with the follown property: For every real $\dot{M}$ there is an integer ${\mathcal{N}}$ such that n ≥ N implies s,≥ M. We then write $$ s_{n}\to+\infty. $$ Similarly,if for every real M there is an integer $\textstyle{\mathcal{N}}$ such that $n\geq N$ implies S,≤ M, we write $$ s_{n}\to-\infty. $$56 raINCIPLES OF MATHEMATICAL ANALYSis It should be noted that we now use the symbol→(introduced in Defini tion 3.1)for certain types of divergent sequences,as well as for convergent sequences, but that the definitions of convergence and of limit, given in Defini- tion 3.1, are in no way changed. 3.16 Definition Let {s,} be a sequence of real numbers. Let E be the set of numbers x(in the extended real number system) such that S,→x for some subsequence Ss,}. This set E contains all subsequential limits as defined in Definition 3.5, plus possibly the numbers +OO,-O0 We now recall Definitions 1.8 and 1.23 and put $$ \begin{array}{l c}{{s^{*}=\sinh{E},}}\\ {{s_{*}=\operatorname*{inf}{E}.}}\end{array} $$ The numbers s*,s are called the upper and lower limits of {s,}; we use the notation $$ \operatorname*{lim}_{n arrow\infty}\operatorname*{sup}_{n arrow\infty}s_{n}=s^{\ast_{i}}\qquad\operatorname*{lim}_{n arrow\infty}\operatorname*{lim}_{n arrow\infty}\operatorname*{lim}_{n arrow\infty}\varepsilon_{n}=s_{\ast}. $$ 3.17 Theorem Let {s,} be a sequence of real numbers. Let E and s* have th same meaning as in Definition 3.16. Then $S^{\mathbb{N}}$ has the following two properties: (a)s* ∈ E. (b)If x> 8*,there is an integer $\mathcal{N}$ such that n ≥ N implies $S_{n}<x$ Moreover,s* is the only number with the properties (a) and (b) Of course, an analogous result is true for sa Proof (a)If s* = +00, then ${\mathcal{H}}^{\nu}$ is not bounded above; hence s,}is not bounded above, and there is a subsequence $\left\{S_{n_{k}}\right\}$ such that $s_{n_{k}} arrow+$ 。 If s* is real, then Eis bounded above, and at least one subsequentia limit exists,so that (a) follows from Theorems 3.7 and 2.28 If s*= =- OO,then $\widehat{\cal H}^{\nu}$ contains only one element, namely -00,and there is no subsequential limit. Hence,for any real M,sn> M for at most a finite number of values of n,so that $s_{n}\to-\infty$ This establishes (a) in all cases. (b)Suppose there is a number x> s* such that s,≥x for infinitel such that many values of n. In that case,there is a number y ∈ ${\widetilde{H}}$ P ≥x> 8*, contradicting the definition of s*. Thus s* satisfies (a) and (b). To show the uniqueness, sppose there are two numbers, p and g which satisfy (a) and (b), and suppose p <q. Choose x such that p <x<4 Since p satisfies (b), we have s,<xfor n ≥ N.But then q cannot satisfy (a)NUMEUICAL SEQUENCEs AND SERuIsS7 3.18Examples (a)Let {s,} be a sequence containing all rationals. Then every real number is a subsequential limit, and lim sup s.= +00, lim inf s,= -OO. 1→o (b) Let s,=(-1")/[1 + (1/n)]. Then n→α lim sup s. = 1, lim inf s,= -1 n→c (c)For a real-valued sequencef{s,}, lim s, = sif and only if 1→α lim sup s. = im inf sm= s n→ We close this section with a theorem which is useful, and whose proof is quite trivial: 3.19 Theorem If s,≤t,for n ≥ N,where N is fixed,then $$ \begin{array}{l}{{\operatorname*{lim}_{n arrow\infty}\operatorname*{inf}_{n}f_{n},}}\\ {{\operatorname*{lim}_{n arrow\infty}}}\\ {{\operatorname*{lim}_{n arrow\infty}\mathbf{p}_{n}\le\operatorname*{lim}_{n arrow\infty}\mathbf{p}_{n}.}}\end{array} $$ SOME SPECIAL SEQUENCES where N is some fixed number, and if We shall now compute the limits of some sequences which occur frequenty then $x_{n}\to0$ The proofs wll ll b based on the following remark:If O≤x,≤8,for n ≥ N $s_{n}-0,$ 3.20 Theorem (a)f p > 0,then lim i nP =0 (b)Ip> 0, then lim "/p =1 (c)lim 、/n =1 1→α (d)If p > 0 and α is real, then lim n = 0. 。(1 + p) (e)If|x|<1,then lim x" = 0. n→c58 PRINCIPLES OF MATHEMATICAL ANALYSiS Proof (a)Take n >((1/e)1/".(Note that the archimedean property of the real number system is used here.) theorem, (b)If p>1, put x,= */p-1. Then x。> 0,and,by the binomial $$ 1+n x_{n}\leq(1+x_{n})^{n}=p, $$ so that $$ 0<x_{n}\leq{\frac{p-1}{n}}. $$ Hence X。→0.Ifp = 1,(b)is trivial, and if $0<p<1,$ the result is obtained by taking reciprocals. (c)Put $x_{n}={\frac{n}{\sqrt{n}}}-1.$ .Then X。≥0, and, by the binomial theorem $$ n=(1+x_{n})^{n}\geq{\frac{n(n-1)}{2}}x_{n}^{2}. $$ Hence $$ 0\leq x_{n}\leq{\sqrt{\frac{2}{n-1}}}\quad\quad(n\geq2). $$ (d) Let ${\big.}{\big/}{\big/}$ be an integer such tha $t~k>\alpha,k>0$ . For n> 2k, $$ (1+p)^{n}>(_{k}^{n})\,p^{k}={\frac{n(n-1)\cdot\cdot\cdot\cdot(n-k+1)}{k!}}p^{k}>{\frac{n^{k}p^{k}}{2^{k}k!}}. $$ Hence $$ 0<\frac{n^{\alpha}}{(1+p)^{n}}<\frac{2^{k}k!}{p^{k}}n^{\alpha-k}\qquad(n>2k). $$ Since $x-k<0,\,n^{x-k}\to0,\;\mathrm{by}\;(a).$ (e)Take α = 0 in (d) SERIES In the remainder of this chapter, all sequences and series under consideration will be complex-valued, unless the contrary is explicitly stated. Extensions of some of the theorems which follow,to series with terms in $R^{k},$ are mentioned in Exercise 15.NUMERICAL SEQUENCEs AND SERIEs59 3.21 Defnition Given a sequence {an}, we use the notation $$ \sum_{n=p}^{q}a_{n}\qquad(p\leq q) $$ to denote the sum $$ a_{p}+a_{p+1}+\cdot\cdot\cdot+a_{q}\,.\quad\mathrm{With~\{}a_{n}\}~\mathrm{v} $$ ve associate a sequence {S,}, where $$ s_{n}={\frac{a}{k-1}}a_{k}. $$ For {s,} we also use the symbolic expression $$ a_{1}+a_{2}+a_{3}+\cdots $$ or, more concisely (4) $$ {\frac{x}{n-1}}a. $$ The symbol(4) we call an infinite series,or just a series. The numbers s。 are called the partial sums of the series. If {s,} converges to s, we say that the series converges, and write $$ {\textstyle\frac{\omega}{n-1}}a_{n}=s. $$ The number s is called the sum of the series;but it should be clearly under- stood that sis the limit of a sequence of sums,and is not obtained simply by addition. If s,} diverges, the series is said to diverge. Sometimes,for convenience of notation, we shall consider series of the form (5) $$ {\frac{\nabla}{a_{1}}}a_{n}. $$ And frequently, when there is no possible ambiguity,or when the distinction is immaterial, we shall simply write Za,in place of(4) or(5) It is clear that every theorem about sequences can be stated in terms of series (putting a = s,and a,= S,一 Sm-1 for n >1), and vice versa. But it is nevertheless useful to consider both concepts form: The Cauchy criterion (Theorem 3.11) can be restated in the following 3.22 Theorem $\textstyle\mathbf{Z}a_{n}$ converges if and only if for every 8 >0 there is an integer N such that (6) $$ \left|z_{k-a}^{\quad\alpha}a_{k}\right|\leq\varepsilon $$ if m ≥n≥ N60 PRINCIPLES OF MATHEMATICAL ANALYsIs In particular, by taking m = n,(6) becomes $$ |a_{n}|\leq\varepsilon\qquad(n\geq N). $$ In other words: 3.23 Theorem If Za,。 converges, then lim-o。 ,= 0 The condition α。→0 is not, however,sufficient to ensure convergence of Za,、For instance, the series $$ \sum_{n=1}^{\infty}{\frac{1}{n}} $$ diverges; for the proof we refer to Theorem 3.28 Theorem 3.14,concerning monotonic sequences,also has an immediate counterpart for series. 3.24 Theorem A series of nonnegativel terms converges if and onlyif its partial sums form a bounded sequence. We now turn to a convergence test of a different nature, the so-called “comparison test.” 3.25 Theorem (a) ${\mathcal{L}}{\mathcal{I}}$ rlanl≤ c, for n ≥ No,where N。is some fixed integer, and if $\textstyle{\mathbf{2}}{\mathcal{C}}_{n}$ converges, then $\textstyle{\mathbf{Z}}a_{n}$ converges. (b)If a,≥ d,≥0 for n ≥ No,and if Ed,diverges, then Ea,diverges. Note that (b) applies only to series of nonnegative terms $Q_{n}$ ProofGiven e> 0, there exists $\mathcal{N}$ ≥ $N_{0}$ such that m>n > N implies $$ {\frac{\pi}{\lambda_{\mathrm{s}}}}\,c_{k}\leq\varepsilon, $$ by the Cauchy criterion. Hence $$ \left|\sum_{k=n}^{m}a_{k}\right|\leq\sum_{k=n}^{m}|a_{k}|\leq\sum_{k=n}^{m}c_{k}\leq6, $$ and (a) follows Next,(b)follows from (a), for if Ea,converges,so must Ed,[note that (b) also follows from Theorem 3.24]. 1 The expression“nonnegatie" always refers to real numbers.NuMERICAL SEQUENCES AND SERus 6 The comparison test is a very useful one; to use it efficiently, we have to become familiar with a number of series of nonnegative terms whose conver- gence or divergence is known. SERIES OF NONNEGATIVE TERMS The simplest of all is perhaps the geometric series. 3.26 Theorem If 0 ≤x<1,ther $$ \sum_{n=0}^{\infty}x^{n}={\frac{1}{1-x}}. $$ Ifx≥1,the series diverges Proof Ifx≠ 1, $$ s_{n}=\sum_{k=0}^{n}\,x^{k}=\frac{1-x^{n+1}}{1-x}\,. $$ The result follows if we let n→OO. For $\scriptstyle x\;=\;1.$ we get $$ \vdash\ \vdash\ \vdash\ \vdash\ \cdot\cdot\cdot\ , $$ which evidently diverges. In many cases which occur in applications, the terms of the series decrease monotonically.、 The following theorem of Cauchy is therefore of particular interest. The striking feature of the theorem is that a rather “thin’subsequence of $\{a_{n}\}$ determines the convergence or divergence of Ea, verges if and only if the series 3.7 heorem Suppose a a,≥4 2 ≥ 0. Then he series E.-n a。con (7) $$ \sum_{k=0}^{\infty}\,2^{k}a_{2k}=a_{1}+2a_{2}+4a_{4}+8a_{8}+\cdot\cdot\cdot $$ converges. Proof By Theorem 3.24,,it suffices to consider boundedness of the partial sums. Let $$ \begin{array}{l}{{s_{n}=a_{1}+a_{2}+\mathbf{\nabla}\cdot\mathbf{\nabla}+a_{n}\,,}}\\ {{t_{k}=a_{1}+2a_{2}+\mathbf{\nabla}\cdot\mathbf{\nabla}\cdot\mathbf{\nabla}+2^{k}a_{2k}\,.}}\end{array} $$62 PRINCIPLES OF MATHEMATICAL ANALYsIs For n <2*, $$ \begin{array}{c}{{s_{n}\leq a_{1}+(a_{2}+a_{3})+\cdots+(a_{2k}+\cdots+a_{2k+1-1})}}\\ {{\leq a_{1}+2a_{2}+\cdots+2^{k}a_{2k}}}\\ {{=t_{k},}}\end{array} $$ so that (8) S,≤ tk. On the ot $$ \begin{array}{l}{{\mathrm{her~hand,~if~}n>2^{k},}}\\ {{s_{n}\geq a_{1}+a_{2}+(a_{3}+a_{4})+\cdots+(a_{2k-1}+1+\cdots+a_{2k},}}\\ {{\geq\frac{1}{2}a_{1}+a_{2}+2a_{4}+\cdots+2^{k-1}a_{2k}}}\\ {{=\frac{1}{2}t_{k},}}\end{array} $$ so that (9) $$ 2S_{n}\geq t_{k}\,. $$ By(8)and(9), the sequences $\{S_{n}\}$ and {tp} are either both bounded or both unbounded. This completes the proof 3.28 Theorem 1 conerges if p >1 and diverges i $f p\leq1$ Proof 1f $p\leq0,$ divergence follows from Theorem 3.23. If $\scriptstyle p\;>\;0$ Theorem 3.27 is applicable, and we are led to the series $$ \sum_{k=0}^{\infty}2^{k}\cdot{\frac{1}{2^{k}p}}\,=\,\sum_{k=0}^{\infty}2^{(1-p)k}. $$ Now,21- <1 if and only if $1-p<0,$ , and the result follows by com- parison with the geometric series (take $x=2^{1-p}$ in Theorem 3.26) As a further application of Theorem 3.27, we prove: 3.29 Theorem $\;U P>1,$ (10) $$ \begin{array}{l l}{{\overset{\alpha}{n}}}&{{1}}\\ {{n=2\,n(\log n)^{p}}}\end{array} $$ converges; if p ≤1,the series diverges Remark““log n”denotes the logarithm of n to the base e(compare Exercise T Chap. 1); the number e will be defined in a moment (see Definition 3.30). We let the series start with n = 2, since log $\mathbf{\tau}_{1}=0$NUMERICAL sEoUENCES AND SERIEs 63 Proof The monotonicity of the logarithmic function(which will be discussed in more detail in Chap.8) implies that {log n} increases. Hence {1/n log ny decreases, and we can apply Theorem 3.27 to(10);this leads us to the series (11) $$ \sum_{k=1}^{\infty}2^{k}\cdot\frac{1}{2^{k}(\log2^{k})^{p}}=\sum_{k=1}^{\infty}\frac{1}{(k\;\log2)^{p}}=\frac{1}{(\log2)^{p}}\sum_{k=1}^{\infty}\frac{1}{k^{p}}, $$ and Theorem 3.29 follows from Theorem 3.28. This procedure may evidently be continued. For instance, (12) $$ \sum_{n=3}^{\infty}{\frac{1}{n\log n\log\log n}} $$ diverges,whereas (13) $$ \begin{array}{l l}{{\overset{\textrm{a}}{n=}}}&{{\frac{1}{n\log n(\log\log n)^{2}}}}\end{array} $$ converges. We may now observe that the terms of the series (12) differ very little from those of (13). Still, one diverges,the other converges. If we continue the process which led us from Theorem 3.28 to Theorem 3.29,and then to (12) and (13),we get pairs of convergent and divergent series whose terms differ even less than those of(12) and(13). One might thus be led to the conjecture that there is a limiting situation of some sort, a“boundary”with all convergent series on one side, all divergent series on the other side -at least as far as series with monotonic coefficients are concerned. This notion of “boundary”is of course quite vague. The point we wish to make is this: No matter how we make this notion precise, the conjecture is false.Exercises 11(b) and 12(b) may serve as illustrations. We do not wish to go any deeper into this aspect of convergence theory and refer the reader to Knopp's “Theory and Application of Infinite Series,’ Chap. IX, particularly Sec. 41. THE NUM BER e 3.30 Definitione 31 ,=o n! Here n!=1·2 3 …n ifn≥l, and 0!=164 PRINCIPLES OF MATHEMATICAL ANALYSIS Since $$ \begin{array}{c}{{s_{n}=1+1+\displaystyle{\frac{1}{1\cdot2}}+\displaystyle{\frac{1}{1\cdot2\cdot3}}+\cdots+\displaystyle{\frac{1}{1\cdot2\cdots n}}}\\ {{\mathrm{~}}}\\ {{<1+1+\displaystyle{\frac{1}{2}}+\displaystyle{\frac{1}{2^{2}}}+\cdots+\displaystyle{\frac{1}{2^{n-1}}}<3,}}\end{array} $$ the series converges, and the definition makes sense. In fact, the series converges very rapidly and allows us to compute e with great accuracy. It is of interest to note that e can also be defined by means of another limit process; the proof provides a good illustration of operations with limits: 3.31 Theorem lim 1 Proof Let $$ s_{n}=\sum_{k=0}^{n}{\frac{1}{k!}},\qquad t_{n}=\left(1+{\frac{1}{n}}\right)^{n}. $$ By the binomial theorem $$ t_{n}=1+1+{\frac{1}{2!}}\left(1-{\frac{1}{n}}\right)+{\frac{1}{3!}}\left(1-{\frac{1}{n}}\right)\left(1-{\frac{2}{n}}\right)+\cdots $$ $$ +\,{\frac{1}{n!}}\left(1-{\frac{1}{n}}\right)\left(1-{\frac{2}{n}}\right)\cdot\cdot\cdot\left(1-{\frac{n-1}{n}}\right) $$ Hence $t_{n}\leq s_{n},$ so that (14) lim sup t。≤e by Theorem 3.19. Next, if n≥m, $$ t_{n}\ge1+1+\frac{1}{2^{2}}\biggl(1-\frac{1}{n}\biggr)+\cdot\cdot\cdot+\frac{1}{m!}\biggl(1-\frac{1}{n}\biggr)\cdot\cdot\cdot\biggl(1-\frac{m-1}{n}\biggr). $$ Let n→OO, keeping m fixed. We get $$ \operatorname*{lim}_{n\to\infty}\operatorname{in}_{m}\mathbf{f}_{n}\geq1+1+{\frac{1}{2!}}+\cdots+{\frac{1}{m!}}, $$ so that $$ s_{m}\leq\operatorname*{lim}_{n\to\infty}\operatorname{inf}_{n}t_{n}\,. $$ Lettin $J/\mathcal{I}\;\longrightarrow\;\Omega S_{\mathfrak{I}}$ we finally get (15) $$ e\leq\operatorname*{lim}_{n\to\infty}\operatorname{inf}_{n}. $$ The theorem follows from (14) and(15)NUMERICAL SEOUENCES AND SERIs 65 The rapidity with which the series $\textstyle\sum{\frac{1}{n}}$ converges can be estimated as follows: If $\bar{\bigcup_{p,p}^{}}$ has the same meaning as above, we have $$ e-s_{n}=\frac{1}{(n+1)!}+\frac{1}{(n+2)!}+\frac{1}{(n+3)!}+\cdots $$ so that $$ <\frac{1}{(n+1)!}\biggl\{1+\frac{1}{n+1}+\frac{1}{(n+1)^{2}}+\cdots\biggr\}=\frac{1}{n!n} $$ (16) $$ 0<e-s_{n}<\frac{1}{n!n}. $$ Thus Sio,for instance, approximates e with an err less than 10-7. The inequality (16) is of theoretical interest as well, since it enables us to prove the irrationality of e very easily 3.32 Theorem e is irrational. Proof Suppose e is rational. Then $e=p/q,$ where p and q are positive integers. By (16), (17) $$ 0<q!(e-s_{q})<\frac{1}{q}. $$ By our assumption, q!e is an integer. Since $$ q!s_{q}=q!\biggl(1+1+\frac{1}{2!}+\cdots+\frac{1}{q!}\biggr) $$ is an integer, we see that q!(e一s,)is an integer. Sinceq≥1,(17) implies the existence of an integer between O and 1 We have thus reached a contradiction. Actually, e is not even an algebraic number. For a simple proof of this see page 25 of Niven's book, or page 176 of Herstein's, cited in the Bibliography THE ROOT AND RATIO TESTS 3.33 Theorem (Root Test) Given $\Sigma a_{n},$ put α = lim sup "/ Ta, Then (b) if α>1, if α <1,Za, converges; diverges; a) $\textstyle{\bar{\mathbb{Z}}}a_{n}$ (c) if α = 1, the test gives no information.66 PaINCIPLEs Or MATHEMATICAL ANALYSsis Proof If α<1, we can choose β so that α<β<1, and an integer $\mathcal{N}$ such that $$ \sqrt{|a_{n}|}<\beta $$ for n ≥ N [by Theorem 3.17(6)] That is, n ≥ N implies $$ \left\vert\alpha_{n}\right\vert<\beta^{n}. $$ Since 0<β<1,工β" converges. Convergence of Za,follows now from the comparison test If α > 1,then, again by Theorem 3.17, there is a sequence {n,} sucl that $$ \stackrel{n a}{\sqrt{}}|\stackrel{n_{n a}}{a_{n a}}| arrow\alpha. $$ Hence a,|>1 for infinitely many values of n,so that the condition α。→ 0, necessary for convergence of Ea $\Sigma a_{n}\,,$ a,, does not hold (Theorem 3.23) To prove (c), we consider the series $$ \Sigma_{n}^{1},\;\Sigma_{n^{-}}^{1}. $$ For each of these series x = 1, but the first diverges, the second converges 3.34 Theorem (Ratio Trest)The series $\Sigma a_{n}$ (a)converges if lim sup (b)diverges ≥1 for all n≥no,where $\mathbb{Z}$ is some fixed integer that Proot f condition (a) holds, we can find $\beta<1,$ and an integer $\bar{N},$ such $$ \left|{\frac{a_{n+1}}{a_{n}}}\right|<\beta $$ for n ≥ N. In particula, $$ \begin{array}{l}{{|a_{N+1}|<\beta|a_{N}|,}}\\ {{|a_{N+2}|<\beta|a_{N+1}|<\beta^{2}|a_{N}|,}}\\ {{\cdot\cdot\cdot\cdot\cdot\cdot\cdot\cdot\cdot\cdot\cdot\cdot\cdot\cdot\cdot\cdot\cdot\cdot\cdot\cdot\cdot\cdot\cdot\cdot\cdot\cdot}}\end{array} $$NUMERICAL sEQUENCES AND SERIEs 67 That is, $$ .|a_{n}|<|a_{N}|\beta^{-N}\cdot\beta^{n} $$ for n≥ N, and (a) follows from the comparison test, since Eβ” converges. If lan+1 ≥ lal for n ≥ no,it is easilyscen that the condition $a_{n}\to V$ does not hold, and (b) follows. Note: The knowledge that lim a,+1/0。=1 implies nothing about the convergence of Za,、 The series I/n and Xi/m demonstrate this 3.35Examples (a)Consider the series $$ {\frac{1}{2}}+{\frac{1}{3}}+{\frac{1}{2^{2}}}+{\frac{1}{3^{2}}}+{\frac{1}{2^{3}}}+{\frac{1}{2^{4}}}+{\frac{1}{3^{4}}}+\cdots $$ for which $$ \operatorname*{lim}_{n\to\infty}\frac{a_{n+1}}{a_{n}}=\operatorname*{lim}_{n\to\infty}\left(\frac{2}{3}\right)^{n}=0, $$ The root test indicates convergence; the ratio test does not apply (b) The same is true for the series $$ {\frac{1}{2}}+1+{\frac{1}{8}}+{\frac{1}{4}}+{\frac{1}{32}}+{\frac{1}{16}}+{\frac{1}{128}}+{\frac{1}{64}}+\cdots, $$ where but $$ \begin{array}{r}{\operatorname*{lim}_{n\to\infty}{\frac{a_{n+1}}{a_{n}}}={\frac{1}{8}},}\\ {\operatorname*{lim}_{n\to\infty}{\frac{a_{n}}{a_{n}}}=2,}\end{array} $$ lim ya, = +68 PRINCIPLEs OF MATHEMATICAL ANALYsis 3.36 Remarks The ratio test is frequently easier to apply than the root test since it is usually easier to compute ratios than nth roots. However, the root test has wider scope. More precisely: Whenever uhe ratio test shows conver gence, the root test does too; whenever the root test is inconclusive, the ratio test is too. This is a consequence of Theorem 3.37,and is iustrated by the above examples Neither of the two tests is subtle with regard to divergence. Both deduce divergence from the fact that a, does not tend to zero as n→00. 3.37 Theorem For any sequence {c) of positive numbers $$ \begin{array}{l}{{\operatorname*{lim}_{n\to\infty}{\frac{c_{n}+1}{c_{n}}}\leq\operatorname*{lim}_{n\to\infty}{\frac{\operatorname*{lim}_{n}\operatorname{in}\Gamma\left(c_{n}\right)}{n\to\infty}}\left(\operatorname*{lim}_{n\to\infty}\right)}}\\ {{\operatorname*{lim}_{n\to\infty}\operatorname*{lim}_{n\to\infty}{\frac{c_{n+1}}{c_{n}}}\left(\operatorname*{lim}_{n\to\infty}\right)}.}}\end{array} $$ Proof We shall prove the second inequality; the proof of the first is quite similar. Put $$ x=\operatorname*{lim}_{n\to\infty}\operatorname*{sup}_{C_{n}}{\frac{C_{n}+1}{C_{n}}}. $$ is an integer $\mathcal{N}$ If α = + o, there is nothing to prove. If is finite, choose $\textstyle{\hat{\mathcal{J}}}$ >α. There such that $$ {\frac{c_{n+1}}{c_{n}}}\leq\beta $$ for $\scriptstyle n\geq N$ In particular, for any $\rho>0,$ $$ c_{N+k+1}\leq\beta c_{N+k}\qquad(k=0,\,1,\,\cdot\cdot,\,p-1). $$ Multiplying these inequalities, we obtain $$ c_{N+p}\leq\beta^{p}c_{N}, $$ or $$ c_{n}\le c_{N}\,\beta^{-N}\cdot\beta^{n}\qquad(n\ge N). $$ Hence $$ \Lambda\overline{{{C_{n}}}}\leq\ 4\overline{{{\sqrt{C_{N}\,\beta-N}}}}\cdot\beta, $$ so that (18) lim sup ,≤ βNUMERICAL SEQUENCES AND SERIES 69 by Theorem 3.20(b)、Since (18) is true for every ${\boldsymbol{\beta}}>{\boldsymbol{\alpha}},$ we have $$ \operatorname*{lim}_{n\to\infty}\operatorname{sup}_{\Omega}n{\stackrel{n}{\underset{n}}}\leq\alpha. $$ POWER SERIES 3.38 Definition Given a sequence $\{c_{n}\}$ of complex numbers, the series (19) $$ \sum_{n=0}^{\infty}c_{n}\,\mathbb{Z}^{n} $$ is called a power series. The numbers ${\mathcal{C}}_{\eta\bar{\eta}}$ are called the coeffcents of the series; z is a complex number. In general, the series will converge or diverge,depending on the choice of z. More specifically, with every power series there is associated a circle,the circle of convergence, such that (19) converges if z is in the interior of the circle and diverges if z is in the exterior (to cover all cases, we have to consider the plane as the interior of a circle of infinite radius, and a point as a circle of radius not be described so simply zero). The behavior on the circle of convergence is much more varied and can 3.39 Theorem Given the power series Ec,.z", pu $$ \alpha=\operatorname*{lim}_{n\to\infty}\mathrm{eqp}\stackrel{n}{\sqrt{|c_{n}|}},\qquad R=\frac{1}{\alpha}. $$ (Ifα = 0,R = + 00;if α = + CO,R = 0.)Then Ec,z" converges if|z」< R,and diverges if |z|> R Proof Pu $a_{n}=c_{n}z^{n},$ and apply the root test: $$ \operatorname*{lim}_{n\to\infty}\operatorname*{sup}_{a\to\infty}\;^{n{\sqrt{|a_{n}|}}}=|z|\operatorname*{lim}_{n\to\infty}\operatorname*{sup}_{a\to\infty}\;^{n{\sqrt{|c_{n}|}}}={\frac{|z|}{R}}. $$ Note:R is called the radius of convergence of E $\sum_{i=1}^{n}C_{n}\G_{i}\ S_{i}\ S_{i}\Phi_{i}-1$ z” 3.40 Examples (a)The series En" z” has $\scriptstyle{R=0}$ (b)The series $\textstyle\sum_{n}^{\mathbb{Z}^{n}}$ has R = +00.(In this case the ratio test is easier to apply than the root test.)70 PRINCTPLEs Oor MATHEMA TICAL ANALxsis (c)The series Zz” has R = 1. If |z| = 1, the series diverges, since {z”} does not tend to O as n→0O. (d)The series $\textstyle\sum_{n}^{\mathbb{Z}^{n}}$ has R = 1. It diverges if z =1. It converges for all other z with $|z|=1.$ (The last assertion will be proved in Theorem 3.44. (e)The series $\sum{\frac{Z^{n}}{p^{2}}}$ has R = 1. It converges for all z with $\scriptstyle{|z|-1}$ by the comparison test, since |z"/n-| = 1/n2 SUMMATION BY PARTS 3.41 Theorem Given two sequences {a)},{b}, pur $$ A_{n}={\frac{\pi}{\geq}}a_{k} $$ if n ≥0; put A-1 = 0.Then, if O≤p ≤q,we have (20) $$ \sum_{n=p}^{q}a_{n}b_{n}=\sum_{n=p}^{q-1}A_{n}(b_{n}-b_{n+1})+A_{q}b_{q}-A_{p-1}b_{p}. $$ Proof $$ \sum_{n=p}^{q}a_{n}b_{n}=\sum_{n=p}^{q}(A_{n}-A_{n-1})b_{n}=\sum_{n=p}^{q}A_{n}b_{n}-\sum_{n=p-1}^{q-1}A_{n}b_{n+1}, $$ and the last expression on the right is clearly equal to the right side of (20). We shall now give applications. Formula (20), the so-called “partial summation formula,” is usefu in the investigation of series of the form Za,b,, particularly when {b,} is monotonic 3.42 Theorem Suppose n→O (b)b。≥b,≥b,≥· (a)the parial sums A,of Za, form a bounded sequence; ic) im b,= 0 Then Ea,bnconverges.NUMERICAL SEQUENCES AND SERIES 71 ProofChoose M such that |A,≤ M for all n. Given é> 0, there is an integer $\mathcal{N}$ such that bx ≤(c/2M)、For $N\leq p\leq q,$ we have $$ \begin{array}{l}{{\displaystyle\sum_{n=p}^{q}a_{n}b_{n}\left|=\left|\sum_{n=p}^{q-1}\!\!A_{n}(b_{n}-b_{n+1})+A_{q}b_{q}-A_{p-1}b_{p}\right|\!\!}}\\ {{\displaystyle\leq M\left|\sum_{n=p}^{q-1}\!\! (b_{n}-b_{n+1}\right)+b_{q}+b_{p}\right|\!\!}}\\ {{}}&{{=2M b_{p}\leq2M b_{N}\leq\varepsilon.}}\end{array} $$ Convergence now follows from the Cauchy criterion. We note that the first inequality in the above chain depends of course on the fact that bn一 b,+1 ≥ 0. 3.43 Theorem Suppose (a)|c|≥ 「cz|≥|ca|≥·… (m=1,2,3,..) (のczm- 2 , czm≤ 0 (o)lim,-。c.= 0. Then Zc,converges. Series for which(b) holds are called “alternating series"; the theorem was known to Leibnitz. Proof Apply Theorem 3.42, with $a_{n}=(-1)^{n+1},\,b_{n}=|c_{n}|$ 3.44 Theorem Suppose the radius of convergence of Zcn,2”is 1, and suppose Co≥ C,≥C2≥…,lim,→。Cn = 0. Then $\Sigma c_{x}x^{*}$ converges at every point on the circle |z|=1, except possibly at z =1. Proof Put $a_{n}=z^{n},\,\,b_{n}=c_{n}.$ The hypotheses of Theorem 3.42 are then satisfied, since $$ \left|A_{n}\right|=\left|\sum_{m=0}^{n}z^{m}\right|=\left|{\frac{1-z^{n+1}}{1-z}}\right|\leq{\frac{2}{\left|1-z\right|}}\,, $$ if |z|= 1,z ≠ 1 ABSOLUTE CONVERGENCE The series $\textstyle\mathbf{Z}a_{n}$ is said to converge absolutely if the series E|a, converges 3.45 Theorem I Za, converes absolutely,then Zα, converges72 PRINCIPLES OF MATHEMATICAL ANALYsIS Proof The assertion follows from the inequality $$ \left|\sum_{k=n}^{m}a_{k}\right|\leq\sum_{k=n}^{m}|a_{k}|\,, $$ plus the Cauchy criterion 3.46 Remarks For series of positive terms, absolute convergence is the same as convergence. If $\Sigma a_{n}$ converges,but E|a,diverges,we say that Ea,converges non- absolutely. For instance, the series $$ \textstyle\sum{\frac{(-1)^{n}}{n}} $$ converges nonabsolutely (Theorem 3.43). The comparison test, as well as the root and ratio tests, is really a test for absolute convergence,and therefore cannot give any information about non absolutely convergent series. Summation by parts can sometimes be used to handle the latter. In particular, power series converge absolutely in the interior of the circle of convergence. We shall see that we may operate with absolutely convergent series very much as with finite sums. We may multiply them term by term and we may change the order in which the additions are carried out, without affecting the sum of the series. But for nonabsolutely convergent series this is no longer true and more care has to be taken when dealing with them. ADDITION AND MULTIPLICATION OF SERIES 3.47 Theorem IfEa,= A,and Zb,= B,then E(α,+ b,) = A + B,and Eca, = cA, for any fixed c ProofLet $$ A_{n}=\sum_{k=0}^{n}a_{k}\,,\qquad B_{n}=\sum_{k=0}^{n}b_{k}\,. $$ Then $$ A_{n}+B_{n}=\sum_{k=0}^{n}(a_{k}+b_{k}). $$ Since $\operatorname*{lim}_{n\to\infty}A_{n}=A$ and $\operatorname*{lim}_{n\to\infty}B_{n}=B,$ we see that $$ \operatorname*{lim}_{n arrow\infty}(A_{n}+B_{n})=A+B. $$ The proof of the second assertion is even simpler.NUMERICAL sEoUENCES AND SERIEs 73 Thus two convergent series may be added term by term, and the result ing series converges to the sum of the two series. The situation becomes more complicated when we consider multiplication of two series. To begin with, we have to define the product. This can be done in several ways; we shall consider the so-called “Cauchy product.” 3.48 Definition Given $\textstyle\mathbf{Z}a_{n}$ and $\Sigma b_{n},$ we put $$ c_{n}=\sum_{k=0}^{n}a_{k}\,b_{n-k}\qquad(n=0,\,1,\,2,\,\dots) $$ and cal $\textstyle\mathbf{Z}c_{n}$ the product of the two given series. This definition may be motivated as follows. If we take two power series $\scriptstyle{{\sum}a_{n}z^{n}}$ and $\Sigma b_{n}z^{\prime}$ ”, multiply them term by term, and collect terms contain- ing the same power of z,we get $$ \begin{array}{l}{{\frac{\alpha}{\hbar}\ a_{n}z^{n}\cdot\sum_{n=0}^{\infty}b_{n}z^{n}=(a_{0}+a_{1}z+a_{2}z^{2}+\cdots)(b_{0}+b_{1}z+b_{2}z^{2}+\cdots)}}\\ {{\ }}&{{=a_{0}\,b_{0}+(a_{0}\,b_{1}+a_{1}b_{0})z+(a_{0}\,b_{2}+a_{1}b_{1}+a_{2}\,b_{0})z^{2}+\cdots,}}\\ {{=c_{0}+c_{1}z+c_{2}z^{2}+\cdots.}}\end{array} $$ Setting $z=1,$ 1, we arrive at the above definition 3.49 Example If $$ A_{n}=\sum_{k=0}^{n}a_{k}\,,\qquad B_{n}=\sum_{k=0}^{n}b_{k}\,,\qquad C_{n}=\sum_{k=0}^{n}c_{k}\,, $$ and A,→ A $B_{n}\to B,$ then it is not at all clear that {Ch} will converge to AB, since we do not have ${\cal C}_{n}=A_{n}\,B_{n}$ The dependence of{CA} on {A} and {B} i quite a complicated one (see the proof of Theorem 3.50). We shall now show that the product of two convergent series may actually diverge The series $$ \sum_{n=0}^{\infty}{\frac{(-1)^{n}}{\sqrt{n+1}}}=1-{\frac{1}{\sqrt{2}}}+{\frac{1}{\sqrt{3}}}-{\frac{1}{\sqrt{4}}}+\cdot\cdot\cdot $$ converges (Theorem 3.43). We form the product of this series with itself and obtain $$ \sum_{n=0}^{\infty}c_{n}=1-\left(\frac{1}{\sqrt{2}}+\frac{1}{\sqrt{2}}\right)+\left(\frac{1}{\sqrt{3}}+\frac{1}{\sqrt{2}\sqrt{2}}+\frac{1}{\sqrt{3}}\right) $$ $$ -\left({\frac{1}{\sqrt{4}}}+{\frac{1}{\sqrt{3}\,\sqrt{2}}}+{\frac{1}{\sqrt{2}\,\sqrt{3}}}+{\frac{1}{\sqrt{4}}}\right)+\cdots. $$74 PRINCIPLES OF MATHEMATICAL ANALYSIS so that Since $$ c_{n}=(-1)_{k=0}^{n}\frac{1}{\sqrt{(n-k+1)(k+1)}}. $$ $$ (n-k+1)(k+1)=\left({\frac{n}{2}}+1\right)^{2}-\left({\frac{n}{2}}-k\right)^{2}\leq\left({\frac{n}{2}}+1\right)^{2}. $$ we have $$ |c_{n}|\geq_{k=0}^{n}{\frac{2}{n+2}}={\frac{2(n+1)}{n+2}}, $$ so that the condition $\scriptstyle c_{n}=0,$ which is necessary for the convergence of $\sum_{i=1}^{n}C_{n}\ln O_{i}\ln O_{i}\ln O_{i}\Delta_{i}\Delta_{i}\Delta_{i}\Delta_{i}\Delta_{i}\Delta_{i}\Delta_{i}\Delta_{i}\Delta_{i}\Pi^{2}$ is not satisfied In view of the next theorem, due to Mertens, we note that we have here considered the product of two nonabsolutely convergent series. 3.50 Theorem Suppose Then $$ \begin{array}{l l}{{(a)}}&{{\stackrel{ harpoonup}{\to}a_{n}\,c o n v e r g e s\,a b s o l a t e l y,}}\\ {{(b)}}&{{\stackrel{ harpoonup}{\to}a_{n}=A,}}\\ {{(c)}}&{{\stackrel{ harpoonup}{\to}b_{n}=B,}}\\ {{(d)}}&{{c_{n}=\sum_{k=0}^{n}a_{k}b_{n-k}}}\end{array}\qquad{(n=0,1,2,\dots).} $$ $$ \sum_{n=0}^{\infty}c_{n}=A B. $$ That is, the product of two convergent series converges, and to the right value,if at least one of the two series converges absolutely. Proof Put $$ A_{n}=\sum_{k=0}^{n}a_{k}\,,\qquad B_{n}=\sum_{k=0}^{n}b_{k}\,,\qquad C_{n}=\sum_{k=0}^{n}c_{k}\,,\qquad\beta_{n}=B_{n}-B. $$ Then $$ \begin{array}{l}{{C_{n}=a_{0}\,b_{0}+(a_{0}\,b_{1}+a_{1}b_{0})+\cdot\cdot\cdot+(a_{0}\,b_{n}+a_{1}b_{n-1}+\cdot\cdot\cdot\cdot\cdot+(a_{0}\,b_{n}+a_{n}b_{0})}}\\ {{\ \ \ \ \ \ \ \ \ \ \ \ \ \ \ \ \ \ \ \ \ \ \ \ \ \ \ \ \ \ \ \ \ \ \ \ \ \ \ \ \ \ \ \ \ \ \ \ \ \ \ }}\\ {{\ =a_{0}B_{n}+a_{1}B_{n-1}+\cdot\cdot\cdot+a_{n}\beta_{0}\qquad}}\end{array} $$NUMERICAL SEQUENCES AND SERIES 75 Put $$ \gamma_{n}=a_{0}\,\beta_{n}+a_{1}\beta_{n-1}+\cdot\cdot\cdot+a_{n}\beta_{0}\,. $$ We wish to show that C,→ AB. Since $A_{n}B\to A B,$ it suffices to show that (21) $$ \operatorname*{lim}_{n arrow\infty}\gamma_{n}=0. $$ Put $$ x=\sum_{n=0}^{\infty}\left|a_{n}\right|. $$ [t is here that we use (a).] Let B> 0 be given.By (c),β。→0.Hence we can choose $\mathcal{N}$ such that β,|≤efor n ≥ N, in which case $$ \begin{array}{c}{{\gamma_{n}|\leq|\beta_{0}a_{n}+\cdots+\beta_{N}a_{n-N}|+\textstyle|\beta_{N+1}a_{n-N-1}+\cdots+\beta_{n}a_{0}|}}\\ {{\leq|\beta_{0}a_{n}+\cdots+\beta_{N}a_{n-N}|+\varepsilon x.}}\end{array} $$ Keeping N fxed, and letting n→O, we get $$ \operatorname*{lim}_{n\to\infty}\operatorname{up}\mid\gamma_{n}|\leq\varepsilon\alpha, $$ since ak→0 as k→O0.Since s is arbitrary,(21) follows. Another question which may be asked is whether the series Ec,,if con vergent, must have the sum AB.Abel showed that the answer is in the affirma tive. 3.51 Theorem f’the series Za,,2D,,Zc。converge to A,B,C,and C,= a。b。+ … + α,bo,then C = AB. Here no assumption is made concerning absolute convergence. We shal give a simple proof(which depends on the continuity of power series) after Theorem 8.2. REA RRANGEMENTS J onto ${\widehat{\mathcal{Y}}}_{9}$ 3.52 Definition Let {k,),n = 1,2,3,.., be a sequence in which every positive integer appears once and only once (that is, {k,}is a 1-1 function from in the notation of Definition 2.2)、 Putting $$ a_{n}^{\prime}=a_{k_{n}}\quad\quad(n=1,\,2,\,3,\,\cdot\cdot), $$ we say that Eaf,is a rearrangement of $\textstyle\mathbf{Z}a_{n}$76 PRINCIPLES OF MATHEMATICAL ANALYsIS lf {s,},{s;} are the sequences of partial sums of Ea $\Sigma a_{n},$ 。,Ea,,it is easily seen that, in general,these two sequences consist of entirely different numbers We are thus led to the problem of determining under what conditions all rearrangements of a convergent series will converge and whether the sums are necessarily the same. 3.53 Example Consider the convergent serie (22) $$ 1-{\frac{1}{2}}+{\frac{1}{3}}-{\frac{1}{4}}+{\frac{1}{3}}-{\frac{3}{6}}+\cdots. $$ and one of its rearrangements (23) $$ 1\,+\,{\frac{1}{3}}\,-\,{\frac{1}{2}}\,+\,{\frac{1}{9}}\,+\,{\frac{1}{7}}\,-\,{\frac{1}{4}}\,+\,{\frac{1}{9}}\,+\,{\frac{1}{1}}\,-\,{\frac{1}{6}}\,+\,\cdot\,\cdot\,\cdot $$ in which two positive terms are always followed by one negative. If s is the sum of (22), then $$ s<1-{\frac{1}{2}}+3={\frac{8}{8}}. $$ Since $$ {\frac{1}{4k-3}}+{\frac{1}{4k-1}}-{\frac{1}{2k}}>0 $$ for k≥1,we see that $S_{3}^{\prime}\ <S_{6}^{\prime}\ <S_{9}^{\prime}\ <\cdot\cdot\cdot\cdot\ ,$ where $\mathbf{S}_{P R}^{Y}$ ’is nth partial sum of (23) Hence $$ \operatorname*{lim}_{n arrow\infty}\operatorname*{sup}_{\mathcal{S}_{n}^{\prime}}>s_{3}^{\prime}={\dot{\mathbb{S}}}, $$ so that (23) certainly does not converge to s[we leave it to the reader to verify that (23) does, however, converge] This example illustrates the following theorem, due to Riemann. 3.54 Theorem Let $\Sigma a_{n}$ be a series of real numbers which converges,but not1 absolutely. Suppose 一00 ≤α≤β≤oo Then there exists a rearrangement $\Sigma a_{n}^{\prime}$ with partial sums s' such that (24) $$ \operatorname*{lim}_{n arrow\infty}\operatorname*{s}_{n}^{\prime}=\alpha,\qquad\operatorname*{lim}_{n arrow\infty}\operatorname*{sup}_{S_{n}^{\prime}}=\beta. $$ Proot Let $$ p_{n}={\frac{\mid a_{n}\mid+a_{n}}{2}},\qquad q_{n}={\frac{\mid a_{n}\mid-a_{n}}{2}}\qquad(n=1,2,3,\ldots). $$NUMERICAL SEQUENCES AND SERIES 77 Then $p_{n}-q_{n}=a_{n},\,p_{n}+q_{n}=\,\left|\,a_{n}\,\right|,\,p_{n}\geq0,\,q_{n}\geq0.$ The series Epm,Eq, must both diverge. For $\overset{\circ}{\bigcup{}}$ both were convergent, then $$ \Sigma(p_{n}+q_{n})=\Sigma|a_{n}| $$ would converge, contrary to hypothesis. Since $$ \sum_{n=1}^{N}\,a_{n}=\sum_{n=1}^{N}\,\left(p_{n}-q_{n}\right)=\sum_{n=1}^{N}\,p_{n}-\sum_{n=1}^{N}\,q_{n}\,, $$ divergence of $\Sigma p_{n}$ and convergence of $\Sigma q_{n}$ (or vice versa) implies diver- gence of $\Sigma a_{n}\,,$ , again contrary to hypothesis Now let $P_{i},P_{2},P_{3},\dots$ denote the nonnegative terms of $\Sigma a_{n},$ in the order in which they occur, and let Q,Q2,Q,.. be the absolute values of the negative terms of Za,, also in their original order. The series EP,,EQ,differ from $\Sigma p_{n},\ \Sigma q_{n}$ only by zero terms, and are therefore divergent We shall construct sequences $\{m_{n}\},\{k_{n}\},$ such that the series (25) $$ \begin{array}{l}{{P_{1}+\cdots+P_{m_{1}}-Q_{1}-\cdots-Q_{k_{1}}+P_{m_{1}+1}+\cdots}}\\ {{+P_{m_{2}}-Q_{k_{1}+1}-\cdots-Q_{k_{2}}+\cdots,}}\end{array} $$ which clearly is a rearrangement of $\Sigma a_{n}\,,$ satisfies(24) Choose real-valued sequences {C,},{B.} such that α,→α,β,→β $\alpha_{n}<\beta_{n}\,,\,\beta_{1}>0$ $\operatorname{Let}\ m_{1},\,k_{1}$ be the smallest integers such that $$ \begin{array}{c}{{P_{1}+\cdots+P_{m_{1}}>\beta_{1},}}\\ {{P_{1}+\cdots+P_{m_{1}}-Q_{1}-\cdots-Q_{k_{l}}<\alpha_{1};}}\end{array} $$ let mz,kz be the smallest integers such that $$ \begin{array}{l}{{P_{1}+\cdots+P_{m_{1}}-Q_{1}-\cdots-Q_{k_{1}}+P_{m_{1}+1}+\cdots+P_{m_{2}}-Q_{k_{1}+1}}}\\ {{P_{1}+\cdots+P_{m_{1}}-Q_{1}-\cdots-Q_{k_{i}}+P_{m_{1}+1}+\cdots+P_{m_{2}}-Q_{k_{1}+1}}}\\ {{P_{1}+\cdots+P_{m_{1}}-Q_{1}+\cdots+P_{m_{2}}-Q_{k_{2}}<\alpha_{2};}}\end{array} $$ and continue in this way. This is possible since EP,and $\sum_{n=1}^{\infty}\sum_{n}\ln_{n}\left(-\frac{B_{n}}{n}\right)_{n}$ diverge If $X_{n}\,,\,y_{n}$ denote the partial sums of (25) whose last terms are $\textstyle{\mathcal{P}}_{m_{n}}\,,$ - Q..,then $$ |x_{n}-\beta_{n}|\leq P_{m_{n}},\qquad|y_{n}-\alpha_{n}|\leq Q_{k_{n}}. $$ Since $p_{n}\to0$ and $\scriptstyle G_{n} arrow0$ b as $\mathcal{I}\mathcal{I}\,\longrightarrow\L\mathcal{C}\mathcal{I}\,\supset$ we see that xn→β,y。→α B can Finally it is clear that no number less than α or greater than $\textstyle\iint$ be a subsequential limit of the partial sums of(25)78 PRINCIPLES Or MATHEMATICAL ANALYSIs 3.55 Theorem If Ea,is a series of complex numbers which converges absolutely then every rearrangement of Ea,converges, and they all converge to the same sum Proof Let $\textstyle\sum\!a_{n}^{\prime}$ ,be a rearrangement, with partial sums s;.Given 8> 0. there exists an integer ${\mathcal{D}}\backslash$ such that m ≥n≥ $\mathcal{N}$ implies (26) $$ \sum_{i=n}^{m}\;\left|a_{i}\right|\leq\varepsilon. $$ Now choose p such that the integers 1,2,.., N are ll contained in the set $\:K_{1},\,K_{2}\,,\,\cdot\,\cdot\,\cdot\,s_{p}\:$ (we use the notation of Definition 3.52). Then ifn> p the numbers $a_{1},\,\cdot\cdot\cdot,\,a_{N}$ will cancel in the difference S。一S',so that lS,一sA|≤8,by (26). Hence {S)} converges to the same sum as {s,} EX ERCISES 1. Prove that convergence of {s,} implies convergence of {ls,}、Is the converse true 2. Calculate lim (Vn*+n- n). 3.I1f $s_{1}={\sqrt{2}}.$ 2, and $$ s_{n+1}={\sqrt{2+{\sqrt{s_{n}}}}}\qquad(n=1,2,3,\ldots), $$ prove that $\{s_{n}\}$ converges, and that $\scriptstyle{\mathfrak{s}}\,<\,2$ 2 for n =1,2,3,.. 4.Find the upper and lower limits of the sequence {sn} defined by $$ s_{1}=0;\qquad s_{2m}={\frac{s_{2m-1}}{2}}\cdot\qquad s_{2m+1}={\frac{1}{2}}\,+\,s_{2m}\,. $$ S. For any two real sequences {a,},{b,), prove that $$ \operatorname*{lim}_{n\to\infty}\operatorname{sup}_{n\to\infty}\beta_{n\to\infty}\delta_{n\to\infty}\delta_{n\to\infty}\delta_{n\to\infty}\delta_{n\times\sum\sum_{n\to\infty}} $$ provided the sum on the right is not of the form o CO 6. Investigate the behavior (convergence or divergence) of $\textstyle\Sigma a_{n}$ if $$ \begin{array}{l}{{(a)\ \ a_{n}=\sqrt{n+1}-\sqrt{n};}}\\ {{(b)\ a_{n}=\displaystyle\frac{\sqrt{n+1}-\sqrt{n}}{n};}}\\ {{(c)\ a_{n}=\displaystyle\frac{1}{n}}}\\ {{(d)\ a_{n}=\displaystyle\frac{1}{1+z^{*}},}}\end{array} $$ T. Prove that the convergence of $\Sigma a_{n}$ implies the convergence of $$ \textstyle\sum{\frac{\sqrt{a_{n}}}{n}}, $$ if an ≥0NUMERICAL SEQUENCES AND SERIES 79 8. If EZa, converges, and if {bis monotonic and bounded, prove that Za,。con verges 9. Find the radius of convergence of each of the following power series: 2” $$ \begin{array}{c}{{(b)\,\sum\!\!\!\!\frac{2^{n}}{n!}z^{n},}}\\ {{(d)\sum\!\!\!\!\frac{n^{3}}{3^{n}}z^{n}.}}\end{array} $$ (a)艺nz”, (e)王第” 10. Suppose that the coefficients of the power series Ea,z" are integers, infinitely many of which are distinct from zero.Prove that the radius of convergence is at most 1. 11. Suppose a,>0, $s_{n}=a_{1}+\cdot\cdot\cdot+$ Qn, and Za,diverges. (a) Prove that Z Qn diverges 1 + αn (b) Prove that $$ \frac{Q_{N+1}}{S_{N+1}}\ +\ ^{*}*\dots\star\frac{Q_{N+k}}{S_{N+k}}\geq1\longrightarrow\frac{S_{N}}{S_{N+k}} $$ and deduce that Z Sn fives (c) Prove that $$ {\frac{a_{n}}{s_{n}^{2}}}\leq{\frac{1}{s_{n-1}}}-{\frac{1}{s_{n}}} $$ and deduce that $\sum\frac{Q_{n}}{S_{n}^{2}}$ converges. (d) What can be said about $$ \begin{array}{l l l}{{\underline{{a_{n}}}}}&{{\mathrm{and}}}&{{\sum\underline{{a_{n}}}}}\\ {{\underline{{1+n a_{n}}}}}&{{}}\end{array} $$ 12.Suppose $\scriptstyle a_{n}\gg0$ and Ean Converges. Put $$ r_{n}=\sum_{m=n}^{\infty}a_{m}\,. $$ (a) Prove that $$ {\frac{a_{m}}{r_{m}}}+\cdot\cdot\cdot+{\frac{a_{n}}{r_{n}}}>1-{\frac{r_{n}}{r_{m}}} $$ if $m<n,$ and deduce that $\textstyle\sum_{T_{n}}^{{\underline{{a}}}_{n}}\mathrm{diverges}$80 PRINCIPLES OF MATHEMATICAL ANALYSIs (b)Prove that $$ \frac{a_{n}}{\sqrt{r_{n}}}<2(\sqrt{r_{n}}-\sqrt{r_{n+1}}) $$ and deduce that $\textstyle\sum{\frac{a_{n}}{\sqrt{r_{n}}}}$ converges. 13. Prove that the Cauchy product of two absolutely convergent series converges absolutely. 14.If{sh is a complex sequence, define its arithmetic means o,by $$ \sigma_{n}={\frac{s_{0}+s_{1}+\cdot\cdot\cdot+s_{n}}{n+1}}\qquad(n=0,1,2,\cdot\cdot). $$ (a) If lim $S_{n}=S,$ prove that lim $\sigma_{n}=S.$ (b)Construct a sequence {sn} which does not converge, although lim $\sigma_{n}\doteq0$ (c)Canit happen that s ${\mathfrak{s}}_{n}\geqslant0$ for ll mand that im sup s.= o,although lim $\sigma_{n}=0^{\circ}$ (d) Put an = Sn一Sn-, for n ≥1.Show that $$ s_{n}-\sigma_{n}={\frac{1}{n+1}}\sum_{k=1}^{n}k a_{k}\,. $$ Assume that $\operatorname*{lim}\left(n a_{n}\right)=0$ and that {o,} converges. Prove that {s) converges [This gives a converse of (a), but under the additional assumption that na,→0.] (e)Derive the last conclusion from a weaker hypothesis:Assume $M<\infty,$ lnan|≤M for all n, and lim $\sigma_{n}=o$ o. Prove that lim sn= 0, by completing the following outline: If $m<n,$ then $$ s_{n}-\sigma_{n}=\frac{m+1}{n-m}(\sigma_{n}-\sigma_{m})\;+\frac{1}{n-m}\;\sum_{i=m+1}^{n}(s_{n}-s_{i}). $$ For these i, $$ |s_{n}-s_{i}|\leq{\frac{(n-i)M}{i+1}}\leq{\frac{(n-m-1)M}{m+2}}. $$ Fix $\varepsilon>0$ and associate with each n the integer m that satisfies $$ m\leq{\frac{n-\varepsilon}{1+\varepsilon}}<m+1. $$ Then (m + 1)/(n - m) ≤1/e and $|{\it s_{n}}-{\it s_{i}}|\ <M\varepsilon.$ Hence $$ \operatorname*{lim}_{n\to\infty}\operatorname{sup}|S_{n}-\sigma|\leq M\varepsilon $$ Since e was arbitrary, lim $S_{n}=\sigma$NUMERICAL sEoUENCES AND SERIES 81 15. Definition 3.21 can be extended to the case in which the a, lie in some fixed $R^{k}$ Absolute convergence is defined as convergence of E|a,|. Show that Theorems 3.223.23,3.25(a),3.33,3.34、3.42,3.45, 3.47, and 3.55 are true in this more general sctting.(Only slight modifications are required in any of the proofs. 16.Fix a positive number α. Choose x> Vα,and define $x_{2}\,,$ x.,X4,...,by the recursion formula $$ x_{n+1}={\frac{1}{2}}\left(x_{n}+{\frac{\alpha}{x_{n}}}\right). $$ (a) Prove that {x,} decreases monotonically and that $\operatorname*{lim}x_{n}={\sqrt{\alpha}}$ (b)Put $\varepsilon_{n}\Longrightarrow{x}_{n}\longrightarrow\surd{\overset{\sigma}{\alpha}},$ and show that $$ \varepsilon_{n+1}=\frac{\varepsilon_{n}^{2}}{2x_{n}}<\frac{\varepsilon_{n}^{2}}{2\sqrt{\alpha}} $$ so that,setting $\beta=2\sqrt{\ }\alpha,$ $$ \varepsilon_{n+1}<\beta\left(\frac{\varepsilon_{1}}{\beta}\right)^{2^{n}}\qquad(n=1,\,2,\,3,\,\cdot\cdot). $$ (c) This is a good algorithm for computing square roots. since the recursion formula is simple and the convergence is extremely rapid. For example, if α =3 and x = 2, show that é ${\bf s_{1}}/\beta<0{\frac{1}{1\,0}}$ and that therefore $$ \varepsilon_{5}<4\cdot10^{-16},\qquad\varepsilon_{6}<4\cdot10^{-32}. $$ 17. Fix α>1. Take x>Vα,and define $$ x_{n+1}={\frac{\alpha+x_{n}}{1\ +x_{n}}}=x_{n}+{\frac{\alpha-x_{n}^{2}}{1+x_{n}}}\,. $$ (a) Prove that $x_{1}\geqslant x_{3}\geqslant x_{s}$ (b) Prove that x,<X4<x。< (c) Prove that lim $x_{n}={\sqrt{\alpha}}.$ (d) Compare the rapidity of convergence of this process with the one described in Exercise 16. 18.Replace the recursion formula of Exercise 1 $16$ by $$ x_{n+1}={\frac{p-1}{p}}x_{n}+{\frac{\alpha}{p}}x_{n}^{-p+1} $$ where p is a fixed positive integer, and describe the behavior of the resulting sequences {xn}. 19.Associate to each sequence $a=\{x_{n}\}$ , in which ${\mathcal{Q}}_{n}$ is O or 2, the real number $$ x(a)=\sum_{n=1}^{\infty}{\frac{\alpha_{n}}{3^{n}}}. $$ Prove that the set of al $x(a)$ )is precisely the Cantor set described in Sec. 2.4482 PRINCIPLES OF MATHEMATICAL ANALYSIS 20. Suppose {p) is a Cauchy scquence in a metric space X, and some subsequence tpn) converges to a point pe X. Prove that theful scquence (p,} converges to p 21. Prove the following analogue of Theorem 3.106b): I1 {EA}is a sequence of closed nonempty and bounded sets in a complete metric space X, if E.→ $E_{n+1},$ and if $$ \operatorname*{lim}_{n\to\infty}\operatorname{diam}E_{n}=0, $$ then OPEn consists of exactly one point 22. Suppose Xis a nonempty complete metric space, and (G.,} is a sequence of dense open subsets of X. Prove Baire's theorem, namely,that FG。is no empty.(In fact,it is dense in X.) Hint: Find a shrinking sequence of neighbor hoods E。such that E.c G,and apply Exercise 21 23.Suppose {p} and {q} are Cauchy sequences in a metric space X. Show that the sequence (d(p。,)} converges. lint: For any m,, $$ d(p_{n},q_{n})\leq d(p_{n},p_{m})+d(p_{m},q_{m})+d(q_{m},q_{n}); $$ it follows that $$ \left|d(p_{n},q_{n})-d(p_{m},q_{m})\right| $$ is small if m and n are large 24. Let X be a metric space (a) Call two Cauchy sequences {pn},{q,} in X equivalent if $$ \operatorname*{lim}_{x\to\infty}d(p_{n},q_{n})=0. $$ Prove that this is an equivalence relation. (b) Let X* be the set of all equivalence classes so obtained.If $\hat{\mathcal{D}}$ e X*, 9e X* {p} P,{q}e Q, define $$ \Delta(P,\,Q)=\operatorname*{lim}_{n arrow\infty}d(p_{n},q_{n}); $$ by Exercise 23,this limit exists. Show that the number A(P,Q) is unchanged i {pa} and {qg.} are replaced by equivalent sequences, and hence that △ is a distance function in $X^{\bullet}$ (c) Prove that the resulting metric space $X^{\bullet}$ is complete (d) For each pe X, there is a Cauchy sequence all of whose terms are p;let P。 be the element of $X^{\bullet}$ which contains this sequence. Prove that $$ \Delta(P,P_{e})=d(p,q) $$ for all p,qe X In other words, the mapping gp defined b g(p)= Ppis an isometry $X^{\bullet}$ (i.e., a distance-preserving mapping) of X into (e) Prove that g(X)is dense in X*,and that g(X) = X* if X is complete.By (d) we may identify X and g(X) and thus regard ${\mathcal{A}}$ as embedded in the complete metric space X*. We call Xt the completion of X 25. Let X be the metric space whose points are the rational numbers, with the metric d(x, y) =|x一P|. What is the completion of this space?(Compare Exercise 24.