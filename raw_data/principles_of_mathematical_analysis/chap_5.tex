5 DIFFERENTIATION In this chapter we shall (except in the final section) confine our attention to real functions defined on intervals or segments. This is not just a matter of con- venience, since genuine differences appear when we pass from real functions to vector-valued ones. Differentiation of functions defined on $\textstyle{\mathcal{R}}^{k}$ will be discussed in Chap. 9. THE DERIVATIVE OF A REAL FUNCTION form the quotient 5.1 Defnitin Let f be defined (and real-valued) on [a,b] For any xe [a,b1 (1) $$ \phi(t)={\frac{f(t)-f(x)}{t-x}}\qquad(a<t<b,t\neq x), $$ and define (2) f'(x) = lim db(t))104 PRINCTPLES OF MATHEMATICAL ANALYSIs provided this limit exists in accordance with Definition 4.1. We thus associate with the function f a function ${\mathcal{J}}^{\prime}$ whose domain is the set of points x at which the limit (2) exists; f’is called the derivative of f. If f’is defined at a point x, we say that fis differentiable at x.If f’is defined at every point of a set Ec [a,b], we say that fis differentiable on $\textstyle E.$ It is possible to consider right-hand and left-hand limits in (2); this leads to the definition of right-hand and left-hand derivatives. In particular, at the endpoints a and b, the derivative, if it exists,is a right-hand or left-hand deriva- tive, respctively. We shall not, however, discuss one-sided derivatives in any detail If f is defined on a segment (a,b) and if a<x<b,then f'(x) is defined by (I) and (2), as above. But f'(a) and f(b) are not defined in this case. 5.2 Theorem Let f be deined on [α,]If is dfferentiable at a point xe [a, b1 then f is continuous at x Proof As t→X, we have,by Theorem 4.4 $$ f(t)-f(x)=\frac{f(t)-f(x)}{t-x}\cdot(t-x)\to f^{\prime}(x)\cdot0=0. $$ The converse of this theorem is not true. It is easy to construct continuous functions which fail to be differentiable at isolated points. In Chap.7 we shal even become acquainted with a function which is continuous on the whole line without being differentiable at any point! 5.3Theorem Suppose f and g are defined on [a,b]and are diffeniable at a point xe [a,b]. Then f + g, fg,and flg are differentiable at x, and (a)(f + 9)'(x) =/"(x) + 9(x) (b)(O)'(x) =/"(x)9(x) +/(x)o′(x): gOx)f′(x)-g′(x)f(x) g-(x) In (c), we assume of course that g(x) ≠ 0. Proof(a) is clear, by Theorem 4.4. Let h = f9、Then ht)- h(x) =ft J[g(t)- g(x)]+ g(x)[f(t)一f(x)]DIPFERENTIATION 105 If we divide this by t- xand note that f(t)→f(x) as t→x(Theorem 5.2) (b) follows. Next, Iet $h=f/g.$ Then $$ {\frac{h(t)-h(x)}{t-x}}={\frac{1}{g(t)g(x)}}\bigg[g(x){\frac{f(t)-f(x)}{t-x}}-f(x){\frac{g(t)-g(x)}{t-x}}\bigg]. $$ Letting t→x, and applying Theorems 4.4 and 5.2,we obtain (c) 5.4 Examples The derivative of any constant is clearly zero. If fis defined by f(x)= x, then f'(x) = 1. Repeated application of (b) and (c) then shows that x” is differentiable, and that its derivative is $n X^{n-1},$ for any integer n(if n<0, we have to restrict ourselves to X≠ 0). Thus every polynomial is differentiable and so is every rational function,except at the points where the denominator is zero. The following theorem is known as the“chain rule”for differentiation lt deals with differentiation of composite functions and is probably the most important theorem about derivatives. Weshall meet more general versions of it in Chap. 9. 5.5 Theorem Suppose fis continuous on [a,b],./"(x) exists at some point xe [a, b], g is defned on an interval I which contains the range off, and g is differentiable at the point f(x).1f $$ h(t)=g(f(t))\qquad(a\leq t\leq b), $$ then h is differentiable at $X,$ , and (3) $$ h^{\prime}(x)=g^{\prime}(f(x))f^{\prime}(x). $$ (5) Proof Let y = $$ \begin{array}{l}{{(x).~~{\mathrm{By~the~defnition~of~the~derivative,~we~have}}}}\\ {{f(t)-f(x)=(t-x)[f^{\prime}(x)+u(t)],}}\\ {{g(s)-g(y)=(s-y)[g^{\prime}(y)+v(s)],}}\\ {{\epsilon^{4}\operatorname{t~then~}(4)\cdots v(s)\to0~a s~t\to y\cdots~{\mathrm{Let}}~s=y\cdots~{\mathrm{Let}}~s=J}}\end{array} $$ (t) (4) wheret∈ [a,b], s Using first (S) and or, if t ≠x, $$ \begin{array}{c}{{h(t)-h(x)-g(f(x))}}\\ {{}}\\ {{=[f(t)-f(x)]\cdot[g^{\prime}(y)+v(s)]}}\\ {{=(t-x)\cdot[f^{\prime}(x)+u(t)]\cdot[g^{\prime}(y)+v(s)]}}\end{array} $$ (6) $$ {\frac{h(t)-h(x)}{t-x}}=[g^{\prime}(y)+v(s)]\cdot[f^{\prime}(x)+u(t)]. $$ Letting t→x, we see that s→y, by the continuity of f,so that the right side of (6) tends to g'O)f"(x), which gives (3)106 PRINCIPLEs OF MATHEMATICAL ANALYSIs 5.6 Examples (a)Letf be defined by (7) $$ f(x)={\left\{\begin{array}{l l}{x\sin{\frac{1}{x}}}&{(x\neq0),}\\ {0}&{(x=0).}\end{array}\right.} $$ 5.3 and 5.5 whenever $x\neq^{\prime}0.$ Taking for granted that the derivative of sin x is cos x(we shall discuss the trigonometric functions in Chap. 8), we can apply Theorems and obtain (8) $$ f^{\prime}(x)=\sin{\frac{1}{x}}-{\frac{1}{x}}\cos{\frac{1}{x}}\qquad(x\neq0). $$ At $\scriptstyle x\;=\;0.$ 0,these theorems do not apply any longer,since 1/x is not defined there, and we appeal directly to the definition: for $t\neq0$ $$ {\frac{f(t)-f(0)}{t-0}}=\sin{\frac{1}{t}}. $$ As1→0,this does not tend to any limit, so that f'(O) does not exist (b) Let f be defined by (9) $$ f(x)={ \{x^{2}\sin{\frac{1}{x}}}\qquad(x\neq0), $$ As above, we obtain (10) $$ f^{\prime}(x)=2x\sin{\frac{1}{x}}-\cos{\frac{1}{x}}\qquad(x\neq0). $$ At $\scriptstyle x\;=\;0.$ we appeal to the definition, and obtain $$ {\frac{|f(t)-f(0)|}{t-0}}=\left|t\sin{\frac{1}{t}}\right|\leq|t|\qquad(t\neq0); $$ letting t→0, we se that (11) $$ f^{\prime\prime}(0)=0. $$ Thus f is differentiable at all points x,but f’is not a continuous function, since cos (1/x) in (10) does not tend to a limit as x→0.DIFPERENTIATION 107 MEAN VALUE THEOREMS 5.7 Definition Let f be a real function defined on a metric space X. We say that f has a local maximum at a point p∈ X if there exists $\delta>0$ such that f(q)≤ JOP) for alg X with dp, g)<员 Local minima are defined likewise Our next theorem is the basis of many applications of differentiation. 5.8 Theorem Let f be defined on [口,b]; if f has a local maximum at a point xe(a,b), and if f’(x) exists, then f'(x) = 0. The analogous statement for local minima is of course also true. Proof Choose S in accordance with Definition 5.7, so that $$ a<x-\delta<x<x+\delta<b. $$ If x-6<t<x, then $$ {\frac{f(t)-f(x)}{t-x}}\geq0. $$ Letting t→X, we see that f′(x)≥0 If x<t<x+ ,then $$ {\frac{f(t)-f(x)}{t-x}}\leq0, $$ which shows that f'(x)≤ 0. Hence f′(x)= 0 5. Thcorem If f and g are continuous real functions on [a,b]which are differentiable in (a,b), then there is a point xe(a,b) at which $$ [f(b)-f(a)]g^{\prime}(x)=[g(b)-g(a)]f^{\prime}(x). $$ Note that differentiability is not required at the endpoints Proof Put $$ h(t)=[f(b)-f(a)]g(t)-[g(b)-g(a)]f(t)\qquad(a\leq t\leq b). $$ Then $\textstyle{\int}_{\theta}$ is continuous on [a,b), his differentiable in(α, b), and (12) $$ h(a)=f(b)g(a)-f(a)g(b)=h(b). $$ To prove the theorem, we have to show that h(x) = 0 for some xe (α,b) If h is constant, this holds for every xe(a,b)、If ht)> h(a) for some te (a,b), Iet x be a point on [a, b] at which h attains its maximum108 pRINCIPLES OF MATHEMATICAL ANALysis (Theorem 4.16). By (12),xe(a,b), and Theorem 5.8 shows that h(x) = 0 If ht)< h(a) for some te (a, b), the same argument applies if we choose for x a point on [a, b] where $\textstyle{\int_{\partial}}$ attains its minimum. This theorem is often called a generalized mean value theorem; the following special case is usually referred to as “the”’ mean value theorem 5.10 Theorem If fis a real continuous function on [a, b】 which is differentiable in (a,b), then there is a point xe(a,b) at which $$ f(b)-f(a)=(b-a)f^{\prime}(x). $$ Proot Take g(x) = x in Theorem 5.9. 5.11 Theorem Suppose f is differentiable in(a,b) (a)lf f(x) ≥0 for all xe (a,b), then f is monotonically increasing (b)If f'(x) = 0 for all xe(a,b), then f is constant. (c)I厂"(x) ≤0 forall x e(a,b), hen fis monotonically decreasing. Proof All conclusions can be read off from the equation $$ f(x_{2})-f(x_{1})=(x_{2}-x_{1})f^{\prime}(x), $$ ${\mathcal{X}}_{1}$ and xz which is valid, for each pair of numbers $x_{1},x_{2}$ in (a,b), for some x between THE CONTINUITY OF DERIYATIVES We have already seen [Example 5.6(6)] that a function f may have a derivative ${\mathcal{J}}^{\prime}$ which exists at every point, but is discontinuous at some point. However, not every function is a derivative.In particular, derivatives which exist at every point of an interval have one important property in common with functions which are continuous on an interval: Intermediate values are assumed (compare Theorem 4.23). The precise statement follows. 5.12 Theorem Suppose f is a real diferentiable function on [a, b] and suppose f′a)<入<f′(b).Then there is a point xe (a,b) such that f'(x) = 入. A similar result holds of course if f(a) >J(b) Proof Put g(t)=/(t)一 入t. Then g'(a)<0,so that gt,)< g(a) for some te (a,b), and g'(b) > 0, so that $g(t_{2})<g(b)$ for some tz∈ (a,b). Hence g attains its minimum on [α,b](Theorem 4.16) at some point x such that Q $<x<b$ .By Theorem 5.8,g′(x)= 0.Hence f(x)= 入.DIFERIENTIATioN109 CorollaryIf f is differentiable on [a,b],then f’cannot have any simple dis continuities on [a,b]. But f’ may very well have discontinuities of the second kind L'HOSPITAL'S RULE The following theorem is frequently useful in the evaluation of limits 5.13 Theorem Suppose f and g are real and dierentiable in (a, b), and g'(x) =0 for allxe (a,b),where -0O ≤a<b≤ + OO.Suppose (13) $$ \frac{f^{\prime}(x)}{g^{\prime}(x)}\to A\ a s\ x\to a. $$ 1f (14) f(x)→O and g(x)→0 as x→a or if (15) 9(x)→ +0O as x→a, then (16) $$ {\frac{f(x)}{g(x)}}\to A\ a s\ x\to a. $$ The analogous statement is of course also true ifx→b, or if g(x)→ 一 OC in (15). Let us note that we now use the limit concept in the extended sense of Definition 4.33. Proof We first consider the case in which -00 ≤ A< +O0. Choose a real number g such that $A<q,$ and then choose r such that A<r<q By (13) there is a point ce (a, b) such that a<x<c implies (17) $$ {\frac{f^{\prime}(x)}{g^{\prime}(x)}}<r. $$ If a <x<y < c, then Theorem 5.9 shows that there is a point te(x, y such that (18) $$ {\frac{f(x)-f(y)}{g(x)-g(y)}}={\frac{f^{\prime}(t)}{g^{\prime}(t)}}<r. $$ Suppose (14) holds. Letting x→a in (18), we see that (19) $$ {\frac{f(y)}{g(y)}}\leq r<q\qquad(a<y<c). $$110 PRINCIPLES OF MATHEMATICAL ANALYSIS Next, suppose (15) holds. Kceping y fixed in (18),we can choose a point cie (a, J) such that g(x) > 90) and g(x) > O if a <x<C. Multi- plying (18) by [g(x)- 90D)]/9(X), we obtain (20) $$ {\frac{f(x)}{g(x)}}<r-r\,{\frac{g(y)}{g(x)}}+{\frac{f(y)}{g(x)}}\qquad(a<x<c_{1}). $$ If we let x→a in (20),(15) shows that there is a point cz∈ (a, c such that (21) $$ {\frac{f(x)}{g(x)}}<q\qquad(a<x<c_{2}). $$ Summing up,(19) and (21) show that for any q, subject only to the condition $A<q,$ there is a point ${\boldsymbol{C}}_{\textstyle2}$ such that f(x)/9(x)< g if a $<x<c_{2}$ In the same manner,if $-\alpha0<A\leq+\infty$ ,and p is chosen so that p< A, we can find a point $C_{3}$ such that (22) $$ p<\frac{f(x)}{g(x)}\;\;\;\;\;\;\;(a<x<c_{3}), $$ and(16) follows from these two statements. DERIVATIVES OF HIGHER ORDER 5.14 Defnition Iff has a derivative f’on an interval, and if f' is itself differen- tiable, we denote the derivative of ${\mathcal{f}}^{\prime}$ by f”and call f”the second derivative of f Continuing in this manner, we obtain functions $$ \mathcal{J},\mathcal{J}^{\prime},\mathcal{J}^{\prime\prime},\mathcal{J}^{(3)},\star\cdot,\mathcal{J}^{(n)}, $$ each of which is the derivative of the preceding one. $f^{(n)}$ is called the nth deriva- tive,or the derivative of order n, of f In order for ${\mathcal{f}}^{(n)}$ (x) to exist at a point $x,f^{(n-1)}\left(t\right)$ must exist in a neighbor- hood of x (or in a one-sided neighborhood, if x is an endpoint of the interval on which s eineij, an f"“S must bedifetabiea .Since - must exist in a neighborhood o x,fu"-2 must be diferentialein that neighborhoo TAYLOR'S THEOR EM 5.15 Theorem Suppose f is a real function on [a,b),n is a positive integer, be distict fm- is continvous on α, b1,./("() exists for every te(a,b)、 Let x, $\textstyle\iint$ points of Ia, b], and define (23) $$ P(t)=\sum_{k=0}^{n-1}{\frac{f^{(k)}(x)}{k!}}\,(t-\alpha)^{k}. $$DTFERENTIATION 111 Then there exists a point x between α and β such tha (24) $$ f(\beta)=P(\beta)+\frac{f^{(n)}(x)}{n!}\,(\beta-\alpha)^{n}. $$ For n = 1,this is just the mean value theorem. In general, the theorem shows that f can be approximated by a polynomial of degree $n-1.$ and that (24) allows us to estimate the error, if we know bounds on f("(x). Proof Let M be the number defined by (25) $$ f(\beta)=P(\beta)+M(\beta-\alpha)^{n} $$ and put (26) $$ g(t)=f(t)-P(t)-M(t-x)^{n}\qquad(a\leq t\leq b). $$ We have to show that n!M =f("(x) for some x between α and β. By (23) and (26), (27) $$ g^{(n)}(t)=f^{(n)}(t)-n!M\qquad(a<t<b). $$ Hence the proof will be complete if we can show that $g^{(n)}(x)=0$ for some x between α and β. Since P((α) =f(*(c) for k = 0,...,n 一1,we have (28) $$ g(\alpha)=g^{\prime}(\alpha)=\cdot\cdot\cdot=g^{(n-1)}(\alpha)=0. $$ Our choice of M shows that $g(\beta)=0,$ so that g $\textstyle\prime(x_{1})=\mathbb{C}$ O for some x between c and β,by the mean value theorem. Since ${\mathit{g}}^{\prime}(x)=0,$ we conclude similarly that $g^{\prime}(x_{2})=0$ for some x, between a and x.After n steps we arrive at the conclusion that $g^{(n)}(x_{n})=0$ for some $\textstyle{X_{n}}$ between α and Xn-1 -1 that is, between α and β DIFFERENTIATION OF VECTOR-VALUED FUNCTIONS 5.16 Remarks Definition 5.l applies without any change to complex functions J defined on [α,bJ, and Theorems 5.2 and 5.3, as well as their proos, remain vald. If and fA are the real and imaginary parts of f, ht is, i $$ f(t)=f_{1}(t)+i f_{2}(t) $$ for a ≤t≤ b,where /( and f,t) are real, then we clearly have (29)) $$ f^{\prime}(x)=f_{1}^{\prime}(x)+i f_{2}^{\prime}(x); $$ also, fis differentiable at x if and only if both f, and f, are differentiable at x112PRINCIPLES OF MATHEMATICAL ANALYSIs Passing to vector-valued functions in general, i.e., to functions f which map [a,b] into some R*, we may still apply Definition 5.l to define f'(x). The term p(t)in (1)is now, for each ${\widehat{ |}}_{3}$ a point in $R^{k},$ and the limit in (2)is taken with respect to the norm of $\textstyle{\mathcal{R}}^{k}$ In other words,f'(x)is that point of R'(if there is one) for which (30) $$ \operatorname*{lim}_{t arrow x}{\frac{\left\|t(t)-\mathbf{f}(x)}{t-x}}-\mathbf{f}^{\prime}(x)\right|=0, $$ and f’is again a function with values in $R^{k}.$ If ,,.….., are the components of , as defined in Theorem 4.10, ten (31) $$ \underline{{{\P}}}^{\prime}\,\underline{{{\downarrow}}}^{\prime}\,\underline{{{\circ}}}\!\cdot\!\iota_{\circ}\,\flat_{k}^{\prime}\!\flat_{\flat} $$ and fis differentiable at a point ${\mathcal{X}}$ if and only if each of the functions f,.….,, is differentiable at x Theorem 5.2 is true in this context as well, and so is Theorem $\mathbb{S}{\cdot}/(a)$ and (b), if f is replaced by the inner product f·g (see Deinition 4.3 When we turn to the mean value theorem, however, and to one of its consequences, namely, L'Hospital's rule, the situation changes. The next two examples will show that each of these results fails to be true for complex-valued functions. 5.17 Example Define, for real x, (32) $$ f(x)=e^{i x}=\cos x+i\sin x. $$ (The last expression may be taken as the definition of the complex exponential el"; see Chap. 8 for a full discussion of these functions.)Then (33) $$ f(2\pi)-f(0)=1-1=0, $$ but (34) $$ f^{\prime}(x)=i e^{i x}, $$ so that |f′(x) = 1 for all real x Thus Theorem 5.10 fails to hold in this case. 5.18 Example On the segment (O, l), define f(x)= xand (35) $$ g(x)=x+x^{2}e^{i/x^{2}}. $$ Since $|e^{i t}|=1$ for all real t, we see that (36) $$ \operatorname*{lim}_{x\to0}{\frac{f(x)}{g(x)}}=1. $$DIFERENTIATION 113 Next, $$ \begin{array}{r l}{y=1+ \lbrace2x-{\frac{2i}{x}} \rbrace e^{i/x^{2}}\qquad{\bigl(}0<x<1{\bigr)}}\\ {\left\vert{\frac{f^{\prime}(x)}{g^{\prime}(x)}}\right\vert={\frac{1}{\vert g^{\prime}(x)\vert}}\leq{\frac{x}{2-x}}}\end{array} $$ (37) g′(X so that (38) Hence (39) and so (40) $$ \operatorname*{lim}_{x\to0}{\frac{f^{\prime}(x)}{g^{\prime}(x)}}=0. $$ By (36) and(40),L'Hospital's rule fails in this case. Note also that g(x)≠ 0 on (0,1), by (38) However, there is a consequence of the mean value theorem which,for purposes of applications, is almost as useful as Theorem 5.10,and which re mains true for vector-valued functions: From Theorem 5.10 it follows that (41) $$ \left|f(b)-f(a)\right|\leq(b-a)\operatorname*{sup}_{a<x<b}\ |f^{\prime}(x)|\,. $$ 5.19 Theorem Suppose fis a continwous mapping of [q,b)into $\textstyle{\mathcal{R}}^{k}$ and f is differentiable in (a,b)、 Then there exists $x\in(a,b)$ such that $$ |f(b)-{\bf f}(a)|\leq(b-a)|f^{\prime}(x)|\,. $$ Proof' Put $\mathbf{z}=\mathbf{f}(b)-\mathbf{f}(a),$ and define $$ \varphi(t)=\mathrm{z}\cdot\xi(t)\qquad(a\leq t\leq b). $$ Then cp is a real-valued continuous function on [a,b] which is differentia_ ble in (a,b). The mean value theorem shows therefore that $$ \varphi(b)-\varphi(a)=(b-a)\varphi^{\prime}(x)=(b-a)z*\xi^{\prime}(x) $$ for some xe (a,b)On the other hand, $$ \varphi(b)-\varphi(a)=z\cdot\mathbf{f}(b)-\mathbf{z}\cdot\mathbf{f}(a)=\mathbf{z}\cdot\mathbf{z}=|\mathbf{z}|^{2} $$ The Schwarz inequality now gives $$ |z|^{2}=(b-a)|z\cdot{\mathsf{f}}^{\prime}(x)|\leq(b-a)|z|\,|{\mathsf{f}}(x)|. $$ Hence lzl ≤(b - a)|f(x)|, which is the desired conclusion. proof to the original one. iv. P. Havin translated the second edition of this book into Russianand added this114 PRINCIPLES OF MATHEMATICAL ANALYSIS EXERCISES 1. Let fbe defined for all real x, and suppose that $$ |f(x)-f(y)|\leq(x-y)^{2} $$ for all real x and y. Prove that fis constant 2. Suppose f'(X)>0in (a,b)Prove that fis strictly increasing in (a,b), and let g be its inverse function. Prove that $\mathcal{J}$ is differentiable, and that $$ g^{\prime}(f(x))={\frac{1}{f^{\prime}(x)}}\qquad(a<x<b). $$ 3. Suppose g is a real function on $R^{1},$ with bounded derivative (say $|g^{\prime}|\leq M\rangle$ 。Fix &>0, and define f(x) $=x+\varepsilon g(x)$ 、 Prove that fis one-to-one if sis small enough (A set of admissible values of e can be determined which depends only on M.) 4.If $$ C_{0}+{\frac{C_{1}}{2}}+\cdots+{\frac{C_{n-1}}{n}}+{\frac{C_{n}}{n+1}}=0, $$ where $C_{0},\ldots,C,$ A are real constants, prove that the equation $$ C_{0}+C_{1}x+\cdot\cdot\cdot\cdot+C_{n-1}x^{n-1}+C_{n}x^{n}=0 $$ has at least one real root between O and ${\begin{array}{r}{\mathfrak{I}}\\ {\mathfrak{I}}\end{array}},$ 5.Suppose f is defined and differentiable for every x>0, and f(x)→ 0 as x→+ CO Put g(x) =/(x + 1)-f(x).Prove that g(x)→ 0 as $x\to+\infty$ 6. Suppose (a) fis continuous for x≥0, (b) f(x) exists for x>0, (c) f(0) =0, (d) f′ is monotonically increasing Put ${\mathcal{f}}^{\prime}$ $$ g(x)={\frac{f(x)}{x}}\qquad(x>0) $$ and prove that g is monotonically increasing 7. Suppose f'(x), g'(x) exist,g $(x)\neq0,$ and f(x) = g(x)= 0. Prove that $$ \operatorname*{lim}_{t arrow x}\frac{f(t)}{g(t)}=\frac{f^{\prime}(x)}{g^{\prime}(x)}. $$ (This holds also for complex functions.) 8. Suppose ${\mathcal{f}}^{\prime}$ is continuous on [a, b] and e>0. Prove that there exists >0 such that $$ {\frac{\left|f(t)-f(x)}{t-x}}-f^{\prime}(x)\right|<\varepsilon $$DeFERENTIATIoN 115 whenever 0<|t-x<8,a≤x≤b,a≤t≤b.(This could be expressed by saying that fis uniformly diferentiable on [a,b] if f’is continuous on [a, b].) Does this hold for vector-valued functions too? 9. Let fbe a continuous real function on R',of which it is known that f'(x) exists for all x ≠ 0 and that f(x)→3 as x→0、Does it follow that f'(0O) exists ? 10.Suppose f and g are complex differentiable functions on (0,1), f(x)→0, gOx)→0, J"(x)→ A,g(x)→Bas x→0, where A and Bare complex numbers, B ≠0. Prove that $$ \operatorname*{lim}_{x\to0}{\frac{f(x)}{g(x)}}={\frac{A}{B}}. $$ Compare with Example 5.18.Hint: $$ {\frac{f(x)}{g(x)}}=\left\{{\frac{f(x)}{x}}-A\right\}\cdot{\frac{x}{g(x)}}+A\cdot{\frac{x}{g(x)}}. $$ Apply Theorem 5.13 to the real and imaginary parts of f(x)/x and g(x)/x: 11. Suppose fis defined in a neighborhood of x, and suppose f“(x) exists. Show that $$ \operatorname*{lim}_{h^{\prime}\to0}{\frac{f(x+h)+f(x-h)-2f(x)}{h^{2}}}=f^{\prime}(x). $$ Show by an example that the limit may exist even if f"(x) does not Hint: Use Theorem 5.13 12.It /(x) =|x|", compute 厂"(x), F"(x) for all el x, and show that /*(0) does not exist 13. Suppose a and c are real numbers,c>0, and fis defined on [-1,1] by $$ f(x)={\binom{x^{a}\sin{(\left|x\right|^{-c})}}{(0}}\qquad{\mathrm{(if~}}x\neq0), $$ Prove the following statements: (a) f is continuous if and only if $a>0,$ (b) F(O) exists if and only if a >1. (c) f’is bounded if and only if a≥1- c. (d) f’is continuous if and only if $a>1+c.$ (e) f"(O) exists if and only if a> 2+c (f) f”is bounded if and only if a≥2+ 2c. (9)f”is continuous if and only if a >2十2c 14. Let fbe a differentiable real function defined in(a,b)、Prove that f is convex if and only if f’is monotonically increasing、Assume next that f"(x) exists for every xe(a,b), and prove that fis convex if and only if f“(x)≥0 for all x e(a,b) 15. Suppose a e R', is a twic-diffentabl real function on (α,の),and Mo,M,,M are the least upper bounds of lf(x)|、|f(x)|,J"(x)|,respectively,on (a,o). Prove that Mf≤4M。 M116 PRINCTPLES OF MATHEMATICAL ANALYSIS Hint: If h>0,Taylor's theorem shows that $$ f^{\prime}(x)={\frac{1}{2h}}\left[f(x+2h)-f(x)\right]-h f^{\prime\prime}(\xi) $$ for some s e(x,X+ 2h). Hence $$ |f^{\prime}(x)|\leq h M_{2}+{\frac{M_{o}}{h}}. $$ To show that $M_{1}^{2}=4M_{0}\,M_{2}$ can actually happen, take $a=-1,$ define $$ f(x)={\sqrt{x^{2}-1}}\qquad(-1<x<0), $$ and show that M。=1, M = 4,M, = 4. Does Mf≤4M。 M,hold for vector-valued functions too? 16. Suppose fis twice-diffrentiable on (0, 0), f”is bounded on (0, o) and /(x)→0 as x→O.Prove that f'(x)→0 as x→0 Hint: Let a→α in Exercise 15. 17. Suppose fis a real,tree times ifrentiable function on 【-1,1, such that $$ f(-1)=0,\;\;\;\;\;\;f(0)=0,\;\;\;\;\;\;f(1)=1,\;\;\;\;\;\;f^{\prime}(0): $$ =0. Prove that f(3(x) ≥3 for some $x\in(-1,1).$ Note that equality holds for $|(x^{3}+x^{2}).$ Hint: Use Theorem 5.15, with α = 0 and β = ±1, to show that there exist s e (0,1) and $t\in(-1,0)$ such that $$ f^{(3)}(s)+f^{(3)}(t)=6. $$ 18. Suppose fis a real function on [a,b] ${\mathcal{N}}$ is a positive integer, and f"- xsts for every te [a, b]. Let α,β, and ${\bar{\mathcal{B}}}$ be as in Taylor's theorem (5.15). Define $$ Q(t)={\frac{f(t)-f(\beta)}{t-\beta}} $$ for te [a, b),1 + β, differentiate $$ f(t)-f(\beta)=(t-\beta)Q(t) $$ n-1 times at $t\equiv\alpha,$ and derive the following version of Taylor's theorem: $$ f(\beta)=P(\beta)+{\frac{Q^{(n-1)}(x)}{(n-1)!}}\,(\beta-\alpha)^{n}. $$ 19。SuppOSe ${\mathcal{J}}$ is defined in(一1,1) and f(O) exists. Suppose -1<αn<β,<1, Cn $\ \: arrow0,$ ,and β. $\to0$ as n→0O、Define the difference quotients $$ D_{n}={\frac{f(B_{n})-f(\alpha_{n})}{\beta_{n}-\alpha_{n}}}\,. $$DIFFERENTIATION 117 Prove the following statements: (a) If αn<0<B.,then lim D,=f*(0). (b)If O<α,<βB,and {β,/(8.一α,)}is bounded, then lim D,=f/(0O). (c) If f’is continuous in(一1,1), then lim D,=f(O) Give an example in which fis differentiable in(一1,1)(but f'is not contin- uous at O) and in which cm,,βA tend to Oin such a way that lim Dnexists but is differ ent from f(0). 20.Formulate and prove an inequality which follows from Taylor's theorem and which remains valid for vector-valued functions. 21. Let $\overline{{\#\mathcal{H}}}$ be a closed subset of R. We saw in Exercise 22, Chap.4, that there is a real continuous function fon R whose zero set is E. Is it possible, for each closed set E,to find such an f which is differentiable on R,or one which is n times differentiable, or even one which has derivatives of all orders on $\textstyle{R^{1}}$ 22.Suppose f is a real function on(一O0,Go)Call x a fixed point of fif f(x) = X (a) If fis differentiable and f'()≠1 for every real t, prove that f has at most one fixed point. (b) Show that the function f defined by $$ f(t)=t+(1+e^{t})^{-1} $$ has no fixed point, although 0<f'(t)<1 for all real t. (c) However, if there is a constant A <1 such that |f(t)≤ A for all real t, prove that a fixed point x of f exists, and that $x=\operatorname*{lim}x_{n},$ where x is an arbitrary real number and $$ x_{n+1}=f(x_{n}) $$ for n =1, 2,3,. (d)) Show that the process described in (c) can be visualized by the zig-zag path $$ (x_{1},\,x_{2}) arrow(x_{2},\,x_{2}) arrow(x_{2}\,,\,x_{3}) arrow(x_{3}\,,\,x_{3}) arrow(x_{3}\,,\,x_{4}) arrow\cdot\cdot. $$ 23. The function f defined by $$ f(x)={\frac{x^{3}+1}{3}} $$ has three fixed points, say α,β,y, where $$ -2<\alpha<-1,~~~~~0<\beta<1,~~~~~1<\gamma<2. $$ For arbitrarily chosen x, define {x,} by setting Xn+1=/(xn) (a) If $\mathcal{X}\,_{\mathrm{I}}$ <α, prove that $x_{n}\to-\infty$ o as $n arrow\infty.$ (b)If α<X<Y, prove that xn→β as n→ o0. (c)If $\gamma<x_{\mathrm{n}}$ prove that xn→ +o as n→ o0. Thus β can be located by this method, but c and y cannot.118 PRINCIPLES OF MATHEMATICAL ANALYsis 24. The process described in part (c of Exercise 22 can of course also be applied to functions that map (O,o) to (0,oo) Fix some α >1, and put $$ f(x)={\frac{1}{2}}{\bigg(}x+{\frac{\alpha}{x}}{\bigg)},\qquad g(x)={\frac{\alpha+x}{1+x}}. $$ Both f and g have V& as their only fixed point in (0, o). Try to explain, on the basis of properties of f and g, why the convergence in Exercise 16,Chap. 3,is sc much more rapid than it is in Exercise 17.(Compare f' and o', draw the zig-zags suggested in Exercise 22.) Do the same when 0<α<1 25. Suppose fis twice diferntiable on L,b1, /(a)<0,/(b) >0,/(Ox)≥8> 0, and 0≤J"(x)≤M for all xe [a, b]. Let S be the unique point in (a,b) at which f(E) = 0. Complete the details in the following outline of Newton's method for com puting f. (a) Choose x∈(,b), and define {x) by $$ x_{n+1}=x_{n}-{\frac{f(x_{n})}{f^{\prime}(x_{n})}}. $$ Interpret this geometically, in termsof a tangent to the graph of f (b) Prove that $x_{n+1}<x_{n}$ and that $$ \operatorname*{lim}_{n\to\infty}x_{n}=\xi. $$ (c) Use Taylor's theorem to show that $$ x_{n+1}-\xi={\frac{f^{\prime\prime}(t_{n})}{2f^{\prime}(x_{n})}}\left(x_{n}-\xi\right)^{2} $$ for some t,∈(f,x) (d)If A = M/28, deduce that $$ 0\leq x_{n+1}-\xi\leq{\frac{1}{2}}\left[A(x_{1}-\xi)\right]^{2n}. $$ (Compare with Exercises 16 and 18,Chap. 3.) (e)Show that Newton's method amounts to finding a fixed point of the function g defined by $$ g(x)=x-{\frac{f(x)}{f^{\prime}(x)}}. $$ How does g(x) behave for x near S? (f)Put f(x) = x1 on(-O,oo) and try Newton's method. What happens?DIFFERENTIArIoN 119 26. Suppose fis differentiable on [a, b], f(a) == 0, and there is a real number A such that | f(x)|≤ A|f(Xx)| on [a, b]. Prove that f(x) $\mathbf{\partial}=\mathbf{0}$ for all x∈ [a,b]Hint: Fix Xo∈ [a,b], let $$ {\cal M}_{0}=\mathrm{sup}|\,f(x)|\,\!,\qquad{\cal M}_{1}=\mathrm{sup}|f^{\prime}(x)| $$ for a ≤x≤Xo. For any such x, l /(x)|≤M(xo -a)≤ A(Xo一 a)Mo. Hence M。= 0 if A(xo-a)<1. That is, f = 0 on [a, Xo]. Proceed. 27. Let p be a real function defined on a rectangle $\mathcal{F}\mathcal{F}$ in the plane, given by a≤x≤b, α≤y≤β. A solution of the initial-value problem $$ y^{\prime}=\phi(x,y),\qquad y(a)=c\qquad(\alpha\leq c\leq\beta) $$ is, by definition, a differentiable functionfon [a, b]such that f(a) = C,α≤f(x)≤β and $$ f^{\prime}(x)=\phi(x,f(x))\qquad(a\leq x\leq b). $$ Prove that such a problem has at most one solution if there is a constant A such that $$ |\phi(x,y_{2})-\phi(x,y_{1})|\leq A\,|y_{2}-y_{1}| $$ whenever (x, y)e R and (x,y2)∈ R. Hint: Apply Exercise 26 to the difference of two solutions. Note that this uniqueness theorem does not hold for the initial-value problem $$ y^{\prime}=y^{1/2},\qquad y(0)=0, $$ which has two solutions: f(x) = 0 and $f(x)=x^{2}/4$ Find all other solutions. 28. Formulate and prove an analogous uniqueness theorem for systems of differential equations of the form $$ y_{j}^{\prime}=\phi_{j}(x,y_{1},\ldots,y_{k}),\qquad y_{j}(a)=c_{j}\qquad(j=1,\ldots,\,k). $$ Note that this can be rewriten in the form $$ \mathbf{y}^{\prime}=\Phi(x,\mathbf{y}),\qquad\mathbf{y}(a)=\mathbf{c} $$ where $\mathbf{y}=(y_{1},\cdot\cdot\cdot,y_{k})$ ranges over a k-cell, $\xi_{\beta}^{\mathsf{H}}\mathcal{\}}$ is the mapping of a(k + 1)-cell into the Euclidean ${\bar{\cal K}},$ space whose components are the functions pn, ,gbk,and c is the vector Cc,…,Cc)、Use Exrcise 26, or vector-valued functions 29. Specialize Exercise 28 by considering the system $$ \begin{array}{c}{{y^{\prime}=y_{I+1}\qquad(j=1,\ \dots,\ k-1),}}\\ {{y_{k}^{\prime}=f(x)-\sum_{j=1}^{k}{g_{j}(x)y_{j}\,,}}}\end{array} $$ where ,g1, .,9nare continuous real functions on [a, b], and derive a uniqueness theorem for solutions of the equation y $$ {}^{(k)}+g_{k}(x)y^{(k-1)}+\cdot\cdot\cdot+g_{2}(x)y^{\prime}+g_{1}(x)y=f(x) $$ subject to initial conditions $$ y(a)=c_{1},\qquad y^{\prime}(a)=c_{2},\qquad\cdot\cdot\cdot\cdot\qquad y^{(k-1)}(a)=c_{k}\,. $$6 THE RIEMANN-STIELTJES INTEGRAL The present chapter is based on a definition of the Riemann integral which depends very explicitly on the order structure of the real line.Accordingly we begin by discussing integration of real-valued functions on intervals.EX tensions to complex- and vector-valued functions on intervals follow in later sections. Integration over sets other than intervals is discussed in Chaps. 10 and 11. DEFINITION AND EXISTENCE OF THE INTEGRAL 6.1 Definition Let [a,b]be a given interval. By a partition P of [a,句 we mean a finite set of points xo, X,,.…X,,where $$ a\,=x_{0}\leq x_{1}\leq\,\cdot\,\cdot\,\leq x_{n-1}\leq x_{n}=b. $$ We write Ax = X1一 Xi-1 (i= 1,.….,n)THE RIEMANN-STIEL IES INTECRAL 121 Now suppose $\mathbb{P}$ is a bounded real function defined on [α,J、 Corresponding to each partition $\ D$ of [a,b] we put $$ \begin{array}{r l}{M_{i}=\operatorname*{sup}f(x)}&{{}\quad(x_{i-1}\leq x\leq x_{i}),}\\ {m_{i}=\operatorname*{inf}f(x)}&{{}\quad(x_{i-1}\leq x\leq x_{i}),}\\ {\operatorname*{}}\\ {I(P,f)=\sum_{i=1}^{n}M_{i}\,\Delta x_{i}\,,}\end{array} $$ and finally (2) $$ \begin{array}{l l}{{\overline{{\sharp}}_{a}^{b}f d x=\mathrm{inf}\ U(P,f),}}\\ {{\overline{{\ K}}_{a}^{b}f d x=\mathrm{sup}\ L(P,f),}}\end{array} $$ (1) where the inf and the sup are taken over all partitions P of [a,b]. The lef members of(1) and(2) are called the wpper and lower Riemann integrals of ) over [a, b], respectively. is Riemann if the upper and lower integrals are equal, we say that $\mathbb{P}$ integrable on a, b],we write f∈ O%((that is, JR denotes the set of Riemann integrable functions), and we denote the common value of (1) and (2) by (3) $$ {\mathfrak{l}}_{a}^{b}f\,d x, $$ or by (4) $$ \textstyle\int_{a}^{b}f(x)\,d x. $$ This is the Riemann integral of f over [a,b].Since f is bounded, there exist two numbers, m and M, such that $$ m\le f(x)\le M\qquad(a\le x\le b). $$ Hence, for every P, $$ m(b-a)\leq L(P,f)\leq U(P,f)\leq M(b-a), $$ so that the numbers L(P,/) and U(P,/) form a bounded st. This shows that the upper and lower integrals are defined for every bounded function f The question of their equality,and hence the question of the integrability of f, is a more delicate one. Instead of investigating it separately for the Riemann integral we shall immediately consider a more general situation.122 PRUINCIPLES OF MATHEMATICAL ANALYSIS 6.2 Defnition Let α be a monotonically increasing function on [a,b](since α(a) and α(b) are finite, it follows that cis bounded on [a,b). Corresponding to each partition P of [a, b], we write $$ \Delta\alpha_{i}=\alpha(x_{i})-\alpha(x_{i-1}). $$ It is clear that $\Delta{\boldsymbol{\alpha}}_{i}$ ≥0. For any real function $\oint_{}^{}d$ which is bounded on [a,b we put $$ \begin{array}{l c r}{{U(P,f,\alpha)=\displaystyle\sum_{i=1}^{n}M_{i}\,\Delta\alpha_{i},}}\\ {{{\cal L}(P,f,\alpha)=\displaystyle\sum_{i=1}^{n}m_{i}\,\Delta\alpha_{i},}}\end{array} $$ where $M_{i},$ m, have the same meaning as in Definition 6.1,and we define (6) $$ \textstyle\int_{a}^{b}f\,d x=\operatorname*{inf}_{b}U(P,f,x), $$ (5) the inf and sup again being taken over all partitions If the left members of(5)and (6) are equal, we denote their common value by (7) $$ \textstyle\int_{a}^{b}f\,d x $$ or sometimes by (8) $$ \textstyle\int_{a}^{b}f(x)\,d x(x). $$ This is the Riemann-Stieltjes integral (or simply the Stieltjes integral) of f with respect to α, over [a,b]. 1f (T) exists, i.e.,if (S) and(6) are equal, we say that fis integrable with respect to α, in the Riemann sense, and write fe SR(α). By taking $\alpha(x)=x,$ the Riemann integral is seen to be a special case of the Riemann-Stielties integral. Let us mention explicitly, however, that in the general case a need not even be continuous. A few words should be said about the notation. We prefer (T) to(8),since the letter x which appears in(8) adds nothing to the content of (T). It is im- material whc letter we use to reprsent the so-caled "variable of integration.’ For instance,(8) is the same as $$ \textstyle\int_{a}^{b}f(y)\,d x(y). $$