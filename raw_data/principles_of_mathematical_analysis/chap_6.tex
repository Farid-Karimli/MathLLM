THB RIEMAN-STiELTJEBS INTEORAL 123 The integral depends on f,α,a and ${\overline{{\mathcal{D}}}}_{9}$ but not on the variable of integration, which may as well be omitted The role played by the variable of integration is quite analogous to that of the index of summation: The two symbols $$ \sum_{i=1}^{n}c_{i},\qquad\sum_{k=1}^{n}c_{k} $$ mean the same thing, since each means $c_{1}+c_{2}+\,\cdot\,\cdot\,\cdot\,+\,c_{n}\,.$ Of course, no harm is done by inserting the variable of integration, and in many cases it is actually convenient to do so. We shall now investigate the existence of the integral (T). Without saying so every time, f will be assumed real and bounded, and a monotonically increas ing on [a, b]; and, when there can be no misunderstanding, we shall writefir place 6.3 Definition We say that the partition ${\boldsymbol{P}}^{\star}$ is a refinement of P if P* = P (that is, if every point of $\bar{\cal M}$ $\textstyle P_{1}$ and P, is a point of P*). Given two partitions, F we say that P* is their common refinement if $P^{*}=P_{1}\cup P_{2}$ 6.4 Theorem If P* is a refinement of P,then (10) $$ \begin{array}{c l c r}{{{\cal L}(P,f,\alpha)\leq{\cal L}(P^{*},f,\alpha)}}\\ {{}}&{{}}&{{U(P^{*},f,\alpha)\leq U(P,f,\alpha).}}\end{array} $$ (9) and Proof To prove (9),suppose first that ${\mathcal{P}}^{\star}$ contains just one point more than P. Let this extra point be x*, and suppose x:-1<x*之x,, where X;-1 and x, are two consecutive points of ${\mathcal{P}},$ Put $$ \begin{array}{r l}{w_{1}=\operatorname*{inf}f(x)}&{{}\quad(x_{i-1}\leq x\leq x)^{*}}\\ {w_{2}=\operatorname*{inf}f(x)}&{{}\quad(x^{*}\leq x\leq x_{i}).}\end{array} $$ ), Clearly w ≥ m, and $w_{2}\geq m_{i},$ where, as before, Hence $$ m_{i}=\operatorname*{inf}f(x)\qquad(x_{i-1}\leq x\leq x_{i}) $$ L(P $$ \begin{array}{l}{{\vphantom[\bullet_{i}f,\alpha)-L(P_{i}f,\alpha)}}\\ {{\qquad\qquad\qquad\qquad\qquad\qquad\qquad\qquad\qquad\qquad\qquad\qquad\qquad\qquad\qquad\qquad\qquad\qquad\qquad\qquad\qquad\qquad\qquad\qquad\qquad\qquad\qquad\qquad\qquad\qquad\qquad\qquad\qquad\qquad\qquad\qquad\qquad\qquad\qquad\qquad\qquad\qquad\qquad\qquad\qquad\qquad\qquad\qquad\qquad\qquad\qquad\qquad\qquad\qquad\qquad\qquad\qquad\qquad\qquad\qquad\qquad\qquad\qquad\qquad\qquad\qquad\qquad\qquad\qquad\qquad\qquad\qquad\qquad\qquad\qquad\qquad\qquad\qquad\qquad\qquad\qquad\qquad\qquad\qquad\qquad\qquad\qquad\qquad\qquad\qquad\qquad\qquad\qquad\qquad(\qquad\qquad\qquad\qquad\qquad\qquad\qquad\qquad\qquad\qquad\qquad\qquad\qquad\qquad\qquad\qquad\qquad\qquad\qquad\qquad\qquad\qquad\qquad\qquad(\qquad\qquad\qquad\qquad\qquad\qquad\qquad\qquad\qquad\qquad\qquad\qquad\qquad\qquad\qquad\qquad\qquad\qquad\qquad\qquad\qquad(\qquad\qquad\qquad\qquad\qquad\qquad}\qquad\qquad\qquad\qquad\qquad\qquad\qquad\qquad\qquad\qquad\qquad\qquad\qquad\qquad\qquad\qquad\qquad\qquad\qquad\qquad\qquad\qquad\qquad\qquad\qquad\qquad\qquad\qquad\qquad\qquad\qquad\qquad\qquad(\qquad\qquad\qquad\qquad\qquad\qquad\qquad\qquad\qquad\qquad\qquad\qquad\qquad\qquad(\qquad\qquad\qquad\qquad\qquad\qquad\qquad\qquad}^{\qquad\qquad(\qquad\qquad\qquad\qquad(\qquad}\qquad\qquad(\qquad(\qquad\qquad\qquad(1}&{\qquad\qquad\qquad\qquad\qquad\qquad\qquad\qquad\qquad\qquad\qquad\qquad\ $$ If P* contains k points more than P,we repeat this reasoning k times, and arrive at (9). The proof of (10) is analogous.124 PRINCIPLES OF MATHEMATICAL ANALYSIs 6.5 Theorem Proof Let P* be the common refinement of two partitions P, and P, By Theorem 6.4, $$ L(P_{1},f,\alpha)\leq L(P^{*},f,\alpha)\leq U(P^{*},f,\alpha)\leq U(P_{2},f,\alpha) $$ Hence (11) $$ L(P_{1},f,\alpha)\leq U(P_{2},f,\alpha). $$ lf P, is fixed and the sup is taken over all P,(11) gives (12) $$ \textstyle\int_{-}f\,d x\leq U(P_{2},f,\alpha). $$ The theorem follows by taking the inf over all $\textstyle P_{2}$ in (12) 6.6 Theorem f∈ JRCc) on [a,b]if and only if for every 8> O there exists a partition P such that (13) $$ U(P,f,\alpha)-L(P,f,\alpha)<\varepsilon $$ Proof For every $\bar{\mathcal{D}}$ we have $$ L(P,f,\alpha)\leq\int f\,d x\leq{\overline{{\ |}}}f\,d x\leq U(P,f,\alpha). $$ Thus (13) implies $$ 0\leq{\overline{{\coprod}}}f\,d x-\bigcup_{x}f\,d x<s. $$ Hence,if(13) can be satisfied for every &>0, we have $$ \hat{\mathbf{I}}f d x=\hat{\mathbf{i}}f d x, $$ that is, f∈ X(α) Conversely, suppose fe XR(α), and let s> 0 be given. Then there exist partitions $\ P_{1}$ and $\textstyle P_{2}$ such that (15) $$ \begin{array}{c}{{U(P_{2},f,\alpha)-\displaystyle\int f d\alpha<{\frac{\varepsilon}{2}},}}\\ {{\vdots}}\\ {{\int f d\alpha-L(P_{1},f,\alpha)<{\frac{\varepsilon}{2}}.}}\end{array} $$ (14)THE RIEMANN-STIELTJES INTEGRAL 125 We choose $\bar{\mathcal{D}}$ to be the common refinement of $\textstyle{P_{1}}$ and P,,Then Theorem 6.4,together with(14) and (15), shows that $$ U(P,f,\alpha)\leq U(P_{2},f,\alpha)<\int f\,d x+{\frac{s}{2}}<L(P_{1},f,\alpha)+\varepsilon\leq L(P,f,\alpha)+\varepsilon, $$ so that (13) holds for this partition $\bar{\mathcal{D}}$ Theorem 6.6 furnishes a convenient criterion for integrability. Before we apply it, we state some closely related facts. 6.7 Theorem (a)I(13) holds for some $\widehat{\cal P}$ and some s, then (13) holds (with the same e) for every refinement of P (b) If(13) holds for $P=\{x_{0}\,,\,\cdot\,\cdot\,,\,x_{n}\}$ and if s;, ${\Big\langle}{\Big\rangle}_{\hat{l}}$ are arbitrary points in Lxr-1, x., then $$ \sum_{i=1}^{n}|f(s_{i})-f(t_{i})|\,\,\Delta\alpha_{i}<\varepsilon. $$ (c)If f∈ SW(c) and the hypotheses of (b) hold, then $$ \left|\sum_{i=1}^{n}f(t_{i})\Delta a_{i}-\int_{a}^{b}f\ d x\right|<\varepsilon. $$ Proof Theorem 6.4 implies (a).Under the assumptions made in(b) both f(s) and f(t) lie in [m;,M;], so that |f(S,)一f(t)≤ M;- m,.Thus $$ \begin{array}{l c r}{{\binom{n}{\displaystyle i=1}}\{f(s_{i})-f(t_{i})|\,\,\,\Delta x_{i}\leq U(P,f,\alpha)-L(P,f,\alpha),}}\end{array} $$ which proves (b). The obvious inequalities and $$ L(P,f,\alpha)\leq\sum f(t_{i})\,\Delta\alpha_{i}\leq\ U(P,f,\alpha) $$ $$ L(P,f,\alpha)\leq\operatorname{J}f d\alpha\leq U(P,f,\alpha) $$ prove (c). 6.8 Theorem Iffis coninuous on [a, b] then fe OI(α)on [a,b1 Proof Let c>0 be given. Choose n > 0 so that $$ [\alpha(b)-\alpha(a)]\eta<\varepsilon. $$ Since f is uniformly continuous on [a,b}(Theorem 4.19), there exists a 8 > 0 such that (16) $$ |f(x)-f(t)|<\eta $$126 PaINCTPLEs Oor MATHEMATICAL ANALYsis if xe [a,b], t∈ [a,b], and |x- t|<6. If P is any partition of [a,b] such that Ax,< for all i, then(16) implies that (17) $$ M_{i}-m_{i}\leq\eta\qquad(i-1,\,\ldots,\,n) $$ and therefore $$ U(P,f,\alpha)-L(P,f,\alpha)=\sum_{i=1}^{n}\left(M_{i}-m_{i}\right)\Delta\alpha_{i} $$ $$ \leq\eta\sum_{i=1}^{n}\Delta x_{i}=\eta[\alpha(b)-\alpha(a)]<\kappa. $$ By Theorem 6.6, fe g(α) 6.9 Theorem Iff is monotonic on [α,b], and if α is continwows on [a, b), then fe S2(oa).(We still assume,of course, that α is monotonic.) Proof Lete>0 be given. For any positive integern, choose a partition such that $$ \Delta\alpha_{i}=\frac{\alpha(b)-\alpha(a)}{n}\qquad(i=1,\dots,n). $$ This is possible since αis continuous (Theorem 4.23) We suppose that fis monotonically increasing the proof is analogous in the other case). Then so that $$ M_{i}=f(x_{i}),\qquad m_{i}=f(x_{i-1})\qquad(i=1,\,\ldots,n), $$ $$ U(P,f,\alpha)-L(P,f,\alpha)={\frac{\alpha(b)-\alpha(a)}{n}}\sum_{i=1}^{n}\,[f(x_{i})-f(x_{i-1})] $$ $$ ={\frac{\alpha(b)-x(a)}{n}}\cdot\left[f(b)-f(a)\right]<t $$ if n is taken large enough. By Theorem 6.6, fe UP(αa) 6.10 Theorem Suppose f is bounded on [α,b],f has only finitely many point of discontinuity on [a,b],and α is continuous at every point at which f is discon tinuous. Then fe S(α) Proof Lete>0 be given. Put M = sup Jf(x), 1et ${\widehat{\mathcal{N}}}$ be the set of points at which f is discontinuous. Since ${\widehat{\mathcal{O}}}^{\mathcal{V}}$ is finite and α is continuous at every point of E, we can cover E by finitely many disjoint intervals [u,,vy]一 [a, b] such that the sum of the corresponding differences $\alpha(v_{j})$ - α(u)is less than E.Furthermore, we can place these intervals in such a way that every point of En(α,b) lies in the interior of some lw, ,1THE RIEMANN-STIELTJES INTEGRAL 127 Remove the segments uj,vy)from [a,bJ、 The remaining set $\textstyle K$ is compact. Hence f is uniformly continuous on $K_{\!\cdot\!}$ and there exists > 0 Each ${\hat{d}}{\hat{d}}_{\cdot}$ such that |J(s)-Ju)|<e if se K, te K,|s -1<员 Now form a partition P = {×0,x,….…,} of [a,如],as follows: , occurs in P. Each v, occurs in P.No point of any segment (u,,v,) $x_{i-1}$ occurs in P. If x $x_{i-1}$ is not one of the u;,then $u_{j\,\cdot}$ $\Delta x_{i}<\delta$ ≤s unless Note that M,- m, ≤ 2M for every i, and that $M_{i}-m_{i}$ is one of the w、 Hence, as in the proof of Theorem 6.8 $$ U(P,f,\alpha)-L(P,f,\alpha)\leq[\alpha(b)-\alpha(a)]\varepsilon+2M\varepsilon $$ Since e is arbitrary, Theorem 6.6 shows that fe O2(c). Note: If f and a have a common point of discontinuity, then f need not be in JX(c).Exercise 3 shows this. 6.11 Theorem Suppose fe M(G)on l,b),m ≤/≤ M,pis continwous on [m, M], and h(x) = p(f(x)) on [a,b] Then h e S(α) on [a,b] Proot Choose e > 0. Since p is uniformly continuous on [m,M), ther exists员> 0 such that 8<: and 」p(8)- p1t)」<: if 」s -tl ≤6 and s,t∈ [m,M] that Since fe OI(a), there is a partition P ={xo,x,.…x) of [Q,b) such (18) $$ U(P,f,\alpha)-L(P,f,\alpha)<\delta^{2}. $$ Let M,,,m, have the same meaning as in Definition 6.1, and let M;", mt be the analogous numbers for h.Divide the numbers 1,...,n into two classes: ie A if $M_{i}-m_{i}<\delta,\;i\in B$ if M,- m, ≥6. Forie A,our choice of S shows that M,- mt ≤ :. For ie B,M:* - m* ≤ 2K, where K = suplp(t),m ≤t≤ M. By (18), we have (19) $$ \delta\!\sum_{i\in B}\Delta\alpha_{i}\leq\!\sum_{i\in B}(M_{i}-m_{i})\,\Delta\alpha_{i}<\delta^{2} $$ so that $\textstyle\sum_{i\in B}\Delta\alpha_{i}<\delta.$ t follows that $$ \begin{array}{c}{{U(P,h,\alpha)-L(P,h,\alpha)=\sum_{i\in A}(M_{i}^{*}-m_{i}^{*})\,\Delta\alpha_{i}+\sum_{i\in B}(M_{i}^{*}-m_{i}^{*})\,\Delta\alpha_{i}}}\\ {{\ \ \ \ \ \ \ \ \mathbf{~}\leq s[\alpha(b)-\alpha(a)]+2K\delta<\varepsilon[\alpha(b)-\alpha(a)+2K]}}\end{array} $$ Since e was arbitrary, Theorem 6.6 implies that he O2(α) Remark:This theorem suggests the question: Just what functions are Riemann-integrable?The answer is given by Theorem 1336).128 PRINCIPLEs Or MATHEMATICAL ANALYsIs PROPERTIES OF THE INTEGRAL 6.12Theorem (a)If fe SW(o) and f, e OM(α) on [a, b], then $$ f_{1}+f_{2}\in{\mathcal{R}}(\alpha), $$ cf e P(o) for every constant ${\mathcal{C}}_{!}$ and $$ \int_{a}^{b}(f_{1}+f_{2})\,d x=\int_{a}^{b}\!f_{1}\,d\alpha+\int_{a}^{b}\!f_{2}\,d\alpha, $$ $$ \bigcap_{a}^{b}c f d x=c\int_{a}^{b}f d x. $$ (b)If f,(x)≤/,(x)on [a,b], then $$ \textstyle\int_{a}^{b}f_{1}\,d\alpha\leq\int_{a}^{b}f_{2}\,d\alpha. $$ [c,b], and (c)If fe SM(c) on [a, b] and if $\iota<c<b,$ then f∈ 8r(c) on [a,c] and on $$ \bigcap_{a}^{c}f\,d\alpha+\int_{c}^{b}f\,d\alpha=\int_{a}^{b}f\,d\alpha. $$ (d)If fe MI(a) on [a, b] and f|f(x)| ≤ M on [a, b), then $$ \left|\int_{a}^{b}f\,d x\right|\leq M[\alpha(b)-\alpha(a)]. $$ (e)If f∈ OR(α) and f∈ SY(α2), then fe SY(α + α,) and $$ \textstyle\int_{a}^{b}f d(\alpha_{1}+\alpha_{2})=\int_{a}^{b}f d\alpha_{1}+\int_{a}^{b}f d\alpha_{2}\ ; $$ if f∈ (α) and c is a positive constant, then fe S(cc) and $$ \textstyle\int_{a}^{b}f\,d(c\alpha)=c\int_{a}^{b}f\,d x. $$ Proof If f = f1 +fz and $\bar{\mathcal{D}}$ is any partition of $(a,b)$ we have $$ \begin{array}{r l}{(20)\qquad{\begin{array}{r l}{L(P,f_{1},\alpha)+L(P,f_{2},\alpha)\leq L(P,f,\alpha)}\\ {\leq U(P,f,\alpha)\leq U(P,f,\alpha)\leq U(P,f_{1},\alpha)+U(P,f_{2},\alpha).}\end{array}}}\end{array} $$ If fe 3R(c)and f,e OR(x),Iet e > 0 be given. There are partitions P (j= 1,2) such that $$ U(P_{j},f_{j},\alpha)-L(P_{j},f_{j},\alpha)<\varepsilon. $$THE RIEMANN-STELTJES INTEGRAL 129 These inequalities persist if $\ P_{1}$ and $\textstyle P_{2}$ are replaced by their common refinement $\widehat{\cal P}$ .Then (20) implies $$ U(P,f,\alpha)-L(P,f,\alpha)<2s_{i} $$ which proves that fe SY(α) With this same $\bar{\mathcal{D}}$ we have $$ U(P,f_{J},\alpha)<{\int}f_{j}\,d\alpha+s\;\;\;\;\;\;(j=1,?k) $$ 2); hence (20) implies $$ \textstyle\int f\,d\alpha\leq U(P,f,\alpha)<\textstyle\int f_{1}\,\,d\alpha+\textstyle\int f_{2}\,\,d\alpha+2\varepsilon $$ Since & was arbitrary, we conclude that (21) $$ \textstyle\int f\,d x\leq\textstyle\int f_{1}\,d x+\textstyle\int f_{2}\,d x. $$ If we replace f, and fz in(21)by -f, and -f,the inequality is reversed, and the equality is proved. The proofs of the other assertions of Theorem 6.12 are so similar that we omit the details. In part c the point is that (by passing to refine- ments) we may restrict ourselves to partitions which contain the point c in approximating f da. 6.13 Theorem If fe SW(a) and ge S%(C)on [a,b), then (a) fg e SR(α), (6)If| e .W(a) anc f」 du Proof If we take dp(t) = 1-,Theorem 6.11 shows that fl e S(α) if fe S(o) The identity $$ 4f g=(f+g)^{2}-(f-g)^{2} $$ completes the proof of(a) If we take b(t)=|t|,Theorem 6.11 shows similarly that ffl e SP(α) Choose c = ±1,so that c Jfda ≥0. Then Sf dα」= cJf dα =(cf do ≤Gf」 do since cf ≤ If 6.14 Definition The uni step function $\textstyle{\mathcal{P}}$ is deined by $$ I(x)={\binom{0}{1}}\quad\quad(x\leq0), $$130 PRINCIPLES OF MATHEMATICAL ANALYsIs 6.15 Theorem If a<s<bf is bounded on [a,bJ,f is continuous at s, and α(x) = I(x - s),then $$ \textstyle\int_{a}^{b}f\,d x=f(s). $$ Proof Consider partitions P={xo,X,X,,x)},where xo = 、and $x_{1}=s<x_{2}<x_{3}=b$ Then $$ U(P,f,\alpha)=M_{2}\,,\qquad L(P,f,\alpha)=m_{2}\,. $$ Since fis continuous at s, we see that $\therefore1_{2}$ and $m_{2}$ converge to f(s)as X2→s 6.16 Theorem Suppose ${\mathcal{C}}_{\gamma_{j}}$ ≥0 for 1,2,3,... $\Sigma C_{n}$ converges,{s, is a sequence of distinct points in (a,b), and (22) $$ \alpha(x)=\sum_{n=1}^{\infty}c_{n}I(x-s_{n}). $$ Let f be continuous on [a,b]. Then (23) $$ \textstyle\int_{a}^{b}f\,d x=\sum_{n=1}^{\infty}c_{n}f(s_{n}). $$ Proof The comparison test shows that the series(22) converges for every x.Its sum α(x) is evidently monotonic, and $x(a)=0,\ x(b)=\Sigma c_{n}$ (This is the type of function that occurred in Remark 4.31.) Let e>0 be given, and choose $\mathcal{N}$ so that $$ \textstyle{\frac{\pi}{g\]}}c_{n}<s. $$ Put $$ \alpha_{1}(x)=\sum_{n=1}^{N}c_{n}I(x-s_{n}),\qquad\alpha_{2}(x)=\sum_{N\mp1}^{\infty}c_{n}I(x-s_{n}). $$ By Theorems 6.12 and 6.15 (24) $$ \textstyle\bigcap_{a}^{b}f\,d a_{1}=\sum_{i=1}^{N}c_{n}f(s_{n}). $$ Since αx,(b)-αz(a)<。, (25) $$ \left|\right|_{a}^{b}f\,d\alpha_{2} |\leq M\varepsilon, $$THE RIEMAN-STIELTJEs INTECRAL131 that where M = suplf(x)、Since α = 1 + α,,it follows from (24) and (25) (26) $$ \left|\uparrow_{a}^{b}f\,d\alpha-\triangle_{i=1}^{N}c_{n}f(s_{n})\right|\leq M\varepsilon. $$ If we let N→0o, we obtain(23) be a bounded real function on [a,b 6.17 Theorem Assume α increases monotonically and u'∈ JR on [a,b] Let Then fe SM(a) if and only if fu’e S0、In that case (27) $$ \textstyle\int_{a}^{b}f\,d x=\int_{a}^{b}f(x)x^{\prime}(x)\,d x. $$ tition Proof Let é> 0 be given and apply Theorem 6.6 to α': There is a par- $P=\{x_{0}\,,\,\ldots,x_{n}\}$ of [a,b] such that (28) $$ U(P,\alpha^{\prime})-L(P,\alpha^{\prime})<\kappa. $$ The mean value theorem furnishes points ${\overline{{\int}}}_{\underline{{j}}}$ e LKxi-1,x] such tha $$ \Delta{x}_{i}=\alpha^{\prime}(t_{i})\,\Delta x_{i} $$ for i= 1,.….,.1f $s_{i}\in[x_{i-1},x_{i}],$ , then (29) $$ \sum_{i=1}^{n}\left|\alpha^{\prime}(s_{i})-\alpha^{\prime}(t_{i})\right|\,\Delta x_{i}<s, $$ by (28) and Theorem 6.7(b)、 Put M = supl fOx)|. Since $$ \begin{array}{l l}{{\displaystyle{\sum_{i=1}^{n}f(s_{i})\Delta\alpha_{i}=\sum_{i=1}^{n}f(s_{i})\alpha^{\prime}(t_{i})\,\Delta x_{i}}}\end{array} $$ it follows from(29) tha (30) $$ \left|\sum_{i=1}^{n}f(s_{i})\,\Delta\alpha_{i}-\sum_{i=1}^{n}f(s_{i})\alpha^{\prime}(s_{i})\,\Delta x_{i}\right|\leq M\varepsilon. $$ In particular, for alchoices of s, $$ \begin{array}{l}{{\frac{1}{\epsilon=1}f(s_{i})\,\Delta\alpha_{i}\leq U(P,f\alpha^{\prime})+M\varepsilon,}}\\ {{\ =\,\left.\varepsilon [x_{i-1},x_{i}\right],\,\sin\mathrm{that}}}\\ {{ .\ }}&{{\ddots}}\end{array} $$ The same argument leads from (30) to Thus U(P,fa')≤ U(P, f,α) + Me. (31) l U(P,.f,x)- U(P, fa’)| ≤ Me.132 PRINCTPLES OF MATHEMATCAL ANALYSIS Now note that (28) remains true if $\ D$ is replaced by any refinement Hence (31) also remains true.We conclude that $$ \left|\overline{{{\ D}}}_{a}^{b}f d\alpha-\overline{{{\int_{a}^{b}}}}f(x)x^{\prime}(x)\,d x\right|\leq M\varepsilon. $$ But sis arbitrary. Hence (32) $$ \overline{{{\int_{a}^{b}}}}f\,d x=\overline{{{\int_{a}^{b}}}}f(x)x^{\prime}(x)\,d x, $$ for any bounded f.The equality of the lower integrals follows from (30 in exactly the same way.The theorem follows. 6.18 Remark The two preceding theorems illustrate the generality and flexibility which are inherent in the Stieltjes process of integration. If c is a pure step function [this is the name often given to functions of the form(22)], the integral reduces to a finite or infinite series. If α has an integrable derivative the integral reduces to an ordinary Riemann integral. This makes it possibl in many cases to study series and integrals simultaneously, rather than separately To illustrate this point, consider a physical example. The moment or inertia of a straight wire of unit length, about an axis through an endpoint, at right angles to the wire, is (33) $$ \textstyle{\int_{0}^{1}}x^{2}\,d m $$ where m(x) is the mass contained in the interval [O,x]、If the wire is regarded as having a continuous density p,that is, if m′(x) = p(x), then (33) turns into (34) $$ \textstyle\int_{0}^{1}x^{2}\;\rho(x)\;d x. $$ On the other hand, if the wire is composed of masses ${\mathcal{N}}_{i}$ concentrated at points x,,(33) becomes (35) $$ \sum_{i}x_{i}^{2}\,m_{i}. $$ Thus (33) contains (34) and (35) as special cases, but it contains much more; for instance,the case in which m is continuous but not everywherc differentiable. 6.19 Theorem (change oft variabe)Suppose p is a strictly increasing contimwows function that maps an interval [A,B] onto [a, b]. Suppose α is monotonicall increasing on [a, b] amd Je stda) on [a,b]、 Define β and g on 【A,B)b (36) β(y) = α(p()), gO) = /(p(y))TE RUEMAN-sTELTrEs IrrcRAL 133 Then g ∈ SR(B) and (37) $$ \textstyle\int_{A}^{B}g\,d\beta=\int_{a}^{b}f\,d x. $$ Proof To each partition $P=\{x_{0}\,,\,\ldots,\,x_{n}\}$ of [a, b] corresponds a partition Q ={y。,…, y, ot [A, B], so that x,= 0(V)、All partitions ot LA, B1 are obtained in this way.Since the values taken by $\widehat{\mathcal{J}}$ on Lx:-1,x] are exactly the same as those taken by g on LVi-1, Jl, we see that (38) $$ U(Q,g,\beta)=U(P,f,\alpha),\qquad L(Q,g,\beta)=L(P,f,\alpha). $$ Since f∈ SR(c), P can be chosen so that both U(P,f, c) and L(P, f, α) are close toIf dc. Hence (38),combined with Theorem 6.6, shows that g e SR(β) and that (37) holds. This completes the proof Let us note the following special case: Take α(x) = x.Then β = p.Assume op’∈ Sr on [A,B].If Theorem 6.17 is applied to the left side of (37), we obtain (39) $$ \textstyle\int_{a}^{b}f(x)\,d x=\int_{A}^{B}f(\varphi(y))\varphi^{\prime}(y)\,d y. $$ INTEGRATION AND DIFFERENTIATION We stil confine ourselves to real functions in this section. We shall show that integration and differentiation are, in a certain sense, inverse operations. 6.20 Theorem Let fe &R on [a, b]、 For $a\leq x\leq b,p u$ $$ F(x)=\int_{a}^{x}f(t)\,d t. $$ Then ${\mathcal{R}}^{\nu}$ is continuous on [a,b)]; furthermore,if f is continuous at a point xo0f [a, b], then ${\mathcal{H}}^{\nu}$ is differentiable at xo,and $$ F^{\prime}(x_{0})=f(x_{0}). $$ If a ≤x<y≤b, then Proot Since fe LR,f is bounded. Suppose lf(t) ≤ M for a≤t≤b $$ |F(y)-F(x)|=\left|\int_{x}^{y}f(t)\,d t\right|\leq M(y-x), $$ by Theorem 6.12(c) and(d).Given &> 0, we see that $$ |F(y)-F(x)|<\varepsilon, $$134 PRINCIPLES OF MATHEMATICAL ANALYSIS provided that ly一x|<=/M. This proves continuity (and, in fact uniform continuity) of F. Now suppose fis continuous at xo. Given 8 > 0, choose > 0 such that $$ |f(t)-f(x_{0})|<\varepsilon $$ if $|t-x_{0}|<\delta,$ and a≤t<b.Hence, i xo- <s≤X0≤t<xo+6 and a≤s<t≤b, we have, by Theorem 6.12(d) $$ \left\vert{\frac{F(t)-F(s)}{t-s}}-f(x_{0})\right\vert=\left\vert{\frac{1}{t-s}}\int_{s}^{t}[f(u)-f(x_{0})]\,d u\right\vert<t_{0}^{2}\ . $$ 8. It follows that $F^{\prime}(x_{0})=f(x_{0}).$ 6.21 The fundamental theorem of calculusIf on [a,b] such that $F^{\prime}=f,$ then ${\mathcal{O}}]$ and if there is $\int_{0}^{T}H_{\mathbf{u}}(E_{\mathbf{u}})_{\mathbf{t}}(E_{\mathbf{u}})_{\mathbf{t}}(E_{\mathbf{u}})_{\mathbf{t}}(E_{\mathbf{u}})_{\mathbf{u}}(E_{\mathbf{u}})_{\mathbf{t}}(E_{\mathbf{u}})_{\mathbf{u}}(E_{\mathbf{u}})_{\mathbf{u}}(E_{\mathbf{u}})_{\mathbf{u}}(E_{\mathbf{u}})_{\mathbf{u}}(E_{\mathbf{u}})_{\mathbf{u}}(E_{\mathbf{u}})_{\mathbf{u}}(E_{\mathit{U}})_{\mathbf{u}}(E_{\mathit{U}})_{\mathit{U}}(E_{\mathbf{U}})_{\mathit{U}}(E_{\mathbf{U}}(E_{\mathbf{U}})_{\mathbf{U}}(E_{\mathcal{U}})_{\boldsymbol{U}}|K_{\mathcal{U}}(U(T(E_{U})$ ∈ SW on [a, a differentiable function ${\mathcal{F}}^{\gamma}$ $$ \textstyle\int_{a}^{b}f(x)\,d x=F(b)-F(a). $$ Proof LetB> 0 be given.Choose a partition P = {×0,.….,X,}of [a, b1 so that U(P,f)- L(P,f)< 8. The mean value theorem furnishes points t,∈ [X:-1, X] such that $$ F(x_{i})-F(x_{i-1})=f(t_{i})\ \Delta x_{i} $$ for i= 1,...,n.Thus $$ \displaystyle{\sum_{i=1}^{n}f(t_{i})\,\Delta x_{i}=F(b)-F(a).} $$ It now follows from Theorem 6.7(c) that $$ \left\vert F(b)-F(a)-\int_{a}^{b}f(x)\,d x\right\vert<\varepsilon. $$ Since this holds for every e> 0, the proof is complete. 6.22 Theorem (integration by parts)Suppose ${\mathcal{F}}^{\prime}$ and G are differentiaoble func- tions on [a,b], ${\boldsymbol{F}}^{\prime}$ =fe O,and $\scriptstyle G\neq g\gamma$ E 92.Then $$ \textstyle\int_{a}^{b}\!F(x)g(x)\,d x=F(b)G(b)-F(a)G(a)-\int_{a}^{b}\!f(x)G(x)\,d x. $$ Proof Put H(x) = F(x)G(Xx) and apply Theorem 6.21 to $\textstyle H$ and its deriv ative. Note that ${\bar{H}}^{\prime}$ ∈ S,by Theorem 6.13.THE RIEMAN-sTILTEs NTEGRAL 135 INTEGRATION OF VECTOR-VALUED FUNCTIONS 6.23 Definition Let f,...,./ be real functions on [a,b], and let f= (f,...,人 be the corresponding mapping of [a,b] into R*. If α increases monotonicall on [a, b], to say that f∈ SR(α) means that f,e SP(α) for j= 1,...,k.If this is the case, we define $$ \textstyle{\int_{a}^{b}}\mathbf{f}\,d x=\left(\int_{a}^{b}f_{1}\,d x,\,\ldots,\int_{a}^{b}f_{k}\,d x\right). $$ In other words, f dox is the point in $\textstyle{\mathcal{R}}^{k}$ whose jth coordinate is [f, dα It is clear that parts (a),(C), and (e) of Theorem 6.12 are valid for these vector-valued integrals; we simply apply the earlier results to each coordinate The same is true of Theorems 6.17,6.20,and 6.21.To illustrate, we state the analogue of Theorem 6.21. 6.24 Theorem Iff and F map [a, b] into $R^{k},$ iff∈ Jt on [a,b], and if F’= f, then $$ \textstyle\int_{a}^{b}\!f(t)\;d t=\operatorname{F}(b)-\operatorname{F}(a). $$ The analogue of Theorem 6.13(b) offers some new features, however, at least in its proof. 6.25 Theorem If t maps [a, b] into $\textstyle{\mathcal{R}}^{k}$ Rt and if f∈ 82(o) for some monotonically increasing function α on [a,b], then |fl e SR(c), and (40) $$ \left|\int_{a}^{b}{}_{b}\,d\alpha\right|\leq\int_{a}^{b}\left|\xi\right|\,d x. $$ Proof If f,… f are the components of f, then (41) $$ |\mathbb{F}|=(f_{1}^{2}+\cdot\cdot\cdot+f_{k}^{2})^{1/2}. $$ By Theorem 6.11, each of the functions f belongs to SR(α); hence so does their sum. Since x $\chi^{2}$ is a continuous function of x, Theorem 4.17 shows that the square-root function is continuous on [O,M], for every real $\mathcal{M}$ If we apply Theorem 6.11 once more,(41) shows that |fle OI(a) y = f dox, and To prove (40), put y =(y+, …, x) where y, = ,/, d. Then we have $$ |\mathbf{y}|^{2}=\sum y_{i}^{2}=\sum y_{j} \|f_{j}\,d x=\int(\sum y_{j}f_{j})\,d x. $$ By the Schwarz inequality, (42) $$ \sum y_{j}f_{j}(t)\leq|{\mathrm{y}}|\,|{\mathrm{f}}(t)|\quad\quad(a\leq t\leq b); $$136 PRINCIPLES OF MATHEMATICAL ANALYSIS hence Theorem 6.12(6) implies (43) lyl ≤|y「ifl d If y = 0,(40) is trivial. If y ≠ 0, division of(43) by lyl gives(40) RECTIFIABLE CURVES We conclude this chapter with a topic of geometric interest which provides ar application of some of the preceding theory. The case k = 2 (i.e.,the case of plane curves)is of considerable importance in the study of analytic functions of a complex variable 6.26 Definition A continuous mapping y of an interval [a, b] into $\textstyle{\mathcal{R}}^{k}$ is calle a curve in R*. To emphasize the parameter interval [a,b], we may also say that y is a curve on Ja,b1. If y is one-to-one, y is called an arc If y(a) = 2(b),y is said to be a closed curve. It should be noted that we define a curve to be a mapping, not a point set. Of course, with each curve y in $\textstyle{\mathcal{R}}^{k}$ there is associated a subset of R',namely the range of y,but different curves may have the same rarge. We associate to each partition $P=\{x_{0}\,,\,\ldots,\,x_{n}\}$ of [a, b] and to each curve y on [a, b] the number $$ \Lambda(P,\gamma)=\sum_{i=1}^{n}|\gamma(x_{i})-\gamma(x_{i-1})|. $$ The ith term in this sum is the distance (in R') between the points y(X:-1) and y(x)、 Hence A(P,y)is the length of a polygonal path with vertices at (xo) yix,.……,x),in tis order.As our partition becomes finer and finer, thi polygon approaches the range of y more and more closely. This makes it seem reasonable to define the length of yas $$ \Lambda(\gamma)=\operatorname{sup}\,\Lambda(P,\,\gamma), $$ where the supremum is taken over all partitions of [a,b]. If A(y)<Oo, we say that y is rectifiable. In certain cases,A(y) is given by a Riemann integral. We shall prove this for coninuouly difeniable curves, i., for curves y whose derivtive y'is continuous.THE RIEMANN-STlELTJES INTEGRAL 137 6.27 Theorem If’is continuous on [a,b], then y is rectifiable, and $$ \Lambda(\gamma)=\int_{a}^{b}|\gamma^{\prime}(t)|\;d t. $$ Proof 1f $a\leq x_{i-1}<x_{i}\leq b,$ then Hence $$ |\gamma(x_{i})-\gamma(x_{i-1})|=\left|\right|_{x_{i-1}}^{x_{i}}\gamma^{\prime}(t)\,d t |\leq\int_{x_{i-1}}^{x_{i}}|\gamma^{\prime}(t)|\,d t. $$ $$ \Lambda(P,\,\gamma)\,\leq\,\int_{a}^{b}|\gamma^{\prime}(t)|\,\,d t $$ for every partition $\underline{{{\iint^{3}}}}$ of [a, b]. Consequently, $$ \Lambda(\gamma)\leq\int_{a}^{b}|\gamma^{\prime}(t)|\ d t. $$ To prove the opposite inequality,let &> 0 be given.Since’is uniformly continuous on [a,b], there exists > 0 such that $$ |\,\gamma^{\prime}(s)-\gamma^{\prime}(t)|<\varepsilon\qquad\mathrm{i}\mathrm{f}\ |s-t|<\delta. $$ Let $P=\{x_{0}\,,\,\ldots,x_{n}\}$ be a partition of [a,b], with $\Delta x_{i}<\delta$ for all i. If Xi-1 S1≤ $X_{i}\,.$ ;,it follows that Hence ly(t)|≤「y'(x) + e. $$ \begin{array}{r l}{\prime(t)|\,d t\leq\left|\,\gamma^{\prime}(x_{i})\right|\,\Delta x_{i}+\varepsilon\,\,\Delta x_{i}}\\ {\ }&{{}=\left|\int_{x_{i}-1}^{x_{i}}\gamma^{\prime}(t)+\gamma^{\prime}(x_{i})-\,\gamma^{\prime}(t)\right|\,d t |+\varepsilon\,\Delta x_{i}}\\ {\leq\left|\,\int_{x_{i}-1}^{x_{i}}\gamma^{\prime}(t)\,d t\right|+\left|\int_{x_{i}-1}^{x_{i}}\left|\gamma^{\prime}(x_{i})-\gamma^{\prime}(t)\right|\,d t\right|+\varepsilon\,\,\Delta x_{i} | |+\varepsilon\,\Delta x_{i}\,,}\end{array} $$ If we add these inequalities, we obtain Since s was arbitrary $$ \begin{array}{r l}{\int_{a}^{b}|\gamma^{\prime}(t)|\,d t\leq\Lambda(P,\gamma)+2s(b-a)}\\ {\leq\,\Lambda(\gamma)+2s(b-a).}\end{array} $$ $$ \int_{a}^{b}\left|\gamma^{\prime}(t)\right|\,d t\leq\Lambda(\gamma). $$ This completes the proof138 PRINCTPLES OF MATHEMATICAL ANALYSIs EXERCISES 1. Suppose α increases on [a,b],a ≤x≤b,α is continuous at Xo,厂(xo) = 1, and J(x)= 0 if xf xo. Prove that fe SC(c) and that Jf dx = 0. 2、Suppose f≥0,/is continuous on [α,b], and [fXx) dx = 0. Prove that F(x)= 0 for all xe [a,b].(Compare this with Exercise 1. $x\geqslant0$ 3.Define three functions BA,β,,βs as follows: βJ(x)= 0 if x<0,B(x) =1 if for j= 1,2,3; and β,(0) = 0,β,(0) =1, Ba(0) = 麦。Let f be a bounded function on [-1, 1]. (a) Prove that fe R(β,) if and only if f(0+)=/(0))and that then $$ \textstyle\int f\,d\!\!\!/\beta_{1}=f(0). $$ (b) State and prove a similar result for β, (c) Prove that fe SW(βa) if and only iffis continuous at O (d) If fis continuous at Oprove that $$ \int f\,d\beta_{1}=\int f\,d\beta_{2}=\int f\,d\beta_{3}=f(0). $$ 4。If f(x) = 0 for all irrational x,f(x)= 1 for all rational x, prove that f生8R on[a,b1 for any $a<b$ 5.Suppose $\textstyle{\oint}$ is a bounded real function on [a, b], and ${\mathcal{f}}^{2}$ e SR on [a, b]. Does it follow that fe SR?Does the answer change if we assume that fse S2? 6. Let P be the Cantor set constructed in Sec. 2.44. Let f be a bounded real function on 【0,1] which is continuous at every point outside P. Prove that f∈ g on 【0,1] Hint: $\bar{\cal D}$ can be covered by finitely many segments whose total length can be made as small as desired. Proceed as in Theorem 6.10 7.Suppose fis a real function on (0,1] and f∈ Se on [c, 1] for every c>0.Define $$ \textstyle\bigcap_{o}^{1}f(x)\,d x=\operatorname*{lim}_{e\to0}\int_{e}^{1}f(x)\,d x $$ if this limit exists (and is finite) (o) Ife JN on I0.1, show that tisdefinition of the integral agrces with the old one (b)Construct a function f such that the above limit exists, although it fails to exist with|f| in place of 代 8. Supposc fe Cn on [α,b」 for every b>a where a is fixed. Definc $$ \int_{a}^{\infty}f(x)\;d x=\operatorname*{lim}_{b\to\infty}\int_{a}^{b}f(x)\;d x $$ if this it exis (and is fnite)、In that case, we say that the integral on the e converges、If it also converges after $\mathbb{P}$ has been replaced by $|f|,$ , it is said to con- verge absolutely.THE RIEMANN-STIELTJES INTEGRAL 139 Assume that fx)≥0 and that f decreases monotonically on [1, co)、Prove that $$ \textstyle{\int_{1}^{\infty}f(x)\,d x} $$ converges if and only i $$ {\frac{\exists}{2}}f(m) $$ converges.(This is the so-called “integral test”for convergence of series) 9. Show that integration by parts can sometimes be applied to the“improper' integrals defined in Exercises T7 and 8.(State appropriate hypotheses, formulate a theorem, and prove it.)For instance show that $$ \bigcap_{0}^{\infty}{\frac{\cos x}{1+x}}\,d x=\int_{0}^{\infty}{\frac{\sin x}{(1+x)^{2}}}\,d x. $$ Show that one of these integrals converges absolutely, but that the other does not 10.Let p and qbe positive real numbers such that $$ {\frac{1}{p}}+{\frac{1}{q}}=1. $$ Prove the following statements. then (a) f u≥0 and $v\geq0.$ $$ u v\leq{\frac{u^{p}}{p}}+{\frac{v^{q}}{q}}\,. $$ Equality holds if and only if $u^{p}=v^{q}.$ (b)If fe SR(α), g e S2(a),f≥ $0,\,g\geq0,$ and $$ \int_{a}^{b}f^{p}\,d x=1=\int_{a}^{b}g^{q}\,d x, $$ then $$ \textstyle{\int_{a}^{b}f g\,d x\leq1.} $$ (c) Iff and g are complex functions in ${\mathcal{R}}(x),$ then $$ \left|\int_{0}^{b}f g\,d\alpha\right|\leq \langle\int_{a}^{b}\left|f\right|^{p}d\alpha \rangle^{1/p} \langle\int_{a}^{b}\left|g\right|^{q}\,d\alpha \rangle^{1/q}. $$ This is Hblder's inequality.When $p=q=2$ 2 it is usually called the Schwarz inequality.(Note that Theorem 1.35 is a very special case of this.) (d) Show that H6lder's inequality is also true for the“improper”integrals de- scribed in Exercises T and 8.140 PRINCIPLES OF MATHEMATICAL ANALYSIS 11. Let a be a fixed increasing function on [a,b]. Forue SY(a), define $$ \|u\|_{2}=\left\{\int_{a}^{b}|u|^{2}\,d x\right\}^{1/2}. $$ Suppose 八,g,he SXc), and prove the triangle inequality $$ \|f-h\|_{2}\leq\|f-g\|_{2}+\|g-h\|_{2} $$ as a consequence of the Schwarz inequality, as in the proof of Theorem 1.37. 12. With the notations of Exercise 1, suppose fe S0(o) and e>0. Prove that there exists a continuous function g on [a,b] such that lf一 glln<& Hint: Let P={Xo,.,X,} be a suitable partition of [α,b], define $$ g(t)={\frac{x_{t}-t}{\Delta x_{t}}}f(x_{t-1})+{\frac{t-x_{t-1}}{\Delta x_{t}}}f(x_{t}) $$ 13.Define it x-1≤1≤x $$ f(x)=\int_{x}^{x+1}\sin{(t^{2})}\,d t. $$ (a) Prove that lf(x)|<1/x if $x>0.$ Hint: Put $\scriptstyle t^{2}=u$ and integrate by parts, to show that f(x) is equal to $$ {\frac{\cos\left(x^{2}\right)}{2x}}-{\frac{\cos\left[\left(x+1\right)^{2}\right]}{2(x+1)}}-\int_{x^{2}}^{(x+1)^{2}}{\frac{\cos\mu}{4u^{3/2}}}\,d t $$ 本。 Replace cos u by $-1.$ (b) Prove that $$ 2x f(x)=\cos\left(x^{2}\right)-\cos\left[(x+1)^{2}\right]+r(x) . $$ where |r(x)|<c/x and c is a constant. (c)Find the upper and lower limits of xf(x), as $x\to\varnothing.$ (d) Does Tsin (t*) dt converge? 14.Deal similarly with $$ f(x)=\int_{x}^{x+1}\sin{(e^{t})}\,d t. $$ Show that e*|f(x)|<2 and that $$ e^{x}f(x)=\cos\left(e^{x}\right)-e^{-1}\cos\left(e^{x+1}\right)+r(x), $$ where lr(x)|<Ce-*, for some constant C.THE RIEMANN-STIELTJES INTEIGRAL 141 15. Suppose f is a real, continuously differentiable function on [a,b], f(a) =/(b) = 0 and $$ \textstyle\int_{a}^{b}f^{2}(x)\;d x=1. $$ Prove that $$ \int_{a}^{b}x f(x)f^{\prime}(x)\,d x=-{\frac{1}{2}} $$ and that $$ \textstyle\int_{a}^{b}[f^{\prime}(x)]^{2}\,d x\cdot\int_{a}^{b}x^{2}f^{2}(x)\,d x>\frac{1}{4}. $$ 16.For 1<s<Co,define $$ \gamma_{(S)}=\sum_{n=1}^{\infty}{\frac{1}{n^{s}}}. $$ (This is Riemann's zeta function, of great importance in the study of the distri- bution of prime numbers.) Prove that (a)【(s)= ”[x] ; d x and that (b)C(s) = S一1 文-H Jx+1 where [x] denotes the greatest integer ≤x Prove that the integral in (b) converges for al $s>0$ Hint: To prove (a), compute the difference between the integral over [,N and the Nth partial sum of the series that defines (s) 17. Suppose α increases monotonically on [a,b)],g is continuous, and $g(x)=G^{\prime}(x)$ for a≤x≤b.Prove that $$ \textstyle\int_{a}^{b}\!\!\!\alpha(x)g(x)\,\,d x=G(b)x(b)-G(a)\alpha(a)-\int_{a}^{b}\!G\,\,d x. $$ Hint: Take g real, without loss of generality、 Given P={xo,x, .…,xA小, choose tne (x.-1, x) so that $g(t_{i})\,\Delta x_{i}=G(x_{i})-G(x_{i-1})$ .Show that $$ \mathbf{\omega}_{t=1}^{n}\alpha(x_{t})g(t_{t})\,\Delta x_{t}=G(b)\alpha(b)-G(a)\alpha(a)-\mathbf{\Lambda}_{t=1}^{n}G(x_{t-1})\,\Delta\alpha_{t}\,. $$ 18. Let y. 92,ys be curves in the complex plane, defned on 【O,2r7】 by $$ \gamma_{1}(t)=e^{i t},\qquad\gamma_{2}(t)=e^{2i t},\qquad\gamma_{3}(t)=e^{2\pi i t\cdot\mathrm{sin~(1/t)}}. $$ Show that these thrce curves have the same range, that $\gamma_{1}$ and yn are rectifiable that the length of $\gamma_{1}$ is 2m, that the ength of yais 4r, and that ${\mathcal{V}}_{3}$ is not rectifiable.142 PRINCIPLES OF MATHEMATICAL ANALYSIS 19. Let y be a curve in R', defined on [a,b]; let d be a continuous 1-1 mapping of [c, dl onto [a,bJ, such that p(c) = a; and define yxds)= 1(6の)) Prove that y i an arc, a closed curve, or a rectifiable curve if and only if the same is true of yi Prove that yn and $\gamma_{1}$ have the same length.