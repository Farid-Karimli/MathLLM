THE REAL AND COMPLEX NUMBER SYSTEMS INTRODUCTION A satisfactory discussion of the main concepts of analysis (such as convergence continuity,differentiation,and integration) must be based on an accurately defined number concept. We shall not, however, enter into any discussion of the axioms that govern the arithmetic of the integers,but assume familiarity with the rational numbers (i.e., the numbers of the form m/n,where m and n are integers and n ≠ 0) The rational number system is inadequate for many purposes,both as a field and as an ordered set.(These terms will be defined in Secs. 1.6 and 1.12.) For instance,there is no rational $\textstyle{\mathcal{P}}$ such that p- = 2.(We shall prove this presently.)This leads to the introduction of so-called “irrational numbers' which are often written as infinite decimal expansions and are considered to be “approximated”by the corresponding finite decimals. Thus the sequence $$ 1,\ 1.4,\ 1.41,\ 1.414,\ 1.4142,\ \cdot\ . $$ ttends to 、/2.”But unless the irrational number $\sqrt{2}$ has been clearly defined, the question must arise: Just what is it that this sequence “tends to"”?2PRINCIPLBs OF MATHEMATICAL ANALYSIS This sort of question can be answered as soon as the so-called “real number system” is constructed. 1.1 Example We now show that the equation (1) $$ p^{2}=2 $$ is not satisfied by any rational p.If there were such a p, we could write p = m/n where m and n are integers that are not both even. Let us assume this is done Then (1) implies (2) $$ m^{2}=2n^{2}, $$ This shows that m is even. Hence m is even (if m were odd,m would be odd) and so m- is divisible by 4.It follows that the right side of (2) is divisible by 4 so that $\mathcal{H}^{2}$ is even, which implies that n is even. The assumption that (1) holds thus leads to the conclusion that both m and n are even, contrary to our choice of m and n, Hence (I)is impossible for rational p We now examine this situation a little more closely. Let A be the set of all positive rationals p such that $p^{t}<2$ and let B consist of all positive rationals p such that p > 2. We shall show that A contains no largest number and B con- tains no smallest. More explicitly,for every p in A we can find a rational g in A such tha P < q, and for every p in ${\widetilde{\mathcal{D}}}$ we can find a rational g in B such that q< p To do this, we associate with each rational p>0 the number (3) $$ q=p-{\frac{p^{2}-2}{p+2}}={\frac{2p+2}{p+2}} $$ Then (4) $$ q^{2}-2={\frac{2(p^{2}-2)}{(p+2)^{2}}}. $$ If p is in A then p一2<0,(3) shows that g> p, and(4) shows that g<2.Thus qis in A. g3> 2. Thus q is in B. f p is in B then pP -2 > 0,(3) shows that O<9<p, and((4) shows that 1.2 Remark The purpose of the above discussion has been to show that the rational number system has certain gaps, in spite of the fact that between any two rationals there is another: Ifr<s then r<(r + 8)/2<s. The real number system fills these gaps. This is the principal reason for the fundamental role which it plays in analysis.THIE RBAL AND COMPLEX NUMBBR SYSTEMs3 In order to elucidate its structure,as well as that of the complex numbers we start with a brief discussion of the general concepts of ordered set and field Here is some of the standard set-theoretic terminology that will be used throughout this book. 1.3 Defnitions If A is any set (whose elements may be numbers or any other objects), we write xe A to indicate that xis a member (or an element) of ${\mathcal{A}}_{\mathrm{A}}^{\otimes}$ Ifxis not a member of A, we write:x生A The set which contains no element will be called the empty set. If a set has at least one element, it is called nonempty If A and B are sets, and if every element of ${\mathcal{A}}_{\mathrm{A}}^{\otimes}$ is an element of B, we say that A is a subset of B, and write A c B,or B 5 A.If, in addition, there is an element of $\mathbb{P}$ which is not in A, then A is said to be a proper subset of B.Note that AC A for every set A. If Ac B and Bc A,we write A = B.Otherwise A ≠ B 1.4 Definitton Throughout Chap.1,the set of all rational numbers will be denoted by Q. ORDERED SETS 1.3 Definition Let $\operatorname{sgn}$ be a set. An order on $|\sup\sim^{\nu}$ is a relation, denoted by <, with the following two properties: (i)Ifxe S and y e S then one and only one of the statements X<y, x=y, y <x is true. (ii)If x,y, z ∈ S, if x< y and y< z,then x<Z The statement “x<y” may be read as "x sles than y” or“xis smaller than ${\mathcal{Y}}^{\sim}$ or“x precedes y” It is often convenient to write y>xin place of x<y The notationx≤y indicates that x<y or x= y, without specifying which of these two is to hold.In other words, x≤y is the negation of x>y. 1.6 Definitlon An ordered set is a set $\operatorname{sgn}$ in which an order is defined. For example, $\bigotimes_{\mathbb{Z}}$ is an ordered set ifr< s is defined to mean that s-r is a positive rational number. 1.7 Defhnition Suppose S is an ordered set, and E S. If there exists a β e S such that x≤ $\textstyle\bigwedge$ for every x∈ E, we say that ${\mathcal{H}}^{\vee}$ is bounded above,and call $\textstyle\bigwedge$ an wpper bound of E Lower bounds are defined in the same way (with ≥ in place of ≤)4PRINCIPLBS OF MATHEMATICAL ANALYSIS 1,8 Definiton Suppose S is an ordered set,E c S, and Eis bounded above Suppose there exists an α∈ S with the following properties: (i)c is an upper bound of E (i)If y<α then y is not an upper bound of $\textstyle{E},$ Then α is called the least upper bound of $\widehat{\mathcal{Q}}$ [that there is at most one such α is clear from (ii)] or the supremum of ${\bar{E}},$ and we write $$ x=\operatorname*{sup}E. $$ The greatest lower bound, or infimum,of a set ${\widehat{\operatorname{F}_{d}}}$ which is bounded below is defined in the same manner: The statement $$ \alpha=\mathrm{inf}\;E $$ means that α is a lower bound of ${\widehat{\mathcal{R}}}_{d}^{\nu}$ and that no $\textstyle\iint$ with β >αis a lower bound of E. 1.9 Examples (a)Consider the sets A and B of Example 1. as subsets of the ordered set Q. The set A is bounded above. In fact, the upper bounds of ${\mathcal{A}}_{\Delta}$ are exactly the members of B.Since B contains no smallest member, A has no least upper bound in $\underline{{{Q}}}$ ${\widehat{\operatorname{G}}}$ Similarly, B is bounded below:The set of all lower bounds of consists of A and of all re Q with r ≤0.Since A has no lasgest member, B has no greatest lower bound in Q. (b)If α = sup E exists, then α may or may not be a member of E.For instance, let $E_{1}$ be the set of all re Q with r<0. Let $E_{2}$ be the set of al re Q with r≤0. Then $$ \operatorname*{sup}E_{1}=\operatorname*{sup}E_{2}=0, $$ and O生E,,0 ∈ E,, (c)Let E consist of all numbers I/n, where n= 1,2,3,. Then sup E = 1, which is in E, and inf E = 0, which is not in E. 1.10 Definition An ordered set S is said to have the least-upper-bound property if the following is true: If Ec S,E is not empty, and E is bounded above, then sup ${\mathcal{H}}^{\vee}$ exists in S Example 1.9(a) shows that Q doces not have te least-upper-bound property We shall now show that there is a close relation between greatest lower bounds and least upper bounds, and that every ordered set with the least-upper bound property also has the greatest-lower-bound propertyTHE REAL AND COMPLEX NUMBER SYSTEMSS 1.11 Theorem Suppose S is an ordered set with the least-upper-bound property B c S,B is not empty, and ${\mathcal{D}}$ is bounded below. Let L be the set of all lower bounds of B.Then $$ \alpha=\operatorname*{sup}L $$ exists in ${\mathfrak{S}}_{s}$ and α = inf B. In particular, inf B exists in S. ProofSince B is bounded below, $\textstyle\int_{}^{}{\frac{}{}}$ is not empty. Since ${\widehat{f}}_{i}$ consists of exactly those ye S which satisfy the inequality y ≤x for every xe B,we see that every xe B is an upper bound of L. Thus L is bounded above Our hypothesis about $|\operatorname{a}{\mathfrak{p}}^{\nu}$ implies therefore that $\textstyle{\int}$ has a supremum in S; call t If)<α then (see Definition 1.8),is not an upper bound of $\textstyle{L_{\mathrm{z}}}$ hence ys B、It fllows that a≤x for every xe B. Thusαe L If α<β then $\textstyle\bigwedge$ 生L, since α is an upper bound of L. We have shown that α e L but β生L ifβ>α. In other words, is a lower bound of B,but $\textstyle\bigwedge$ is not if β > α. This means that α = inf B. FIELDS 1.12 Definition A field is a set ${\mathcal{H}}^{\nu}$ with two operations,called addition and multiplication, which satisfy the following so-called“field axioms'’(A),(M) and (D): (A)Axioms for addition (A1)If xe F and y∈ F, then their sum x + y is in F (A2)Addition is commutative:x +y = ) + x for all x, ye F (A3)Addition is associative: (x +y)十 Z =X + (y + z) for all x, y,z ∈ ${\textstyle\bigwedge}^{\nu}$ (A4)F contains an element O such that O+X= x for every xe F (A5)To every xe F corresponds an element一xe F such that $$ x+(-x)=0. $$ (M)Axioms for multiplication (M1)If xe F and y e F, then their product xy is in F (M2) Multiplication is commutative: xy = yx for all x,y∈ F. (M3) Multiplication is associative: (xy)z = x(yz) for all x,y, z∈ F. (M4) ${\mathcal{H}}^{\vee}$ contains an element 1≠O such that lx = x for every xe F. (M5) If xe ${\mathcal{H}}^{\vee}$ and x≠ 0 then there exists an element 1/x e F such that $$ x\cdot(1/x)=1. $$6 PRINCIPLEs Or MATHEMATCAL ANALYsIs (D)The dlstributive aw x(y + 2)= xy + xZ holds for all x, y, ze F 1.13 Remarks (o)One usually writes Gin any feld $$ x-y,{\frac{x}{y}},x+y+z,x y z,x^{2},x^{3},2x,3x,\dots. $$ in place of x+(一y +y)+ z,(xy)z, XX,XXX,X十x,X十x十x (6)The field axioms clearly hold in Q,the set of all rational numbers, i addition and multiplication have their customary meaning.Thus Q is a field. (c)Athough it is not our purpose to study fields (or any other algecbraic structures) in detail, it is worthwhile to prove that some familiar properties of Q are consequences of the field axioms; once we do this, we will not need to do it again for the real numbers and for the complex numbers 1.14 Proposition The axioms for addition imply the following statements (a)Ifx + y =X+ z then y = z. (b)If x +) =x then y = 0. (c)Ifx+ y =0 hen y = 一x (d)一(一x) = x. Statement (a) is a cancellation law.Note that (b)) asserts the uniqueness of the element whose existence is assumed in(A4), and that (c) does the same for (A5). Proof Ifx+y =X + z, the axioms (A) give y = 0 +y =(一X + x)+ y = -X + (x +少) = -x + (x + 2) (一X+X) + z = 0 + z = z This proves (a).Take z = 0 in(a) to obtain (b). Take z = -x in (a) to obtain (c). Since 一x+× = 0,(c)(with -xin place of x) gives (d)THE REAL AND COMPLEX NUMBER SYSTEMS 7 1.15 Proposition The axioms for multiplication imply the following statements (a)Ifx ≠ O and xy = xz then = 2 (b)Ifx ≠ 0 and xy =x then y = 1. (c)I/x≠ 0 and xy =1 then y = 1/x (d)1fx ≠ 0 then 1/(1/x) = x. The proof is so similar to that of Proposition 1.14 that we omit it 1.16 Propositio Te feld axioms imply the following statements, for any x,, z ∈ F. (a)Ox = 0. (6)If x ≠ 0 and y + 0 then xy ≠ 0 (c)(一x)y = -(xy) = X(一y) (d)(一x)(一)) = xy. Proof Ox + 0x = (0 + 0)x = 0x. Hence 1.14(b) implies that Ox =0,and (a) holds Next assume $x\neq0,\;y\neq0,$ but xy = 0. Then (a) gives $$ 1={\bigg(}{\frac{1}{y}}{\bigg)}\left({\frac{1}{x}}\right)x y={\bigg(}{\frac{1}{y}}{\bigg)}\left({\frac{1}{x}}\right)0=0, $$ a contradiction. Thus (b) holds The first equality in (c) comes from $$ (-x)y+x y=(-x+x)y=0y=0, $$ Finally combined with 1.14c); the other half of (c) is proved in the same way $$ (-x)(-y)=-[x(-y)]=-[-(x y)]=x y $$ by (C) and 1.14(d) 1.17 Definition An ordered field is a field F which is also an ordered set, such that (i)x +y <x+ z if x, P,ze F and y < z, ti)xy > Oif xe F,ye F,x> 0, and y > 0. If x>0, we call x positive; if x<0,xis negative. For example, Q is an ordered field All the familiar rules for working with inequalities apply in every ordered field:Multiplication by positive [negative] quantities preserves [reverses] in these equalities, no square is negative, etc. The following proposition lists some of8 PRINCIPLEs Or MATHEMATICAL ANALYss 1.18 Proposition The following statements are trwe in every ordered field (a)Ix>0 hen -x<0, and vice versa. (b)Ifx> O and y < z then xy <XZ. (c)Ifx<O and y < z then xy > Xz (d)If x≠ O then x > 0.In particular,1> 0 (e)If O <x<y then O<1/y < 1/x Proof (の)Ifx> 0 then 0 = 一X+x> -×+ 0,so that -x<0. Ifx<O then 0 = -x +x< -x + 0, so that -x> 0. This proves (a). (b)Since z>)、we have z-y>y-y = 0, hence xz - y)> 0, and therefore $$ x z=x(z-y)+x y>0+x y=x y. $$ (c)By (a),(b), and Proposition 1.16(c) $$ -\left[x(z-y)\right]=(-x)(z-y)>0, $$ so that x(z 一)<0, hence xz <xy (d)If x>0, part (i) of Definition 1.17 gives x>0. If x<0,then -x > 0, hence(-x)) >0. But x =(-x),by Proposition 1.16(4) Since 1= 1-,1 > 0. (e)Ify > O and v≤0, then yv ≤0. But y·(1/)) =1 >0. Hence 1l/y > 0. Likewise,1/x > 0. If we multiply both sides of te inequalityx<y by the positive quantity (1/.x)(1/y)、 we obtain 1/y <1/x. THE REAL FIELD We now state the existence theorem which is the core of this chapter. 1.19 Theorem There exists an ordered field R which has the least-wpper-bound property. Moreover,R contains Q as a subfield The second statement means that Qc R and that the operations of addition and multiplication in R, when applied to members of Q, coincide with the usual operations on rational numbers; also, the positive rational numbers are positive elements of R. The members of R are called real numbers. The proof of Theorem 1.19 is rather long and a bit tedious and is therefore presented in an Appendix to Chap. 1. The proof actually constructs R from QTHE REAL AND COMPLEX NUMBER sYSTEMS 9 The next theorem could be extracted from this construction with very litte extra effort. However, we prefer to derive it from Theorem 1.19 since thi provides a good illustration of what one can do with the least-upper-bound property. 1.20Theorem (a)f xe R,ye R, and x>0, then there is a positiveinteger n such tha $$ n x>y. $$ (6)Ixe R,ye R, andx<J, then there exists ape Q such that x<P<y Part (a) is usually referred to as the archimedean property of R、 Part (D) may be stated by saying that Q is dense in R: Between any two real numbers there is a rational one. Proof (a)Let A be the set of ll nx, where n runs through the positive integcrs lf (a) were false, then y would be an upper bound of A. But then A has a least upper bound in R. Put α = sup A. Since x> 0,α一x<α,and α一xis not an upper bound of A. Hence α一X<mx for some positive integer m.But then α<(m + 1)x ∈ A, which is impossible, since α is an upper bound of ${\mathcal{A}}_{\Delta}$ (b)Since x<y, we have $y-x>0$ , and(a) furnishes a positive integer n such that $$ n(y-x)>1. $$ Apply Ga) again, to obtain positive integers m, and m,such that m, > nx m,> -nx. Then $$ -m_{2}<n x<m_{1} $$ Hence there is an integer m(with -m,≤m ≤ m,) such that m -1≤nx < m. If we combine these inequalities, we obtain $$ n x<m\leq1+n x<n y. $$ Since $\scriptstyle n\;>\;0.$ it follows that $$ x<{\frac{m}{n}}<y. $$ This proves (b), with $p=m/n.$10 PRINCIPLEs OF MATHEMATICAL ANALYsIs We shall now prove the existence of nth roots of positive reals. This proof wil show how the dificuty pointed out in the Introduction Girration- ality of 、/2) can be handled in R. 1.21 Theorem For every real x>0 and every integer n> 0 there is one and only one positive real y such that y" = x. This number y switen ;/x or $x^{1/n}$ Proof That there is at most one such yis clar snce O<Y<y, implies y'<y. Let $\widehat{H_{d}}$ be the set consisting of all positive real numbers t such that t”<x If t = x/(1 +x) then 0≤t<1. Hence t"≤t<x.Thus te E, and E is not empty If >1+x then t"≥1>x, so that ts E. Thus 1+ xis an upper bound of E Hence Theorem 1.19 implies the existence o $$ y=\operatorname*{sup}E. $$ and y">x leads to a contradiction. To prove that y"=x we wil show that each of the incqualities y"<X The identityb" $$ \mathbf{\tau}^{*}-a^{n}=(b-a)(b^{n-1}+b^{n-2}a+\mathbf{\epsilon}\cdot\mathbf{\epsilon}+a^{n-1} $$ )yields the inequality $$ b^{n}-a^{n}<(b-a)n b^{n-1} $$ when 0<α<b Assume y"<x. Choose $\textstyle{\hat{J}}$ so that O<h<1 and $$ h<{\frac{x-y^{n}}{n(y+1)^{n-1}}}\,. $$ Put a = y, b = ) + h.Then $$ (y+h)^{n}-y^{n}<h n(y+h)^{n-1}<h n(y+1)^{n-1}<x-y^{n}. $$ Thus (y + h)"<x, and y + he E. Since $y+h>y,$ this contradicts the fact that y is an upper bound of E. Assume y">x. Put $$ k={\frac{y^{n}-x}{n y^{n-1}}}. $$ Then 0<k<y. Ift≥y一k,we conclude that y $$ \l_{}^{n}-t^{n}\leq y^{n}-(y-k)^{n}<k n y^{n-1}=y^{n}-x. $$ Thus t">x,and t生 E. It follows that $y-k$ is an upper bound of E.THE REAL AND COMPLEX NUMBER sYSrEMS 11 But y-k<y, which contradicts the fact that y is the least upper bound of E. Hence $y^{n}=x,$ and the proof is complete Corollary If a and b are positive real numbers and n is a positive integer, then $$ (a b)^{1/n}=a^{1/n}b^{1/n}, $$ Proof Put $x=a^{1/n},\,\beta=b^{1/n}$ . Then $$ a b=\alpha^{n}\beta^{n}=(\alpha\beta)^{n}, $$ since multiplication is commutative. [Axiom(M2)in Definition 1.12. The uniqueness assertion of Theorem 1.21 shows therefore that $$ (a b)^{1/n}=\alpha\beta=a^{1/n}b^{1/n}. $$ 1.22 Decimals We conclude this section by pointing out the relation between real numbers and decimals. Let $\scriptstyle x\;>\;0$ be real. Let ${\mathcal{N}}_{0}$ be the largest integer such that n。≤x.(Note tha the existence of ${\mathcal{H}}_{0}$ , depends on the archimedean property of R.)Having chosen $n_{0}\,,\,\,n_{1},\,\,\ldots\,,\,\,n_{k-1},\,\,\mathrm{fet}\,\,\gamma_{i}\,$ n, be the largest integer such that $$ n_{0}+{\frac{n_{1}}{10}}+\cdot\cdot\cdot+{\frac{n_{k}}{10^{k}}}\leq x. $$ Let I ${\mathcal{F}}^{\dagger}$ be the set of these numbers (5) $$ n_{0}+{\frac{n_{1}}{10}}+\cdot\cdot\cdot\cdot+{\frac{n_{k}}{10^{k}}}\qquad(k=0,1,2,\cdot\cdot). $$ Then x = sup E. The decimal expansion of xis (6) $$ n_{0}\cdot n_{1}n_{2}n_{3}\cdot\cdot\cdot\cdot. $$ Conversely, for any infinite decimal (6) the set E of numbers (5)is bounded above, and (6)is the decimal expansion of sup E Since we shall never use decimals, we do not enter into a detailed discussion. THE EXTENDED REAL NUMBER SYSTEM 1.23 Definition The extended real number system consists of the real field R and two symbols,+OO and -OO. We preserve the original order in R,and define 一 OO <X<十 OO for every xe R12 PaINCIPLEs Or MATHEMATICAL ANALYsrs It is then clear that +oo is an upper bound of every subset of the extended real number system, and that every nonempty subset has a least upper bound. If, for example, E is a nonempty set of real numbers which is not bounded above in R,then sup E = +oo in the extended real number system. Exactly the same remarks apply to lower bounds The extended real number system does not form a field, but it is customary to make the following conventions: (a)If x is real then $$ x+\,\infty=\,+\infty,\qquad x-\,\infty=\,-\infty,\qquad{\underset{+\,\infty}{x}}={\frac{x}{-\,\infty}}=0. $$ (b)Ifx>O then x·(十0)= +00,x·(-α)= -oo (c)Ifx<O then x·(+ 00)= -O0,x·(-00)= +00. When it is desired to make the distinction between real numbers on the one hand and the symbols +0O and -OO on the other quite explicit, the former are called finite. THE COMPLEX FIELD 1.24 Definition A complex number is an ordered pair (a,b) of real numbers “Ordered”’means that (a,b) and(b,a) are regarded as distinct if a ≠ b. Let x = (α,b), = (c, d) be two complex numbers. We write x = y if and only if a = c and b= 4.(Note that this definition is not entirely superfluous: think of equality of rational numbers, represented as quotients of integers.)W define x + y = (a + c, b + d), Xy = (ac - bd, ad + bc) 1.25 Theorem These definitons of addition and mutiplication turn the set 0 all complex numbers into a field,with (0,0) and (1,0) in the role of O and 1 Proof We simply verify the field axioms, as listed in Definition 1.12 (Of course, we use the field structure of R.) Let x = (a,b),y = (c, d), z = (e,f) (A1)is clear. (A2)x+y = (a + c,b + d)= (c + α, d + b)=) + xTHE REAL AND COMPLEX NUMBER sYSTEMS 13 (A3)(x + y) + =(α + c,b + d)+(e,f) =(a +c + e,b+ d +/) = (a,b)+ (c + e,d +/)=X+ (y + z). (A4)X十0 = (a,b)+ (0,0) = (a,b) = x (A5) Put -X =(-a,-b). Then X +(一x)= (0,0) = 0 (M1)is clear. (M2) xy = (αc - bd, ad + bc) =(ca - db, da + cb) = yx. (M3)Oxp)2 = (c - bd, ad + bc)(e,f) = (ace - bde - adf - bcf, acf - bdf + ade + bce) = (a,b)(ce - df, cf + de)= x(yz) (M4) 1x =(1,0Xa,b) = (a,b) = X (M5)Ifx≠ 0 then (a,b)≠ (0,0), which means that at least one of the real numbers a、 $\dot{\mathcal{J}}$ is different from O. Hencc $a^{2}+b^{2}>0$ ,by Proposition 1.18(d), and we can define $$ \frac{1}{x}=\left(\frac{a}{a^{2}+b^{2}},\ \ \frac{-b}{a^{2}+b^{2}}\right). $$ Then (D)xy $$ \begin{array}{c c}{{x\stackrel{1}{x}=(a,b)\left(\frac{a}{a^{2}+b^{2}},\quad\frac{-b}{a^{2}+b^{2}}\right)=(1,0)=1.}}\\ {{+\,z)=(a,b)(c+e,d+f)}}&{{}}\\ {{=(a c+b d-b f,a d+b c)+(a e-b f,a f+b e)}}&{{}}\\ {{=x y+x z.}}&{{}}&{{}}\end{array} $$ 1.26 Theorem For any real numbers a and $\stackrel{\cal J}{\cal J}$ we have $$ (a,\,0)+(b,\,0)=(a+\,b,\,0),\qquad(a,\,0)(b,\,0)=(a b,\,0). $$ The proof is trivial Theorem 1.26 shows that the complex numbers of the form (a,O) have the same arithmetic properties as the corresponding real numbers a. We can there- fore identify (a,0) with a. This identification gives us the real field as a subfield of the complex field. The reader may have noticed that we have defined the complex numbers without any reference to the mysterious square root of -1. We now show that the notation (a,b)is equivalent to the more customary a + bi. 1.27 Definition i=(0,1)14 PRINCIPLES OF MATHEMATICAL ANALYSIs 1.28 Theorem i1 = -1. Proofi = (0,1)(0,1) =(-1,0) = -1 1.29Theorem If a and b are real, then (a,b) = 4 + bi. Proo $$ \begin{array}{c}{{a+b i=(a,0)+(b,0)(0,1)}}\\ {{}}\\ {{=(a,0)+(0,b)=(a,b).}}\end{array} $$ 1.30 Definition If a, $\boldsymbol{\beta}$ are real and z = a + bi, then the complex numbe 2 = a- bi is called the conjugate of z. The numbers a and b are the real part and the imaginary part of z, respectively. We shall occasionally write $$ a=\mathrm{Re}(z),\qquad b=\mathrm{Im}(z). $$ 1.31 Theorem If z and w are complex, then (a)2 + W =2 + w, (b)Zw = 2·w (c)z + 2 = 2 Re(z), z-2= 2i 1m(z) (d)z is real and positive (except when = 0) Proof(a),(b), and (c) are quite trivial. To prove (d), write z = a + bi and note that zz = a3 + b3. 1.32 Definition If z is a complex number, its absolute value |z」 is the non negative square root of z; tat is, z|= z) The existence (and uniqueness)of |z| follows from Theorem 1.21 and part (d) of Theorem 1.31. if x≥0,|x| = -xif Note that when xis real, then = x, hence |x| =,/x.Thus $\operatorname{tri}-x$ $x<0$ 1.33 Theorem Let z and w be complex numbers. Then (a) |z|> O unless z = 0,|0| = 0 (b) |2 =|z}, (c) |zw|= |z||w|, (d) | Re zl≤|z|, (e) }z + wl≤」z」+」w」.THB REAL AND COMPLEX NUMBER SYsTEMS 15 Proof(a) and (b) are trivial. Put $z=a+b i,$ w =C + di, with a,b,c,d real. Then $$ |z w|^{2}=(a c-b d)^{2}+(a d+b c)^{2}=(a^{2}+b^{2})(c^{2}+d^{2})=|z|^{2}|w|^{2} $$ or |zw| = (|z||w). Now (c) follows from the uniqueness assertion of Theorem 1.21. To prove (d), note that al ≤a+ b,hence $$ \vert a\vert=\sqrt{a^{2}}\le\sqrt{\overline{{{a^{2}+b^{2}}}}}. $$ To prove (e), note that zw is the conjugate of zw,so that zw + zw 2 Re (zw)、 Hence $$ \begin{array}{l l}{{|z+w|^{2}=(z+w)(\bar{z}+\bar{w})=z\bar{z}+z\bar{w}+\bar{z}w+\bar{w}}}\\ {{=\ |z|^{2}+2\,{\mathrm R e}\,(z\bar{w})+\left|w\right|^{2}}}\\ {{\ }}&{{\le\left|z\right|^{2}+2\left|z\bar{w}\right|\,+\,\left|w\right|^{2}}}\\ {{=\,\left|z\right|^{2}+2\left|z\right|\left|w\right|^{2}}}&{{=(\left|z\right|^{2}+\left|w\right|^{2}}.}}\end{array} $$ Now (e) follows by taking square roots 1.34 Notation If x, ……,xnare complex numbers, we write $$ x_{1}+x_{2}+\cdot\cdot\cdot+x_{n}=\sum_{j=1}^{n}x_{j}\,. $$ We conclude this section with an important inequality, usually known as the Schwarz inequality 1.35 Theorem If a,,.…, ,and b $b_{1},\ \ldots,\ \ ,$ b。are complex numbers, then $$ \left|\sum_{j=1}^{n}a_{j}b_{j}\right|^{2}\leq\sum_{j=1}^{n}\left|a_{j}\right|^{2}\sum_{j=1}^{n}\left|b_{j}\right|^{2}. $$ Proof Put $A=\Sigma\,|\,a_{j}\,|^{2},\,B=\Sigma\,|\,b_{j}\,|^{2},\,C=\Sigma a_{j}\,b_{j}$ 5, (in all sums in this proof jruns over the values 1,.….,n) If B = 0, then b,= = b,= 0,and the conclusion is trivial. A ssume therefore that B>0、By Theorem 1.31 we have $$ \begin{array}{c}{{\sum|B a_{j}-C b_{j}|^{2}=\sum(B a_{j}-C b_{j})(B\bar{a}_{j}-\overline{{{C b}}}_{j})}}\\ {{=B^{2}\sum|a_{j}|^{2}-B\overline{{{C}}}\sum a_{j}b_{j}-B C\sum\bar{a}_{j}b_{j}+|C|^{2}\sum|b_{j}|^{2}}}\\ {{=B^{2}A-B|C|^{2}}}\\ {{=B(A B-|C|^{2}).}}\end{array} $$16 PRINCIPLES OF MATHEMATICAL ANALYsIS Since each term in the first sum is nonnegative, we see tha $$ B(A B-|C|^{2})\geq0. $$ Since $\scriptstyle B>0,$ it follows that AB -|C ≥0. This is the desired inequality EUCLIDEAN SPACES 1.36 Definitions For each positive integer k, let $\textstyle{\mathcal{P}}^{k}$ be the set of all ordered k-tuples $$ \mathbf{x}=(x_{1},x_{2},\dots,x_{k}), $$ where x, .,Xx are real numbers, called the coordinates of x. The elements of $\textstyle{\mathcal{P}}^{k}$ are called points,or vectors, especially when $k>1$ .We shall denote vectors by boldfaced letters.If ${\mathbf y}=\left(y_{1},\ \ast\ ,\ y_{k}\right)$ and if α is a real number, put $$ \begin{array}{c}{{\mathbf{x}+\mathbf{y}=(x_{1}+y_{1},\mathbf{\ddots}\cdot\mathbf{\ldots}\cdot x_{k}+y_{k}),}}\\ {{\alpha\mathbf{x}=(\alpha x_{1},\mathbf{\ldots}\cdot\mathbf{\alpha}\cdot\cdot\cdot\cdot x x_{k})}}\end{array} $$ so that x + y ∈ $\textstyle{\mathcal{P}}^{k}$ R'and αX ∈ $\textstyle{\mathcal{N}}^{k}$ .This defines addition of vectors, as well as multiplication of a vector by a real number (a scalar). These two operations satisfy the commutative,associative, and distributive laws (the proof is trivial in view of the analogous laws for the real numbers) and make $\textstyle{\mathcal{P}}^{k}$ into a vector space over the real field. The zero element of Rk(sometimes called the origin or the null vector)is the point O, all of whose coordinates are O We also define the so-called“inner product'’or scalar product) of x and y by $$ \mathbf{x}\cdot\mathbf{y}=\sum_{i=1}^{k}x_{i}y_{i} $$ and the norm of x by $$ |\mathbf{x}|=(\mathbf{x}\cdot\mathbf{x})^{1/2}={\binom{k}{1}}\,x_{i}^{2}{\Big)}^{1/2}. $$ The structure now defined (the vector space $\textstyle{\mathcal{P}}^{k}$ with the above inner product and norm) is called euclidean k-space 1.37 Theorem Suppose x, y, ze R*, and cα is real.、 Then (a) }x」≥ 0; if and only if x = 0; (b) $\scriptstyle|x|=0$ (c) lαx| = |αl|x|; (d) |x·y≤|x||y; (e) ix +y|≤|x +Iyl; (f) |x - z|≤|x- y」 + |y - z|rEE REAL AND COMPLEX NUMBER SYSTEMS 17 Proof(a)(b), and (c) are obvious, and (d)is an immcdiate consequence of the Schwarz inequality.By (d) we have $$ \begin{array}{r l}{|x+y|^{2}=(x+y)\cdot(x+y)}\\ {=\mathbf{x}\cdot\mathbf{x}+2x\cdot y+y\cdot y}\\ {=\mathbf{x}\cdot\mathbf{x}+2\mathbf{x}\cdot\mathbf{y}+y\cdot\mathbf{y}}\\ {\mathbf{z}\cdot|x|^{2}+2|\mathbf{x}|y|+|y|^{2},}\\ {=\langle|\mathbf{x}|+|y|)^{2},}\end{array} $$ so that (e) is proved. Finally, (O) follows from (e) if we replace x by X - y and y by y - Z. 1.38 Remarks Theorem 1.37(a), (b), and(f) will allow us (see Chap. 2)to regard $\textstyle{\mathcal{R}}^{k}$ * as a metric space. $\textstyle{\mathcal{Q}}^{1}$ (the set of all real numbers) is usually called the line, or the real line Likewise, R is called the plane,or the complex plane (compare Definitions 1.24 and 1.36). In these two cases the norm is just the absolute value of the corre sponding real or complex number. APPENDIX Theorem 1.19 will be proved in this appendix by constructing R from Q. We shall divide the construction into several steps. Step 1 The members of R will be certain subsets of Q, called cuts.A cut is, by definition,any set α c Q with the following three properties (I)α is not empty, and α≠Q (IDIf peα,ge Q, and q< p, then geα. (III)If peα, then p<r for some re α will denote cuts. The letters p q, ,.….will always denote rational numbers, and α,β,y,. facts which will be used freely: Note that (III) simply says that a has no largest member;(ID implies two If peα and qsaα then p<q Ifrα and r<s then s生α. Step Define “α<β” to mean: α is a proper subset of β Let us check that this meets the requirements of Definition 1.5. If α <β and β<y it is clear that α<y.(A proper subset ofa proper sub set is a proper subset.) It is also clear that at most one of the three relations $$ \alpha<\beta,\ \ \ \ \ \alpha=\beta,\ \ \,\ \ \ \beta<\alpha $$18 PRINCIPLEs OF MATHEMATICAL ANALysis Since β≠ α, we conclude: β<α can hold for any pair α,β、 To show that at least one holds, assume that the first two fail. Then α is not a subset of β、Hence there is a peα with psβ If q e β,it fllows that q<p (Since p≠ D), hence qe α, by Gin). Thus βea Thus R is now an ordered set Step 3 The ordered set R has the least-upper-bound property To prove this,let A be a nonempty subset of R, and assume that βe R is an upper bound of A. Define y to be the union of all c A. In other words $\ D$ ∈ y if and only if p e cα for some α∈ A. We shall prove that y∈ R and that ) = sup A. Since A is not empty, there exists an ${\mathcal{Q}}_{0}$ e A. This o ${\mathcal{Q}}_{0}$ co is not empty. Since αoC ),y is not empty. Next,)C β(since αc β for every α∈ A), and therefore y ≠ Q. Thus y satisfies property (I). To prove(II) and (III),pick p ∈ y. Then p e α, for some α,∈ A.If q< p, then g∈ α, hence qe y; this proves (II)、 1f re α, is so chosen that r> p, we see that re,(since αC 》),and therefore satisfies (II1). Thus ye R. It is clear thatα≤ , for every α e A. Suppose 员< ":Then there is an sey and that st . Since se y,seα for some αe A. Hence 8<α,and Sis not an upper bound of A. This gives the desired result: y = sup A. Step 4 Ifoe R and $\underline{{{\oint}}}$ ∈ R we define α + β to be the set of all sums r +s, where r ∈ α and se β. We define 0* to be the set of all negative rational numbers. It is clear that 0* is a cut. We verify that the axioms for addition (see Definition 1.12) hold in ${\mathcal{R}},$ with 0* playing the role of 0O. (A1)We have to show that α + β is a cut. It is clear that α + β is a nonempty subset of Q.Take r’生α,s′生 β. Then r’十S′>r +s for al choices of re α,s ∈ β. Thus r′+ s′生α + β.It follows that α + β has property (I). Pick p eα+ β、 Then p =7+s,with reα,se β. If q<p, then 4 -8<r,sO9- s∈ α,and q= (q - s)+s∈α+ β. Thus(II) holds Choose t∈ α so that t> r. Then p <1+s andt+ s∈ α +β、 Thus (III holds. (A2)α ${}+\beta$ is the set of all r +s,with r∈ α,s ∈ β. By the same definition, β + αis the set of all s + r.Since r + 8 = 8 + r for all r∈ Q,s∈ Q, we have α + β = β十α (A3)As above, this follows from the associative law in Q. (A4)Ifre o and se 0*,then r + 8<r, hence r + s∈ α. Thus $\alpha+0^{\bullet}<\alpha$ To obtain the opposite inclusion, pick p e α, and pick reα,r> p.ThenTHB REAL AND COMPLEX NUMBER SYSTEMs 19 P -re 0*, and p =7 +(p -r)eα + 0*. Thus αCα + 0*.We conclude that α + 0* = α. (A5)Fix αe R. Let β be the set of all p with the following property There exists r>0 such that -p -r≠α. In other words,some rational number smaller than -p fails to be in α. We show that β ∈ R and that α + β = 0*. If sα and p = -8-1, then -p -1≠α, hence p ∈ β. So β is not empty. If g ∈ α, then -9生β.So β≠ Q. Hence β satisfies (I) Pick p ∈ β,and pick r> 0,so that -p -r生α.If g<p,then -Q 一r> 一p -r, hence 一q-r生α. Thus q∈ β,and(II) holds. Put 1 = p +(r/2). Then t>p,and -t- (r/2) = -p -r≠α,so that te β Hence $\underline{{{\mathcal{O}}}}$ satisfies (III). We have proved that β e R If reα and s e β,then -s生 α, hence r< -s,,+8< 0. Thus α + β c 0*. To prove the opposite inclusion, pick v∈ 0*, put w = -D/2.Then w > 0, and there is an integer n such that nw eα but (n + 1)w fα.(Note that this depends on the fact that $\begin{array}{c}{{\langle\Im\rangle}}\\ {{\pm\rangle}}\end{array}$ has the archimedean property!)Put p = -(n + 2)w. Then p e β,since $-p-w\not\in\omega$ 、and Thus 0* e α + β. $$ v=n w+p\in\alpha+\beta. $$ We conclude that $x+\beta=0^{n}.$ This β will of course be denoted by -α. Step SHaving proved that the addition defined in Step satisfies Axioms (A) of Definition 1.12,it follows that Proposition 1.14 is valid in R, and we can prove one of the requirements of Definition 1.17: If α,β,y ∈ ${\mathcal{Q}}_{\mathrm{d}}$ and β<y, then α + β <α +y β = ) Indeed,itis obvious from the definition of + in R that α + βC α + y; it we had α+ β =α + y, the cancellation law (Proposition 1.14) would imply It also follows that α > 0* if and only if -α <0*. Step 6 Multiplication is a litle more bothersome than addition in the present context, since products of negative rationals are positive. For this reason we confine ourselves first to Rt,the set of all c e R withα > 0*. If ae Rt and βe Rt, we define aβ to be the set of ll p such that $p\leq r s$ for some choice of re α,s∈ β,r> 0,s > 0 We define 1* to be the set of all $\scriptstyle q\,<\,1$20 raINCIPLEs Oor MATHBMATICAL ANALYSis Then the axioms (M)and(D) of Definition 1.12 hold, with A ${\boldsymbol{R}}^{\dagger}$ in place of F and with 1* in the role of 1 The proofs are so similar to the ones given in detail in Step 4 that we omit them. Note, in particular, that the second requirement of Definition 1.17 holds: If α > 0* and β > 0* then αβ > 0*. Step T We complete the definition of multiplication by setting α0* = 0*α = 0* and by setting $$ \begin{array}{c c}{{\mathrm{if~}\alpha<0^{\bullet},\,\beta<0^{\bullet},}}\\ {{\alpha\beta= \{-[(-\alpha)\beta]}}&{{\mathrm{if~}\alpha<0^{\bullet},\,\beta>0^{\bullet},}}\\ {{-[\alpha\cdot(-\beta)]}}&{{\mathrm{if~}\alpha>0^{\bullet},\,\beta<0^{\bullet}.}}\end{array} $$ The products on the right were defined in Step 6. Having proved in Step 6)that the axioms(M) hold in $R^{\star},$ ,it is now perfectly simple to prove them in R,by repeated application of the identity )二一(一)) which is part of Proposition 1.14.(See Step 5.) The proof of the distributive law $$ \alpha(\beta+\gamma)=\alpha\beta+\alpha\gamma $$ breaks into cases. For instance,suppose α> 0*,β< 0*,β+ ) > 0*. Then = (β + ))+(一β), and (since we already know that the distributive law holds in Rt) $$ \alpha\gamma=\alpha(\beta+\gamma)+\alpha\cdot(-\beta). $$ But α·(-β)= -(αβ).Thus $$ \alpha\beta+\alpha\gamma=\alpha(\beta+\gamma). $$ The other cases are handled in the same way. We have now completed the proof that ${\mathcal{F}}_{\mathrm{e}}^{y}$ is an ordered field with the least- upper-bound property. Step &We associate with each r∈ Q the set $f^{\prime}{}^{\mathrm{sg}}$ which consists of all p ∈ Q such that p<r. It is clear that each r* is a cut; that is, r* ∈ R. These cuts satisfy the following relations: (a)r* + * = (r + 8)*, (の)* = (rs)*, (c)r* <s* if and only ir<s To prove (a),choose p ∈ r* + s*. Then p = U + D,where u<r,1<s. Hence p <r+s, which says that p e (r + 8)*.THE REAL AND coMPLEx NUMBER sYsTEMS 21 Conversely,suppose p ∈ (r + s)*. Then p <r +s. Choose t so that 21 =7 + 8-p, put $$ \displaystyle r^{\prime}=r-t,s^{\prime}=s-t. $$ Then $\textstyle{\int}^{\rho}{}$ er*,s′∈ s*, and p = r′+ s',so that p e r* + s* This proves (a). The proof of (b) is similar. If r<s then re s*,but r≠r*; hence r*<s* If r*< s*,then there is a p∈ s* such that p生r*. Hencer≤p<s,so that r<s. This proves (c) Step We saw in Step 8 that the replacement of the rational numbers r by the corresponding “rational cuts”r*e R preserves sums, products, and order. This fact may be expressed by saying that the ordered field Q is isomorphic to the ordered field Q* whose elements are the rational cuts. Of course, r* is by no means the same as r, but the properties we are concerned with (arithmetic and order) are the same in the two fields It is this identification ofQ with Q* which allows us to regard Q as a subfield of R The second part of Theorem 1.19 is to be understood in terms of this identification. Note that the same phenomenon occurs when the real numbers are regarded as a subfield of the complex field, and it also occurs at a much more elementary level, when the integers are identified with a certain subset of Q It is a fact, which we will not prove here, that any two ordered fields with the least-upper-bound property are isomorphic. The first part of Theorem 1.19 therefore characterizes the real field R completely. The books by Landau and Thurston cited in the Bibliography are entirely devoted to number systems. Chapter 1 of Knopp's book contains a more leisurely description of how R can be obtained from Q. Another construction in which each real number is defined to be an equivalence class of Cauchy sequences of rational numbers (see Chap.3),is carried out in Sec. S of the book by Hewitt and Stromberg The cuts in Q which we used here were invented by Dedekind. The construction of R from Q by means of Cauchy sequences is due to Cantor. Both Cantor and Dedekind published their constructions in 1872. EXERCISES Unless the contrary is explicitly stated, all numbers that are mentioned in these exer cises are understood to be real 1.If r is rational (r + 0) and xis irrational, prove that $\scriptstyle{r+x}$ and rx are irrational.22 PRINCIPLES OF MATHEMATICAL ANALYsIs 2. Prove that there is no rational number whose square is 12. 3. Prove Proposition 1.15. 4. Let $\widehat{H}$ be a nonempty subset of an ordered set; suppose α is a lower bound of E and $\textstyle{\hat{\mathcal{N}}}$ is an upper bound of E. Prove that α≤β. 5. Let ${\mathcal{A}}\,$ be a nonempty set of real numbers which is bounded below. Let -A be the set of all numbers 一x, where x∈ A. Prove that $$ \mathrm{in}f\,A=-\operatorname{sup}(-A). $$ 6。 Fix b>1. (a) If m, 入,P,q are integers, $n>0,\,q>0,$ and r= m/n = P/g, prove tha $$ (b^{m})^{1/n}=(b^{p})^{1/q}. $$ Hence it makes sense to define $b^{r}=(b^{m})^{1/n}$ (b)Prove that $b^{r+s}=b^{r}b^{s}$ if r and $\begin{array}{l}{{\frac{\partial\mathbb{C}^{\sqrt{1}}}{\hbar\partial\!\!\!{\big/}}}}\end{array}$ y are rational (c) Ifx is real, define ${\mathcal{B}}(x)$ to be the set of all numbers b', where t is rational and 1≤x. Prove that $$ b^{r}=\operatorname{sup}B(r) $$ when r is rational. Hence it makes sense to define $$ b^{x}=\operatorname*{sup}B(x) $$ for every real x (d)Prove that $b^{x+y}=b^{x}b^{y}$ for all real x and y. 7. Fixb>1,y >0,and prove that there is a unique real x such that $b^{x}=y,$ by completing the following outline.(This x is called the logarithm of y to the base b.) (a) For any positive integer n, $\scriptstyle{b^{n}-1}$ ≥n(b- 1). (b) Hence b-1≥n(b1/"-1). (c) Ift>1 and n>(b-1)/(t- 1), then $b^{1/n}$ <1 Cb r wis sch ha b~ <), thenb"*4< for sficty large , o se ths apply part (c) with t=y·b-* (e)If b”>y,then bw -41/D>y for sufficintly large n (f) Let A be the set of all w such that b”<y, and show that $x=\operatorname*{sup}A$ l satisfies b* = y. (g) Prove that this xis unique 8. Prove that no order can be defined in the complex field that turns it into an ordered field. Hint:-1 is a square. 9. Suppose z =α+ bi,w = c+ di. Define z <w ifa<c, and also if a=c but b<d. Prove that this turns the set of all complex numbers into an ordered set. (This type of order relation is called a dictionary order, or lexicographic order, for obvious reasons.)Does this ordered set have the least-upper-bound property ? 10. Suppose $z=a+\quad$ bi, $w=u+i v,$ , and $$ a=\left(\frac{|w|+u}{2}\right)^{1/2},\qquad b=\left(\frac{|w|-u}{2}\right)^{1/2}. $$THE REAL AND COMPLEX NUMBER sYSTEMS23 Prove that $z^{2}\equiv1$ = w ifv≥0 and that (z) = w if v ≤0.Conclude that cvery complex number (with onc cxception!) has two complex square roots. 11. If z is a complex number, prove that there exists an r≥0 and a complex numbe w with lw|=1 such that z == rw. Are w and r always uniquely determined by z? 12。If z,……,Z。 are complex, prove that $$ |z_{1}+z_{2}+\cdots+z_{n}|\leq|z_{1}|+|z_{2}|+\cdots+|z_{n}| $$ 13. Ifx,y are complex, prove that $$ ||x|-|y|\leq|x-y|. $$ 14.If z is a complex number such that |z|=1, that is, such that zz =1, compute $$ |1+z|^{2}+|1-z|^{2}. $$ 15. Under what conditions does equality hold in the Schwarz inequality ? 16. Suppose k≥3,X,y e Rt,|x一y| =d>0, and r>0.Prove: (a) If 2r >d, there are infinitely many z ∈ $\textstyle{\mathcal{R}}^{k}$ such that $$ |z-\mathbf{x}|=|z-\mathbf{y}|=r. $$ (b)If 2r = d, there is exactly one such z (c) If 2r <d, there is no such z How must these statements be modified if ${\mathcal{N}}$ is 2 or 1? 17。Prove that $$ |\mathbf{x}+\mathbf{y}|^{2}+|\mathbf{x}-\mathbf{y}|^{2}=2|\mathbf{x}|^{2}+2|\mathbf{y}|^{2} $$ if xeI $R^{k}$ and y∈ R'. Interpret this geometrically, as a statement about parallel- ograms 18.If k≥2 and x∈ $Q^{k},$ prove that there exists $\scriptstyle y\in R^{n}$ such that $\scriptstyle y\neq0$ but $\mathbf{x}\cdot\mathbf{y}=0.$ Is this also true if $\scriptstyle{k=1}$ 1 ? 19. Suppose ae R', b e R*. Find ce $R^{k}$ and r>0O such that $$ |\mathbf{x}-\mathbf{a}|=2|\mathbf{x}-\mathbf{b}| $$ if and only if lx - c| =r (Solution: 3c = 4b一Q,3r = 2|b - a|.) 20.、With reference to the Appendix, suppose that property (III) were omitted from th definition of a cut. Keep the same definitions of order and addition. Show that the resulting ordered sct has the least-upper-bound property, that addition satisfies axioms (A1)to (A4)(with a slightly different zero-element!) but that (AS) fails