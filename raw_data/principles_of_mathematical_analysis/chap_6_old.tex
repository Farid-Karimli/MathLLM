\documentclass[10pt]{article}
\usepackage[utf8]{inputenc}
\usepackage[T1]{fontenc}
\usepackage{amsmath}
\usepackage{amsfonts}
\usepackage{amssymb}
\usepackage[version=4]{mhchem}
\usepackage{stmaryrd}
\usepackage{mathrsfs}

\begin{document}
\section{6}
\section{THE RIEMANN-STIELTJES INTEGRAL}
The present chapter is based on a definition of the Riemann integral which depends very explicitly on the order structure of the real line. Accordingly, we begin by discussing integration of real-valued functions on intervals. Extensions to complex- and vector-valued functions on intervals follow in later sections. Integration over sets other than intervals is discussed in Chaps. 10 and 11 .

\section{DEFINITION AND EXISTENCE OF THE INTEGRAL}
6.1 Definition Let $[a, b]$ be a given interval. By a partition $P$ of $[a, b]$ we mean a finite set of points $x_{0}, x_{1}, \ldots, x_{n}$, where

$$
a=x_{0} \leq x_{1} \leq \cdots \leq x_{n-1} \leq x_{n}=b
$$

We write

$$
\Delta x_{i}=x_{i}-x_{i-1} \quad(i=1, \ldots, n)
$$

Now suppose $f$ is a bounded real function defined on $[a, b]$. Corresponding to each partition $P$ of $[a, b]$ we put

$$
\begin{array}{rlrl}
M_{i} & =\sup f(x) & & \left(x_{i-1} \leq x \leq x_{i}\right), \\
m_{i} & =\inf f(x) & & \left(x_{i-1} \leq x \leq x_{i}\right), \\
U(P, f) & =\sum_{i=1}^{n} M_{i} \Delta x_{i}, & \\
L(P, f) & =\sum_{i=1}^{n} m_{i} \Delta x_{i}, &
\end{array}
$$

and finally

$$
\begin{aligned}
& \int_{a}^{b} f d x=\inf U(P, f) \\
& \int_{a}^{b} f d x=\sup L(P, f)
\end{aligned}
$$

where the inf and the sup are taken over all partitions $P$ of $[a, b]$. The left members of (1) and (2) are called the upper and lower Riemann integrals of $f$ over $[a, b]$, respectively.

If the upper and lower integrals are equal, we say that $f$ is Riemannintegrable on $[a, b]$, we write $f \in \mathscr{R}$ (that is, $\mathscr{R}$ denotes the set of Riemannintegrable functions), and we denote the common value of (1) and (2) by

$$
\int_{a}^{b} f d x,
$$

or by

$$
\int_{a}^{b} f(x) d x
$$

This is the Riemann integral of $f$ over $[a, b]$. Since $f$ is bounded, there exist two numbers, $m$ and $M$, such that

$$
m \leq f(x) \leq M \quad(a \leq x \leq b)
$$

Hence, for every $P$,

$$
m(b-a) \leq L(P, f) \leq U(P, f) \leq M(b-a)
$$

so that the numbers $L(P, f)$ and $U(P, f)$ form a bounded set. This shows that the upper and lower integrals are defined for every bounded function $f$. The question of their equality, and hence the question of the integrability of $f$, is a more delicate one. Instead of investigating it separately for the Riemann integral, we shall immediately consider a more general situation.

6.2 Definition Let $\alpha$ be a monotonically increasing function on $[a, b]$ (since $\alpha(a)$ and $\alpha(b)$ are finite, it follows that $\alpha$ is bounded on $[a, b])$. Corresponding to each partition $P$ of $[a, b]$, we write

$$
\Delta \alpha_{i}=\alpha\left(x_{i}\right)-\alpha\left(x_{i-1}\right)
$$

It is clear that $\Delta \alpha_{i} \geq 0$. For any real function $f$ which is bounded on $[a, b]$ we put

$$
\begin{aligned}
& U(P, f, \alpha)=\sum_{i=1}^{n} M_{i} \Delta \alpha_{i} \\
& L(P, f, \alpha)=\sum_{i=1}^{n} m_{i} \Delta \alpha_{i}
\end{aligned}
$$

where $M_{i}, m_{i}$ have the same meaning as in Definition 6.1 , and we define

$$
\begin{aligned}
& \int_{a}^{b} f d \alpha=\inf U(P, f, \alpha) \\
& \int_{a}^{b} f d \alpha=\sup L(P, f, \alpha)
\end{aligned}
$$

the inf and sup again being taken over all partitions.

If the left members of (5) and (6) are equal, we denote their common value by

$$
\int_{a}^{b} f d \alpha
$$

or sometimes by

$$
\int_{a}^{b} f(x) d \alpha(x)
$$

This is the Riemann-Stieltjes integral (or simply the Stieltjes integral) of $f$ with respect to $\alpha$, over $[a, b]$.

If (7) exists, i.e., if (5) and (6) are equal, we say that $f$ is integrable with respect to $\alpha$, in the Riemann sense, and write $f \in \mathscr{R}(\alpha)$.

By taking $\alpha(x)=x$, the Riemann integral is seen to be a special case of the Riemann-Stieltjes integral. Let us mention explicitly, however, that in the general case $\alpha$ need not even be continuous.

A few words should be said about the notation. We prefer (7) to (8), since the letter $x$ which appears in (8) adds nothing to the content of (7). It is immaterial which letter we use to represent the so-called "variable of integration." For instance, (8) is the same as

$$
\int_{a}^{b} f(y) d \alpha(y)
$$

The integral depends on $f, \alpha, a$ and $b$, but not on the variable of integration, which may as well be omitted.

The role played by the variable of integration is quite analogous to that of the index of summation: The two symbols

$$
\sum_{i=1}^{n} c_{i}, \quad \sum_{k=1}^{n} c_{k}
$$

mean the same thing, since each means $c_{1}+c_{2}+\cdots+c_{n}$.

Of course, no harm is done by inserting the variable of integration, and in many cases it is actually convenient to do so.

We shall now investigate the existence of the integral (7). Without saying so every time, $f$ will be assumed real and bounded, and $\alpha$ monotonically increasing on $[a, b]$; and, when there can be no misunderstanding, we shall write $\int$ in place of $\int_{a}^{b}$.

6.3 Definition We say that the partition $P^{*}$ is a refinement of $P$ if $P^{*} \supset P$ (that is, if every point of $P$ is a point of $P^{*}$ ). Given two partitions, $P_{1}$ and $P_{2}$, we say that $P^{*}$ is their common refinement if $P^{*}=P_{1} \cup P_{2}$.

6.4 Theorem If $P^{*}$ is a refinement of $P$, then

and

$$
L(P, f, \alpha) \leq L\left(P^{*}, f, \alpha\right)
$$

$$
U\left(P^{*}, f, \alpha\right) \leq U(P, f, \alpha)
$$

Proof To prove (9), suppose first that $P^{*}$ contains just one point more than $P$. Let this extra point be $x^{*}$, and suppose $x_{i-1}<x^{*}<x_{i}$, where $x_{i-1}$ and $x_{i}$ are two consecutive points of $P$. Put

$$
\begin{array}{ll}
w_{1}=\inf f(x) & \left(x_{i-1} \leq x \leq x^{*}\right) \\
w_{2}=\inf f(x) & \left(x^{*} \leq x \leq x_{i}\right)
\end{array}
$$

Clearly $w_{1} \geq m_{i}$ and $w_{2} \geq m_{i}$, where, as before,

Hence

$$
m_{i}=\inf f(x) \quad\left(x_{i-1} \leq x \leq x_{i}\right)
$$

$$
\begin{aligned}
& L\left(P^{*}, f, \alpha\right)-L(P, f, \alpha) \\
& \quad=w_{1}\left[\alpha\left(x^{*}\right)-\alpha\left(x_{i-1}\right)\right]+w_{2}\left[\alpha\left(x_{i}\right)-\alpha\left(x^{*}\right)\right]-m_{i}\left[\alpha\left(x_{i}\right)-\alpha\left(x_{i-1}\right)\right] \\
& \quad=\left(w_{1}-m_{i}\right)\left[\alpha\left(x^{*}\right)-\alpha\left(x_{i-1}\right)\right]+\left(w_{2}-m_{i}\right)\left[\alpha\left(x_{i}\right)-\alpha\left(x^{*}\right)\right] \geq 0
\end{aligned}
$$

If $P^{*}$ contains $k$ points more than $P$, we repeat this reasoning $k$ times, and arrive at (9). The proof of (10) is analogous.

6.5 Theorem $\int_{a}^{b} f d \alpha \leq \int_{a}^{b} f d \alpha$.

Proof Let $P^{*}$ be the common refinement of two partitions $P_{1}$ and $P_{2}$. By Theorem 6.4,

$$
L\left(P_{1}, f, \alpha\right) \leq L\left(P^{*}, f, \alpha\right) \leq U\left(P^{*}, f, \alpha\right) \leq U\left(P_{2}, f, \alpha\right)
$$

Hence

$$
L\left(P_{1}, f, \alpha\right) \leq U\left(P_{2}, f, \alpha\right)
$$

If $P_{2}$ is fixed and the sup is taken over all $P_{1}$, (11) gives

$$
\int_{\underline{1}} f d \alpha \leq U\left(P_{2}, f, \alpha\right)
$$

The theorem follows by taking the inf over all $P_{2}$ in (12).

6.6 Theorem $f \in \mathscr{R}(\alpha)$ on $[a, b]$ if and only if for every $\varepsilon>0$ there exists a partition $P$ such that

$$
U(P, f, \alpha)-L(P, f, \alpha)<\varepsilon
$$

Proof For every $P$ we have

$$
L(P, f, \alpha) \leq \int_{0} f d \alpha \leq \bar{\int} f d \alpha \leq U(P, f, \alpha)
$$

Thus (13) implies

$$
0 \leq \bar{\int} f d \alpha-\int f d \alpha<\varepsilon .
$$

Hence, if (13) can be satisfied for every $\varepsilon>0$, we have

$$
\bar{\int} f d \alpha=\int f d \alpha
$$

that is, $f \in \mathscr{R}(\alpha)$.

Conversely, suppose $f \in \mathscr{R}(\alpha)$, and let $\varepsilon>0$ be given. Then there exist partitions $P_{1}$ and $P_{2}$ such that

$$
\begin{aligned}
& U\left(P_{2}, f, \alpha\right)-\int f d \alpha<\frac{\varepsilon}{2} \\
& \int f d \alpha-L\left(P_{1}, f, \alpha\right)<\frac{\varepsilon}{2}
\end{aligned}
$$

We choose $P$ to be the common refinement of $P_{1}$ and $P_{2}$. Then Theorem 6.4 , together with (14) and (15), shows that

$$
U(P, f, \alpha) \leq U\left(P_{2}, f, \alpha\right)<\int f d \alpha+\frac{\varepsilon}{2}<L\left(P_{1}, f, \alpha\right)+\varepsilon \leq L(P, f, \alpha)+\varepsilon
$$

so that (13) holds for this partition $P$.

Theorem 6.6 furnishes a convenient criterion for integrability. Before we apply it, we state some closely related facts.

\subsection{Theorem}
(a) If (13) holds for some $P$ and some $\varepsilon$, then (13) holds (with the same $\varepsilon$ ) for every refinement of $P$.

(b) If (13) holds for $P=\left\{x_{0}, \ldots, x_{n}\right\}$ and if $s_{i}, t_{i}$ are arbitrary points in $\left[x_{i-1}, x_{i}\right]$, then

$$
\sum_{i=1}^{n}\left|f\left(s_{i}\right)-f\left(t_{i}\right)\right| \Delta \alpha_{i}<\varepsilon
$$

(c) If $f \in \mathscr{R}(\alpha)$ and the hypotheses of (b) hold, then

$$
\left|\sum_{i=1}^{n} f\left(t_{i}\right) \Delta \alpha_{i}-\int_{a}^{b} f d \alpha\right|<\varepsilon
$$

Proof Theorem 6.4 implies $(a)$. Under the assumptions made in $(b)$, both $f\left(s_{i}\right)$ and $f\left(t_{i}\right)$ lie in $\left[m_{i}, M_{i}\right]$, so that $\left|f\left(s_{i}\right)-f\left(t_{i}\right)\right| \leq M_{i}-m_{i}$. Thus

$$
\sum_{i=1}^{n}\left|f\left(s_{i}\right)-f\left(t_{i}\right)\right| \Delta \alpha_{i} \leq U(P, f, \alpha)-L(P, f, \alpha)
$$

which proves $(b)$. The obvious inequalities

and

$$
L(P, f, \alpha) \leq \sum f\left(t_{i}\right) \Delta \alpha_{i} \leq U(P, f, \alpha)
$$

$$
L(P, f, \alpha) \leq \int f d \alpha \leq U(P, f, \alpha)
$$

prove $(c)$.

6.8 Theorem If $f$ is continuous on $[a, b]$ then $f \in \mathscr{R}(\alpha)$ on $[a, b]$.

Proof Let $\varepsilon>0$ be given. Choose $\eta>0$ so that

$$
[\alpha(b)-\alpha(a)] \eta<\varepsilon
$$

Since $f$ is uniformly continuous on $[a, b]$ (Theorem 4.19), there exists a $\delta>0$ such that

$$
|f(x)-f(t)|<\eta
$$

if $x \in[a, b], t \in[a, b]$, and $|x-t|<\delta$.

If $P$ is any partition of $[a, b]$ such that $\Delta x_{i}<\delta$ for all $i$, then (16) implies that

$$
M_{i}-m_{i} \leq \eta \quad(i-1, \ldots, n)
$$

and therefore

$$
\begin{aligned}
U(P, f, \alpha)-L(P, f, \alpha) & =\sum_{i=1}^{n}\left(M_{i}-m_{i}\right) \Delta \alpha_{i} \\
\leq \eta \sum_{i=1}^{n} \Delta \alpha_{i} & =\eta[\alpha(b)-\alpha(a)]<\varepsilon
\end{aligned}
$$

By Theorem 6.6, $f \in \mathscr{R}(\alpha)$.

6.9 Theorem If $f$ is monotonic on $[a, b]$, and if $\alpha$ is continuous on $[a, b]$, then $f \in \mathscr{R}(\alpha)$. (We still assume, of course, that $\alpha$ is monotonic.)

Proof Let $\varepsilon>0$ be given. For any positive integer $n$, choose a partition such that

$$
\Delta \alpha_{i}=\frac{\alpha(b)-\alpha(a)}{n} \quad(i=1, \ldots, n)
$$

This is possible since $\alpha$ is continuous (Theorem 4.23).

We suppose that $f$ is monotonically increasing (the proof is analogous in the other case). Then

so that

$$
M_{i}=f\left(x_{i}\right), \quad m_{i}=f\left(x_{i-1}\right) \quad(i=1, \ldots, n)
$$

$$
\begin{aligned}
U(P, f, \alpha)-L(P, f, \alpha) & =\frac{\alpha(b)-\alpha(a)}{n} \sum_{i=1}^{n}\left[f\left(x_{i}\right)-f\left(x_{i-1}\right)\right] \\
& =\frac{\alpha(b)-\alpha(a)}{n} \cdot[f(b)-f(a)]<\varepsilon
\end{aligned}
$$

if $n$ is taken large enough. By Theorem 6.6, $f \in \mathscr{R}(\alpha)$.

6.10 Theorem Suppose $f$ is bounded on $[a, b], f$ has only finitely many points of discontinuity on $[a, b]$, and $\alpha$ is continuous at every point at which $f$ is discontinuous. Then $f \in \mathscr{R}(\alpha)$.

Proof Let $\varepsilon>0$ be given. Put $M=\sup |f(x)|$, let $E$ be the set of points at which $f$ is discontinuous. Since $E$ is finite and $\alpha$ is continuous at every point of $E$, we can cover $E$ by finitely many disjoint intervals $\left[u_{j}, v_{j}\right] \subset$ $[a, b]$ such that the sum of the corresponding differences $\alpha\left(v_{j}\right)-\alpha\left(u_{j}\right)$ is less than $\varepsilon$. Furthermore, we can place these intervals in such a way that every point of $E \cap(a, b)$ lies in the interior of some $\left[u_{j}, v_{j}\right]$.

Remove the segments $\left(u_{j}, v_{j}\right)$ from $[a, b]$. The remaining set $K$ is compact. Hence $f$ is uniformly continuous on $K$, and there exists $\delta>0$ such that $|f(s)-f(t)|<\varepsilon$ if $s \in K, t \in K,|s-t|<\delta$.

Now form a partition $P=\left\{x_{0}, x_{1}, \ldots, x_{n}\right\}$ of $[a, b]$, as follows: Each $u_{j}$ occurs in $P$. Each $v_{j}$ occurs in $P$. No point of any segment $\left(u_{j}, v_{j}\right)$ occurs in $P$. If $x_{i-1}$ is not one of the $u_{j}$, then $\Delta x_{i}<\delta$.

Note that $M_{i}-m_{i} \leq 2 M$ for every $i$, and that $M_{i}-m_{i} \leq \varepsilon$ unless $x_{i-1}$ is one of the $u_{j}$. Hence, as in the proof of Theorem 6.8,

$$
U(P, f, \alpha)-L(P, f, \alpha) \leq[\alpha(b)-\alpha(a)] \varepsilon+2 M \varepsilon
$$

Since $\varepsilon$ is arbitrary, Theorem 6.6 shows that $f \in \mathscr{R}(\alpha)$.

Note: If $f$ and $\alpha$ have a common point of discontinuity, then $f$ need not be in $\mathscr{R}(\alpha)$. Exercise 3 shows this.

6.11 Theorem Suppose $f \in \mathscr{R}(\alpha)$ on $[a, b], m \leq f \leq M, \phi$ is continuous on $[m, M]$, and $h(x)=\phi(f(x))$ on $[a, b]$. Then $h \in \mathscr{R}(\alpha)$ on $[a, b]$.

Proof Choose $\varepsilon>0$. Since $\phi$ is uniformly continuous on $[m, M]$, there exists $\delta>0$ such that $\delta<\varepsilon$ and $|\phi(s)-\phi(t)|<\varepsilon$ if $|s-t| \leq \delta$ and $s, t \in[m, M]$.

Since $f \in \mathscr{R}(\alpha)$, there is a partition $P=\left\{x_{0}, x_{1}, \ldots, x_{n}\right\}$ of $[a, b]$ such that

$$
U(P, f, \alpha)-L(P, f, \alpha)<\delta^{2}
$$

Let $M_{i}, m_{i}$ have the same meaning as in Definition 6.1, and let $M_{i}^{*}, m_{i}^{*}$ be the analogous numbers for $h$. Divide the numbers $1, \ldots, n$ into two classes: $i \in A$ if $M_{i}-m_{i}<\delta, i \in B$ if $M_{i}-m_{i} \geq \delta$.

For $i \in A$, our choice of $\delta$ shows that $M_{i}^{*}-m_{i}^{*} \leq \varepsilon$.

For $i \in B, M_{i}^{*}-m_{i}^{*} \leq 2 K$, where $K=\sup |\phi(t)|, m \leq t \leq M$. By (18), we have

$$
\delta \sum_{i \in B} \Delta \alpha_{i} \leq \sum_{i \in B}\left(M_{i}-m_{i}\right) \Delta \alpha_{i}<\delta^{2}
$$

so that $\sum_{i \in B} \Delta \alpha_{i}<\delta$. It follows that

$$
\begin{aligned}
U(P, h, \alpha)-L(P, h, \alpha) & =\sum_{i \in A}\left(M_{i}^{*}-m_{i}^{*}\right) \Delta \alpha_{i}+\sum_{i \in B}\left(M_{i}^{*}-m_{i}^{*}\right) \Delta \alpha_{i} \\
& \leq \varepsilon[\alpha(b)-\alpha(a)]+2 K \delta<\varepsilon[\alpha(b)-\alpha(a)+2 K]
\end{aligned}
$$

Since $\varepsilon$ was arbitrary, Theorem 6.6 implies that $h \in \mathscr{R}(\alpha)$.

Remark: This theorem suggests the question: Just what functions are Riemann-integrable? The answer is given by Theorem $11.33(b)$.

\section{PROPERTIES OF THE INTEGRAL}
\subsection{Theorem}
(a) If $f_{1} \in \mathscr{R}(\alpha)$ and $f_{2} \in \mathscr{R}(\alpha)$ on $[a, b]$, then

$$
f_{1}+f_{2} \in \mathscr{R}(\alpha)
$$

$c f \in \mathscr{R}(\alpha)$ for every constant $c$, and

$$
\begin{aligned}
\int_{a}^{b}\left(f_{1}+f_{2}\right) d \alpha & =\int_{a}^{b} f_{1} d \alpha+\int_{a}^{b} f_{2} d \alpha \\
\int_{a}^{b} c f d \alpha & =c \int_{a}^{b} f d \alpha .
\end{aligned}
$$

(b) If $f_{1}(x) \leq f_{2}(x)$ on $[a, b]$, then

$$
\int_{a}^{b} f_{1} d \alpha \leq \int_{a}^{b} f_{2} d \alpha
$$

(c) If $f \in \mathscr{R}(\alpha)$ on $[a, b]$ and if $a<c<b$, then $f \in \mathscr{R}(\alpha)$ on $[a, c]$ and on $[c, b]$, and

$$
\int_{a}^{c} f d \alpha+\int_{c}^{b} f d \alpha=\int_{a}^{b} f d \alpha
$$

(d) If $f \in \mathscr{R}(\alpha)$ on $[a, b]$ and if $|f(x)| \leq M$ on $[a, b]$, then

$$
\left|\int_{a}^{b} f d \alpha\right| \leq M[\alpha(b)-\alpha(a)]
$$

(e) If $f \in \mathscr{R}\left(\alpha_{1}\right)$ and $f \in \mathscr{R}\left(\alpha_{2}\right)$, then $f \in \mathscr{R}\left(\alpha_{1}+\alpha_{2}\right)$ and

$$
\int_{a}^{b} f d\left(\alpha_{1}+\alpha_{2}\right)=\int_{a}^{b} f d \alpha_{1}+\int_{a}^{b} f d \alpha_{2}
$$

if $f \in \mathscr{R}(\alpha)$ and $c$ is a positive constant, then $f \in \mathscr{R}(c \alpha)$ and

$$
\int_{a}^{b} f d(c \alpha)=c \int_{a}^{b} f d \alpha
$$

Proof If $f=f_{1}+f_{2}$ and $P$ is any partition of $[a, b]$, we have

$$
\begin{aligned}
L\left(P, f_{1}, \alpha\right)+L\left(P, f_{2}, \alpha\right) \leq L(P, f, \alpha) & \\
& \leq U(P, f, \alpha) \leq U\left(P, f_{1}, \alpha\right)+U\left(P, f_{2}, \alpha\right)
\end{aligned}
$$

If $f_{1} \in \mathscr{R}(\alpha)$ and $f_{2} \in \mathscr{R}(\alpha)$, let $\varepsilon>0$ be given. There are partitions $P_{j}$ $(j=1,2)$ such that

$$
U\left(P_{j}, f_{j}, \alpha\right)-L\left(P_{j}, f_{j}, \alpha\right)<\varepsilon
$$

These inequalities persist if $P_{1}$ and $P_{2}$ are replaced by their common refinement $P$. Then (20) implies

$$
U(P, f, \alpha)-L(P, f, \alpha)<2 \varepsilon
$$

which proves that $f \in \mathscr{R}(\alpha)$.

With this same $P$ we have

$$
U\left(P, f_{J}, \alpha\right)<\int f_{j} d \alpha+\varepsilon \quad(j=1,2)
$$

hence (20) implies

$$
\int f d \alpha \leq U(P, f, \alpha)<\int f_{1} d \alpha+\int f_{2} d \alpha+2 \varepsilon
$$

Since $\varepsilon$ was arbitrary, we conclude that

$$
\int f d \alpha \leq \int f_{1} d \alpha+\int f_{2} d \alpha
$$

If we replace $f_{1}$ and $f_{2}$ in (21) by $-f_{1}$ and $-f_{2}$, the inequality is reversed, and the equality is proved.

The proofs of the other assertions of Theorem 6.12 are so similar that we omit the details. In part $(c)$ the point is that (by passing to refinements) we may restrict ourselves to partitions which contain the point $c$, in approximating $\int f d \alpha$.

6.13 Theorem If $f \in \mathscr{R}(\alpha)$ and $g \in \mathscr{R}(\alpha)$ on $[a, b]$, then

(a) $f g \in \mathscr{R}(\alpha)$;

(b) $|f| \in \mathscr{R}(\alpha)$ and $\left|\int_{a}^{b} f d \alpha\right| \leq \int_{a}^{b}|f| d \alpha$.

Proof If we take $\phi(t)=t^{2}$, Theorem 6.11 shows that $f^{2} \in \mathscr{R}(\alpha)$ if $f \in \mathscr{R}(\alpha)$. The identity

$$
4 f g=(f+g)^{2}-(f-g)^{2}
$$

completes the proof of $(a)$.

If we take $\phi(t)=|t|$, Theorem 6.11 shows similarly that $|f| \in \mathscr{R}(\alpha)$. Choose $c= \pm 1$, so that

Then

$$
c \int f d \alpha \geq 0
$$

$$
\left|\int f d \alpha\right|=c \int f d \alpha=\int c f d \alpha \leq-\int|f| d \alpha
$$

since $c f \leq|f|$.

6.14 Definition The unit step function $I$ is defined by

$$
I(x)= \begin{cases}0 & (x \leq 0) \\ 1 & (x>0)\end{cases}
$$

6.15 Theorem If $a<s<b, f$ is bounded on $[a, b], f$ is continuous at $s$, and $\alpha(x)=I(x-s)$, then

$$
\int_{a}^{b} f d \alpha=f(s)
$$

Proof Consider partitions $P=\left\{x_{0}, x_{1}, x_{2}, x_{3}\right\}$, where $x_{0}=a$, and $x_{1}=s<x_{2}<x_{3}=b$. Then

$$
U(P, f, \alpha)=M_{2}, \quad L(P, f, \alpha)=m_{2}
$$

Since $f$ is continuous at $s$, we see that $M_{2}$ and $m_{2}$ converge to $f(s)$ as $x_{2} \rightarrow s$.

6.16 Theorem Suppose $c_{n} \geq 0$ for $1,2,3, \ldots, \Sigma c_{n}$ converges, $\left\{s_{n}\right\}$ is a sequence of distinct points in $(a, b)$, and

$$
\alpha(x)=\sum_{n=1}^{\infty} c_{n} I\left(x-s_{n}\right)
$$

Let $f$ be continuous on $[a, b]$. Then

$$
\int_{a}^{b} f d \alpha=\sum_{n=1}^{\infty} c_{n} f\left(s_{n}\right)
$$

Proof The comparison test shows that the series (22) converges for every $x$. Its sum $\alpha(x)$ is evidently monotonic, and $\alpha(a)=0, \alpha(b)=\Sigma c_{n}$. (This is the type of function that occurred in Remark 4.31.)

Let $\varepsilon>0$ be given, and choose $N$ so that

$$
\sum_{N+1}^{\infty} c_{n}<\varepsilon
$$

Put

$$
\alpha_{1}(x)=\sum_{n=1}^{N} c_{n} I\left(x-s_{n}\right), \quad \alpha_{2}(x)=\sum_{N+1}^{\infty} c_{n} I\left(x-s_{n}\right)
$$

By Theorems 6.12 and 6.15,

$$
\int_{a}^{b} f d \alpha_{1}=\sum_{i=1}^{N} c_{n} f\left(s_{n}\right)
$$

Since $\alpha_{2}(b)-\alpha_{2}(a)<\varepsilon$

$$
\left|\int_{a}^{b} f d \alpha_{2}\right| \leq M \varepsilon
$$

where $M=\sup |f(x)|$. Since $\alpha=\alpha_{1}+\alpha_{2}$, it follows from (24) and (25) that

$$
\left|\int_{a}^{b} f d \alpha-\sum_{i=1}^{N} c_{n} f\left(s_{n}\right)\right| \leq M \varepsilon
$$

If we let $N \rightarrow \infty$, we obtain (23).

6.17 Theorem Assume $\alpha$ increases monotonically and $\alpha^{\prime} \in \mathscr{R}$ on $[a, b]$. Let $f$ be a bounded real function on $[a, b]$.

Then $f \in \mathscr{R}(\alpha)$ if and only if $f \alpha^{\prime} \in \mathscr{R}$. In that case

$$
\int_{a}^{b} f d \alpha=\int_{a}^{b} f(x) \alpha^{\prime}(x) d x .
$$

Proof Let $\varepsilon>0$ be given and apply Theorem 6.6 to $\alpha^{\prime}$ : There is a partition $P=\left\{x_{0}, \ldots, x_{n}\right\}$ of $[a, b]$ such that

$$
U\left(P, \alpha^{\prime}\right)-L\left(P, \alpha^{\prime}\right)<\varepsilon .
$$

The mean value theorem furnishes points $t_{i} \in\left[x_{i-1}, x_{i}\right]$ such that

$$
\Delta \alpha_{i}=\alpha^{\prime}\left(t_{i}\right) \Delta x_{i}
$$

for $i=1, \ldots, n$. If $s_{i} \in\left[x_{i-1}, x_{i}\right]$, then

$$
\sum_{i=1}^{n}\left|\alpha^{\prime}\left(s_{i}\right)-\alpha^{\prime}\left(t_{i}\right)\right| \Delta x_{i}<\varepsilon
$$

by (28) and Theorem 6.7(b). Put $M=\sup |f(x)|$. Since

$$
\sum_{i=1}^{n} f\left(s_{i}\right) \Delta \alpha_{i}=\sum_{i=1}^{n} f\left(s_{i}\right) \alpha^{\prime}\left(t_{i}\right) \Delta x_{i}
$$

it follows from (29) that

$$
\left|\sum_{i=1}^{n} f\left(s_{i}\right) \Delta \alpha_{i}-\sum_{i=1}^{n} f\left(s_{i}\right) \alpha^{\prime}\left(s_{i}\right) \Delta x_{i}\right| \leq M \varepsilon
$$

In particular,

$$
\sum_{i=1}^{n} f\left(s_{i}\right) \Delta \alpha_{i} \leq U\left(P, f \alpha^{\prime}\right)+M \varepsilon
$$

for all choices of $s_{i} \in\left[x_{i-1}, x_{i}\right]$, so that

$$
U(P, f, \alpha) \leq U\left(P, f \alpha^{\prime}\right)+M \varepsilon
$$

The same argument leads from (30) to

Thus

$$
U\left(P, f \alpha^{\prime}\right) \leq U(P, f, \alpha)+M \varepsilon
$$

$$
\left|U(P, f, \alpha)-U\left(P, f \alpha^{\prime}\right)\right| \leq M \varepsilon
$$

Now note that (28) remains true if $P$ is replaced by any refinement. Hence (31) also remains true. We conclude that

$$
\left|\int_{a}^{b} f d \alpha-\bar{\int}_{a}^{b} f(x) \alpha^{\prime}(x) d x\right| \leq M \varepsilon
$$

But $\varepsilon$ is arbitrary. Hence

$$
\int_{a}^{b} f d \alpha=\bar{\int}_{a}^{b} f(x) \alpha^{\prime}(x) d x
$$

for any bounded $f$. The equality of the lower integrals follows from (30) in exactly the same way. The theorem follows.

6.18 Remark The two preceding theorems illustrate the generality and flexibility which are inherent in the Stieltjes process of integration. If $\alpha$ is a pure step function [this is the name often given to functions of the form (22)], the integral reduces to a finite or infinite series. If $\alpha$ has an integrable derivative, the integral reduces to an ordinary Riemann integral. This makes it possible in many cases to study series and integrals simultaneously, rather than separately.

To illustrate this point, consider a physical example. The moment of inertia of a straight wire of unit length, about an axis through an endpoint, at right angles to the wire, is

$$
\int_{0}^{1} x^{2} d m
$$

where $m(x)$ is the mass contained in the interval $[0, x]$. If the wire is regarded as having a continuous density $\rho$, that is, if $m^{\prime}(x)=\rho(x)$, then (33) turns into

$$
\int_{0}^{1} x^{2} \rho(x) d x
$$

On the other hand, if the wire is composed of masses $m_{i}$ concentrated at points $x_{i},(33)$ becomes

$$
\sum_{i} x_{i}^{2} m_{i}
$$

Thus (33) contains (34) and (35) as special cases, but it contains much more; for instance, the case in which $m$ is continuous but not everywhere differentiable.

6.19 Theorem (change of variable) Suppose $\varphi$ is a strictly increasing continuous function that maps an interval $[A, B]$ onto $[a, b]$. Suppose $\alpha$ is monotonically increasing on $[a, b]$ and $f \in \mathscr{R}(\alpha)$ on $[a, b]$. Define $\beta$ and $g$ on $[A, B]$ by

$$
\beta(y)=\alpha(\varphi(y)), \quad g(y)=f(\varphi(y)) .
$$

Then $g \in \mathscr{R}(\beta)$ and

$$
\int_{A}^{B} g d \beta=\int_{a}^{b} f d \alpha .
$$

Proof To each partition $P=\left\{x_{0}, \ldots, x_{n}\right\}$ of $[a, b]$ corresponds a partition $Q=\left\{y_{0}, \ldots, y_{n}\right\}$ of $[A, B]$, so that $x_{i}=\varphi\left(y_{i}\right)$. All partitions of $[A, B]$ are obtained in this way. Since the values taken by $f$ on $\left[x_{i-1}, x_{i}\right]$ are exactly the same as those taken by $g$ on $\left[y_{i-1}, y_{i}\right]$, we see that

$$
U(Q, g, \beta)=U(P, f, \alpha), \quad L(Q, g, \beta)=L(P, f, \alpha)
$$

Since $f \in \mathscr{R}(\alpha), P$ can be chosen so that both $U(P, f, \alpha)$ and $L(P, f, \alpha)$ are close to $\int f d \alpha$. Hence (38), combined with Theorem 6.6, shows that $g \in \mathscr{R}(\beta)$ and that (37) holds. This completes the proof.

Let us note the following special case:

Take $\alpha(x)=x$. Then $\beta=\varphi$. Assume $\varphi^{\prime} \in \mathscr{R}$ on $[A, B]$. If Theorem 6.17 is applied to the left side of (37), we obtain

$$
\int_{a}^{b} f(x) d x=\int_{A}^{B} f(\varphi(y)) \varphi^{\prime}(y) d y
$$

\section{INTEGRATION AND DIFFERENTIATION}
We still confine ourselves to real functions in this section. We shall show that integration and differentiation are, in a certain sense, inverse operations.

6.20 Theorem Let $f \in \mathscr{R}$ on $[a, b]$. For $a \leq x \leq b$, put

$$
F(x)=\int_{a}^{x} f(t) d t
$$

Then $F$ is continuous on $[a, b]$; furthermore, if $f$ is continuous at a point $x_{0}$ of $[a, b]$, then $F$ is differentiable at $x_{0}$, and

$$
F^{\prime}\left(x_{0}\right)=f\left(x_{0}\right)
$$

Proof Since $f \in \mathscr{R}, f$ is bounded. Suppose $|f(t)| \leq M$ for $a \leq t \leq b$. If $a \leq x<y \leq b$, then

$$
|F(y)-F(x)|=\left|\int_{x}^{y} f(t) d t\right| \leq M(y-x)
$$

by Theorem $6.12(c)$ and $(d)$. Given $\varepsilon>0$, we see that

$$
|F(y)-F(x)|<\varepsilon
$$

provided that $|y-x|<\varepsilon / M$. This proves continuity (and, in fact, uniform continuity) of $F$.

Now suppose $f$ is continuous at $x_{0}$. Given $\varepsilon>0$, choose $\delta>0$ such that

$$
\left|f(t)-f\left(x_{0}\right)\right|<\varepsilon
$$

if $\left|t-x_{0}\right|<\delta$, and $a \leq t \leq b$. Hence, if

$$
x_{0}-\delta<s \leq x_{0} \leq t<x_{0}+\delta \quad \text { and } \quad a \leq s<t \leq b
$$

we have, by Theorem 6.12(d),

$$
\left|\frac{F(t)-F(s)}{t-s}-f\left(x_{0}\right)\right|=\left|\frac{1}{t-s} \int_{s}^{t}\left[f(u)-f\left(x_{0}\right)\right] d u\right|<\varepsilon .
$$

It follows that $F^{\prime}\left(x_{0}\right)=f\left(x_{0}\right)$.

6.21 The fundamental theorem of calculus If $f \in \mathscr{R}$ on $[a, b]$ and if there is a differentiable function $F$ on $[a, b]$ such that $F^{\prime}=f$, then

$$
\int_{a}^{b} f(x) d x=F(b)-F(a)
$$

Proof Let $\varepsilon>0$ be given. Choose a partition $P=\left\{x_{0}, \ldots, x_{n}\right\}$ of $[a, b]$ so that $U(P, f)-L(P, f)<\varepsilon$. The mean value theorem furnishes points $t_{i} \in\left[x_{i-1}, x_{i}\right]$ such that

$$
F\left(x_{i}\right)-F\left(x_{i-1}\right)=f\left(t_{i}\right) \Delta x_{i}
$$

for $i=1, \ldots, n$. Thus

$$
\sum_{i=1}^{n} f\left(t_{i}\right) \Delta x_{i}=F(b)-F(a)
$$

It now follows from Theorem 6.7(c) that

$$
\left|F(b)-F(a)-\int_{a}^{b} f(x) d x\right|<\varepsilon
$$

Since this holds for every $\varepsilon>0$, the proof is complete.

6.22 Theorem (integration by parts) Suppose $F$ and $G$ are differentiable functions on $[a, b], F^{\prime}=f \in \mathscr{R}$, and $G^{\prime}=g \in \mathscr{R}$. Then

$$
\int_{a}^{b} F(x) g(x) d x=F(b) G(b)-F(a) G(a)-\int_{a}^{b} f(x) G(x) d x
$$

Proof Put $H(x)=F(x) G(x)$ and apply Theorem 6.21 to $H$ and its derivative. Note that $H^{\prime} \in \mathscr{R}$, by Theorem 6.13 .

\section{INTEGRATION OF VECTOR-VALUED FUNCTIONS}
6.23 Definition Let $f_{1}, \ldots, f_{k}$ be real functions on $[a, b]$, and let $\mathbf{f}=\left(f_{1}, \ldots, f_{k}\right)$ be the corresponding mapping of $[a, b]$ into $R^{k}$. If $\alpha$ increases monotonically on $[a, b]$, to say that $\mathbf{f} \in \mathscr{R}(\alpha)$ means that $f_{j} \in \mathscr{R}(\alpha)$ for $j=1, \ldots, k$. If this is the case, we define

$$
\int_{a}^{b} \mathbf{f} d \alpha=\left(\int_{a}^{b} f_{1} d \alpha, \ldots, \int_{a}^{b} f_{k} d \alpha\right)
$$

In other words, $\int \mathbf{f} d \alpha$ is the point in $R^{k}$ whose $j$ th coordinate is $\int f_{j} d \alpha$.

It is clear that parts $(a),(c)$, and $(e)$ of Theorem 6.12 are valid for these vector-valued integrals; we simply apply the earlier results to each coordinate. The same is true of Theorems 6.17, 6.20, and 6.21. To illustrate, we state the analogue of Theorem 6.21.

6.24 Theorem If $\mathbf{f}$ and $\mathbf{F}$ map $[a, b]$ into $R^{k}$, if $\mathbf{f} \in \mathscr{R}$ on $[a, b]$, and if $\mathbf{F}^{\prime}=\mathbf{f}$, then

$$
\int_{a}^{b} \mathbf{f}(t) d t=\mathbf{F}(b)-\mathbf{F}(a)
$$

The analogue of Theorem 6.13(b) offers some new features, however, at least in its proof.

6.25 Theorem If $\mathbf{f}$ maps $[a, b]$ into $R^{k}$ and if $\mathbf{f} \in \mathscr{R}(\alpha)$ for some monotonically increasing function $\alpha$ on $[a, b]$, then $|\mathbf{f}| \in \mathscr{R}(\alpha)$, and

$$
\left|\int_{a}^{b} \mathbf{f} d \alpha\right| \leq \int_{a}^{b}|\mathbf{f}| d \alpha
$$

Proof If $f_{1}, \ldots, f_{k}$ are the components of $\mathbf{f}$, then

$$
|\mathbf{f}|=\left(f_{1}^{2}+\cdots+f_{k}^{2}\right)^{1 / 2} .
$$

By Theorem 6.11, each of the functions $f_{i}^{2}$ belongs to $\mathscr{R}(\alpha)$; hence so does their sum. Since $x^{2}$ is a continuous function of $x$, Theorem 4.17 shows that the square-root function is continuous on $[0, M]$, for every real $M$. If we apply Theorem 6.11 once more, (41) shows that $|\mathbf{f}| \in \mathscr{R}(\alpha)$.

To prove (40), put $\mathbf{y}=\left(y_{1}, \ldots, y_{k}\right)$, where $y_{j}=\int f_{j} d \alpha$. Then we have $\mathbf{y}=\int \mathbf{f} d \alpha$, and

$$
|\mathbf{y}|^{2}=\sum y_{i}^{2}=\sum y_{j} \int f_{j} d \alpha=\int\left(\sum y_{J} f_{j}\right) d \alpha .
$$

By the Schwarz inequality,

$$
\sum y_{j} f_{j}(t) \leq|\mathbf{y}||\mathbf{f}(t)| \quad(a \leq t \leq b)
$$

hence Theorem 6.12(b) implies

$$
|\mathbf{y}|^{2} \leq|\mathbf{y}| \int|\mathbf{f}| d \alpha
$$

If $\mathbf{y}=\mathbf{0}$, (40) is trivial. If $\mathbf{y} \neq \mathbf{0}$, division of (43) by $|\mathbf{y}|$ gives (40).

\section{RECTIFIABLE CURVES}
We conclude this chapter with a topic of geometric interest which provides an application of some of the preceding theory. The case $k=2$ (i.e., the case of plane curves) is of considerable importance in the study of analytic functions of a complex variable.

6.26 Definition A continuous mapping $\gamma$ of an interval $[a, b]$ into $R^{k}$ is called a curve in $R^{k}$. To emphasize the parameter interval $[a, b]$, we may also say that $\gamma$ is a curve on $[a, b]$.

If $\gamma$ is one-to-one, $\gamma$ is called an arc.

If $\gamma(a)=\gamma(b), \gamma$ is said to be a closed curve.

It should be noted that we define a curve to be a mapping, not a point set. Of course, with each curve $\gamma$ in $R^{k}$ there is associated a subset of $R^{k}$, namely the range of $\gamma$, but different curves may have the same rarige.

We associate to each partition $P=\left\{x_{0}, \ldots, x_{n}\right\}$ of $[a, b]$ and to each curve $\gamma$ on $[a, b]$ the number

$$
\Lambda(P, \gamma)=\sum_{i=1}^{n}\left|\gamma\left(x_{i}\right)-\gamma\left(x_{i-1}\right)\right|
$$

The $i$ th term in this sum is the distance (in $R^{k}$ ) between the points $\gamma\left(x_{i-1}\right)$ and $\gamma\left(x_{i}\right)$. Hence $\Lambda(P, \gamma)$ is the length of a polygonal path with vertices at $\gamma\left(x_{0}\right)$, $\gamma\left(x_{1}\right), \ldots, \gamma\left(x_{n}\right)$, in this order. As our partition becomes finer and finer, this polygon approaches the range of $\gamma$ more and more closely. This makes it seem reasonable to define the length of $\gamma$ as

$$
\Lambda(\gamma)=\sup \Lambda(P, \gamma)
$$

where the supremum is taken over all partitions of $[a, b]$.

If $\Lambda(\gamma)<\infty$, we say that $\gamma$ is rectifiable.

In certain cases, $\Lambda(\gamma)$ is given by a Riemann integral. We shall prove this for continuously differentiable curves, i.e., for curves $\gamma$ whose derivative $\gamma^{\prime}$ is continuous.

6.27 Theorem If $\gamma^{\prime}$ is continuous on $[a, b]$, then $\gamma$ is rectifiable, and

$$
\Lambda(\gamma)=\int_{a}^{b}\left|\gamma^{\prime}(t)\right| d t
$$

Proof If $a \leq x_{i-1}<x_{i} \leq b$, then

Hence

$$
\left|\gamma\left(x_{i}\right)-\gamma\left(x_{i-1}\right)\right|=\left|\int_{x_{i-1}}^{x_{t}} \gamma^{\prime}(t) d t\right| \leq \int_{x_{i-1}}^{x_{t}}\left|\gamma^{\prime}(t)\right| d t
$$

$$
\Lambda(P, \gamma) \leq \int_{a}^{b}\left|\gamma^{\prime}(t)\right| d t
$$

for every partition $P$ of $[a, b]$. Consequently,

$$
\Lambda(\gamma) \leq \int_{a}^{b}\left|\gamma^{\prime}(t)\right| d t
$$

To prove the opposite inequality, let $\varepsilon>0$ be given. Since $\gamma^{\prime}$ is uniformly continuous on $[a, b]$, there exists $\delta>0$ such that

$$
\left|\gamma^{\prime}(s)-\gamma^{\prime}(t)\right|<\varepsilon \quad \text { if }|s-t|<\delta \text {. }
$$

Let $P=\left\{x_{0}, \ldots, x_{n}\right\}$ be a partition of $[a, b]$, with $\Delta x_{i}<\delta$ for all $i$. If $x_{i-1} \leq t \leq x_{i}$, it follows that

Hence

$$
\left|\gamma^{\prime}(t)\right| \leq\left|\gamma^{\prime}\left(x_{i}\right)\right|+\varepsilon \text {. }
$$

$$
\begin{aligned}
\int_{x_{i-1}}^{x_{i}}\left|\gamma^{\prime}(t)\right| d t & \leq\left|\gamma^{\prime}\left(x_{i}\right)\right| \Delta x_{i}+\varepsilon \Delta x_{i} \\
& =\left|\int_{x_{i-1}}^{x_{i}}\left[\gamma^{\prime}(t)+\gamma^{\prime}\left(x_{i}\right)-\gamma^{\prime}(t)\right] d t\right|+\varepsilon \Delta x_{i} \\
& \leq\left|\int_{x_{i-1}}^{x_{i}} \gamma^{\prime}(t) d t\right|+\left|\int_{x_{i-1}}^{x_{i}}\left[\gamma^{\prime}\left(x_{i}\right)-\gamma^{\prime}(t)\right] d t\right|+\varepsilon \Delta x_{i} \\
& \leq\left|\gamma\left(x_{i}\right)-\gamma\left(x_{i-1}\right)\right|+2 \varepsilon \Delta x_{i} .
\end{aligned}
$$

If we add these inequalities, we obtain

Since $\varepsilon$ was arbitrary,

$$
\begin{aligned}
\int_{a}^{b}\left|\gamma^{\prime}(t)\right| d t & \leq \Lambda(P, \gamma)+2 \varepsilon(b-a) \\
& \leq \Lambda(\gamma)+2 \varepsilon(b-a)
\end{aligned}
$$

$$
\int_{a}^{b}\left|\gamma^{\prime}(t)\right| d t \leq \Lambda(\gamma)
$$

This completes the proof.

\section{EXERCISES}
\begin{enumerate}
  \item Suppose $\alpha$ increases on $[a, b], a \leq x_{0} \leq b, \alpha$ is continuous at $x_{0}, f\left(x_{0}\right)=1$, and $f(x)=0$ if $x \neq x_{0}$. Prove that $f \in \mathscr{R}(\alpha)$ and that $\int f d \alpha=0$.

  \item Suppose $f \geq 0, f$ is continuous on $[a, b]$, and $\int_{a}^{b} f(x) d x=0$. Prove that $f(x)=0$ for all $x \in[a, b]$. (Compare this with Exercise 1.)

  \item Define three functions $\beta_{1}, \beta_{2}, \beta_{3}$ as follows: $\beta_{J}(x)=0$ if $x<0, \beta_{J}(x)=1$ if $x>0$ for $j=1,2,3$; and $\beta_{1}(0)=0, \beta_{2}(0)=1, \beta_{3}(0)=\frac{1}{2}$. Let $f$ be a bounded function on $[-1,1]$.

\end{enumerate}

(a) Prove that $f \in \mathscr{R}\left(\beta_{1}\right)$ if and only if $f(0+)=f(0)$ and that then

$$
\int f d \beta_{1}=f(0)
$$

(b) State and prove a similar result for $\beta_{2}$.

(c) Prove that $f \in \mathscr{R}\left(\beta_{3}\right)$ if and only if $f$ is continuous at 0 .

$(d)$ If $f$ is continuous at 0 prove that

$$
\int f d \beta_{1}=\int f d \beta_{2}=\int f d \beta_{3}=f(0)
$$

\begin{enumerate}
  \setcounter{enumi}{3}
  \item If $f(x)=0$ for all irrational $x, f(x)=1$ for all rational $x$, prove that $f \notin \mathscr{R}$ on $[a, b]$ for any $a<b$.

  \item Suppose $f$ is a bounded real function on $[a, b]$, and $f^{2} \in \mathscr{R}$ on $[a, b]$. Does it follow that $f \in \mathscr{R}$ ? Does the answer change if we assume that $f^{3} \in \mathscr{R}$ ?

  \item Let $P$ be the Cantor set constructed in Sec. 2.44. Let $f$ be a bounded real function on $[0,1]$ which is continuous at every point outside $P$. Prove that $f \in \mathscr{R}$ on $[0,1]$. Hint: $P$ can be covered by finitely many segments whose total length can be made as small as desired. Proceed as in Theorem 6.10.

  \item Suppose $f$ is a real function on $(0,1]$ and $f \in \mathscr{R}$ on $[c, 1]$ for every $c>0$. Define

\end{enumerate}

$$
\int_{0}^{1} f(x) d x=\lim _{c \rightarrow 0} \int_{c}^{1} f(x) d x
$$

if this limit exists (and is finite).

(a) If $f \in \mathscr{R}$ on $[0,1]$, show that this definition of the integral agrees with the old one.

(b) Construct a function $f$ such that the above limit exists, although it fails to exist with $|f|$ in place of $f$.

\begin{enumerate}
  \setcounter{enumi}{7}
  \item Suppose $f \in \mathscr{R}$ on $[a, b]$ for every $b>a$ where $a$ is fixed. Define
\end{enumerate}

$$
\int_{a}^{\infty} f(x) d x=\lim _{b \rightarrow \infty} \int_{a}^{b} f(x) d x
$$

if this limit exists (and is finite). In that case, we say that the integral on the left converges. If it also converges after $f$ has been replaced by $|f|$, it is said to converge absolutely.

Assume that $f(x) \geq 0$ and that $f$ decreases monotonically on $[1, \infty)$. Prove that

$$
\int_{1}^{\infty} f(x) d x
$$

converges if and only if

$$
\sum_{n=1}^{\infty} f(n)
$$

converges. (This is the so-called "integral test" for convergence of series.)

\begin{enumerate}
  \setcounter{enumi}{8}
  \item Show that integration by parts can sometimes be applied to the "improper" integrals defined in Exercises 7 and 8. (State appropriate hypotheses, formulate a theorem, and prove it.) For instance show that
\end{enumerate}

$$
\int_{0}^{\infty} \frac{\cos x}{1+x} d x=\int_{0}^{\infty} \frac{\sin x}{(1+x)^{2}} d x
$$

Show that one of these integrals converges absolutely, but that the other does not.

\begin{enumerate}
  \setcounter{enumi}{9}
  \item Let $p$ and $q$ be positive real numbers such that
\end{enumerate}

$$
\frac{1}{p}+\frac{1}{q}=1
$$

Prove the following statements.

(a) If $u \geq 0$ and $v \geq 0$, then

$$
u v \leq \frac{u^{p}}{p}+\frac{v^{q}}{q}
$$

Equality holds if and only if $u^{p}=v^{q}$.

(b) If $f \in \mathscr{R}(\alpha), g \in \mathscr{R}(\alpha), f \geq 0, g \geq 0$, and

$$
\int_{a}^{b} f^{p} d \alpha=1=\int_{a}^{b} g^{a} d \alpha
$$

then

$$
\int_{a}^{b} f g d \alpha \leq 1
$$

(c) If $f$ and $g$ are complex functions in $\mathscr{R}(\alpha)$, then

$$
\left|\int_{a}^{b} f g d \alpha\right| \leq\left\{\int_{a}^{b}|f|^{p} d \alpha\right\}^{1 / p}\left\{\int_{a}^{b}|g|^{a} d \alpha\right\}^{1 / a}
$$

This is Hölder's inequality. When $p=q=2$ it is usually called the Schwarz inequality. (Note that Theorem 1.35 is a very special case of this.)

(d) Show that Hölder's inequality is also true for the "improper" integrals described in Exercises 7 and 8.

\begin{enumerate}
  \setcounter{enumi}{10}
  \item Let $\alpha$ be a fixed increasing function on $[a, b]$. For $u \in \mathscr{R}(\alpha)$, define
\end{enumerate}

$$
\|u\|_{2}=\left\{\int_{a}^{b}|u|^{2} d \alpha\right\}^{1 / 2}
$$

Suppose $f, g, h \in \mathscr{R}(\alpha)$, and prove the triangle inequality

$$
\|f-h\|_{2} \leq\|f-g\|_{2}+\|g-h\|_{2}
$$

as a consequence of the Schwarz inequality, as in the proof of Theorem 1.37.

\begin{enumerate}
  \setcounter{enumi}{11}
  \item With the notations of Exercise 11, suppose $f \in \mathscr{R}(\alpha)$ and $\varepsilon>0$. Prove that there exists a continuous function $g$ on $[a, b]$ such that $\|f-g\|_{2}<\varepsilon$.
\end{enumerate}

Hint: Let $P=\left\{x_{0}, \ldots, x_{n}\right\}$ be a suitable partition of $[a, b]$, define

$$
g(t)=\frac{x_{i}-t}{\Delta x_{i}} f\left(x_{i-1}\right)+\frac{t-x_{i-1}}{\Delta x_{i}} f\left(x_{i}\right)
$$

if $x_{t-1} \leq t \leq x_{i}$.

\begin{enumerate}
  \setcounter{enumi}{12}
  \item Define
\end{enumerate}

$$
f(x)=\int_{x}^{x+1} \sin \left(t^{2}\right) d t
$$

(a) Prove that $|f(x)|<1 / x$ if $x>0$.

Hint: Put $t^{2}=u$ and integrate by parts, to show that $f(x)$ is equal to

$$
\frac{\cos \left(x^{2}\right)}{2 x}-\frac{\cos \left[(x+1)^{2}\right]}{2(x+1)}-\int_{x^{2}}^{(x+1)^{2}} \frac{\cos u}{4 u^{3 / 2}} d u
$$

Replace $\cos u$ by -1 .

(b) Prove that

$$
2 x f(x)=\cos \left(x^{2}\right)-\cos \left[(x+1)^{2}\right]+r(x)
$$

where $|r(x)|<c / x$ and $c$ is a constant.

(c) Find the upper and lower limits of $x f(x)$, as $x \rightarrow \infty$.

(d) Does $\int_{0}^{\infty} \sin \left(t^{2}\right) d t$ converge?

\begin{enumerate}
  \setcounter{enumi}{13}
  \item Deal similarly with
\end{enumerate}

$$
f(x)=\int_{x}^{x+1} \sin \left(e^{t}\right) d t
$$

Show that

$$
e^{x}|f(x)|<2
$$

and that

$$
e^{x} f(x)=\cos \left(e^{x}\right)-e^{-1} \cos \left(e^{x+1}\right)+r(x)
$$

where $|r(x)|<C e^{-x}$, for some constant $C$.

\begin{enumerate}
  \setcounter{enumi}{14}
  \item Suppose $f$ is a real, continuously differentiable function on $[a, b], f(a)=f(b)=0$, and
\end{enumerate}

$$
\int_{a}^{b} f^{2}(x) d x=1
$$

Prove that

$$
\int_{a}^{b} x f(x) f^{\prime}(x) d x=-\frac{1}{2}
$$

and that

$$
\int_{a}^{b}\left[f^{\prime}(x)\right]^{2} d x \cdot \int_{a}^{b} x^{2} f^{2}(x) d x>1
$$

\begin{enumerate}
  \setcounter{enumi}{15}
  \item For $1<s<\infty$, define
\end{enumerate}

$$
\zeta(s)=\sum_{n=1}^{\infty} \frac{1}{n^{s}}
$$

(This is Riemann's zeta function, of great importance in the study of the distribution of prime numbers.) Prove that

(a) $\zeta(s)=s \int_{1}^{\infty} \frac{[x]}{x^{s+1}} d x$

and that

(b) $\zeta(s)=\frac{s}{s-1}-s \int_{1}^{\infty} \frac{x-[x]}{x^{s+1}} d x$,

where $[x]$ denotes the greatest integer $\leq x$.

Prove that the integral in $(b)$ converges for all $s>0$.

Hint: To prove (a), compute the difference between the integral over $[1, N]$ and the $N$ th partial sum of the series that defines $\zeta(s)$.

\begin{enumerate}
  \setcounter{enumi}{16}
  \item Suppose $\alpha$ increases monotonically on $[a, b], g$ is continuous, and $g(x)=G^{\prime}(x)$ for $a \leq x \leq b$. Prove that
\end{enumerate}

$$
\int_{a}^{b} \alpha(x) g(x) d x=G(b) \alpha(b)-G(a) \alpha(a)-\int_{a}^{b} G d \alpha .
$$

Hint: Take $g$ real, without loss of generality. Given $P=\left\{x_{0}, x_{1}, \ldots, x_{n}\right\}$, choose $t_{i} \in\left(x_{t-1}, x_{i}\right)$ so that $g\left(t_{i}\right) \Delta x_{i}=G\left(x_{i}\right)-G\left(x_{i-1}\right)$. Show that

$$
\sum_{i=1}^{n} \alpha\left(x_{i}\right) g\left(t_{i}\right) \Delta x_{i}=G(b) \alpha(b)-G(a) \alpha(a)-\sum_{i=1}^{n} G\left(x_{i-1}\right) \Delta \alpha_{i} .
$$

\begin{enumerate}
  \setcounter{enumi}{17}
  \item Let $\gamma_{1}, \gamma_{2}, \gamma_{3}$ be curves in the complex plane, defined on $[0,2 \pi]$ by
\end{enumerate}

$$
\gamma_{1}(t)=e^{i t}, \quad \gamma_{2}(t)=e^{2 t t}, \quad \gamma_{3}(t)=e^{2 \pi l t \sin (1 / t)}
$$

Show that these three curves have the same range, that $\gamma_{1}$ and $\gamma_{2}$ are rectifiable, that the length of $\gamma_{1}$ is $2 \pi$, that the length of $\gamma_{2}$ is $4 \pi$, and that $\gamma_{3}$ is not rectifiable.

\begin{enumerate}
  \setcounter{enumi}{18}
  \item Let $\gamma_{1}$ be a curve in $R^{k}$, defined on $[a, b]$; let $\phi$ be a continuous 1-1 mapping of $[c, d]$ onto $[a, b]$, such that $\phi(c)=a$; and define $\gamma_{2}(s)=\gamma_{1}(\phi(s))$. Prove that $\gamma_{2}$ is an arc, a closed curve, or a rectifiable curve if and only if the same is true of $\gamma_{1}$. Prove that $\gamma_{2}$ and $\gamma_{1}$ have the same length.
\end{enumerate}

\end{document}