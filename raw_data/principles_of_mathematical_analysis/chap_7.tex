7 SEQUENCES AND SERIES OF FUNCTIONS In the present chapter we confine our attention to complex-valued functions (including the real-valued ones, of course), although many of the theorems and proofs which follow extend without difficulty to vector-valued functions, and even to mappings into general metric spaces. We choose to stay within this simple framework in order to focus attention on the most important aspects of the problems that arise when limit processes are interchanged. DISCUSSION OF MAIN PROBLEM 7.1 Definition Suppose {f,),n = 1,2,3,.…,is a sequence of functions defined on a set E, and suppose that the sequence of numbers {fA(x)} converges for every xe E. We can then define a function f by (1) $$ f(x)=\operatorname*{lim}_{n\to\infty}f_{n}(x)\qquad(x\in E). $$144 PRuINcCTPLES OF MATHEMLATTCAL ANALYSis Under these circumstances we say that {JA} converges on ${\widetilde{\mathcal{H}}}_{i}$ and that fis the limit, or the limit function, of{fA)、Sometimes we shall use a more descriptive terminology and shall say that“{ f,} converges tof pointwise on $E^{\prime\prime}$ if () holds Similarly,if Ef,(x) converges for every xe ${\bar{\cal E}},$ and if we define (2) $$ f(x)=\sum_{n=1}^{\infty}f_{n}(x)\qquad(x\in E), $$ the function f is called the sum of the series Ef, The main problem which arises is to determine whether important properties of functions are preserved under the limit operations (1) and (2). For instance, if the functions f, are continuous,or diffrentiabl,or integrable, is the same true of the limit function ?What are the relations between f' and f' say, or between the integrals of f and that of f? To say that fis continuous at a limit point x means $$ \operatorname*{lim}_{t arrow\infty}f(t)=f(x). $$ Hence, to ask whether the limit of a sequence of continuous functions is con- tinuous is the same as to ask whether (3) $$ \operatorname*{lim}_{t arrow x}\;\operatorname*{lim}_{n arrow\infty}f_{n}(t)=\operatorname*{lim}_{n arrow\infty}\;\operatorname*{lim}_{t arrow x}f_{n}(t), $$ i.e., whether the order in which limit processes are carried out is immaterial On the left side of(3), we first let n→OO,then t→X;on the right side,1→、 first, then n→00. We shall now show by means of several examples that limit processes cannot in general be interchanged without affecting the result. Afterward,we shall prove that under certain conditions the order in which limit operations are carried out is immaterial Our first example, and the simplest one, concerns a“double sequence.” 7.2Example For m = 1,2,3,...,n = 1,2,3,..., let $$ s_{m,n}={\frac{m}{m+n}}. $$ Then, for every fixed n, m→α lim s. 1 so that (4) lim limsm.= 1SEQUENCES AND SERIES Or FUNCrIONs 145 On the other hand, for every fixed m $$ \operatorname*{lim}_{n\to\infty}s_{m,n}=0, $$ so that (5) $$ \operatorname*{lim}_{m\to\infty}\operatorname*{lim}_{n\to\infty}s_{m,n}=0. $$ 7.3 Example Let $$ f_{n}(x)={\frac{x^{2}}{(1+x^{2})^{n}}}\qquad(x\operatorname{real};n=0,1,2,\dots), $$ and consider (6) $$ f(x)=\sum_{n=0}^{\infty}f_{n}(x)=\sum_{n=0}^{\infty}{\frac{x^{2}}{(1+x^{2})^{n}}}. $$ Since f,(0) = 0, we have f(0) = 0. For x ≠ 0, the last series in (6) is a convergent geometric series with sum ${\bf i}\,+\,{\bf x}^{2}$ (Theorem 3.26). Hence (7) $$ f(x)={\binom{0}{1+x^{2}}}\quad\quad(x=0), $$ so that a convergent series of continuous functions may have a discontinuous sum. 7.4Example For m = 1,2,3,.…,put $$ f_{m}(x)=\operatorname*{lim}_{n\to\infty}(\cos\,m\,\d t\,\pi x)^{2n}. $$ When m!xis an integer,fx)= 1. For all oher values of x, fmCx) = 0.Now let $$ f(x)=\operatorname*{lim}_{m\to\infty}f_{m}(x). $$ For irrational x,fA(x) = 0 for every m; hence f(x) = 0. For rational x,say x = p/q, where p and $\underline{{{\mathcal{O}}}}$ are integers, we see that m!x is an integer if m ≥ ${\mathcal{L}}_{\mathrm{s}}$ so that f(x) = 1. Hence (8) $$ \operatorname*{lim}_{m arrow\infty\;n arrow\infty}(\cos m!\pi x)^{2n}={\binom{0}{1}}\;\;\;\;\;\;\;{\binom{x\;{\mathrm{irrational}})}{(x\;\mathrm{rational}).}} $$ We have thus obtained an everywhere discontinuous limit function, which is not Riemann-integrable (Exercise , Chap. 6)146 PRINCIPLES OF MATHEMATTICAL ANALYSIS 7.5 Example Let (9) f.(x)= sin nx (x real, n = 1,2,3,.….) and √n f(x) = lim fA(x) = 0 Then f”(x) = 0, and $$ f_{n}^{\prime}(x)={\sqrt{n}}\cos n x, $$ so that {f'} does not converge to f’ For instance, $$ f_{n}^{\prime}(0)=\sqrt{n}\to+\infty $$ as n→ 00,whereas f'(0) = 0 7.6 Example Let (10) $$ f_{n}(x)=n^{2}x(1-x^{2})^{n}\qquad(0\leq x\leq1,\,n=1,\,2, $$ , 3,.…. For 0<x≤l, we have $$ \operatorname*{lim}_{n\to\infty}f_{n}(x)=0, $$ by Theorem 3.20(d).Since f,(0) = 0, we see that (11) $$ \operatorname*{lim}_{n\to\infty}f_{n}(x)=0\qquad(0\leq x\leq1). $$ A simple calculation shows that $$ \textstyle\bigcap_{0}^{1}x(1-x^{2})^{n}\,d x={\frac{1}{2n+2}}. $$ Thus,in spite of (11) $$ \bigcap_{o}^{1}f_{n}(x)\,d x={\frac{n^{2}}{2n+2}}\to+\infty $$ as n→0 If, in (10), we replace $\gamma_{A}^{2}$ by n,(1I) still holds, but we now have $$ \operatorname*{lim}_{n\to\infty}\,\int_{0}^{1}f_{n}(x)\,d x=\operatorname*{lim}_{n\to\infty}{\frac{n}{2n+2}}={\frac{1}{2}}. $$ whereas $$ \textstyle\int_{0}^{1}\left[\operatorname*{lim}_{n\to\infty}f_{n}(x)\right]\,d x=0. $$SEQUENCES AND SERIES OF FUNCrIONs 147 Thus the limit of the integral need not be equal to the integral of the limit, even if both are finite After these examples, which show what can go wrong if limit processes are interchanged carelessly, we now define a new mode of convergence, stronger than pointwise convergence as defined in Definition 7.1, which will enable us to arrive at positive results. UNIFORM CONVERGENCE 7.7 Definition We say that a sequence of functions {f)},n = 1, 2,3,.… converges uniformly on E to a function f if for every e> O there is an integer $\mathcal{N}$ such that $n\geq N$ implies (12) lfA(x)一f(x)」≤8 for allx ∈ E It is clear that every uniformly convergent sequence is pointwise con- vergent. Quite explicitly, the difference between the two concepts is this: If {f元 converges pointwise on E,then there exists a function fsuch that, for every 8> 0, and for every xe E, there is an integer N, depending on e and on x, such that (12) holds if n ≥ N; if{fA converges uniformly on E, it is possible, for each c > 0, to find one integer N which will do for all xe E. We say that the series Ef,(x) converges uniformly on E if the sequence {s,} of partial sums defined by $$ \sum_{i=1}^{n}\ f_{i}(x)=s_{n}(x) $$ converges uniformly on ${\boldsymbol{E}}.$ The Cauchy criterion for uniform convergence is as follows 7.8 Theorem The sequence of functions {f,), defined on $\textstyle E_{\!,\!\qquad}$ converges uniformly o0n ${\widetilde{\mathcal{H}}}_{?}$ if and only if for every é>0 there exists an integer N such that m ≥ N, n≥ N,xe E implies (13) $$ |f_{n}(x)-f_{m}(x)\,|\leq\kappa. $$ Proof Suppose{f} converges uniformly on E,and let f be the limit function. Then there is an integer $\mathcal{N}$ such that n ≥ N,xe E implies $$ |f_{n}(x)-f(x)|\leq{\frac{\varepsilon}{2}}, $$ so that if.(x)-/m(x)|≤ /A(x)一f(x)」十|f(x)-f(X)」≤e if n≥ N,m ≥N,xe E.148 PRINCIPLES OF MATHEMATTCAL ANALYSIS Conversely, suppose the Cauchy condition holds. By Theorem 3.11 the sequence{f,(x)} converges, for every x,to a limit which we may cal f(x). Thus the sequence{fA} converges on E,to f. We have to prove that the convergence is uniform Let e>0 be given, and choose $\mathcal{N}$ such that (13) holds. Fix n, and let m→ oo in (13)、 Since fx)→f(x) as m→ O,this gives (14) lf,(x)-f(x)|≤& for every n ≥ $\mathcal{N}$ and every xe E, which completes the proof The following criterion is sometimes useful. 7.9 Theorem Suppose n→c lim /A(x) =/0x) (x e E) Put M。= sup IfA(x)一J(x)| Then f。→f uniformly on E if and only if M。→ 0 as n→ OO. Since this is an immediate consequence of Definition 7.7, we omit th details of the proof. For series, there is a very convenient test for uniform convergence, due to Weierstrass. 7.10Theorem Suppose {fA} is a sequence of functions defined on E, and suppose $$ |f_{n}(x)\,|\leq M_{n}\ \;\;\;\;\;\;(x\in E,n=1,2,3,\dots). $$ Then Ef, converges uniformly on E i EM, converges Note that the converse is not asserted (and is, in fact, not true) Proof If EM, converges,then, for arbitrary e > 0, $$ \left|\sum_{i=n}^{m}f_{i}(x)\right|\leq_{i=n}^{m}M_{i}\leq s\qquad(x\in E), $$ provided m and n are large enough. Uniform convergence now follows from Theorem 7.8.sEQUENCES AND SERIES OP FUNcrroNs 149 UNIFORM CONVERGENCE AND CONTINUITY 7.11 Theorem Suppose f,→f uniformly on a set ${\widehat{H}}^{1}$ in a metric space. Let x be a limit point of E, and suppose that (15) $$ \operatorname*{lim}_{t arrow x}f_{n}(t)=A_{n}\;\;\;\;\;\;\;(n=1,2,\,3,\,.\,.). $$ Then {A} converges, and (16) $$ \operatorname*{lim}_{t arrow x}f(t)=\operatorname*{lim}_{n arrow\infty}A_{n}. $$ In other words, the conclusion is that (17) $$ \operatorname*{lim}_{t arrow x}\;\operatorname*{lim}_{n arrow\infty}f_{n}(t)=\operatorname*{lim}_{n arrow\infty}\;\operatorname*{lim}_{t arrow x}f_{n}(t). $$ exists $\mathcal{N}$ such that n ≥ N,m ≥ Proof Let s> 0 be given. By the uniform convergence of {JA), there imply $N,\,t\in E$ (18) $$ |f_{n}(t)-f_{m}(t)\,|\leq\varepsilon. $$ Letting 1→x in(18),we obtain $$ |A_{n}-A_{m}|\leq t $$ for n ≥ N, m ≥ N,so that {A}is a Cauchy sequence and therefore converges, say to A Next, (19) $$ J(t)-A\vdash\sqcup\textstyle\in{\cal J}_{n}(t)\uparrow+\ |f_{n}(t)-A_{n}|\downarrow+|A_{n}-A|. $$ We first choose n such that (20) $$ \left|f(t)-f_{n}(t)\right|\leq{\frac{\varepsilon}{3}} $$ for allt ${\widetilde{F}}^{1}$ (this is possible by the uniform convergence), and such that (21) $$ \left|A_{n}-A\right|\leq_{\overline{{3}}}^{E}. $$ Then, for this ${\mathcal{I}}{\bar{l}}_{3}$ we choose a neighborhood V of x such that (22 V40-~.1号 if teVn E,1≠x Substituting the inequalities (20) to (22) into (19), we see tha $$ |f(t)-A|\leq $$ e, provided te Vn E, t≠x.This is equivalent to(16)150 pRINCTPLEs OF MATHEMATICAL ANALYSIs 7.12 Theorem If{f} is a sequence of continuous functions on $\textstyle E_{\mathrm{{J}}}$ and iff。→ uniformly on $\textstyle E,$ then f is continuous on ${\widehat{\mathcal{H}}}^{\times}$ This very important result is an immediate corollary of Theorem 7.11 The converse is not true; that is, a sequence of continuous functions may converge to a continuous function,although the convergence is not uniform Example 7.6 is of this kind (to see this, apply Theorem 7.9).But there is a case in which we can assert the converse. 7.13 Theorem Suppose K ${\ddot{\vec{k}}}_{b}\vec{\jmath}$ compact, and (a){f}is a sequence of continuous functions on K, (のJA converges pointwise to a continuous function f on K ic)J,(x)≥/.+100 for all xe K,n= 1,2,3,. Then f,→f uniformly on K (K,is empty. Hence Proof Put g。= /。-f. Then g。is continuous, $K_{n}$ be the set of all xe K with gA(x) ≥8 pointwise,and 9。≥9.+1. We have to prove that g $\scriptstyle g_{n}\, arrow0$ $K.$ ${\mathcal{G}}_{n}$ 。→O uniformly on Let e> 0 be given. Let we see that x生 $K_{n}$ $K_{N}$ Since g,is continuous, K,is closed (Theorem 4.8), hence compact (Theorem (Theorem 2.36). It follows 2.35). Since g。.≥9.+1. we have K, Kn+:Fix xeK Since g.(x)→0, $\mathcal{N}$ is empty for some if nisuficiently large. Thusx生[K。In other words, that O ≤9,(x)<efor all $x\in K$ and for all n ≥ N. This proves the theorem. Let us note that compactness is really needed here、For instance, if $$ f_{n}(x)=\frac{1}{n x+1}\;\;\;\;\;\;\;(0<x<1;n=1,\,2,\,3,\,\ldots $$ . then J.(x)→0 monotonically in (0,1), but the convergence is not uniform 7.14 Definition If Xis a metric space, G(X will denote the set of all complex valued, continuous,bounded functions with domain X INote that boundedness is redundant if X is compact(Theorem 4.15) Thus G(X) consists of all complex continuous functions on X if X is compact.] We associate with each f∈G(X) its supremum norm $$ \|f\|=\operatorname*{sup}_{x\neq x}|f(x)\,|. $$ Since $\mathbb{P}$ is assumed to be bounded,l」fl|< o.It is obvious that f/| = only i Jf(x) =0 for every xe X, that is, only if f = 0. If h =/+9,then lA(Cx)」≤ If(x)| + lg(x)| ≤ lfl + Hl for all x e X; hence l/+ l ≤ Ifl 1glsEQUENCES AND SERIES Or FUNCrIoNs 151 If we define the distance between feG(X) and g eG(X) to be |If-gl it follows that Axioms 2.15 for a metric are satisfied We have thus made G(X) into a metric space. Theorem 7.9 can be rephrased as follows: A sequence{f,} converges to fwith respect to the metric of G(X) if and only iff。→f uniformly on X. Accordingly,closed subsets of G(X) are sometimes called uniformly closed, the closure of a set .sl c 6(X) s called its unform closure, and so on 7.15 Theorem The above metric makes G(X)into a complete metric space. Proof Let{fA} be a Cauchy sequence in G(X). This means that to each 8 > 0 corresponds an $\mathcal{N}$ such that lf。一-fml<e if n ≥1 $\mathcal{N}$ and m ≥ N t follows(by Theorem 7.8) that there is a function f with domain X to Moreover, which {f} converges uniformly.By Theorem 7.12,fis continuous $\mathsf{\mathcal{J}}$ is bounded, since there is an n such that f(x) - JA(x)|<1 for all xe X, and f。 is bounded. Thus fe 6(X), and since f.→f uniformly on X,we have $\|f-f_{n}\|\to0$ as n 一> 0O, UNIFORM CONVERGENCE AND INTEGRATION 7.16 Theorem Let α be monotonically ncreasing on [a,b]. Swppose f,e OR(α on [a, b], for n=1,2,3,.…,and suppose f。→f uniformly on [a,bj].Then fe 8*(α on [a,b], and (23) $$ \textstyle{\int_{a}^{b}f\,d x=\operatorname*{lim}_{n\to\infty}\int_{a}^{b}f_{n}\,d\alpha.} $$ (The existence of the limit is part of the conclusion.) Proof It suffices to prove this for real f.. Put (24) $$ s_{n}=\operatorname*{sup}\;|f_{n}(x)-f(x)\;|, $$ the supremum being taken over a≤x≤b.Then $$ f_{n}-\varepsilon_{n}\leq f\leq f_{n}+\varepsilon_{n}, $$ so that the upper and lower integrals of f(see Definition 6.2) satisy (25) Hence $$ \textstyle\int_{a}^{b}(f_{n}-s_{n})\,d x\leq\underbrace{\int f}_{-}f\,d x\leq\~\stackrel{\textstyle D}{\int}_{d}d x\leq\int_{a}^{b}(f_{n}+s_{n})\,d x. $$ $$ 0\leq\lceil f d x-\rceil f d x\leq2\varepsilon_{n}\lfloor x(b)-\alpha(a)\rfloor. $$152PRINCIPLES OF MATHEMATICAL ANALYSIs Since $s_{n}\to0$ 0 as n→O(Theorem 7.9),the upper and lower integrals of f are equal Thus f ∈ 9X(α). Another application of(25) now yields (26) $$ \left|\int_{a}^{b}f\,d x-\int_{a}^{b}f_{n}\,d x\right|\leq s_{n}[\alpha(b)-\alpha(a)]. $$ This implies (23). Corollary If f,∈ 82(c) on [a,b] and i $$ f(x)=\sum_{n=1}^{\infty}f_{n}(x)\qquad(a\leq x\leq b), $$ the series converging uniformly on [a,b], ther $$ \textstyle{\int_{a}^{b}f\,d x=\sum_{n=1}^{\infty}\int_{a}^{b}f_{n}\,d x.} $$ In other words, the series may be integrated term by term UNIFORM CONVERGENCE AND DIFFERENTIA TION We have already seen,in Example 7.5, that uniform convergence of{f,} implies nothing about the sequence {f;}. Thus stronger hypotheses are required for the assertion that f'→f'if f。→了. 7.17 Theorem Suppose {fA} is a sequence of fumctions, differentiable on [a,b o0n [a,b].If{f'}converges and and such that {/,(xo)} converges for some point ${\mathcal{X}}_{0}$ ${\widehat{\operatorname{f}_{3}}}$ uniformly on [a, b], hn f/A comerges umiformly on a,bJ, 1o a fumctio (27) $$ f^{\prime}(x)=\operatorname*{lim}_{n\to\infty}f_{n}^{\prime}(x)\qquad(a\leq x\leq b). $$ Proof Let &>0 be given. Choose $\mathcal{N}$ such that $n\geq N,$ , m≥ N,implies (28) J-0)1- and (29) l f/(t)一fMt)|< 8 (a ≤1≤b) 2(6 -a)sroUENCEs AND SERIs or rUNcroNs 153 If we apply the mean value theorem 5.19 to the function f,一fm,(29 shows that (30) 1/ $$ \begin{array}{c l c r}{{f_{n}(x)-f_{m}(x)-f_{n}(t)+f_{m}(t)\mid\leq\frac{\left|x-t\mid\!\!\varepsilon}{2\! |0-a\right.\leq\!\!\frac{\delta}{2}}}}&{{\leq}}\end{array} $$ for any x and t on [a,b], ifn≥ N,m ≥ N. The inequality lf.Ax) 一fmX)|≤ IfAx)-/Xx) 一/,(xo) +/(×o)| + 「/Axo)一/(×xo) implies, by (28) and(30), that $$ |f_{n}(x)-f_{m}(x)\,|<\varepsilon\qquad(a\leq x\leq b,\,n\geq N,\,m\geq\Lambda $$ V), so that{f,} converges uniformly on [a,b]. Let $$ f(x)=\operatorname*{lim}_{n\to\infty}f_{n}(x)\qquad(a\leq x\leq b). $$ Let us now fx a point x on La, b] and define (31) $$ \phi_{n}(t)=\frac{f_{n}(t)-f_{n}(x)}{t-x}\,,\qquad\phi(t)=\frac{f(t)-f(x)}{t-x} $$ for a ≤t ≤b,1≠x. Then (32) $$ \operatorname*{lim}_{t arrow x}\phi_{n}(t)=f_{n}^{\prime}(x)\qquad(n=1,2,\,3,\,\cdot\cdot). $$ The first inequality in (30) shows that $$ |\phi_{n}(t)-\phi_{m}(t)\,|\leq{\frac{s}{2(b-a)}}\qquad(n\geq N,m\geq N), $$ so that {p,} converges uniformly, for t≠ x.Since {fm} converges to f,w conclude from(31) that (33) $$ \operatorname*{lim}_{n\to\infty}\phi_{n}(t)=\phi(t) $$ uniformly for a ≤t≤b,t ≠ x. If We now apply Theorem 7.11 to {中,),(32) and (33) show that $$ \operatorname*{lim}_{t arrow x}\phi(t)=\operatorname*{lim}_{n arrow\infty}f_{n}^{\prime}(x); $$ and this is (27), by the definition of $\phi(t)$ Remark:If the continuity of the functions $f_{n}^{\prime}$ is assumed in addition to the above hypotheses,then a much shorter proof of(27) can be based or Theorem 7.16 and the fundamental theorem of calculus154 pnnNcTPLEs Oor MATHiEMATICAL ANALYsis 7.18 Theorem There exists a real continuous function on the real ine which is nowhere differentiable Proof Define (34) $$ \varphi(x)=\;|x\ |\;\;\;\;\;\;\;(-1\leq x\leq1) $$ and extend the definition of p(x) to all real x by requiring that (35) $$ \varphi(x+2)=\varphi(x). $$ Then, for all s and t (36) $$ |\varphi(s)-\varphi(t)|\leq|s-t|. $$ In particular,p is continuous on $R^{1}.$ . Define (37) $$ f(x)=\sum_{n=0}^{\infty}(\lambda)^{n}\varphi(4^{n}x). $$ Since O ≤9 ≤1, Theorem 7.10 shows that the series (37) converges uniformly on $\textstyle{\mathcal{R}}^{\prime}$ ,By Theorem 7.12,f is continuous on Rl Now fix a real number ${\mathcal{N}}$ and a positive integer m.Put (38) $$ \delta_{m}=\pm\textstyle{\frac{1}{2}}\cdot4^{-m} $$ where the sign is so chosen that no integer lies between 4"x and 4"(x + 6) This can be done,since $4^{m}\left|\delta_{m}\right|=\textstyle{\frac{1}{2}}$ Define (39) $$ \gamma_{n}=\frac{\varphi(4^{n}(x+\delta_{m}))-\varphi(4^{n}x)}{\delta_{m}}. $$ When n > m, then 4 $4^{n}\delta_{m}$ is an even integer, so that $\gamma_{n}=0.$ When $0\leq n\leq m,$ (36) implies that lyml ≤ 4". Since $|\gamma_{m}|=4^{m},$ we conclude that $$ \begin{array}{l}{{\left|\mathcal{f}(x+\delta_{m})-f(x)\right|}}\\ {{\left.\bar{\partial}_{m}\right|}}\\ {{=\frac{1}{2}(3^{m}+1).}}\end{array} $$ As m→00,6m→0. It follows that fis not differentiable at x EOUICONTINUOUS FAMILIES OF FUNCTIONS In Theorem 3.6 we saw that every bounded sequence of complex numbers contains a convergent subsequence, and the question arises whether something similar is true for sequences of functions. To make the question more precise, we shall define two kinds of boundedness.SEQUENCES AND SERIES OF FUNCTIONS 155 7.19 Defnition Let {fA} be a sequence of functions defined on a set E We say that {J,} is pointwise bounded on Eif the sequencet JAXx)} is bounded for every xe E, that is,if there exists a finite-valued function p defined on ${\widehat{\mathbb{H}^{\nu}}}$ such that $$ |f_{n}(x)|<\phi(x)\qquad(x\in E,n=1,2,3,\ldots). $$ We say that f)} is uniformly bounded on ${\widehat{\operatorname{fr}}}$ if there exsts a number M such that $$ |f_{n}(x)\,|<M $$ Now if {f) is pointwise bounded on ${\widehat{\operatorname{P}^{\nu}}}$ and $\textstyle{E_{1}}$ is a countable subset of $\textstyle E_{\mathrm{{J}}}$ it is always possible to find a subsequence $\{f_{n_{k}}\}$ such that {fm,(x)} converges for every xe E. This can be done by the diagonal process which is used in the proof of Theorem 7.23. However, even if {f} is a uniformly bounded sequence of continuous functions on a compact set E, there need not exist a subseq;ience which con- verges pointwise on E. In the following example,this would be quite trouble some to prove with the equipment which we have at hand so far, but the proof is quite simple if we appeal to a theorem from Chap. 11. 7.20Example Let $$ f_{n}(x)=\sin n x\qquad(0\leq x\leq2\pi,\,n=1,\,2,\,3,\,\ldots). $$ Supose there exists a sequence {n} such that {sin n,x} converges, for ever x∈ [0,2元]. In that case we must have $$ \operatorname*{lim}_{k\to\infty}(\sin n_{k}x-\sin n_{k+1}x)=0\qquad(0\le x\le2\pi); $$ hence (40) $$ \operatorname*{lim}_{s\to\infty}(\sin n_{k}x-\sin n_{k+1}x)^{2}=0\qquad(0\leq x\leq2\pi). $$ By Lebesgue's theorem concerning integration of boundedly convergen sequences (Theorem 11.32),(40) implies (41) $$ \operatorname*{lim}_{k arrow\infty}\int_{0}^{2\pi}(\sin n_{k}x-\sin n_{k+1}x)^{2}\,d x=0. $$ But a simple calculation shows that $$ \int_{0}^{2\pi}(\sin n_{k}x-\sin n_{k+1}x)^{2}\,d x=2\pi, $$ which contradicts (41)156 PRINCIPLEs Or MATHEMATICAL ANALYs Another question is whether every convergent sequence contains a uniformly convergent subsequence. Our next example will show that this need not be so,even if the sequence is uniformly bounded on a compact set (Example 7.6 shows that a sequence of bounded functions may converge without being uniformly bounded;but it is trivial to see that uniform conver gence of a sequence of bounded functions implies uniform boundedness.) 7.21 Example Let $$ f_{n}(x)=\frac{x^{2}}{x^{2}+(1-n x)^{2}}\qquad(0\leq x\leq1,\,n=1,\,2,\,3,\,\cdot\cdot). $$ Then If,(x) ≤1,so that{fA} is uniformly bounded on [0,1]. Also $$ \operatorname*{lim}_{n\to\infty}f_{n}(x)=0\qquad(0\leq x\leq1), $$ but $$ f_{n}\biggl(\frac{1}{n}\biggr)=1\qquad(n=1,2,3,\dots), $$ so that no subsequence can converge uniformly on [0, 1] The concept which is needed in this connection is that of equicontinuity; it is given in the following definition. 7.22Definition A family F of complex functions f defined on a set ${\mathcal{F}}^{\downarrow}$ in a metric space $X$ is said to be equicontinuous on $\stackrel{\mathcal{P}}{\mathcal{P}}$ if for everyB> O there exists a $\scriptstyle\delta>0$ such that $$ |f(x)-f(y)|<s $$ whenever dx, y)<6,x∈ E,y e E, and fe 多.Here $\mathcal{Q}$ denotes the metric of X It is clear that every member of an equicontinuous family is uniformly continuous. The sequence of Example 7.21 is not equicontinuous. Theorems 7.24 and 7.25 will show that there is a very close relation between equicontinuity,on the one hand, and uniform convergence of sequences of continuous functions, on the other. But first we describe a selection process which has nothing to do with continuity. a countable set 7.23 Theorem If{f,} is a pointwise bounded sequence of complex functions on $\textstyle E,$ then {J,} as a subsequence f. such that {f,Xx)} converges for every x∈ ${\widetilde{\mathcal{H}}}_{t}$SEQUENCES AND SERIEs OF FUNCrONs 157 Proof Let {x;},i= 1,2,3,...,be the points of E, arranged in a sequence Since{f,(x))}is bounded, there exists a subsequence,which we shal by the array denote by {fA.,}, such that{f,kxi)} converges as k→O S,S,……, which we represent Let us now consider sequences $S_{1},$ $$ \begin{array}{c c c c c c c c c c c c c}{{S_{1}:}}&{{f_{1,1}}}&{{f_{1,2}}}&{{f_{1,3}}}&{{f_{1,4}}}&{{\cdots}}&{{\cdots}}\\ {{S_{2}:}}&{{f_{2,1}}}&{{f_{2,2}}}&{{f_{2,3}}}&{{f_{2,4}}}&{{\cdots}}&{{\cdots}}&{{\cdots}}&{{\cdots}}&{{\cdots}}&{{\cdots}}&{{\cdots}}&{{\cdots}}&{{\cdots}}&{{\cdots}}&{{\cdots}}&{{\cdots}}&{{\cdots}}&{{\cdots}}&{{\cdots}}&{{\cdots}}&{{\cdots}}&{{\cdots}}&{{\cdots}}&{{\cdots}}&{{\cdots}}&{{\cdots}}&{{\cdots}}&{{\cdots}}&{{\cdots}}&{{\cdots}}&{{\cdots}}&{{\cdots}}&{{\cdots}}&{~.}}&{{\cdots}}&{{{\cdots}}&{~.}}&{{\end{array} $$ and which have the following properties: (a) $\mathrm{S}_{n}$ is a subsequence of ${\mathfrak{S}}_{n-1},$ for n = 2,3,4,. makes it possble to choose (b){fm,(x,)} converges,as 人→OO(the boundedness of{J,(x,) $\mathrm{{S}}_{n}$ in this way); (c) The order in which the functions appear is the same in each se quence;ie.,if one function precedes another in $S_{1},$ they are in the same relation in every $S_{n}\,,$ until one or the other is deleted. Hence, when going from one row in the above array to the next below, functions may move to the left but never to the right. We now go down the diagonal of the array;ie., we consider the sequence S:/,,/z,2 /3,3J.,… By(c),the sequence S(except possibly its first n - 1 terms)is a sub sequence of $S_{n},$ , for n=1,2,3,.… Hence ()implies that f..x0 converges, as n→OO,for every x;∈ E. 7.24 Theorem If K is a compact metric space, if f,∈ G(K) for n = 1,2,3, … and if{f} converges uniformly on K, then {f} is equicontinuous on K. Proof Letc> 0 be given. Since{ $\{f_{n}\}$ converges uniformly,there is an integer $\mathcal{N}$ I such that (42) l1/ 一 ,ll < (n > N) (See Definition 7.14.)Since continuous functions are uniformly con tinuous on compact sets,there is a $\delta>0$ such that (43) If,(x) -f(y)|<8 if l≤i≤ N and d(x,)<6. If n > N and dx, J)<6,it follows that lf.(x)-f,Oy)≤ IfA(x)-f,(x)|+ |f,(x)一f爪(y)|+ |fx(y)一/,0)|<38 In conjunction with (43), this proves the theorem.158 PxINCTPLES Or MATHEMATICAL ANALYsis 7.25 Theorem If K is compact, ü f/, G(K) for n =1,2,3,.…,ad iV {U.} pointwise bounded and equicontinuous on K, then (a){f,} is uniformly bounded on K (b){ fA} contains a uniformly convergent subsequence. Proof 7.22, so that (a) Let e >0 be given and choose 6> 0,in accordance with Definition (44) lf,(x)-f.Oy)|<B for all n, provided that d(x, y)<6. xe K. This proves (a) Since K is compact, there are finitely many points p……,, in K with d(x,p)<员 such that to every xe K corresponds at least one $\mathbf{\mathcal{P}}_{i}$ SincefA}is pointwise bounded, there exist M,< oo such tht IfAdp)<M, for all n. If M = max(M,.…,M,),then IJ,Ax))< M +e for every (b)Let E be a countable dense subset of K.(For the existence of such a set E,see Exercise 25, Chap. 2.) Theorem 7.23 shows that {/,}has a subsequencef,} such that {f/,Xx)} converges for every xe E Put /.= 91,to simplify the notation. We shall prove that {g} converges uniformly on K. V(x, 8) be the set of all Let e> 0, and pick 6> 0 as in the beinning of this proof. Let $\nu\in K$ with d(x, y)<6. Since Eis dense in K, and K is compact, there are finitely many points x,……,xmin E such that (45) $$ K\subset V(x_{1},\,\delta)\,\cup\,\cdots\,\cup\,V(x_{m},\,\delta). $$ that Since {g9,0x)} converges for every xe E, there is an integer $\mathcal{N}$ such (46) $$ |g_{i}(x_{s})-g_{j}(x_{s})\,|<\varepsilon $$ whenever i≥ N,j≥ N,1 ≤s≤m If xe K,(45) shows that xe V(x,,の) for some s, so that $$ |g_{i}(x)-g_{i}(x_{s})\,|<\varepsilon $$ for every i.Ifi≥ $\mathcal{N}$ and j≥ N,it follows from (46) tha lg ,(x)-9,x)|≤ lg,(0x) -g,(xs)|+|9,(x,)= 9,(x.)」+ g ,Cx,)- 9(x) < 38. This completes the proofSEQUENCES AND SERIES OF FUNCrONS 159 THE STONE-WEIERSTRASS THEOREM 7.26 Theorem Iffis a continuous complex function on [a,b], there exists a sequence of polynomials $\textstyle{\mathit{P}}_{n}$ such thai $$ \operatorname*{lim}_{n\to\infty}P_{n}(x)=f(x) $$ uniformly on [a, b].If f is real, the $\textstyle{\mathit{P}}_{n}$ may be taken real. This is the form in which the theorem was originally discovered by Weierstrass Proof We may assume, without loss of generality,that [a, b] = [0,1] We may also assume that f(0) =f(1) = 0. For if the theorem is proved for this case, consider $$ g(x)=f(x)-f(0)-x[f(1)-f(0)]\qquad(0\leq x\leq1). $$ Here g(0) = 9(1) = 0, and if g can be obtained as the limit of a uniformly convergent sequence of polynomials, it is clear that the same is true for f, since f-g is a polynomial Furthermore, we define f(x) to be zero for x outside [0,1]. Then f is uniformly continuous on the whole line. We put (47) $$ Q_{n}(x)=c_{n}(1-x^{2})^{n}\qquad(n=1,\,2,\,3,\,\dots), $$ where ${\cal{C}}_{p i}$ is chosen so that (48) $$ \bigcap_{-1}^{1}\ Q_{n}(x)\,d x=1\qquad(n=1,2,3,\dots). $$ We need some information about the order of magnitude of cn.Since $$ \bigcap_{-1}^{1}\;(1-x^{2})^{n}\,d x=2\int_{0}^{1}(1-x^{2})^{n}\,d x\geq2\int_{0}^{1/\sqrt{n}}(1-x^{2})^{n}\,d x $$ > 2 dx 4 3、/n 1 √n it follows from(48) that (49) $$ c_{n}<\sqrt{n}. $$160 PRINCIPLES OF MATHIEMATICAL ANALYsis The inequality (1 -x)"≥1-nx” which we used above is casil shown to be true by considering the function $$ (1-x^{2})^{n}-1+n x^{2} $$ For any $\delta>0,$ which is zero at x = 0 and whose derivative is positive in (O,1) (49) implies (50) $$ Q_{n}(x)\leq{\sqrt{n}}\,(1-\delta^{2})^{n}\qquad(\delta\leq|x\,|\leq1), $$ so that $\scriptstyle G_{s} arrow\{$ o uniformly in $\delta\leq|x|\leq1$ Now set (51) $$ P_{n}(x)=\int_{-1}^{1}f(x+t)Q_{n}(t)\,d t\qquad(0\leq x\leq1). $$ Our assumptions about f show, by a simple change of variable, that $$ P_{n}(x)=\int_{-x}^{1-x}f(x+t)Q_{n}(t)\,d t=\int_{0}^{1}\!f(t)Q_{n}(t-x)\,d t, $$ and the last integral is clearly a polynomial in x.Thus {P, is a sequence of polynomials, which are real if fis real Given 8> 0,we choose $\delta>0$ such that $y-x\mid<\delta$ implies $$ |f(y)-f(x)|<{\frac{s}{2}}. $$ see that for 0≤x≤1, Let M = sup I/(X)|. Using(48),(50), and the fact that QAXx)≥0,we $$ \begin{array}{r l}{P_{n}(x)-f(x)|=\left|\int_{-1}^{1}\left[f(x+t)-f(x)\right]Q_{n}(t)\;d t\right|}\\ {\leq\int_{-1}^{1}\left|f(x+t)-f(x)\left|Q_{n}(t)\;d t}\\ {\leq2M\int_{-1}^{-s}Q_{n}(t)\;d t+{\frac{s}{2}}\right.}\\ {\leq4M\sqrt{n} (1-\delta^{2}\right)^{n}+{\frac{s}{2}}}\end{array} $$ <8 for all large enough n,which proves the theorem t is instructive to sketch the graphs of $\textstyle{Q_{n}}$ for a few values of n; also note that we needed uniform continuity of fto deduce uniform convergence of {Pm}.sroUENCES AND SERIEs Or FUNcrioNs 161 In the proof of Theorem 7.32 we shall not need the full strength of Theorem 7.26, but only the following special case, which we state as a corollary 7.27Corollary For every interval[- a,al there is a sequence of real poly nomials P。such that P,(0) = 0 and such that n→co lim P,.x) = |xl uniformly on[-a,a]. Proof By Theorem 7.26,there exists a sequence{P*} of real polynomial which converges to |x| uniformly on[-a,a].In particular,P*(0)→0 as n→OO. The polynomials $$ P_{n}(x)=P_{n}^{*}(x)-P_{n}^{*}(0)\qquad(n=1,2,3,\ldots) $$ have desired properties. We shall now isolate those properties of the polynomials which make the Weierstrass theorem possible. 7.28 Definition A family .s of complex functions defined on a set $\textstyle{\mathcal{R}}$ is said to be an algebra if (i)f+ g ∈ .ol,,(ii) fg e ol, and (ii) cf ∈ s for all f∈ .od,g ∈ .l and for all complex constants c, that is,if ${\mathcal{Q}}$ is closed under addition, multi- plication, and scalar multiplication. We shall also have to consider algebras of real functions; in this case,(ii) is of course only required to hold for all real c If .d has the property that f∈ s whenever f,∈ d(n = 1,2, 3,...) and f。→f uniformly on ${\cal K},$ then .d is said to be uniformly closed Let S be the set of all functions which are limits of uniformly convergent sequences of members of .l. Then W is called the uniform closure of .oM.(See Definition 7.14.) For example,the set of all polynomials is an algebra, and the Weierstras theorem may be stated by saying that the set of continuous functions on [a,b is the uniform closure of the set of polynomials on [a,b] 7.29 Theorem Let % be the uniform closure of am algebra .dof bounded functions. Then B is a uniformly closed algebra. Proof If f∈ Z and g ∈ 38,there exist uniformly convergent sequences { f},{9,} such that f,→f,9。→g and f,∈ ,sl,9,∈ .0.Since we are dealing with bounded functions, it is easy to show that $$ \begin{array}{c c c}{{f_{n}+g_{n} arrow f+g,}}&{{\quad f_{n}g_{n} arrow f g,}}&{{\quad c f_{n} arrow c f.}}\end{array} $$ where c is any constant, the convergence being uniform in each case. Hence f+9 ∈ 43, fg e .M, and cf e &W,so that M is an algebra. By Theorem 2.27,W is (uniformly) closed.162 PRINCIPLEs Or MATHEMATICAL ANALYSIs 7.30 Definition Let sd be a family of functions on a set E. Then d is said to separate points on ${\widetilde{\mathcal{P}}}_{?}^{\vee}$ if to every pair of distinct points x1,Xz∈ E there corre- sponds a function f∈ .d such that f(x1)≠/(xz). If to each x∈ E there corresponds a function g∈ d such that g(x)≠0 we say that sd vanishes at no point of E. The algebra of all polynomials in one variable clearly has these properties O力 $\textstyle{\mathcal{R}}^{1}$ .An example of an algebra which does not separate points is the set of all even polynomials, say on[-1,1], since f(一x) =/(Kx) for every even function f. The following theorem wil illustrate these concepts further 7.31 Theorem Suppose d is an algebra of functions on a set E, s separates points on E, and s vanishes at no point of E. Suppose x,X2 are distinct points Of $\textstyle{E_{3}}$ E、and $C_{13}$ ,C, are constants (real )d is a real algebra)、 Then sl contains c function f such that $$ f(x_{1})=c_{1},\qquad f(x_{2})=c_{2}\,. $$ Proof The assumptions show that s contains functions g,h、and k such that $$ g(x_{1})\neq g(x_{2}),\qquad h(x_{1})\neq0,\qquad k(x_{2})\neq0. $$ Put $$ u=g k-g(x_{1})k,\qquad v=g h-g(x_{2})h. $$ Then u ∈ .d,v ∈ .s, $$ \iota(x_{1})=v(x_{2})=0,u(x_{2})\neq0, $$ and v(x,)≠ 0. Therefore $$ f={\frac{c_{1}v}{v(x_{1})}}+{\frac{c_{2}u}{u(x_{2})}} $$ has the desired properties. We now have all the material needed for Stone's generalization of thc Weierstrass theorem. 7.32Theorem Let d be an algebra of real continuous functions on a compac set K. If .sl separates points on $\textstyle K$ and if sd vanishes at no point of K, then the uniform closure U》 of w consists of all real continuous functions on K. We shall divide the proof into four steps STEP 1 If f∈ 3,then Lfle W Proof Let (52) 4 = sup IfOx) (xe K)seouENCESs AND SERIs or FuNcroNs 163 and let s> 0 be given. By Corollry 7.27 there exist real numbers $c_{19}\ .\cdot\cdot\cdot\cdot\times c_{n}$ such that (53) $$ \left|\sum_{i=1}^{n}c_{i}y^{i}-\ |y|\right|<\varepsilon\qquad(-a\leq y\leq a). $$ Since ${\mathcal{B}}$ is an algebra,the function $$ g={\frac{a}{i-1}}c J^{2} $$ is a member of g. By (52) and (53), we hav $$ |g(x)-|f(x)||<\varepsilon\qquad(x\in K). $$ Since 8 is uniformly closed, this shows that lf| ∈ % STEP 2If fe W and g e %,then max(f, g)∈ SW and min(f, g)∈ 8% By max(,g) we mean the function $\textstyle{\hat{J}}$ defined by $$ h(x)=\!{\binom{f(x)}{g(x)}}\quad{\begin{array}{r l}{{\mathrm{if}}f(x)\geq g(x),}\\ {{\mathrm{if}}f(x)<g(x),}\end{array}} $$ and min(f, g) is defined likewise. ProofStep 2 follows from step l and the identities $$ \begin{array}{r}{\operatorname*{max}\left(f,g\right)={\frac{f+g}{2}}+{\frac{|f-g\,|}{2}},}\\ {\operatorname*{min}\left(f,g\right)={\frac{f+g}{2}}-{\frac{|f-g\,|}{2}}.}\end{array} $$ By iteration, the result can of course be extended to any finite set of functions:If, …./, M,then max (C, …..,) %, and min C,…….))e W e xists a function STEP 3 Given a real function f, continuous on K, a point xe K, and s > 0, there ${\mathcal{O}}_{x}$ e G such that g,(x) =/(x) and (54) $$ g_{x}(t)>f(t)-s\;\;\;\;\;\;\;\;(t\in K). $$ Proof Since d c S and d satisfies the hypotheses of Theorem 7.31 sc does Ca、 Hence, for every ye $K_{\it3}$ we can find a function $h_{y}$ e MG such that (55) $$ h_{y}(x)=f(x),\qquad h_{y}(y)=f(y). $$164 PRINCIPLES OF MATHEMLATICAL ANALYSIs By the continuity of ,,there cxists an open set J ${\cal J}_{\mathrm{v}}\,,$ , containing y such that (56) $$ h_{y}(t)>f(t)-s\;\;\;\;\;\;\;\;(t\in J_{y}). $$ Since $\textstyle{\bar{\mathbf{X}}}$ is compact, there is a finte set of points ,…,》Asuch tha (57) $$ K\subset J_{y_{1}}\cup\cdots\cup J_{y_{n}}. $$ Put $$ \ell_{x}=\operatorname*{max}\,(h_{y_{1}}\,,\,\cdot\cdot,\,h_{y_{n}}). $$ By step 2,g,∈ 8, and the relations(55) to (5T) show that g,has the other required properties. STEP 4 Given a real function f, continuous on $K_{\mathrm{{J}}}$ and s > 0, there exists a functior $\textstyle{\int}q$ ∈ g such that (58) $$ |h(x)-f(x)|<s\quad\quad(x\in K). $$ Since M is uniformly closed, this statement is equivalent to the conclusion of the theorem. such that Proot Let us consider the functions g.,for each xeK, constructed i step 3.By the continuity of g., there exist open sets V. containing x (59) $$ g_{x}(t)<f(t)+stextrm{\qquad}(t\in V_{x}). $$ Since K is compact, there exists a finite set of points $x_{1},\ldots,\quad$ Xm such that (60) $$ K\subset V_{x_{1}}\cup\cdots\cup V_{x_{m}}. $$ Put $$ h=\operatorname*{min}\left(g_{x_{1}},\ldots,g_{x_{m}}\right). $$ By step 2,h e 8,and(54) implies (61) $$ h(t)>f(t)-s\quad\quad(t\in K), $$ whereas (59) and (60) imply (62) $$ h(t)<J(t)+s\quad\quad(t\in I $$ K). Finally,(S) follows from (61) and (62)SEOUENCES AND SERIES OF FUNCTIONS165 Theorem 7.32 does not hold for complex algebras. A counterexample is given in Exercise 21. However, the conclusion of the theorem does hold, even for complex algebras, if an extra condition is imposed on .sl,namely, that 、a be self-adjoint. This means that for every fe sd its complex conjugate f must also belong to .s4;fis defined by f(x)=7(x). 7.33 Theorem Suppose Mis a self-adjoint algebra of complex continuous functions on a compact set K,.M separates points on K, and .l vanishes at no point of K. Then the uniform closure CB of.od consists of all complex continuous functions on K. In other words,.sl is dense (K) Proof Let ,s、be the set of all real functions on ${\hat{H}}_{\perp}$ which belong to .o If f∈ .od and f = u + it, with u,v real, then 2u =/ + f, and since .s is self-adjoint,we see that u∈ .VR·If x,≠Xz,there exists f∈ .ol such that J(x)= 1,/(×2) = 0; hence 0 = udxz) ≠ ux,)= 1, which shows that ${\mathcal{A}}_{R}$ separates points on K. Ifxe K, then g(x) + 0 for some g ∈ ,o4,and there is a complex number such that Ag(x) > 0;if f = 2.9,/= + iv, it follows that u(x)> 0;hence 。 ${\mathcal{Q}}_{R}$ R vanishes at no point of K. Thus ${\mathcal{A}}_{R}$ e satisfies the hypotheses of Theorem 7.32.It follows that every real continuous function on $\textstyle K$ lies in the uniform closure of sp hence lies in W.If f is a complex continuous function on K, f= u + iv then u ∈ 9W,v ∈ .W,hence f ∈ .3.This completes the proof. EXERCISES 1. Prove that every uniformly convergent sequence of bounded functions is uni formly bounded. 2. If {f} and $\{g_{n}\}$ converge uniformly on a set L $\textstyle E,$ prove that {f, ${\frac{1}{1}}$ gy converges uniformly on $\widehat{H}$ If, in addition,{f} and {gny are sequences of bounded functions, prove that {f.g) converges uniformly on $\widehat{H}$ but such 3.Construct sequences (f.,{g.} which converge uniformly on some set $\textstyle E,$ that {f.go. does not convere uniformly on $\widetilde{{\cal P}^{\prime}}$ (of course,{/o) must coverge o E). 4、Consider $$ f(x)=\sum_{n=1}^{\infty}\frac{1}{1+n^{2}x}. $$ For what values of x does the series converge absolutely ?On what intervals does it converge uniformly?On what intervals does it fail to converge uniformly ? Is 广 continuous wherever the series converges ?Is f bounded ?166 PInNCIPLES Oor MATHEMATICAL ANALyxsis S. Let $$ f_{n}(x)= \langle\operatorname{\left(n-\frac{1}{n}+1\right)}, $$ Show that {fA} converges to a continuous function, but not uniformly. Use the serics Ef,to show that absolute convergence, even for all x, does not imply uni form convergence. 6. Prove that the scries $$ \begin{array}{l c r}{{\stackrel{\circ}{\to}}}&{{}}\\ {{=1}}&{{}}\end{array} $$ converges uniformly in every bounded interval, but does not converge absolutely for any value of x. T. For n= 1,2,3.…x real, pu $$ f_{n}(x)={\frac{x}{1+n x^{2}}}\,. $$ Show that {fA} converges uniformly to a function f, and that the equation $$ f^{\prime}(x)=\operatorname*{lim}_{n\to\infty}f_{n}^{\prime}(x) $$ is correct if $x\neq0,$ but false if $\scriptstyle x\;=\;0.$ 8、If $$ I(x)= \{{\frac{0}{1}}\quad\quad(x\leq0),\quad $$ if $\{x_{n}\}$ is asequence of distinct points of (,、b), and if Elcl converges prove that the series $$ f(x)=\sum_{n=1}^{\infty}c_{n}I(x-x_{n})\qquad(a\leq x\leq b) $$ converges uniformly, and that fis continuous for every $\chi\div\cdot\chi_{n}\ .$ 9. Let {JA} be a sequence of continuous functions which converges uniformly to a function f on a set $\widehat{\overline{{B}}},$ Prove that $$ \operatorname*{lim}_{n\to\infty}f_{n}(x_{n})=f(x) $$ for every sequence of points x。∈ E such that xn→x, andx∈ E. Is the converse of this true ?sEoUENCES AND SERIEs Or FUNCrroNs 167 10. Letting Cx) denote the fractional part of the real number x(see Exercise 16, Chap. 4, for the definition), consider the function $$ f(x)={\frac{\alpha}{\pm}}{\frac{(n x)}{n^{2}}}\qquad(x\,\mathrm{real}). $$ Find all discontinuities of ${\widetilde{\mathcal{Y}}}_{3}.$ and show that they form a countable dense set. Show that fis neverthcless Riemann-integrable on every bounded interval 11. Suppose {J.},{g} are defined on E, and (a)Ef,has uniformly bounded partial sums; (b)g,→O uniformly on E; (c) g(x)≥04x)≥90x)≥… for every xe E. Prove that Sf.g. converges uniformly on E. Hint: Compare with Theore 3.42. 12. Suppose g and f.n 1, ,... re defined on (0,o),are Riemann-integrable on t, T] whenever O <1<T<00,I/|≤9,/。→f uniformly on every compact sub- set of (0,o), and $$ \int_{0}^{\infty}g(x)\,d x<\infty. $$ Prove that $$ \operatorname*{lim}_{n\to\infty}\ \int_{0}^{\infty}f_{n}(x)\,d x=\int_{0}^{\infty}f(x)\,d x. $$ (See Exercises 7 and 8 of Chap. 6 for the relevant definitions.) This is a rather weak form of Lebesgue's dominated convergence theorem (Theorem 11.32). Even in the context of the Riemann integral, uniform conver gence can be replaced by pointwise convergence if it is assumed that fe J.(Se the articles by F. Cunningham in Math. Mag., vo01.40,1967, pp. 179-186, and by H. Kestelman in Amer. Math. Monthly,vol. 77, 1970, pp. 182-187.) 13。Assume that {f} is a sequence of monotonically increasing functions on $\textstyle{R^{2}}$ with 0≤/(X)≤1 for all x and all n. (a) Prove that there is a function f and a sequence {n;} such that $$ f(x)=\operatorname*{lim}_{k\to\infty}f_{n_{k}}(x) $$ for cvery xe R'.(The existence of such a pointwise convergent subsequence is usually called Helly's selection theorem.) (b) If, moreover, f is continuous, prove that fA→f uniformly on compact sets Hin::(i) Some subsequence {fm} converges at all rational points r,say, to f(r).(ii) Define f(x), for any xe R',to be sup f(r),the sup being taken over all r≤x、(ii)Show that fm(x)→>f(x) at every x at which $\mathbb{Z}$ is continuous.(This is wherc monotonicity is strongly used.)(iv)A subsequence of {fm} converges a every point of discontinuity off since there are at most countably many such points. This proves (a). To prove (b), modify your proof of (i)) appropriately.168 PRINCIPLEs Or MATHEMATICAL ANALYSsis 14. Let f be a continuous real function on ${\boldsymbol{R}}^{1}$ with the following properties: 0≤f(t)≤1,f(1 $\left.+2\right)=f(t)$ for every ${\hat{f}}_{3}$ and $$ f(t)= \{0\qquad(0\leq t\leq\frac{1}{3})\qquad(0\leq t\leq1). $$ Put D(t)= (x(t), y(t)), where $$ x(t)={\frac{\omega}{s_{\pm}}}2^{-s}f(3^{2s-1}t),\qquad y(t)={\frac{\omega}{s_{\pm}}}2^{-s}f(3^{2s}t). $$ Prove that O is continuous and that ${\widehat{\mathbb{D}}}$ $\scriptstyle T\scriptstyle-\oplus(\ln1)$ onto the unit square 1 c R p maps If fact, show that ${\mathsf{C}}{\mathsf{D}}$ maps the Cantor set onto $\textstyle\int\!^{2}$ Hint:Each $(x_{0},y_{0})\in I^{2}$ has the form $$ x_{0}=\sum_{n=1}^{\infty}2^{-n}a_{2n-1},\qquad y_{0}=\sum_{n=1}^{\infty}2^{-n}a_{2n} $$ where each ${\mathcal{Q}}_{j}$ is O or 1.If $$ t_{0}=\sum_{i=1}^{\infty}3^{-i-1}(2a_{i}) $$ show that f(3*to) = ak,and hence that $x(t_{0})=x_{0}\,,\,y(t_{0})=y_{0}\,.$ (This simple example of a so-called “space-filling curve”is due to I. J. Schoenberg,Bull. A.M.S., vol.44,1938, pp. 519.) 15.Suppose fis a real continuous function on R',f.(t) = f(nt)for n=1,2, 3,.…., and {JA} is equicontinuous on [0,1. What conclusion can you draw about f? 16.Suppose $\{f_{n}\}$ is an equicontinuous sequence of functions on a compact set $K_{\mathrm{{J}}}$ and {fA} converges pointwise on $K.$ Prove that {fh} converges uniformly on K 17. Define the notions of uniform convergence and equicontinuity for mappings into any metric space. Show that Theorems 7.9 and 7.12 are valid for mappings into any metric space, that Theorems 7.8 and 7.11 are valid for mappings into any complete metric space, and that Theorems 7.10, 7.16, 7.17, 7.24, and 7.25 hold for vector-valued functions, that is, for mappings into any $R^{k}.$ 18. Let {f,} be a uniformly bounded sequence of functions which are Riemann-inte grable on [a,b], and put $$ F_{n}(x)=\int_{a}^{x}\!f_{n}(t)\,d t\qquad(\alpha\leq x\leq b). $$ Prove that there exists a subsequence { $\{F_{n_{k}}\}$ ;}which converges uniformly on [a,b]. 19. Let K bea compact metric space, let ${\mathfrak{s}}_{i\to}^{\prime}$ bea subset of G(K). Prove that $*_{k}^{\infty}$ is compact (with respect to the metric defined in Section 7.14) if and only if S is uniformly closed, pointwise bounded,and equicontinuous.(If S is not equicontinuous then S contains a sequence which has no equicontinuous subsequence, hence has no subsequence that converges uniformly on K.)sEoUENCES AND SERIEs Or FUNCTroNs 169 20. If fis continuous on [O, 1] and if $$ \bigcap_{o}^{1}f(x)x^{n}\,d x=0\qquad(n=0,1,2,\ldots), $$ prove that /(x) = 0 on [O,1]. Hint: The integral of the product of f with any polynomial is zero、 Use the Weierstrass theorem to show that $ \langle\begin{array}{l}{{\mathfrak{I}}}\\ {{}}\\ {{}}\\ {{\mathfrak{U}}}\end{array} \rangle$ f3(x) dx = 0. 21. Let K be the unit circle in the complex plane (i.e., the set of all z with $|z|=1\mathrm{{,}}$ and let s be the algebra of all functions of the form $$ f(e^{i\theta})=\sum_{n=0}^{N}c_{n}e^{i n\theta}\qquad(\theta\mathrm{~real}). $$ Then os separates points on $\textstyle{\mathcal{N}}$ and .ol vanishes at no point of $K_{\!_{J}}$ but nevertheless there are continuous functions on $\textstyle{\mathcal{H}}$ which are not in the uniform closure of .sl Hint: For every fe .sl $$ \textstyle{\int_{0}^{2\pi}f(e^{i\theta})e^{i\theta}\,d\theta=0,} $$ and this is also true for every f in the closure of .sl 22.Assume f∈ OXα) on [a, b], and prove that there are polynomials $\textstyle P_{n}$ such that $$ \operatorname*{lim}_{n arrow\infty}\ \int_{a}^{b}|f-P_{n}|^{2}\ d x=0. $$ (Compare with Exercise 12,Chap. 6.) 23. Put $p_{n}=0.$ and define, for $n=0,1,2$ 2,.… $$ P_{n+1}(x)=P_{n}(x)+{\frac{x^{2}-P_{n}^{2}(x)}{2}}\,. $$ Prove that lim P,(x)=|x|, uniformly on [-1, 1]. (This makes it possible to prove the Stone-Weierstrass theorem without firs proving Theorem 7.26.) Hint: Use the identity $$ |x|-P_{n+1}(x)=[|x|-P_{n}(x)]\biggl[1-{\frac{|x|+P_{n}(x)}{2}}\biggr] $$ to prove that O≤P,(x)≤P.+1(x)≤|x| if |x|≤1, and that $$ |x|-P_{n}(x)\leq|x|\left(1-{\frac{|x|}{2}}\right)^{n}<{\frac{2}{n+1}} $$ if }xl ≤1.170 PRINCIPLES OF MATHEMATICAL ANALYSIs 24. Let X be a metric space, with metric $Q^{\prime}$ Fix a point a e X.Assign to each p e X the function f, defined by $$ f_{p}(x)=d(x,p)-d(x,a)\qquad(x\in X) $$ Prove that 」f,(x)|≤d(a, p)) for all $x\in X,$ Y, and that therefore f,e G(X) Prove that for all ,。 e X. $$ |f_{p}-f_{\alpha}||=d(p,q) $$ If D(p)=f,it follows that ${\mathsf{C P}}$ is an isometry(a distance-preserving mappingp of X onto D(X)c G(X). Let Y be the closure of Q(X)in G(X).Show that $\textstyle\gamma$ is complete Conclusion: X is isometric to a dense subset of a complete metric space Y. (Exercise 24,Chap. 3 contains a different proof of this.) 25. Supposepis a continuous bounded real function in the strip defined by 0≤x≤1,-0 <y<CO. Prove that the initial-value problem $$ y^{\prime}=\phi(x,y),\qquad y(0)=c $$ has a solution.(Note that the hypotheses of this existence theorem are less stringen than those of the corresponding uniqueness theorem;see Exercise 27, Chap. 5.) Hint: Fix n. Fori= 0,..,n, put x = i/n. Let f, be a continuous function on [O,1] such that $\scriptstyle f_{i(0)=1}$ C, $$ f_{n}^{\prime}(t)=\phi(x_{i},f_{n}(x_{i}))\qquad{\mathrm{if~}}x_{i}<t<x_{i+1}, $$ and put $$ \Delta_{n}(t)=f_{n}^{\prime}(t)-\phi(t,f_{n}(t)), $$ except at the points $X_{i}\,,$ where $\Delta_{n}(t)=0$ Then $$ f_{n}(x)=c+\int_{0}^{x}[\phi(t,f_{n}(t))+\Delta_{n}(t)]\,d t. $$ Choose M <O so that lp|≤M. Verify the following assertions. (a)|f'≤M,|An≤2M,A。 UR,and」,|≤lcl+ M= M, say, on 【[0,1,for all n. (6){f,} is equicontinuous on [0,1], since「f州≤M (c) Some {f} converges to some f uniformly on [0,1] (d)Since d is uniformly continuous on the rectangle 0≤x≤1,|y|≤M, $$ \phi(t,f_{n_{k}}(t)) arrow\phi(t,f(t)) $$ uniformly on [0,1] (e)A,(t)→0 uniformly on [0,1], since $$ \Delta_{n}(t)=\phi(x_{i},f_{n}(x_{i}))-\phi(t,f_{n}(t)) $$ in (xi,X1+1)SEQUENCES AND SERIES Or FUNCTiONs 171 (f) Hence $$ f(x)=c+\int_{0}^{x}\!\phi(t,f(t))\,\,d t. $$ This fis a solution of the given problem. 26. Prove an analogous existence theorem for the initial-value problem $$ \mathbf{y}^{\prime}=\Phi(x,\mathbf{y}),\qquad\mathbf{y}(0)=\mathbf{c}, $$ where now ce R,y e R', and 1 $\scriptstyle{\frac{\Gamma_{*}}{\hbar}}\;_{\phantom{\mathrm{Gl}}}$ is a continuous bounded mapping of the part of $R^{k+1}$ defined by $0\leq x\leq1,\ y\in R^{n}$ into R*.(Compare Exercise 28, Chap. 5.) Hint: Use the vector-valued version of Theorem 7.25.