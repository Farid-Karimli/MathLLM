8 SOME SPECIALFUNCTIONS POWER SERIES In this section we shall derive some properties of functions which are represented by power series, i.e., functions of the form (1) $$ f(x)=\sum_{n=0}^{\infty}c_{n}x^{n} $$ or, more generally (2) $$ f(x)=\sum_{n=0}^{\infty}c_{n}(x-a)^{n}. $$ These are called analytic functions We shall restrict ourselves to real values of x.Instead of circles of con vergence (see Theorem 3.39) we shall therefore encounter intervals of conver- gence If (I) converges for all x in(-R,R),for some R > 0(R may be +OO) we say that fis expanded in a power series about the point x = 0. Similarly, i (2) converges for $|x-a|<R,f$ is said to be expanded in a power series about the point x= a. As a matter of convenienc, we shall ofen take a = 0 withou any loss of generality.SOME SPECIAL FUNCTIONS173 8.1 Theorem Suppose the series (3) $$ \sum_{n=0}^{\infty}c_{n}\,x^{n} $$ converges for |x|< $\textstyle{\mathcal{R}},$ and define (4) $$ f(x)=\sum_{n=0}^{\infty}c_{n}\,x^{n}\qquad(|x|<R). $$ Then(3) converges uniformly on $[-R+s,\,R-$ B], no matter whichc> 0 is chosen. The function f is continuous and differentiable in(-R,R), and (5) $$ f^{\prime}(x)=\sum_{n=1}^{\infty}n c_{n}\,x^{n-1}\qquad(|x|<R). $$ Proof Lets> 0 be given. For $|x|\leq R-:$ 8, we have $$ \left|\,c_{n}x^{n}\,\right|\leq\,\left|\,c_{n}(R-s)^{n}\,\right|\,; $$ and since $$ \Sigma_{n}(R-\varepsilon)^{n} $$ converges absolutely(every power series converges absolutely in the interior of its interval of convergence,by the root test), Theorem 7.10 shows the uniform convergence of(3) on $[-R+s,R-s]$ Since $\textstyle{n{\sqrt{n}}\to1}$ as $n arrow\infty,$ we have $$ \operatorname*{lim}_{n arrow\infty}\operatorname*{sup}_{n arrow\infty}\ ^{n\sqrt{n\left|\,c_{n}\right|}}=\operatorname*{lim}_{n arrow\infty}\left.\bigwedge^{n\sqrt{ |\,c_{n}\right|}\,}, $$ so that the series (4) and (S) have the same interval of convergence Since(5)is a power series,it converges uniformly in [-R +8 R-e], for every e> 0, and we can apply Theorem 7.17(for series in- stead of sequences). It follows that (S)holds if |x|≤ R-6. But, given any x such that $\scriptstyle x_{1}<R,$ we can find an s> 0 such that |x|<R-E. This shows that (S) holds for Ix|<R. Continuity of f follows from the existence of f’(Theorem 5.2). Corollary Under the hypotheses of Theorem 8.1, $\mathbb{P}$ has derivatives of al orders in(一R,R),which are given by (6) $$ f^{(k)}(x)=\sum_{n=k}^{\infty}n(n-1)\cdot\cdot\cdot(n-k+1)c_{n}x^{n-k}. $$ In particular (7) $$ f^{(k)}(0)=k!c_{k}\qquad(k=0,1,\,2,\,\cdot\cdot). $$ (Here f(O) means f, and ${\mathcal{f}}^{(k)}$ is the kth derivative of f, for k = 1,2,3,...)174 PRINCIPLEs OF MATHEMATICAL ANALYSIs Proof Equation(6) follows if we apply Theorem 8.1 successively to f f′,f”,.….. Putting x= 0 in(6), we obtain(T) Formula(T)is very interesting. It shows,on the one hand,that the coefficients of the power series development offare determined by the values of fand of its derivatives at a single point. On the other hand, if the coefficients are given, the values of the derivatives of f at the center of the interval of con vergence can be read off immediately from the power series Note, however, that although a function f may have derivatives of al orders, the series Zc,X", where ${\mathcal{C}}_{\mathfrak{p}}$ is computed by(T), need not converge to f(x) for any x ≠ 0.In this case, f cannot be expanded in a power series about $\scriptstyle x\;=\;0$ For if we had f(x)= Ea,x", we should have $$ n!a_{n}=f^{(n)}(0); $$ hence $a_{n}=c_{n}$ An example of this situation is given in Exercise 1 then fis continuous If the series (3) converges at an endpoint, say at $x=R_{\circ}$ not only in(-R,R), but also at x= R. This follows from Abel's theorem (for simplicity of notation, we take R= 1): 8.2 Theorem Suppose $\sum_{j=1}^{\infty}(C_{\bar{\beta}\bar{\beta}})$ converges. Put $$ f(x)=\sum_{n=0}^{\infty}c_{n}x^{n}\qquad(-1<x<1). $$ Then (8) $$ \operatorname*{lim}_{x\to1}f(x)=\sum_{n=0}^{\infty}c_{n}. $$ Proot Let s, = C。+ … + C,,s-1 = 0. Then $$ \sum_{n=0}^{m}c_{n}x^{n}=\sum_{n=0}^{m}(s_{n}-s_{n-1})x^{n}=(1-x)\sum_{n=0}^{m-1}s_{n}x^{n}+s_{m}x^{m}. $$ For x|<1, we let m→Oo and obtain (9) $$ f(x)=(1-x)\sum_{n=0}^{\infty}s_{n}x^{n}. $$ Suppose s = lim s。、Let e> O be given. Choose $\mathcal{N}$ so that $\scriptstyle n\geq N$ implies $$ \left|s-s_{n}\right|<{\frac{s}{2}}. $$soME SPECIAL FUNcroNs 175 Then,since $$ (1-x){\sum_{n=0}^{\infty}x^{n}}=1\qquad(|x|<1), $$ we obtain from (9 $$ \left|f(x)-s\right|=\left|(1-x)\sum_{n=0}^{\infty}(s_{n}-s)x^{n}\right|\leq(1-x)\sum_{n=0}^{N}\left|s_{n}-s\right|\left|x\right|^{n}+{\frac{s}{2}}\leq s $$ if $x>1-\delta,$ for some suitably chosen ${\boldsymbol{\delta}}>0$ .This implies(8) ZC,,converge to A,B, C, and iy As an application, let us prove Theorem 3.51, which asserts: If Za,,Ebm 1B. We Ie $c_{n}=\alpha_{0}\,\partial_{n}+\,\cdot\,\cdot\,\cdot\,\cdot\,\cdot\,\cdot\,\cdot\,\cdot\,\cdot\,\nonumber\,\cdot\,\cdot\,\cdot\,\cdot\,a_{n}\,\partial_{0}\,,\,\,t h e n\,\,C=A$ $$ f(x)=\sum_{n=0}^{\infty}a_{n}x^{n},\qquad g(x)=\sum_{n=0}^{\infty}b_{n}x^{n},\qquad h(x)=\sum_{n=0}^{\infty}c_{n}x^{n}, $$ for 0<X≤1. For x<1,these series converge absolutely and hence may be multiplied according to Definition 3.48; when the multiplication is carried out we see that (10) $$ f(x)\cdot g(x)=h(x)\qquad(0\leq x<1). $$ By Theorem 8.2, (11) $$ f(x) arrow A,\qquad g(x) arrow B,\qquad h(x) arrow C $$ as x→1. Equations (10) and(11) imply $\scriptstyle A\theta=C$ We now require a theorem concerning an inversion in the order of sum- mation.(See Exercises 2 and 3.) 8.3 Theorem Given a double sequence {a,}, i= 1,2,3,.., = 1,2,3、.… suppose that (12) $$ \sum_{j=1}^{\infty}\left|a_{i j}\right|=b_{i}\qquad(i=1,2,3,\dots) $$ and $\Sigma b_{i}$ converges.Then (13) $$ \sum_{i=1}^{\infty}\sum_{j=1}^{\infty}a_{i j}=\sum_{j=1}^{\infty}\sum_{i=1}^{\infty}a_{i j}\,. $$ Proof We could establish(13) by a direct procedure similar to (although more involved than) the one used in Theorem 3.55. However, the following method seems more interesting176 PRINCIPLEs OF MATHEMATICAL ANALYSis Let ${\overline{{\mathcal{F}}}}$ be a countable set,consising o the points xo,X,,×,…. and suppose xn→Xo as n→OO. Define (16) $$ \begin{array}{l l}{{f_{i}(x_{0})=\sum_{j=1}^{\infty}a_{i j}}}&{{\qquad(i=1,2,3,\dots),}}\\ {{f_{i}(x_{n})=\sum_{j=1}^{\infty}a_{i j}\qquad(i,n=1,2,3,\dots),}}\\ {{g(x)=\sum_{i=1}^{\infty}a_{i}(x)\qquad(x\in E).}}&{{\qquad(x\in E).}}\end{array} $$ (14) (15) Now,(14) and(15),together with (12), show that each f,is con tinuous at xo.Since $|f_{i}(x)|\leq b_{i}$ for xe E,(16) converges uniformly,so that g is continuous at xo(Theorem 7.11). It follows that $$ \begin{array}{r l}{{\frac{\alpha}{i=1}}\sum_{j=1}^{\infty}a_{i j}=\sum_{i=1}^{\infty}f_{i}(x_{0})=g(x_{0})=\operatorname*{lim}_{n\to\infty}{\sum_{i=1}^{\infty}}\sum_{i=1}^{\infty}\sum_{j=1}^{n}\sum_{j=1}^{n}a_{i j}}\\ {{}=\operatorname*{lim}_{n\to\infty}\sum_{i=1}^{\infty}f_{i}(x_{n})=\operatorname*{lim}_{n\to\infty}\sum_{i=1}^{\infty}a_{i j}=\sum_{j=1}^{\infty}a_{i j}.}\end{array} $$ 8.4 Theorem Suppose $$ f(x)=\sum_{n=0}^{\infty}c_{n}x^{n}, $$ the series converging in「x|< R. ${\mathcal{I}}-R<a<R,$ then f can be expanded in a power series about the point $x=a$ which converges in $|x-a|<R-|a|,$ and (17) $$ f(x)=\sum_{n=0}^{\infty}{\frac{f^{(n)}(a)}{n!}}\,(x-a)^{n}\qquad(\lfloor x-a\rfloor<R-\lfloor a\rfloor). $$ This is an extension of Theorem 5.15 and is also known as Taylor's theorem. Proof We have $$ \begin{array}{c}{{f(x)=\sum_{n=0}^{\infty}c_{n}[(x-a)+a]^{n}}}\\ {{\qquad\qquad\qquad\qquad\qquad\qquad\qquad\qquad\qquad\qquad\qquad\qquad\qquad\qquad\qquad\qquad\qquad\qquad\qquad\qquad\qquad\qquad\qquad\qquad\qquad\qquad\qquad\qquad\qquad\qquad\qquad\qquad\qquad\qquad\qquad\qquad\qquad\qquad\qquad\qquad\qquad\qquad\qquad\qquad\qquad\qquad\qquad\qquad\qquad\qquad\qquad\qquad\qquad\qquad\qquad\qquad\qquad\qquad\qquad\qquad\qquad\qquad\qquad\qquad\qquad\qquad\qquad\qquad\qquad\qquad\qquad\qquad\qquad\qquad\qquad\qquad\qquad\qquad\qquad\qquad\qquad\qquad\qquad\qquad\qquad\qquad\qquad\qquad\qquad\qquad\qquad\qquad\qquad\qquad\qquad\qquad\qquad\qquad\qquad\qquad\qquad\qquad\qquad\qquad\qquad\qquad\qquad\qquad\qquad\qquad\qquad\qquad\qquad\qquad\qquad\qquad\qquad\qquad\qquad\qquad\qquad\qquad\qquad\qquad\qquad\qquad\qquad\qquad\qquad\qquad\qquad\qquad\qquad\qquad\qquad\qquad\qquad\qquad\qquad\qquad\qquad}}\qquad\qquad\qquad\qquad\qquad\qquad\qquad\qquad\qquad\qquad\qquad\qquad\qquad\qquad\qquad\qquad\qquad\qquad\qquad\quad\qquad\qquad\qquad\qquad\qquad\qquad\qquad}{\qquad\qquad\qquad\quad\qquad}\quad\qquad(1}&{\qquad\quad(1}&{\qquad\quad\qquad\quad\qquad\qquad\qquad}&{\quad{\qquad\quad\qquad\quad\qquad\qquad\qquad\quad\quad.}}&{\qquad\qquad}&{\qquad=\qquad=1}&{\qquad\qquad\qquad\qquad}&{\qquad}&{\quad.}&{\qquad{\quad.}&{\ $$soME SPECIAL FUNCTIONS 177 This is the desired expansion about the point $x=a.$ To prove its validity we have to justify the change which was made in the order of summation. Theorem 8.3 shows that this is permissible if (18) $$ \sum_{n=0}^{\infty}\sum_{m=0}^{n}\left|c_{n}\left({n\atop m}\right)a^{n-m}(x-a)^{m}\right| $$ converges. But (18)is the same as (19) $$ \sum_{n=0}^{\infty}|c_{n}|\cdot(|x-a|+|a|)^{n}, $$ and(19) converges if $|x-a|+|a|<R.$ Finally, the form of the coefficients in(17) follows from(7) It should be noted that (17) may actually converge in a larger interval than the one given by lx-a< R - al If two power series converge to the same function in(-R,R),(7) shows that the two series must be identical, i.e., they must have the same coefficients lt is interesting that the same conclusion can be deduced from much weake hypotheses: 8.5 Theorem Suppose the series $\scriptstyle\Sigma a_{a}x^{a}$ ’and Zb,X” converge in the segment S =(-R,R). Let E be the set of all $\mathcal{N}$ e S at which (20) $$ \sum_{n=0}^{\infty}a_{n}x^{n}=\sum_{n=0}^{\infty}b_{n}x^{n}. $$ If E has a limit point in S, then $a_{n}=b_{n}j o r\;n=0,$ 1,2,.... Hence (20) holds for all xe S Proof Put c,= α,一 $\partial_{n}$ and (21) $$ f(x)=\sum_{n=0}^{\infty}c_{n}x^{n}\qquad(x\in S). $$ Then f(x)= 0 on E. Let A be the set of all limit points of E in S, and let B consist of al other points of S. It is clear from the definition of ““limit point”that B is open. Suppose we can prove that A is open. Then A and $\textstyle{\mathcal{D}}$ are disjoint open sets. Hence they are separated(Definition 2.45).Since S= A U B, and S is connected, one of A and B must be empty.By hypothesis, A is not empty. Hence Bis empty, and A = S.Since f is continuous in S, is the desired conclusion. A e E. Thus E = S, and (T) shows that c, = 0 for n = 0, 1, 2, .…, which178 PRINCIPLES OF MATHEMATICAL ANALYSIs Thus we have to prove that Ais open.If xo∈ A, Theorem 8.4 shows that (22) $$ f(x)=\sum_{n=0}^{\infty}d_{n}(x-x_{0})^{n}\qquad(|x-x_{0}|<R-|x_{0}|). $$ We claim that $d_{a}=0$ for all n.Otherwise, let k be the smallest non negative integer such that d、≠ 0. Then (23) $$ f(x)=(x-x_{0})^{k}g(x)\qquad(|x-x_{0}|<R-|x_{0}|), $$ where (24) $$ g(x)=\sum_{m=0}^{\infty}d_{k+m}(x-x_{0})^{m}. $$ Since g is continuous at ${\mathcal{X}}_{0}$ , and $$ g(x_{0})=d_{k}\neq0, $$ there exists a > 0 such that $\scriptstyle{m\cdot{n}}$ if |x- Xo|<6.It follows from (23) that f(x)≠ 0 if $0<\vert\,x-x_{0}\vert$ <6. But this contradicts the fact that xois a limit point of ${\mathcal{H}}^{\nu}$ Thus d,= 0 for all 1 ${\mathcal{H}}_{\mathfrak{g}}$ so that f(x) = 0 for all x for which (22) holds i.e., in a neighborhood of xo. This shows that ${\mathcal{A}}_{\widehat{\Lambda}}$ is open, and completes the proof. THE EXPONENTIAL AND LOGARITHMIC FUNCTIONS We define (25) $$ E(z)=\sum_{n=0}^{\infty}\frac{z^{n}}{n!} $$ The ratio test shows that this series converges for every complex z.Applying Theorem 3.50 on multiplication of absolutely convergent series, we obtain $$ \begin{array}{r l r}{{E(z)E(w)=\sum_{n=0}^{\infty}{\frac{z^{n}}{n!}}\sum_{m=0}^{\infty}{\frac{w^{m}}{m!}}=\sum_{n=0}^{\infty}\sum_{k=0}^{n}{\frac{z^{k}w^{n-k}}{k!(n-k)!}}}\\ {{{}}}&{{=\sum_{n=0}^{\infty}{\frac{1}{n!}}\sum_{k=0}^{n}{\frac{n}{k!(n-k)!}}}\end{array} $$ which gives us the important addition formula (26 $$ E(z+w)=E(z)E(w)\qquad(z,w\cot\mathrm{ex}) $$ One consequence is that (27) E(z)E(-z)= E(z - Z) = E(0) = 1 (z complex)soME SPECIAL FUNCTTONS 179 This shows that $E(z)\neq0$ for all z. By (25),E(x)> 0 ifx> 0; hence(27) shows that $E(x)>0$ for all real x、By (25),E(x)→+OO as x→+Oo;hence(27) shows that $E(x)\to0$ as X→ 一O along the real axis.By (25), 0<x<y implies that E(x)< E(y);by (27),it follows that E(一y)< E(-x); hence E is strictly in- creasing on the whole real axis. The addition formula also shows that (28) $$ \operatorname*{lim}_{h=0}{\frac{E(z+h)-E(z)}{h}}=E(z)\operatorname*{lim}_{h=0}{\frac{E(h)-1}{h}}=E(z); $$ the last equality follows directly from(25) Iteration of (26) gives (29) $$ E(z_{1}+\cdot\cdot\cdot\cdot+z_{n})=E(z_{1})\cdot\cdot\cdot E(z_{n}). $$ Let us take z,= = Z,= 1. Since E(1) = e, where e is the number defined in Definition 3.30, we obtain ((30) $$ E(n)=e^{n}\qquad(n=1,\,2,\,3,\,\cdot\cdot). $$ If p = n/m,where n, m are positive integers, then (31) $$ [E(p)]^{m}=E(m p)=E(n)=e^{n}, $$ so that (32) $$ E(p)=e^{p}\qquad(p>0,p\ \mathrm{rational}). $$ lt follows from(27) that E(-p) = eP if p is positive and rational. Thus (32) holds for all rational p In Exercise 6, Chap. 1, we suggested the definition (33) $$ x^{\nu}=\operatorname{sup}\,x^{p}, $$ where the sup is taken over all rational $\ D$ such that p<y,for any real y, and x>1.If we thus define, for any real x, (34) $$ \varepsilon^{x}=\operatorname{sup}\,e^{p}\qquad(p<x,\,p{\mathrm{~rational}}), $$ the continuity and monotonicity properties of E, together with (32), show that (35 $$ E(x)=e^{x} $$ for all real x.Equation (35) explains why $\widehat{H}^{\nu}$ is called the exponential function The notation exp (x) is often used in place of e*,expecially when x is a complicated expression. Actually one may very well use(35) instead of (34) as the definition of e~ (35)is a much more convenient starting point for the investigation of the properties of e*.We shall see presently that (33) may also be replaced by a more convenient definition [see (43)]180 PRINCIPLES OF MATHEMATICAL ANALYSIS We now revert to the customary notation、e*, in place of E(x), and sum marize what we have proved so far. 8.6 Theorem Let e” be defined on $\textstyle{\mathcal{R}}^{1}$ by (35) and (25). Then (a)e* is continuous and differentiable for all x (b)(e")'= e*; (d)e*+ = e*e”; (c)e*is a strctly increasing function of x, and e* > 0 -OO; (e)e*→ + 00 as x→ + 00,e”→0 as x→ ()lim,-+-X"e"* = 0, for every n. Proof We have already proved (a) to (e);(25) shows that $$ e^{x}>{\frac{x^{n+1}}{(n+1)!}} $$ for x> 0,so that $$ x^{n}e^{-x}<{\frac{(n+1)!}{x}}, $$ and(f) follows.Part (f) shows that ${\mathcal{C}}^{X}$ tends to + “faster”than any power of x, as $x\to+\infty$ Since ${\mathcal{H}}^{\nu}$ is strictly increasing and differentiable on Rl,it has an inverse function L which is also strictly increasing and differentiable and whose domain is E(R'),that is, the set of all positive numbers. L is defined by (36) $$ E(L(y))=y\qquad(y>0), $$ or, equivalently,by (37) $$ L(E(x))=x\qquad(x\ {\mathrm{real}}). $$ Differentiating (37), we get (compare Theorem 5.5) $$ L^{\prime}(E(x))\cdot E(x)=1. $$ Writing y = E(x), this gives us (38) $$ L^{\prime}(y)={\frac{1}{y}}\;\;\;\;\;\;(y>0). $$ Taking $\scriptstyle x\;=\;0$ in (37), we see that $L(1)=0$ .Hence (38) implies (39) $$ L(y)=\textstyle\int_{1}^{y}{\frac{d x}{x}}. $$sOME SPECIAL FUNCTIONs181 Quite frequently,(3) is taken as the starting point of the theory of the logarithm and the exponential function. Writing u= E(x),0 = E(y'),(26) gives $$ L(w)=L(E(x)\cdot E(y))=L(E(x+y))=x+y, $$ so that (40) $$ L(u v)=L(u)+L(v)\qquad(u>0,\,v>0). $$ This shows that L has the familiar property which makes logarithms useful tools for computation. The customary notation for L(x) is of course log x. As to the behavior of log x as x→+0o and as x→0, Theorem 8.6(e) shows that $$ \begin{array}{r l}{\log x\to+\infty}&{{}\quad{\mathrm{as~}}x\to+\infty}\\ {\log x\to-\infty}&{{}\quad{\mathrm{as~}}x\to0.}\end{array} $$ It is easily seen that (41) $$ x^{n}=E(n L(x)) $$ if x> 0O and nis an integer. Similarly,if m is a positive integer, we have (42) $$ x^{1/m}=E\left(\frac{1}{m}\,L(x)\right)\!, $$ since each term of (42)、 when raised to the mth power, yields the corresponding term of(36). Combining (41) and(42), we obtain (43) $$ x^{\alpha}=E(\alpha L(x))=e^{\alpha\log x} $$ for any rational α We now define x,for any real α and any x> 0, by (43). The continuity and monotonicity of ${\mathcal{F}}^{\dagger}$ and $\textstyle{\int}$ show that this definition leads to the same result as the previously suggested one. The facts stated in Exercise 6 of Chap.1, are trivial consequences of (43) If we differentiate(43), we obtain, by Theorem 5.5, (44) $$ (x^{\alpha})^{\prime}=E(\alpha L(x))\cdot{\frac{\alpha}{x}}=\alpha x^{\alpha-1}. $$ Note that we have previously used(44) only for integral values of z, in which case (44) follows easily from Theorem 5.3(6). To prove (44) directly from the definition of the derivative, if x" is defined by (33) and αis irrational is quite troublesome The well-known integration formula for xs follows from (44 if α ≠ -1 (45) x→+ and from (38) if α = -1. We wish to demonstrate one more property of log x, namely, limx- log x = 0182 PRINCIPLEs Or MATHEMATICAL ANALYsis for everyα> 0. That is, log x→+00“slower” than any positive power of x asX→十 0O For if O<8<α, and x> 1,then $$ \begin{array}{r l}{x^{-\alpha}\log x=x^{-\alpha}\int_{1}^{x}t^{-1}\,d t<x^{-\alpha}\int_{1}^{x}t^{\varepsilon-1}\,d t}\\ {=x^{-\alpha}\cdot{\frac{x^{\varepsilon}-1}{\varepsilon}}<{\frac{x^{\varepsilon-\alpha}}{\varepsilon}},}\end{array} $$ and(45) follows. We could also have used Theorem 8.6(f)to derive (45) THE TRIGONOMETRIC FUNCTIONS Let us define (46) $$ C(x)={\frac{1}{2}}\left[E(i x)+E(-i x)\right],\qquad S(x)={\frac{1}{2i}}\left[E(i x)-E(-i x)\right]. $$ We shall show that ${\boldsymbol{C}}({\boldsymbol{x}})$ and S $S(x)$ ) coincide with the functions cos x and sin x whose definition is usually based on geometric considerations.By (25),E(2)= E(z). Hence (46) shows that ${\mathcal{C}}(x)$ and S(x) are real for real x.Also, (47) $$ {\cal E}(i x)=C(x)+i S(x). $$ x is real. By (27) Thus C(x) and S(x) are the real and imaginary parts, respectively,of E(ix), if $$ \mid{\cal E}(i x)\mid^{2}={\cal E}(i x)\overline{{{\cal E}(i x)}}={\cal E}(i x)\cal E(-i x)=1, $$ so that (48) $$ |E(i x)|\,=1 $$ From(46) we can read off that $C(0)=1,\ S(0)=0,$ , and (28)) shows that (49) $$ C^{\prime}(x)=-\,S(x),\qquad S^{\prime}(x)=C(x). $$ x> 0, hence $S^{\prime}(x)>0,$ by (49), hence $*\subseteq\operatorname{\nabla}_{k}^{\vee}$ we assert that there exist positive numbers x such that C(x) = 0. For we have we have suppose this is not so.Since $C(0)=1,$ it then follows that C(x)> 0 for al $S(x)>0$ if x > 0. Hence $i\Gamma\;0<x<y,$ is strictly increasing; and since S(O) = 0, (50) $$ S(x)(y-x)<\int_{x}^{y}\!S(t)\,d t=C(x)-C(y)\leq2. $$ The last inequality follows from(48) and (47).Since $S(x)>0,$ (50) cannot be true for large y, and we have a contradiction.soMB SPECIAL FUNCroNs 183 Let xo be the smallest positive number such that C(x)) = 0. This exists, since the set of zeros of a continuous function is closed, and C(O) ≠ 0.We define the number r by (51) $$ \pi=2x_{0}\,. $$ Then C(元/2) = 0,and(48))shows that S(元/2) = ±1. Since C(x)>0 in (0,7/2), S is increasing in (0,7/2); hence $S(\pi/2)=:\;$ 1. Thus $$ E\left(\frac{\pi i}{2}\right)=i, $$ and the addition formula gives (52) $$ E(\pi i)=\,-\,1,\;\;\;\;\;\;\;E(2\pi i)=\,1\,; $$ hence (53) E(z + 2mi)= E(z) (z complex). 8.7 Theorem (a)The function E is periodic, with period 2ri (b) The functions C and S are periodic, with period 2元. (c) If O<t< 2元, then E(it)≠1. such that E(in) = 2. (d)1f z is a complex number with |z|=1, there is a uniquet in [0,2r) Proof By (53),(a) holds; and (b) follows from (a) and (46) Suppose 0<1< r/2 and E(it) = × +iy, with x, real. Our preceding work shows that O<x<1,0<y<1. Note that $$ E(4i t)=(x+i y)^{4}=x^{4}-6x^{2}y^{2}+y^{4}+4i x y(x^{2}-y^{2}). $$ If O ≤t<1<2m, then If E(ait) is real, it follows that x- y = 0; since x + ) = 1, by (48) we have x = y = , hence E(4i1) = -1. This proves (c) $$ E(i t_{2})[E(i t_{1})]^{-1}=E(i t_{2}-i t_{1})\neq1, $$ by (c).This establishes the uniqueness assertion in (d) To prove the existence assertion in (d),fix z so that |z|= 1. Write 2 = X+iy, with x and y real. Suppose first that x≥0 and y≥ 0.On [0,7/2], C decreases from I to 0.Hence Ct) = x for some te [O, r/21 Since $Z^{2}+S^{2}=1$ and S≥0 on [0, 7/2], it follows that z = E(it) If x<0 and y≥ 0, the preceding conditons are satisfied y -iz. Hence -iz = E(it) for some te [O, 7/2j, and since i= E(ri/2),we obtain 2 = E(i(t + r/2))、 Finally,if y<0, the preceding two cases show that184 PRINCIPLEs Or MATHEMATICAL ANALYSis - z = E(it) for some te(0, r). Hence $z=-\mathcal{E}(i t)=\mathcal{E}(i(t+\pi)).$ This proves (d), and hence the theorem It follows from (d) and (48) that the curve y defined by (54) $$ \gamma(t)=E(i t)\qquad(0\leq t\leq2\pi) $$ is a simple closed curve whose range is the unit circle in the plane. Since y′(t)=iE(it), the length of yis $$ \int_{0}^{2\pi}\vert\gamma^{\prime}(t)\vert\,d t=2\pi, $$ by Theorem 6.27. This is of course the expected result for the circumference of a circle of radius 1.It shows that r, defined by(51), has the usual geometric significance. In the same way we see that the point y(t)describes a circular arc of length to as t increases from O to ${\mathcal{F}}_{0}$ .Consideration of the triangle whose vertices are $$ z_{1}=0,\qquad z_{2}=\gamma(t_{0}),\qquad z_{3}=C(t_{0}) $$ shows that C(t) and S(t)are indeed identical with cos t and sin t, if the latter are defined in the usual way as ratios of the sides of a right triangle. It should be stressed that we derived the basic properties of the trigono- metric functions from(46) and (25), without any appeal to the geometric notion of angle. There are other nongeometric approaches to these functions. The papers by W. F. Eberlein (Amer. Math. Monthly, vol.74,1967, pp. 1223-1225) and by G. B. Robison (Math. Mag., vol. 41,1968,pp.66-70) deal with these topics. THE ALGEBRAIC COMPLETENESS OF THE COMPLEX FIELD We are now in a position to give a simple proof of the fact that the complex field is algebraically complete, that is to say, that every nonconstant polynomia with complex coefficients has a complex root. 8.8 Theorem Suppose $a_{0}\,,\,\cdot\cdot,$ $Q_{n}$ are complex numbers,n≥ 1,a,≠ 0, $$ P(z)=\sum_{0}^{n}\,a_{k}\,z^{k}. $$ Then $P(z)=0$ for some complex number z. Proof Without loss of generality, assume a, = 1. Put (55) $$ \mu=\operatorname*{inf}\ |P(z)|\qquad(z\cos\mathrm{plex}) $$ If |z|= R, then (56) $$ |P(z)|\geq R^{n}[1-|a_{n-1}|R^{-1}-\cdots-|a_{0}|R^{-n}]. $$soME SPECIAL FUNCTONs 185 The right side of(56) tends to o as R→OO.Hence there exists $\textstyle R_{0}$ Such that |P(z)|>p if $|z|>R_{0}$ 。Since $|P|$ Pis continuous on the closed disc with center at O and radius $\textstyle{\mathcal{R}}_{0}$ ,Theorem 4.16 shows that |P(z0)= p for some zo We claim that $\scriptstyle\mu=0$ If not, put Q(z)= P(z + Zz0)/P(zo). Then $\bigotimes_{\mathbb{Z}}$ is a nonconstant poly- nomial,Q(0) = 1,andQ(z)|≥1 for all z. There is a smallest integer k 1≤k≤n,such that (57) $$ Q(z)=1\,+\,b_{k}z^{k}+\,\cdot\,\cdot\,+\,b_{n}z^{n},\qquad b_{k}\ne0, $$ By Theorem 8.7(d) there is a real such that (58) $$ e^{i k\theta}b_{k}=-\left|\,b_{k}\right|. $$ If r> 0 and r"|b,<1,(58) implies $$ |1+b_{k}r^{k}e^{i k\theta}|=1-r^{k}|b_{k}|\,, $$ so that $$ Q(r e^{i\theta})|\leq1-r^{k}\{|b_{k}|-r|b_{k+1}|-\cdot\cdot\cdot-r^{n-k}|b_{n}|\}. $$ For suficiently small r, the expression in braces is positive; hence Q(re')<1,a contradiction Thus u = 0, that is, P(zo) = 0 Exercise 27 contains a more general result. FOURIER SERIES 8.9 Definition A trigonometric polynomial is a finite sum of the form (59) $$ f(x)=a_{0}+\sum_{n=1}^{N}\left(a_{n}\cos n x+b_{n}\sin n x\right)\qquad{\mathrm{(xrea}} $$ al), where $a_{0}\,,\,\cdot\cdot\cdot,\,a_{N}\,,\,b_{1},\,\cdot\cdot\cdot,\,b_{N}$ are complex numbers. On account of the identities (46),(59) can also be written in the form (60) $$ f(x)=\sum_{-N}^{N}c_{n}e^{i n x}\qquad(x\,\mathrm{real}), $$ which is more convenient for most purposes. It is clear that every trigonometric polynomial is periodic, with period 2r. If ${\mathcal{N}}$ is a nonzero integer, e'"” is the derivative of ${e}^{i n x}/i\eta,$ which also has period 2r.Hence (61) $$ {\frac{1}{2\pi}}\int_{-\pi}^{\pi}e^{i n x}\,d x= \{0\qquad({\mathrm{if~}}n=0), $$186 PRINCIPLEs OF MAXTHEMATICAL ANALYSIs Let us multiply (60)by e-tmx, where m is an integer; if we integrate th product,(61) shows that (62) $$ c_{m}={\frac{1}{2\pi}}\int_{-\pi}^{\pi}f(x)e^{-i m x}\,d x $$ for |m|≤ N.If」m| > N, the integral in (62) is O. The following observation can be read off from(60)and(62): The trigonometric polynomial f, given by(60), is real if and only if c-, = C。 for $n=0,\ldots,$ N. In agreement with (60), we define a trigonometric series to be a series of the form (63) $$ \sum_{-\infty}^{\infty}c_{n}e^{i n x}\qquad(x\ \mathrm{real}); $$ the Nth partial sum of (63) is defined to be the right side of (60) If fis an integrable function on [-r,r], the numbers $C_{m}$ defined by (62) for all integers m are called the Fourier coefficients of f, and the series (63) formed with these coefficients is called the Fourier series of f. The natural question which now arises is whether the Fourier series of f converges to f, or, more generally, whether fis determined by its Fourier series That is to say,if we know the Fourier coefficients of a function, can we find the function, and if so, how? The study of such series, and, in particular, the problem of representing a given function by a trigonometric series, originated in physical problems such as the theory of oscillations and the theory of heat conduction (Fourier's “Théorie analytique de la chaleur”was published in 1822). The many difficul and delicate problems which arose during this study caused a thorough revision and reformulation of the whole theory of functions of a real variable.Among many prominent names, those of Riemann, Cantor, and Lebesgue are intimately connected with this field, which nowadays, with all its generalizations and rami- fications, may well be said to occupy a central position in the whole of analysis. We shall be content to derive some basic theorems which are easily accessible by the methods developed in the preceding chapters. For more thorough investigations,the Lebesgue integral is a natural and indispensable too1. We shall frst study more general systems of functions which share a property analogous to (61). 8.10 Definition Let {pm}(n = 1,2,3,…..) be a sequence of complex functions on [a,b], such that (64) $$ \int_{a}^{b}\phi_{n}(x){\overline{{\phi_{m}(x)}}}\,d x=0\qquad(n\neq m). $$soMe SPECIAL FUNCTIONs 187 Then {qp} is said to be an orthogonal system of functions on [a, b]. If, in addition, (65) $$ \textstyle\int_{a}^{b}\left|\phi_{n}(x)\right|^{2}\,d x=1 $$ for all n,{p,} is said to be orthonormal. For example, the functions(2r)- telns form an orthonormal system on [-r,T]. So do the real functions $$ \frac{1}{\sqrt{2\pi}},\frac{\cos x}{\sqrt{\pi}},\frac{\sin x}{\sqrt{\pi}},\frac{\cos2x}{\sqrt{\pi}},\frac{\sin2x}{\sqrt{\pi}},\cdot\cdot\cdot. $$ If {dp,} is orthonormal on [a, b] and if (66) $$ c_{n}=\int_{a}^{b}f(t){\overline{{\phi_{n}(t)}}}\,d t\qquad(n=1,\,2,\,3,\,\ldots), $$ we call c, the nth Fourier coefficient of f relative to {p,}. We write (67) $$ f(x)\sim\sum_{1}^{\infty}c_{n}\phi_{n}(x) $$ and call this series the Fourier series of f (relative to {p}) Note that the symbol ~ used in(67)) implies nothing about the conver- gence of the series; it merely says that the coefficients are given by (66) The following theorems show that the partial sums of the Fourier serie of f have a certain minimum property.We shall assume here and in the rest of this chapter that f∈ S2, although this hypothesis can be weakened. 8.11 Theorem Let {p,} be orthonormal on [a,b] Let (68) $$ s_{n}(x)=\sum_{m=1}^{n}c_{m}\,\phi_{m}(x) $$ be the nth partial sum of the Fourier series of f, and suppose (69) $$ t_{n}(x)=\sum_{m=1}^{n}\gamma_{m}\phi_{m}(x). $$ Then (70) $$ \int_{a}^{b}|f-s_{n}|^{2}\;d x\leq\int_{a}^{b}|f-t_{n}|^{2}\;d x, $$ and equality holds if and only i (71) $$ \gamma_{m}=c_{m}\qquad(m=1,\,\ldots,\,n). $$ That is to say, among all functions $t_{n},\ S_{n}$ gives the best possible mean square approximation to f188 PRNCPLES or MATHEMATICAL ANALYSIs Proof Letfdenote the integral over [a, b], the sum from l to n. Then $$ \textstyle\int\!f\overline{{{t}}}_{n}=\int\!f\sum\overline{{{\gamma}}}_{m}\widetilde{\phi}_{m}=\sum c_{m}\widetilde{\gamma}_{m} $$ by the definition of c} $$ \int|t_{n}|^{2}=\int t_{n}\bar{t}_{n}=\int\sum\gamma_{m}\phi_{m}\sum\bar{\gamma}_{k}\bar{\phi}_{k}=\sum|\gamma_{m}|^{2} $$ since {p}is orthonormal, and so $$ \begin{array}{r l}{\int|f-t_{n}|^{2}=\int|f|^{2}-\int\!\!f{\vec{t}}_{n}-\int\!f{\vec{t}}_{n}+\int\!\!\!f_{n}|^{2}\!\!\!\!\!}\\ {\ }&{{=\int\!\!\left|f\right|^{2}-\sum c_{m}{\vec{\gamma}}_{m}-\sum{\vec{c}}_{m}\,\gamma_{m}+\sum\vert\gamma_{m}-c_{m}|^{2},}\end{array} $$ Putting $\gamma_{m}\equiv\,c_{m}$ which is evidently minimized if and only if $\gamma_{m}=c_{m}$ in this calculation, we obtain (72) $$ \int_{a}^{b}|s_{n}(x)|^{2}\,d x=\sum_{1}^{n}\,|c_{m}|^{2}\leq\int_{a}^{b}|f(x)|^{2}\,d y $$ 文 since JIf- 1,| ≥20 8.12 Theorem If {oA)is orthonormal on [a, b), and i then $$ f(x)\sim\sum_{n=1}^{\infty}c_{n}\,\phi_{n}(x), $$ (73) $$ \sum_{n=1}^{\infty}|c_{n}|^{2}\leq\int_{a}^{b}|f(x)|^{2}\,d x $$ In particular, (74) n→ o lim c,= 0 Proof Letting n→oo in(72), we obtain(73)、the so-called“Bessel inequality.’ 8.13 Trigonometric series From now on we shall deal only with the trigono metric system.We shall consider functions f that have period 2r and that are Riemann-integrable on [-T, r](and hence on every bounded interval). The Fourier series of $\mathbb{J}$ is then the series (63) whose coefficients c。 are given by the integrals (62), and (75) $$ s_{N}(x)=s_{N}(f;\,x)=\sum_{-N}^{N}c_{n}e^{i n x} $$soME SPECIAL FUNCroNs 189 is the Nth partial sum of the Fourier series of f The inequality (72) now takes the form (76) $$ {\frac{1}{2\pi}}\int_{-\pi}^{\pi}|s_{N}(x)|^{2}\;d x=\sum_{-N}^{N}|c_{n}|^{2}\leq{\frac{1}{2\pi}}\int_{-\pi}^{\pi}|f(x)|^{2}\;d x. $$ In order to obtain an expression for ${\mathbf{}}S_{\mathbf{}}$ that is more manageable than (75) we introduce the Dirichlet kernel (77) $$ D_{N}(x)=\sum_{n=-N}^{N}e^{i n x}=\frac{\sin\left(N+\frac{1}{2}\right)x}{\sin\left(x/2\right)}. $$ The first of these equalities is the definition of $D_{k}(x)$ ). The second follows if both sides of the identity $$ (e^{i x}-1)D_{N}(x)=e^{i(N+1)x}-e^{-i N x} $$ are multiplied by $e^{-i x/2}$ By (62) and(75), we have $$ \begin{array}{c}{{s_{N}(f;x)=\sum_{-N}^{N}\frac{1}{2\pi}\int_{-\pi}^{\pi}f(t)e^{-i n t}\,d t\,e^{i n x}}}\\ {{}}&{{=\frac{1}{2\pi}\int_{-\pi}^{\pi}f(t)\sum_{-N}^{N}e^{i n(x-t)}\,d t,}}\end{array} $$ so that (78 $$ \begin{array}{r l}{\cdot\;\;\;\;\;s_{N}(f;x)={\frac{1}{2\pi}}\int_{-\pi}^{\pi}f(t)D_{N}(x-t)\,d t={\frac{1}{2\pi}}\int_{-\pi}^{\pi}f(x-t)D_{N}(t)\,d t.}\end{array} $$ The periodicity of all functions involved shows that it is immaterial over which interval we integrate, as long as its length is 2m. This shows that the two integrals in(78) are equal. We shall prove just one theorem about the pointwise convergence of Fourier series. 8.14 Theorem I, for some x,there are constants 6> 0 and M <oo such that (79) $$ \left|{\mathcal{F}}(x+\,t)-f(x)\right|\leq\,M\, |t $$ forall te(-6, 6), hen (80) lim s(/,x)=/(x) N→0 Proof Define (81) $$ g(t)=\frac{f(x-t)-f(x)}{\sin\,(t/2)} $$190 PRINcrPLES OF MATHEMATICAL ANALYsrs for 0<lt|≤T, and put $g(0)=0$ D.By the definition (77) $$ \frac{1}{2\pi}\int_{-\pi}^{\pi}D_{N}(x)\;\;d x=1. $$ Hence(78) shows that sx( $$ \begin{array}{l}{{f\colon x)-f(x)=\displaystyle\frac{1}{2\pi}\int_{-\pi}^{\pi}g(t)\sin\left(N+\displaystyle\frac{1}{2}\right)t\,d t}}\\ {{\displaystyle=\displaystyle\frac{1}{2\pi}\int_{-\pi}^{\pi}\left[g(t)\cos\displaystyle\frac{t}{2}\right]\sin N t\,d t\,+\displaystyle\frac{1}{2\pi}\int_{-\pi}^{\pi}\left[g(t)\sin\displaystyle\frac{t}{2}\right]\cos N t\,-\displaystyle\frac{1}{2\pi}\int_{-\infty}^{\pi}\left[x\right]\cos\left[\left.\frac{t}{2}\right]\cos\right.\left({\frac{t}{2}}\right) ] . .}}\end{array} $$ dt. By(79) and (81), gt) cos t/2)and g(t) sin (t/2) are bounded. The last two integrals thus tend to O as N→OO,by (74). This proves (80) CorollaryIf f(x)= 0 for all x in some segment J, then lim s(f; $\scriptstyle x\;=\;0$ for every x∈ J. Here is another formulation of this corollary If f(t)= 9(t) for all t in some neighborhood of x, then $$ s_{N}(f;x)-s_{N}(g;\,x)=s_{N}(f-g;\,x)\to0\,a s\,N\to\infty. $$ This is usually called the localization theorem. It shows that the behavior of the sequence {sx(厂; x)} as far as convergence is concerned, depends only on the values of f in some (arbitrarily small) neighborhood of x. Two Fourie series may thus have the same behavior in one interval, but may behave in entirely different ways in some other interval. We have here a very striking contrast between Fourier series and power series (Theorem 8.5) We conclude with two other approximation theorems. 8.15 Theorem If f is continuous(with period 2z) and if& >0, then there is a trigonometric polynomial $\overline{{{\mathcal{P}}}}$ such that $$ |P(x)-f(x)|<s $$ for all real x Proot f we identify x and x+ 2m, we may regard the 2zr-periodic func- tions on $\textstyle{\mathcal{Q}}^{1}$ l as functions on the unit circle T, by means of the mapping x→ei*. The trigonometric polynomials,i.e., the functions of the form (60),form a self-adjoint algebra d,which separates points on ${\mathcal{D}}^{\mathcal{N}}$ is compact, Theorem 7.33 te ${\mathcal{J}}_{\mathbf{\Lambda}}^{\mathbf{v}}$ and which vanishes at no point of T. Since us that sd is dense in G(T). This is exactly what the theorem asserts A more precise form of this theorem appcars in Exercise 15SOME SPECIAL FUNcTIONs 191 8.16 Parseval's theoren Supposef and g are Riemann-integrable functions with period 2x, and (82) $$ f(x)\sim\sum_{-\infty}^{\infty}c_{n}e^{i n x},\qquad g(x)\sim\sum_{-\infty}^{\infty}\gamma_{n}e^{i n x}. $$ Then (85) $$ \operatorname*{lim}_{\kappa\to\infty}\frac{1}{2\pi}\int_{-\pi}^{\pi}|f(x)-s_{N}(f;x)|^{2}\;d x=0, $$ (83) (84) Proof Let us use the notation (86) $$ \|h\|_{2}= \lbrace{\frac{1}{2\pi}}\int_{-\pi}^{\pi}|h(x)|^{2}\,d x \rbrace^{1/2}. $$ Let e>0 be given. Since f∈ O and $f(\pi)=f(-\pi),$ the construction described in Exercise 12 of Chap. 6 yields a continuous 2m-periodic func- tion h with (87) $$ \|f-h\|_{2}<s. $$ By Theorem 8.15, there is a trigonometric polynomial $\bar{\mathcal{J}}$ such that h(x)- P(x)」<efor all x. Hence lh - Plz<E.If $\bar{\mathcal{D}}$ has degree $N_{0}$ Theorem 8.11 shows that (88) $$ \|h-s_{N}(h)\|_{2}\leq\|h-P\|_{2}<\varepsilon $$ for all N ≥ ${\mathcal{N}}_{0}$ .By(72), with h -fin place of f, (89) $$ \left\|s_{N}(h)-s_{N}(f)\right\|_{2}=\left\|s_{N}(h-f)\right\|_{2}\leq\left\|h-f\right\|_{2}<s. $$ Now the triangle inequality (Exercise 11, Chap. 6), combined with (87),(88), and(89), shows that (90) $$ \|f-s_{N}(f)\|_{2}<3\varepsilon\qquad(N\geq N_{0}). $$ This proves (83).Next, (91) $$ \frac{1}{2\pi}\int_{-\pi}^{\pi}s_{N}(f)\bar{g}\,d x=\sum_{-N}^{N}c_{n}\ \frac{1}{2\pi}\int_{-\pi}^{\pi}e^{i n x}\overline{{{g(x)}}}\,d x=\sum_{-N}^{N}c_{n}\,\bar{\gamma}_{n}, $$ and the Schwarz inequality shows that (92) $$ \left|\int\!f\bar{g}-\int\!s_{N}(f)\bar{g}\right|\leq\int\!\mid f-s_{N}(f)\mid\!\mid\!g\!\mid\leq\left<\int\!\mid f-s_{N}|^{2}\int\!\mid\!g\!\mid^{2}\right>^{1/2}, $$192 PRINCIPLES OF MATHEMATICAL ANALYSIs which tends to O, as N→0, by (83).Comparison of (9l) and(92) gives (84). Finally,(85) is the special case g = f of(84) A more general version of Theorem 8.16 appears in Chap. 11 THE GAMMA FUNCTION This function is closely related to factorials and crops up in many unexpected places in analysis. Its origin, history, and development are very well described in an interesting article by P J. Davis (Amer. Math. Monthly, vol. 66, 1959 pp. 849-869).Artin's book (cited in the Bibliography) is another good elemen tary introduction. Our presentation will be very condensed, with only a few comments afte each theorem. This section may thus be regarded as a large exercise, and as an opportunity to apply some of the material that has been presented so far. 8.17 Definition For O $<x<\infty$ (93) $$ \Gamma(x)={\int_{0}^{\infty}{t^{x-1}}e^{-t}}\,d t. $$ The integral converges for these x.(When x<1, both O and co have to be looked at.) 8.18 Theorem (a)The functional equation $$ \Gamma(x+1)=x\Gamma(x) $$ holds if O<x<OO. (b)T(n +1)= n! for n = 1,2,3,. (c)log $\sqrt{1}$ is convex on (0, oo) Proof An integration by parts proves a). Since T(1) = 1,(a) implies $(\mid\rlap/p)\d p)\d t\h(\mid\int\theta)$ -1, apply Hoider (b),by induction. If 1<p<Oo and inequality (Exercise 10, Chap. 6)to (93), and obtain $$ \Gamma\left(\frac{x}{p}+\frac{y}{q}\right)\leq\Gamma(x)^{1/p}\Gamma(y)^{1/q}. $$ This is equivalent to (c) $\stackrel{\Vert}{\longrightarrow}\bigcup$ is a rather surprising fat, discovered by Bohr and Mollerup、 that these three properties characterize $\textstyle{\frac{\mathrm{f}}{\mathrm{h}}}^{-}$ completely.soME SPECIAL FUNCTIONS 193 8.19 Theorem Iff is a positive function on(0,oo) such thai (a)/(x + 1)= x/(x), (b)f(1) = 1 (c)log f is convex, then J(x)= F(x) Proof Since T satisfies (a),(b), and (c),it is enough to prove that f(x) is uniquely determined by (a),(b),(c) for all x > 0. By (a), it is enough to do this for xe (0,1) Put p = log f. Then (94) $$ \varphi(x+1)=\varphi(x)+\log x\qquad(0<x<\infty), $$ cp(1) = 0, and p is convex.Suppose O<x<1, and ${\mathcal{N}}$ is a positive integer By (94),p(n + 1) = log(n!).Consider the difference quotients of p on th intervals [n,n + 1], [n + 1, n + 1 + x],[n + 1, n + 2]、Since p is convex $$ \log n\leq{\frac{\varphi(n+1+x)-\varphi(n+1)}{x}}\leq\log\left(n+1\right). $$ Repeated application of (94) gives $$ \varphi(n+1+x)=\varphi(x)+\log\,[x(x+1)\cdot\cdot\cdot(x+n)]. $$ Thus $$ 0\leq\varphi(x)-1\circ\mathrm{g}\left[\frac{n!n^{x}}{x(x+1)\cdot\cdot\cdot(x+n)}\right]\leq x\log\left(1+\frac{1}{n}\right). $$ The last expression tends to $\left(\right)\,$ as $n arrow\infty.$ .Hence p(x) is determined, and the proof is complete. As a by-product we obtain the relation (95) $$ \Gamma(x)=\operatorname*{lim}_{n\to\infty}{\frac{n!n^{x}}{x(x+1)\cdots(x+n)}} $$ since T(X $+\mid=x\Gamma(x).$ at least when 0<x<1; from this one can deduce that (95) holds for all x > 0 8.20 Theorem Ifx> $\mathbf{\nabla}(\geqslant)\mathbf{\psi}$ and $y>0,$ then (96) $$ \bigcap_{0}^{1}t^{x-1}(1-t)^{y-1}\,d t={\frac{\Gamma(x)\Gamma(y)}{\Gamma(x+y)}}. $$ This integral is the so-called beta function B(x, y)194 PRINCIPLES OF MATHEMATICAL ANALYSIs Proof Note that B(1,y) = 1/y,that log B(x, y)is a convex function of x, for each fixed y,by H6lder's inequality, as in Theorem 8.18,and that (97) $$ B(x+1,y)={\frac{x}{x+y}}\,B(x,y). $$ To prove (97), perform an integration by parts on $$ B(x+1,y)=\int_{0}^{1}\left({\frac{t}{1-t}}\right)^{x}(1-t)^{x+y-1}\,d t. $$ These three properties of $\scriptstyle D(x,y)$ show,for each y, that Theorem 8.19 applies to the function f defined by Hence f(x) = T(x) $$ f(x)={\frac{\Gamma(x+y)}{\Gamma(y)}}\,B(x,y). $$ 8.21 Some consequences The substitution $t=\sin^{2}$ O turns (96) into (98) $$ 2\int_{0}^{x/2}(\sin\theta)^{2x-1}\,(\cos\theta)^{2y-1}\,d\theta={\frac{\Gamma(x)\Gamma(y)}{\Gamma(x+y)}}. $$ The special case $x=y={\frac{1}{2}}$ gives (99 $$ \Gamma({\textstyle\frac{1}{2}})=\sqrt{\pi}. $$ The substitution $\scriptstyle t\,=\,s^{2}$ turns (93) into (100) $$ \Gamma(x)=2\int_{0}^{\infty}s^{2x-1}\,e^{-s^{2}}\,d s\qquad(0<x<\infty). $$ The special case $x=\frac{1}{2}$ gives (101) $$ \{\stackrel{\alpha0}{\sim}_{-\,\infty}^{\alpha}e^{-\,s^{2}}\,Q S=\sqrt{\pi}. $$ By (99), the identity (102) $$ \Gamma(x)={\frac{2^{x-1}}{\sqrt{\pi}}}\ \Gamma\left({\frac{x}{2}}\right)\Gamma\left({\frac{x+1}{2}}\right) $$ follows directly from Theorem 8.19 8.22Stirling's formula This provides a simple approximate expression fo T(×+ 1) when x is large hence forn! when nis are). The formula is (103) $$ \operatorname*{lim}_{x arrow\infty}\frac{\Gamma(x+1)}{(x/e)^{x}\,\sqrt{2\pi x}}=1. $$soME SPECIAL FUNCTIONs 195 Here is a proof. Put t = x(1 + u) in (93). This gives (104) $$ \Gamma(x+1)=x^{x+1}\,e^{-x}\int_{-1}^{\infty}\,[(1+u)e^{-u}]^{x}\,d u. $$ Determine hu) so that h(0) = 1 and (105) $$ (1+u)e^{-u}=\exp\left[-{\frac{u^{2}}{2}}h(u)\right] $$ if $-1<u<\infty,$ u ≠ 0. Then (106) $$ h(u)={\frac{2}{u^{2}}}\,[u-\log\left(1+u\right)]. $$ lt follows that $J_{\mathrm{f}}$ is continuous, and that h(u) decreases monotonically from oo $\complement\bigcap$ as u increases from -l to oo. to C The substitution u = 8、/2/x turns (104) into (107) $$ \Gamma(x+1)=x^{x}e^{-x}\sqrt{2x}\int_{-\infty}^{\infty}\psi_{x}(s)\,d s $$ where $$ \psi_{x}(s)=\left\{\!\!\mathrm{exp}\!\left[-s^{2}h(s\,\sqrt{2/x})\right]\right.\ \ \ \ (-\sqrt{x/2}<s< $$ oO), Note the following facts about w,(s): (a) For every s,业,(s)→e-* as x→0O. (b) The convergence in (a) is uniform on [- A,A], for every A < O. (c) When s<0, then O< W.(s)<e-” (d) When s> 0 and x>1,then 0<y,(s)<V,(s) (e)了8 W(s)ds < Oo. The convergence theorem stated in Exercise 12 of Chap. T can therefore be appied to the integral (107), and shows that this integral converges to 、/元 as x→0,by (101). This proves (103) A more detailed version of this proof may be found in R. C. Buck's “"Advanced Calculus,'pp 216-218. For two other, entirely diferent, proofs see W. Feller's article in Amer. Math. Monthly, vo1. 74,1967, pp. 1223-1225 (with a correction in vol. 75,1968, p. S18) and pp. 20-24 of Artin's book Exercise 2O gives a simpler proof of a less precise result.196 PRINCIPLES OF MATHEMATICAL ANALYSIs EXERCISES 1. Define $$ f(x)= \{_{0}^{e-1/x^{2}}\ \ \ \ \ \ (x\neq0), $$ Prove that f has derivatives of all orders at x= 0,and that f"(0) = 0 for n= 1,2,3,.… 2. Let a, be the number in the ith row and jth column of the array 一1 0 0 0 。·· 女 -1 0 0 .… 主由 -1 0 · 吾无玉 -1 so that ai」 (0 1 Gi-23 (i<j) L2/-1 ti>苏) Prove that ZZα,= -2, 王买a,=0 3. Prove that 22ay >α if a,≥0 for all iand j(the case +0 = + oo may occur) 4. Prove the following limit relations: (a) lim b*-1 = log b (b>0) x (b) lim log (1+x) =1 x (c) lim (1+x)1/* = e (d) linsoME SPECIAL FUNCTiONS 197 5. Find the following limits (a) lim e一 X -(1 + x)1* X→0 (b) lim n [n"/"- 11 *o log n (c) lim tan x一X o x(1- cos x) do i一x -o tan x一x 6. Suppose f(x)f(Oy)= f(x + y))for all real x and y. (a)Assuming that fis diferentiable and not zero, prove that $$ f(x)=e^{e x} $$ where c is a constant (b) Prove the same thing, assuming only that f is continuous T. If O $\textstyle{\bar{\lambda}}<x<{\frac{\pi}{2}},$ prove that $$ {\frac{2}{\pi}}<{\frac{\sin x}{x}}<1. $$ 8. For n = 0,1,2,.…, and x real, prove that $$ |\!\sin n x|\leq n|\!\sin x|. $$ Note that this inequality may be false for other values of n、 For instance, $$ |\sin\mathrm{\frac{~3\pi}{2\pi}|\ >\scriptstyle{\frac{~3}{k}}|\sin\pi|\,. $$ 9.(a) Put sx=1+())+ +(1/V)、 Prove that $$ \operatorname*{lim}_{N arrow\infty}\;^{(S_{N} arrow\vert_{O}\Theta\Lambda V)} $$ exists.(The limit, often denoted by y,is called Euler's constant. Its numerical value is O.5772... It is not known whether y is rational or not.) (b) Roughly how large must m be so that N = 10* satisfies sx>100? 10. Prove that Z 1/p diverges; the sum extends over all primes. (This shows that the primes form a fairly substantial subset of the positive integers.)198 PRINCIPLEs Or MATHEMATICAL ANALYs Hint: Given N, Iet p,.……,,phbe those primes that divide at least one in- teger ≤N. Then $$ \begin{array}{l}{{\frac{3}{2}}\prod_{i=1}^{5}\overbrace{p_{i}}^{1}\left(1+{\frac{1}{p_{j}}}+{\frac{1}{p_{j}^{2}}}+\cdots\right)}}\\ {{\langle\begin{array}{l}{{\frac{1}{p_{j}}}+{\frac{1}{p_{j}^{2}}}+\cdots\rangle}}\\ {{\bf\Phi=\prod_{i=1}^{k}\left(1-{\frac{1}{p_{j}}}\right)^{-1}}}\end{array} $$ 文 2 csp。, The last inequality holds because $$ (1-x)^{-1}\leq e^{2x} $$ if O≤x≤如 (There are many proofs of this result. See,for instance, the article by I. Niven in Amer. Math. Monthly, vol. 78,1971,pp.272-273, and the one by R. Bellman in Amer. Math. Monthly, vol. 50,1943, pp. 318-319.) 11. Suppose fe SR on [0, A] for all $A<\infty,$ and f(x)→1 as x→+0.Prove that $$ \operatorname*{lim}_{t\to0}t\int_{0}^{\infty}e^{-t x}f(x)\,d x=1\qquad(t>0). $$ 12. Suppose $0<\delta<\pi,f(x)=1$ if |x|≤8,f(x) = 0 if 8<|x|≤T, and f(x + 2)= f(x) for all x (a) Compute the Fourier coefficients of f (b) Conclude that $$ \begin{array}{r l r}{{\stackrel{\circ}{\operatorname{\cong}}}}&{{\sin{(n\delta)}}}&{{}}&{{}}&{{(0<\delta<\pi).}}\end{array} $$ (c) Deduce from Parseval's theorem that $$ \sum_{n=1}^{\infty}{\frac{\sin^{2}\left(n\delta\right)}{n^{2}\delta}}={\frac{\pi-\delta}{2}}\,. $$ (d) Let 8→0 and prove that $$ \bigcap_{0}^{\infty}\left({\frac{\sin x}{x}}\right)^{2}d x={\frac{\pi}{2}}. $$ 13.Put f(x)= x if (e)Put 8= /2 in (c). What do you get ? , and apply Parseval's theorem to conclude that $0\leq x<2\pi,$ $$ \sum_{n=1}^{\infty}{\frac{1}{n^{2}}}={\frac{\pi^{2}}{6}}\,. $$soMr sPECIAL FUNcroNs 199 14。If f(x) = (T一x|) on [一T, r], prove that $$ f(x)={\frac{\pi^{2}}{3}}+\sum_{n=1}^{\infty}{\frac{4}{n^{2}}}\cos n x $$ and deduce that $$ \begin{array}{c c}{{\sim_{n=1}^{\infty}\ \frac{1}{n^{2}}=\frac{\pi^{2}}{6},~~~~~~\sum_{n=1}^{\infty}\displaystyle\frac{1}{n^{4}}=\frac{\pi^{4}}{90}.}}\end{array} $$ (A recent article by E. L. Stark contains many references to series of the form >n-", where sis a positive integer. See Math. Mag., vol.47,1974, pp. 197-202. 15. With Dn as defined in (77), put $$ K_{N}(x)={\frac{1}{N+1}}\sum_{n=0}^{N}D_{n}(x). $$ Prove that $$ K_{N}(x)={\frac{1}{N+1}}\cdot{\frac{1-\cos\left(N+1\right)}{1-\cos x}} $$ Dx and that (a) K,≥0, (b) 2r Kx(x) dx = 1 (c)Kx(x)≤ 1 2 if O<8≤x≤m N+1 1- cos S If $s_{N}=s_{N}(f\colon x)$ is the Nth partial sum of the Fourier series of f, consider the arithmetic means $$ \sigma_{N}={\frac{s_{0}+s_{1}+\cdot\cdot\cdot\cdot\cdot\cdot}{N+1}}. $$ Prove that $$ \sigma_{N}(f;x)={\frac{1}{2\pi}}\int_{-\pi}^{\pi}f(x-t)K_{N}(t)\,d t, $$ and hence prove Fejér's theorem: I/ f is continuous, with period 2mr, then ox(f;x)→f(x) uniformly on [-m, m] Hint: Use properties (a),(b),Cc) to proceed as in Theorem 7.26 16. Prove a pointwise version of Fejer's theorem: If fe BR and f(x +), f(x-) exist for some x, then $$ \operatorname*{lim}_{x arrow\infty}\;\sigma_{N}(f;x)=\Phi[f(x+)+f(x-)]. $$200PRINCIPLEs Or MATHEMATICAL ANALYsrs 17.Assumefis bounded and monotonic on [一T,, T), with Fourier coefficients Cn,as given by (62) (a) Use Exercise 17 of Chap. 6 to prove that {ncn} is a bounded sequence (b) Combine(a) with Exercise 16 and with Exercise 14(e) of Chap. 3, to conclude that lim sm(f;x)= [/(x+)+/(x-)1 for every x (c)Assume only that fe J on [-m, m] and that f is monotonic in some segment (α,β)e [-rm, rl.、Prove that the conclusion of (b) holds for every xe(α, B) (This is an application of the localization theorem.) 18。Define $$ \begin{array}{l}{{f(x)=x^{3}-\sin^{2}x\tan x}}\\ {{g(x)=2x^{2}-\sin^{2}x-x\tan x.}}\end{array} $$ Find out, for each of these two functions, whether it is positive or negative for al xe (0,r/2), or whether it changes sign. Prove your answer. 19.Suppose fis a continuous function on $R^{1},f(x+2\pi)=f(x),$ and c/rr is irrational. Prove that $$ \operatorname*{lim}_{N\to\infty}{\frac{1}{N}}\sum_{n=1}^{N}f(x+n x)={\frac{1}{2\pi}}\int_{-\pi}^{\pi}f(t)\,d t $$ for every x.Hint:Do it first for f(x)= e* 20. The following simple computation yields a good approximation to Stirling's formula. For m = 1, 2,3,...,define $$ f(x)=(m+1-x)\log m+(x-m)\log\left(m+1\right) $$ if m ≤x≤m+1, and define $$ g(x)={\frac{x}{m}}-1+\log m $$ if m一如≤x<m+.Draw the graphs of fand g. Note that f(x)≤log x≤9(x) if x≥1 and that $$ \int_{1}^{n}f(x)\,d x=\log{(n!)}-\frac{1}{1}\log n>\,-\o{\lg}+\int_{1}^{n}g(x)\,d x. $$ Integrate log x over [1, n]. Conclude that $$ i_{g}<\log\,(n!)-(n+i)\log\,n+n<1 $$ for n= 2,3,4,.…….(Note:: log V2m~ 0.918... Thus $$ e^{7/8}<{\frac{n!}{(n/e)^{n}{\sqrt{n}}}}<e. $$sOME SPECIAL FUNCTONs 201 21. Let $$ L_{n}=\frac{1}{2\pi}\int_{-\pi}^{\pi} \lfloor D_{n}(t) \rfloor\,d t\qquad(n=1,2,3,\dots). $$ Prove that there exists a constant $C>0$ O such that $$ L_{n}>C\log n\qquad(n=1,\,2,\,3,\,\dots), $$ or, more precisely, that the sequence $$ \left\{L_{n}-{\frac{4}{\pi^{2}}}\log n\right\} $$ is bounded. 2.。If c is real and $-1<x<1.$ prove Newton's binomial theorem $$ (1+x)^{*}=1+\sum_{n=1}^{\infty}{\frac{\alpha(\alpha-1)\cdots(\alpha-n+1)}{n!}}x^{*}. $$ Hint: Denote the right side by f(x)、 Prove that the series converges. Prove that $$ (1+x)f^{\prime}(x)=\alpha f(x) $$ and solve this differential equation. Show also that $$ (1-x)^{-s}=\sum_{n=0}^{\infty}{\frac{\Gamma(n+\alpha)}{n!\,\Gamma(\alpha)}}\,x^{n} $$ if -1<x<1 and α >0 23. Let y be a continuously differentiable closed curve in the complex plane, with parameter interval [a,b], and assume that $\gamma(t)\neq0$ ) for every te [a, b]. Define the index of y to be $$ \operatorname{Ind}\left(\gamma\right)={\frac{1}{2\pi i}}\int_{a}^{b}{\frac{\gamma^{\prime}(t)}{\gamma(t)}}\,d t. $$ Prove that Ind (y is always an integer Hint:: There exists p on [a,b] with gp′ = ′/y, p(la)= 0.Hence y exp(-gp) is constant. Since y(a) = y(b) it follows that exp $\varphi(b)=\exp\varphi(a)=1$ ,Note that gp(b) = 2i Ind (y) Compute Ind(y) when y(t)= e",a = 0, b= 2 Explain why Ind(y) is often called the winding number of y around O 24. Let y be as in Exercise 23, and assume in addition that the range of y does no intersect the negative real axis. Prove that Ind $\scriptstyle(\gamma_{i})\;=\;0$ )、Hint: For O≤c< O0, Ind (y + c) is a continuous integer-valued function of c.Also,Ind (y + c)→0 as C→ O.202 PRINCIPLES OF MATHEMATICAL ANALYSIs 25. Suppose yi and ya are curves as in Exercise 23,and lyt)一y4(t)|<|y(t) (a ≤t≤b) Prove that Ind (yn)= Ind (ya). Hint: Puty = 2/1. Then $|1-\gamma|<1,$ hence Ind (y) = 0,by Exercise 24. Also, $$ \begin{array}{l}{{\frac{V}{V}=V_{n}^{2}-\gamma t}}\\ {{\gamma}}&{{\gamma_{n}}}\end{array} $$ 26. Let y be a closed curve in the complex plane (not necessarily differentiable) with parameter interval [O,2], such that $\gamma(t)\neq($ O for every te [0, 2]. Choose 8>0 so that ly(t)>8 for all t∈ [0,2m]. If P,and P,are trigo nometric polynomials such that |P,(t)一y(t)|<8/4 for all t e [0,2m] (their exis tence is assured by Theorem 8.15), prove that $$ {\mathrm{Ind~}}(P_{1})={\mathrm{Ind~}}(P_{2}) $$ by applying Exercise 25. Define this common value to be Ind (y) Prove that the statements of Exercises 24 and 25 hold without any differenti- ability assumption. 27. Let f be a continuous complex function defined in the complex plane. Suppose there is a positive integer n and a complex number c≠ O such that $$ \operatorname*{lim}_{|z|\to\infty}z^{-z}f(z)=c. $$ Prove that f(z)= 0 for at least one complex number z Note that this is a generalization of Theorem 8.8 Hint:Assume f(z) ≠ 0 for all z, define $$ \gamma_{r}(t)=f(r e^{t}) $$ for O≤r<0,0≤t≤2r, and prove the following statements about the curves y,: (a) Ind (yo) = 0 (b) Ind (y,)= n for all sufficiently large r. (c) Ind (y,)is a continuous function of ${\mathcal{P}}_{\it3}$ on [O,o) [In (b) and (c), use the last part of Exercise 26.1 Show that (a),(b), and (c) are contradictory, since $\scriptstyle n\;>0$ 28. Let ${\widehat{\operatorname{f}}}$ be the closed unit disc in the complex plane.(Thus $\scriptstyle z\in D$ if and only if |z|≤1.) Let g be a continuous mapping of $\widehat{\cal D}$ into the unit circle T.(Thus lg(z)| =1 for every z e D) Prove that g(z)= -z for at least one z e T Hint: For O≤r≤1,0≤t≤2m, put Y-(t)= g(reli)) and put yt) = -“y:(t).、 If g(z)≠ -z for every z e T, then yt) + -1 for every t e[0,2mI. Hence Ind (y) = 0, by Exercises 24 and 26.It follows that Ind $(\gamma_{1})=1.$ But Ind Gyo) = 0. Derive a contradiction, as in Exercise 27sOME SPECIAL FUNcrioNs 203 29. Prove that every continuous mapping f of $\widehat{\mathcal{D}}$ into ${\widehat{\operatorname{Q}}}$ has a fixed point in ${\mathcal{D}}.$ (This is the 2-dimensional case of Brouwer's fixed-point theorem.) Hint:: Assume f(z)≠ z for every z e D、Associate to each z e D the point 9(z) ∈ T which lies on the ray that starts at $f(z)$ and passes through z. Then g maps ${\widehat{\mathcal{D}}}$ into $T,g(z)=z$ if z e T, and g is continuous,because $$ g(z)=z-s(z)[f(z)-z], $$ where ${\mathfrak{s}}(z)$ is the unique nonnegative root of a certain quadratic equation whose coefficients are continuous functions of f and z. Apply Exercise 28. 30.Use Stirling's formula to prove that $$ \operatorname*{lim}_{x\to\infty}{\frac{\Gamma(x+c)}{x^{c}\Gamma(x)}}=1 $$ for every real constant c. 31. In the proof of Theorem 7.26 it was shown that $$ \textstyle\int_{-1}^{1}(1-x^{2})^{n}\,d x\geq{\frac{4}{3\,\sqrt{n}}} $$ for n=1, 2, 3,.... Use Theorem 8.20 and Exercise 30 to show the more precise result $$ \operatorname*{lim}_{n arrow\infty}\sqrt{n}\,\int_{-1}^{1}\,(1-x^{2})^{n}\,d x=\sqrt{\pi}. $$