CONTINUITY The function concept and some of the related terminology were introduced in Definitions 2.1 and 2.2. Although we shall in later chapters) be mainly interested in real and complex functions (i.e.,in functions whose values are real or complex numbers) we shall also discuss vector-valued functions (i.e.,functions with values in R') and functions with values in an arbitrary metric space. The theo- rems we shall discuss in this general setting would not become any easier if we restricted ourselves to real functions, for instance, and it actually simplifes and clarifies the picture to discard unnecessary hypotheses and to state and prove theorems in an appropriately general context. The domains of definition of our functions will aso bc metric spaces suitably specialized in various instances. LIMITS OF FUNCTIONS y, and 4. Defnition Let $X$ and ${\mathfrak{J}}^{\gamma}$ be mtric spaces; suppose ${\mathcal{H}}^{\nu}$ c X,f maps E into $\textstyle{\mathcal{D}}$ is a limit point of ${\mathcal{F}}$ . We write F(x)→g asx→p,or (1) $$ \operatorname*{lim}_{x\to y}f(x)=q $$84 PRINCIPLES OF MATHEMATICAL ANALYsIs if there is a point qe $\textstyle\nabla$ with the following property: For every s> 0 there exists a $\scriptstyle\delta>0$ such that (2) $$ d_{Y}(f(x),q)<s $$ for all points xe E for which (3) $$ 0<d_{x}(x,p)<\delta. $$ The symbols ${\mathcal{A}}_{\chi}$ and ${\mathcal{A}}_{Y}$ refer to the distances in $X$ and Y, respectively If X and/or Y are replaced by the real line, the complex plane, or by some euclidean space R', the distances dx, dy are of course replaced by absolute values, or by norms of differences (see Sec. 2.16). lt should be noted that pe X, but that p need not be a point of E in the above definition.Moreover, even if pe E, we may very well have f(p) ≠ lim,→pf(x) We can recast this definition in terms of limits of sequences: 4.2 Theorem Let X,Y, E,f, and p be as in Definition 4.1. Then (4) $$ \operatorname*{lim}_{x\to J}(x)=q $$ if and only if (5) $$ \operatorname*{lim}_{n arrow\infty}f(p_{n})=q $$ for every sequence {p,} in E such tha (6) $$ p_{n}\neq p,\\operatorname*{lim}_{n\to\infty}p_{n}=p. $$ Proof Suppose (4) holds. Choose (p in ${\widetilde{H}}^{\gamma}$ satisfying (6)、 Let s>0 be given. Then there exists > 0 such that drGfOx),g)<e if xe E and 0< dx(x,p)<6.Aso、there existsN such that n> N implies 0 <dx(Pm,p)<8. Thus,for n> N,we have dyGfOP,)q)<B,which shows that (S) holds. Conversely, suppose (4) is fals. Then there exists some s> 0 such that for every > 0 there exists a point xe E(depending on O), for which drdf(x)g) ≥2s but O< dx(x,p)< 员.Taking 6,= 1/n (n = 1, 2,3,..) we thus find a sequence in ${\widetilde{\mathcal{H}}}$ satisfying (6) for which (5) is false. Corollary Iff has a limit at p,this limit is unique This follows from Theorems 3.2(6) and 4.2.coNTnNUrr 85 4.3 Definition Suppose we have two complex functions, f and g, both defined on E.By f+ g we mean the function which assigns to each point x of E the number f(x) + 9(x).Similarly we define the difference f-9,the product fg, and the quotient flg of the two functions, with the understanding that the quo- tient is defined only at those points x of E at which g(x) ≠ 0.If f assigns to each point x of E the same number c,then fis said to be a constant function,o1 simply a constant, and we write f = c.If f and g are real functions, and i f(x)≥ 9(x)) for every x $\mathrm{e}\,E,$ we shall sometimes write f≥ g, for brevity. Similarly, if f and g map Einto $R^{k},$ we define f $\mathbf{f}+\mathbf{g}$ and f·g by $$ ({\bf f}+{\bf g})(x)={\bf f}(x)+{\bf g}(x),\qquad({\bf f}\cdot{\bf g})(x)={\bf f}(x)\cdot{\bf g}(x); $$ and if入is a real number,(入f)(x)= Af(x) 4.4 Theorem Suppose Ec X, a metric space, p is a limit point of E, f and g are complex functions on $\textstyle E,$ and lim f(x) = 4, lim g(x) = B. 女→口 x→p Then (a)lim(/ + 9)(x)= 4 + B; (6) lim (fg)(x) = AB (c)lim 秀,”*。 Proof In view of Theorem 4.2, these assertions follow immediately from the analogous properties of sequences (Theorem 3.3) Remark Iff and g map ${\overline{{\mathcal{F}}}}^{\nu}$ into R', then (a) remains true, and(b) becomes (b′)lim(f·g)(x) = A·B. (Compare Theorem 3.4.) CONTINUOUS FUNCTIONS 4.5 Definition Suppose X and Y are metric spaces, $E\subset X,\,p\in E,$ and f maps E into Y. Then f is said to be continuous at p if for everyé> 0 there exists a 8> 0 such that $$ d_{Y}(f(x),f(p))<\varepsilon $$ for all points xe ${\mathcal{H}}^{\nu}$ for which $d_{x}(x,p)<\delta.$ If fis continuous at every point of E, then fis said to be continuous on ${\mathcal{H}}^{\nu}$ It should be noted that f has to be defined at the point p in order to be continuous at p、(Compare this with the remark following Definition 4.1.)86 PRINCIPLES OF MATHEMATICAL ANALYSIs 1f $\mathcal{P}$ is an isolated point of E,then our definition implies that every function $\mathbb{Z}$ which has $\widehat{F}^{\dagger}$ as its domain of definition is continuous at p. For, no matte which e> O we choose, we can pick $\delta>0$ so that the only point xe E for which dx(x,p)<8is x = p; then $$ d_{Y}(f(x),f(p))=0<\varepsilon $$ 4.6 Theorem In the situation given in Definition 4.5, assume also that p is a limit point of E. Then f is continuous at pif and only if lim-,/(x) =/(p) Proof This is clear if we compare Definitions 4.1 and 4.5. We now turn to compositions of functions. A brief statement of th continuous. following theorem is that a continuous function of a continuous function is maps the range o/ 代,J(E), into Z, and 4.7 Theorem Suppose X, y, Z are metric spaces, E e X,J maops E into Y is he mapping oO $\textstyle{\mathcal{F}}$ into Z deined by ${\overline{{\cal J}}}^{\gamma}{}_{,{\overline{{{\cal J}}}}}^{\gamma}{}_{,{\overline{{{\cal J}}}}}^{\gamma}{}_{,{\overline{{{\cal J}}}}}^{\gamma}{}_{,{\overline{{{\cal J}}}}}^{\gamma}{}_{,{\overline{{{\cal J}}}}^{\gamma}}{}_{,{\overline{{{\cal J}}}}^{\gamma}}{}_{,{\overline{{{\cal J}}}}^{\gamma}}{}_{,{\overline{{{\cal J}}}}^{\gamma}}{}_{,{\overline{{{\cal J}}}^{\gamma}}}{}_{,{\overline{{{\cal J}}}}^{\gamma}{}^{\gamma}{}_{,{\overline{{{\cal J}}}}^{\gamma}}{}^{\gamma}{}_{,{\overline{{{\gamma}}}}},$ $J_{\mathit{l}}$ $$ h(x)=g(f(x))\qquad(x\in E). $$ If fis continuous at a point $\rho\in E$ E and if g is continuous at the point f(p), then hi continuous at p This function his called the composition or the composite of f and g.The notation $$ h=g\circ f $$ is frequently used in this context n > O such that Proof Let e > 0 be given. Since g is continuous at Jp), there exist $$ d_{z}(g(y),g(f(p)))<\varepsilon\,\mathrm{if}\,d_{Y}(y,f(p))<\eta\,\mathrm{and}\,y\in f(E). $$ Since is continuous at p,there exists $\scriptstyle\delta>0$ such that $$ d_{Y}(f(x),f(p))<\eta{\mathrm{~if~}}d_{X}(x,p)<\delta{\mathrm{~and~}}x\in E. $$ It follows that $$ d_{z}(h(x),\,h(p))=d_{z}(g(f(x)),\,g(f(p)))<\varepsilon $$ if ${\mathcal{A}}_{X}(x,p)<\delta$ and xe E. Thus h is continuous at $\textstyle{\mathcal{J}}$ 上 4.8 Theorem A mapping f of a metric space $X$ into a metric space Y is con- tinvous on Xif and onlyif f (V)is open in X for every open set ${\big.}{\big/}^{\prime}$ in Y (Inverse images are defined in Definition 2.2.)This is a very useful charac terization of continuity.cONrINUrr 87 Proof Suppose fis continuous on X and V is an open set in Y. We have to show that every point of f"(V)is an interior point of f"1(V)、So suppose pe X and f(p)e V. Since V is open,there exists e > such that y e V if dyCf(p), y)<8; and since f is continuous at p, there exists $\delta>0$ such that dy(J(x), f(p))< &if dx(x,p)<6. Thus xef-1(V)as soon as dx(x, p)<6. for every open set V in Y. Conversely, suppose f-1(V)is open in $X$ Fix p ∈ X ande> 0, let V be the set of all y ∈ ${\mathfrak{J}}^{\prime}$ such that dy(y,f(p)))<8. Then Vis open; hence f"1(V)is open; hence there exists > 0 such that xef-1(V )as soon as dx(p,x)<6. But if xe f-1(V), then f(x)e V, so that drdf(x), f(p))<8. This completes the proof. CorollaryA mapping f of a metric space X into a metric space Y is continuous if and only if f"(C)is closed in X for every closed set C in Y This follows from the theorem, since a set is closed if and only if its com- plement is open,and since f"1(E')= [f1(E)]° for every Ec Y We now turn to complex-valued and vector-valued functions,and to functions defined on subsets of R*. 4.9Theorem Let f and g be complex continuous functions on a metric space X Then f + 9, fg,and flg are continuous on X. In the last case, we must of course assume that g(x)≠ 0, for all xe X. Proof At isolated points of X there is nothing to prove. At limit points the statement follows from Theorems 4.4 and 4.6. 4.10 Theorem (a)Let f,...,f be real functions on a metric space X, and let f be the mapping of X into $\textstyle{\mathcal{R}}^{k}$ defined by (7) $$ \mathbf{f}(x)=(f_{1}(x),\cdot\cdot\cdot,f_{k}(x))\qquad(x\in X); $$ thenfis continuous if and only if each of the functions f,... , is continuous (b)If f and g are continuous mappings of X into R', then f + g and f·g are continuous on X. The functions f, .,f, are called the components of f. Note that f + g is a mapping into Rt, Whereas t- g is a real function on X88 rRINCIPLEs OF MATHEMATCAL ANALYSis Proof Part (a) follows from the inequalities $$ f_{j}(x)-f_{j}(y)|\leq|{\bf f}(x)-{\bf f}(y)|\ =\left\{\sum_{i=1}^{k}|f_{i}(x)-f_{i}(y)|^{2}\right\}^{\frac{1}{4}}, $$ for j =1,...,k.Part (b) follows from(a) and Theorem 4.9 4.11 Examples If x, .…,X。 are the coordinates of the point xe R',the functions $\emptyset_{i}$ defined by (8) $$ \phi_{i}(\mathbf{x})=x_{i}\qquad(\mathbf{x}\in R^{k}) $$ are continuous on $R^{k}{}_{i}$ *,since the inequality $$ |\phi_{i}({\bf x})-\phi_{i}({\bf y})|\leq|{\bf x}-{\bf y}| $$ shows that we may take E $\delta=\varepsilon$ in Definition 4.5. The functions p, are sometimes called the coordinate functions. Repeated application of Theorem 4.9 then shows that every monomial (9) x{x3 ... X" where n,...,n, are nonnegative integers, is continuous on $\textstyle{\mathcal{R}}^{k}$ .The same is true of constant multiples of (9),since constants are evidently continuous. It follows that every polynomial ${\mathcal{P}},$ given by (10) $$ P({\bf x})=\Sigma c_{n_{1}}..._{m_{k}}x_{1}^{n_{1}}\cdot\cdot\cdot x_{k}^{n_{k}}\qquad({\bf x}\in R^{k}), $$ is continuous on $\textstyle{\mathcal{R}}^{k}$ .Here the coefficients c n,are complex numbers. $\mathcal{V}\mathcal{V}_{\perp}\mathbf{\varepsilon}\circ\mathbf{\varepsilon}\circ\mathbf{\varepsilon}\phantom{\gamma}\mathcal{U}_{\cal A}$ are nonnegative integers, and the sum in(10) has finitely many terms. Furthermore, every rational function in x,.… ,Xx,that is, every quotient of two polynomials of the form (10), is continuous on $\textstyle{\mathcal{R}}^{k}$ wherever the denomi- nator is different from zero. From the triangle inequality one sees easily that (11) $$ \mid\mathbf{x}|\,-\,\vert\mathbf{y}\vert\,\mid\,\leq\,\vert\mathbf{x}-\mathbf{y}\vert\quad(\mathbf{x},\,\mathbf{y}\in{\boldsymbol{R}}^{k}). $$ Hence the mapping X→|x| is a continuous real function on $\textstyle{\mathcal{R}}^{k}$ If now fis a continuous mapping from a metric space X into R*, and if p is defined on X by setting d(p) = |f(p)|, it follows, by Theorem 4.7, that $\hat{\phi}$ is a continuous real function on X. 4.12 Remark We defined the notion of continuity for functions defined on a subset E of a metric space X. However,the complement of ${\widetilde{H}}$ in X plays no role whatever in this definition (note that the situation was somewhat different for limits of functions).Accordingly, we lose nothing of interest by discarding the complement of the domain of f. This means that we may just as well talk only about continuous mappings of one metric space into another, rather thancOrrNurr89 of mappings of subsets. This simplifies statements and proofs of some theorems We have already made use of this principle in Theorems 4.8 to 4.10, and wil continue to do so in the following section on compactness. CONTINUITY AND COMPACTNESS 4.13 Definition A mapping f of a set ${\widetilde{F}}^{\dagger}$ into $\textstyle{\mathcal{R}}^{k}$ is said to be bounded if there is a real number M such that |f(x)」≤ M for all $x\in L$ 4.14 Theorem Supose fis a continuous mapping of a compact metric space X into a metric space Y.Then f(X) is compact Proof Let{V)}be an open cover of (X). Since fis continuous, Theorem 4.8 shows that each of the sets -(V)is open. Since X is compact, there are finitely many indices, say a, …,,,such tha (12) $$ X\subset f^{-1}(V_{\alpha_{1}})\cup\cdot\cdot\cdot\cup f^{-1}(V_{\alpha_{n}}). $$ Since f(f"(E)) E for every Ee Y,(12) implies that (13) $$ f(X)\subset V_{\alpha_{1}}\cup\cdots\cup V_{\alpha_{n}}. $$ This completes the proof. Note: We have used the relation f(f-(E))C E, valid for $\scriptstyle E\in Y$ 1f Ec X, then f-1(f(E))→ E;equality need not hold in either case. We shall now deduce some consequences of Theorem 4.14. 4.15 Theorem If fis a continuous mapping of a compact metric space $X$ into R', then f(X) is closed and bounded.Thus, is bounded This follows from Theorem 2.41. The result is particularly important when fis real: 4.16 Thecorem Suppose is a continwous real fumnction on a compact metric space X, and (14) M = sup f(p), m = inf J(p) p e p e X Then the exist points p,g e $X$ such that /(p) = M and J(q) = m of this set of numbers. The notation in(14) means that M is the least upper bound of the set o all numbers (p), where p ranges over x,and that m is the greatest lower boun90 PRINCIPLEs OF MATHEMATICAL ANALYsi The conclusion may also be stated as follows: There exist points p and g (at p) and its minimum (at q) in X such that J(q) ≤f(x) ≤f(p) for all xe X; that is,f attains its maximum Proof By Theorem 4.15, f(X)is a closed and bounded set of real num- bers; hence f(X) contains $$ M=\operatorname*{sup}f(X)\qquad{\mathrm{and}}\qquad m=\operatorname*{inf}f(X), $$ 。 by Theorem 2.28 4.17 Theorem Suppose f is a continuous -l mapping of a compact metric 1 defined on Y by space X onto a meric space Y.Then the imverse mapping 广 $$ f^{-1}(f(x))=x\qquad(x\in X) $$ is a continuous mapping of Y onto X Proof Applying Theorem 4.8 to f" in place of f, we see that i uffice to prove that f(V)is an open set in Y for every open set V in X. Fix such a set V. The complement V° of ${\mathfrak{I}}^{\prime}$ is closed in X, hence compact (Theorem 2.35); hence f(V')is a compact subset of Y(Theorem 4.4)and so is closed in Y(Theorem 2.34). Since fis one-to-one and onto, f(V)is the complement of f(V"). Hence f(V)is open. 4.18 Definition Let f be a mapping of a metric space Xinto a mtric space $\textstyle\iint_{0}^{r}$ We say that fis uniformly coninuows on Xif for every s> 0 there exists6 > 0 such that (15) $$ d_{Y}(f(p),f(q))<\varepsilon $$ for all p and g in X for which $d_{X}(p,q)<\delta$ Let us consider the diferences between the concepts of continuity and of uniform continuity. Firs, uniform continuity is a property of a function on a set, whereas continuity can be defined at a single point. To ask whether a given function is uniformly continuous at a certain point is meaningless. Second, if $\oint_{0}^{x}$ is continuous on X,then it is possible to find, for eachc> 0 and for each point p of X, a number > O having the property specified in Definition 4.5. This depends on s and on p.Iffis, however, uniformly continuous on X, then it is possible, for each e> 0,to find one numberS> 0 which will do for all points p of X Evidently,every uniformly continuous function is continuous. That the two concepts are equivalent on compact sets follows from the next theoremCONTINUITr91 4.19 Theorem Let f be a coninuous mapping of a compact metric space X into a metric space Y. Then f is uniformly continuous on $X$ Proof Let > 0 be given.Since fis continuous,we can associate to each point p ∈ Xa positive number $\phi(p)$ such that (16) $$ q\in X,\,d_{x}(p,\,q)<\phi(p)\quad{\mathrm{implies}}\quad d_{Y}(f(p),\,f(q))<{\frac{\varepsilon}{2}}. $$ Let J(p) be the set of all g ∈ X for which (17) $$ d_{X}(p,q)<{\frac{1}{2}}\phi(p). $$ Since p ∈ .J(p),the collection of all sets ${\mathcal{I}}(p)$ is an open cover of X; and since X is compact, there is a finite set of points $p_{1},$ ...,P,in X, such that (18) $$ X\subset J(p_{1})\cup\cdot\cdot\cdot\cup J(p_{n}). $$ We put (19) $$ \delta={\frac{1}{2}}\;\mathrm{min}\;[\phi(p_{1}),\,\cdot\cdot,\,\phi(p_{n})]. $$ Then6> 0.(This is one point where the finiteness of the covering,in herent in the definition of compactness,is essential. The minimum of a finite set of positive numbers is positive, whereas the inf of an infinite set of positive numbers may very well be O.) Now let g and p be points of X, such that $d_{x}(p,q)<\delta$ ,By(18), there is an integer m,1 ≤m ≤n,such that $p\in J(p_{m});$ hence (20) $$ d_{x}(p,p_{m})<\textstyle{\frac{1}{2}}\phi(p_{m}), $$ and we also have $$ d_{x}(q,p_{m})\le d_{x}(p,q)+d_{x}(p,p_{m})<\delta+\textstyle{\frac{1}{2}}\phi(p_{m})\le\phi(p_{m}). $$ Finally,(16) shows that therefore $$ d_{Y}(f(p),f(q))\leq d_{Y}(f(p),f(p_{m}))+d_{Y}(f(q),f(p_{m}))<\varepsilon. $$ This completes the proof. An alternative proof is sketched in Exercise 10 We now proceed to show that compactness is essential in the hypotheses of Theorems 4.14,4.15,4.16,and 4.19. 4.20 Theorem Let $\widehat{H^{\tau}}$ be a noncompact set in $\textstyle{\mathcal{R}}^{1}$ . Then (a)there exists a continuous function on ${\widehat{\operatorname{P}}}^{\prime}$ which is not bounded (b)there exists a continuous and bounded function on E which has no maximum 1/, in addition,E is bounded, then92 PRINCIPLES OF MATHEMATICAL ANALYSIS (c)there exists a continuous function on ${\widetilde{\mathcal{H}}}^{\nu}$ which is not uniformly continuous. Proof Suppose first that $\widehat{\mathcal{D}}$ is bounded,so that there exists a limit point Xo oOf E which is not a point of ${\widehat{\operatorname{{\mathcal{H}}}^{*}}}$ . Consider (21) $$ f(x)={\frac{1}{x-x_{o}}}\qquad(x\in E). $$ This is continuous on E(Theorem 4.9),but evidently unbounded. To se that (21) is not uniformly continuous, let B> 0 and 6> 0 be arbitrary, and choose a point xe E such that |x- Xo|<员.Taking t close enough to Xo,we can then make the diffrence |f(t)-/(x)| greater than e, although |t-x<6. Since this is true for every > 0,fis not uniformly continu- ous on ${\widehat{\mathcal{H}}}^{\nu}$ The function g given by (22) $$ g(x)={\frac{1}{1+(x-x_{0})^{2}}}\qquad(x\in E) $$ is continuous on $\textstyle E,$ and is bounded, since $\scriptstyle0\,<\,g(x)\,<\,1$ .It is clear that $$ \operatorname*{sup}_{x\neq E}g(x)=1, $$ whereas g(x)<1 for all xe E. Thus $\mathcal{J}$ has no maximum on ${\overline{{\mathcal{P}}}}$ Having proved the theorem for bounded sets E, let us now suppose that ${\widehat{\mathcal{H}}}^{\mathcal{+}}$ is unbounded. Then f(x) = x establishes (a), whereas (23) $$ h(x)={\frac{x^{2}}{1+x^{2}}}\qquad(x\in E) $$ establishes (b), since $$ \operatorname*{sup}_{x\neq E}h(x)=1 $$ and A(x)<1 for all xe E hypotheses. For, let Assertion (c) would be false if boundedness were omitted from the ${\widetilde{\mathcal{H}}}^{\nu}$ be the set of all integers. Then every function defined on ${\tilde{\mathcal{H}}}^{\gamma}$ is uniformly continuous on E. To see this, we need merely take员<lin Definition 4.18. We conclude this section by showing that compactness is also essential in Theorem 4.17.coNrINUIr 93 4.21 Example Let X be the half-open interval [0,2r) on the real line, and let f be the mapping of X onto the circle Y consisting of all points whose distance from the origin is 1,given by (24) $$ \mathbf{f}(t)=(\cos t,\sin t)\qquad(0\leq t<2\pi). $$ The continuity of the trigonometric functions cosine and sine, as well as thei periodicity properties,will be established in Chap. 8. These results show that f is a continuous 1-1 mapping of X onto Y. However,the inverse mapping(which exists,since f is one-to-one and onto) fails to be continuous at the point (1,0) = (O). Of course,X is not com pact in this example.(t may be of interest to observe that f- fails to be continuous in spite of the fact that Yis compact!) CONTINUITY AND CONNECTEDNESS 4.22 Theorem I/ Jis a continuous mapping of a metric space X into a metric space Y, and if E is a connected subset of X, then f(E)is connected. Proof Assume,on the contrary, that f(E)= A U B, where A and ${\mathcal{L}}$ are nonempty separated subsets of Y.Pu $\mathrm{tr}\,G=E\,\cap\,f^{-1}(A),\,H=E\,\cap\,f^{-1}(B)$ Then E = G u H, and neither G nor H is empty Since A C A(the closure of A), we have G cf-1(A);the latter set is closed, since fis continuous; hence $G\subset f^{-1}({\bar{A}}).$ It follows that f(G) c A Since $f(H)=B$ and A n B is empty, we conclude that G o H is empty are The same argument shows that Go His empty. Thus G and $\textstyle H$ separated. This is impossible if $\widehat{H}$ is connected. 4.23 Theorem Let fbe a continuous real function on the interval [a,b1.、1 f(a)<f(b) and if c is a number such that f(a) <c<f(b),then there exists a point xe (a,b) such that f(x) = C A similar result holds,of course,if f(a) > /(b). Roughly speaking,the theorem says that a continuous real function assumes all intermediate values on an interval Proof By Theorem 2.47,[a, b]is connected; hence Theorem 4.22 shows that f([a,b]) is a connected subset of $R^{1}.$ , and the assertion follows if we appeal once more to Theorem 2.47. 4.24 Remark At first glance, it might seem that Theorem 4.23 has a converse must be continuous That s, one might think that if for any two points x,<xz and for any number c between f(x) and J(x>) there is a point xin (x, xz) such that /(x) = c, then That this is not so may be concluded from Example 4.27(d)94 PRNCIPLEs OF MATHEMATICAL ANALYSis DIScONTINUITIES If xis a point in the domain of definition of the function f at whichfis not continuous, we say that fis discontinuous at x, or that f has a discontinuity at x If fis defined on an interval or on a segment, it is customary to divide discon tinuities into two types.Before giving this classification, we have to define the right-hand and the lefi-hand limits of f at x, which we denote by f(x+)and f(x-) respectively. 4.25 Definition Let f be defined on(a,b). Consider any point x such that a ≤x<b. We write $$ f(x+)=q $$ if f(t,)→q as n→0O, for all sequences {t,) in (x, b) such that t,→X.To obtain the definition of f(x一), for $a<x\leq b,$ we restrict ourselves to sequences $\{t_{n}\}$ in (a, x) It is clear that any point x of (a,b), lim f(t) exists if and only if $$ f(x+\ )=f(x-\ )=\operatorname*{lim}_{t\to x}f(t). $$ 4.26 Definition Let f be defined on (a,b). Iffis discontinuous at a point x, and if f(x+) and f(x一) exist, then fis said to have a discontinuity of the first kind, or a simple discontinuity, at x.Otherwise the discontinuity is said to be of the second kind. There are two ways in which a function can have a simple discontinuity either f(x+)≠./(x一)[in which case the value f(x) is immaterial], or f(x+) = f(x-)≠f(x). 4.27 Examples (a)Define $$ f(x)=\left\{0\right.\ \ \ \ (x\ {\mathrm{rational}}), $$ Then f has a discontinuity of the second kind at every point x,since neither f(x+)nor f(x-)exists. (b)Define $$ f(x)={ \{0\atop(x{\mathrm{~rational}}),} $$cONTINUITY 95 Then f is continuous at x= 0 and has a discontinuity of the second kind at every other point. (c)Define $$ f(x)={ \langle-x-2{\quad}\begin{array}{l l}{{\ \ (-3<x<-2)}}\\ {{-x-2}}&{{\ \ (-2\leq x<0),}}\\ {{x+2}}&{{\ \ \ (0\leq x<1).}}\end{array} .} $$ Then f has a simple discontinuity at $\scriptstyle x\;=\;0$ and is continuous at every other point of(一3,1). (d)Define $$ f(x)={ \{\sin{\frac{1}{x}}\quad\quad(x\neq0),} $$ Since neither f(0+) nor f(0-) exists, $\widehat{{\mathcal J}}$ has a discontinuity of the second kind at x=0.We have not yet shown that sin xis a continuous function. If we assume this result for the moment, Theorem 4.7 implies that fis continuous at every point x≠ 0. MONOTONIC FUNCTIONS We shall now study those functions which never decrease (or never increase) on a given segment. 4.28 Definition Let f be real on(a,b). Then fis said to be monotonicall increasing on(a,b)if $a<x<y<b$ implies f(x) ≤f(y). If the last inequality is reversed, we obtain the definition of a monotonically decreasing function. The class of monotonic functions consists of both the increasing and the decreasing functions 4.29 Theorem Let f be monotonically increasing on (α,b)、Then f(x+) and f(x-) exist at every point ofxof(a,b)、More precisely, (25) $$ \operatorname*{sup}_{a<t<x}f(t)=f(x-)\leq f(x)\leq f(x+)=\operatorname*{inf}_{x<t<b}f(t) $$ Furthermore,if a<X<y<b,then (26) $$ f(x+)\leq f(y-). $$ Analogous results evidently hold for monotonically decreasing functions.96 raINCIPLEs or MATHEMATICAL ANALYSis A = f(x一)、 Proo' By hypothesis, theset of numbers.f(t), where a <1<x,is bounded above by the number f(x), and therefore has a least upper bound which we shall denote by A. Evidently A ≤/(x).We have to show that Let e > O be given. It follows from the definition of A as a least upper bound that there exists > 0 such that $a<x-\delta<x$ and (27) $$ A-s<f(x-\delta)\leq A. $$ Since f is monotonic, we have (28) $$ f(x-\delta)\leq f(t)\leq A\qquad(x-\delta<t<x). $$ Combining (27) and (28), we see that $$ |f(t)-A|<s\qquad(x-\delta<t<x). $$ Hence f(x-)= A Next, if The second half of (25) is proved in precisely the same way. $a<x<y<b$ ,we see from (25) that (29) $$ f(x+)=\operatorname*{inf}_{x<t<b}f(t)=\operatorname*{inf}_{x<t<J}f(t). $$ Similarly, The last equality is obtained by applying(25) to (a,J) in place of (α,b) (30) $$ f(y-)=\operatorname*{sup}_{a<t<y}f(t)=\operatorname*{sup}_{x<t<y}f(t). $$ Comparison of (29) and (30) gives (26) Corollary Monotonic functions have no discontinuities of the second kind. This corollary implies that every monotonic function is discontinuous a a countable set of points at most. Instead of appealing to the general theorem whose pof is sketched in Exercise I7,we give here a simple proof which is applicable to monotonic functions. 4.30 Theorem Let f be monotonic on (a,b)、Then the set of poins of (a,b)a1 which fis discontinuous is at most countable. let Proof Suppose, for the sake of definiteness, that f is increasing, and $\widehat{H}$ be the set of points at which fis discontinuous. With every point x of $\widehat{H}$ we associate a rational number r(x) such that $$ f(x-)<r(x)<f(x+). $$coNrINUIr97 Since x<x。implies f(x1+)≤/(xz-)、 we see that r(x) f r(x2)if X、≠X We have thus established a 1-1 correspondence between the set $\textstyle{\bar{H}}^{\prime}$ and a subset of the set of rational numbers. The latter, as we know,is count able. 4.31 Remark It should be noted that the discontinuities of a monotonic function need not be isolated. In fact, given any countable subset E of(α,b) which may even be dense,we can construct a function f, monotonic on (a,b) discontinuous at every point of I $\textstyle E,$ and at no other point of(a,b). To show this,let the points of $\textstyle{\hat{H}}^{\nu}$ be arranged in a sequence{x,}, n = 1, 2、3,.… Let {c,} be a sequence of positive numbers such that Zc, converges. Define (31) $$ f(x)=\sum_{x_{n}<x}c_{n}\qquad(a<x<b). $$ The summation is to be understood as follows: Sum over those indices r for which $x_{n}<x.$ If there are no points x,to the left of x, the sum is empty; following the usual convention, we define it to be zero. Since(31) converges absolutely,the order in which the terms are arranged is immaterial. We leave the verification of the following properties of f to the reader: (a)fis monotonically increasing on (a, b); (b)f is discontinuous at every point of $\textstyle E$ ; in fact $$ f(x_{n}+)-f(x_{n}-)=c_{n}. $$ (c)f is continuous at every other point of (a,b) Moreover, it is not hard to see that f(x-)=/(x) at all points of (a,b)、If a function satisfies this condition, we say that f is continuous from the left. I the summation in (31) were taken over all indices n for which $X_{n}$ ≤x, we would have f(x+) =/(x) at every point of (a,b); that is, f would be continuous fron the right. Functions of this sort can also be defined by another method;for an example we refer to Theorem 6.16. INFINITE LIMITS AND LIMITS AT INFINITY To enable us to operate in the extended real number system, we shall now enlarge the scope of Definition 4., by reformulating itin terms of neighborhoods. For any real number ${\mathfrak{X}}_{\mathfrak{Y}}$ we have already defined a neighborhood of x to be any segment $(x-\delta,\,x+\delta).$98 PRINCIPLES OF MATHEMLATICAL ANALYSIS 4.32 Definiton For any real c, the set of real numbers x such that x> ci calld a neighborhood of + o and is written (Cc,+O). Similarly, the set(-0,C is a neighborhood of -. 4.33Detfnition Let f be a real function defined on EC R. We say that f(t)→ A as t→X,, where A and xare in the extended real number system, if for every neighborhood $\textstyle\zeta\bar{\zeta}$ of A there is a neighborhood V of x such that $\scriptstyle V\cdot J\cdot E$ is not empty, and such that f(t)e U for all te Vo E,1 ≠ X. A moment's consideration will show that this coincides with Definition 4.1 when A and x are real The analogue of Theorem 4.4 is still true, and the proof offers nothing new. We state it, for the sake of completeness 4.34 Theorem Let f and g be defined on E C R. Suppose Then f(t)→ A, g(t)→B as t→x. (a)f(t)→A’implies $\scriptstyle A\sp\,=\,A.$ (6( + 9)(t)→A+B (c)(fgXt)→ AB (d)(f/9)(1)→A/B, provided the right members of (b),(c), and(d) are defned Note that oo - 00, 0·00, olo0,AJO are not defined (see Definition 1.23) EXERCISES 1. Suppose fis a real function defined on $\textstyle{\mathcal{R}}^{1}$ which satisfies $$ \operatorname*{lim}_{h\to0}\left[f(x+h)-f(x-h)\right]=0 $$ for every xe R. Does this imply that fis continuous? 2. If fis a continuous mapping ofa metric space Xinto a metric space ${\textstyle\bigwedge}^{\gamma}\!_{3}$ prove tha $$ f(E)\subset{\overline{{f(E)}}} $$ for every set Ec X. (E denotes the closure of E)) Show, by an example, that $f({\mathcal{E}})$ can be a proper subset of f(E) 3. Let fbe a continuous real function on a metric space X. Let Z(/)(the zero set of /) be the set of all p e X at which f(p) = 0.Prove that Z(f)is closed. 4. Let f and g be continuous mappings of a metric space $X$ into a metric space YcONTINUTTr99 and let E be a dense subset of X. Prove that F(EB)is dense in f(X). If g(p) =/(p) for all pe E, prove that g(p) =/(p) for all pe X. (In other words, a continuous mapping is determined by its values on a dense subset of its domain.) s. If fis a real continuous function defined on a closed set Ec R', prove that therc exist continuous real functions g on $\textstyle R^{1}$ such that g(x) =/(x) for all xe E.(Such functions g are called continuous extensions of f from E to RI.))Show that the result becomes false if the word “closed” is omitted.Extend the result to vector- valued functions. Hint: Let the graph of g be a straight line on each of the seg- ments which constitute the complement of E(compare Exercise 29, Chap.2D. The result remains true if Rl is replaced by any metric space, but the proof is not so simple. 6、Ifis defined on E,the graph of fis the set of points Kx, f(x)) for xe E. In partic- ular, if E is a se of real numbers, and fis real-valued, the graph of fis a subset of the plane. Suppose $\widehat{\overline{{\mathcal{H}}}}$ is compact, and prove tha is continuous on $\widehat{\cal H}^{\nu}$ if and only i its graph is compact. 7。If $\underline{{\mathbb{Z}}}$ c Xand if fis a function defined on X, the restriction of fto $\widehat{\overline{{\cal H}}}^{\nu}$ is the function g whose domain of definition is E, such that g(p) =/(p) for pe E. Define f and g on $R^{2}$ by:f(0,0) = 9(0,0) = 0,f(x, ) = xy-/(x* + y")、g9(x, y) = xy-/(x +y°) if (x,y) = (0, 0).、 Prove that $\mathbb{S}$ is bounded on $\textstyle R^{2},$ that ${\mathcal{O}}$ is unbounded in every neighborhood of (0,0), and that fis not continuous at (0,0); nevertheless, the restrictions of both f and g to every straight line in $R^{2}$ are continuous! 8. Let f be a real uniformly continuous function on the bounded set $\widehat{\cal H}^{\nu}$ in R'.Prove that fis bounded on E Show that the conclusion is false if boundedness of $\underline{{\iiint}}$ is omited from the hypothesis 9. Show that the requirement in the definition of uniform continuity can be rephrased as follows, in terms of diameters of sets: To every é>0 there exists a 8>0 such that diam f(E)<efor all Ec X with diam E<8. 10. Complete the details of the following alternative proof of Theorem 4.19: If fis not uniformly continuous, then for some &>O there are sequences {pm}, {q} in ${\mathcal{N}}$ such that dx(pn, g))→O but d,(fOp.) (q,.)>。:. Use Theorem 2.37 to obtain a contra- diction. 11.Suppose fis a uniformly continuous mapping of a metric space ${\mathcal{X}}$ into a metric space Y and prove that {f(x)} is a Cauchy sequence in Y for every Cauchy se- quence {x}in X. Use this result to give an alternative proof of the theorem stated in Exercise 13. 12.A uniformly continuous function of a uniformly continuous function is uniformly continuous. State this more preciscly and prove it 13. Let $\overline{{\theta^{4}}}$ Z be a dense subset of a metric space X, and let f be a uniformly continuous real function defined on E. Prove that f has a continuous extension from E to X100 rnINcTPLEs Or MATHEMLATICAL ANALxsis (see Exercise S for terminology).(Uniqueness follows from Exercise 4.) Hint: For each pe X and each positive integer n,Iet VA(p) be the set of all qe E with d(p,q)<1/n、 Use Exercise 9 to show that the intersection of the closures of the sets f(Vip), f(VA(p)), …,consists of a single point, say g(p),of R. Prove that the function g so defined on Xis the desired extension of 天 Could the range space $\textstyle R^{1}$ be replaced by R*? By any compact metric space? R1 By any complete metric space? By any metric space? 14. Let 1= [0,1] be the closed unit interval. Suppose fis a continuous mapping of I into I. Prove that f(x)=x for at least one xeI. 15. Call mapping of X into Yopenit OP is an open set in Y whenever Vis an open set in X. Prove that every continuous open mapping of $\textstyle R^{1}$ into $\textstyle{\mathcal{R}}^{1}$ is monotonic. 16. Let [xy denote the largest integer contained in x, that is,[xJis the integer such that x-1<[x]≤x; and let (x) =x- [x] denote the fractional part of x.What discontinuities do the functions [x] and (x) have? 17. Let fbe a real function defined on(α,b).Prove that the set of points at which 广 has a simple discontinuity is at most countable. Hint: Let E be the set on which f(x-)<f(x+). With each point x of E, associate a triple(p,q,r) of rationa numbers such that (a) f(x-)<p<f(x十) (b)a<9<1<ximplics f(t)<p, (c)x<1<r<b implies f(t)>P The set of all such triples is countable. Show that each triple is associated with at most one point of E. Deal similarly with the other possible types of simple dis- continuities 18.。Every rational x can be written in the form $x=r n/n,$ where n >0, and m and n are integers without any common divisors. When x =0, we take n =1. Consider the function f defined on $\textstyle R^{1}$ by $$ f(x)={\Bigg\{}{\frac{0}{n}}\quad(x\ {\mathrm{iratonal}}, $$ ) Prove that fis continuous at every irrational point, and that f has a simple discon- tinuity at every rational point. 19. Suppose fis a real function with domain Rl which has the intermediate valuc property: If f(a)<c<f(b), then f(x) = c for some x between a and b. Suppose also, for every rational r, that the set of allx with f(x) =ris closed Prove that f is continuous. Hint: If $x_{n}\to x_{0}$ but F(x)>r>/(xo) for somer and all n, then f(t) = for some t,between xo and xn;thus t。→Xo,Find a contradiction. (N. J, Fine, Amer. Math. Monthly, vo1.73,1966, p.782.)coNTINUrr101 20. If Eis a nonempty subset of a metric space X, define the distance from xe X to E by $$ \rho_{E}(x)=\operatorname*{inf}_{x:E}d(x,z). $$ (a) Prove that px(x) = 0 if and only if xe E (b) Prove that pz is a uniformly continuous function on X,by showing that $$ \left|\rho_{E}(x)-\rho_{E}(y)\right|\leq d(x,y) $$ for all xe X,y∈ X. Hint: pA(x)≤dx, z)≤d(x, y) +dOy, z),so that $$ \rho_{E}(x)\leq d(x,y)+\rho_{E}(y). $$ 21.Suppose ${\mathcal{H}}$ and F are disjoint sets in a metric space X, K is compact, $\widehat{\cal H}^{\prime}$ is closed Prove that there exists 8>0 such that d(p,g)>8if pe K,q∈ F. Hint: pr is a continuous positive function on K Show that the conclusion may fai for two disjoint closed sets if neither is compact 22. Let A and B be disjoint nonempty closed sets in a metric space $\star_{\mathrm{{A}}}$ , and define $$ f(p)={\frac{\rho_{A}(p)}{\rho_{A}(p)+\rho_{B}(p)}}\quad(p\in X). $$ Show that fis a continuous function on X whose range lies in [0,1], that f(p) = 0 preciscly on A and $f(p)=1$ precisely on B. This establishes a converse of Exercise 3: Every closed set A C Xis Z(f) for some continuous real f on X. Settin $$ V=f^{-1}([0,\mathrm{\bar{i}})),\qquad W=f^{-1}((\mathrm{\bar{i}},1)\ . $$ ]), show that ${\mathfrak{I}}^{\prime}$ and ${\mathcal{W}}$ are open and disjoint, and that $A\subset V,$ Bc W.(Thus pairs of disjoint closed sets in a metric space can be covered by pairs of disjoint open sets. This property of metric spaces is called normality.) 23.A real-valued function f defined in (α、b) is said to be convex if $$ f(\lambda x+(1-\lambda)y)\leq\lambda f(x)+(1-\lambda)f(y) $$ whenever a<x<b,a<y <b,0<入<1. Prove that every convex function is continuous. Prove that every increasing convex function of a convex function is convex.(For example, if f is convex,so is e/.) If f is convex in (a,b) and if a<s<1<u<b, show that $$ {\frac{f(t)-f(s)}{t-s}}\leq{\frac{f(u)-f(s)}{u-s}}\leq{\frac{f(u)-f(t)}{u-t}}\,. $$ 24.Assume that fis a continuous real function defined in (α,b) such that $$ f{\biggl(}{\frac{x+y}{2}}{\biggr)}\leq{\frac{f(x)+f(y)}{2}} $$ for all x, ye(α,b). Prove that fis convex.102 PRINCIPLES OF MATHEMATICAL ANALYSis 25.If A c R and Bc R*,define $A+B$ to be the set of all sums $\mathbf{x}\,+\,\mathbf{y}$ with xe A ye B. (a) If ${\widehat{\operatorname{R}_{a}}}$ is compact and C is closed in R', prove that ${\widehat{\mathbb{N}}}$ + Cis closed. Hint: Take z生 K + C, put F= Z-C, the set of all z-y with ye C. Then K and ${\mathcal{H}}^{\prime}$ are disjoint. Choose Sas in Exercise 21. Show that the open ball with center z and radius does not intersect $R+C,$ (b) Let c be an irrational real number. Let ${\mathit{C}}_{1}$ be the set of all integers, let ${\cal{C}}_{2}$ be the set of all nc with ne C. Show that ${\cal C}_{1}$ and C, are closed subsets of $\textstyle{R^{1}}$ whose sum C,+ C,is not closed, by showing that CI+C,is a countable dense subset of Rl. 26.Suppose X, Y, Z are metric spaces, and Y is compact. Let f map X into Y, le g be a continuous one-to-one mapping of ${\mathcal{V}}$ into Z, and put Mx)=90(O) for x∈ X Prove that f is uniformly continuous if $J_{\ell}$ is uniformly continuous Hint: ${\mathfrak{g}}^{-1}$ l has compact domain g(Y), and $\scriptstyle{\sqrt{\langle x\rangle}}\,=$ g-"(A(rx) Prove also that f is continuous if $J_{\mathit{l}}$ is continuous Show (by modifying Example 4.21, or by finding a different example) that the compactness of Y cannot be omitted from the hypotheses, even when X and Z are compact