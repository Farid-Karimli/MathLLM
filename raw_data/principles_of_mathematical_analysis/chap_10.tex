$$ {\big|}\backslash{\big|}\backslash{\bigrangle} $$ INTEGRATION OF DIFFERENTIALFORMS lntegration can be studied on many levels. In Chap. G6,the theory was developed for reasonably well-behaved functions on subintervals of the real line. In Chap. 11 we shall encounter a very highly developed theory of integration that can be applied to much larger classes of functions, whose domains are more or less arbitrary sets, not necessarily subsets of $R^{n}$ The present chapter is devoted to those aspects of integration theory that are closely related to the geometry of euclidean spaces, such as the change of variables formula,line integrals, and the machinery of differential forms that is used in the statement and proof of the n-dimensional analogue of the fundamental theorem of calculus, namely Stokes’theorem. INTEGRATION 10.1 Definition Suppose ${\mathcal{L}}^{k}$ is a k-cell in R', consisting of al $$ {\bf x}\,\underline{{{=}}}\,\underline{{{\left(x_{1\,\,\,,\iota\ \circ\right.,\eta_{\ell} )}}}}\,\underline{{{x_{k}}}} $$ such that (1) $$ a_{i}\leq x_{i}\leq b_{i}\qquad(i=1,\ldots,k), $$246 PRINCIPLES OF MATHEMATICAL ANALYSIs ${\mathcal{L}}^{j}$ is the j-cell in $\textstyle R^{\prime}$ defined by the first ${\hat{\mathcal{J}}}^{*}$ inequalities(1), and f is a real con tinuous function on $\textstyle{\int}^{k}.$ Put f =/k,and define fk-1 on ${\mathit{I}}^{k-1}$ -1 by $$ f_{k-1}(x_{1},\cdot\cdot\cdot,\,x_{k-1})=\int_{a_{k}}^{b_{k}}f_{k}(x_{1},\cdot\cdot\cdot,\,x_{k-1},\,x_{k})\,d x_{k}\,. $$ The uniform continuity of $\textstyle{\int_{k}}$ on $\textstyle{\mathit{J}}^{k}$ shows that $f_{k-1}$ is continuous on ${}j^{k-1}.$ Hence we can repeat this process and obtain functions f;,continuous on $\textstyle{\mathcal{J}}_{\cdot}$ such that f,- is the integral off,,with respect to xy,over [a,,b,]、After ${\big.}{\binom{~}{\cal K}$ steps we arrive at a number fo,which we call the integral of f over I'; we write it in the form (2) $$ \int_{I^{k}}f(\mathbf{x})\;d\mathbf{x}\qquad{\mathrm{or}}\qquad\int_{I^{k}}f. $$ A priori, this definition of the integral depends on the order in which the k integrations are carried out. However, this dependence is only apparent. T prove this, let us introduce the temporary notation L(f)for the integral (2) and $L^{\prime}(f)$ for the result obtained by carrying out the k integrations in some other order. 10.2 Theorem For every fe G(I"), L(/) = L(/) Proof If $h(\mathbf{x})=h_{1}(x_{1})\cdot\cdot\cdot h_{k}(x_{k}),$ where h e G([a,,b,), then $$ L(h)=\prod_{i=1}^{k}\int_{a_{i}}^{b_{i}}h_{i}(x_{i})\,d x_{i}=L^{\prime}(h). $$ lf .s isthe set of all fnite sums of such functions $\textstyle{\iint_{3}}$ it follows tha ${\cal L}(g)$ L(g) for all g sdAlso, .d is an algebra of functions on $\textstyle{\mathcal{J}}^{k}$ to which the Stone-Weierstrass theorem applies Put V =1(b,一α)、If fe 6(I') and s> 0, there exists $\mathcal{J}$ e .sd such that llf- gl<E/V, where lfl is defined as max IJ(x)|(xe T)、 Then $|L(f-g)|<\varepsilon,\ |L^{\prime}(f-g)|<$ B,and since $$ L(f)-L^{\prime}(f)=L(f-g)+L^{\prime}(g-f), $$ we conclude that |L(f)- L’(f)|< 28 In this connection, Exercise 2 is relevant 10.3 Definition The support of a (real or complex) function $\mathbb{J}$ on $\textstyle{\mathcal{R}}^{k}$ is the closure of the set of all points x∈ R' at which f(x)≠ 0.Iff is a continuousINTEGRATION OF DIFFERENTIAL FORMs247 function with compact support, let $\textstyle{\mathcal{J}}^{k}$ be any ${\mathcal{N}},$ -cell which contains the support of f, and define (3) $$ \int_{R^{k}}f=\int_{I^{k}}f. $$ The integral so defined is evidently independent of the choice of $\textstyle{\int}^{k},$ provided only that ${\boldsymbol{J}}^{k}$ contains the support of f It is now tempting to extend the definition of the integral over $\textstyle{\mathcal{R}}^{k}$ k to functions which are limits (in some sense) of continuous functions with compact support. We do not want to discuss the conditions under which this can be done; the proper setting for this question is the Lebesgue integral. We shal merely describe one very simple example which will be used in the proof of Stokes' theorem. 10.4 Example Let $\Omega^{k}$ be the k-simplex which consists of all points x (x, ….x)in $\textstyle{\mathcal{R}}^{k}$ for which x,+· + X≤1 and $x_{i}\geq0$ O foi i= 1, .,k. If k =3, for example, $\textstyle{\mathcal{Q}}^{k}$ is a tetrahedron, with vertices at O, e1,e2,e,.If f∈ G(Q"). extend f to a function on ${\mathcal{J}}^{k}$ by setting f(x) = 0 off Q', and define (4) $$ \mathit{\int_{Q^{k}}}\ f=\mathit{\int_{I^{k}}}\ f. $$ Here I' is the “unit cube”defined b $$ 0\leq x_{i}\leq1\qquad(1\leq i\leq k). $$ Since f may be discontinuous on $\textstyle{\int}^{k},$ the existence of the integral on the right of(4) needs proof. We also wish to show that this integral is independent of the order in which the ${\mathcal{N}}$ single integrations are carried out. To do this, suppose $0<\delta<1,$ put (5) $$ \varphi(t)\!=\!{\sqrt{\frac{1}{(1-t)}}}\qquad(t\leq1-\delta) $$ and define (6) 上 $$ \quad(\mathbf{x})=\varphi(x_{1}+\mathbf{\epsilon}\cdot\mathbf{\epsilon}+x_{k})f(\mathbf{x})\qquad(\mathbf{x}\in I^{k}). $$ Then Fe 6(I*) Put y=(x,.….*x-1), x=(y, x)、For each $\mathbf{y}\in I^{k-1},$ the set of all x such that Fy. x) ≠ f(y; x) is either empty or is a segment whose length does not exceed 6. Since $0\leq\varphi$ p ≤ 1,it follows that (7) $$ \begin{array}{c l}{{|F_{k-1}({\bf y})-f_{k-1}({\bf y})|\le\delta||f|)}}&{{\ \ \ ({\bf y}\in I^{k-1}),}}\end{array} $$248 PRINCIPLES OF MATHEMATICAL ANALYsIS where lfl has the same meaning as in the proof of Theorem 10.2,and Fx-1 fk-t are as in Definition 10.1. As 6→0,(7) exhibits fk- as a uniform limit of a sequence of continuous functions. Thus fk-1∈6(I*-l), and the further integrations present no problem. This proves the existence of the integral (4)、 Moreover,(T) shows that (8) $$ \left|\int_{I k}F(\mathbf{x})\,d\mathbf{x}-\int_{I k}f(\mathbf{x})\,d\mathbf{x}\right|\leq\delta\|f\|. $$ Note that (8) is true, regardless of the order in which the k single integrations are carried out. Since F∈ 6(I*),JF is unaffected by any change in this order Hence(8) shows that the same is true of If. This completes the proof Our next goal is the change of variables formula stated in Theorem 10.9 To facilitate its proof, we first discuss so-called primitive mappings, and parti tions of unity. Primitive mappings will enable us to get a clearer picture of the local action of a C′-mapping with invertible derivative, and partitions of unity are a very useful device that makes it possible to use local information in a global setting PRIMITIVE MAPPINGS 10.5 Definition If G maps an open set $\scriptstyle E\in K^{n}$ into $\textstyle R^{n},$ and if there is an integer m and a real function g with domain ${\underline{{\mathcal{D}}}}_{x}^{\prime}$ such that (9) $$ \mathrm{G}({\bf x})=\sum_{i\neq m}x_{i}\,{\bf e}_{i}+g({\bf x}){\bf e}_{m}\qquad({\bf x}\in E), $$ then we call G primitive. A primitive mapping is thus one that changes at most one coordinate. Note that (9) can also be written in the form (10) $$ \mathrm{G(x)}=\mathrm{x}+\,\left[g(\mathbf{x})-x_{m}\right]\mathrm{e}_{m}\,. $$ operator G'(a) has If g is differentiable at some point ${\mathfrak{a}}\in E.$ so is $\overline{{\left(\underline{{{\nu}}}\underline{{{\nu}}}}\underline{{\right)}}}$ .The matrix $[\alpha_{i j}]$ of the (11) $$ (D_{1}g)({\bf{a}}),\;.\;.\;.\;(D_{m}g)({\bf{a}}),\;.\;.\;.\;(D_{n}\;g)({\bf{a}}) $$ as its mth row. For $j\neq m,$ we have $\scriptstyle x_{j}=1$ and $\scriptstyle x_{i}=0$ if i≠j. The Jacobian of G at ais thus given by (12) $$ J_{\mathrm{G}}(\mathbf{a})=\operatorname*{det}[G^{\prime}(\mathbf{a})]=(D_{m}\,g)(\mathbf{a}), $$ and we se by Theorem 9.36) that C(a) is inverible f and only iG(DmsJ)a) = 0INTEGRATION OF DIFFERENTIAL FORMS 249 10.6 Definition A linear operator $\mathbb{S}$ On $R^{n}$ that interchanges some pair of members of the standard basis and leaves the others fixed will be called a ftip For example, the flip $\widehat{\cal D}$ on ${\mathcal{R}}^{\lambda}$ that interchanges e, and ea has the form (13) B(x e,+ ×2e。 + ×se。 + xa ea)= ×1e, + X,e2。 + ×. 。 + ×x4e2 or, equivalently, (14) B0x e,+ X12+ ×9e。+ ×4e4) = ×e, + ×4 2 + ×3e.+ ×,e。 Hence B can also be thought of as interchanging two of the coordinates, rather than two basis vectors. In the proof that follows, we shall use the projections $\begin{array}{l}{{\displaystyle\int_{\mathbf{0}}~,~\L~,~\l~,~\L~\L~\nabla~\L}}\end{array}\$ in R", defined by $P_{0}\times=0$ and (15) $$ P_{m}\,{\bf x}={\o{\chi}_{1}\,\mathrm{e}_{1}}\,+\,\mathbf{\circ}^{*}\,\mathbf{\circ}\cdot\mathbf{\circ}\,+\,{\o{x}_{m}}\,\mathbf{e}_{m} $$ for 1≤m ≤n. Thus $\textstyle P_{m}$ is the projection whose range and null space are spanned by {e,.…,e and $\left\{\Theta_{m+1,{\mathrm{~\ell\ell~}}\circ{\mathrm{~\Theta_{B}}}}\right\};\;$ , respectively. 10.7 Theorem Suppose $\frac{\mathbb{P}}{\mathbb{C}}$ is a B'-mapping of an open set Ec R" into R",0∈ E F(0) = 0, and F'(0) is invertible. Then there is a neighborhood of O in A $R^{n}$ in which a representation (16) $$ \mathbf{F}(\mathbf{x})=B_{1}\cdot\cdot\cdot B_{n-1}\mathbf{G}_{n}\circ\cdot\cdot\cdot\cdot\circ\mathbf{G}_{1}(\mathbf{x}) $$ is valid. In(16), each $\textstyle{\binom{n}{N_{i}}}$ is a primitiveC'-mapping in some neighborhood of0: G,(0) = 0, G(0) is invertible, and each ${\boldsymbol{B}}_{i}$ is either a ftip or the identity operator Briefly,(16) represents F locally as a composition of primitive mapping and flips Proof Put F = F,.Assume 1≤m ≤n -1,and make the following induction hypothesis (which evidently holds for m = 1): Vmis a neighborhood of O, Fm∈ C'(Vm),F(0) = 0,F'.(0) is invertible and (17) $$ P_{m-1}\mathbf{F}_{m}(\mathbf{x})=P_{m-1}\mathbf{x}\qquad(\mathbf{x}\in V_{m}). $$ By(17), we have (18) $$ \mathbf{F}_{m}(\mathbf{x})=P_{m-1}\mathbf{x}+\sum_{i=m}^{n}\alpha_{i}(\mathbf{x})\mathbf{e}_{i}\,, $$ where αm, .., α。 are real G'-functions in $V_{\mathit{m}}$ Hence (19) $$ {\bf F}_{m}^{\prime}({\bf0})\mathrm{e}_{m}=\sum_{i=m}^{n}(D_{m}\alpha_{i})({\bf0})\mathrm{e}_{i}\,. $$250 PRINCIPLES OF MATHEMATICAL ANALYSIS Since F'(0) is invertible, the left side of (19) is not O, and therefore there is a k such that m ≤k ≤n and(Dmαx,)(0) ≠ 0. Let ${\boldsymbol{B}}_{m}$ be the fip that interchanges m and this k (ifk = m, Bmis the identity) and define (20) $$ \mathrm{G}_{m}({\bf x})={\bf x}+[\alpha_{k}({\bf x})-x_{m}]{\bf e}_{m}\qquad({\bf x}\in V_{m}) $$ Then Gm∈ C′(Vm),Gmis primitive, and GA,(0) is invertible, since (Dmα,)(0) + 0. The inverse function theorem shows therefore that there is an open set ${\mathcal{U}}_{m\;,\;\;\;\;\;}$ with Oe UmC Vm,such that Gmis a 1-1 mapping of $U_{m}$ onto a neighborhood $V_{m+1}$ of 0,in which Gm is continuously differentiable Define $\mathbb{R}_{m+1}$ by (21) $$ \mathrm{F}_{m+1}({\bf y})=B_{m}\,\mathrm{F}_{m}\circ\mathrm{G}_{m}^{-\,1}({\bf y})\qquad({\bf y}\in\mathrm{~}V_{m+1}). $$ Then Fm+1∈ 6'(Vm+1),Fm+1(0) = 0,and F'm+1(0))is invertible (by the chain rule)、 Also, for xe Um, (22) $$ \begin{array}{r l}{P_{m}\operatorname{F}_{m+1}(\mathbf{G}_{m}(\mathbf{x}))=P_{m}\,B_{m}\operatorname{F}_{m}(\mathbf{x})}\\ {\,}&{{}=P_{m}[P_{m-1}\mathbf{x}+\alpha_{k}(\mathbf{x})\mathbf{e}_{m}+\ \cdots]}\\ {=P_{m-1}\mathbf{x}+\alpha_{k}(\mathbf{x})\mathbf{e}_{m}}\\ {=P_{m}\mathbf{G}_{m}(\mathbf{x})}\end{array} $$ so that (23) $$ P_{m}\,\mathbf{F}_{m+1}(\mathbf{y})=P_{m}\mathbf{y}\qquad(\mathbf{y}\in V_{m+1}). $$ Our induction hypothesis holds therefore with m + 1 in place of m the definition of $\bar{\cal P}_{m\;3}$ and finally (20). $\ B_{m}\,,$ then [In (22), we first used (21), then (18)and the definition of B Since $B_{m}B_{m}=I,$ (21), with $\mathbf{y}=\mathbf{G}_{m}(\mathbf{x}),$ is equivalent to (24) $$ \operatorname{F}_{m}(\mathbf{x})=B_{m}\operatorname{F}_{m+1}(\mathbf{G}_{m}(\mathbf{x}))\qquad(\mathbf{x}\in U_{m}). $$ If we apply this with $m=1,\ \ldots,n-1$ , we successively obtain $$ \begin{array}{r}{\mathbf{F}=\mathbf{F}_{1}=B_{1}\mathbf{F}_{2}\circ\mathbf{G}_{1}}\\ {=B_{1}B_{2}\mathbf{F}_{3}\circ\mathbf{G}_{2}\circ\mathbf{G}_{1}=\dotsb}\\ {=B_{1}\dotsb B_{n}-\mathbf{F}_{n}\circ\mathbf{G}_{n-1}\circ\mathbf{G}_{n}\circ\mathbf{\dots}\circ\mathbf{G}_{n}}\end{array} $$ in some neighborhood of O、By(17), F。is primitive. This completes the proof.INTEGRATION OF DIFFERENTIAL FORMs 251 PARTITIONS OF UNITY 10.8 Theorem Suppose K is a compact subset of R",and {Va} is an open cover of K. Then there exist functions Wi, ..., s∈ 6(R") such that (a)0≤业,≤1 for 1 ≤i≤s (b)each W, has its support in some Va,and (c)W(x)+·…+业,(x) = 1 for every xe K Because of(c),{y}is called a partition of unity,and(b)is sometimes expressed by saying that $\{\psi_{i}\}$ is subordinate to the cover {Va) Corollary If fe 6(R") and the support of f lies in K, then (25) $$ f=\sum_{i=1}^{s}\psi_{i}f. $$ Each y,f has its support in some $V_{\alpha}$ The point of(25) is that it furnishes a representation of f as a sum o continuous functions $\psi_{i}f$ with“"small’supports. Proof Associate with each xe K an index α(x)so that $\mathbf{x}\in V_{\alpha(x)}$ Then there are open balls B(x) and W(x), centered at x, with (26) $$ \overline{{{B({\bf x})}}}\subset W({\bf x})\subset\overline{{{W({\bf x})}}}\subset V_{\alpha({\bf x})}\,. $$ Since K is compact, there are points $\mathbf{x}_{1},\dots,\mathbf{x}_{t}$ in $\textstyle K$ such that (27) $$ K\subset B(\mathbf{x}_{1})\cup\cdot\cdot\cdot\cup B(\mathbf{x}_{s}). $$ By (26),there are functions $\varphi_{1},$ .,0,e G(KR"), such that on $R^{n}$ 、Define $\psi_{1}=\,\varphi_{1}$ on B(x), p,(x) = outside W(x,), and $0\leq\varphi_{i}(\mathbf{x})\leq1$ $\varphi_{i}(\mathbf{x})=1$ and (28) $$ \psi_{i+1}=(1-\varphi_{1})\cdot\cdot\cdot(1-\varphi_{i})\varphi_{i+1} $$ for i= 1,.. .,8-1 Properties (a) and (b) are clear. The relation (29) $$ \psi_{1}\,+\,\cdot\,\cdot\,\cdot\,+\psi_{i}=1\,-\,(1\,-\,\varphi_{1})\cdot\cdot\cdot\,(1\,-\,\varphi_{i}) $$ is trivial for i= 1. If(29) holds for some i<s, addition of(28) and (29 yields (29) with ${\mathit{i}}+1$ in place of i.It follows that (30) $$ \begin{array}{l l}{{\frac{s}{\hbar}\psi_{i}({\bf x})=1-\prod_{i=1}^{s}[1-\varphi_{i}({\bf x})]\qquad\quad({\bf x}\in R^{n}).}}\end{array} $$ If xe K, then xe B(x)) for some i, hence cp ,(x) = 1, and the product in (30) is O. This proves (c).252 rnnNCIPLEs Or MATHEMATICAL ANALYsis CHANGE OF VARIABLES We can now describe the ffect of a change of variables on a multiple integral For simplicity, we confine ourselves here to continuous functions with compact support, although this is too restrictive for many applications. This s illustrated by Exercises 9 to 13. 10.9 Theorem Suppose $^{'}\!{\hat{J}}^{'}$ is a 1-1 6'-mapping of an open set ${\widetilde{F}},$ c $\textstyle{\mathcal{R}}^{k}$ into R such that J-(x) = 0 for all xe E.I/ f is a continuous function on $R^{k}$ * whose suppor R” is compact and lies in ${\boldsymbol{T}}({\boldsymbol{E}}),$ then (31) $$ \int_{\partial k}f(\mathbf{y})\ d\mathbf{y}=\int_{\partial k}f(T(\mathbf{x}))|J_{T}(\mathbf{x})|\ d\mathbf{x} $$ x. We recall that $\textstyle{J_{T}}$ is the Jacobian of T. The assumption .J-(x) + 0 implies by the inverse function theorem, that $T^{-1}$ is continuous on T(E),and this ensures that the integrand on the right of(31) has compact support in $\bar{F}^{\prime}$ (Theorem 4.14). The appearance of the absolute value of J-(x)in (31) may call for a com ment. Take the case A $K=1,$ , and suppose T is a i-1 6'-mepping of $\textstyle R^{1}$ lonto $\textstyle{\mathcal{R}}^{1}$ Then J-(x) = 7′(x); and if ${\mathcal{P}}$ is increasing, we have (32) $$ \int_{R^{1}}f(y)\,d y=\int_{R^{1}}f(T(x))T^{\prime}(x)\,d x, $$ T decreases,then $T^{\prime}(x)<0$ by Theorems 6.19 and 6.17, for all continuous f with compact support. But if ; and if f is positive in the interior of its support, the left side of (32) is positive and the right side is negative. A correct equation is obtained if T’is replaced by |T′} in (32). The point is that the integrals we are now considering are integrals of functions over subsets of R*,and we associate no direction or orientation with these subsets. We shall adopt a different point of view when we come to inte gration of differential forms over surfaces. Proof It follows from the remarks just made that(31) is true if ${\mathcal{D}}^{\gamma}$ is a primitive B'-mapping(see Definition 10.5),and Theorem 10.2 show that (31) is true if T is a linear mapping which merely interchanges two coordinates. Ifthe theorem is true for transformations P, Q, and if S(x) = P(Q(x)) then G(c) dz =(/(PGy),JAy)| dy -[/(P(Qx).J(Qx)| /(x)| dx =「/S()/.(x) xINTEGRATION OF DIFFERENTIAL FORMs 253 since JI (Q(x))./o(x) = det P'(Q(x)) det Q'(x = det P'(Q(x))Q′(x) = det S'(x)= 15(x) by the multiplication theorem for determinants and the chain .rule. Thus the theorem is also true for S. Each point a e ${\widetilde{H}}^{\prime}$ has a neighborhood U C E in which (33) $$ T(\mathbf{x})=T(\mathbf{a})+B_{1}\cdot\cdot\cdot B_{k-1}G_{k}\circ\mathbf{G}_{k-1}\circ\cdot\cdot\cdot\circ\circ\mathbf{G}_{1}(\mathbf{x}-\mathbf{a}), $$ where $\complement_{i}$ ; and ${\boldsymbol{B}}_{i}$ are as in Theorem 10.7. Setting $V=T(U)$ ,it follows that (31) holds if the support of f lies in V. Thus: Each point y e T(E) lies in an open set $V_{\mathrm{y}}\subset T(E)$ ) such that (31) holds for all continuous functions whose support lies in $V_{\mathbf{y}}$ Since $\{V_{\mathfrak{p}}\}$ Now let f be a continuous function with compact support $K\subset T(E)$ covers K, the Corollary to Theorem 10.8 shcivws that f= ZW, where each v,is continuous,and each ${\mathcal{Y}}_{i}$ has its support in some V, Thus(31)holds for each $\psi_{i}f,$ and hence also for their sum f. DIFFER ENTIAL FORMS We shall now develop some of the machinery that is needed for the n-dimen- sional version of the fundamental theorem of calculus which is usually called Stokes'theorem. The original form of Stokes' theorem arose in applications of vector analysis to electromagnetism and was stated in terms of the curl of a vector field. Green's theorem and the divergence theorem are other special cases. These topics are briefly discussed at the end of the chapter. lt is a curious feature of Stokes' theorem that the only thing that is difficul about it is the elaborate structure of definitions that are needed for its statement These definitions concern differential forms,their derivatives,boundaries, and orientation. Once these concepts are understood, the statement of the theorem is very brief and succinct, and its proof presents litle dificulty Up to now we have considered derivatives of functions of several variables only for functions defined in open sets. This was done to avoid difficulties that can occur at boundary points.It will now be convenient, however, to discuss differentiable functions on compact sets. We therefore adopt the following convention: $D\subset R^{k}$ into $\textstyle R^{n}$ $W\subset R^{k}$ To say that f is a '-mapping (or a C"-mapping) of a compact set and such that g(x) = f(x)for an open set into R" such that I means that there is a G'-mapping (or a C"-mapping)g of $D\subset W$ all xe ${\mathcal{D}}$254 PRINCIPLES OF MATHEMATICAL ANALYSIs 10.10 Defnition Suppose ${\widetilde{\mathcal{H}}}$ is an open set in $R^{n}$ A k-surface in E is a C- mapping db from a compact set $\scriptstyle{D={R^{n}}}$ into E D is called the parameter domain of D、Points of ${\hat{H}}{\hat{H}}$ will be denoted by u =(ui,,...,Ux) We shall confine ourselves to the simple situation in which D is either a k-cell or the k-simplex $\underline{{{G}}}^{k}$ described in Example 10.4. The reason for this s that we shall have to integrate over D, and we have not yet discussed ntegration over more complicated subsets of R'. It will be seen that this restriction on D (which wil be tacitly made from now on) entails no significant loss of generality in the resulting theory of differential forms. we stess ta k-surfaces in E are defined to be mappings into E. no subsets of E. This agrees with our earlier definition of curves(Definition 6.26) In fact,,l-surfaces are precisely the same as continuously differentiable curves 10.11 Definition Suppose ${\overline{{\mathcal{H}}}}^{\gamma}$ is an open set in $\textstyle R^{n}\!.$ A differential form of orde k ≥1 in E(briefly, a k-form in E) is a function の, symbolically represented by the sum (34) $$ \omega=\sum a_{i_{1}}\cdot\cdot\cdot_{i_{k}}({\bf x})\,d x_{i_{1}}\wedge\cdot\cdot\cdot\wedge\,d x_{i_{k}} $$ (the indices $i_{1}\!_{}^{}_{\!1}\!,\ \cdot\cdot\cdot\ ,\ i_{k}$ range independently from l to n), which assigns to each k-surface $({\hat{\mathbb{P}}})$ ) in ${\mathcal{H}}^{\nu}$ a number $\omega(\Phi)=\int_{\Phi}\omega,$ according to the rule (35) $$ \int_{\phi}\omega=\int_{D}\sum a_{i_{1}}\cdot\cdot\cdot\cdot_{i_{k}}(\Phi(\mathbf{u})){\frac{\hat{\mathcal{C}}(x_{i_{1}}\cdot\cdot\cdot\cdot,\,x_{i_{k}})}{\hat{\mathcal{C}}(u_{1},\cdot\cdot\cdot,\,u_{k})}}d\mathbf{u}, $$ where ${\hat{H}}\rangle$ is the parameter domain of の The functions $Q_{i_{1}}\ \ ,\ \L,\ \ \L,\ \ \i_{i_{k}}$ are assumed to be real and continuous in E. lf $\phi_{1},\,\cdot\,\cdot\,,\,\phi_{n}$ are the components of O, the Jacobian in(35) is the one determined by the mapping $$ (u_{1},\cdot\cdot\cdot,\,u_{k})\to(\phi_{i_{1}}(\mathbf{u}),\cdot\cdot\cdot,\,\phi_{i_{k}}(\mathbf{u})). $$ Note that the right side of (35) is an integral over ${\mathcal{D}},$ as defined in Defini- tion 10.1(or Example 10.4) and that (35)is the definition of the symbol (。w A k-form o is said to be of class $\mathcal{C}^{\prime}$ ’or G” if the functions $a_{i_{1}}\cdot\cdot\cdot i_{k}$ in (34) are all of class C’or G” A O-form in ${\mathcal{H}}^{\dagger}$ is defined to be a continuous function in E 10.12 Examples (a) Let )be a l-surface (a curve of class G))in R,with parameter domain [0,1]. Write(x,y, 2 in place of (x, X2,X), and put $$ \omega=x\,d y+y\,d x. $$INTEGRATION OF DIFFERENTIAL rORMs 255 Then $$ \int_{\gamma}\omega=\int_{0}^{1}\left[\gamma_{1}(t)\gamma_{2}^{\prime}(t)+\gamma_{2}(t)\gamma_{1}^{\prime}(t)\right]d t=\gamma_{1}(1)\gamma_{2}(1)-\gamma_{1}(0)\gamma_{2}(0). $$ Note that in this example[,o depends only on the initial point y(O and on the end point y(l) of . In particular, J,0 = 0 for every closed curve y.(As we shall Selater, this is true for évery l-form o which is exact.) integrals of l-forms are often called line integrals (b)Fix $a>0,\;b>0,$ and define $$ \gamma(t)=(a\cos t,\,b\sin t)\qquad(0\leq t\leq2\pi), $$ so that yis a closed curve in $R^{2}$ .((Its range is an ellipse.)Then $$ \left\{x\;d y=\right\}_{0}^{2\pi}a b\;\mathrm{cos}^{2}\;t\;d t=\pi a b, $$ whereas $$ \left\{_{\gamma}y\,d x=-\int_{0}^{2\pi}a b\,\sin^{2}t\,d t=-\pi a b.\right. $$ Note that $\stackrel{\left\{}{\right\}}y$ x dy is the area of the region bounded by y. This is a (c) Let special case of Green's theorem ${\widehat{\operatorname{f}}}{\widehat{\operatorname{f}}}$ be the 3-cll defined by $$ 0\leq r\leq1,\qquad0\leq\theta\leq\pi,\qquad0\leq\varphi\leq2\pi. $$ Define O(r,0,) = (x,y, z), where $$ \begin{array}{l}{{x=r\ \mathrm{sin}\ \theta\cos\varphi}}\\ {{y=r\ \mathrm{sin}\ \theta\ \mathrm{sin}\ \varphi}}\\ {{z=r\ \mathrm{cos}\ \theta.}}\end{array} $$ Then $$ J_{\Phi}(r,\theta,\,\varphi)={\frac{{\hat{\sigma}}(x,y,z)}{{\hat{\sigma}}(r,\theta,\,\varphi)}}=r^{2}\,\sin\,\theta $$ Hence (36) $$ \int_{\Phi}d x\wedge d y\wedge d z=\int_{D}J_{\Phi}={\frac{4\pi}{3}}. $$ Note that o maps $\hat{H}\hat{J}$ onto the closed unit ball of R, that the mapping is 1-1 in the interior of ${\hat{H}}\rangle$ (but certain boundary points are identified by D), and that the integral (36) is equal to the volume of p(D)256 PRINCIPLEs Or MATHEMATICAL ANALYsIs 10.13Llementary propertis Lt o, 0,0D be k-forms in E. We wrie $\omega_{1}=\omega_{2}$ if and only if 0,(0) = 02(0) for every k-surface O in E. In particular, 0 =6 means that o(0D) = 0 for every k-surface in E. If c is a real number, then co is the k-form defined by (37) $$ \int_{\Phi}c\omega=c\int_{\Phi}\omega, $$ and の = 01 + 0のz means that (38) $$ \int_{\Phi}\omega=\int_{\Phi}\omega_{1}+\int_{\Phi}\omega_{2} $$ that for every k-surface D in E.As a special case of (37), note that -o is defined so (39) $$ \int_{\Phi}(-\omega)=-\int_{\Phi}\,d\omega. $$ Consider a k-form (40) $$ \omega=a(\mathbf{x})\,d x_{i_{1}}\wedge\,\cdot\cdot\,\wedge\,d x_{i_{k}} $$ and let の be the k-form obtained by interchanging some pair of subscripts i (40). If (35) and (39) are combined with the fact that a determinant changes sign if two of its rows are interchanged, we see that (41) $$ \vec{\omega}=-\omega. $$ As a special case of this, note that the anticommutative relation (42) $$ d x_{i}\wedge d x_{j}=-d x_{j}\wedge d x_{i} $$ holds for all iand jIn particular (43) $$ d x_{i}\wedge d x_{i}=0\qquad(i=1,\dots,n). $$ More generally, let us return to(40), and assume that ${\overline{{\omega}}}=\omega,$ hence $\scriptstyle\omega=0,$ by r ≠ s.If these two subscripts are interchanged, then $i_{r}=i_{s}$ for some (41). In other words, if ois given by(40),then = 0 unless the subscripns i,.…., i, are all distinct If o is as in (34), the summands with repeated subscripts can therefore be omitted without changing o. forms. lt follows that O is the only k-form in any open subset of R", if ${\boldsymbol{k}}>{\boldsymbol{n}}$ The anticommutativity expressed by (42) is the reason for the inordinat amount of atention that has to be paid to minus signs when studying diffrentialINTEGRATIOoN or DIF FERENTIAL rOxMs 257 10.14 Basic ${\hat{H}}\gamma$ -forms lf i,. $\stackrel{*}{\bigcup}_{K}$ are integers such that $1\leq i_{1}<i_{2}<\cdots$ <i。≤n, and if Iis the ordered k-tuple i,,.……,iA}, then we call $\mathcal{T}$ an increasing k-index, and we use the brief notation (44) $$ d x_{I}=d x_{i_{1}}\wedge\dots\wedge d x_{i_{k}}. $$ These forms dxy are the so-called basic k-forms in R” It is not hard to verify that there are precisely n!/k!(n - k)! basic k-forms in $R^{n};$ we shall make no use of this, however. Much more important is the fact that every k-form can be represented in terms of basic k-forms. To see this, note that every k-tuple {,.…. A} of distinct integers can be converted to an increasing k-index J by a finite number of inter- changes of pairs; each of these amounts to a multiplication by -1, as we saw in Sec. 10.13; hence (45) $$ d x_{j_{1}}\wedge\cdot\cdot\cdot\wedge d x_{j_{k}}=s(j_{1},\ldots\cdot\jmath_{k})\,d x_{J} $$ where &(j, .…., ))is l or -1, depending on the number of interchanges that are needed. In fact, it is easy to see that (46) $$ s(j_{1},\ \cdot\ ,\ \cdot\ ,j_{k})=s(j_{1},\ \cdot\cdot\ ,\ j_{k}) $$ where s is as in Definition 9.33. For example, $$ d x_{1}\wedge d x_{5}\wedge d x_{3}\wedge d x_{2}=-d x_{1}\wedge d x_{2}\wedge d x_{3}\wedge d x_{5} $$ and $$ d x_{4}\wedge d x_{2}\wedge d x_{3}=d x_{2}\wedge d x_{3}\wedge d x_{4}\,. $$ If every k-tuple in (34))is converted to an increasing k-index, then we obtain the so-called standard presentation of o: (47) $$ \omega=\sum_{I}b_{I}(\mathbf{x})\,d x_{I}. $$ ${\boldsymbol{\partial}}_{I}$ For example, The summation in(47) extends over all increasing k-indices I.[Of course, every increasing k-index arises ftrom many (from k!, to be precise) k-tuples. Each in (47) may thus be a sum of several of the coeffcents that occur in (34).1 $$ x_{1}\,d x_{2}\land d x_{1}-x_{2}\,d x_{3}\land d x_{2}+x_{3}\,d x_{2}\land d x_{3}+d x_{1}\land d x_{2} $$ is a 2-form in $R^{3}$ whose standard presentation is $$ (1-x_{1})\,d x_{1}\wedge d x_{2}+(x_{2}+x_{3})\,d x_{2}\wedge d x_{3}\,. $$ The following uniqueness theorem is one of the main reasons for the introduction of the standard presentation of a k-form.258 PRNCIPLEs OF MATHEMATICAL ANALYsis 10.15 Theorem Suppose (48) $$ \omega=\sum_{I}b_{I}(\mathbf{x})\,d x_{I} $$ is the standard presentation of a k-form o in an open set $E=R^{n}$ If 0 = 0 in E, then $b_{I}(\mathbf{x})=0$ for every increasing k-index $\textstyle{\mathcal{J}}$ and for every x∈ E. Note that the analogous statement would be false for sums such as (34) since, for example, $$ d x_{1}\wedge d x_{2}+d x_{2}\wedge d x_{1}=0. $$ Proof Assume, to reach a contradiction, that b,v) > 0 for some v e $\widehat{F}^{+}$ and for some increasing k-index J = {, .….)}. Since $R^{n}$ whose coordinates there exists h > 0 such that b,(x)> 0 for all xe $\partial_{J}$ is continuous satisfy $|\,x_{i}-v_{i}|\leq h.$ Let D be the k-cell in $\textstyle{\mathcal{R}}^{k}$ such that ue 工 ${\hat{H}}\rangle$ if and only if |u,」≤ $\textstyle{\int}$ for r = 1,..,k.Define (49) $$ \Phi(\mathbf{u})=\displaystyle\forall+\sum_{r=1}^{k}u_{r}\mathbf{e}_{j r}\qquad(\mathbf{u}\in D). $$ Then D is a k-surface in $\textstyle E_{\mathrm{{J}}}$ with parameter domain ${\mathcal{D}},$ and b,(0(u))>0 for every ue $\neq{\mathcal{J}}$ We claim that (50) b,(0(u)) du Since the right side of (50) is positive,it follows that (0) ≠ 0.Hence (50) gives our contradiction To prove (50), apply (35) to the presentation(48). More specically compute the Jacobians that occur in (35)、By (49) $$ {\frac{\partial(x_{j_{1}},\ \cdot\cdot\cdot\cdot\cdot\cdot\cdot\cdot x_{j_{k}})}{\partial(u_{1},\ \cdot\cdot\cdot\cdot,u_{k})}}=1. $$ For any other increasing k-index $\scriptstyle I\neq J,$ the Jacobian is O, since it is the determinant of a matrix with at least one row of zeros 10.16 Products of basic k-forms Suppose (51) $$ I=\{i_{1},\ldots,i_{p}\},\qquad J=\{j_{1},\ldots,j_{q}\} $$ where 1≤i $\iota\prec\cdot\cdot\ <i_{p}\leq n$ and $1\leq j_{1}<\cdot\cdot<j_{q}\leq n.$ The product of the cor responding basic forms $d x_{I}$ and $d x_{J}$ in R” is a (p + 9)-form in R", denoted by the symbol dx, A dxy, and defined by (52) $$ d x_{I}\wedge d x_{J}=d x_{i_{1}}\wedge\cdot\cdot\cdot\times d x_{i_{p}}\wedge d x_{j_{1}}\wedge\cdot\cdot\cdot\wedge d x_{j_{q}}. $$INTEGRATION OF DIFFERENTLAL FORMS25S If $\underline{{\gamma}}$ and J have an element in common,then the discussion in Sec. 10.13 shows that dx,八 dx,= 0. If I and J have no element in common, let us write L,J) for the increasing (p + 9)-index which is obtained by arranging the members of I Jin increasing order. Then c $\scriptstyle a\!q_{U,\,v}$ is a basic (p + 9)-form.We claim that (53) $$ d x_{I}\wedge d x_{J}=(-\,1)^{\alpha}\,d x_{[I,J]} $$ where α is the number of differences j,- i, that are negative.(The number of positive differences is thus P9 - C.) To prove (53), perform the following operations on the numbers (54) $$ \dot{l}_{1>}\;\ast\;\ast\;\cdot\;\cdot\;\cdot\;\cdot\dot{l}_{p}\;\cdot\cdot\cdot\cdot,\dot{J}_{q}\;. $$ Move i, to the right, step by step, unti its right neighbor is larger than i, $i_{p-1},\ \cdot\cdot,\ i$ 。The total The number of steps is the number of subscriptst such thati,<A.(Note that O steps are a distinct possibility.)Then do the same for number of steps taken is α. The final arrangement reached is [I, J]、 Each step when applied to the right side of (52), multiplies dx,A dx, by -1. Hence(53) holds. Note that the right side of(53) is the standard presentation of dx,A dx, Next, let K =(k,…….,k,) be an increasing rindex in {1,...,,). We shal use(53)) to prove that (55) $$ (d x_{I}\wedge d x_{J})\wedge d x_{K}=d x_{I}\wedge(d x_{J}\wedge d x_{K}). $$ lf any two of the sets I, J, K have an element in common, then each side of (55) is O, hence they are equal So let us assume that I, ${\mathcal{Y}}_{\mathfrak{Y}}$ K are pairwise disjoint. Let [I, J, K] denote the increasing (p + 9 + r)-index obtained from their union.Associate β with the ordered pair (J, K) and y with the ordered pair (I, K)in the way that α was associated with (I, J)in(53). The left side of (55) is then $$ (-1)^{x}\,d x_{[I,J]}\wedge d x_{K}=(-1)^{x}(-1)^{\beta+\gamma}\,d x_{[I,J,J]}, $$ K〕 by two applications of(53), and the right side of (55) is $$ (-1)^{\beta}\,d x_{I}\wedge d x_{[J,K]}=(-1)^{\beta}(-1)^{x+\gamma}\,d x_{[I,J,K]}. $$ Hence (5S)is correct 10.17Multiplicatin Suppose o and are p- and g-forms, respectively, in some open set E c R", with standard presentations (56) $$ \omega=\sum_{I}b_{I}(\mathbf{x})\,d x_{I},\qquad\lambda=\sum_{J}c_{J}(\mathbf{x})\,d x_{J} $$ where $\textstyle{\mathcal{F}}$ and $\mathcal{P}$ range over all increasing p-indices and over all increasing g-indices taken from the set {1,...,n}.260 PRINCIPLES OF MATHEMATICAL ANALYSIS Their product, denoted by the symbol の八入,is defined to be (57) $$ \omega\wedge\lambda=\sum_{I J}b_{I}({\bf x})c_{J}({\bf x})\,d x_{I}\wedge d x_{J}. $$ In this sum, Iand Jrange independently over their possible values, and dx, A dx is as in Sec. 10.16. Thus 0八入is a (p + q)-form in E. It is quite easy to see(we leave the details as an exercise) that the distribu tive laws $$ (\omega_{1}+\omega_{2})\wedge\lambda=(\omega_{1}\wedge\lambda)+(\omega_{2}\wedge\lambda) $$ and $$ \omega\wedge(\lambda_{1}+\lambda_{2})=(\omega\wedge\lambda_{1})+(\omega\wedge\lambda_{2}) $$ hold, with respect to the addition defined in Sec. 10.13. If these distributive laws are combined with (55), we obtain the associative law (58) $$ (\omega\wedge\lambda)\wedge\sigma=\omega\wedge(\lambda\wedge\sigma) $$ for arbitrary forms 0,入,c in ${\mathcal{H}}^{\prime}$ In this discussion it was tacitly assumed that p≥1 and g≥1. The produc of a O-form f with the p-form o given by (56) is simply defined to be the p-form $$ f\omega=\omega f=\sum_{I}f(\mathbf{x})b_{I}(\mathbf{x})\,d x_{I}. $$ It is customary to write fo, rather than f入 の, when fis a O-form 10.18 Differentiation Weshall now define a diferentiation operator d which associates a (k + 1)-form do to each ${\hat{N}}.$ -form o of class ${\mathcal{C}}^{\prime}$ in some open set Ec R". A O-form of class ${\mathcal{C}}^{\prime}$ in ${\widetilde{F}}^{\prime}$ is just a real function f∈ 6′(E), and we define (59) $$ d f=\sum_{i=1}^{n}(D_{i}f)({\bf x})\,d x_{i}. $$ If o = Zb,(x) dx, is the standard presentation of a k-form o, and bye G′(E) for each increasing k-index I, then we define (60) $$ d\omega=\sum_{T}\left(d b_{\cal I}\right)\wedge d x_{\cal I}. $$ 10.19Example Suppose $\widehat{H}$ is open in R",f∈ S′(E), and y is a continuously differentiable curve in $\textstyle E_{\mathrm{{J}}}$ with domain [O,1]. By (59) and (35) (61) $$ \int_{\gamma}d f=\int_{0}^{1}\sum_{i=1}^{n}\,(D_{i}f)(\gamma(t))\gamma_{i}^{\prime}(t)\,\,d t. $$INTEGRATION OF DIFFERENTIAL FORMS 261 By the chain rule, the last integrand is(fop)(t)Hence (62)) $$ \int_{\gamma}d f=f(\gamma(1))-f(\gamma(0)), $$ and we see that f, df is the same for all y with the same initial point and the same end point, as in(a) of Example 10.12. Comparison with Example 10.12(b) shows therefore that the l-form x dy is not the derivative of any O-form f.This could also be deduced from part (b of the following theorem, since d(x dy) = dx A dy ≠ 0. 10.20 Theorem (a)If o and )are k- and m-forms, respectively, of class G’in $E,$ then (63) $$ d(\omega\times\lambda)=(d\omega)\wedge\lambda+(-1)^{k}\,\omega\wedge d\lambda. $$ (b)1 o is of clas G” in E, then d-o = 0 Here d-o means, of course, ddo) Proof Because of(57))and(60),(a) follows if (63)is proved for the special case (64) $$ \omega=f d x_{I},\qquad\lambda=g\;d x_{J} $$ where f,g ∈ 6′(E), dx, is a basic k-form, and $d x_{J}$ is a basic m-form.[I k or m or both are O, simply omit $d x_{I}$ or dx, in (64); the proof that follows is unaffected by this.]Then $$ \omega\wedge\lambda=j g\;d x_{I}\wedge d x_{J}. $$ Let us assume that $\hat{J}\,$ and $\mathcal{J}$ have no element in common. [In the other case each of the three terms in(63) is O.] Then, using (53), $$ d(\omega\wedge\lambda)=d(j g\,d x_{I}\wedge d x_{J})=(-1)^{\alpha}\,d(j g\,d x_{[I,J]})\,. $$ By (59), d(fg) = f dg + g df. Hence (60) gives $$ \begin{array}{c}{{d(\omega\wedge\lambda)=(-1)^{x}\,(f d g+g\,d f)\wedge d x_{[I,J]}}}\\ {{}}\\ {{}}\\ {{=(g\,d f+f d g)\wedge d x_{I}\wedge d x_{J}.}}\end{array} $$ Since dg is a 1-form and dx $d x_{I}$ : is a k-form, we have $$ d g\wedge d x_{I}=(-1)^{k}\,d x_{I}\wedge d g, $$262 PRINCIPLES OF MATHEMATICAL ANALYSIs by (42). Hence d(0o人入)= (df人dx)人(g dx,)+(-1)*(f dx)人(dg A dx,) = (do)八入+(-1*0人 d儿 which proves (a). Note that the associative law(58) was used freely Let us prove(b) first for a O-form fe G”: $$ d^{2}f=d{\biggl(}\sum_{j=1}^{n}(D_{j}f)(\mathbf{x})\,d x_{j}{\biggr)} $$ $$ =\sum_{j=1}^{n}d(D_{j}f)\wedge d x_{j} $$ $$ =_{i,j=1}^{n}(D_{i j}f)(\mathbf{x})\,d x_{i}\wedge d x_{j}. $$ Since D,/f= D,uf(Theorem 9.41) and dx, A dx, = -dx,八4 $d x_{i},$ we see that d = 0 Hence (63) shows that If o = dxr, as in (64), then do =(d)A dx,By (60), $d(d x_{t})=0.$ $$ d^{2}\omega=(d^{2}f)\wedge d x_{I}=0. $$ 10.21 Change of variables Suppose ${\widetilde{\mathcal{H}}}^{\dagger}$ is an open set in R",T is a B'-mapping of ${\widetilde{\mathcal{H}}}^{\nu}$ into an open set V c R",and o is a k-form in V, whose standard presenta- tion is (65) $$ \omega=\sum_{I}b_{I}(y)\,d y_{I}. $$ (We use y for points of V, x for points of E. Let t,...,1mbe the components of T:If $$ \mathbf{y}=\left(y_{1},\mathbf{\epsilon}\circ\mathbf{\epsilon}\mathbf{\epsilon}\cdot\mathbf{\epsilon}\mathbf{\epsilon}\cdot\mathbf{y}\mathbf{\epsilon}\right)=T(\mathbf{x}) $$ then $y_{i}=t_{i}(\mathbf{x})$ .As in(59), (66) $$ d t_{i}=\underline{{{\pi}}}_{j=1}^{n}(D_{j}t_{i j}(x)\,d x_{j}\qquad(1\leq i\leq m). $$ Thus each $d t_{i}$ is a 1-form in E The mapping ${\mathcal{J}}^{\mathcal{J}}$ transforms o into a k-form のr in E, whose definition is (67) $$ \omega_{T}=\sum_{I}b_{I}(T({\bf x}))\,d t_{i_{I}}\wedge\cdots\wedge d t_{i_{k}}. $$ In each summand of (67), $I=\{i_{1},\,\ldots,\,i_{k}\}$ is an increasing k-index Our next theorem shows that addition, multiplication, and differentiation of forms are defined in such a way that they commute with changes of variablesINTEGRATION Or DFFERENTIAL FORMs 263 10.22 Theorem Wih E and ${\mathcal{J}}^{\mathcal{J}}$ as in Sec. 10.21, let o and Xbe k- and m-forms in V, respectively. Then (a)(0 + A)- = 0r + .r if k = m; (b)(0人入) = 0r八 入r; (c d(or) =(do) if o is o clss G" and T is o class G” Proof Part (a) follows immediately from the definitions. Part(b)s almost as obvious, once we realize that (68) $$ (d y_{i_{1}}\wedge\cdots\wedge d y_{i_{r}})_{T}=d t_{i_{1}}\wedge\cdots\wedge d t_{i_{r}} $$ regardless of whether $\{i_{1},\ldots,i_{r}\}$ }is increasing or not;(68) holds because the same number of minus signs are needed on each side of (68) to produce increasing rearrangements we turn to the proof of (c)、If fis a O-form of class G” in V,then $$ f_{T}({\bf x})=f(T({\bf x})),\qquad d f=\sum_{i}\,(D_{i}f)({\bf y})\,d y_{i}. $$ By the chain rule, it follows that (69) $$ \begin{array}{c}{{d(f_{T})=\sum_{J}(D_{j}f_{T})({\bf x})\,d x_{j}}}\\ {{=\sum_{J}\sum_{i}(D_{i}f)(T({\bf x}))(D_{J}t_{i})({\bf x})\,d x_{j}}}\\ {{=(d f)_{T}.}}\end{array} $$ 1f $d y_{t}=d y_{i}$ …人 dy。, then $(d y_{\boldsymbol{I}})_{\boldsymbol{T}}=d t_{i_{1}}\wedge\mathbf{\boldsymbol{\cdot\cdot\cdot\cdot}}\wedge d t_{i_{k}},$ and Theorem 10.20 shows that (70) $$ d((d y_{T})_{T})=0. $$ (This is where the assumption $r\in{\mathcal{Q}}^{n}$ is used.) Assume now that $\omega=f\,d y_{I}.$ Then $$ \omega_{T}=f_{T}(\mathbf{x})\,(d y_{I})_{T} $$ and the preceding calculations lead to $$ \begin{array}{c c c}{{d(\omega_{T})=d(f_{T})\wedge(d y_{T})_{T}=(d f)_{T}\wedge(d y_{T})_{T}}}\\ {{}}&{{=(d f)\wedge d y_{T})_{T}=(d\omega)_{T}.}}\end{array} $$ The first equality holds by (63) and (70), the second by (69), the third by part(b),and the last by the definition of do. The general case of (c) follows from the special case just proved, if we apply (a). This completes the proof264 PRINCIPLES OF MATHEMATICAL ANALYSIS Our next objective is Theorem 10.25. This will follow directly from twc other important transformation properties of differential forms, which we state first. 10.23Theorem Suppose ${}^{\prime}{\overline{{\beta}}}^{\mathrm{W}}$ is a G'-mapping of an open set $\stackrel{\beta}{\cal F},$ C 中 ” into an open $\textstyle{\mathcal{R}}^{n}$ set V c Rm,,S is a B'-mapping of V into an open set W c RP,and o is a k-form in W,so that ${\mathcal{O}}_{S}$ is a k-form in V and both $(\omega_{S})_{T}$ and $\omega_{S T}$ are k-forms in ${\mathcal{E}},$ where ST' is defined by (STXKX) = S(T(x)Then (71) $$ (\omega_{S})_{T}=\omega_{S T}. $$ Proof If o and . are forms in W, Theorem 10.22 shows that $$ ((\omega\wedge\lambda)_{S})_{T}=(\omega_{S}\wedge\lambda_{S})_{T}=(\omega_{S})_{T}\wedge(\lambda_{S})_{T} $$ and $$ (\omega\wedge\lambda)_{S T}=\omega_{S T}\wedge\lambda_{S T}. $$ Thus if(71) holds for o and for A,it follows that(71) also holds for 0人入 Since every form can be built up from O-forms and l-forms by addition and multiplication, and since(71) is trivial for O-forms, it is enough to prove (71)in the case $\omega=d z_{q},\,q=$ 1,...,pP.(We denote the points of E, V,W by x,y,z, respectively.) nents of $\mathcal{T}_{\mathbf{D}}(t\tilde{\mathit{f}}_{1},\ldots\alpha_{\mathbf{\delta}}\tilde{\mathbf{f}}_{m}$ be the components of T $T_{\mathrm{{J}}}$ let si,...,S,be the compo $\omega=d z_{q},$ then $\mathbf{S}_{\!_{J}}$ and $\vert\in\hat{\Gamma}~J_{1}^{\prime},~\cdot~,~\cdot~,~{\mathcal{T}}_{p}$ be the components of ${\boldsymbol{S}}{\boldsymbol{T}}$ .If $$ \omega_{S}=d s_{q}=\sum_{j}\left(D_{j}s_{q}\rangle(\mathbf{y})\,d y_{j},\right) $$ so that the chain rule implies $$ \begin{array}{l c r}{{(\omega_{S})_{T}=\sum_{J}\left(D_{J}s_{q}\right)\left(T({\bf x})\right)\,d t_{J}}}\\ {{=\sum_{J}\left(D_{J}s_{q}\right)\sum_{i}\left(D_{i}t_{J}({\bf x})\right)\frac{1}{i} (D_{i}t_{J}=d r_{q}=\omega_{S T}.}}\end{array} $$ 10.24 Theorem Suppose o is a k-form in an open set $E\subset R^{n},$ p is a k-surface in E, with parameter domain $\scriptstyle D=R^{n}$ , and $\mathbf{\psi}/\O_{\mathrm{B}}$ is the k-surface in $R^{k},$ with parameter domain D, defined by $\Delta({\mathbf{u}})={\mathbf{u}}({\mathbf{u}}\in D).$ Then $$ \Bigr\{_{\L,\L}\omega=\Bigr\}_{\L,\L}\omega_{\L,\L} $$ Proof We need only consider the case $$ \omega=a(\mathbf{x})\,d x_{i_{1}}\wedge\mathbf{\nabla}\cdot\mathbf{\nabla}\cdot\mathbf{\nabla}\wedge d x_{i_{k}}. $$INTEGRATION OF DIFFERENTIAL FORMS 265 If p1,.….,pnare the components of p, then $$ \omega_{\Phi}=a(\Phi(\mathbf{u}))\,d\phi_{i_{1}}\wedge\mathbf{\cdot}\cdot\cdot\cdot\wedge\ d\phi_{i_{k}}. $$ The theorem will fllow if we can show tha (72) $$ d\phi_{i_{1}}\wedge\cdot\cdot\cdot\times\,d\phi_{i_{k}}=J({\bf u})\,d u_{1}\,\wedge\,\cdot\cdot\times\,d u_{k}\,, $$ where $$ J(\mathbf{u})={\frac{{\hat{\sigma}}(x_{i_{1}}\cdot\mathbf{\hat{\cdots}}\cdot\mathbf{\hat{\cdots}}\iota_{i_{k}})}{{\hat{\sigma}}(u_{1},\mathbf{\hat{\cdots}},u_{k})}}; $$ since (72) implies $$ \int_{\Phi}\omega=\int_{D}a(\Phi({\bf u}))J({\bf u})\,d{\bf u} $$ $$ =\int_{\Delta}a(\Phi(\mathbf{u})J(\mathbf{u})\;d u_{1}\;\wedge\;\cdots\;\wedge\;d u_{k}=\int_{\Delta}\omega_{\Phi} $$ Let [A] be the ${\hat{I}}\,\!$ by k matrix with entries $$ \begin{array}{l l}{{\alpha(p,q)=(D_{q}\phi_{i_{p}})({\bf u})\qquad(p,q=1,\ldots,k).}}\end{array} $$ Then $$ d\phi_{i_{p}}=\sum_{q}\alpha(p,q)\,d u_{q} $$ so that $$ d\phi_{i_{1}}\wedge\cdot\cdot\cdot\wedge\,d\phi_{i_{k}}=\sum\alpha(1,q_{1})\cdot\cdot\cdot\alpha(k,q_{k})\,d u_{q_{1}}\wedge\cdot\cdot\cdot\cdot\wedge\,d u_{q_{k}}. $$ In this last sum $q_{1},\ \ldots,q_{k}$ range independently over 1,...,k.The anti- commutative relation (42) implies that $$ d u_{q_{1}}\wedge\cdot\cdot\cdot\wedge\;d u_{q_{k}}=s(q_{1},\ .\cdot\cdot\cdot\cdot\cdot q_{k})\,d u_{1}\wedge\cdot\cdot\cdot\wedge\;d u_{k}, $$ where s is as in Definition 9.33; applying this definition, we see that $$ d\phi_{i_{1}}\wedge\cdot\cdot\cdot\cdot\wedge\ d\phi_{i_{k}}=\mathrm{det}\ [A]\,d u_{1}\wedge\cdot\cdot\cdot\wedge\ d u_{k} $$ 安 and since J(u) = det [A],(72) is proved The final result of this section combines the two preceding theorems 10.235 Theorem Supose Tis a G'-mapping of an open se $\scriptstyle E\in K^{n}$ into an open set V e R”,o is a k-surface in E, and o is a k-form in V Then $$ \Bigr\{_{T\circ}\omega=\Bigr\}_{\omega}\omega_{T}. $$266 PRINCIPLES OF MATHEMATICAL ANALYSIs Proof Let D be the parameter domain of の(hence also of T0) and define A as in Theorem 10.24. Then $$ \int_{T\Phi}\omega=\int_{\bf a}\omega_{T\Phi}=\int_{\bf a}(\omega_{T})_{\Phi}=\int_{\bf\Phi}\omega_{T}\,. $$ The first of these equalities is Theorem 10.24, applied to Tp in place of 中 The second follows from Theorem 10.23.The third is Theorem 10.24 with or in place of o. SIMPLEXES AND CHAINS 10.26 Affine simplexes A mapping f that carries a vector space $\chi$ into a vector space Y is said to be affne iff- f(O) is linear. In other words, the require ment is that (73) $$ \mathbf{f}(\mathbf{x})=\mathbf{f}(0)+A\mathbf{x} $$ for some A∈ L(X,Y) An affine mapping of $\textstyle{\mathcal{R}}^{k}$ into $R^{n}$ is thus determined if we know f(O) and f(e,) for 1 ≤i≤ $K_{\cdot}^{\cdot}$ ; as usual, {e,, $\in_{k}!$ is the standard basis of Rk of the form We define the standard simplex $\textstyle{G^{k}}$ s to be the set of all u∈ $\textstyle{\mathcal{R}}^{k}$ (74) $$ \mathbf{u}={\frac{k}{i-1}}\mathbf{x}|\mathbf{e}| $$ k-simplex such that α,≥ 0 for i= 1,...,k and $\textstyle\mathbf{\textstyle\sum}\!{\boldsymbol{\alpha}}_{i}$ ≤1. ”. The oriented affne Assume now that po,P $\mathbb{P}_{k}$ are points of $\textstyle R^{n}$ (75) $$ \sigma=[{\mathfrak{p}}_{0},\,{\mathfrak{p}}_{1},\,\ldots,\,{\mathfrak{p}}_{k}] $$ is defined to be the ${\hat{N}}{\hat{N}}.$ -surface in $\textstyle R^{n}$ with parameter domain $\Omega^{k}$ which is given by the afine mapping (76) $$ \sigma(\alpha_{1}\mathrm{e}_{1}+\cdots+\alpha_{k}\mathrm{e}_{k})=\mathrm{p}_{0}+\sum_{i=1}^{k}\alpha_{i}(\mathrm{p}_{i}-\mathrm{p}_{0}). $$ Note that o is characterized by (77 $$ \sigma(0)={\bf p_{0}}\,,\qquad\sigma({\bf e}_{i})={\bf p}_{i}\qquad(\mathrm{for~}1\le i\le k $$ ) and that (78) $$ \sigma(\mathbf{u})=\mathbf{p}_{0}+A\mathbf{u}\qquad(\mathbf{u}\in Q^{k}) $$ where A ∈ L(R',R") and $$ A\mathbf{e}_{i}=\mathbf{p}_{i}-\mathbf{p}_{0}\,\mathrm{for}\ 1\leq i\leq k $$INTEGRATION OF DIFFERENTIAL FORMs267 is taken into account. If we call o orented to emphasize thathe ordering of te vertices p。.…, p (79) $$ \vec{\sigma}=[p_{i_{0}},p_{i_{1}},\ldots,p_{i_{k}}], $$ where {io,1,………,i,} is a permutation of the ordered set {O, 1,.…,人}, we adopt the notation (80) $$ \vec{\sigma}=s(i_{0}\,,\,i_{1},\,\cdot\,\cdot\,,\,i_{k})\sigma, $$ where sis the function defined in Definition 9.33. Thus 6= ±o,depending on whether s = 1 or s= -1.Strictly speaking, having adopted (75)and (76))a $i_{i}=k_{i}$ even the definition of o, we should not write =o unless i。 0, if s(i。,……, i,) = 1; what we have here is an equivalence relation, not an equality However, for our purposes the notation is justified by Theorem 10.27 If G = 8o (using the above convention) and if := 1, we say that G and o have the same orientation; if 8= -1,6 and o are said to have opposite orienta- tions.Note that we have not defined what we mean by the“orientation of a simplex.” What we have defined is a relation between pairs of simplexes having the same set of vertices, the relation being that of“having the same orientation.’ There is, however, one situation where the orientation of a simplex can be defined in a natural way. This happens when n = k and when the vectors P;一 Po(I ≤i≤k) are independent. In that case, the linear transformation A that appears in (78) is invertible, and its determinant (which is the same as the Jacobian of o)is not O. Then c is said to be positively (or negatively) oriented if det A is positive (or negative). In particular, the simplex [0, $\mathbf{e_{1}},\mathbf{\epsilon}\cdot\mathbf{\epsilon}\cdot\mathbf{\epsilon}\cdot\mathbf{\epsilon}\cdot\mathbf{\epsilon}\cdot\mathbf{\epsilon}\cdot\mathbf{\epsilon}$ e,] in $\textstyle{\mathcal{R}}^{k}$ given by the identity mapping, has positive orientation. So far we have assumed that k≥1. An oriented O-simplex is defined to be a point with a sign attached. We write = + Do or o = - Po·If o= epo (8 =±1)and if f is a O-form (i.e., a real function), we define $$ \bigcap_{p}f=s f(p_{0}). $$ 10.27 Theorem If o is an oriented rectilinear k-simplex in an open set $E=R^{n}$ and if G = 8o then (81) $$ \int_{\bar{\sigma}}\omega=\varepsilon\triangleright_{\sigma}\omega $$ for every k-form o in ${\mathcal{H}}^{\gamma}$ Proof For k = 0,(81) follows from the preceding definition. So we assume k ≥l and assume that o is given by (75).268 PRINCIPLES OF MATHEMATICAL ANALYSIs Suppose 1 ≤j≤k,and suppose G is obtained from o by inter- changing po and py.Then 8= -1, and $$ \tilde{\sigma}({\bf u})={\bf p}_{j}+B{\bf u}\qquad({\bf u}\in Q^{k}), $$ $B\mathbf{e}_{i}=\mathbf{p}_{i}-\mathbf{p}_{j}$ where B is the linear mapping of $\textstyle{\mathcal{R}}^{k}$ into $\textstyle{\mathcal{R}}^{n}$ defined by Be, = Po一 P), if i ≠ j. If we write $A\mathbf{e}_{i}=\mathbf{x}_{i}$ (1 <i≤k), where A is given by (78),the column vectors of B(that is,the vectors Be;) are $$ {\bf{X}_{1}}-{\bf{X}_{j}},\cdot\cdot\cdot,{\bf{X}_{j-1}}-{\bf{X}_{j}},\ -{\bf{X}_{j}},\cdot{\bf{x}_{j+1}}-{\bf{X}_{j}},\cdot\cdot\cdot,{\bf{X}_{k}}-{\bf{X}_{j}}. $$ If we subtract the jth column from each of the others, none of the deter- minants in(35) are affected, and we obtain columns $\mathbf{x}_{1},\mathbf{\boldsymbol{\cdot}}\cdot\cdot\cdot,\mathbf{x}_{j-1},\mathbf{\Lambda}-\mathbf{x}_{j}$ Xy+, .…,。、These difer from those of ${\mathcal{A}}\,$ only in the sign of the jth column. Hence (81) holds for this case. Suppose next that O<i<j≤ ${\big.}{\binom{~}{\cal K}$ and that Gis obtained from o by interchanging p: and py Then d(u) = Po + Cu, where C has the same columns as A,except that the ith and jth columns have been inter- changed. This again implies that (81) holds, since E= The general case follows,since every permutation of {0 $1,\,\ldots,\,k\rangle$ is a composition of the special cases we have just dealt with 10.28 Affine chains An affine k-chain T in an open set $E\in R^{n}$ is a collection of finitely many oriented affine k-simplexes o,….…,o, in ${\mathcal{F}}^{\dagger}$ These need not be distinct; a simplex may thus occur in T with a certain multiplicity IfT is as above, and if o is a k-form in ${\mathit{E}},$ E, we define (82) $$ \int_{\mathbf{T}}\omega=\sum_{i=1}^{r}\int_{\sigma_{i}}\omega. $$ We may view a k-surface O in ${\mathcal{F}}^{\dagger}$ as a function whose domain is the collec tion of all k-forms in ${\mathcal{F}}^{\dagger}$ and which assigns the number $\textstyle\bigcap_{\Phi}$ o to 0. Since real valued functions can be added (as in Definition 4.3), this suggests the use of the notation (83) $$ \Gamma=\sigma_{1}+\ \cdot\ \cdot\ +\ \sigma_{r} $$ or, more compactly, (84) $$ \textstyle\Gamma=\sum_{i=1}^{r}\sigma_{i} $$ to state the fact that (82) holds for every $\textstyle{\hat{F}}*$ form o in E To avoid misunderstanding, we point out explicitly that the notations introduced by (83) and(80) have to be handled with care.The point is that every oriented afine ${\hat{N}}.$ -simplex c in $\textstyle{\mathcal{R}}^{n}$ ” is a function in two ways, with different domains and different ranges, and that therefore two entirely different operationsINTEGRATION OF DIFFERENTIAL FORMs _269 of addition are possible.Originally, o was defined as an R"-valued function with domain Q'; accordingly, 0,+0。could be interpreted to be the function o that assigns the vector o,(u)十 0,(u) to every u∈ Q'; note that o is then again an oriented affine k-simplex in $\textstyle R^{n}!$ This is not what is meant by (83). For example, if o = -oq as in (80)(that is to say, if ${\cal O}_{\bf1}$ and c ${\cal O}_{\mathrm{2}}$ have the same set of vertices but are oppositely oriented) and if $1\,=\sigma_{1}+\sigma_{2}\,,$ then Jr 0 = 0 for all o, and we may express this by writing T =0 0r $\sigma_{1}+\sigma_{2}=0.$ This does not mean that o,(u) + 2(u) is the null vector of R”. 10.29 Boundaries For k ≥ 1, the boundary of the oriented afine k-simplex $$ \sigma=[{\mathfrak{p}}_{0},{\mathfrak{p}}_{1},\cdot\cdot,{\mathfrak{p}}_{k}] $$ is defined to be the affine (k - 1)-chain (85) $$ {\hat{\sigma}}\sigma=\sum_{j=0}^{k}(-1)^{j}[{\bf{p}}_{0},\ldots,{\bf{p}}_{j-1},{\bf{p}}_{j+1},\ldots,{\bf{p}}_{k}]. $$ For example,ifo = Ipo,P, P>l, then 0 = Ip, P2」- [Po,P2] + 【Po,p,]= 【po,P;]+ Ipi, P2] + Ip,PoJ which coincides with the usual notion of the oriented boundary of a triangle which occurs in(85) has For l≤i≤k, observe that the simplex o, = 【Po。,………, P;-1,Pj+1,.…,PA $Q^{k-1}$ l as its parameter domain and that it is defined by (86) $$ \sigma_{j}({\bf u})={\bf p}_{0}+B{\bf u}\qquad({\bf u}\in Q^{k-1}), $$ where $\mathbf{\mathcal{D}}$ is the linear mapping from R*- to R" determined by $$ \left.\begin{array}{l l}{{B\mathbf{e}_{i}=\mathbf{p}_{0}}}&{{(\mathbf{if}\mathbf{\beta}\mathbf{\beta}\mathbf{\beta}\mathbf{\beta}\mathbf{\beta}\geq i\leq j-1),}}\\ {{B\mathbf{e}_{i}=\mathbf{p}_{i+1}-\mathbf{p}_{0}}}&{{(\mathbf{if}\mathbf{\beta}\ {\mathit{\Phi}}\cdot i\mathbf{\beta}-1).}}\end{array}\right. $$ The simplex $$ \sigma_{0}=[{\mathfrak{p}}_{1},{\mathfrak{p}}_{2},\dotsc,{\mathfrak{p}}_{k}], $$ which also occurs in(85), is given by the mapping $$ \sigma_{0}(\mathbf{u})=\mathbf{p}_{1}+B\mathbf{u}, $$ where Be, = P:+1- Ph for l≤i≤k-1 10.30 Differentiable simplexes and chains Let ${\mathcal{J}}$ be a f"-mapping of an open set $E\subset R^{n}$ into an open set V c R"; T need not be one-to-one.lf ois an oriented affine k-simplex in ${\mathcal{F}}\$ then the composite mapping $\Phi=T\,\circ$ G(which we shall sometimes write in the simpler form To)is a k-surface in V, with parameter domain Q*. We call an oriented k-simplex of class ”.270 PRINCIPLES OF MATHEMATICAL ANALYSIS A finite collection $\Psi$ of oriented k-simplexes $\Phi_{1},\,\ldots,$ 0,of class C”in V 中 is called a k-chain of class ${\mathcal{C}}^{\prime\prime}$ in V. If o is a k-form in V, we define (87) $$ \int_{\Psi}\omega=\sum_{i=1}^{r}\ \int_{\Phi_{i}}\omega $$ and use the corresponding notation $\Psi=\Sigma\Phi_{i}$ If $\Gamma=\Sigma\sigma_{i}$ is an affine chain and if $\Phi_{i}=T\circ\sigma_{i},$ we also write Y = T。T or (88) $$ T(\sum\sigma_{i})=\sum T\sigma_{i}. $$ The boundary ap of the oriented k-simplex $\Theta=T,$ o o is defined to be the $(k-1)$ chain (89) $$ \hat{\mathcal{Q}}\Phi=T(\hat{\mathcal{Q}}\sigma). $$ In justification of(89),observe that if T is affine,then 4= T。c is an oriented affine k-simplex, in which case (89) is not a matter of definition, but is seen to be a consequence of(85). Thus(89) generalizes this special case. It is immediate that Od is of class C”if this is true of p Finally,we define the boundary OY of the k-chain Y = ED,to be the (k -1) chain (90) $$ \hat{\phi}\Psi=\sum\delta\Phi_{i}. $$ 10.31 Positively oriented boundaries So far we have associated boundaries to chains, not to subsets of R". This notion of boundary is exactly the one that is most suitable for the statement and proof of Stokes'theorem、 However,in applications, especially in $R^{2}$ or $R^{3}$ ,it is customary and convenient to talk about"oriented boundaries”’of certain sets as well.We shall now describe this briefly Let Q" be the standard simplex in T ${\boldsymbol{R}}^{n},$ ",Iet ${\mathcal{O}}_{0}$ be the identity mapping with domain Q".As we saw in Sec. 10.26, ${\mathcal{D}}_{0}$ may be regarded as a positively oriented n-simplex in R".Its boundary Oo。is an affine (n - )-chain. This chain is called the positively oriented boundary of the set ${\underline{{\mathbf{}}}}\,$ For example, the positively oriented boundary of $\textstyle\mathbb{Q}^{3}$ is $$ [\mathbf{e}_{1},\mathbf{e}_{2},\mathbf{e}_{3}]-[0,\mathbf{e}_{2},\mathbf{e}_{3}]+[0,\mathbf{e}_{1},\mathbf{e}_{3}]-[0,\mathbf{e}_{1},\mathbf{e}_{2}]. $$ theorem, Now let ${\mathcal{J}}^{\prime}$ be a 1-1 mapping of $Q^{n}$ ” into ${\boldsymbol{R}}^{n},$ of class 6”, whose Jacobian is ${\mathcal{F}}^{\gamma}$ positive (at least in the interior of Q"). Let E= T(Q"). By the inverse function is the closure of an open subset of R". We define the positively oriented boundary of the set $\stackrel{\textstyle}{\cal K}^{\prime}$ to be the (n - 1)-chain $$ {\mathcal{U}}=T(\partial\sigma_{0}), $$ and we may denote this (n - 1)-chain by OEINTEGRATION OF DIFFERENTIAL FORMS 271 An obvious question occurs here: If $E=T_{1}(Q^{n})=T_{2}(Q^{n}),$ and if both $\textstyle T_{1}$ and $\textstyle T_{2}$ have positive Jacobians,is it true that ${\mathcal{O T}}_{1}={\mathcal{O T}}_{2}.$ ?That is to say, does the equality $$ \int_{\partial T_{1}}^{\epsilon}\omega=\int_{\partial T_{2}}\omega $$ hold for every (n- 1)-form 0?The answer is yes,but we shall omit the proof (To see an example,compare the end of this section with Exercise 17.) One can go further. Let $$ \Omega=\operatorname{E}_{1}\cup\cdots\cup E_{r}, $$ where $E_{i}=T_{i}(Q^{n}),$ each $\textstyle T_{i}$ has the properties that T had above, and the interiors of the sets $\textstyle E_{i}$ are pairwise disjoint. Then the (n - l)-chain $$ {\hat{O}}T_{1}\,+\,\cdot\,\cdot\,\cdot\,\cdot\,\cdot\,\cdot\,\cdot\,\cdot\,\cdot\,\hat{O}T_{r}=\hat{O}\Omega. $$ is called the positively oriented boundary of Q For example, the unit square $\textstyle{\mathit{f}}^{2}$ in $\textstyle R^{2}$ is the union of o(Q) and o2(Q) where $$ \sigma_{1}({\bf u})={\bf u},\qquad\sigma_{2}({\bf u})={\bf e}_{1}+{\bf e}_{2}-{\bf u}. $$ Both ${\mathcal{O}}_{1}$ and ${\cal{O}}_{2}$ have Jacobian 1> 0. Since $$ \sigma_{1}=[0,\mathbf{e}_{1},\mathbf{e}_{2}],\qquad\sigma_{2}=[\mathbf{e}_{1}+\mathbf{e}_{2},\mathbf{e}_{2},\mathbf{e}_{1}. $$ we have $$ \begin{array}{l}{{\displaystyle\left.\left.\left(\left.\nu\right.1\right)\right.-\left[\displaystyle\Phi_{1},\,\mathrm{e}_{2}\right]\right.-\left[\left.\Phi_{1}+\mathrm{e}_{2},\,\mathrm{e}_{1}\right]+\left[\left.\Phi_{1}+\mathrm{e}_{2},\,\mathrm{e}_{2}\right];}}\\ {{\displaystyle\left.\hat{\left.\left.\hat{\right.}}\right. [ .\mathrm{e}_{2},\,\mathrm{e}_{1}\right]- [\mathrm{e}_{1}+\mathrm{e}_{2},\,\mathrm{e}_{1}\right]+ [\mathrm{e}_{1}+\mathrm{e}_{2},\,\mathrm{e}_{2}\right];}}\end{array} $$ The sum of these two boundaries is $$ {\mathcal{M}}^{2}=[0,\,{\mathbf{e}}_{1}]+[{\mathbf{e}}_{1},\,{\mathbf{e}}_{1}+{\mathbf{e}}_{2}]+[{\mathbf{e}}_{1}+{\mathbf{e}}_{2},\,{\mathbf{e}}_{2}]+[{\mathbf{e}}_{2},\,{\mathbf{e}}], $$ the positively oriented boundary of $\textstyle{\int}^{2}$ Note that [e,ez] canceled [ez,e,] lf O is a -surface in R", with parameter domain 1', then O(regarded as a function on 2-forms)is the same as the 2-chain $$ \Phi\circ\sigma_{1}+\Phi\circ\sigma_{2}. $$ Thus C $$ \begin{array}{c}{{\mathrm{i}\Phi=\hat{\sigma}(\Phi\circ\sigma_{1})+\hat{\sigma}(\Phi\circ\sigma_{2})}}\\ {{=\Phi(\hat{\sigma}_{1})+\Phi(\hat{\sigma}\sigma_{2})=\Phi(\hat{\sigma}{\cal I}^{2}).}}\end{array} $$ ln other words, if the parameter domain of O is the square $\nonumber\,T^{2},$ we need not refer back to the simplex Q', but can obtain Ob directly from ${\partial}I^{2}$ Other examples may be found in Exercises 17 to 19.272 PRINCIPLES OF MATHEMATICAL ANALYSIs 10.32 Example For 0≤u ≤ r,0≤v≤ 2z, define E(u,D) = (sin u cos v,sin u sin v, cos u) Then Eis a 2-surface in $R^{3}.$ whose parameter domain is a rectangle Dc R-, and whose range is the unit sphere in $\textstyle R^{3}$ .Its boundary is o× = Z(D)= 1 + "2 + 13 + 74 where $$ \begin{array}{l}{{\gamma_{1}(u)=\Sigma(u,\,0)=(\sin u,\,0,\,\cos u),}}\\ {{\gamma_{2}(v)=\Sigma(v)=(0,\,0,\,-1),}}\\ {{\gamma_{3}(u)=\Sigma(\pi-u,\,2\pi)=(\sin u,\,0,\,-\cos u),}}\\ {{\gamma_{4}(v)=\Sigma(0,\,2\pi-v)=(0,\,0,\,1),}}\end{array} $$ with [O, r] and [O,2r] as parameter intervals for u and v, respectively Since $\gamma_{2}$ and $\gamma_{4}$ a are constant, their derivatives are O, hence the integral of any l-form over yz Or Y is O.See Example 1.12(a).] Since $\gamma_{3}(u)=\gamma_{1}(\pi-u),$ direct application of (35)shows that $$ \int_{\gamma_{3}}\omega=-\int_{\gamma_{1}}\omega $$ for every l-form o. Thus $\textstyle{\left[}_{\alpha_{\mathrm{{L}}}\,\omega=0\right]}$ , and we conclude that OE = 0. (n geographic terminology,a2 starts at the north pole N, runs to the south pole $\operatorname{\it{cgn}}$ along a meridian, pauses at S, returns to V along the same meridian, and finally pauses at N. The two passages along the meridian are in opposite directions. The corresponding two line integrals therefore cancel each other In Exercise 32 there is also one curve which occurs twice in the boundary, but without cancellation.) STOKES’THEOR EM 10.33 Theorem 1f ${\mathfrak{A}}{\mathfrak{I}}$ is a k-chain of class ${\mathcal{C}}^{\prime\prime}$ in an open set $V\subset R^{m}$ and if o is a (k - 1)-form of class B’in V, then (91) $$ \int_{\Psi}d\omega=\int_{\bar{\alpha}\Psi}\omega. $$ The case $k=m=1$ is nothing but the fundamental theorem of calculus theorem, and $k=m=3$ (with an additional differentiability assumption).The case $k=m=2$ is Green'S The case $k=2,\ m=3$ gives the so-caled "divergence theorem”of Gauss is the one originally discovered by Stokes.(Spivak'snNTEGRATION OF DIFPERENTIAL FORMs273 book describes some of the historical background.) These special cases will b discussed further at the end of the present chapter. Proof It is enough to prove that (92) $$ \int_{\Phi}d\omega=\int_{\partial\Phi}\omega $$ for every oriented k-simplex $({\overline{{\mathfrak{p}}}})$ of class G”in V. For if (92) is proved and if Y = 2Q;,then (87) and (89) imply (91) Fix such a Q and put (93) $$ \sigma=[0,\mathbf{e}_{1},\dots,\mathbf{e}_{k}]. $$ Thus o is the oriented affine k-simplex with parameter domain $\textstyle{G^{k}}$ which is defined by the identity mapping.Since O is also defined on $\textstyle\Omega^{k}$ (see Definition 10.30) and Oe 6", there is an open set $E\in R^{n}$ which contains Q',and there is a C"-mapping $^{'}\!\!\!\!\!/$ of $\overline{{F}}^{+}$ into ${\boldsymbol{\gamma}}$ such that O = T。o、 By Theorems 10.25 and 10.22(c), the left side of(92) is equal to $$ \left\{\ _{T\sigma}d\omega=\int_{\sigma} (d\omega\right)_{T}=\int_{\sigma}d(\omega_{T}). $$ Another application of Theorem 10.25 shows,by (89), that the right side of (92) is $$ \hat{\tilde{\ |}_{\delta(T\sigma)}}\,\omega=\hat{\rangle}_{T(\delta\sigma)}\,\omega=\hat{\rangle}_{\hat{\alpha}\sigma}\,\omega_{T}\,. $$ Since $\omega_{T}$ is a $(k-1)$ -form in $E,$ we see that in order to prove (92) we merely have to show tha (94) $$ \int_{\sigma}d\lambda=\int_{\partial\sigma}\lambda $$ for the special simplex(93) and for everyk - 1)-form of classG’ in E lf k = 1,the definition of an oriented O-simplex shows that(94) merely asserts that (95) $$ \{\O_{0}^{1}f^{\prime}(u)\,d u=f(1)-f(0) $$ for every continuously differentiable function $\mathbb{Z}_{\mathbf{Y}}$ on [0,1], which is true by the fundamental theorem of calculus. From now on we assume that k> 1,fix an integer r(1 ≤r≤k) and choose f∈ 6'((E). It is then enough to prove (94) for the case (96) $$ \lambda=f(\mathbf{x})\,d x_{1}\wedge\mathbf{\partial}\cdot\cdot\cdot\mathbf{\partial}\wedge d x_{r-1}\wedge d x_{r+1}\wedge\cdot\cdot\cdot\mathbf{\partial}\wedge d x_{i} $$ since every (k - l)-form is a sum of these special ones, for r=1,...,人k274 PRINCIPLES OF MATHEMATICAL ANALYSIs By (85), the boundary of the simplex (93) is $$ {\partial}\sigma=[{\bf e}_{1},\cdot\cdot\cdot,{\bf e}_{k}]+\sum_{i=1}^{k}(-1)^{i}\tau_{i} $$ where $$ {\mathcal{C}}_{i}\Longrightarrow\left[\Theta_{\right.,{\phantom{\frac{1}{2}}}} .\circ_{i}\circ\circ\O_{\left.i-1,\phantom{\frac{.}{\theta}_{i}}\right.\kern\cdot .,\Pi}\circ\left.\right.\circ_{i} ] $$ for i= 1,.….,k.Put $$ \tau_{0}=[\mathbf{e}_{r},\,\mathbf{e}_{1},\,\cdot\,\cdot\,.\,\mathbf{e}_{r-1},\,\mathbf{e}_{r+1},\,\cdot\,\cdot\,.\,\mathbf{e}_{k}]. $$ Note that $\tau_{0}$ is obtained from $ [\Theta_{1},\ \cdot\cdot\cdot,$ e,] by r-l successive interchanges of e, and its left neighbors. Thus (97) $$ \partial\sigma=(-1)^{r-1}\tau_{0}+\sum_{i=1}^{k}(-1)^{i}\tau_{i}\,. $$ Each r; has $Q^{k-1}$ as parameter domain. I1 $\mathbf{f}\mathbf{x}=\tau_{0}(\mathbf{u})$ and u∈ $Q^{k-1}$ , then (98) $$ x_{j}=\underbrace{\binom{u_{j}}{1-(u_{1}+\cdot\cdot\cdot\cdot\cdot+u_{k-1})}}_{(u_{j-1}}\qquad\underbrace{(j\leq j<r),}_{(r<j\leq k).} $$ lf l≤i≤k,ue $Q^{k-1}$ ,and x = ,u), then (99) $$ x_{j}=\begin{array}{c c}{{\left(u_{j}\qquad}}&{{(1\le j<i),}}\\ {{0\qquad}}&{{(j=i),}}\\ {{u_{j-1}\qquad}}&{{(i<j\le k).}}\end{array}\right. $$ For O≤i≤k, let $\vartheta_{\mathrm{~\Big<}}\bigg>_{\frac{1}{2}}$ be the Jacobian of the mapping (100) $$ (u_{1},\cdot\cdot\cdot,\ u_{k-1})\to(x_{1},\ .\cdot\cdot\cdot\cdot r_{r-1},\ x_{r+1},\ \cdot\cdot\cdot,\ x_{k}) $$ induced by $\textstyle\bar{{\cal Z}_{\mathit{\Phi}}}_{\stackrel{.}{\bar{i}}}$ $i=0$ and when i=r,(98) and(99) show that (100 .Thus T;,When ×.= 0 in (99) sows that is the identity mapping. Thus J。 = 1, J, = 1. For other i, the fact that $\scriptstyle J_{i}=0$ ${\widehat{\mathcal{Y}}}_{i}$ has a row of zeros, hence (101) $$ \int_{\mathbf{r}_{l}}\lambda=0\qquad(i\neq0,i\neq r), $$ by (35) and (96). Consequently,(97) gives (102) $$ \begin{array}{r l}{\int_{\partial\sigma}{\lambda}=(-1)^{r-1}\int_{\tau_{0}}{\lambda}+(-1)^{r}\int_{\tau_{r}}{\lambda}}\\ {\ }&{{}}\\ {=(-1)^{r-1}\int~~~~~[f(\tau_{0}({\bf u}))-f(\tau_{r}({\bf u}))]\,d{\bf u}}\end{array} $$nNTECGRATioN or DFrRENTIAL FoRMs 275 On the other hand, C $$ \begin{array}{l}{{J\rangle=(D_{r}f)(\mathbf{x})d x_{r}\wedge d x_{1}\wedge\ \cdots\wedge d x_{r-1}\wedge d x_{r+1}\ \wedge\ \cdots\wedge d x_{k}}}\\ {{=(-1)^{r-1}[D_{-}f](\mathbf{x}]\,d x_{1}\ \wedge\ \cdots\wedge d x_{t}}}\end{array} $$ so that (103) $$ \int_{\sigma}d\lambda=(-1)^{r-1}\int_{Q^{k}}(D_{r}f)(\mathbf{x})\,d\mathbf{x} $$ We evaluate (103) by first integrating with respect to x, over the interval $$ [0,1-(x_{1}+\cdots+x_{r-1}+x_{r+1}+\cdots+x_{k})], $$ put (x,.….,x,-1,X,+1, …,x) = $\left({\cal U}_{1},\ \circ\ ,\ \cdot\ {\cal U}_{k-1}\right),$ ,and see with the aid of $Q^{\kappa-1}$ (98) that the integral over Q'in(103)is equal to the integral over in(102)、 Thus (94) holds, and the proof is complete. CLOSED FORMS AND EXACT FORMS 10.34 Definition Let o be a k-form in an open set Ec R”.If there is a (k -1) form Xin E such that 0 = dR, then o is said to be exact in $\widehat{F},$ lf o is of class G' and do = 0, then o is said to be closed Theorem 10.20(6) shows that every exact form of class ${\mathcal{C}}^{\prime}$ is closed In certain sets E, for example in convex ones,the converse is true; this is the content of Theorem 10.39(usually known as Poincaré's lemma) and Theorem 10.40. However, Examples 10.36 and 10.37 will exhibit closed forms that are not exact. 10.35 Remarks (a) Whether a given k-form o is or is not closed can be verified by simply differentiating the coefficients in the standard presentation of For example,a 1-form (104) $$ \omega=\sum_{i=1}^{n}f_{i}({\bf x})\,d x_{i}, $$ with f;∈ 6′(E)for some open set $E\subset R^{n},$ ,,is closed if and only if the equations (105 $$ (D_{j}f_{i})({\bf x})=(D_{i}f_{j})({\bf x}) $$ hold for all i, jin{1,...,n} and for all xe E276 PRINCTPLES OF MATHEMATICAL ANALYSIS Note that (105) is a“pointwise” condition; it does not involve any global properties that depend on the shape of E. On the other hand, to show that o is exact in $\textstyle E,$ one has to prove 。 the existence of a form 入,defined in E, such that dA = 0. This amounts to solving a system of partial differential equations, not just locally,but in all of E. For example,to show that (104) is exact in a set $\textstyle E,$ , one has to find a function (or O-form) g ∈ G'(E) such that (106) ( $$ \begin{array}{r l}{D_{i}g)(\mathbf{x})=f_{i}(\mathbf{x})}&{{}\quad(\mathbf{x}\in E,\;1\leq i\leq n).}\end{array} $$ Of course,(105) is a necessary condition for the solvability of (106) (b) Let o be an exact k-form in E. Then there is a (k - 1)-form 入 in ${\mathcal{F}}$ with $d\lambda=\omega,$ and Stokes' theorem asserts that (107) $$ \int_{\Psi}\omega=\int_{\Psi}d\lambda=\int_{\omega\Psi}\lambda $$ for every k-chain P of class ${\mathcal{C}}^{\prime\prime}$ in E. $\Psi$ If $\Psi_{1}$ and Y, are such chains, and if they have the same boundaries it follows that $$ \Bigr\{_{\Psi_{1}}\omega=\int_{\Psi_{2}}\omega. $$ In particular, the integral of an exact k-form in E is O over every k-chain in ${\mathcal{F}}$ whose boundary is O. As an important special case of this,note that integrals of exact 1-forms in ${\widetilde{F}},$ are O over closed (differentiable) curves in E. (c) Let o be a closed k-form in E. Then do = 0,and Stokes' theorem asserts that (108) $$ \int_{\partial\Psi}\omega=\int_{\Psi}d\omega=0 $$ for every (k +1)-chain Y of class ${\mathcal{C}}^{\prime\prime}$ ”in E In other words,integrals of closed k-forms in E are O over k-chains that are boundaries of (k +1)-chains in E. (d)Let Y be a (k + 1-chain in $d^{2}\lambda=0,$ two applications of Stokes' theorem show tha ${\mathcal{H}}$ and let be a(k- 1)-form in E, both of class G”.Since (109) $$ \int_{\partial\Phi}\lambda=\int_{\partial\Psi}d\lambda=\int_{\Psi}d^{2}\lambda=0. $$ We conclude that a-P = 0. In other words,the boundary of boundary is O. See Exercise 16 for a more direct proof of thisINTEGRATION OF DIFFERENTIAL FORMS277 10.36 Example Let $E=R^{2}-\{0\},$ the plane with the origin removed. The 1-form (110) $$ \eta={\frac{x\,d y-y\,d x}{x^{2}+y^{2}}} $$ is closed in $\scriptstyle R T=\langle D|$ 、 This is easily verified by differentiation. Fix r> 0, and define (111) $$ \gamma(t)=(r\cos t,\,r\sin\,t)\qquad(0\leq t\leq2\pi). $$ Then yis a curve (an“oriented l-simplex'") in R - 0}. Since y(O = y(2r) we have (112) $$ {\bar{\partial}}\gamma=0. $$ Direct computation shows that (113) $$ \int_{\gamma}\eta=2\pi\neq0. $$ The discussion in Remarks 10.35(b) and (c) shows that we can draw two conclusions from(113): First, nis not exact in $R^{2}$ -{0}, for otherwise(112) would forc the integral (113) to be O. Secondly,y is not the boundary of any 2-chain in $\scriptstyle R T=(0)$ (of class G") for otherwise the fact that ${\mathcal{V}}_{j}$ is closed would force the integral(113) to be O. 10.37Example Let E = R - 0},3-space with the origin removed. Define (114) $$ \displaystyle{\mathrm{\Large~v}_{\mathrm{~}}}={\frac{x\,d y\,\wedge\,d z+y\,d z\,\wedge\,d x+z\,d x\,\wedge\,d y}{(x^{2}+y^{2}+z^{2})^{3/2}}} $$ where we have written(x,y, z)in place of $(x_{1},\,x_{2}\,,\,x_{3})$ Differentiation shows that dt = 0, so that C is a closed 2-form in $\scriptstyle R T=(0)$ recall that Let E be the 2-chain in $R^{3}-\{0\}$ that was constructed in Example 10.32 $\sum_{i\in i}$ is a parametrization of the unit sphere in $R^{\ddagger}$ . Using the rectangle $\underline{{{\J}}}$ of Example 10.32 as parameter domain, it is easy to compute that (115) $$ \int_{\Sigma}\mathbf{v}=\int_{D}\sin u\,d u\,d v=4\pi\neq0. $$ As in the preceding example, we can now conclude that Cis not exact in $\scriptstyle R T=(0)$ (since O2 = 0, as was shown in Example 10.32) and that the sphere $\sum$ is not the boundary of any 3-chain in $\scriptstyle R T=(M)$ (of class C”), although OE = 0. The following result will be used in the proof of Theorem 10.39278 PRINCIPLES OF MATHEMATICAL ANALYsIS 10.38 Theorem Suppose $\widehat{R}^{*}$ ${\dot{\bar{\iota}}}{\boldsymbol{\mathsf{J}}}$ a convex open set in ${\boldsymbol{R}}_{',\ }$ f∈ G(E), p is an integer, 1 ≤p≤n, and (116) $$ (D_{j}f)({\bf x})=0\;\;\;\;\;\;\;\;(p<j\leq n,\;{\bf x}\in E). $$ Then there exists an $F\in{\mathcal{G}}^{\prime}(E)$ such that (117) (D,FXx) =/(x) (D,F)(x) = 0 (p <j≤n,x∈ E) Proof Write $\mathbf{x}=(\mathbf{x}^{\prime},\,x_{p},\,\mathbf{x}^{\prime}),$ where $$ {\bf x}^{\prime}=(x_{1},\ .\ .\ ,\ x_{p-1}),\ {\bf x}^{\prime\prime}=(x_{p+1},\ .\cdot\cdot,\,x_{n}). $$ (When $p=1,$ $\chi_{\mathrm{{B}}}^{\prime}$ is absent; when $p=n,\ \mathbf{x}^{*}$ ”is absent.) Let V be the projection of $E,\;V$ is a convex open set in $\textstyle R^{p}.$ P、Since x,, X”)∈ E for some x”. Being a set of all $(\mathbf{x}^{\prime},\,x_{p})\in R^{p}$ such that (x,X $X_{p}$ is convex and (116 ${\mathcal{F}}$ holds, J(x) does not depend on $\mathbf{x}^{\prime\prime}$ .Hence there is a function (p,with domain ${\mathit{V}}_{\mathrm{,}}$ such that for all $$ f(\mathbf{x})=\varphi(\mathbf{x}^{\prime},\,x_{p}) $$ $x\in E.$ If $\scriptstyle p\;=\;1,$ V is a segment in $\textstyle{\mathcal{R}}^{1}$ (possibly unbounded). Pick c∈ V and define $$ F(\mathbf{x})=\int_{c}^{x_{1}}\varphi(t)\,d t\qquad(\mathbf{x}\in E). $$ If p > 1,let $\bar{\boldsymbol{\{}}$ l be the set of al $\mathbf{x}^{\prime}\in R^{p-1}$ such that (x,xp)∈ ${\mathfrak{I}}^{\gamma}$ for some x, Then $\textstyle{\hat{U}}$ is a convex open set in $R^{p-1}.$ ,and there is a function αe6(U) such that ((x’,c(x'))e V for every x'e U;in other words, the graph of c lies in V (Exercise 29). Define $$ {\cal F}({\bf x})=\int_{\alpha({\bf x}^{\prime})}^{\alpha r}\,\varphi({\bf x}^{\prime},\,t)\,d t\qquad({\bf x}\in{\cal E}). $$ In either case, ${\mathcal{H}}^{\prime}$ satisfies (117) (Note:Recll the usual convention that $\stackrel{\bullet}{\bigcup}\varnothing$ means -Gs if b<a.) 10.3 Theorem If Ec R” is convex and open, if k≥1, ifのis a k-form o class ${\mathcal{C}}^{\prime}$ in $\textstyle E,$ and if do = 0, then there is a (k- 1)-form 入 in $\overline{{G}}$ such that o = dA Briefly, closed forms are exact in convex sets Proof For p = ,.…, let $\textstyle{\bar{Y}}_{p}$ denote the set of all k-forms oの, of class C'′in E, whose standard presentation (118) $$ \omega=\sum_{I}f_{I}(\mathbf{x})\,d x_{I} $$ does not involve $d x_{p+1},\,\ast,\,d x_{n}$ . In other words,I∈ {1,..…,p}if fAKx) 0 for some $\mathbf{x}\in E$INrEGRAroN or DlrrERENTIAL roRMs 279 we shall proced by induction on p Assume first that o∈ Y. Then 0 = f(x) dx.Since do = 0 (D,f)(x) = 0 for 1<j≤,,x∈ E. By Theorem 10.38 there is an FeW(E such that D,F=f and D,F = 0 for 1<j≤n、 Thus $$ d F=(D_{1}F)(\mathbf{x})\,d x_{1}=f(\mathbf{x})\,d x_{1}=\omega. $$ Now we take $p>1$ and make the following induction hypothesis: Every closed k-form that belongs to $\textstyle Y_{p}$ so that do = 0.By (118), $Y_{p-1}$ is exact in E Choose o ∈ (119) $$ \sum_{I}\sum_{j=1}^{n}(D_{J}f_{I})({\bf x})\;d x_{J}\wedge d x_{I}=d\omega=0. $$ Consider a fixed j, with p<j≤n. Each I that occurs in(118)lies in 1,..…,p)、If I, I, are two of these k-indices, and if I, ≠ I,,then the (k + 1)-indices (I,),(I,,j) are distinct. Thus there is no cancellation, and we conclude from (119)that every coefficient in (118) satisfies (120) $$ (D_{j}f_{I})({\bf x})=0\;\;\;\;\;\;\;({\bf x}\in E,p<j\leq n). $$ We now gather those terms in (118) that contain dx $d x_{p}$ , and rewrite o in the form (121) $$ \omega=\alpha+\sum_{I_{0}}f_{I}({\bf x})\;d x\,t_{0}\,\wedge\,d x_{p}\,, $$ and where ae Y,-, cach $\mathbf{\mathcal{L}}_{0}$ 。is an increasing (k - 1)-index in 1, .…, -1} ∈ B′(E $I=(I_{0},p)$ By (120), Theorem 10.38 furnishes functions ${\mathcal{F}}_{Y}$ such that (122) $$ D_{p}F_{I}=f_{I},\qquad D_{J}F_{I}=0\qquad(p<j\leq n). $$ Put (123) $$ \beta=\sum_{I_{0}}F_{I}(\mathbf{x})\,d x_{I_{0}} $$ and define $\gamma=\omega-(-1)^{\kappa-1}$ :- dβ. Since β is a (k- 1)-form, it follows that $$ \begin{array}{c}{{\gamma=\omega-\sum_{I_{0}}\sum_{l=1}^{P}(D_{J}F_{I})({\bf x})\,d x_{I_{0}}\wedge d x_{J}}}\\ {{=\alpha-\sum_{I_{0}}^{P-1}(D_{J}F_{I})({\bf x})\,d x_{I_{0}}\wedge d x_{J}\,,}}\end{array} $$ which is clearly in Yp-1:Since do = 0 and $d^{2}\beta=0$ 、we have $\scriptstyle d_{f^{-}}\;d_{f^{-}}$ (k -1)-form p in E. If Our induction hypothesis shows therefore that y = du for some $\scriptstyle{|{\hat{A}}=\mu|}$ 1 +(-1*-1 β,we conclude that 0 = dA、 By induction, this completes the proof.280 rNCIPLEs Or MATHEMArICAL ANALxsis 10.40 Theorem Fix k,1≤k≤n. Let Ec R" be an open set in which every onto an open set $U\vDash R^{n}$ closed k-form is exact.、 Let T be a 1-1 6"-mapping of $\widehat{F}$ whose inverse S is also of class B”. Then every closed k-form in U is exact in $U$ Note that every convex open set E satisfies the present hypothesis,by Theorem 10.39. The relation between ${\widehat{\operatorname{F}^{\prime}}}$ and ${\hat{G}}J$ may be expressed by saying that they are G"-equivalent. Thus every closed form is exact in any set which is G"-equivalent to a convex open set 0- is a k-form in E for which Proof Let oo be a k-form in U, with do = 0. By Theorem 10.22(c 0. Hence 0- = dA for some Theorem 10.22c0, $d(\omega_{T})=1$ k一 )-form in E. By Theorem 10.23,and another application of $$ \omega=(\omega_{T})_{S}=(d\lambda)_{S}=d(\lambda_{S}). $$ Since As is a (k -1)-form in U/,o is exact in $U,$ 10.41 Remark In applications, cells (see Definition 2.17) are often more con venient parameter domains than simplexes. If our whole development had been based on cells rather than simplexes, the computation that occurs in the proof of Stokes'theorem would be even simpler.(It is done that way in Spivak's book.) The reason for preferring simplexes is that the definition of the boundary of an oriented simplex seems easier and more natural than is the case for a cell. (See Exercise 19.))Also, the partitioning of sets into simplexes (called “triangu- lation') plays an important role in topology, and there are strong connections between certain aspects of topology,on the one hand,and differential forms on the other. These are hinted at in Sec. 10.35. The book by Singer and Thorpe contains a good introduction to this topic Since every cell can be triangulated, we may regard it as a chain. For dimension 2, this was done in Example 10.32;for dimension 3, see Exercise 18. Poincaré's lemma(Theorem 10.39) can be proved in several ways. See. for example, page 94 in Spivak's book, or page 280 in Fleming's. Two simple proofs for certain special cases are indicated in Exercises 24 and 27. VECTOR ANALYSIS We conclude this chapter with a few applications of the preceding material to theorems concerning vector analysis in R3. These are special cases of theorems about differential forms, but are usually stated in different terminology. We are thus faced with the job of translating from one language to another.INTEGRATION oF DIFFERENTIAL FORMs 281 an open set $E\subset R^{3}$ 10.42 ector fields Let F = F,e + F,e2 + Fses be a continuous mapping of into R. Since F associates a vector to each point of E,F is sometimes called a vector field,especially in physics. With every such Fis associated a 1-form (124) $$ \lambda_{\mathrm{F}}=F_{1}\,d x+F_{2}\,d y+F_{3}\,d z $$ and a2-form (125) 0, = F, dy 八 dz + Fz dz A dx + F, dx A dy Here, and in the rest of this chapter, we use the customary notation (x, y, 2 in place of (x,,Xz,Xa). It is clear, conversely, that every l-form 入in ${\mathcal{H}}$ is kr for some vector field F in E, and that every 2-form o is op for some F. In $R^{3}.$ the study of l-forms and 2-forms is thus coextensive with the study of vector fields. If ue G′(E) is a real function, then its gradieni $$ \nabla u=(D_{1}u)\mathbf{e}_{1}+(D_{2}u)\mathbf{e}_{2}+(D_{3}u)\mathbf{e}_{3} $$ is an example of a vector field in E. Suppose now that $\textstyle{\frac{\nabla}{\Delta^{\dagger}}}$ is a vector field in E, of class ${\mathcal{C}}^{\prime}$ 、Its cur1 $\nabla\times\mathbf{F}$ F is the vector field defined in ${\overline{{\mathcal{H}}}},$ by V × F=(D, Fs- D, Fz)e +(D, F- D,F)e. +(D,Fz- D, F)e and its divergence is the real function V·F defined in $\widehat{F}$ by $$ \nabla\cdot{\mathbf{F}}=D_{1}F_{1}+D_{2}\,F_{2}+D_{3}\,F_{3}\,. $$ These quantities have various physical interpretations. We refer to the book by O. D. Kellogg for more details. Here are some relations between gradients, curls, and divergences 10.43 Theorem Suppose E is an open set in $R^{3}$ 。ueG"(E), and $\widehat{\left({\begin{array}{c}{\strut}\\ {\Re}\cdots}\end{array}}\widehat{ ({\strut{\Re}}\right)}$ is a vector field in E, of class C”. (a)If F = Vu, then V× F = 0. (b)If F = V× G, then V·F = 0. Furthermore,if E is G"-equivalent to a convex set, then(a) and (b) have converses, in which we assume that $\frac{\mathbb{Q}}{\mathbb{Q}^{1}}$ is a vector field in $\textstyle E,$ of class G′: (a’)If V× F = 0, then F = Vu for some u∈ G”(E) (b′)IfV·F = 0,then F =V× G for some vector feld G in E,of lass GC following four statements: Proof If we compare the definitions of Vu,V×F,and V·F with the diferential forms $\lambda_{\mathrm{F}}$ and or given by (124) and (12), we obtain th282 PRINCrPLEs Or MATHEMATICAL ANALYsis $$ \begin{array}{r l}{{\mathrm{F}=\nabla u}}&{{\mathrm{if~and~only~if~}}~~\lambda_{\mathrm{F}}=d u.}\\ {{\mathrm{V}\times{\mathrm{F}}=0}}&{{\mathrm{if~and~only~if~}}~~\ d\lambda_{\mathrm{F}}=0.}\\ {{\mathrm{C}\times{\mathrm{F}}=0}}&{{\mathrm{if~and~only~if~}}~~\omega_{\mathrm{F}}=d\lambda_{\mathrm{G}}.}\\ {{\mathrm{V}\cdot{\mathrm{F}}=0}}&{{\mathrm{if~and~only~if~}}~~~d\omega_{\mathrm{F}}=0.}\end{array} $$ Now if F = Vu, then Ap = du, hence dAp = d-u = 0(Theorem 10.20) which means that V×F = 0. Thus (a) is proved. As regards (a'), the hypothesis amounts to saying that $d\lambda_{\mathrm{F}}=0$ in E. By Theorem 10.40,入,= du for some O-form u. Hence F = Vu. The proofs of (b) and(b’)follow exactly the same pattern 10.44 Volume elements The k-form $$ d x_{1}\wedge\ \cdot\cdot\cdot\ \wedge\ d x_{k} $$ is called the volume element in $R^{k}$ .It is often denoted by $d V$ (or by $d V_{k}$ if it seems desirable to indicate the dimension explicitly), and the notation (126) $$ \int_{\phi}f(\mathbf{x})\,d x_{1}\wedge\mathbf{\theta}\cdot\mathbf{\theta}\cdot\mathbf{\theta}\wedge d x_{k}=\int_{\phi}f d V $$ is used when $\langle{\overline{{\mathbf{p}}}}\rangle$ is a positively oriented k-surface in $\textstyle{\mathcal{R}}^{k}$ and $\oint_{0}^{a}$ is a continuous function on the range of O domain in $R^{k}{}_{,}$ The reason for using this terminolog is very simple::1f $R^{k},$ with positive Jacobian and if dD is a 1-1 6'-mapping of D into ${\hat{H}}\rangle$ is a parameter Jo,then the left side of (126) is $$ \int_{D}f(\Phi(\mathbf{u}))J_{\Phi}(\mathbf{u})\,d\mathbf{u}=\int_{\Phi(D)}f(\mathbf{x})\,d\mathbf{x}, $$ by (35) and Theorem 10.9. In particular, when f = 1,(126) defines the volume of . We already saw a special case of this in (36) The usual notation for dV is dA 10.45 Green's theorem Suppose ${\widetilde{F}}^{\dagger}$ is an open set in $R^{2},$ ,α∈ B′(E),β ∈ G'(E), and S is a closed subset of ${\mathcal{E}},$ with positively oriented boundary aQ,as described in Sec. 10.31. Then (127) $$ \int_{\partial\Omega}(\alpha\;d x+\beta\;d y)=\int_{\Omega}\left({\frac{{\dot{\phi}}{\dot{g}}}{{\dot{\phi}}x}}-{\frac{{\dot{\omega}}x}{{\dot{\sigma}}y}}\right)d A. $$INTEGRATION OF DIFFERENTIAL FORMS 283 Proof Put.=α dx + β dy. Then $$ \begin{array}{c}{{d\lambda=(D_{2}\alpha)\,d y\,\wedge\,d x+(D_{1}\beta)\,d x\,\wedge\,d y}}\\ {{=(D_{1}\beta-D_{2}\alpha)\,d A,}}\end{array} $$ and (127) is the same as $$ \textstyle\int_{\Omega}\lambda=\int_{\Omega}d\lambda, $$ which is true by Theorem 10.33 With a(x, J) = -y and β(x, J) = x,(127) becomes (128) $$ \textstyle{\frac{1}{2}}\int_{\partial\Omega}(x\,d y-y\,d x)=A(\Omega), $$ the area of $\Omega$ With α= 0, β = x,a similar formula is obtained. Example 10.12(b) con- tains a special case of this. 10.46 Area elements in $\textstyle\mathcal{Q}^{3}$ Let Q be a 2-surface in $R^{3},$ 3,of class G',with pa- rameter domain $\scriptstyle D=R^{n}$ Associate with each point (u, D)e D the vector (129) $$ \mathrm{N}(u,v)={\frac{\partial(y,z)}{\partial(u,v)}}\mathrm{e}_{1}+\frac{\partial(z,x)}{\partial(u,v)}\mathrm{e}_{2}+\frac{\partial(x,y)}{\partial(u,v)}\mathrm{e}_{3}\,. $$ The Jacobians in(129) correspond to the equation (130) $$ (x,y,z)=\Phi(u,v). $$ If f is a continuous function on 0(D),the area integral of f over D is defined to be (131) $$ \int_{\phi}f\,d A=\int_{D}f(\Phi(u,v))\,|\,\mathrm{N}(u,v)|\,\,d u\,d v. $$ In particular, when f = 1 we obtain the area of O, namely (132) $$ A(\Phi)=\int_{D}|\mathbf{N}(u,v)|\;d u\;d v. $$ The following discussion will show that(131) and its special case (132 are reasonable definitions. It will also describe the geometric features of th vector N. N = N(Opo). put Write = 0,e. + 92。+ 93e,,fx a point Po = (uo,Do)e D、 put (133) α,=(Dp)Opo),β.=(D, 9DGPo) (i= 1, 2,3)284 PRINCIPLES OF MATHEMATICAL ANALysIs and let Te L(R3,R')be the linear transformation given by (134) $$ T(u,v)=\sum_{i=1}^{3}(\alpha_{i}u+\beta_{i}v)\mathrm{e}_{i}\,. $$ Note that $T=\Phi^{\prime}({\mathfrak{p}}_{0}),$ in accordance with Definition 9.11 Let us now assume that the rank of T is 2.(If it is l or O,then N = 0,and the tangent plane mentioned below degenerates to a line or to a point.)The range of the affine mapping $$ (u,v)\to\Phi({\mathfrak{p}}_{0})+T(u,v) $$ is then a plane I, called the tangent plane to $\langle{\overline{{\mathbf{p}}}}\rangle$ at po、IOne would like to cal II the tangent plane at D(Po), rather than at $\mathfrak{p}_{0}$ ; if $({\overline{{\mathfrak{h}}}})$ is not one-to-one, this runs into difficulties.】 If we use(133) in(129), we obtain (135) N =(α2βs-αsβ2)e +(αsβ 一α1β3)e2 + (α,β2- α2 β,)e、 and(134) shows that (136) $$ T\mathbf{e}_{1}=\sum_{i=1}^{3}\alpha_{i}\mathbf{e}_{i},\qquad T\mathbf{e}_{2}=\sum_{i=1}^{3}\beta_{i}\mathbf{e}_{i}. $$ A straightforward computation now leads to (137) $$ {\bf N}\cdot(T\mathbf{e_{1}})=0={\bf N}\cdot(T\mathbf{e_{2}}). $$ Hence N is perpendicular to I. t is therefore calldthe mormal to $({\overline{{\mathfrak{h}}}})$ a po A second property of N, also verified by a direct computation based on (135) and (136), is that the determinant of the linear transformation of $R^{\ 3}$ R that takes {e,,e2,ey} to {Te,, Tez,N} is |N| > 0 (Exercise 30). The 3-simplex (138) $$ [0,T\mathrm{e}_{i},T\mathrm{e}_{2},\mathrm{N}] $$ is thus positively oriented The third property of Nthat we shall use is a consequence of the first two The above-mentioned determinant, whose value is IN|,is the volume of the parallelepiped with edges [O, Te,],[0,Te_],[0,N].By(137)、[,N] is perpen- dicular to the other two edges.The area of the parallelogram with vertices (139) $$ 0,T\mathrm{e}_{1},T\mathrm{e}_{2},T(\mathrm{e}_{1}+\mathrm{e}_{2}) $$ is therefore |N This parallelogram is the image under T of the unit square in $R^{2}$ 1f ${\mathcal{F}}^{\dagger}$ is any rectangle in R,it follows (by the linearity of T)that the area of the parallelogram T(E)is (140) $$ A(T(E))=|\mathbf{N}|A(E)=\int_{E}|\mathbf{N}(u_{0},v_{0})|\;d u\;d v. $$rNTEGRATiON OF DIFFERENTIAL FrORMs 285 We conclude that (132) is correct when dis affine. To justify the definition (132) in the general case, divide D into small rectangles, pick a point (uo,Do) in each, and replace O in each rectangle by the corresponding tangent plane The sum of the areas of the resulting parallelograms, obtained via (140), is then an approximation to A(O)、Finally, one can justify 131 from(132) by approxi mating f by step functions. 10.47 Example Let O<a<b be fixed. Let K be the 3-cll determined by $$ 0\leq t\leq a,\qquad0\leq u\leq2\pi,\qquad0\leq v\leq2\pi. $$ The equations (141) $$ \begin{array}{l}{{x=t\cos u}}\\ {{y=(b+t\sin u)\cos v}}\\ {{z=(b+t\sin u)\sin v}}\end{array} $$ describe a mapping Y of $\textstyle R^{3}$ into $R^{\ddagger}$ which is 1-1 in the interior of $K_{\mathrm{{J}}}$ such that Y(K) is a solid torus. Its Jacobian is $$ J_{\Psi}=\frac{\hat{Q}(x,y,z)}{\hat{Q}(t,u,v)}=t(b+t\sin u) $$ which is positive on K,except on the face t = 0. If we integrate $J_{\Psi}$ over K, we obtain $$ \mathrm{vol}\left(\Psi(K)\right)=2\pi^{2}a^{2}b $$ as the volume of our solid torus. Now consider the 2-chain 0 = 0Y.(See Exercise 19.)Y maps the faces u = O and u= 2m of K onto the same cylindrical strip, but with opposite orienta tions.Y maps the faces $v=0$ and v= 2r onto the same circular disc,but with opposite orientations.Y maps the face t = 0 onto a circle, which contributes O to the 2-chain OY.(The relevant Jacobians are O.) Thus dD is simply the 2-surface obtained by setting t= a in(141), with parameter domain D the square defined by O≤u ≤ 2元,0≤v ≤2r According to (129) and (141), the normal to D at(u,D)∈ D is thus the vector N(u, D)= a(b + a sin u)n(u,U) where n(u,D) = (cos u)e + (sin u cos D)e + (sin usin v)ea286 PRINCIPLES OF MATHEMATICAL ANALYs Since ln(u, D)| =1, we have |N(u,D)| = a(b + a sin ub), and if we integrate this over D,(131) gives $$ A(\Phi)=4\pi^{2}a b $$ as the surface area of our torus. If we think of N = N(u,v)as a directed line segment, pointing from D(u,D)to O(u,v)+ N(u,D), then N points outward, that is to say, away from Y(K).This is so because $\ {\bf J}_{\Psi}>0$ when 1= a. For example, take $u\equiv v\simeq\pi/2,$ 1 = Q. This gives the largest value of z on Y(K), and N = a(b + a)es points “upward” for this choic of u, D) 10.48 Integrals of 1-forms in R Let ybe a B'-curve in an open set Ec R3 with parameter interval [O,1], Iet $\frac{\mathbb{P}^{\nu}}{\mathbb{N}^{\nu}}$ be a vector field in $\textstyle E_{\mathrm{{J}}}$ as in Sec. 10.42, and define xp by (124). The integral of .p over y can be rewritten in a certain way which we now describe. For any ue [0, 1 $$ \gamma^{\prime}(u)=\gamma_{1}^{\prime}(u)\mathbf{e}_{1}+\gamma_{2}^{\prime}(u)\mathbf{e}_{2}+\gamma_{3}^{\prime}(u)\mathbf{e}_{3} $$ is called the tangent vector to y at u. We define $\mathbf{t}=\mathbf{t}(u)\,\mathbf{t}$ to be the unit vector in the direction of y'(u)、 Thus $$ \gamma^{\prime}(u)=\mid\gamma^{\prime}(u)|\,[(u). $$ [If y'(u) = 0 for some ${\mathcal{U}}_{>}$ put $\mathbf{t}(u)=\mathbf{e}_{1}$ ; any other choice would do just as well. By (35), (142) $$ \begin{array}{c}{{\lambda_{\mathrm{{F}}}=\sum_{i=1}^{\infty}\int_{0}^{i}F_{i}(\gamma(u))\gamma_{i}^{\prime}(u)\,d u}}\\ {{\ }}\\ {{=\int_{0}^{1}\mathbf{F}(\gamma(u))\cdot\gamma^{\prime}(u)\,d u}}\end{array} $$ Theorem 6.27 makes it reasonable to call ly(u)| du the element of ar length along y.A customary notation for it is ds, and (142) is rewritten in the form (143) $$ \int_{\gamma}\lambda_{F}=\int_{\gamma}(\mathbf{F}\cdot\mathbf{t})\,d s. $$ of F along y Since tis a unit tangent vector to y ${\boldsymbol{\Lambda}}_{>}$ ,F·tis called the tangential componentINTEGRATION OF DIFFERENTIAL FORMS 287 The right side of(143) should be regarded as just an abbreviation for the last integral in(142). The point is that F is defined on the range of y, but t is defined on [0,1]; thus F·t has to be properly interpreted. Of course, when y is one-to-one, then t(u) can be replaced by t(y(u)), and this diffculty disappears. 10.49 Integrals of 2-forms in $\textstyle R^{3}$ Let O be a 2-surface in an open set $E=R^{\prime}$ of class C',with parameter domain $\scriptstyle{D-{R^{\prime}}}$ Let F be a vector field in $\textstyle E,$ and define oのp by(125).As in the preceding section,we shall obtain a different representation of the integral of op over D. By (35) and (129), $$ \begin{array}{c}{{\displaystyle\int_{\Phi}\omega_{F}=\int_{\Phi}(F_{1}\,d y\wedge d z+F_{2}\,d z\wedge d x+F_{3}\,d x\wedge d y)}}\\ {{\displaystyle=\int_{D}\left\{(F_{1}\circ\Phi)\displaystyle{\frac{\partial(y,z)}{\tilde{\vartheta}(u,v)}+(F_{2}\circ\Phi)}\displaystyle{\frac{\partial(x,y)}{\tilde{\vartheta}(u,v)}}\right\}d u\,d v}}\\ {{=\displaystyle\int_{D}\displaystyle\int_{\displaystyle\left(\Phi(u,v)\right)\cdot{\bf N}(u,v) )d u\,d v.}}\end{array} $$ Now let n= n(u,D)be the unit vector in the direction of N(u,D)、[I N(u, D) = 0 for some ((u,D)∈ ${\mathcal{D}},$ take n(u,v))= e,. Then N = |NIn, and there- fore the last integral becomes $$ \int_{D}\mathbf{F}(\Phi(u,\,v))\cdot\mathbf{n}(u,\,v)\,|\,\mathbf{N}(u,\,v)\,|\,d u\,d v. $$ By (131), we can finally write this in the forn (144) $$ \int_{\phi}\omega_{\mathrm{F}}=\int_{\phi}\left(\mathrm{F}\cdot\mathrm{n}\right)d A. $$ With regard to the meaning of F·n, the remark made at the end of Sec. 10.48 applies here as well We can now state the original form of Stokes' theorem. 10.50 Stokes' formula If F is a vecto field of class f’ in an open set Ec R, and if 中 is a 2-surface of class G” in E,then (145) $$ \int_{\Phi}(\nabla\times\mathbf{F})\cdot\mathbf{n}\,d A=\int_{\partial\Phi}(\mathbf{F}\cdot\mathbf{t})\,d s. $$ Proof Put $\mathbf{H}=\nabla\times\mathbf{F}$ 、Then, as in the proof of Theorem 10.43,we have (146) $$ \omega_{\mathrm{H}}=d\lambda_{\mathrm{F}}\,. $$288 PRINCIPLES OF MATHEMATICAL ANALYsIs Hence $$ \int_{\phi}(\nabla\times\mathbf{F})\cdot\mathbf{n}\,d A=\int_{\Phi}(\mathbf{H}\cdot\mathbf{n})\,d A=\int_{\Phi}\omega_{\mathrm{{R}}} $$ $$ =\int_{\phi}d\lambda_{\mathrm{F}}=\int_{\phi\Phi}\lambda_{\mathrm{F}}=\int_{\partial\Phi}(\mathrm{F}\cdot\mathrm{t})\,d s $$ Here we used the definition of H, then(144) with H in place of F then (146),then--the main step- -Theorem 10.33,,and finally(143) extended in the obvious way from curves to l-chains. 10.51 The divergence theorem I/ F is a vector field of class Gf’ in an open set E C $R^{3},$ and if Q is a closed subset of ${\widetilde{R}}^{\dagger}$ with positively oriented boundary aQ (as described in Sec. 10.31) then (147) $$ \int_{\Omega}(\nabla\cdot\mathbf{F})\;d V=\int_{\partial\Omega}(\mathbf{F}\cdot\mathbf{n})\;d A. $$ Proof By(125) $$ d\omega_{F}=\left(\nabla\cdot\mathbf{F}\right)d x\wedge d y\wedge d z=\left(\nabla\cdot\mathbf{F}\right)d V. $$ Hence $$ \int_{\Omega}\left(\nabla\cdot\mathbf{F}\right)d V=\int_{\Omega}d\omega_{\mathrm{F}}=\int_{\partial\Omega}\omega_{\mathrm{F}}=\int_{\partial\Omega}\left(\mathbf{F}\cdot\mathbf{n}\right)d A, $$ by Theorem 10.33, applied to the 2-form or,and (144) 卫XERCISES 1。Let $\textstyle{H}$ be a compact convex set in $\textstyle R^{k},$ with nonempty interior. Let fe G(H), put J(x) = 0 in the complement of $\textstyle H,$ and define m as in Definition 1.3 $\mathbf{j}_{n}f$ f is independent of the order in which the ${\mathcal{N}}$ integrations are Prove that f carried out. Hint:Approximate f by functions that are continuous on $R^{k}$ and whose 2. For supports are in $\textstyle H,$ , as was done in Example 10.4. $i=1,$ 2,3,.…. lt qp G(R') have support in (2-,21-′), such that Jgn=1 Put $$ f(x,y)=\textstyle{\frac{\pi}{t-1}}[\varphi_{i}(x)-\varphi_{i+1}(x)]\varphi_{i}(y) $$ Then f has compact support in R3,f is continuous except at $(0,0),$ and $$ \int\!d y\int\!f(x,y)\,d x=0\qquad\mathrm{but}\qquad\int\!d x\int\!f(x,y)\,d y=1. $$ Observe that fis unbounded in every neighborhood of(0,0)INTEGRAToN or DrFERENYTLAL roRMs 289 3.(a) If ${\mathcal{H}}^{\nu}$ is as in Theorem 10.7, put A = F(O), F,(x)= A-1F(x). Then F(0) = 7 Show that $$ \mathrm{F}_{1}({\bf x})={\bf G}_{n}\circ{\bf G}_{n-1}\circ\cdot\cdot\cdot\circ\;{\bf G}_{1}({\bf x}) $$ in some neighborhood of O, for certain primitive mappings ${\bf G}_{1},{\bf\cdot}\cdot{\bf\cdot}$ G.This gives another version of Theorem 10.7: $$ \operatorname{F}(\mathbf{x})=\operatorname{F}^{\prime}(0)\mathrm{G}_{n}\circ\mathrm{G}_{n-1}\circ\cdot\cdot\cdot\cdot\mathrm{G}_{1}(\mathbf{x}). $$ (b)Prove that the mapping $(x,y)\to(y,x)$ of $R^{2}$ onto $R^{2}$ is not the composition of any two primitive mappings,in any neighborhood of the origin.(This shows that the flips $\ B_{i}$ cannot be omitted from the statement of Theorem 10.7.) 4. For $(x,y)\in R^{2}$ ,define $$ \operatorname{F}(x,y)=(e^{x}\cos y-1,\,e^{x}\sin y). $$ Prove that $\scriptstyle\mathbf{F}=\mathbf{G}_{2}\circ\mathbf{G}+S.$ where $$ \begin{array}{c}{{G_{1}(x,y)=(e^{x}\,\cos y-1,\,y)}}\\ {{{}}}\\ {{G_{2}(u,v)=(u,(1+u)\:\tan v)}}\end{array} $$ are primitive in some neighborhood of (0,0) Compute the Jacobians of G,G,,F at (0,0).Define $$ \mathrm{H}_{2}(x,y)=(x,e^{x}\sin y) $$ and find $$ {\mathrm{H}}_{1}(u,v)=(h(u,v),v) $$ so that $\mathbf{F}=\mathbf{H}_{1}\circ\mathbf{H}_{2}$ is some neighborhood of (0,0). 5. Formulate and prove an analogue of Theorem 10.8,in which K is a compact subset of an arbitrary metric space.(Replace the functions gpi that occur in the proof of Theorem 10.8 by functions of the type constructed in Exercise 22 of Chap.4.) 6.Strengthen the conclusion of Theorem 10.8 by showing that the functions yn can be made diferentiable, and even infinitely differentiable.(Use Exercise 1 of Chap.8 in the construction of the auxiliary functions gpn.) 7.(a)Show that the simplex $Q^{k}$ is the smallest convex subset of $R^{k}$ t that contains 0,e,.….,en (b) Show that affine mappings take convex sets to convex sets 8。 Let ${\mathbf{}}H$ be the parallelogram in $R^{2}$ whose vertices are(1,1),(3,2),(4,5),(2,4) Find the affine map T which sends(0,0) to(1,1),(1,0)to (3,2),(0,1)to(2,4) Show that $\scriptstyle J_{T}=3$ .Use ${\mathcal{T}}^{\prime}$ to convert the integral $$ \alpha=\int_{\cal H}e^{x-y}\,d x\,d y $$ to an integral over $\textstyle{\mathcal{V}}^{2}$ and thus compute c290 PRINCIPLES OF MATHEMATICAL ANALYSIS 9. Define $(x,y)=T(r,\ \theta)$ on the rectangle 0≤r≤a, 0≤0≤2 by the equations $$ x=r\,\cos\theta,\qquad y=r\,\sin\theta. $$ Show that ${\mathcal{J}}^{\gamma}$ maps this rectangle onto the closed disc D with center at $(0,0)$ and radius a, that ${\mathcal{V}}{\mathcal{J}}^{1}$ is one-to-one in the interior of the rectangle, and that Jr(r,0) = r. If fe 6(D), prove the formula for integration in polar coordinates: $$ \int_{\scriptscriptstyle D}f(x,y)\,d x\,d y=\int_{\scriptscriptstyle0}^{e}\int_{\scriptscriptstyle0}^{2\pi}f(T(r,\theta))r\,d r\,d\theta. $$ Hint: Let ${\mathcal{D}}_{0}$ be the interior of $\underline{{{\J}}}\underline{{{\longrightarrow}}}$ , minus the interval from(0,0) to (0,a) As it stands, Theorem 10.9 applies to continuous functions f whose support lies in Do.To remove this restriction, proceed as in Example 10.4. 10. Let a→Oo in Exercise 9 and prove that $$ \int_{R^{2}}f(x,y)\,d x\,d y=\int_{0}^{\infty}\int_{0}^{2x}f(T(r,\theta))r\,d r\,d\theta, $$ for continuous functions f that decrease suficiently rapidly as $|x|+|y| arrow\alpha|$ (Find a more precise formulation.)Apply this to $$ f(x,y)=e x p\left(-x^{2}-y^{2}\right) $$ to derive formula(101) of Chap.8 11. Define (u,D)= T(s,t) on the strip $$ 0<s<\infty,\;\;\;\;\;\;\;0<t<1 $$ by setting $u=s-s t,v=s t$ in $\textstyle R^{2}$ .Show that J,A(,1)= s. is a l-1 mapping of the strip onto the .Show that ${\mathcal{T}}$ positive quadrant $\underline{{\langle\chi\rangle}}$ For $x>0,y>0,$ integrate $$ w^{x-1}e^{-u}v^{y-1}e^{-v} $$ over Q, use Theorem 10.9 to convert the integral to one over the strip, and derive formula(96) of Chap.8 in this way. (For this application, Theorem 10.9 has to be extended so as to cover certain improper integrals. Provide this extension.) 12. Let $\textstyle\int\!k$ be the set of all u= (u,.…A)∈ R' with O≤u≤1 for all ; let Q be the set of all $\mathbf{x}=(x_{1},\dots,$ x)e R* with x≥0,Xx ≤1.(I* is the unit cube; $Q^{k}$ is the standard simplex in R*.) Define x= T(u) by $$ \begin{array}{l}{{x_{1}=u_{1}}}\\ {{x_{2}=(1-u_{1})u_{2}}}\\ {{\cdots\cdots\cdots\cdots\cdots\cdots\cdots\cdots\cdots\cdots\cdots\cdots\cdots\cdots\cdots\cdots\cdots\cdots\cdot\cdot\cdot\cdot\cdot\cdot\cdot\cdot\cdot\cdot\cdot\cdot\cdot\cdot\cdot\cdot\cdot\cdot\cdot.}}\\ {{x_{k}=(1-u_{1})\cdot\cdot\cdot\cdot(1-u_{k-1})u_{k}\,.}}\end{array} $$INTEGRATION OF DIFFERENTIAL FORMS 291 Show that $$ \mathop{\stackrel{\cong}{\geq}}{}_{i=1}^{k}\,x_{i}=1-\prod_{i=1}^{k}\,(1-u_{i}). $$ inverse $\stackrel{ arrow}{k\overline{{3}}}^{\prime}$ is defined in the interior of $Q^{k}$ by $u_{1}=x_{1}$ is 1-1 in the interior of $\textstyle{\mathit{J}}^{k},$ and that its Show that T maps $\textstyle\int\!k$ onto $Q^{k},$ that ${\mathcal{T}}$ and $$ u_{t}={\frac{x_{t}}{1-x_{1}-\cdots-x_{t-1}}} $$ for i= 2,...,k.Show that $$ J_{T}(\mathbf{u})=(1-u_{1})^{k-1}(1-u_{2})^{k-2}\cdots(1-u_{k-1}), $$ and $$ J_{s}(\mathbf{x})=[(1-x_{1})(1-x_{1}-x_{2})\cdots(1-x_{1}-\cdots-x_{k-1})]^{-1} $$ 13.Let r,...,rx be nonnegative integers, and prove that $$ \int_{q k}x_{1}^{r_{1}}\cdots x_{k}^{r_{k}}\,d x={\frac{r_{1}!\cdots r_{k}!}{(k+r_{1}+\cdots+r_{k})!}} $$ Hint: Use Exercise 12, Theorems 10.9 and 8.20 Note that the special case $r_{1}=\mathbf{\ddot{\}}\cdot\cdot\cdot=r_{k}=\mathbf{0}$ shows that the volume of $Q^{k}$ is 1/k ! 14. Prove formula(46) 15. If w and $\lambda_{\mathbf{\theta}}$ are k- and m-forms, respectively, prove that $$ \omega\ \wedge\vartheta\sim(-1)^{k m}\lambda\ \wedge\ \omega. $$ 16.I1f k≥ and = [po,P…… olis an oreted affn k-simplex, prove hat 20o = 0 directly from the definition of the boundary operator e. Deduce from this that a24*= 0 for every chain Y. Hint: For orientation, do it first for $k=$ 2,k = 3. In general, if i<j,let ou 3 $\textstyle\bigcap j \langle$ and py from o.Show that each ou be the (k- 2)-simplex obtained by deleting occurs twice in a3o, with opposite sign 17. Put J = T1十Tz, where $$ \tau_{1}=[0,\mathrm{e}_{1},\mathrm{e}_{1}+\mathrm{e}_{2}],\qquad\tau_{2}=-\left[0,\mathrm{e}_{2},\mathrm{e}_{2}+\mathrm{e}_{1}\right] $$ Explain why it is reasonable to call ${\mathcal{I}}^{2}$ the positively oriented unit square in $R^{2}$ Show that ${\mathcal{O}}J^{2}$ is the sum of 4 oriented affne 1-simplexes. Find these. What is 2(r:-7)? 18.Consider the oriented affine 3-simplex $$ \sigma_{1}=[0,\mathbf{e}_{1},\mathbf{e}_{1}+\mathbf{e}_{2},\mathbf{e}_{1}+\mathbf{e}_{2}+\mathbf{e}_{3}] $$ in $R^{3}$ Show that oi (regarded as a linear transformation) has determinant 1. Thus α, is positively oriented.292 PRINCIPLES or MATHEMATICAL ANALYSIs Let 。,……,O。 be five other orinted 3-simplexes, obtained as follows There are five permutations (i,iz,is) of (1,2,3), distinct from (1,2,3).Associate with each (i,ia,io) the simplex $$ s(i_{1},i_{2},i_{3})[0,\mathrm{e}_{i_{1}},\mathrm{e}_{i_{1}}+\mathrm{e}_{i_{2}},\mathrm{e}_{i_{1}}+\mathrm{e}_{i_{2}}+\mathrm{e}_{i_{3}}] $$ where s is the sign that occurs in the definition of the determinant.(This is how r was obtained from ${\overline{{\mathcal{P}}}}_{\perp}$ in Exercise 17. Show that oz,.…., s are positively oriented. Put $J^{3}=\sigma_{1}$ 十…+ 0。.Then J may be called the positively oriented unit cube in R3. Show that aJ3 is the sum of 12 oriented affine 2-simplexes.(These 12 tri- angles cover the surface of the unit cube I*.) Show that x= (x, X2,,×) is in the range of on if and only if O≤x.≤x ≤M≤1. Show that the ranges of $\sigma_{1},\,\cdot\cdot\cdot,$ os have disjoint interiors, and that their unicn covers ${\cal{V}}^{3}.$ (Compare with Exercise 13; note that 3!= 6.) 19. Let ${\mathcal{J}}^{2}$ r and. ${\boldsymbol{J}}^{3}$ be as in Exercise 17 and 18.Define $$ \begin{array}{l l}{{B_{01}(u,v)=(0,u,v),\;\;\;\;\;\;\;B_{11}(u,v)=(1,u,v),}}\\ {{B_{02}(u,v)=(u,0,v),\;\;\;\;\;\;B_{12}(u,v)=(u,1,v),}}\\ {{B_{03}(u,v)=(u,v,0),\;\;\;\;\;B_{i3}(u,v)=(u,v,1).}}\end{array} $$ These are affine, and map Rl into R3. Put B,= B,(J7-),for r= 0,1,i= 1, 2,3.Each β, is an affine-oriented 2-chain.(See Sec. 10.30.) Verify that $$ {\delta J}^{3}=\sum_{i=1}^{3}(-1)^{\dagger}(\beta_{o i}-\beta_{1,i}), $$ in agreement with Exercise 18. 20. State conditions under which the formula $$ \int_{\phi}f\,d\omega=\int_{a\varphi}f\omega-\int_{\phi}(d f)\,\wedge\,a $$ is valid, and show that it generalizes the formula for integration by parts. Hint: d(fo) = (df)入α+f do 21. As in Example 10.36, consider the 1-form $$ \eta={\frac{x\,d y-y\,d x}{x^{2}+y^{2}}} $$ in $\scriptstyle{R-0t}$ (a) Carry out the computation that leads to formula (113), and prove that $a_{I}=0$ (b) Let y(t)= (r cos t,r sin t), for some r >0, and let T be a G"-curve in $R^{2}-\{0\},$nNTEGRATION OF DIFFERENTIAL FORMS 293 with parameter interval [O,2m], with T(O)= T(2m), such that the intervals y(n) T(t)] do not contain O for any t e [0,2m]、 Prove that $$ \int_{\Gamma}\eta=2\pi. $$ Hint: For 0≤t ≤2m, 0≤u≤1, define $$ \Phi(t,u)=(1-u)\,\Gamma(t)+u\gamma(t). $$ Then $\langle{\mathbb{D}}\rangle$ is a 2-surface in $\scriptstyle n\cdot n=0.$ whose parameter domain is the indicated rect angle. Because of cancellations (as in Example 10.32), $$ {\mathcal{C}}\Phi=\Gamma-\gamma. $$ Use Stokes' theorem to deduce that $$ \int_{\mathbf{r}}\eta=\int_{\gamma}\eta $$ because $d\eta=0$ (c) Take T(t)= (a cos t, b sin t)where $a>0,\;b>0$ are fixed. Use part (b))to show that $$ \displaystyle\int_{0}^{2n}{\frac{a b}{a^{2}\cos^{2}t+b^{2}\sin^{2}t}}\,d t=2\pi. $$ (d)Show that $$ \eta=d\left(\arctan{\frac{y}{x}}\right) $$ in any convex open set in which $x\neq0,$ and that $$ \eta=d\bigg(-\mathrm{~arc~tan}\frac{x}{y}\bigg) $$ in any convex open set in which $y\neq0$ Explain why this justifies the notation $\eta=d\theta,$ in spite of the fact that $\mathcal{V}$ is not exact in R -{0} (e) Show that(b) can be derived from (d) (f) If T is any closed C'-curve in $R^{2}-$ {0}, prove that $$ {\frac{1}{2\pi}}\biggr\}_{\mathbf{r}}\eta=\mathrm{Ind}(\Gamma). $$ (See Exercise 23 of Chap. 8 for the definition of the index of a curve.294 PRINCIPLES OF MATHEMATICAL ANALYSIS 22.。As in Example 10.37, define Z in R -{0} by $$ \begin{array}{c}{{\gamma=\frac{x\,d y\,\wedge\,d z+y\,d z\,\wedge\,d x+z\,d x\,\wedge\,d y}{r^{3}}}}\\ {{\qquad\qquad\qquad\qquad\qquad\qquad\qquad\qquad\qquad}}\end{array} $$ where $\displaystyle r=({x^{2}}+{y^{2}}+{z^{2}})^{1/2},\,\left|{e}\right.$ t $\underline{{{\J}}}\underline{{{\longrightarrow}}}$ be the rectangle given by 0≤u≤m,0≤v≤2m and let E be the 2-surface in $\textstyle{R^{3}}$ , with parameter domain D, given by $$ x=\sin u\cos v,\;\;\;\;\;\;\;y=\sin u\sin v,\;\;\;\;\;\;z=\cos u. $$ (a) Prove that d, = 0 in $\scriptstyle{R-(0)}$ (b) Let S denote the restriction of $\sum_{i=1}$ to a parameter domain Ec D. Prove that $$ \int_{s}\zeta=\int_{E}\sin u\,d u\,d v=A(S), $$ where A denotes area,as in Sec.10.43.Note that this contains(115) as a special case (c) Suppose g,h, hz,ha,are G“-functions on [0,1, 9>0. Let $(x,y,z)=\Phi(s,t)$ ) define a 2-surface O, with parameter domain 1, by $$ x=g(t)h_{1}(s),~~~~~~y=g(t)h_{2}(s),~~~~~z=g(t)h_{3}(s). $$ Prove that directly from ((35) Note the shape of the range of O: For fixed s,D(s,t) runs over an interva on a line through 0. The range of O thus lies in a“cone”with vertex at the origin. (d) Let E be a closed rectangle in D, with edges parallel to those of D.Suppose f ∈ G”(D), f>0. Let S2 be the 2-surface with parameter domain E, defined by $$ \Omega(u,v)=f(u,v)\Sigma(u,v). $$ Define $\operatorname{\mu}^{\infty}$ as in (b) and prove tha $$ \int_{\mathbf{a}}{\zeta}=\int_{s}{\zeta}=A(S). $$ (Since S is the “radial projection”of S into the unit sphere, this result makes it reasonable to call fa the*solid angle”subtended by the range of $\mathbf{\hat{\Pi}}$ at the origin. Hint: Consider the 3-surface Y'given by $$ \Psi(t,u,v)=[1-t+t f(u,v)]\Sigma(u,v), $$ where (u,D)∈ E,0≤t≤1. For fixed ${\mathcal{O}}_{\mathrm{{3}}}$ the mapping((,u)→Y(t,山,D) is a 2-surnNTEGRAToN or DirFBRENTIAL roRMs 295 face $\mathrm{(f)}$ to which (c) can be applied to show that Jot = 0. The same thing holds when uis fixed. By (a) and Stokes' theorem, $$ \int_{\partial\Psi}\zeta=\int_{\Psi}d\zeta=0. $$ (e) Put入= -(z/r)m, where $$ \eta={\frac{x\,d y-y\,d x}{x^{2}+y^{2}}}, $$ as in Exercise 21. Then Ais a 1-form in the open set $\scriptstyle V\in K^{\prime}$ in which $x^{2}+y^{2}>0$ Show that ' is exact in V by showing that $$ \scriptstyle t\,=\,a_{\mathrm{k}} $$ (f)Derive (d)from (e), without using (c). Hint: To begin with, assume $\ 0<u<\pi\ {\mathrm{on}}\ E$ E. By (e), $$ \int_{\mathbf{a}}{\mathbf{\hat{\Pi}}}=\int_{\partial\mathbf{n}}\lambda\qquad\mathrm{and}\qquad{\left\{\begin{array}{l l}{{\gamma=\int_{\partial S}\lambda.}}\\ {{\delta\cdot}}\end{array}\right.} $$ Show that the two integrals of $\lambda$ are equal, by using part (d) of Exercise 21, and by noting that zlr is the same at $\scriptstyle\Sigma(n,v)$ as at $\Omega(u,v)$ (g)Is ( exact in the complement of every line through the origin ? 23, Fix n、 Define r. =(xf+…十x2) for 1≤k≤n, let $E_{k}$ , be the set of all $\textstyle x\in R^{\prime}$ at which $r_{k}>0,$ ), and let $\omega_{k}$ b。 be the (k - 1)-form defined in E。by $$ \omega_{k}=(r_{k})^{-k}\sum_{i=1}^{k}~(-1)^{i-1}x_{i}\,d x_{1}\wedge\cdots\wedge d x_{i-1}\,\wedge d x_{i+1}\,\wedge\cdots\wedge d x_{k}\,. $$ Note that c2= ,0 = , in the terminology of Exercises 21 and 22.Note also that $$ E_{1}\subset E_{2}\subset\cdot\cdot\cdot<E_{n}=R^{n}-\{0\}. $$ (a) Prove that dd $\scriptstyle u_{n}=0$ in $E_{k}$ (b)For $k=2,\ \ldots,n,$ prove that $\cdot\cdot\upsilon_{k}$ is exact in Ek-1,by showing that $$ \omega_{k}=d(f_{k}\omega_{k-1})=(d f_{k})\wedge\omega_{k-1}, $$ where $f_{k}({\bf x})=(-1)^{k}\,g_{k}(x_{k}/r_{k})$ and $$ g_{k}(t)=\int_{-1}^{t}\,(1-s^{2})^{(s-3)/2}\,d s\qquad(-1<t<1). $$ Hint: ${\mathcal{\mathcal{J}}}_{k}$ satisfies the differential equations and $$ \mathbf{x}\cdot(\nabla f_{k})(\mathbf{x})=0 $$ $$ ({\cal D}_{k}f_{k})({\bf x})=\frac{(-{\bf l})^{k}(r_{k}-1)^{k-1}}{(r_{k})^{k}}\,. $$296 PRINCIPLES OF MATHEMATICAL ANALYSIS (c) Is wn exact in $E_{n}$ E,? (d) Note that (b) is a generalization of part (e) of Exercise 22. Try to extend some of the other assertions of Exercises 21 and 22 to w0n,for arbitrary n. 24. Let w=2a,(x) dx, be a 1-form of class C” in a convex open set EC R. Assume dw = 0 and prove that w is exact in E, by completing the following outline: Fix p e E. Define $$ f(\mathbf{x})=\int_{[v,\mathbf{x}]}\omega\qquad(\mathbf{x}\in E). $$ Apply Stokes' theorem to affne-oriented 2-simplexes Ip, x,yJ in E. Deduce that $$ f({\bf y})-f({\bf x})=\sum_{i=1}^{n}{(y_{i}-x_{i})}\int_{0}^{1}\!a_{i}(({\bf1}-t){\bf x}+t{\bf y})\,d t $$ for xe E, y e E. Hence $(D_{i}f)(\mathbf{x})=a_{i}(\mathbf{x}).$ 25.Assume that w is a 1-form in an open set $\overline{{F}}$ c R" such that $$ \int_{\mathcal{V}}\omega=\mathbf{0} $$ for every closed curve $\textstyle{\mathcal{Y}}$ in $\textstyle E,$ of class f’、 Prove that w is exact in ${\overline{{F}}},$ by imitatin part of the argument sketched in Exercise 24. 26. Assume w is a 1-form in R- {0}, of class $\mathcal{C},$ and dw =0. Prove that w is exact in 6′ R 一 {0}. Hint: Every closed continuously differentiable curve in R*-{0} is the boundary of a 2-surface in R-{0}. Apply Stokes' theorem and Exercise 25 27. Let E be an open 3-cell in $R^{3},$ with edges parallel to the coordinate axes. Suppose (a,b, c)e E,fe G'(E) for i=1, 2,3 $$ \omega=f_{1}\:d y\:\wedge d z+f_{2}\:d z\:\wedge\:d x+f_{3}\:d x\:\wedge\:d y $$ and assume that $d\omega=0$ in $\widehat{F}.$ Define $$ \lambda=g_{1}\,d x+g_{2}\,d y $$ where $$ \begin{array}{c}{{g_{1}(x,y,z)=\displaystyle\int_{c}^{\pi}f_{2}(x,y,s)\,d s-\int_{b}^{y}f_{3}(x,t,c)\,d t}}\\ {{g_{2}(x,y,z)=-\displaystyle\int_{c}^{\pi}f_{1}(x,y,s)\,d s,}}\end{array} $$ for $(x,y,z)\in E.$ Prove that ${\bar{u}}=a$ in $\underline{{\land}}$ that occurs in Evaluate these integrals when w = and thus find the form ${\boldsymbol{\lambda}}_{\downarrow}$ part (e) of Exercise 22INTEGRATION OF DIFFERENTIAL FORMS 297 28. Fix $b>a>0,$ define $$ \Phi(r,\,\theta)=(r\,\cos\,\theta,\,r\,\sin\,\theta) $$ and compute both for a ≤r≤b,.0 ≤0 ≤2m.(The range of O is an annulus inR') Put $\scriptstyle w\colon x^{2}$ dy, $$ \int_{\phi}d\omega\qquad\mathrm{and}\qquad\int_{\phi\Phi}\omega $$ to verify that they are equal 30。If $\Re{}_{\mathfrak{L}}$ assertions become trivial if $\bar{T}$ 29. Prove the existence of a function α with the properties needed in the proof of is of class f'(Both Theorem 10.38, and prove that the resulting function $\textstyle{\mathcal{H}}$ taken to be a constant. Refer to Theorem 9.42.) is an open cell or an open ball, since α can then be is the vector given by (135), prove that $$ d e t\left[{\frac{\alpha_{1}}{\alpha_{2}}}\begin{array}{l l l}{{\beta_{1}}}&{{\alpha_{2}\beta_{3}-\alpha_{3}\beta_{2}}}\\ {{\alpha_{2}}}&{{\beta_{2}}}&{{\alpha_{3}\beta_{1}-\alpha_{4}\beta_{3}}}\\ {{\alpha_{3}}}&{{\beta_{3}}}&{{\alpha_{1}\beta_{2}-\alpha_{2}\beta_{1}}}\end{array}\right]=|{\bf N}|^{2}. $$ Also, verify Eq.(137). 31,Let $\underline{{\#}}$ C $\textstyle R^{3}$ be open, suppose g e G”(E), $h\in{\mathcal{C}}^{\prime}(E),$ and consider the vector field $$ \mathbf{F}=g\nabla h. $$ (a) Prove that $$ \nabla\cdot F=g\,\nabla^{2}h+(\nabla g)\cdot(\nabla h) $$ where $\nabla^{2}h=\nabla\cdot(\nabla h)=\Sigma\delta^{2}h/\delta x_{i}^{2}$ is the so-called “Laplacian”of h. (b) If Q is a closed subset of ${\widehat{H}},$ with positively oriented boundary a (as in Theorem 10.51), prove that $$ \int_{\Omega}\left[g\,\nabla^{2}h+(\nabla g)\cdot(\nabla h)\right]d V=\int_{\mathrm{an}}g\,{\frac{\dot{\sigma}h}{\dot{\sigma}n}}\,d A $$ where (as is customary) we have written Oh/an in place of(Vh)·n.(Thus Oh/en is the directional derivative of $\mathcal{J}_{\ell}$ in the direction of the outward normal to aQ, thc so-called normal derivative of h.)Interchange g and h。 subtract the resulting $J_{\ell}{}_{>}$ formula from the first one, to obtain $$ \int_{\mathbf{a}}(g\,\nabla^{2}h-h\,\nabla^{2}g)\,d V=\int_{\partial\mathbf{a}}\left(g\,{\frac{\partial h}{\partial n}}-h{\frac{d g}{\partial n}}\right)\,d A. $$ These two formulas are usually called Green's identities. (c)Assume that $J_{\ell}$ is harmonic in $\textstyle K.$ this means that $\nabla^{2}h=0$ D.Take $\scriptstyle g=1$ l and con- clude that $$ \int_{s\Omega}{\frac{\delta h}{\delta n}}d A=0. $$298 PRINCTPLES Or MATHEMATICAL ANALYSIs Take g = h, and conclude that $\displaystyle h=$ O in $(\underline{{{\gg}}})$ if $\scriptstyle h\,=\,0$ on aQ (d)Show that Green's identities are also valid in $R^{2}$ 32. Fix 8,0 $0<\delta<1$ .Let $\bar{\boldsymbol{D}}$ be the set of all((,t) $\in R^{2}$ l such that O≤0≤m,-8≤1≤8 Let O be the 2-surface in $\textstyle R^{3},$ with parameter domain D,given by $$ \begin{array}{c}{{x=(1-t\sin\theta)\cos2\theta}}\\ {{y=(1-t\sin\theta)\sin2\theta}}\end{array} $$ where( $x,y,z)=\Phi(\theta,t).$ Note that OD(r,t)= 0(0,-t), and that O is one-to-one on the rest of $\underline{{f}}$ The range $\mathcal{M}=\Phi(\mathcal{D})$ of Op is known as a Mobius band. It is the simplest example of a nonorientable surface Prove the various assertions made in the following description: Put P: = (0, - 6),p2= (, -6),ps = (m,8), P4 = (0,6), $\scriptstyle y,\;=\;y,$ . Put y = 【p,P:+1 i= 1,.….,4, and put $\Gamma_{i}=\Phi\circ\gamma_{i}$ Then $$ \bar{\Phi}\Phi=\Gamma_{1}+\Gamma_{2}+\Gamma_{3}+\Gamma_{4}\,. $$ Put a =(1,0, -8), b = (1,0,8). Then $$ \Phi({\bf p}_{1})=\Phi({\bf p}_{3})=a,\qquad\Phi({\bf p}_{1})=\Phi({\bf p}_{4})={\bf b}, $$ and eO can be described as follows. T spirals up from a to b; its projection into the (x, y)-plane has winding number +l around the origin.(See Exercise 23, Chap. 8. T.= 【b, a]. Ts spirals up from a to b; its projection into the(x,y)) plane has winding number -l around the origin. T4 = [b, a] Thus 20 = I+ Ts十2T . If we go from a to b along $\mathbf{T_{1}}$ and continue along the “edge”of M until we return to a, the curve traced out is $$ \mathbf{r}=\mathbf{r}_{z}-\mathbf{r}_{z} $$ which may also be represented on the parameter interval [, 2] by the equations $$ \begin{array}{c}{{x=(1+\delta\sin\theta)\cos2\theta}}\\ {{y=(1+\delta\sin\theta)\sin2\theta}}\\ {{z=-\delta\cos\theta.}}\end{array} $$ It should be emphasized that $\textstyle\int\!\!\!\!\!\perp^{1}$ = 2D: Let n be the 1-form discussed in Exercises 21 and 22. Since $d\eta=0,$ Stokes' theorem shows that $$ \Bigr\{_{g\oplus}\eta=0. $$INTEGRATION OF DIFERENTIAL FORMS 299 But although $\prod$ is the “geometric”boundary of ${\cal M}_{4}$ we have $$ \dagger_{\mathrm{r}}\eta=4\pi. $$ In order to vid this psiblesource of confusion, Stokes formula (Theorem 10.50) is frequently stated only for orientable surfaces