2 BASICTOPOLOGY FINITE,COUNTABLE,AND UNCoUNTABLE SETS We begin this section with a definition of the function concept 2.1 Definition Consider two sets A and B, whose elements may be any objects whatsoever, and suppose that with each element x of A there is associated, in some manner, an element of ${\mathcal{B}},$ which we denote by f(x). Then f is said to be a function from A to $\widehat{\cal D}$ (or a mapping of A into B). The set ${\mathcal{\mathbf{\nabla}}}$ is called the domain of f(we also say fis defined on A), and the elements f(x) are called the values of f The set of all values of fis called the range of f 2.2 Definition Let A and B be two sets and Iet f be a mapping of A into B If E c A,f(E) is defined to be the set of all elements f(x), for xe E. We cal f(E) the image of ${\overline{{R}}},$ under f. In this notation, f(A)is the range of f It is clea that f(A)c B. If f(A) = B, we say that f maps A onto $\bar{\cal D}_{\L}^{j}$ (Note that, according If to this usage, onto is more specific than into.) denotes the set of all xe A such that f(x)∈ E. We call $E\subset B,f^{-1}(E)$ f-1(E) the inverse image of E under fIf ye B,f(y) is the set of all xe ABASIC ToroLocv 25 such that f(x)= y、If, for each ye B, f-1y) consists of at most one element wise we write x = ×2.) of A, then fis said to be a 1-1 (one-to-one) mapping of A into B、This may , are distinct elements; other- also be expressd as follows: f is a 1-1 mapping of A into B provided tha and x $X_{\mathbf{2}}$ f(x,) ≠f(xz) whenever x ≠ ×2,xie A,x。e A. ${\mathcal{X}}_{1}$ (The notation x ≠xz means that 2.3 Definition If there exists a 1-l mapping of A onto ${\mathcal{B}}_{\l}$ we say that ${\mathcal{A}}_{\mathrm{II}}^{\widehat{Q}}$ and ${\mathcal{L}}{\mathcal{I}}$ can be put in 1-l correspondence, or that A and ${\mathfrak{S}}\backslash$ have the same cardinal number, or, briefly,that A and B are equivalent, and we write A ~ B. This relation clearly has the following properties It is reflexive: A~ A. $\bigcap{}\phi$ is symmetric: If A ~ B, then B~ A. It is transitive: If A~ B and $\beta\sim\mathbb{C},$ then A ~ C Any relation with these three properties is called an equivalence relation. 2.4 Definition For any positive integer n, let J,be the set whose elements are the integers 1,2,...,n; let J be the set consisting of all positive integers. For any set A, we say: finite) (a)A is finite if A~J, for some n(the empty set is also considered to be (b)A is infinite if ${\mathcal{A}}$ is not finite. (e)A is countable if A ~.J (d)A is uncountable if Ais neither finite nor countable (e)A is at most countable if A is finite or countable. Countable sets are sometimes called enumerable, or denumerable. For two finite sets A and B, we evidently have A ~ B if and only if A and B contain the same number of elements. For infinite sets, however, the idea of “having the same number of elements" becomes quite vague, whereas the notion of 1-1 correspondence retains its clarity 2.5 Example Let A be the set of all integers. Then A is countable. For, consider the following arrangement of the sets A and J: $$ \begin{array}{r l}{A:}&{{}{\begin{array}{r l}{{3}},1,-1,2,-2,3,-3,\dots}\\ {J:}&{{}}&{{}}\\ {\quad}&{{}}&{{1,2,3,4,5,6,7,\dots}\end{array} $$26 RtINcrPLEs or MaATHrMArTCcL ANALYSis We can, in this example, even give an explicit formula for a function ) from ${\mathcal{Y}}_{\phi}$ to A which sets up a 1-1 correspondence: $$ f(n)={\Bigg\vert}{\Bigg\vert}{\frac{n}{2}}\qquad(n\,\,\mathrm{even}), $$ 2.6 Remark A finite set cannot be cquivalent to one of its proper subsets That this is,however, possible for infinite sets,is shown by Example 2.5,in which J is a proper subset of A. In fact, we could replace Definition 2.4(b) by the statement: A is infinite if A is cquivalent to one of its proper subsets. f by the symbol $\left\{\mathcal{X}_{m}\right\},$ 2.7 Definiton By a sequence, we mean a function f defined on the set $\mathcal{J}$ of al positive integers. If (n) = X,, for neJ,it is customary to denote the sequenc or sometimes by x,x2,X,,..… The values of , that is the elements $\textstyle x_{n}\,\!,$ are called the terms of the sequence. If A is a set and if x,∈ A for all n e J, then {x,} is said to be a sequence in A,or a sequence of elements of A ${\widehat{\mathcal{Y}}}_{9}$ WG Note tha the terms x,.×2,×。,. of sequence need not be distinct Since every countable set is the range of a 1-1 function defined on J may regard every countable set as the range of a sequence of distinct terms Speaking more loosely, we may say that the elements of any countable set can be“arranged in a sequence. Sometimes it is convenient to replace J in this definition by the set of al nonnegative integers, i.e., to start with O rather than with 1. 2.8 Theorem Eery infinie subset of a countble set A is countable. Proof Suppose ${\widehat{\mathcal{H}}}$ C A, and E is infinite.Arrange the elements x of A in a sequence{ $ \lbrace x_{n} \rbrace$ } of distinct elements. Construct a sequence {nm, as follows: e E. Having Let n,be the smallest positive integer such that $x_{n_{1}}$ than between E and J. chosen n,.….,1x-1(k = 2,3,4,...), let $\mathbf{\Omega}r_{k}$ be the smallest integer greater $n_{k-1}$ such that xome E Futting J(k X(k = 1,2,3. .. weobtin a -l corresondenc The theorem shows that, roughly speaking, countable sets represent the“smallest' infinity: No uncountable set can be a subset of a countable set. 2.9 Definition Let A and $\ \{\sum_{i=1}\}$ be sets, and suppose that with each element aα of A there is associated a subset of Q which we denote by EaBASIC TOPOLoGY27 The set whose elements are the sets $E_{\alpha}$ will be denoted by {E}. Instead of speaking of sets of sets,we shall sometimes speak of a collection of sets,o1 a family of sets. The union of the sets Eais defined to be the set $|\operatorname{a}{\big)}^{\dagger}$ such that xe S if and only if xe E。for at least one α∈ A.We use the notation (1) $$ S=\bigcup_{\alpha\neq A}E_{\alpha}. $$ If A consists of the integers 1, 2,...,n,one usually writes (2) $$ S=\textstyle\bigwedge_{m}^{n}{\cal E}_{m} $$ or (3) $$ S=E_{1}\cup E_{2}\cup\dotsb\cup E_{n}. $$ If A is the set of all positive integers,the usual notation is (4) $$ S=\bigcup_{m=1}^{\infty}E_{m}\,. $$ The symbol oo in (4) merely indicates that the union of a countable col- lection of sets is taken, and should not be confused with the symbols +00,-oo introduced in Definition 1.23 The intersection of the sets L $E_{\alpha}$ is defined to be the set ${\mathfrak{J}}^{\mathfrak{g}}$ such that xe $\textstyle{\iint}$ if and only if xe E。for every α∈ A. We use the notation (5) $$ P=\bigcap_{s\in A}E_{s}, $$ or (6) $$ P=\bigcap_{m=1}^{n}E_{m}=E_{1}\cap E_{2}\cap\ ^{\cdot\;\cdot\;\cdot}\cap E_{h}\,, $$ or (7) $$ P=\bigcap_{m=1}^{\infty}E_{m}\,, $$ as for unions. If A o B is not empty, we say that A and ${\mathfrak{S}}^{\mathfrak{g}}$ intersect; otherwise they are disjoint. 2.10 Examples (a)Suppose $E_{1}$ consists of 1,2,3 and $E_{2}$ consists of 2,3,4. Then Ei u Ez consists of 1, 2,3,4, whereas E n Ez consists of 2, 3.23 PRINCIPLES Or MATHLEMLATICAL ANALYSIS xe A, let (b)Lct A be the set of real numbers x such that O<x≤1. For every $E_{x}$ be the set of real numbers y such that O<y<x. Then () $$ \begin{array}{r l}{E_{x}\subset E_{x}{\mathrm{~if~and~only~if~}}0<x\leq z\leq1}\\ {\bigcup}&{{}{\frac{x^{\prime}x}{x^{A}}}x=E_{1};}\\ {\qquad{\frac{\exp A}{x^{\prime}}}x{\mathrm{~is~empty}};}\end{array} $$ ) (i) (ii (i) and (ii) are clear. To prove (ii) we note that for every y > 0,P生E if x<y. Hence y生OseaE、 2.11 Remarks Many properties of unions and intersections are quite similar to those of sums and products;in fact, the words sum and product were some times used in this connection, and the symbols E and II were written in plac of U and 门 The commutative and associative laws are trivial: (9) (A U $$ \begin{array}{r l}{A\cup B=B\cup A;}&{{}\qquad A\cap B=B\cap A.}\\ {,B)\cup C=A\cup(B\cup C);\quad}&{{}(A\cap B)\cap C=A\cap(B\cap C).}\end{array} $$ (8) Thus the omission of parentheses in (3) and (G) is justified. The distributive law also holds: (10) $$ A\cap(B\cup C)=(A\cap B)\cup(A\cap C). $$ To prove this, let the left and right members of (10) be denoted by E and ${\boldsymbol{F}},$ respectively. Suppose xe E. Then xE A and xe ${\hat{\mathcal{N}}}$ U C, that is, x∈ ${\hat{\mathcal{D}}}$ or xe C (pos- sibly both). Hence xe A o B or xe A n C, so that xe F. Thus $E\subset F$ Next, suppose xe F. Then xe Ao B or xe AnC. That is,xe A, and xe B U C. Hence xe A o(B U C), so that FC E. It follows that $E=F,$ We list a few more relations which are easily verified: (11) $$ \begin{array}{c}{{A\in A\cup B,}}\\ {{A\setminus B\in A}}\end{array} $$ (12) If O denotes the empty set, then (13) $$ A\cup0=A,\qquad A\cap0=0. $$ If A c B, then (14) $$ A\cup B=B,\ \ \ \ \ A\cap B=A. $$BASIC ToPoLOGY 29 2.12 Theorem Let {E,},n = 1,2,3,...,be a sequence of countable sets, and pu (15) $$ S=\bigcup_{n=1}^{\infty}E_{n}\,. $$ Then S is countable. Proof Let every set L $E_{n}$ be arranged in a sequence {xmk},人 = 1,2,3,. and consider the infinite array (16) $$ \begin{array}{l l l l l l l}{{x_{\overline{{{\mathrm{r}}}}\overline{{{\mathrm{t}}}}}-x_{\overline{{{\mathrm{r}}}}}^{\mathrm{~}\gamma}-x_{\mathrm{2}\overline{{{\mathrm{}}}}}^{\mathrm{~}\gamma}}}&{{\cdots.}}&{{}}&{{\cdots\cdot}}\\ {{x_{\overline{{{\mathrm{2}}}}\mathrm{~}>}^{\mathrm{~}\mathrm{~}\mathrm{~}\mathrm{~}\mathrm{~}\mathrm{~}\mathrm{~}\mathrm{~}\mathrm{~}\mathrm{~}\mathrm{~}\scriptstyle{\bar{x}}_{\overline{{{\mathrm{3}}}}}\mathrm{~}\mathrm{~}\quad}\quad}}&{{x_{\mathrm{~}\mathrm{~}\quad}\cdots}}&{{}}&{{\cdots.}}&{{}}&{{}}\\ {{x_{\overline{{{\mathrm{~}\mathrm{~}}}}\quad}\quad x_{\mathrm{~}\quad}\quad}}&{{\cdots}}&{{}\quad}}&{{{\cdots}\quad}}&{{{{\quad}\quad}\cdots}}&{{{{\boldsymbol{\boldsymbol{\boldsymbol{\boldsymbol{\quad}\quad}\boldsymbol{\bar{\quad}\downarrow}\quad}} .}}}}&{{{{{{{\begin{array}\begin{array}\begin{array}\begin{array} $$ in which the elements of $E_{n}$ form the nth row.The array contains al elements of S.As indicated by the arrows,these elements can be arranged in a sequence (17) $$ x_{11}\colon X_{21},\cdot X_{12};\ x_{31},\cdot X_{22},\cdot x_{13};\cdot x_{32},\cdot x_{23},\cdot x_{14};\cdot\cdot\cdot $$ If any two of the sets $E_{n}$ have elements in common, these will appear more than once in(17). Hence there is a subset ${\mathcal{P}}$ of the set of all positive integers such that S~ T, which shows that S is at most countable (Theorem 2.8). Since $E_{1}\subset S,$ and $\textstyle E_{1}$ is infinite, $\mathbf{\sigma}_{k\geq}^{\nu}$ is infinite, and thus countable. Corollary Suppose A is at most countable,and, for every α∈ A, $B_{\alpha}$ is at most countable. Put Then T is at most countable $$ T=\bigcup_{\alpha\in A}B_{\alpha}\,. $$ For ${\mathcal{D}}^{\dagger}$ is equivalent to a subset of(15) 2,13 Theorem Let A be a countable set, and let ${\mathcal{B}}_{n}$ be the set of all n-tuples (a,...,a,) where a,∈ A (k = 1,..., n), and the elements $u_{1},\ \cdot\cdot\cdot,\ u_{n}$ , need not be distinct. Then ${\mathcal{B}}_{n}$ is countable. Proof That ${\mathcal{B}}_{1}$ is countable is evident, since $B_{1}=A$ Suppose $B_{n-1}$ is 3 countable (n = 2,3,4,...). The elements of ${\mathcal{B}}_{n}$ are of the form (18) $$ (b,\,a)\qquad(b\in B_{n-1},\,a\in A). $$ For every fixed b,the set of pairs (b,a)is equivalent to A, and hence countable. Thus ${\mathcal{B}}_{n}$ is the union of a countable set of countable sets. By Theorem 2.12, $\textstyle{B_{n}}\,$ is countable. The theorem follows by induction30 PRuNcCiPLs or MATHEMArTcCAL ANALvs Corollary The set of all ational numbers is countable Proof We apply Theorem 2.13, with n = 2, noting that every rational r is of the form bla, where a and b are integers. The set of pairs (a,b), and therefore the set of fractions b/a,is countable. In fact, even the set of all algebraic numbers is countable (see Exer cise 2). That not all infinite sets are, however, countable, is shown by the next theorem. 2.14 Theorem Let A be the set of all sequences whose elements are the digits O and 1. This set A is uncountable. The elements of A are sequences ike 1,0,0, 1,0,1,1,1,.. Proof Let E be a countable subset of ${\mathcal{A}},$ and let ${\mathcal{F}}^{\star}$ consist of the se- quences Ss,S2,83,… We construct a sequence s as follows. If the nth digit in $\mathbf{s}_{I I}^{\mathsf{S}}$ is 1, we Iet the nth digit ofsbe O, and vice versa. Then the sequence s differs from every member of ${\widetilde{F}}^{\dagger}$ in at least one place; hence s生E. But clearly se A,so that ${\mathcal{H}}^{\nu}$ is a proper subset of A. We have shown that every countable subset of ${\mathcal{A}}\,$ is a proper subset of A. It follows that A is uncountable (for otherwise A would be a proper subset of A,which is absurd) The idea of the above proof was first used by Cantor,and iscalled Cantor's diagonal process; for,if the sequences s, S2,83,. are placed in an array like (16), it is the elements on the diagonal which are involved in the construction of the new sequence. Readers who are familiar with the binary representation of the real numbers (base 2 instead of 10) will notice that Theorem 2.14 implies that the fact in Theorem 2.43. set of all real numbers is uncountable.、 We shall give a second proof of this METRIC SPACES 2.15 Defnition A set X, whose elements we shall call points,is said to be a metric space if with any two points p and g of X there is associated a rea number d(p,q), called the distance from p to ${\cal Q}_{1}$ such that (a)d(p,9)> 0 fp≠g;d(p, p) = 0; (b)d(p, 9) = d(q,p); (c)d(p,q)≤ d(p, r) + d(r,q), for any re X. Any function with these three properties is called a distance function,or a metric.BASic ToroLoov 31 2.16 Examples The most important examples of metric spaces, from our standpoint, are the euclidean spaces $R^{k},$ especially $\textstyle{R^{1}}$ $\textstyle{\mathcal{R}}^{2}$ (the R (the real line) and complex plane); the distance in $\textstyle{\mathcal{R}}^{k}$ is defined by (19) $$ d(\mathbf{x},\mathbf{y})=|\mathbf{x}-\mathbf{y}|\qquad(\mathbf{x},\mathbf{y}\in R^{k}). $$ By Theorem 1.37, the conditions of Definition 2.15 are satisfied by (19) It is important to observe that every subset ${\mathcal{Y}}$ of a metric space $\bigvee_{\mathcal{N}}$ is a metric space in its own right, with the same distance function. For it is clear that if conditions (a) to (c) of Definition 2.15 hold for p, q,re X, they also hold if we restrict p,q,r to lie in Y. Thus every subset of a euclidean space is a metric space.Other examples tively. are the spaces G(K) and ${\mathcal{C}}^{2}(\mu),$ which are discussed in Chaps. T and 11, respec 2.17 Definition By the segment (a,b)we mean the set of all eal numbers x such that a<x<b By the interval [a,b] we mean the set of all real numbers x such that α ≤x≤b Occasionally we shall also encounter "half-open intervals" [a, b) and (α,小1 the first consists of all x such that a≤x<b,the second of all x such that $a<x\leq b.$ If $a_{i}<b_{i}$ ;for i= 1,...,k,the set of all points $\mathbf{x}=\left(x_{1},\mathbf{\epsilon}\cdot\mathbf{\epsilon}\cdot\mathbf{\epsilon}\cdot x_{k}\right)\mathbf{in}\,R^{k}$ whose coordinates satisfy the inequalities a,≤X≤ b,(1 ≤i≤k)is called a k-cell. Thus a l-cell is an interval, a 2-cell is a rectangle, etc lf xe $\textstyle{\mathcal{R}}^{k}$ and r> 0, the open (or closed)ball B with center at x and radius r We call a set is defined to be the set of all y R such that ly-x|<r(or $|\mathbf{y}-\mathbf{x}|\leq r\rangle$ $E\in R^{n}$ convex if $$ \lambda\mathbf{x}+(1-\lambda)\mathbf{y}\in E $$ whenever xe E, y ∈ E,and $0<\lambda<1$ For example,balls are convex. For if|y -x<r,|z -x<r,and $0<\lambda<1,$ we have $$ \begin{array}{c}{{\lambda{\bf y}+(1-\lambda)z-{\bf x}|=|\lambda({\bf y}-{\bf x})+(1-\lambda)({\bf z}-{\bf x})|}}\\ {{\leq\lambda|\mathbf y-{\bf x}|+(1-\lambda)|z-{\bf x}|<\lambda r+(1-\lambda)r}}\end{array} $$ The same proof applies to closed balls. It is also easy to see that $\textstyle{\mathcal{H}}$ -cells are convex.32 PRINCIPLES OF MATHEMATICAL ANALYSIS 2.18 Definition Let X be a metric space.All points and sets mentioned below are understood to be elements and subsets of X. (a) A neighborhood of p is a set N,(p)consisting of allqsuch that d(p,9)<r,for some r> 0. The number ris called the radius of N,(p) (b) A point p is a limit point of the set E if every neighborhood of p contains a point g≠ p such that q∈ E (c) If p∈ E and p is not a limit point of E, then p is called an isolated point of E (d) E is closed if every limit point of ${\mathcal{H}}^{\nu}$ is a point of E (e) A point p is an interior point of E if there is a neighborhood N of p such that $N\in E$ () E is open if every point of Eis an interior point of E. (9) The complement of ${\widehat{\mathcal{H}}}_{x}$ (denoted by E') is the set of all points p ∈ X such that p生E (h) E is perfect if ${\mathcal{H}}^{\gamma}$ is closed and if every point of E is a limit point of E ( E is bounded if there is a real number M and a point ge X such that d(p, 9)< M for all pe E. (j) E is dense in X if every point of Xis a limit point of E, or a point of E (or both) Let us note that in Rl neighborhoods are segments, whereas in $\textstyle R^{2}$ neigh borhoods are interiors of circles. 2.19 Theorem Every neighborhood $\stackrel{\circ}{\langle}\Theta$ an open set Proof Consider a neighborhood $E=N_{r}(p),$ and let $\underline{{{Q}}}$ be any point of ${\widehat{\mathcal{H}}}$ Then there is a positive real number $\textstyle{\int}\psi$ such that $$ d(p,q)=r-h. $$ For all points s such that d(q,s)<h, we have then $$ d(p,s)\leq d(p,q)+d(q,s)<r-h+h=r, $$ so that se E. Thus q is an interior point of ${\mathcal{F}}^{\prime}$ 2.20 Theorem If p is a limit point of a set E, then every neighborhood of p contains infinitely many points of E. ProofSuppose there is a neighborhood $\textstyle{\cal{N}}$ of p which contains only a finite number of points of E. Let $q_{1},\ldots,q_{n}$ be those points of N o E, which are distinct from p, and put 1≤m ≤n min d0p,9mBASIC TorouooY33 lweuse tisnotation to dnote th smallet o the umbers dp, 4n) d(p, ).、 The minimum of a finite set of positive numbers is clearly posi tive,so that r> 0. The necighborhood N,(p) contains no point q of ${\widehat{\mathcal{H}}}$ such that g ≠ p, so that p is not a Limit point of E.This contradiction establishes th theorem. CorollaryA finite point set has no limit points. 2.21Examples Let us consider the following subsets of A $\textstyle R^{2}$ P (a)The set of all complex z such that |z| <1 (6)The set of all complex z such that |zl ≤ 1. (c)A noncmpty finite set. (d)The set of all integers (e)The set consisting of the numbers l/n (n = 1, 2,3、...). Let us notc that this set E has a limit point (namely,z = 0) but that no point of E is a limit point of E; we wish to stress the difference between having a limit point and containing onc. (f)The set of all complex numbers (that is, R) (g)The segment (a,b) Let us note that (d),(e) Gg) can be regarded also as subsets of $\textstyle{\mathcal{R}}^{1}$ Some properties of these sets are tabulated below: (a) Closed Open Perfect Bounded (6 No Yes No Yes (9) Yes No Yes Yes (C Yes No No Yes (aD Yes No No No (e No No No Yes () Yes Yes Yes No No No Yes In (gJ),we left the second entry blank. The reason is that the segment $\textstyle{\mathcal{R}}^{1}$ α,b)is not open if we regard it as a subset of R,butit is an open subsct of 2.22 Thcorem Let1{E} be a finie or ininie collecio sets E。 Then (20) $$ \left(\bigcup_{\alpha}E_{\alpha}\right)^{c}=\bigcap_{\alpha}\left(E_{\alpha}^{c}\right). $$ Thus Ac B Proof Let A and B be the left and right members of (20). If xe A,then xeU.E.,hence x唯E。for any α, hence xe ES for every α,so that xen Es3 PRINCIPLES OF MATHEMATICAL ANALYSIS Conversely, if xe B, then xe Es for every c, hence x E。 for any α. hence x生Ua Ea,SO that xe(U。 E-)°. Thus B c A. It follows that A = B. 2.23 Theorem A set $\overline{{\beta^{\nu}}}$ is open if and onby 扩is complemen is closed Proof First, suppose E" is closed. Choose xe E. Then xs E",and xis not a limit point of E". Hence therc exists a neighborhood $\textstyle{\mathcal{N}}$ of x such and Eis open. that E" o N is empty, that is, N E. Thus xis an interior point of $\textstyle E,$ Next, suppose E is open. Let xbe a limit point of E*. Then every neighborhood of x contains a point of E",so that xis not an interior point of E. Since $\overline{{\beta\omega^{\nu}}}$ is open, this means that xe E. It follows that ES is closed. Corollary A set F is closed if and only if its complement is open. 2.24 Theorem (a)For any collection {G} of open sets, UJ。 G, is open. (b)For any collection {F} of closed sets,O。 F is closed. (c)For any finite collection G,..., ${\mathcal{C}}_{n}$ of open sets, ("= G, is open (d)For any finite collection F,...,F,of closed sets, U*= F is closed Proof Put G = U G..If xe G, thcn xe G。 for some α.Since xis an interior point of Gm,xis also an interior point of ${\mathcal{O}},$ and G is open. This proves (a) By Theorem 2.22 (21) (0F)*=U(FD and $\textstyle{\mathcal{F}}_{\mathcal{Q}}^{c}$ is open,by Theorem 2.23.Hence (a) implies that (21) is open so that()。 F is closed. Next, put H =0n=G.For any xe H,there exist neighborhoods $\textstyle{N_{i}}$ of x,with radii r,such that $N_{i}\subset G_{i}\left(i=1,\cdot\cdot\cdot,n\right)$ Put $$ r=\operatorname*{min}\,(r_{1},\ldots,r_{n}), $$ and let $\textstyle{\cal{N}}$ be the neighborhood of x of radius r. Then N $N\subset G_{i}$ G,for i=1, .,n,so that N c H, and $\textstyle{H}$ is open. By taking complements,(d follows from (c) $$ \left(\bigcup_{i=1}^{n}F_{i}\right)^{e}=\bigcap_{i=1}^{n}(F_{i}). $$BASic roroLoGx35 2.25Examples In parts (c) and (d) of the preceding theorem, the finiteness of Then ${\mathcal{O}}_{n}$ the collections is essential. For let ${\mathcal{D}}_{n}$ be the segment $\left(-{\frac{1}{n}},{\frac{1}{n}}\right)$ (n = 1,2,3、.….) is an open subset of R'. Put G = 08=1 G,. Then G consists of a singl point (namely,x = 0) and is therefore not an open subset of $\textstyle{\mathcal{R}}^{1}$ Thus the intersection of an infinite collection of open sets need not be open Similarly, the union of an infinite collection of closed sets need not be closed. 2.26 Definition If Y is a metric space,if $E\subset X,$ and if E' denotes the set of all limit points of ${\widetilde{R}}^{\nu}$ in X, then the closure of ${\widehat{\operatorname{P}^{\nu}}}$ is the set $\scriptstyle{E=E}$ L ${\boldsymbol{E}}^{\prime}$ 2.27Theorem If X is a metric space and Ec X, then (a)Eis closed, (b)E = E if and only if E is closed, $E\in F,$ c)Ec F for every closed set Fe X such tha By (a) and (c), E Is the smallest closed subset of X that contains ${\boldsymbol{E}}.$ Proof (a)Ifpe X and p≠ E then p is neither a point of E nor a limit point of E Hence p has a neighborhood which does not intersect E. The complement of E is therefore open. Hence E is closed (b)If E = E,(a) implies that E is closed. If E is closed, then $E^{\prime}\subset E$ [by Definitions 2.18(d) and 2.26], hence $\frac{\widehat{P v}}{\sqrt{}}$ = E (Cc)If F is closed and F5 E, then Fp F', hence F p E’ Thus $F\perp E$ 2.28 Thcorem Let E be a nonempty set of real numbers which is bounded above Let y = sup E. Then y e E.Hence y e E if E is closed. Compare this with the examples in Sec. 1.9. Proof If y∈ E then y∈ E.Assume y生E. For every h > O there exists then a point xe ${\widehat{\mathcal{H}}}$ such that y - h<X<y,for otherwise y- h would be an upper bound of E. Thus y is a limit point of E. Hence ye E. 2.29 Remark Suppose E c Yc X, where Xis a metric space. To say that ${\widetilde{R}}^{\prime}$ is an open subset of X means that to each point $\textstyle{\int}$ e E there is associated a positive number r such that the conditions d(p,9)<r,q∈X imply that qe E But we have already observed (Sec. 2.16) that Y is also a metric space,so tha our definitions may equally well be made within Y.To be quite explicit, Iet us say that ${\mathcal{F}}^{\nu}$ is open relative to Y if to each p∈ ${\mathcal{F}}^{\nu}$ there is associated an r> O such that qe ${\overline{{F}}}^{\nu}$ whenever d(p,9)<r and ge Y、Example 2.21(g) showed that a set36 PRuNCIPLBs Or MATHEMATICAL ANALYSis may be open relative to Y without being an open subset of $X$ However, there is a simple relation between these concepts, which we now state. 2.30 Theorem Suppose Ye X. A subset $\widehat{R}^{\vee}$ of $\textstyle{\mathcal{Y}}$ is open relaive to Yif and only if E = Yn G for some open subset $\bigoplus{\overline{{\gamma}}}$ of X Proof Suppose E is open relative to Y. To each pe E there is a positive number r,such that the conditions dp,9)<r,,qe Y imply that ge E. , and define Let V,be the set of all g e X such that $a(p,q)<r_{p}$ $$ G=\bigcup_{p\in E}V_{p}\,. $$ Then G is an open subset of X,by Theorems 2.19 and 2.24 Since p ∈ ${\mathit{V}}_{p}$ for all p e E, it is clear that E c G n Y for every $p\in E,$ so that By our choice of Vp, we have $V_{p}\cap Y\subset E$ G o Yc E. Thus E = G。Y, and one half of the theorem is proved. neighborhood $\nu_{r}=G.$ Conversely, if G is open in X and $E=G\,\cap\,Y$ every p ∈ E has a Then $V_{p}\cap Y\subset E,$ so that ${\mathcal{H}}^{\gamma}$ is open relative to Y COMPACT SETS 2.31 Definition By an open cover of a set ${\mathcal{H}}$ in a metric space $X$ we mean a collection {Ga} of open subsets of X such that Ec U. G。 2.32 Definition A subset $\textstyle K$ of a metric space Xis said to be compact if every open cover of K contains a finite subcover More explicitly, the requirement is that if {G,} is an open cover of $K_{\mathrm{{J}}}$ then there are finitely many indices o ${\mathcal Q}_{1},\ \circ\ \circ\ ,\ \ {\mathcal Q}_{n}$ L, such that $$ K\subset G_{\alpha_{1}}\cup\cdots\sim G_{\alpha_{n}}. $$ The notion of compactness is of great importance in analysis,especially in connection with continuity (Chap.4)) It is clear that every finite set is compact. The existence of a large class o infinite compact sets in $\textstyle{\mathcal{R}}^{k}$ will follow from Theorem 2.41. We observed earlier (in Sec. 2.29) that if ${\overline{{F}}}^{\nu}$ c Yc X, then ${\mathcal{F}}$ may be open relative to Y without being open relative to X. The property of bcing open thu depends on the space in which ${\widetilde{F}}^{\nu}$ is embedded. The same is true of the property of being closed. Compactness, however, behaves better, as we shall now see. To formu late the next theorem, let us say, temporarily, that K is compact relative to X if the requirements of Definition 2.32 are met.BASIC ToPoLoGY 37 2.33 Theorem Suppose K c Yc X.Then K is compact relative to X if and only if K is compact relative to Y. By virtue of this theorem we are able, in many situations,to regard com- pact sets as metric spaces in their own right, without paying any attention to any embedding space. In particular, although it makes itle sense to talk of open spaces, or of closed spaces (every metric space X is an open subset of itself and is a closed subset of itself), it does make sense to talk of compact metric spaces. Proof Suppose ${\widehat{F}}_{\mathbf{b}}^{\prime}$ is compact relative to X, and let {Va}be a collection of sets, open relative to Y, such that Ke UAV。By theorem 2.30, there are sets Gm,open relative to X,such that $V_{x}=\Upsilon\cap G_{x}\,,$ for all c; and since ${\hat{H}}$ is compact relative to X, we have (22) $$ {\cal K}\subset G_{\alpha_{1}}\cup\ ^{}\cdot\cdot\cdot\cup\ {\cal G}_{\alpha_{n}} $$ for some choice of finitely many indices $Q_{\parallel\cdot\mathbf{\epsilon}\ \circ\ \circ\ \circ\ \circ\ \circ\ \circ\ \circ\ \circ\ \N_{\eta}}$ Since K c Y,(22 implies (23) $$ K\subset V_{\alpha_{1}}\cup\cdot\cdot arrow\cup V_{\alpha_{n}}. $$ This proves that ${\widehat{N}}_{\mathrm{{B}}}^{p}$ is compact relative to Y. Conversely, suppose K is compact relative to Y,let {G,} be a col- implies (22). lection of open subsets of X which covers $K,$ and put $V_{\alpha}=\,Y\left(\mathcal{C}\right)\,\mathcal{C},$ ,Then (23)will hold for some choice of $\alpha_{1},\,\cdot\cdot\cdot,\,\alpha_{n};$ ; and since V.c G。,(23) This completes the proof. 2.34 Theorem Compact subsets of metric spaces are closed. Proof Let ${\hat{H}}$ be a compact subset of a metric space X.We shall prove that the complement of ${\bar{\cal K}}$ is an open subset of X. Suppose p e X, p生 K.If $q\in K,$ let $V_{\boldsymbol{q}}$ and ${\mathcal{W}}_{q}$ be neighborhoods of p and q, respectively, of radius less than dp,q)) see Definition 2.18(a)] Since ${\hat{H}}_{\mathrm{B}}^{\prime}$ is compact, there are finitely many points q、 ${}^{\circ}\qquad\qquad\qquad\qquad\qquad\qquad\qquad\qquad\qquad\qquad\qquad$ g,in K such that $$ K\subset W_{q_{1}}\cup\cdot\cdot\cdot\cup\;W_{q_{n}}=W. $$ intersect W.Hence $\scriptstyle V\in K^{n}$ ,so that then V is a neighborhood of $\textstyle{\mathcal{D}}$ which does no .The If V = V,个…n V $\textstyle\mathit{V}_{q_{n}}~,$ is an interior point of $K^{\mathrm{c}}$ $\mathcal{J}$ K theorem follows. 2.35 Theorem Closed subsets of compact sets are compact. Proof Suppose ${\mathcal{F}}^{\nu}$ c Kc X, ${\mathcal{F}}^{\vee}$ is closed (relative to X), and K is compact Let {Va} be an open cover of ${\mathcal{H}}^{\nu}$ If ${\boldsymbol{F}}^{c}$ is adjoined to $\{V_{\alpha}\},$ we obtain an38 PRnNCIPLES Or MATHEMATICAL ANALYsis open cover Q of K. Since K is compact, there is a finite subcollection o finite subcollection of {Va} covers F of Q which covers K, and hence F. If F is a member of O, we may remove it from o and till retain an open cover of F. We have thus shown that a Corollary If F is closed and Kis compact, then $\mathcal{H}$ o Kis compact. Proof Theorems 2.24(6)) and 2.34 show that FoK is closed;since F o K c K, Theorem 2.35 shows that $F\cap K$ is compact. 2.36 Theorem If{K} is a collection of compact subsets of ametric space X suc that the intersection of every finite subcollection of {K} is nonempty,then O K is nonempty. that and since $K_{1}$ $K_{1}\subset G_{\alpha_{1}}\cup\cdots\cup G_{\alpha_{n}}$ Proof Fix a member K, of {K,} and put G。= K".Assume that no point ${\mathcal O}_{\downarrow}\;\mathrm{~,~\phi~\phi~\phi~\phi~\phi~\sim~\partial}{\mathcal O}_{\not p}$ such of K, belongs to every K。Then the sets G。 form an open cover of K: is compact, there are finitely many indices But this means that $$ K_{1}\cap K_{\alpha_{1}}\cap\cdots\cap K_{\alpha_{n}} $$ is empty, in contradiction to our hypothesis CorollaryIf {K,} s a sequence of nonempty compact sets such hat K,→ K,+ (n = 1,2,3,...) then QY K, is not empty. 2.37 Theorem If E is an infinite subset of a compact set K,then E has alim point in K. Proof If no point of $\textstyle K$ were a limit point of E, then each ge K would and the same is true of K, since ${\overline{{\mathcal{F}}}}^{\nu}$ which contains at most one point of $\{V_{q}\}$ can cover $E{\mathcal{E}},$ have a neighborhood $V_{\boldsymbol{q}}$ ${\mathcal{F}}^{\nu}$ (namely, g,if qe E)、 t s clear that no finite subcollection of C K. This contradicts the compactness of K. 2.38 Theorem If(I.,} is a sequence of intervals in R', such thai I, 1.+ (n =1,2,3,...),then [Y I, is not empty. Proof If 1, = [a,,。,],let E be the set of all a,.Then ${\mathcal{F}}^{\nu}$ is nonempty and bounded above (by b)Let x be the sup of E. If m and n are positiv integers, then $$ a_{n}\le a_{m+n}\le b_{m+n}\le b_{m}\,, $$ so that x $:=b_{n}$ for each m. Since it is obvious that $a_{m}\leq x,$ we see that xe Im for m = 1,2,3,.BASic roroLocv 39 2.39 Theorem Let k be a positive integer.I ({,} is a sequence of k-cells such that I,→ 1.+10n = 1,2,3,...), then Y I, is not empty. ProofLet ${\widehat{\operatorname{f}}}_{m}$ , consist of all points $\mathbf{x}=\left(x_{1},\ldots,\ x_{k}\right)\mathbf{:}$ such that $$ a_{n,j}\leq x_{j}\leq b_{n,j}\qquad(1\leq j\leq k;n=1,\,2,\,3,\,\ldots), $$ and put $I_{n,j}=[a_{n,j,}$ ..」、For cach j,the sequence $\langle{I_{n,j}}\rangle$ satisfies the hypotheses of Theorem 2.38.Hence there are real numbers x;(1 ≤j≤k) such that $$ a_{n,j}\leq x_{j}^{*}\leq b_{n,j}\qquad(1\leq j\leq k;n=1,\,. $$ 2,3,...) Setting x* = (x,..,xt),we see that $\mathbf{x}^{\ast}\in I_{n}$ for n = 1,2,3,.… The theorem follows. 2.40 Theorem Every k-cell is compact. Proo Let I be a kcell consisting of all points x = (x,.…x) such that a, ≤x, ≤b,(1 ≤j≤ k). Put $$ \delta= \langle\sum_{1}^{k}\left(b_{j}-a_{j}\right)^{2} \rangle^{1/2}. $$ Then x-y|≤6,if xe I, y e I. Suppose,to get a contradiction, that there exists an open cover {G Put c, = ( ,+ b,)/2. The of I which contains no finite subcover of ${\mathcal{F}}_{*}$ intervals [a;,c;] and [c,b;] then determine 2 ${\mathcal{L}}^{k}$ k-cells $\textstyle{\underline{{\lambda}}}_{i}$ whose union is I At least one of these sets $Q_{i}.$ , call it ${\mathcal{L}}_{1},$ , cannot be covered by any finite subcollection of {Ga} (otherwise $\mathbb{Z}\not\to$ could be so covered). We next subdivide $\bigwedge$ and continue the process. We obtain a sequence $\{I_{n}\}$ ,} with the following properties: (a) $I arrow I_{1} arrow I_{2} arrow I_{3} arrow\cdot.$ (の)1,is not covered by any finite sbcollection of {fGa}; (c)if xe I。 and y e I,, then|x -yl≤2 " 8. By (a) and Theorem 2.39, there is a point x which lies in every ${\widehat{\operatorname{f}}}_{\mathfrak{p}}$ For some α,x* ∈ G。. Since G。is open,there exists r> 0 such that ly -x*|<r implies that y e G。.If n is so large that 2-"8 <r(there is such an n,for otherwise 2" ≤ /r for all positive integers n,which is $I_{n}\subseteq G_{\alpha},$ which con- absurd since R is archimedean), then (c) impies that tradicts (b). This completes the proof. Borel theorem. The equivalence of (a) and (b) in the nex theorem is known as the Heine40 PRINCIPLES OF MATHEMATICAL ANALYsis 2.41 Theorem Ifa set ${\overline{{\mathcal{F}}}}^{\nu}$ in $\textstyle{\mathcal{R}}^{k}$ has one of thefollowing hree popeties,thn i1 has the other two· (a)E is closed and bounded (b) E is compact. (c)Every infinite subset of E has a limit point in E. Proof If(a) holds, then EeI for some k-cell I, and (b) follows from Theorems 2.40 and 2.35. Theorem 2.37 shows that (b) implies (c)、It remains to be shown that (c) implies (a) If ${\mathcal{F}}^{\dagger}$ is not bounded, then ${\widehat{R}}^{\prime}$ contains points $\mathbf{X}_{n}$ with $$ |{\bf x}_{n}|>n\;\;\;\;\;\;\;(n=1,\,2,3,\,...\,). $$ The set S consisting of these points $\mathbf{X}_{n}$ is infinite and clearly has no limit point in $R^{k},$ , hence has none in $\textstyle E.$ Thus (c) implies that ${\widehat{R}}^{\prime}$ is bounded If ${\mathcal{R}}^{\chi}$ is not closed, then there is a point $\mathbf{x}_{0}\in R^{k}$ k which is a limit poin of ${\mathcal{F}}^{\prime}$ but not a point of E. For n = 1,2,3,..., there are points $x_{n}\in E$ such that $|{\bf x}_{n}-{\bf x}_{0}|<1/n.$ Let S be the set of these points $\mathbf{X}_{n}$ $\operatorname{sg}$ is K,. Then infinite(otherwise $|x_{n}-\mathbf{x}_{0}|$ would have a constant positive value,for has no other limit infinitely many n), S has xoas a limit point, and $*\subseteq\operatorname{\nabla}_{k}$ point in R*. For if y e R",y ≠ xo,then $$ \begin{array}{r}{|\mathbf{x}_{n}-\mathbf{y}|\geq|\mathbf{x}_{0}-\mathbf{y}|\ -|\mathbf{x}_{n}-\mathbf{x}_{0}|}\\ {\geq|\mathbf{x}_{0}-\mathbf{y}|-{\frac{1}{n}}\geq{\frac{1}{2}}\,|\mathbf{x}_{0}-\mathbf{y}|}\end{array} $$ for all but finitely many n; this shows that $\nabla$ is not a limit point of S (Theorem 2.20). Thus $\operatorname{sgn}$ has no limit point in E; hence E must be closed if (c) holds We should remark, at this point, that (b) and (c) are equivalent in any metric space(Exercise 26) but that (a) does not, in general, imply(b) and (c) Examples are furnished by Exercise 16 and by the space 9 ,which is dis- cussed in Chap. 11. point in Rt 2.42 Theorem(Weiersrss)Every bounded infinite subset of $\textstyle{\mathcal{R}}^{k}$ has a limi Proof Being bounded, the set ${\mathcal{F}}^{\prime}$ in question is a subset of a k-cell $I\in R^{*}$ By Theorem 2.40,I is compact, and so E has a limit point in I,by Theorem 2.37.BASIC TOPoLOGY 41 PERFECT SETS 2.43 Theorem Let P be a nonempty perfect set i $R^{k}.$ Then $\underline{{\mathbf{p}}}$ is uncountoble Proof Since ${\mathfrak{P}}$ has limit points, P must be infinite. Suppose $\widehat{\cal M}$ is count able, and denote the points of P by x,,Xz,X.,…… We shall construct a sequence{V,} of neighborhoods, as follows. Let V be any neighborhood of x.If V consists of all $\scriptstyle\gamma\in R^{4}$ such that ly- x<, the closure of V is the set of ll ye R" such that ly 一X1」≤r is a imit point of P,there is a neighborhood V.+ such every poit o Suppose V,。 has been constructed, so that V,o P is not empty. Sinc n P is not empty. By Giii $\underline{{{\beta}}}^{0}$ that (ü))V.+1C P,,(i) x。≠ V.+,(ii ${\mathcal{Y}}_{n+1}$ V.+1 satisfies our induction hypothesis, and the construction can proceed Put $K_{n}=V_{n}\cap P$ Since ${\bar{V}}_{n}$ is closed and bounded,7, is compact. Since that O" K, is empty.But each $K_{n}$ no point of P lies in OK,.Since K,c P, this implies $K_{n}\ni K_{n+1}$ $x_{n}\notin K_{n+1},$ is nonempty, by ii), and by (i); this contradicts the Corollary to Theorem 2.36. Corollary Every interval [a,b](a<b) is uncountable.、In particular, the set o all real numbers is uncountable. 2,44 The Cantor set The set which we are now going to construct shows that there exist perfect sets in 1 which contain no segment $\textstyle{\mathcal{R}}^{1}$ Let I ${\mathit{E}}_{0}$ 。be the interval [0,1.、 Remove the segment(玉,3), and let $E_{1}$ be the union of the intervals [0, }]【3,1] Remove the middle thirds of these intervals, and Iet $E_{2}$ be the union of the intervals [0,}], 【6,号],【8,3],【8,1] Continuing in this way, we obtain a sequence of compact sets E,, such that (a) $E_{1}\Rightarrow E_{2}\Rightarrow E_{3}$ (b) $E_{n}$ ${\mathcal{D}}^{n}$ intervals, each of length 3" E,is the union of The set $$ P={\frac{\tilde{\sigma}}{\kappa_{n}}}E_{n} $$ is called the Cantor set. P is clearly compact, and Theorem 2.36 shows that $\stackrel{\mathcal{}}{\mathcal{}}$ is not empty.42 PRINCIPLES OF MATHEMATICAL ANALYsis No segment of the form (24) $$ \Big(\frac{3k+1}{3^{m}},\;\frac{3k+2}{3^{m}}\Big), $$ where ${\mathcal{N}}$ and m are positive integers, has a point in common with P. Since every segment (α,β) contains a segment of the form (24), if $$ 3^{-m}<{\frac{\beta-\alpha}{6}}, $$ P contains no segment. To show that $\overline{{{\mathcal{D}}}}$ is perfect, it is enough to show that P contains no isolated point. Let xe P,and let S be any segment containing x. Let ${\widehat{\operatorname{f}}}_{\mathrm{f}{\widehat{\mathfrak{g}}}}$ be that interval of $E_{n}$ , which contains x.Choose n large enough,so that ${\widehat{\operatorname{f}}}_{p}$ C S. Let x。be an endpoint of I,such that X。≠X $\hat{\overline{{\mathfrak{l}}}}\,^{\prime}\,\hat{\mathbb{I}}$ follows from the construction of P that x,∈ P. Hence xis a limit point of P, and P is perfect. One of the most interesting properties of the Cantor set is that it provides us with an example of an uncountable set of measure zero(the concept of measure will be discussed in Chap. 11). CONNECTED SETS 2.45 Definition TWo subsets A and B of a metric space X are said to be separated if both A n B and An B are empty, i.e., if no point of A lies in the closure of B and no point of B lies in the closure of A. A set Ec Xis said to be connected if E is not a union of two nonempty separated sets. 2.46 Remark Separated sets are of course disjoint, but disjoint sets need not be separated. For example, the interval [O, 1] and the segment (1,2) are no separated,since l is a limit point of (1,2).However, the segments (0,1) and (1,2) are separated The connected subsets of the line have a particularly simple structure: 2.47 Theorem A subset ${\widehat{\mathcal{H}}}^{\dagger}$ of the real line $\textstyle{\mathcal{R}}^{1}$ is connected if and only if it has the following property: Ifxe E,y e E,and x<z<y, then ze E. Proof If there exist xe E,ye E, and some ze (x, J) such that z生 E, then $E=A_{n}\cup B_{n}$ where $$ A_{z}=E\cap(-\infty,z),\qquad B_{z}=E\cap(z,\infty). $$BASIC roPoLOGY 43 Since xe A, and y∈ $B_{z},$ A and ${\widehat{\operatorname{G}}}$ are nonempty. Since A,C(-00,z) and B,C (z,0O), they are separated.Hence ${\widehat{\operatorname{st}\operatorname{st}}}$ is not connected. To prove the converse,suppose ${\widehat{\mathcal{H}}}$ is not connected. Then there are nonempty separated sets ${\mathcal{A}}$ and B such that $A\odot B=E$ . Pick xe A,ye B, and assume(without loss of generality) that x<y. Define $$ z=\operatorname{sup}\,(A\cap[x,y]). $$ By Theorem 2.28,ze A; hence z生B、In particular, x≤z<y. If z生 A,it follows that $x\prec z\ <y$ and z≠ E. If z e A, then z生B,hence there exists $\mathbb{Z}_{\parallel}$ such that z <z,<y and Z1生 B. Then x<Z,<y and z:生 E EXERCISES 1. Prove that the empty set is a subset of every set 2.A complex number z is said to be algebraic if there are integers ao ,an,not al zero, such that $$ a_{0}\,z^{n}+a_{1}z^{n-1}+\cdot\cdot\cdot+a_{n-1}z+a_{n}=0. $$ Prove that the set of all algebraic numbers is countable. Hint: For every positive integer $\Lambda\not\vdash$ there are only finitely many equations with $$ n+|a_{0}|+|a_{1}|+\cdot\cdot+|a_{n}|=N. $$ 3. Prove that there exist real numbers which are not algebraic. 4. Is the set of all irrational real numbers countable? 5. Construct a bounded set of real numbers with exactly three Iimit points. 6。Let ${\boldsymbol{E}}^{\prime}$ be the set of all limit points of a set E. Prove that ${\boldsymbol{E}}^{\prime}$ is closed. Prove tha E and $\overline{{\rho e}}_{i}$ have the same Himit points.(Recall that $E=E\cup E^{\prime}.$ .) Do Eand ${\boldsymbol{E}}^{\prime}$ always have the same limit points? 7. Let Ai, Az,A.,…. be subsets of a metric space. (a) If B. = Un- A, prove that B.= Ur- A,for n =1,2,3, (b)If B = U2- Ai, prove that B- U2- A Show, by an example,that this inclusion can be proper. 8.Is every point of every open set $\scriptstyle{E\left(E\right)\,R^{\prime}}$ a limit point of E?Answer the same question for closed sets in $R^{2}$ 9。Let $E^{\infty}$ denote the set of all interior points of a set E. [See Definition 2.18(e) $E^{\circ}$ is called the interior of E.] (a)) Prove that ${\boldsymbol{E}}^{\mathrm{o}}$ is always open (b) Prove that $\overline{{R}}$ is open if and only if $E^{\circ}=E.$ (c) If G C E and G is open, prove that G c E° (d)Prove that the complement of $E^{\circ}$ is the closure of the complement of $\overline{{f_{2}}}$ (e) Do $\overline{{R_{2}}}$ and $\overline{{\chi}}$ always have the same interiors ? (f)Do $\overline{{\theta^{\eta}}}$ $\textstyle E^{3}$ always have the same closures ? E and44 PRINcIPLEs OF MATHEMATICAL ANALYSIs 10. Let X be an infinite set. For p e $\underline{{\land}}$ and ge X, define $$ d(p,q)=\left\{_{0}^{1}\ \ \ \ \ \mathrm{(if\}p\neq q)\,\right\}. $$ Prove that this sa metric. Which subsets f te resutin metric space are open ? Which are closed?Which are compact ? 11. For xe Rl and ye R,define $$ \begin{array}{c}{{d_{i}(x,y)=(x-y)^{2},}}\\ {{d_{2}(x,y)=\sqrt{|x-y|},}}\\ {{d_{i}(x,y)=|x-2y|,}}\\ {{d_{i}(x,y)=|x-y|},}\\ {{d_{i}(x,y)=\frac{|x-y|}{1+|x-y|}\,.}}\end{array} $$ Determine. for each of these, whether it is a metric or not 12. Let $\scriptstyle{K\left(R\right)}$ consist of O and the numbers 1/n, for n= 1, 2,3, ... Prove that Kis compact directly from the definition (without using the Heine-Borel theorem) 13. Construct a compact set of real numbers whose Iimit points form a countable set 14. Give an example of an open cover of the segment (0,1) which has no finite sub- cover 15. Show that Theorem 2.36 and its Corollary become false in R', for example) if th word “compact”is replaced by “"closed”or by “bounded.” 16. Regard Q,the set of al rational numbers, as a metric space, with $d(p,q)=|p-q|$ Let E be the set of all $p\in Q$ such that $2<p^{2}<$ 3.Show that $\overline{{R}}$ is closed and bounded in Q, but that $\underline{{\mathbf{}}}$ is not compact. Is $\underline{{\mathcal{H}}}$ open in $\begin{array}{c}{{\left(\gamma\right)}}\\ {{\underline{{\chi}}}}\end{array}$ ? 17. Let E be the set of all xe [0.1] whose decimal expansion contains only the digit 4 and 7.Is E countable?Is $\widehat{H}$ dense in [0,11? Is E compact ? Is E perfect ? 18. Is there a nonempty perfect set in $\textstyle{R^{1}}$ which contains no rational number? 19. (a) It A and B are disioint closed sets in some metric space X, prove that they are separated. (b)Prove the same for disjoint open sets. (c)Fix p∈ X,8>0, define Ato be the set of all q e X for which d(p,q)<8, define B similarly, with > in place of <.Prove that A and B are separated. (d) Prove that every connected metric space with at least two points is uncount able. Hint: Use (c) 20. Are closures and interiors of connected sets always connected?(Look at subsets of R3.) 21. Let ${\mathcal{A}}$ and $\mathbf{\mathcal{D}}$ be separated subsets of some $\textstyle{\mathcal{R}}^{k},$ ', suppose ae A,be B, and define $$ \mathbf{p}(t)=(1-t)\mathbf{a}+t\mathbf{b} $$ for te R'. Put A。= p-(A), B。 = p- (B)、[Thus te Ao if and only if p(t)e A.]BAsic rorouoov 45 (a) Prove that Aoand ${\mathcal{B}}_{0}$ are separated subsets of R (b) Prove that there exists to (0,1) such that p(to)≠ A U B (c) Prove that every convex subset of $\textstyle{R^{k}}$ is connected. 22. A metric space is called separable if it contains a countable dense subset. Show that $R^{k}$ is separable. Hint: Consider the set of points which have only rational coordinates. 23. A collction {Va) of open subsets of Xis said to be a base for Xif the following is the union of a is true: For every xe X and every open set Gc X such that xe G, we have ${\mathcal{X}}$ xe P.c G for some α、In other words, every open set in subcollection of {V,}. Prove that every separable metric space has a countable base.Hint: Take all neighborhoods with rational radius and center in some countable dense subset of X. 24. Let X be a metric space in which every infinite subset has a limit point. Prove that X is separable. Hint: Fix 8>0,and pick xie X. Having chosen xi, $x_{i}\in X_{i}$ choose x+ne X,if possible so tat d(xr,,xy+)≥ for $i=1,$ ..,j、Show that this pocess must stop ater a finite number of steps, and that ${\mathcal{A}}$ can therefore be covered by fnitely many neighborhoods of radius .Take $\vartheta\simeq\,\displaystyle{\bf1}/n\,(n\not{k}\to\displaystyle{\bf1},\,2,\,3,$ . .), and consider the centers of the corresponding neighborhoods. 25. Prove that every compact metric space K has a countable base,and that $\textstyle{\hat{N}}$ is therefore separable、Hint: For every positive integer n, there are finitely many neighborhoods of radius 1/n whose union covers K. 26. Let X be a metric space in which every infinite subset has a limit point. Prove that X is compact. Hint: By Exercises 23 and 24,X has a countable base.I1 follows that every open cover of X has a countable subcover {Gn}, n = 1,2,3, If no finite subcollection of {G,} covers X, then the complement $F_{n}$ of G u……U G is nonempty for each n, but [FA is empty.If $\underline{{\mathbf{}}}$ is a set which contains a point from each Fm,consider a limit point of E, and obtain a contradiction. 27. Define a point $\textstyle{\mathcal{P}}$ in a metric space X to be a condensation point of a set $E\subset X$ if every neighborhood of p contains uncountably many points of E Suppose $E\subset R^{k},$ E is uncountable, and let P be the set of all condensation points of E.Prove that P is perfect and that at most countably many points of $\underline{{\mathbf{}}}$ are not in P. In other words, show that $P^{c}\cap E$ is at most countable.Hint: Let {V} be a countable base of R',Iet ${\mathcal{W}}$ be the union of those V。for which $E\,\cap\,V_{n}$ is at most countable, and show that $P=W^{c}$ 28. Prove that every closed set in a separable metric space is the union of a (possibly empty) perfect set and a set which is at most countable.(Corollary: Every count- able closed set in $\textstyle{\mathcal{R}}^{k}$ has isolated points.)Hint: Use Exercise 27. 29. Prove that every open set in ${\boldsymbol{R}}^{1}$ is the union of an at most countable collection of disjoint segments. Hint: Use Exercise 22.46 PRINctPLES OF MATHEMATICAL ANALYSIs 30. Imitate the proof of Theorem 2.43 to obtain the following result: If R' = UJY Fm,where each $F_{n}$ is a closed subset of R', then at least one F has a nonempty interior. Equivalent statement: If GAis a dense open subset of R*,for $\scriptstyle n=1.$ 2,3,.… then [YG。is not empty (in fact, it is dense in R") (This is a special case of Baire's theorem; see Exercise 22, Chap. 3, for the general case.)