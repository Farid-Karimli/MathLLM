\documentclass[10pt]{article}
\usepackage[utf8]{inputenc}
\usepackage[T1]{fontenc}
\usepackage{amsmath}
\usepackage{amsfonts}
\usepackage{amssymb}
\usepackage[version=4]{mhchem}
\usepackage{stmaryrd}
\usepackage{bbold}
\usepackage{graphicx}
\usepackage[export]{adjustbox}
\graphicspath{ {./images/} }

\title{QUALIFYING EXAMINATION }


\author{HARVARD UNIVERSITY}
\date{}


\begin{document}
\maketitle
Department of Mathematics

Tuesday August 30, 2016 (Day 1)

\begin{enumerate}
  \item (DG)
\end{enumerate}

(a) Show that if $V$ is a $\mathcal{C}^{\infty}$-vector bundle over a compact manifold $X$, then there exists a vector bundle $W$ over $X$ such that $V \oplus W$ is trivializable, i.e. isomorphic to a trivial bundle.

(b) Find a vector bundle $W$ on $S^{2}$, the 2 -sphere, such that $T^{*} S^{2} \oplus W$ is trivializable.

\begin{enumerate}
  \setcounter{enumi}{1}
  \item (RA) Let $(X, d)$ be a metric space. For any subset $A \subset X$, and any $\epsilon>0$ we set
\end{enumerate}

$$
B_{\epsilon}(A)=\bigcup_{p \in A} B_{\epsilon}(p)
$$

(This is the " $\epsilon$-fattening" of $A$.) For $Y, Z$ bounded subsets of $X$ define the Hausdorff distance between $Y$ and $Z$ by

$$
d_{H}(Y, Z):=\inf \left\{\epsilon>0 \mid Y \subset B_{\epsilon}(Z), \quad Z \subset B_{\epsilon}(Y)\right\}
$$

Show that $d_{H}$ defines a metric on the set $\tilde{X}:=\{A \subset X \mid A$ is closed and bounded $\}$.

\begin{enumerate}
  \setcounter{enumi}{2}
  \item (AT) Let $T^{n}=\mathbb{R}^{n} / \mathbb{Z}^{n}$, the $n$-torus. Prove that any path-connected covering space $Y \rightarrow T^{n}$ is homeomorphic to $T^{m} \times \mathbb{R}^{n-m}$, for some $m$.

  \item (CA)

\end{enumerate}

Let $f: \mathbb{C} \rightarrow \mathbb{C}$ be a nonconstant holomorphic function. Show that the image of $f$ is dense in $\mathbb{C}$.

\begin{enumerate}
  \setcounter{enumi}{4}
  \item (A) Let $F \supset \mathbb{Q}$ be a splitting field for the polynomial $f=x^{n}-1$.
\end{enumerate}

(a) Let $A \subset F^{\times}=\{z \in F \mid z \neq 0\}$ be a finite (multiplicative) subgroup. Prove that $A$ is cyclic.

(b) Prove that $G=\operatorname{Gal}(F / \mathbb{Q})$ is abelian.

\begin{enumerate}
  \setcounter{enumi}{5}
  \item (AG) Let $C$ and $D \subset \mathbb{P}^{2}$ be two plane cubics (that is, curves of degree 3 ), intersecting transversely in 9 points $\left\{p_{1}, p_{2}, \ldots, p_{9}\right\}$. Show that $p_{1}, \ldots, p_{6}$ lie on a conic (that is, a curve of degree 2 ) if and only if $p_{7}, p_{8}$ and $p_{9}$ are colinear.
\end{enumerate}

\section*{QUALIFYING EXAMINATION }
Department of Mathematics

Wednesday August 31, 2016 (Day 2)

\begin{enumerate}
  \item (A) Let $R$ be a commutative ring with unit. If $I \subseteq R$ is a proper ideal, we define the radical of $I$ to be
\end{enumerate}

$$
\sqrt{I}=\left\{a \in R \mid a^{m} \in I \quad \text { for some } m>0\right\} \text {. }
$$

Prove that

$$
\sqrt{I}=\bigcap_{\substack{\mathfrak{p} \supseteq I \\ \mathfrak{p} \text { prime }}} \mathfrak{p}
$$

\begin{enumerate}
  \setcounter{enumi}{1}
  \item (DG) Let $c(s)=(r(s), z(s))$ be a curve in the $(x, z)$-plane which is parameterized by arc length $s$. We construct the corresponding rotational surface, $S$, with parametrization
\end{enumerate}

$$
\varphi:(s, \theta) \mapsto(r(s) \cos \theta, r(s) \sin \theta, z(s))
$$

Find an example of a curve $c$ such that $S$ has constant negative curvature -1 .

\begin{enumerate}
  \setcounter{enumi}{2}
  \item (RA) Let $f \in L^{2}(0, \infty)$ and consider
\end{enumerate}

$$
F(z)=\int_{0}^{\infty} f(t) e^{2 \pi i z t} d t
$$

for $z$ in the upper half-plane.

(a) Check that the above integral converges absolutely and uniformly in any region $\operatorname{Im}(z) \geq C>0$.

(b) Show that

$$
\sup _{y>0} \int_{0}^{\infty}|F(x+i y)|^{2} d x=\|f\|_{L^{2}(0, \infty)}^{2}
$$

\begin{enumerate}
  \setcounter{enumi}{3}
  \item (CA) Given that $\int_{0}^{\infty} e^{-x^{2}} d x=\frac{1}{2} \sqrt{\pi}$, use contour integration to prove that each of the improper integrals $\int_{0}^{\infty} \sin \left(x^{2}\right) d x$ and $\int_{0}^{\infty} \cos \left(x^{2}\right) d x$ converges to $\sqrt{\pi / 8}$.

  \item $(\mathrm{AT})$

\end{enumerate}

(a) Let $X=\mathbb{R} P^{3} \times S^{2}$ and $Y=\mathbb{R} P^{2} \times S^{3}$. Show that $X$ and $Y$ have the same homotopy groups but are not homotopy equivalent.

(b) Let $A=S^{2} \times S^{4}$ and $B=\mathbb{C} P^{3}$. Show that $A$ and $B$ have the same singular homology groups with $\mathbb{Z}$-coefficients but are not homotopy equivalent.

\section{6. $(\mathrm{AG})$}
Let $C$ be the smooth projective curve over $\mathbb{C}$ with affine equation $y^{2}=f(x)$, where $f \in \mathbb{C}[x]$ is a square-free monic polynomial of degree $d=2 n$.

(a) Prove that the genus of $C$ is $n-1$.

(b) Write down an explicit basis for the space of global differentials on $C$.

\section*{QUALIFYING EXAMINATION }
Department of Mathematics

Thursday September 1, 2016 (Day 3)

\begin{enumerate}
  \item (AT) Model $S^{2 n-1}$ as the unit sphere in $\mathbb{C}^{n}$, and consider the inclusions
\end{enumerate}

\begin{center}
\includegraphics[max width=\textwidth]{2023_10_29_e08956e1bd7bcd4df6dfg-4}
\end{center}

Let $S^{\infty}$ and $\mathbb{C}^{\infty}$ denote the union of these spaces, using these inclusions.

(a) Show that $S^{\infty}$ is a contractible space.

(b) The group $S^{1}$ appears as the unit norm elements of $\mathbb{C}^{\times}$, which acts compatibly on the spaces $\mathbb{C}^{n}$ and $S^{2 n-1}$ in the systems above. Calculate all the homotopy groups of the homogeneous space $S^{\infty} / S^{1}$.

\begin{enumerate}
  \setcounter{enumi}{1}
  \item (AG) Let $X \subset \mathbb{P}^{n}$ be a general hypersurface of degree $d$. Show that if
\end{enumerate}

$$
\left(\begin{array}{c}
k+d \\
k
\end{array}\right)>(k+1)(n-k)
$$

then $X$ does not contain any $k$-plane $\Lambda \subset \mathbb{P}^{n}$.

\begin{enumerate}
  \setcounter{enumi}{2}
  \item (DG) Let $\mathcal{H}^{2}:=\left\{(x, y) \in \mathbb{R}^{2}: y>0\right\}$. Equip $\mathcal{H}^{2}$ with a metric
\end{enumerate}

$$
g_{\alpha}:=\frac{d x^{2}+d y^{2}}{y^{\alpha}}
$$

where $\alpha \in \mathbb{R}$.

(a) Show that $\left(\mathcal{H}^{2}, g_{\alpha}\right)$ is incomplete if $\alpha \neq 2$.

(b) Identify $z=x+i y$. For $\left(\begin{array}{ll}a & b \\ c & d\end{array}\right) \in \mathrm{SL}(2, \mathbb{R})$, consider the map $z \mapsto \frac{a z+b}{c z+d}$. Show that this defines an isometry of $\left(\mathcal{H}^{2}, g_{2}\right)$.

(c) Show that $\left(\mathcal{H}^{2}, g_{2}\right)$ is complete. (Hint: Show that the isometry group acts transitively on the tangent space at each point.)

\begin{enumerate}
  \setcounter{enumi}{3}
  \item (RA)
\end{enumerate}

(a) Let $H$ be a Hilbert space, $K \subset H$ a closed subspace, and $x$ a point in $H$. Show that there exists a unique $y$ in $K$ that minimizes the distance $\|x-y\|$ to $x$.

(b) Give an example to show that the conclusion can fail if $H$ is an inner product space which is not complete.

\begin{enumerate}
  \setcounter{enumi}{4}
  \item (A)
\end{enumerate}

(a) Prove that there exists a unique (up to isomorphism) nonabelian group of order 21.

(b) Let $G$ be this group. How many conjugacy classes does $G$ have?

(c) What are the dimensions of the irreducible representations of $G$ ?

\begin{enumerate}
  \setcounter{enumi}{5}
  \item (CA) Find (with proof) all entire holomorphic functions $f: \mathbb{C} \rightarrow \mathbb{C}$ satisfying the conditions:

  \item $f(z+1)=f(z)$ for all $z \in \mathbb{C}$; and

  \item There exists $M$ such that $|f(z)| \leq M \exp (10|z|)$ for all $z \in \mathbb{C}$.

\end{enumerate}

\end{document}