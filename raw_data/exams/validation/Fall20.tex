\documentclass[10pt]{article}
\usepackage[utf8]{inputenc}
\usepackage[T1]{fontenc}
\usepackage{amsmath}
\usepackage{amsfonts}
\usepackage{amssymb}
\usepackage[version=4]{mhchem}
\usepackage{stmaryrd}
\usepackage{bbold}

\title{QUALIFYING EXAMINATION }


\author{HARVARD UNIVERSITY\\
Department of Mathematics}
\date{}


\begin{document}
\maketitle
Department of Mathematics

Tuesday September 1, 2020 (Day 1)

\begin{enumerate}
  \item (AG) Let $X$ be a smooth projective curve of genus $g$, and let $p \in X$ be a point. Show that there exists a nonconstant rational function $f$ which is regular everywhere except for a pole of order $\leq g+1$ at $p$.

  \item (CA) Let $U \subset \mathbb{C}$ be an open set containing the closed unit disc $\bar{\Delta}=\{z \in$ $\mathbb{C}:|z| \leq 1\}$, and suppose that $f$ is a function on $U$ holomorphic except for a simple pole at $z_{0}$ with $\left|z_{0}\right|=1$. Show that if

\end{enumerate}

$$
\sum_{n=0}^{\infty} a_{n} z^{n}
$$

denotes the power series expansion of $f$ in the open unit disk, then

$$
\lim _{n \rightarrow \infty} \frac{a_{n}}{a_{n+1}}=z_{0}
$$

\begin{enumerate}
  \setcounter{enumi}{2}
  \item (RA) Let $\left\{a_{n}\right\}_{n=0}^{\infty}$ be a sequence of real numbers that converges to some $A \in \mathbb{R}$. Prove that $(1-x) \sum_{n=0}^{\infty} a_{n} x^{n} \rightarrow A$ as $x$ approaches 1 from below.

  \item (A) Prove that every finite group of order $72=2^{3} \cdot 3^{2}$ is not a simple group.

  \item (AT) Let $X$ be a topological space and $A \subset X$ a subset with the induced topology. Recall that a retraction of $X$ onto $A$ is a continuous map $f: X \rightarrow A$ such that $f(a)=a$ for all $a \in A$.

\end{enumerate}

Let $I=[0,1] \subset \mathbb{R}$ be the closed unit interval, and

$$
M=I \times I /(0, y) \sim(1,1-y) \forall y \in I
$$

the closed Möbius strip; by the boundary of the Möbius strip we will mean the image of $I \times\{0,1\}$ in $M$. Show that there does not exist a retraction of the Möbius strip onto its boundary.

\begin{enumerate}
  \setcounter{enumi}{5}
  \item (DG) Let $S$ be a surface of revolution
\end{enumerate}

$$
\mathbf{r}(u, v)=(x(u, v), y(u, v), z(u, v))=(v \cos u, v \sin u, f(v))
$$

where $0<v<\infty$ and $0 \leq u \leq 2 \pi$ and $f(v)$ is a $C^{\infty}$ function on $(0, \infty)$. Determine the set of all $0 \leq \alpha \leq 2 \pi$ such that the curve $u=\alpha$ (called a meridian) is a geodesic of $S$, and determine the set of all $\beta>0$ such that the curve $v=\beta$ (called a parallel) is a geodesic of $S$.

Hint: To determine whether a meridian or a parallel is a geodesic, parametrize it by its arc-length and use the arc-length equation besides the two secondorder ordinary differential equations for a geodesic. For your convenience the formulas for the Christoffel symbols in terms of the first fundamental form $E d u^{2}+2 F d u d v+G d v^{2}$ are listed below.

$$
\begin{aligned}
\Gamma_{11}^{1} & =\frac{G E_{u}-2 F_{u}+F E_{v}}{2\left(E G-F^{2}\right)} . & \Gamma_{11}^{2} & =\frac{2 E F_{u}-E E_{v}-F E_{u}}{2\left(E G-F^{2}\right)}, \\
\Gamma_{12}^{1} & =\frac{G E_{v}-F G_{u}}{2\left(E G-F^{2}\right)}, & \Gamma_{12}^{2} & =\frac{E G_{u}-F E_{v}}{2\left(E G-F^{2}\right)} \\
\Gamma_{22}^{1} & =\frac{2 G F_{v}-G G_{u}-F G_{v}}{2\left(E G-F^{2}\right)}, & \Gamma_{22}^{2} & =\frac{E G_{v}-2 F F_{v}+F G_{u}}{2\left(E G-F^{2}\right)}
\end{aligned}
$$

where the subscript $u$ or $v$ for the function $E, F$, or $G$ means partial differentiation of the function with respect to $u$ or $v$.

\section*{QUALIFYING EXAMINATION }
Department of Mathematics

Wednesday September 2, 2020 (Day 2)

\begin{enumerate}
  \item (CA) Evaluate the integral
\end{enumerate}

$$
\int_{-\infty}^{\infty} \frac{x \sin x}{x^{2}+1} \mathrm{~d} x
$$

You need to prove that the error terms vanish in the residue calculation.

\begin{enumerate}
  \setcounter{enumi}{1}
  \item (AG) Let $X \subset \mathbb{P}^{n}$ be an irreducible projective variety of dimension $k$. Let $\mathbb{G}(\ell, n)$ be the Grassmannian of $\ell$-planes in $\mathbb{P}^{n}$ for some $\ell<n-k$, and let $C(X) \subset \mathbb{G}(\ell, n)$ the algebraic variety of $\ell$-planes meeting $X$. Prove that $C(X)$ is irreducible, and find its dimension.

  \item (RA) Let $\left\{f_{n}\right\}$ be a sequence of functions on $X=(0,1) \subset \mathbb{R}$, converging almost everywhere to $f$. Suppose moreover that $\sup _{n}\left\|f_{n}\right\|_{L^{2}(X)} \leq M$ for some $M$ fixed. Under these conditions, answer the following questions by giving a counterexample or proving your answer.

\end{enumerate}

(a) Do we know $\|f\|_{L^{2}(X)}<\infty$ ?

(b) Do we know $\lim _{n \rightarrow \infty}\left\|f_{n}-f\right\|_{L^{2}(X)}=0$ ? Do we know that

$$
\lim _{n \rightarrow \infty}\left\|f_{n}-f\right\|_{L^{p}(X)}=0 \quad \text { for } \quad 1<p<2 ?
$$

(c) If we assume, in addition, that $\lim _{n \rightarrow \infty}\left\|f_{n}\right\|_{L^{2}(X)}=\|f\|_{L^{2}(X)}<\infty$, do we know that

$$
\lim _{n \rightarrow \infty}\left\|f_{n}-f\right\|_{L^{2}(X)}=0 ?
$$

\begin{enumerate}
  \setcounter{enumi}{3}
  \item (A) Let $R$ be a commutative ring with 1 . Show that if every proper ideal of $R$ is a prime ideal, then $R$ is a field.

  \item (AT) Let $D=\{z \in \mathbb{C}:|z| \leq 1\}$ be the closed unit disc in the complex plane, and let $X$ be the space obtained from $D$ by identifying points on the boundary differing by multiplication by powers of $e^{2 \pi i / 5}$; that is, we let $\sim$ be the equivalence relation on $D$ given by

\end{enumerate}

$$
z \sim w \text { if }|z|=|w|=1 \text { and }(z / w)^{5}=1 .
$$

(a) Find the homology groups of $X$ with coefficients in $\mathbb{Z}$.

(b) Find the homology groups of $X$ with coefficients in $\mathbb{Z} / 5$.

\begin{enumerate}
  \setcounter{enumi}{5}
  \item (DG) Suppose $G$ is a compact Lie group with Lie algebra $\mathfrak{g}$. Consider an element $g \in G$, and let $\mathfrak{c} \subset \mathfrak{g}$ be the subalgebra $\mathfrak{c}=\left\{X \mid \operatorname{Ad}_{g}(X)=X\right\}$. Show there exists some $\epsilon>0$ such that for all $X \in \mathfrak{g}$ with $|X|<\epsilon$, there exists $Y \in \mathfrak{c} \operatorname{such}$ that $g \exp (X)$ is conjugate to $g \exp (Y)$.
\end{enumerate}

\section*{QUALIFYING EXAMINATION }
Thursday September 3, 2020 (Day 3)

\begin{enumerate}
  \item (AG) Let $C \subset \mathbb{P}^{3}$ be an algebraic curve (that is, an irreducible, one-dimensional subvariety of $\left.\mathbb{P}^{3}\right)$, and suppose that $p_{C}(m)$ and $h_{C}(m)$ are its Hilbert polynomial and Hilbert function respectively. Which of the following are possible?

  \item $p_{C}(m)=3 m+1$ and $h_{C}(1)=3$;

  \item $p_{C}(m)=3 m+1$ and $h_{C}(1)=4$.

  \item (RA) The weak law of large numbers states that the following is correct: Let $X_{1}, X_{2}, \ldots X_{n}$ be independent random variables such that $\left|\mu_{j}\right|=\left|\mathbb{E} X_{j}\right| \leq 1$ and $\mathbb{E}\left(X_{j}-\mu_{j}\right)^{2}=V_{j} \leq 1$. Let $S_{n}=X_{1}+\ldots+X_{n}$. Then for any $\varepsilon>0$

\end{enumerate}

$$
\lim _{n \rightarrow \infty} \mathbb{P}\left(\left|\frac{S_{n}-\sum_{j} \mu_{j}}{n}\right|>\varepsilon\right)=0
$$

Now suppose that we don't know the independence of the sequence $X_{1}, X_{2}, \ldots X_{n}$, but we know that there is a function $g:\{0\} \cup \mathbb{N} \rightarrow \mathbb{R}$ with $\lim _{k \rightarrow \infty} g(k)=0$ such that for all $j \geq i$

$$
\mathbb{E} X_{i} X_{j}=g(j-i)
$$

In other words, the correlation functions vanishing asymptotically. Do we know whether the conclusion $(+)$ still holds? Give a counterexample or prove your answer.

\begin{enumerate}
  \setcounter{enumi}{2}
  \item $(\mathrm{CA})$
\end{enumerate}

(a) Suppose that both $f$ and $g$ are analytic in a neighborhood of a disk $D$ with boundary circle $C$. If $|f(z)|>|g(z)|$ for all $z \in C$, prove that $f$ and $f+g$ have the same number of zeros inside $C$, counting multiplicity.

(b) How many roots of

$$
p(z)=z^{7}-2 z^{5}+6 z^{3}-z+1=0
$$

are there in the unit disc in $|z|<1$, again counting multiplicity?

\begin{enumerate}
  \setcounter{enumi}{3}
  \item (AT) Let $S^{1}=\mathbb{R} / \mathbb{Z}$ be a circle, and let $S^{2}$ be a two-dimensional sphere. Consider involutions on both, with an involution on $S^{1}$ defined by $x \mapsto-x$
for $x \in \mathbb{R}$, and with $j: S^{2} \rightarrow S^{2}$ defined by reflection about an equator. Let $M$ be the space of maps that respects these involutions, i.e.
\end{enumerate}

$$
M=\left\{f: S^{1} \rightarrow S^{2} \mid f(-x)=j(f(x))\right\}
$$

Show $M$ is connected but not simply-connected.

\begin{enumerate}
  \setcounter{enumi}{4}
  \item (DG) Let $\mathbb{H}$ denote the upper half-plane; that is, $\mathbb{H}=\{z \in \mathbb{C}: \operatorname{Im} z>0\}$, with the metric $\frac{1}{y^{2}} d x d y$ for $z=x+i y$. Suppose $\Gamma$ is a group of isometries acting on $\mathbb{H}$ such that $\mathbb{H} / \Gamma$ is a smooth surface $S$, and you are given that a fundamental domain $D$ for the action of $\Gamma$ on $\mathbb{H}$ is given as follows:
\end{enumerate}

$$
D=\left\{x+i y \in \mathbb{H} \mid-\frac{3}{2} \leq x \leq \frac{3}{2},(x-c)^{2}+y^{2} \geq \frac{1}{9} \text { for } c \in\left\{ \pm \frac{1}{3}, \pm \frac{2}{3}, \pm \frac{4}{3}\right\}\right\}
$$

Compute $\chi(S)$ using Gauss-Bonnet. You may use that the (Gaussian) curvature of $\mathbb{H}$ is identically equal to -1 .

\begin{enumerate}
  \setcounter{enumi}{5}
  \item (A) Fix a prime $p$.
\end{enumerate}

i) Suppose $F$ is a field of characteristic $p$, and $c \in F$ is not of the form $a^{p}-a$ for any $a \in F$. Prove that the polynomial $P(X)=X^{p}-X-c$ is irreducible and that if $x$ is any root of $P$ then $F(x)$ is a normal extension of $F$ with Galois group isomorphic with $\mathbf{Z} / p \mathbf{Z}$.

ii) Suppose $Q \in \mathbf{Z}[X]$ is a monic polynomial of degree $p$ such that $Q \equiv$ $X^{p}-X-c \bmod p$ for some integer $c \neq 0 \bmod p$, and that $Q$ has exactly $p-2$ real roots. Prove that the Galois group of $Q$ is the full symmetric group $S_{p}$.


\end{document}