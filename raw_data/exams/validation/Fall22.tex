\documentclass[10pt]{article}
\usepackage[utf8]{inputenc}
\usepackage[T1]{fontenc}
\usepackage{amsmath}
\usepackage{amsfonts}
\usepackage{amssymb}
\usepackage[version=4]{mhchem}
\usepackage{stmaryrd}
\usepackage{bbold}

\title{QUALIFYING EXAMINATION }


\author{HARVARD UNIVERSITY\\
Department of Mathematics}
\date{}


\begin{document}
\maketitle
Department of Mathematics

Tuesday August 30, 2022 (Day 1)

\begin{enumerate}
  \item (AG) Let $V, W$ be complex vector spaces of dimensions $m \geq n \geq 2$, respectively. Let $\mathbb{P H o m}(V, W) \cong \mathbb{P}^{m n-1}$ be the projective space of nonzero linear maps $\phi: V \rightarrow W$ modulo scalars. Further, let $\Phi \subset \mathbb{P H o m}(V, W)$ be the subset of those linear maps $\phi$ which do not have full rank $n$. Prove that $\Phi$ is an irreducible subvariety of $\mathbb{P}^{m n-1}$ and find its dimension.

  \item (AT) Let $S^{n}$ be the standard $n$-sphere

\end{enumerate}

$$
S^{n}=\left\{\left(x_{0}, \ldots, x_{n}\right) \in \mathbb{R}^{n+1} \mid \sum x_{i}^{2}=1\right\}
$$

and let $S^{k} \subset S^{n}$ be the locus defined by the vanishing of the last $n-k$ coordinates $x_{k+1}, \ldots, x_{n}$. Assume $n-1>k>0$.

\begin{enumerate}
  \item Find the homology groups of the complement $S^{n} \backslash S^{k}$.

  \item Suppose now that $T \subset S^{n}$ is the sphere defined by the vanishing of the first $k$ coordinates; that is,

\end{enumerate}

$$
T=\left\{\left(0, \ldots 0, x_{k+1}, \ldots, x_{n}\right) \in \mathbb{R}^{n+1} \mid \sum x_{i}^{2}=1\right\}
$$

What is the fundamental class of $T$ in the homology group $H_{n-k-1}\left(S^{n} \backslash\right.$ $\left.S^{k}\right) ?$

\begin{enumerate}
  \setcounter{enumi}{2}
  \item (CA) Compute
\end{enumerate}

$$
\int_{0}^{2 \pi} \frac{1}{(3+\cos \theta)^{2}} d \theta
$$

using contour integration.

\begin{enumerate}
  \setcounter{enumi}{3}
  \item (A) Show that the symmetric group $S_{n}$ has at least one Sylow $p$-subgroup which is a cyclic group of order $p$. (You may use the fact that for any prime $p$, there exists a prime in the interval $(p, 2 p)$.)

  \item (DG) Let $X=T^{*} \mathbb{C}^{\times}=\mathbb{C}^{\times} \times \mathbb{C}$, where we write $z, w$ for holomorphic coordinates on the base and fiber, respectively. Find all time-1 periodic orbits of the vector field $V=\operatorname{Re}\left(z w \frac{\partial}{\partial z}\right)$ - i.e., all points $x \in X$ such that the time-1 flow of $x$ under $V$ is equal to $x$.

  \item (RA) Let $X_{1}, X_{2}, X_{3}, \ldots$ be independent and identically distributed random variables with finite expected value $\mu$ and finite nonzero variance. Let

\end{enumerate}

$$
\overline{X_{n}}=\frac{1}{n}\left(X_{1}+\cdots+X_{n}\right)
$$

Use Chebyshev's inequality to prove that $\overline{X_{n}}$ converges to $\mu$ in probability as $n \rightarrow \infty$.

\section*{QUALIFYING EXAMINATION }
Department of Mathematics

Wednesday August 31, 2022 (Day 2)

\begin{enumerate}
  \item (CA) Let $\Omega \subset \mathbb{C}$ denote the open set
\end{enumerate}

$$
\Omega=\{z:|z-1|>1 \text { and }|z-3|<3\} \text {. }
$$

Give a conformal isomorphism between $\Omega$ and the unit disk $D=\{z:|z|<1\}$.

\begin{enumerate}
  \setcounter{enumi}{1}
  \item (A) Let $F=\mathbb{Q}(z)$ where
\end{enumerate}

$$
z=\cos \frac{2 \pi}{13}+\cos \frac{10 \pi}{13}
$$

i) Prove that $[F: \mathbb{Q}]=3$ and $F / \mathbb{Q}$ is a Galois extension.

ii) Prove that if $p$ is a prime and $p \neq 13$ then $p$ is unramified in $F$, and that $p$ is split in $F$ if and only if $p \equiv \pm 1$ or $\pm 5 \bmod 13$.

\begin{enumerate}
  \setcounter{enumi}{2}
  \item (DG) Let $u \mapsto \tau(u)$, for $a<u<b$, be a smooth space curve in $\mathbb{R}^{3}$ with both its curvature and torsion nowhere zero. Assume that the parameter $u$ is the arc-length of $u \mapsto \tau(u)$. Suppose $\sigma(v)$, for $c<v<d$, is a smooth function with $\sigma^{\prime}(v)$ nowhere zero. Consider the surface $S$ defined by
\end{enumerate}

$$
(u, v) \mapsto \vec{r}(u, v)=\tau(u)+\sigma(v) \tau^{\prime}(u)
$$

for $a<u<b$ and $c<v<d$. Compute the first and second fundamental forms of $S$ in terms of $\tau(u)$ and $\sigma(v)$ and their derivatives. Determine the condition on the function $\sigma(v)$ so that the Gaussian curvature of the surface $S$ is identically zero.

\begin{enumerate}
  \setcounter{enumi}{3}
  \item (RA) Let $V$ be the vector space of continuous functions $[0,1] \rightarrow \mathbb{R}$, and let $g: V \rightarrow \mathbb{R}$ be the linear functional $f \mapsto \int_{0}^{1} x^{-1 / 3} f(x) d x$. For which $p \in(1, \infty)$ does $g$ extend to a continuous functional $\bar{g}: L^{p}([0,1]) \rightarrow \mathbb{R}$ ? For those $p$, what is the norm of this functional?

  \item (AG) Let $X \subset \mathbb{P}^{n}$ be any hypersurface of degree $d \geq 2$, and $\Lambda \subset X \subset \mathbb{P}^{n}$ a $k$-plane in $\mathbb{P}^{n}$ contained in $X$.

  \item Show that if $k \geq n / 2$, then $X$ is necessarily singular.

  \item If $k=n / 2$ and $X \subset \mathbb{P}^{n}$ is a general hypersurface containing a $k$-plane, describe the singular locus of $X$.

  \item $(\mathrm{AT})$

\end{enumerate}

(a) Given compact oriented manifolds $M$ and $N$, both of dimension $n$, define the degree of a continuous map $f: M \rightarrow N$.

(b) What are the possible degrees of continuous maps $\mathbb{C P}^{4} \rightarrow \mathbb{C P}^{4}$ ? Justify your answer.

\section*{QUALIFYING EXAMINATION }
Thursday September 1, 2022 (Day 3)

\begin{enumerate}
  \item (DG) Let $X$ be a compact Riemannian manifold.
\end{enumerate}

(a) Let $\xi_{i}$ be a smooth 1-form on $X$ which is both $d$-closed and $d^{*}$-closed. Let $\Delta$ denote the Laplacian. Denote by $|\xi|$ the pointwise norm of $\xi$. Denote by $|\nabla \xi|$ the pointwise norm of the covariant differential $\nabla \xi$ of $\xi$. Use the notation Ricci for the Ricci tensor of $X$. Prove the following identity of Bochner on $X$

$$
\frac{1}{2} \Delta\left(|\xi|^{2}\right)=|\nabla \xi|^{2}+\operatorname{Ricci}(\xi, \xi)
$$

by directly computing $\Delta\left(|\xi|^{2}\right)$ and appropriately contracting the commutation formula for $\nabla_{\alpha} \nabla_{\beta} \xi-\nabla_{\alpha} \nabla_{\beta} \xi$ with $\xi$ to yield the Ricci term.

(b) Assume that the Ricci curvature is positive semidefinite everywhere on $X$ and is strictly positive at at least one point of $X$. By integrating Bochner's identity in (a) over $X$ to prove that every harmonic 1-form on $X$ must be identically zero. Here harmonic means $d$-closed and $d^{*}$-closed.

\begin{enumerate}
  \setcounter{enumi}{1}
  \item (RA) Suppose $w:[0,1] \rightarrow(0, \infty)$ is a continuous function.
\end{enumerate}

i) Prove that there exist unique monic polynomials $p_{0}, p_{1}, p_{2}, \ldots \in \mathbb{R}[x]$ such that each $p_{n}$ has degree $n$ and $\int_{0}^{1} w(x) p_{m}(x) p_{n}(x) d x=0$ for all $m, n \geq 0$ such that $m \neq n$.

ii) Prove that for each $n>0$ the four polynomials $p_{n-1}, p_{n}, x p_{n}, p_{n+1}$ are linearly dependent.

\begin{enumerate}
  \setcounter{enumi}{2}
  \item (AG) Let $\Gamma \subset \mathbb{P}^{n}$ be any closed algebraic variety.

  \item Define the Hilbert function $h_{\Gamma}(m)$.

  \item If $\Gamma=D \cap E \subset \mathbb{P}^{2}$ is the transverse intersection of plane curves $D, E$ of degrees $d$ and $e$, what is the Hilbert function of $\Gamma$ ?

  \item (AT) Let $G=\mathbb{Z} / m$ denote a finite cyclic group of odd order $m$. Suppose that we are given a free action of $G$ on $S^{3}$. Compute the homology groups with integer coefficients of of the orbit space $M=S^{3} / G$.

  \item (CA) Let $f(z)$ be an entire function. Assume that for any $z_{0} \in \mathbb{R}$, at least one coefficient in the analytic expansion $f(z)=\sum_{n=0}^{\infty} c_{n}\left(z-z_{0}\right)^{n}$ around $z_{0}$ is equal to zero, i.e. $c_{n}=0$, for some $n \in \mathbb{Z}_{\geq 0}$. Prove that $f$ is a polynomial.

  \item (A) Let $k$ be the finite field $\mathbb{Z} / 13 \mathbb{Z}$; let $C$ be the subgroup $\{1,5,8,12\}$ of $k^{*}$; and let $G$ be the group of 52 permutations of $k$ of the form $g_{a, b}: x \mapsto a x+b$ where $a \in C$ and $b \in k$. Let $(V, \rho)$ be the permutation representation of $G$ acting on complex-valued functions on $k$, and $\chi$ its associated character.

  \item i) Determine $\chi\left(g_{a, b}\right)$ for all $a \in C$ and $b \in g$, and prove that $\langle\mathbf{1}, \chi\rangle=1$ and $\langle\chi, \chi\rangle=4$. Here $\mathbf{1}$ is the character of the trivial 1-dimensional representation $V_{1}$ of $G$.

  \item Deduce that $V$ is the direct sum of four pairwise non-isomorphic irreducible representations of $G$.

\end{enumerate}

\end{document}