\documentclass[10pt]{article}
\usepackage[utf8]{inputenc}
\usepackage[T1]{fontenc}
\usepackage{amsmath}
\usepackage{amsfonts}
\usepackage{amssymb}
\usepackage[version=4]{mhchem}
\usepackage{stmaryrd}
\usepackage{bbold}

\title{QUALIFYING EXAMINATION 
 HARVARD UNIVERSITY 
 Department of Mathematics 
 Tuesday September 2, 2014 (Day 1) }

\author{}
\date{}


\begin{document}
\maketitle
\begin{enumerate}
  \item (AG) For any $0<k<m \leq n \in \mathbb{Z}$, let $M \cong \mathbb{P}^{m n-1}$ be the space of nonzero $m \times n$ matrices mod scalars, and let $M_{k} \subset M$ be the subset of matrices of rank $k$ or less.
\end{enumerate}

(a) Show that $M_{k}$ is closed in $M$ (in the Zariski topology).

(b) Show that $M_{k}$ is irreducible.

(c) What is the dimension of $M_{k}$ ?

(d) What is the degree of $M_{1}$ ?

\begin{enumerate}
  \setcounter{enumi}{1}
  \item (A) Let $S_{3}$ be the group of automorphisms of a 3-element set.
\end{enumerate}

(a) Classify the conjugacy classes of $S_{3}$.

(b) Classify the irreducible representations of $S_{3}$.

(c) Write the character table for $S_{3}$.

\begin{enumerate}
  \setcounter{enumi}{2}
  \item (DG) Let $x, y, z$ be the standard coordinates on $\mathbb{R}^{3}$. Consider the unit sphere $\mathbb{S}^{2} \subset \mathbb{R}^{3}$.

  \item Compute the critical points of the function $\left.x\right|_{\mathbb{S}^{2}}$. Show that they are isolated and non-degenerate.

  \item Equip $\mathbb{S}^{2}$ with the standard metric induced from $\mathbb{R}^{3}$. Compute the gradient vector field of $\left.x\right|_{\mathbb{S}^{2}}$. Compute the integral curves of this vector field.

  \item (RA)

\end{enumerate}

Find a solution for the heat equation

$$
\frac{\partial}{\partial t} u(x, t)-\frac{\partial^{2}}{\partial x^{2}} u(x, t)=0, \quad(t>0, \quad 0<x<1)
$$

with the initial condition $u(x, 0)=A$ where $A$ is a constant and the boundary conditions $u(0, t)=u(1, t)=0, \quad t>0$.

\begin{enumerate}
  \setcounter{enumi}{4}
  \item (AT)
\end{enumerate}

(a) Show that a continuous map $f: X \rightarrow \mathbb{R P}^{n}$ factors through $S^{n} \rightarrow \mathbb{R} \mathbb{P}^{n}$ if and only if the induced map $f^{*}: H^{1}\left(\mathbb{R}^{n} ; \mathbb{Z} / 2\right) \rightarrow H^{1}(X, \mathbb{Z} / 2)$ is zero.
(b) Show that a continuous map $f: X \rightarrow \mathbb{C P}^{n}$ factors through $S^{2 n+1} \rightarrow \mathbb{C P}^{n}$ if and only if the induced map $f^{*}: H^{2}\left(\mathbb{C P}^{n} ; \mathbb{Z}\right) \rightarrow H^{2}(X, \mathbb{Z})$ is zero.

\begin{enumerate}
  \setcounter{enumi}{5}
  \item (CA) Let $f$ be a meromorphic function on a contractible region $U \subset \mathbb{C}$, and let $\gamma$ be a simple closed curve inside that region. Recall that the argument principle for a meromorphic function says that the integral
\end{enumerate}

$$
\frac{1}{2 \pi i} \int_{\gamma} \frac{f^{\prime}}{f}
$$

is equal to the number of zeroes minus the number of poles of $f$ inside $\gamma$.

(a) Prove Rouche's Theorem. That is, assume (1) $f$ and $g$ are holomorphic in $U,(2) \gamma$ is a simple, smooth, closed curve in $U$, and (3) $|f|>|g|$ on $\gamma$. Then the number of zeroes of $f+g$ inside $\gamma$ is equal to the number of zeroes of $f$ inside $\gamma$. You may assume the Argument Principle.

(b) Show that for any $n$, the roots of the polynomial

$$
\sum_{i=0}^{n} z^{i}
$$

all have absolute value less than 2 .

\section{QUALIFYING EXAMINATION 
 HARVARD UNIVERSITY 
 Department of Mathematics 
 Wednesday September 3, 2014 (Day 2)}
\begin{enumerate}
  \item (AT)
\end{enumerate}

(a) Let $X$ and $Y$ be compact, oriented manifolds of the same dimension $n$. Define the degree of a continuous map $f: X \rightarrow Y$.

(b) What are all possible degrees of continuous maps $f: \mathbb{C P}^{3} \rightarrow \mathbb{C P}^{3}$ ?

\begin{enumerate}
  \setcounter{enumi}{1}
  \item (A)
\end{enumerate}

(a) Show that every finite extension of a finite field is simple (i.e., generated by attaching a single element).

(b) Fix a prime $p \geq 2$ and let $\mathbb{F}_{p}$ be the field of cardinality $p$. For any $n \geq 1$, show that any two fields of degree $n$ over $\mathbb{F}_{p}$ are isomorphic as fields.

\begin{enumerate}
  \setcounter{enumi}{2}
  \item (CA) Fix two positive real numbers $a, b>0$. Calculate the value of the integral
\end{enumerate}

$$
\int_{-\infty}^{\infty} \frac{\cos (a x)-\cos (b x)}{x^{2}} d x
$$

\begin{enumerate}
  \setcounter{enumi}{3}
  \item (AG) Let $C \subset \mathbb{P}^{2}$ be the smooth plane curve of degree $d>1$ defined by the homogeneous polynomial $F(X, Y, Z)=0$
\end{enumerate}

(a) If $p \in C$, find the homogeneous linear equation of the tangent line $\mathbb{T}_{p} C \subset$ $\mathbb{P}^{2}$ to $C$ at $p$.

(b) Let $\mathbb{P}^{2 *}$ be the dual projective plane, whose points correspond to lines in $\mathbb{P}^{2}$. Show that the Gauss map $g: C \rightarrow \mathbb{P}^{2 *}$ sending each point $p \in C$ to its tangent line $\mathbb{T}_{p} C \in \mathbb{P}^{2 *}$ is a regular map.

(c) Let $C^{*} \subset \mathbb{P}^{2 *}$ be the dual curve of $C$; that is, the image of the Gauss map. Assuming that the Gauss map is birational onto its image, what is the degree of $C^{*} \subset \mathbb{P}^{2 *}$ ?

\begin{enumerate}
  \setcounter{enumi}{4}
  \item (DG) Let $U$ the be upper half plane $U=\left\{(x, y) \in \mathbb{R}^{2} \mid y>0\right\}$ and introduce the Poincaré metric
\end{enumerate}

$$
g=y^{-2}(d x \otimes d x+d y \otimes d y)
$$

Write the geodesic equations.

\begin{enumerate}
  \setcounter{enumi}{5}
  \item (RA)
\end{enumerate}

(a) Define what is meant by an equicontinuous sequence of functions on the closed interval $[-1,1] \subset \mathbb{R}$.
(b) Prove the Arzela-Ascoli theorem: that if $\left\{f_{n}\right\}_{n=1,2, \ldots}$ is a bounded, equicontinuous sequence of functions on $[-1,1]$, then there exists a continuous function $f$ on $[-1,1]$ and an infinite subsequence $\Lambda \subset\{1,2, \ldots\}$ such that

$$
\lim _{n \in \Lambda \text { and } n \rightarrow \infty}\left(\sup _{t \in[-1,1]}\left|f_{n}(t)-f(t)\right|\right)=0
$$

\section{QUALIFYING EXAMINATION 
 HARVARD UNIVERSITY 
 Department of Mathematics 
 Thursday September 4, 2014 (Day 3)}
\begin{enumerate}
  \item (DG) The symplectic group $S p(2 n, \mathbb{R})$ is defined as the subgroup of $G l(2 n, \mathbb{R})$ that preserves the matrix
\end{enumerate}

$$
\Omega=\left(\begin{array}{cc}
0 & I_{n} \\
-I_{n} & 0
\end{array}\right)
$$

where $I_{n}$ is the $n \times n$ identify matrix. That is, it is composed of elements of $G l(2 n, \mathbb{R})$ that satisfy the relation

$$
M^{T} \Omega M=\Omega
$$

(a) Show that every symplectic matrix is invertible with inverse $M^{-1}=$ $\Omega^{-1} M^{T} \Omega$.

(b) Show that the square of the determinant of a symplectic metric is 1 . (In fact, the determinant of a symplectic matrix is always 1 , but you don't need to show this.)

(c) Compute the dimension of the symplectic group.

\begin{enumerate}
  \setcounter{enumi}{1}
  \item (RA) Suppose that $\sigma$ is a positive number and $f$ is a non-negative function on $\mathbb{R}$ such that
\end{enumerate}

$$
\int_{\mathbb{R}} f(x) d x=1 ; \quad \int_{\mathbb{R}} x f(x) d x=0 \quad \text { and } \quad \int_{\mathbb{R}} x^{2} f(x) d x=\sigma^{2}
$$

Let $\mathcal{P}$ denote the probability measure on $\mathbb{R}$ with density function $f$.

(a) Supposing that $\rho$ is a positive number, give a non-trivial upper bound in terms of $\sigma$ for the probability as measured by $\mathcal{P}$ of the subset $[\rho, \infty)$.

(b) Given a positive integer $N$, let $\left\{X_{1}, \ldots, X_{N}\right\}$ denote $N$ independent random variables on $\mathbb{R}$, each with the same probability measure $\mathcal{P}$. Let $S_{N}$ be the random variable on $\mathbb{R}^{N}$ given by

$$
S_{N}=\frac{1}{N} \sum_{i=1}^{N} X_{i}
$$

What are the mean and standard deviation of $S_{N}$ ?

(c) Let $\left\{X_{1}, X_{2}, \ldots, X_{N}\right\}$ be independent random variables on $\mathbb{R}$, each with the same probability measure $\mathcal{P}$, and let $P_{N}(x)$ denote the function on $\mathbb{R}$ given by the probability that

$$
\frac{1}{\sqrt{N}} \sum_{k=1}^{N} X_{k}<x
$$

Given $x \in \mathbb{R}$, what is the limit as $N \rightarrow \infty$ of the sequence $\left\{P_{N}(x)\right\}$ ?

\begin{enumerate}
  \setcounter{enumi}{2}
  \item (AG) Let $X$ be the blow-up of $\mathbb{P}^{2}$ at a point.
\end{enumerate}

(a) Show that the surfaces $\mathbb{P}^{2}, \mathbb{P}^{1} \times \mathbb{P}^{1}$ and $X$ are all birational.

(b) Prove that no two of the surfaces $\mathbb{P}^{2}, \mathbb{P}^{1} \times \mathbb{P}^{1}$ and $X$ are isomorphic.

\begin{enumerate}
  \setcounter{enumi}{3}
  \item (AT) Suppose that $G$ is a finite group whose abelianization is trivial. Suppose also that $G$ acts freely on $S^{3}$. Compute the homology groups (with integer coefficients) of the orbit space $M=S^{3} / G$.

  \item (CA) Recall that a function $u: \mathbb{R}^{2} \rightarrow \mathbb{R}$ is called harmonic if $\Delta u:=\partial_{x}^{2} u+$ $\partial_{y}^{2} u=0$. Prove the following statements using harmonic conjugates and standard complex analysis.

\end{enumerate}

(a) Show that the average value of a harmonic function along a circle is equal to the value of the harmonic function at the center of the circle.

(b) Show that the maximum value of a harmonic function on a closed disk occurs only on the boundary, unless $u$ is constant.

\begin{enumerate}
  \setcounter{enumi}{5}
  \item (A) Let $G$ be a finite group.
\end{enumerate}

(a) Let $V$ be any $\mathbb{C}$-representation of $G$. Show that $V$ admits a Hermitian, $G$-invariant inner product.

(b) Let $N$ be a $\mathbb{C}[G]$-module which is finite-dimensional over $\mathbb{C}$, and let $M \subset N$ a submodule. Show that the inclusion splits.

(c) Consider the action of $S_{3}$ on $\mathbb{C}^{3}$ given by permuting the axes. Decompose $\mathbb{C}^{3}$ into irreducible $S_{3}$-representations.


\end{document}