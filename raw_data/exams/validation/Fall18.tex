\documentclass[10pt]{article}
\usepackage[utf8]{inputenc}
\usepackage[T1]{fontenc}
\usepackage{amsmath}
\usepackage{amsfonts}
\usepackage{amssymb}
\usepackage[version=4]{mhchem}
\usepackage{stmaryrd}
\usepackage{bbold}

\title{QUALIFYING EXAMINATION }


\author{HARVARD UNIVERSITY}
\date{}


\begin{document}
\maketitle
Department of Mathematics

Tuesday September 4, 2018 (Day 1)

\begin{enumerate}
  \item (AT)
\end{enumerate}

(a) Let $X$ and $Y$ be compact, oriented manifolds of the same dimension $n$. Define the degree of a continuous map $f: X \rightarrow Y$.

(b) Let $f: \mathbb{C P}^{3} \rightarrow \mathbb{C P}^{3}$ be any continuous map. Show that the degree of $f$ is of the form $m^{3}$ for some integer $m$.

(c) Show that conversely for any $m \in \mathbb{Z}$ there is a continuous map $f: \mathbb{C P}^{3} \rightarrow$ $\mathbb{C P}^{3}$ of degree $m^{3}$.

\begin{enumerate}
  \setcounter{enumi}{1}
  \item (A) Let $G$ be a group.
\end{enumerate}

(a) Prove that, if $V$ and $W$ are irreducible $G$-representations defined over a field $\mathbb{F}$, then a $G$-homomorphism $f: V \rightarrow W$ is either zero or an isomorphism.

(b) Let $G=D_{8}$ be the dihedral group with 8 elements. What are the dimensions of its irreducible representations over $\mathbb{C}$ ?

\begin{enumerate}
  \setcounter{enumi}{2}
  \item (CA) Let $f_{n}$ be a sequence of analytic functions on the unit disk $\Delta \subset \mathbb{C}$ such that $f_{n} \rightarrow f$ uniformly on compact sets and such that $f$ is not identically zero. Prove that $f(0)=0$ if and only if there is a sequence $z_{n} \rightarrow 0$ such that $f_{n}\left(z_{n}\right)=0$ for $n$ large enough.

  \item (AG) Let $K$ be an algebraically closed field of characteristic 0 , and let $\mathbb{P}^{n}$ be the projective space of homogeneous polynomials of degree $n$ in two variables over $K$. Let $X \subset \mathbb{P}^{n}$ be the locus of $n^{\text {th }}$ powers of linear forms, and let $Y \subset \mathbb{P}^{n}$ be the locus of polynomials with a multiple root (that is, a repeated factor).

\end{enumerate}

(a) Show that $X$ and $Y \subset \mathbb{P}^{n}$ are closed subvarieties.

(b) What is the degree of $X$ ?

(c) What is the degree of $Y$ ?

\begin{enumerate}
  \setcounter{enumi}{4}
  \item (DG) Given a smooth function $f: \mathbb{R}^{n-1} \rightarrow \mathbb{R}$, define $F: \mathbb{R}^{n} \rightarrow \mathbb{R}$ by
\end{enumerate}

$$
F\left(x_{1}, \ldots, x_{n}\right):=f\left(x_{1}, \ldots, x_{n-1}\right)-x_{n}
$$

and consider the preimage $X_{f}=F^{-1}(0) \subset \mathbb{R}^{n}$.

(a) Prove that $X_{f}$ is a smooth manifold which is diffeomorphic to $\mathbb{R}^{n-1}$.

(b) Consider the two examples $X_{f}$ and $X_{g} \subset \mathbb{R}^{3}$ with $f\left(x_{1}, x_{2}\right)=x_{1}^{2}+x_{2}^{2}$ and $g\left(x_{1}, x_{2}\right)=x_{1}^{2}-x_{2}^{2}$. Compute their normal vectors at every point $\left(x_{1}, x_{2}, x_{3}\right) \in X_{f}$ and $\left(x_{1}, x_{2}, x_{3}\right) \in X_{g}$.

\begin{enumerate}
  \setcounter{enumi}{5}
  \item (RA) Let $K \subset \mathbb{R}^{n}$ be a compact set. Show that for any measurable function $f: K \rightarrow \mathbb{C}$, it holds that
\end{enumerate}

$$
\lim _{p \rightarrow \infty}\|f\|_{L^{p}(K)}=\|f\|_{L^{\infty}(K)}
$$

(Recall that $\|f\|_{L^{p}(K)}=\left(\int_{K}|f|^{p} \mathrm{~d} x\right)^{1 / p}$ and that $\|f\|_{L^{\infty}(K)}$ is the essential supremum of $f$, i.e., the smallest upper bound if the behavior of $f$ on null sets is ignored.)

\section{QUALIFYING EXAMINATION}
\section{Harvard UNiVERsity}
Department of Mathematics

Wednesday September 5, 2018 (Day 2)

\begin{enumerate}
  \item (AG) Let $C \subset \mathbb{P}^{2}$ be a smooth plane curve of degree $d$.
\end{enumerate}

(a) Let $K_{C}$ be the canonical bundle of $C$. For what integer $n$ is it the case that $K_{C} \cong \mathcal{O}_{C}(n)$ ?

(b) Prove that if $d \geq 4$ then $C$ is not hyperelliptic.

(c) Prove that if $d \geq 5$ then $C$ is not trigonal (that is, expressible as a 3 -sheeted cover of $\mathbb{P}^{1}$ ).

\begin{enumerate}
  \setcounter{enumi}{1}
  \item (CA) (The $1 / 4$ theorem). Let $\mathcal{S}$ denote the class of functions that are analytic on the disk and one-to-one with $f(0)=0$ and $f^{\prime}(0)=1$.
\end{enumerate}

(a) Prove that if $f \in \mathcal{S}$ and $w$ is not in the range of $f$ then

$$
g(z)=\frac{w f(z)}{(w-f(z))}
$$

is also in $\mathcal{S}$.

(b) Show that for any $f \in \mathcal{S}$, the image of $f$ contains the ball of radius $1 / 4$ around the origin. You may use the elementary result (Bieberbach) that if $f(z)=z+\sum_{k \geq 2} a_{k} z^{k}$ in $\mathcal{S}$ then $\left|a_{2}\right| \leq 2$.

\begin{enumerate}
  \setcounter{enumi}{2}
  \item (A) Find a polynomial $f \in \mathbb{Q}[x]$ whose Galois group (over $\mathbb{Q}$ ) is $D_{8}$, the dihedral group of order 8 .

  \item (RA)

\end{enumerate}

(a) Let $a_{k} \geq 0$ be a monotone increasing sequence with $a_{k} \rightarrow \infty$, and consider the ellipse,

$$
E\left(a_{k}\right)=\left\{v \in \ell^{2}(\mathbb{Z}): \sum a_{k} v_{k}^{2} \leq 1\right\}
$$

Show that $E\left(a_{n}\right)$ is a compact subset of $\ell^{2}(\mathbb{Z})$.
(b) Let $\mathbb{T}$ denote the one-dimensional torus; that is, $\mathbb{R} / 2 \pi \mathbb{Z}$, or $[0,2 \pi]$ with the ends identified. Recall that the space $H^{1}(\mathbb{T})$ is the closure of $C^{\infty}(\mathbb{T})$ in the norm

$$
\|f\|_{H^{1}(\mathbb{T})}=\sqrt{\|f\|_{L^{2}(\mathbb{T})}+\left\|\frac{d}{d x} f\right\|_{L^{2}(\mathbb{T})}}
$$

Use part (a) to conclude that the inclusion $i: H^{1}(\mathbb{T}) \hookrightarrow L^{2}(\mathbb{T})$ is a compact operator.

\begin{enumerate}
  \setcounter{enumi}{4}
  \item (AT) Consider the following topological spaces:
\end{enumerate}

$$
A=S^{1} \times S^{1} \quad B=S^{1} \vee S^{1} \vee S^{2}
$$

(a) Compute the fundamental group of each space.

(b) Compute the integral cohomology ring of each space.

(c) Show that $B$ is not homotopy equivalent to any compact orientable manifold.

\begin{enumerate}
  \setcounter{enumi}{5}
  \item (DG) Consider the set
\end{enumerate}

$$
G:=\left\{\left(\begin{array}{ccc}
x & 0 & 0 \\
0 & x & y \\
0 & 0 & 1
\end{array}\right): x \in \mathbb{R}_{+}, y \in \mathbb{R}\right\}
$$

and equip it with a smooth structure via the global chart that sends $(x, y) \in$ $\mathbb{R}_{+} \times \mathbb{R}$ to the corresponding element of $G$.

(a) Show that $G$ is a Lie subgroup of the Lie group $G L(\mathbb{R}, 3)$.

(b) Prove that the set

$$
\left\{x \frac{\partial}{\partial x}, x \frac{\partial}{\partial y}\right\}
$$

forms a basis of left-invariant vector fields on $G$.

(c) Find the structure constants of the Lie algebra $\mathfrak{g}$ of $G$ with respect to the basis in (b).

\section*{QUALIFYING EXAMINATION }
Department of Mathematics

Thursday September 6, 2018 (Day 3)

\begin{enumerate}
  \item (AT) Let $p: E \rightarrow B$ be a $k$-fold covering space, and suppose that the Euler characteristic $\chi(E)$ is defined, nonzero, and relatively prime to $k$. Show that any $\mathrm{CW}$ decomposition of $B$ has infinitely many cells.

  \item (RA) Let $W$ be Gumbel distributed, that is $P(W \leq x)=e^{-e^{-x}}$. Let $X_{i}$ be independent and identically distributed Exponential random variables with mean 1; that is, $X_{i}$ are independent, with $P\left(X_{i} \leq x\right)=\exp (-\max x, 0)$.

\end{enumerate}

Let

$$
M_{n}=\max _{i \leq n} X_{i}
$$

Show that there are deterministic sequences $a_{n}, b_{n}$ such that

$$
\frac{M_{n}-b_{n}}{a_{n}} \rightarrow W
$$

in law; that is, such that for any continuous bounded function $F$,

$$
\mathbb{E} F\left(\frac{M_{n}-b_{n}}{a_{n}}\right) \rightarrow \mathbb{E} F(W)
$$

\begin{enumerate}
  \setcounter{enumi}{2}
  \item (DG) Consider $\mathbb{R}^{2}$ as a Riemannian manifold equipped with the metric
\end{enumerate}

$$
g=e^{x} \mathrm{~d} x^{2}+\mathrm{d} y^{2}
$$

(i) Compute the Christoffel symbols of the Levi-Civita connection for $g$.

(ii) Show that the geodesics are described by the curves $x(t)=2 \log (A t+B)$ and $y(t)=C t+D$, for appropriate constants $A, B, C, D$.

(iii) Let $\gamma: \mathbb{R}_{+} \rightarrow \mathbb{R}^{2}, \gamma(t)=(t, t)$. Compute the parallel transport of the vector $(1,2)$ along the curve $\gamma$.

(iv) Are there two vector fields $X, Y$ parallel to the curve $\gamma$, such that $g(X(t), Y(t))$ is non-constant?

\begin{enumerate}
  \setcounter{enumi}{3}
  \item (A) Let $G$ be a group of order 78 .
\end{enumerate}

(a) Show that $G$ contains a normal subgroup of index 6 .

(b) Show by example that $G$ may contain a subgroup of index 13 that is not normal.

\begin{enumerate}
  \setcounter{enumi}{4}
  \item (AG) Let $K$ be an algebraically closed field of characteristic 0 , and consider the curve $C \subset \mathbb{A}^{3}$ over $K$ given as the image of the map
\end{enumerate}

$$
\begin{aligned}
\phi: \mathbb{A}^{1} & \rightarrow \mathbb{A}^{3} \\
& t \mapsto\left(t^{3}, t^{4}, t^{5}\right)
\end{aligned}
$$

Show that no neighborhood of the point $\phi(0)=(0,0,0) \in C$ can be embedded in $\mathbb{A}^{2}$.

\begin{enumerate}
  \setcounter{enumi}{5}
  \item (CA) Let $f(z)$ be an entire function such that
\end{enumerate}

a) $f(z)$ vanishes at all points $z=n, n \in \mathbb{Z}$;

b) $|f(z)| \leq e^{\pi|\operatorname{Im} z|}$ for all $z \in \mathbb{C}$.

Prove that $f(z)=c \sin \pi z$, with $c \in \mathbb{C},|c| \leq 1$.


\end{document}