\documentclass[10pt]{article}
\usepackage[utf8]{inputenc}
\usepackage[T1]{fontenc}
\usepackage{amsmath}
\usepackage{amsfonts}
\usepackage{amssymb}
\usepackage[version=4]{mhchem}
\usepackage{stmaryrd}
\usepackage{bbold}

\title{QUALIFYING EXAMINATION }


\author{HARVARD UNIVERSITY}
\date{}


\begin{document}
\maketitle
Department of Mathematics

Tuesday August 31, 2021 (Day 1)

\begin{enumerate}
  \item (A) Let $G$ be a finite group, let $V$ be a representation of $G$ on a finitedimensional vector space over $\mathbb{C}$, and let $W \subset V$ be a subrepresentation. Show that there is a subrepresentation $W^{\prime} \subset V$ such that
\end{enumerate}

$$
V=W \oplus W^{\prime}
$$

\begin{enumerate}
  \setcounter{enumi}{1}
  \item (AG) Consider the varieties in the affine plane $\mathbb{A}_{\mathbb{C}}^{2}$ with coordinates $(x, y)$ defined by the following polynomials:

  \item $X_{1}=V\left(x^{2}-1\right)$

  \item $X_{2}=V\left(x^{2}-y\right)$

  \item $X_{3}=V\left(x^{2}-y^{2}\right)$

  \item $X_{4}=V\left(x^{2}-y^{3}\right)$

  \item $X_{5}=V\left(x^{2}-y^{4}\right)$.

\end{enumerate}

Prove that no two of the varieties $X_{i}$ are isomorphic. (Note: we are not adopting the convention that varieties are assumed irreducible.)

\begin{enumerate}
  \setcounter{enumi}{2}
  \item (AT) Let $D^{n}$ be a closed disc in $\mathbb{R}^{n}$ and $S^{n-1}=\partial D^{n}$ its boundary. For any topological space $X$ and map $\alpha: S^{n-1} \rightarrow X$, we define the space $Y$ obtained from $X$ by attaching an $n$-cell via the map $\alpha$ to be the quotient of the disjoint union $D^{n} \sqcup X$ by the equivalence relation generated by $p \sim \alpha(p)$ for all $p \in \partial D^{n}$. Assuming that the Betti numbers of $X$ are finite, show that one of the two following statements holds:

  \item the $n$th Betti number of $Y$ is 1 greater than the $n$th Betti number of $X$, and all other Betti numbers are equal; or

  \item the $(n-1)$ st Betti number of $Y$ is 1 less than the $(n-1)$ st Betti number of $X$, and all other Betti numbers are equal.

  \item (CA) Evaluate the series

\end{enumerate}

$$
\sum_{n=-\infty}^{\infty} \frac{n^{2}+n+1}{n^{4}+1}
$$

by integrating $R(z) \cot \pi z$ for some appropriate rational function $R(z)$ over the boundary of the square $C_{n} \subset \mathbb{C}$ whose four vertices are $\left(n+\frac{1}{2}\right)( \pm 1 \pm i)$ and then letting $n \rightarrow \infty$.

\begin{enumerate}
  \setcounter{enumi}{4}
  \item (DG) Let $c>0$. Consider the catenary $C$ defined by
\end{enumerate}

$$
x=c \cosh \left(\frac{z}{c}\right)
$$

in the $x z$-plane. Let $S$ be the catenoid in the $x y z$-space obtained by rotating the catenary $C$ with respect to the $z$-axis. Use $\theta, z$ as coordinates for $S$, where $\theta$ is from the polar coordinates $(r, \theta)$ of the $x y$-plane. In terms of $(\theta, z)$, write down the first and second fundamental forms of $S$ and the mean curvature and Gaussian curvature of $S$.

\begin{enumerate}
  \setcounter{enumi}{5}
  \item (RA) Suppose $f:[-1,1] \rightarrow \mathbf{R}$ is a continuous function such that
\end{enumerate}

$$
\int_{-1}^{1} x^{2 n} f(x) d x=0
$$

for each $n=0,1,2,3, \ldots$ Prove that $f$ is an odd function (i.e., that $f(-x)=$ $-f(x)$ for all $x \in[-1,1])$.

\section*{QUALIFYING EXAMINATION }
Department of Mathematics

Wednesday September 1, 2021 (Day 2)

\begin{enumerate}
  \item $(\mathrm{AT})$
\end{enumerate}

(a) Let $X$ and $Y$ be compact, connected, oriented $n$-manifolds, and $f: X \rightarrow$ $Y$ a continuous map. Define the degree of the map $f$.

(b) Let $S^{n}$ be the unit sphere in $\mathbb{R}^{n+1}$, and let $r_{i}: S^{n} \rightarrow S^{n}$ be the reflection in the $i$ th axis; that is, the map

$$
\left(x_{0}, \ldots, x_{n}\right) \mapsto\left(x_{0}, \ldots, x_{i-1},-x_{i}, x_{i+1}, \ldots, x_{n}\right)
$$

What is the degree of $r_{i}$ ?

(c) Let $S^{n}$ be the unit sphere in $\mathbb{R}^{n+1}$, and let $a: S^{n} \rightarrow S^{n}$ be the antipodal map sending $x$ to $-x$. What is the degree of $a$ ?

\begin{enumerate}
  \setcounter{enumi}{1}
  \item (CA) Suppose that $f:\{z: 0<|z|<1\} \rightarrow \mathbb{C}$ is holomorphic and $|f(z)| \leq$ $A|z|^{-3 / 2}$ for some constant $A$. Prove that there is a complex constant $\alpha$ such that $g(z):=f(z)-\alpha z^{-1}$ can be extended to a holomorphic function on $\{z:|z|<1\}$.

  \item (DG) Which of the following smooth manifolds:

  \item $S^{2}$,

  \item $\mathbb{R P}^{2}$ and

  \item $S^{1} \times S^{1}$

\end{enumerate}

admit a closed, non-exact differential 1-form? In each case, either argue why such form does not exist or give an example.

\begin{enumerate}
  \setcounter{enumi}{3}
  \item (RA) Let $\mathbf{T}$ be the torus $(\mathbf{R} / \mathbf{Z})^{2}$, and let $a: \mathbf{T} \rightarrow \mathbf{R}$ be any continuous function. Prove that the $\mathbf{R}$-vector space of solutions of the partial differential equation
\end{enumerate}

$$
\frac{\partial^{2} f}{\partial x^{2}}+\frac{\partial^{2} f}{\partial y^{2}}=a f
$$

in functions $f: \mathbf{T} \rightarrow \mathbf{R}$ is finite dimensional.

\begin{enumerate}
  \setcounter{enumi}{4}
  \item (A) Consider the polynomial $f(x)=x^{4}+1$.
\end{enumerate}

(a) Prove that the Galois group $G$ of $f$ over $\mathbb{Q}$ has order 4 .

(b) Show that $G$ is in fact isomorphic to the group $\mathbb{Z} / 2 \mathbb{Z} \times \mathbb{Z} / 2 \mathbb{Z}$.

(c) Is there any prime $p>2$ such that $f$ is irreducible over the finite field of order $p$ ?

\begin{enumerate}
  \setcounter{enumi}{5}
  \item (AG) Let $C \subset \mathbb{P}^{3}$ be a smooth, irreducible, nondegenerate curve of degree 4 .
\end{enumerate}

(a) If the genus of $C$ is 0 , show that $C$ is contained in a quadric surface.

(b) If the genus of $C$ is 1 , show that $C$ is equal to the intersection of two quadric surfaces.

(c) Show that the genus of $C$ cannot be greater than 1 .

\section*{QUALIFYING EXAMINATION }
Department of Mathematics

Thursday September 2, 2021 (Day 3)

\begin{enumerate}
  \item (DG) Let $a_{i j}$ for $1 \leq i \leq n-1$ and $1 \leq j \leq n$ be real constants. For $1 \leq i \leq n-1$ consider the vector field
\end{enumerate}

$$
X_{i}=(\underbrace{0, \cdots, 0,1,0 \cdots, 0}_{1 \text { in } i^{\text {th }} \text { position }}, \sum_{j=1}^{n} a_{i j} x_{j})
$$

on $\mathbb{R}^{n}$ (with coordinates $x_{1}, \cdots, x_{n}$ ). Let $\Pi$ be the distribution of the tangent subspace of dimension $n-1$ in $\mathbb{R}^{n}$ spanned by $X_{1}, \cdots, X_{n-1}$. Determine the necessary and sufficient condition for $\Pi$ to be integrable. Express the condition in terms of symmetry properties of the $(n-1) \times(n-1)$ matrix $\left(a_{i j}\right)_{1 \leq i, j \leq n-1}$ and the relation among the ratios $\frac{a_{i k}}{a_{j k}}$ for $1 \leq i<j \leq n-1$ and $1 \leq \bar{k} \leq n$.

\begin{enumerate}
  \setcounter{enumi}{1}
  \item (RA) Suppose $U$ and $V$ are two random variables. We say that $U$ and $V$ are uncorrelated if $\operatorname{Cov}(U, V)=\mathbb{E}[U V]-\mathbb{E}[U] \mathbb{E}[V]=0$.
\end{enumerate}

(a) Is it true that if $U$ and $V$ are uncorrelated, then $U$ and $V$ are independent? Prove it or give a counter example.

(b) Suppose $\mathrm{X}$ and $\mathrm{Y}$ are distributed by the following bivariate normal distribution with density

$$
f(x, y)=\frac{1}{2 \pi} \frac{1}{\sqrt{1-\rho^{2}}} e^{-\frac{x^{2}-2 \rho x y+y^{2}}{2\left(1-\rho^{2}\right)}}
$$

where $0<\rho<1$ is a parameter. Let $U=X+a Y$ and $V=X+b Y$ with $a, b \neq 0$. Find the condition that $\operatorname{Cov}(U, V)=0$. In this case, prove that $U$ and $V$ are independent (you cannot just cite a theorem).

\begin{enumerate}
  \setcounter{enumi}{2}
  \item (A) Suppose $R$ is a commutative ring with unit, $I$ an ideal in $R$, and $M$ a finitely-generated $R$-module. If $I M=M$, prove that there exists $r \in R$ such that $r-1 \in I$ and $r M=0$.

  \item (AG) Let $\mathbb{P}^{n^{2}-1}$ be the variety of nonzero $n \times n$ complex matrices modulo scalars. Consider the set

\end{enumerate}

$$
X:=\left\{[A] \in \mathbb{P}^{n^{2}-1} \mid A \text { is nilpotent }\right\}
$$

(a) Show that $X$ is a closed subvariety of $\mathbb{P}^{n^{2}-1}$.

(b) Show that $X$ is irreducible, and find its dimension.

\begin{enumerate}
  \setcounter{enumi}{4}
  \item (AT) Let $M$ be a connected closed 4-manifold such that $\pi_{1}(M)$ is perfect; that is, does not have any non-trivial abelian quotients. Determine the possible cohomology groups $H^{*}(M, \mathbb{Z})$.

  \item (CA) Let $a<b$ and $f(z)$ be a continuous function on the closed strip $\{a \leq$ $x \leq b\}$ which is holomorphic on its interior $\{a<x<b\}$, where $z=x+i y$, such that $|f(z)|=O\left(e^{\varepsilon|y|}\right)$ on $\{a \leq x \leq b\}$ for every $\varepsilon>0$ as $|y| \rightarrow \infty$. If $|f(z)| \leq M$ on the boundary $\{x=a$ or $x=b\}$ of the strip $\{a \leq x \leq b\}$ and on the interval $[a, b]$ for some positive number $M$, prove that $|f(z)| \leq M$ on the entire closed strip $\{a \leq x \leq b\}$.

\end{enumerate}

Hint: Consider

$$
g_{\varepsilon}(z)=e^{\varepsilon i z} f(z) \text { and } h_{\varepsilon}(z)=e^{-\varepsilon i z} f(z) \text {. }
$$


\end{document}