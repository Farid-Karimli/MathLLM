\documentclass[10pt]{article}
\usepackage[utf8]{inputenc}
\usepackage[T1]{fontenc}
\usepackage{amsmath}
\usepackage{amsfonts}
\usepackage{amssymb}
\usepackage[version=4]{mhchem}
\usepackage{stmaryrd}
\usepackage{bbold}

\title{QUALIFYING EXAMINATION }


\author{HARVARD UNIVERSITY}
\date{}


\begin{document}
\maketitle
Department of Mathematics

Tuesday September 3, 2019 (Day 1)

\begin{enumerate}
  \item (AT) Suppose that $M$ is a compact connected manifold of dimension 3 , and that the abelianization $\left(\pi_{1} M\right)_{\mathrm{ab}}$ is trivial. Determine the homology and cohomology groups of $M$ (with integer coefficients).

  \item (A) Prove that for every finite group $G$ the number of groups homomorphisms $h: \mathbf{Z}^{2} \rightarrow G$ is $n|G|$ where $n$ is the number of conjugacy classes of $G$.

  \item (AG) Let $X \subset \mathbb{P}^{n}$ be a projective variety over a field $K$, with ideal $I(X) \subset$ $K\left[Z_{0}, \ldots, Z_{n}\right]$ and homogeneous coordinate ring $S(X)=K\left[Z_{0}, \ldots, Z_{n}\right] / I(X)$. The Hilbert function $h_{X}(m)$ is defined to be the dimension of the $m$ th graded piece of $S(X)$ as a vector space over $K$.

\end{enumerate}

a. Define the Hilbert polynomial $p_{X}(m)$ of $X$.

b. Prove that the degree of $p_{X}$ is equal to the dimension of $X$.

c. For each $m$, give an example of a variety $X \subset \mathbb{P}^{n}$ such that $h_{X}(m) \neq$ $p_{X}(m)$.

\begin{enumerate}
  \setcounter{enumi}{3}
  \item (CA) Use contour integration to prove that for real numbers $a$ and $b$ with $a>b>0$,
\end{enumerate}

$$
\int_{0}^{\pi} \frac{d \theta}{a-b \cos \theta}=\frac{\pi}{\sqrt{a^{2}-b^{2}}}
$$

\begin{enumerate}
  \setcounter{enumi}{4}
  \item (RA) Dirichlet's function $D$ is the function on $[0,1] \subset \mathbb{R}$ that equals 1 at every rational number and equals 0 at every irrational number. Thomae's function $T$ is the function on $[0,1]$ whose value at irrational numbers is 0 and whose value at any given rational number $r$ is $1 / q$, where $r=p / q$ with $\mathrm{p}$ and q relatively prime integers, $q>0$.

  \item Prove that $D$ is nowhere continuous.

  \item Show that $T$ is continuous at the irrational numbers and discontinuous at the rational numbers.

  \item Show that $T$ is nowhere differentiable.

  \item (DG)

\end{enumerate}

Consider the Riemannian manifold $(\mathbb{D}, g)$ with $\mathbb{D}$ the unit disk in $\mathbb{R}^{2}$ and

$$
g=\frac{1}{1-x^{2}-y^{2}}\left(d x^{2}+d y^{2}\right)
$$

Find the Riemann curvature tensor of $(\mathbb{D}, g)$. Use this to read off the Gaussian curvature of $(\mathbb{D}, g)$.

\section*{QUALIFYING EXAMINATION }
Department of Mathematics

Wednesday September 4, 2019 (Day 2)

\begin{enumerate}
  \item (CA) Fix $a \in \mathbb{C}$ and an integer $n \geq 2$. Show that the equation $a z^{n}+z+1=0$ has a solution with $|z| \leq 2$.

  \item (AG) Let $\mathbb{P}^{N}$ be the space of nonzero homogeneous polynomials of degree $d$ in $n+1$ variables over a field $K$, modulo multiplication by nonzero scalars, and let $U \subset \mathbb{P}^{N}$ be the subset of irreducible polynomials $F$ such that the zero locus $V(F) \subset \mathbb{P}^{n}$ is smooth.

\end{enumerate}

(a) Show that $U$ is a Zariski open subset of $\mathbb{P}^{N}$.

(b) What is the dimension of the complement $D=\mathbb{P}^{N} \backslash U$ ?

(c) Show that $D$ is irreducible.

\begin{enumerate}
  \setcounter{enumi}{2}
  \item (RA) Let $B$ denote the Banach space of continuous, real valued functions on $[0,1] \subset \mathbb{R}$ with the sup norm.

  \item State the Arzela-Ascoli theorem in the context of $B$.

  \item Define what is meant by a compact operator between two Banach spaces.

  \item Prove that the operator $T: \mathcal{B} \rightarrow \mathcal{B}$ defined by

\end{enumerate}

$$
(T f)(x)=\int_{0}^{x} f(y) d y
$$

is compact.

\begin{enumerate}
  \setcounter{enumi}{3}
  \item (A) Let $\mathbb{F}_{q}$ be the finite field with $q$ elements. Show that the number of $3 \times 3$ nilpotent matrices over $\mathbb{F}_{q}$ is $q^{6}$.

  \item (AT) Let $\operatorname{Sym}^{n} X$ denote the $n$th symmetric power of a CW complex $X$, i.e. $X^{n} / S_{n}$, where the symmetric group $S_{n}$ acts by permuting coordinates. Show that for all $n \geq 2$, the fundamental group of $\operatorname{Sym}^{n} X$ is abelian.

  \item (DG) Let $S^{2} \subset \mathbb{R}^{3}$ be the unit 2 -sphere, with its usual orientation. Let $X$ be the vector field generating the flow given by

\end{enumerate}

$$
\left(\begin{array}{ccc}
\cos (t) & -\sin (t) & 0 \\
\sin (t) & \cos (t) & 0 \\
0 & 0 & 1
\end{array}\right) \cdot\left[\begin{array}{l}
x \\
y \\
z
\end{array}\right]
$$

and let $\omega$ be the volume form induced by the embedding in $\mathbb{R}^{3}$ (so the total "volume" is $4 \pi$ ). Find a function $f: S^{2} \rightarrow \mathbb{R}$ satisfying

$$
d f=\iota_{X} \omega
$$

where $\iota_{X} \omega$ is the contraction of $\omega$ by $X$.

\section*{QUALIFYING EXAMINATION }
Department of Mathematics

Thursday September 5, 2019 (Day 3)

\begin{enumerate}
  \item (RA) Let $f:[0,1] \rightarrow \mathbb{R}$ be in the Sobolev space $H^{1}([0,1])$; that is, functions $f$ such that both $f$ and its derivative are $L^{2}$-integrable. Prove that
\end{enumerate}

$$
\lim _{n \rightarrow \infty}\left(n \int_{0}^{1} f(x) e^{-2 \pi i n x} d x\right)=0
$$

\begin{enumerate}
  \setcounter{enumi}{1}
  \item (CA) Given that the sum
\end{enumerate}

$$
\sum_{n \in \mathbb{Z}} \frac{1}{(z-n)^{2}}
$$

converges uniformly on compact subsets of $\mathbb{C} \backslash \mathbb{Z}$ to a meromorphic function on the entire complex plane, prove the identity

$$
\frac{\pi^{2}}{\sin ^{2} \pi z}=\sum_{n \in \mathbb{Z}} \frac{1}{(z-n)^{2}}
$$

\begin{enumerate}
  \setcounter{enumi}{2}
  \item (AG) Let $C \subset \mathbb{P}^{3}$ be a smooth curve of degree 5 and genus 2 .
\end{enumerate}

(a) By considering the restriction map $\rho: H^{0}\left(\mathcal{O}_{\mathbb{P}^{3}}(2)\right) \rightarrow H^{0}\left(\mathcal{O}_{C}(2)\right)$, show that $C$ must lie on a quadric surface $Q$.

(b) Show that the quadric surface $Q$ is unique.

(c) Similarly, show that $C$ must lie on at least one cubic surface $S$ not containing $Q$.

(d) Finally, deduce that there exists a line $L \subset \mathbb{P}^{3}$ such that the union $C \cup L$ is a complete intersection of a quadric and a cubic.

\begin{enumerate}
  \setcounter{enumi}{3}
  \item (A) Show that if $p, q$ are distinct primes then the polynomial $\left(x^{p}-1\right) /(x-1)$ is irreducible $\bmod q$ if an only if $q$ is a primitive residue of $p$ (i.e. if every integer that is not a multiple of $p$ is congruent to $q^{e} \bmod p$ for some integer $e$ ).
\end{enumerate}

ii) Prove that $x^{6}+x^{5}+x^{4}+x^{3}+x^{2}+x+1$ factors mod 23 as the product of two irreducible cubics.

\begin{enumerate}
  \setcounter{enumi}{4}
  \item (DG) Suppose that $G$ is a Lie group.
(a) Consider the map $\iota: G \rightarrow G$ defined by $\iota(g)=g^{-1}$. Show that the derivative of $\iota$ at the identity element is multiplication by -1 .
\end{enumerate}

(b) For $g \in G$ define maps $L_{g}, R_{g}: G \rightarrow G$ by

$$
\begin{aligned}
& L_{g}(x)=g x \\
& R_{g}(x)=x g
\end{aligned}
$$

Show that if $\omega$ is a $k$-form which is bi-invariant in the sense that $L_{g}^{*} \omega=$ $R_{g}^{*} \omega$ then $\iota^{*} \omega=(-1)^{k} \omega$.

(c) Show that bi-invariant forms on $G$ are closed.

\begin{enumerate}
  \setcounter{enumi}{5}
  \item (AT) Suppose that $m$ is odd. Show that if $n$ is odd there is a fixed point free action of $\mathbb{Z} / m$ on $S^{n}$. What happens if $n$ is even?
\end{enumerate}

\end{document}