\documentclass[10pt]{article}
\usepackage[utf8]{inputenc}
\usepackage[T1]{fontenc}
\usepackage{amsmath}
\usepackage{amsfonts}
\usepackage{amssymb}
\usepackage[version=4]{mhchem}
\usepackage{stmaryrd}
\usepackage{bbold}

\title{QUALIFYING EXAMINATION }


\author{HARVARD UNIVERSITY}
\date{}


\begin{document}
\maketitle
Department of Mathematics

Tuesday September 1, 2015 (Day 1)

\begin{enumerate}
  \item (A) The integer 8871870642308873326043363 is the $13^{\text {th }}$ power of an integer $n$. Find $n$.

  \item (AG) Let $C \subset \mathbb{P}^{2}$ be a smooth plane curve of degree 4 .

\end{enumerate}

(a) Describe the canonical bundle of $C$ in terms of line bundles on $\mathbb{P}^{2}$. What are the effective canonical divisors on $C$ ?

(b) What is the genus of $C$ ? Explain how you obtain this formula.

(c) Prove that $C$ is not hyperelliptic.

\begin{enumerate}
  \setcounter{enumi}{2}
  \item (DG) Let $M$ be a $C^{\infty}$ manifold, $T M$ its tangent bundle, and $T^{\mathbb{C}} M=\mathbb{C} \otimes_{\mathbb{R}} T M$ the complexified tangent bundle. An almost complex structure on $M$ is a $C^{\infty}$ bundle map $J: T M \rightarrow T M$ such that $J^{2}=-1$.
\end{enumerate}

(a) Show that an almost complex structure naturally determines, and is determined by, each of the following two structures:

i) the structure of a complex $C^{\infty}$ vector bundle - i.e., with fibres that are complex vector spaces - on $T M$ compatible with its real structure. ii) a $C^{\infty}$ direct sum decomposition $T^{\mathbb{C}} M=T^{1,0} M \oplus T^{0,1} M$ with $T^{0,1} M$ = complex conjugate of $T^{1,0} \mathrm{M}$.

(b) Show that every almost complex manifold is orientable.

(c) If $S$ is a $C^{\infty}$, orientable, 2-dimensional, Riemannian manifold, construct a natural almost complex structure on $S$ in terms of its Riemannian structure, but one that depends only on the underlying conformal structure of $S$.

(d) Does the almost complex structure constructed in (c) determine the conformal structure of $S$ ? You need NOT give a detailed answer to this question; a heuristic one- or two-sentence answer suffices.

\begin{enumerate}
  \setcounter{enumi}{3}
  \item (RA) In this problem $V$ denotes a Banach space over $\mathbb{R}$ or $\mathbb{C}$.
\end{enumerate}

(a) Show that any finite dimensional subspace $U_{0} \subset V$ is closed in $V$.

(b) Now let $U_{1} \subset V$ a closed subspace, and $U_{2} \subset V$ a finite dimensional subspace. Show that $U_{1}+U_{2}$ is closed in $V$.

\begin{enumerate}
  \setcounter{enumi}{4}
  \item (AT) Consider the following three topological spaces:
\end{enumerate}

$$
A=\mathbb{H} \mathrm{P}^{3}, \quad B=S^{4} \times S^{8}, \quad C=S^{4} \vee S^{8} \vee S^{12}
$$

$\left(\mathbb{H} \mathrm{P}^{3}\right.$ denotes quaternionic projective 3 -space.)
(a) Calculate the cohomology groups (with integer coefficients) of all three.

(b) Show that $A$ and $B$ are not homotopy equivalent.

(c) Show that $C$ is not homotopy equivalent to any compact manifold.

\begin{enumerate}
  \setcounter{enumi}{5}
  \item (CA) Let $f(z)$ be a function which is analytic in the unit disc $D=\{|z|<1\}$, and assume that $|f(z)| \leq 1$ in $D$. Also assume that $f(z)$ has at least two fixed points $z_{1}$ and $z_{2}$. Prove that $f(z)=z$ for all $z \in D$.
\end{enumerate}

\section*{QUALIFYING EXAMINATION }
Wednesday September 2, 2015 (Day 2)

\begin{enumerate}
  \item (AT) Let $\mathbb{C P}^{n}=\left(\mathbb{C}^{n+1} \backslash\{0\}\right) / \mathbb{C}^{*}$ be $n$ dimensional complex projective space.
\end{enumerate}

(a) Show that every map $f: \mathbb{C P}^{2 n} \rightarrow \mathbb{C P}^{2 n}$ has a fixed point. (Hint: Use the ring structure on cohomology.)

(b) For every $n \geq 0$, give an example of a map $f: \mathbb{C P}^{2 n+1} \rightarrow \mathbb{C P}^{2 n+1}$ without any fixed points and describe its induced map on cohomology.

\begin{enumerate}
  \setcounter{enumi}{1}
  \item (A) Let $A$ be a commutative ring with unit. Define what it means for $A$ to be Noetherian. Prove that the ring of continuous functions $f:[0,1] \rightarrow \mathbb{R}$ (with pointwise addition and multiplication) is not Noetherian.

  \item (CA) Let $S \subset \mathbb{C}$ be the open half-disc $\left\{x+i y: y>0, x^{2}+y^{2}<1\right\}$.

\end{enumerate}

(a) Construct a surjective conformal mapping $f: S \rightarrow D$, where $D$ is the open unit disc $\{z \in \mathbb{C}:|z|<1\}$.

(b) Construct a harmonic function $h: S \rightarrow \mathbb{R}$ such that:

\begin{itemize}
  \item $h(x+i y) \rightarrow 0$ as $y \rightarrow 0$ from above, for all real $x$ with $|x|<1$, and

  \item $h\left(r e^{i \theta}\right) \rightarrow 1$ as $r \rightarrow 1$ from below, for all real $\theta$ with $0<|x|<\pi$.

\end{itemize}

\begin{enumerate}
  \setcounter{enumi}{3}
  \item (AG) Let $Q$ be the complex quadric surface in $\mathbb{P}^{3}$ defined by the homogeneous equation $x_{0} x_{3}-x_{1} x_{2}=0$.
\end{enumerate}

(a) Show that $Q$ is non-singular.

(b) Show that through each point of $Q$ there are exactly two lines which lie on $Q$.

(c) Show that $Q$ is rational, but not isomorphic to $\mathbb{P}^{2}$.

\begin{enumerate}
  \setcounter{enumi}{4}
  \item (DG) Let $\Omega$ be the 2 -form on $\mathbb{R}^{3}-\{0\}$ defined by
\end{enumerate}

$$
\Omega=\frac{1}{x^{2}+y^{2}+z^{2}}(x d y \wedge d z+y d z \wedge d x+z d x \wedge d y)
$$

(a) Prove that $\Omega$ is closed.

(b) Let $f: \mathbb{R}^{3}-\{0\} \rightarrow S^{2}$ be the map which sends $(x, y, z)$ to $\left(\frac{1}{x^{2}+y^{2}+z^{2}}\right)^{1 / 2}(x, y, z)$. Show that $\Omega$ is the pull-back via $f$ of a 2 -form on $S^{2}$.

(c) Prove that $\Omega$ is not exact.

\begin{enumerate}
  \setcounter{enumi}{5}
  \item (RA) Consider the linear ODE $f^{\prime \prime}+P f^{\prime}+Q f=0$ on the interval $(a, b) \subset \mathbb{R}$, with $P, Q$ denoting $C^{\infty}$ real valued functions on $(a, b)$. Recall the definition of the Wronskian $W\left(f_{1}, f_{2}\right)=f_{1} f_{2}^{\prime}-f_{1}^{\prime} f_{2}$ associated to any two solutions $f_{1}, f_{2}$ of this differential equation.
\end{enumerate}

(a) Show that $W\left(f_{1}, f_{2}\right)$ either vanishes identically or is everywhere nonzero, depending on whether the two solutions $f_{1}, f_{2}$ are linearly dependent or not.

(b) Now suppose that $f_{1}, f_{2}$ are linearly independent, real valued solutions. Show that they have at most first order zeroes, and that the zeroes occur in an alternating fashion: between any two zeroes of one of the solutions there must be a zero of the other solution.

\section*{QUALIFYING EXAMINATION }
Department of Mathematics

Thursday September 3, 2015 (Day 3)

\begin{enumerate}
  \item (DG) Consider the graph $S$ of the function $F(x, y)=\cosh (x) \cos (y)$ in $\mathbb{R}^{3}$ and let
\end{enumerate}

$$
\Phi: \mathbb{R}^{2} \rightarrow S \subset R^{3}
$$

be its parametrization: $\Phi(x, y)=(x, y, \cosh (x) \cos (y))$.

(a) Write down the metric on $\mathbb{R}^{2}$ that is defined by the rule that the inner product of two vectors $v$ and $w$ at the point $(x, y)$ is equal to the inner product of $\Phi_{*}(v)$ and $\Phi_{*}(w)$ at the point $\Phi(x, y)$ in $\mathbb{R}^{3}$.

(b) Define the Gaussian curvature of a general surface embedded in $\mathbb{R}^{3}$.

(c) Compute the Gaussian curvature of the surface $S$ at the point $(0,0,1)$.

\begin{enumerate}
  \setcounter{enumi}{1}
  \item (RA) Let $f(x) \in C(\mathbb{R} / \mathbb{Z})$ be a continuous $\mathbb{C}$-valued function on $\mathbb{R} / \mathbb{Z}$ and let $\sum_{n=-\infty}^{\infty} a_{n} e^{2 \pi i n x}$ be its Fourier series.
\end{enumerate}

(a) Show that $f$ is $C^{\infty}$ if and only if $\left|a_{n}\right|=O\left(|n|^{-k}\right)$ for all $k \in \mathbb{N}$.

(b) Prove that a sequence of functions $\left\{f_{n}\right\}_{n \geq 1}$ in $C^{\infty}(\mathbb{R} / \mathbb{Z})$ converges in the $C^{\infty}$ topology (uniform convergence of functions and their derivatives of all orders) if and only if the sequences of $k$-th derivatives $\left\{f_{n}^{(k)}\right\}_{n \geq 1}$, for all $k \geq 0$, converge in the $L^{2}$-norm on $\mathbb{R} / \mathbb{Z}$.

\begin{enumerate}
  \setcounter{enumi}{2}
  \item (AG) Let $C$ be a smooth projective curve over $\mathbb{C}$ and $\omega_{C}^{\otimes 2}$ the square of its canonical sheaf.
\end{enumerate}

(a) What is the dimension of the space of sections $\Gamma\left(C, \omega_{C}^{\otimes 2}\right)$ ?

(b) Suppose $g(C) \geq 2$ and $s \in \Gamma\left(C, \omega_{C}^{\otimes 2}\right)$ is a section with simple zeros. Compute the genus of $\Sigma=\left\{x^{2}=s\right\}$ in the total space of the line bundle $\omega_{C}$, i.e. the curve defined by the 2 -valued 1 -form $\sqrt{s}$.

\begin{enumerate}
  \setcounter{enumi}{3}
  \item (AT) Show (using the theory of covering spaces) that every subgroup of a free group is free.

  \item $(\mathrm{CA})$

\end{enumerate}

(a) Define Euler's Gamma function $\Gamma(z)$ in the half plane $\operatorname{Re}(z)>0$ and show that it is holomorphic in this half plane.

(b) Show that $\Gamma(z)$ has a meromorphic continuation to the entire complex plane.
(c) Where are the poles of $\Gamma(z)$ ?

(d) Show that these poles are all simple and determine the residue at each pole.

\begin{enumerate}
  \setcounter{enumi}{5}
  \item (A) Let $G$ be a finite group, and $\rho: G \rightarrow G L_{n}(\mathbb{C})$ a linear representation. Then for each integer $i \geq 0$ there is a representation $\wedge^{i} \rho$ of $G$ on the exterior power $\wedge^{i}\left(\mathbb{C}^{n}\right)$. Let $W_{i}$ be the subspace $\left(\wedge^{i}\left(\mathbb{C}^{n}\right)\right)^{G}$ of $\wedge^{i}\left(\mathbb{C}^{n}\right)$ fixed under this action of $G$.
\end{enumerate}

Prove that $\operatorname{dim} W_{i}$ is the $T^{i}$ coefficient of the polynomial

$$
\frac{1}{|G|} \sum_{g \in G} \operatorname{det}\left(\mathbf{1}_{n}+T \rho(g)\right)
$$

where $\mathbf{1}_{n}$ is the $n \times n$ identity matrix.


\end{document}