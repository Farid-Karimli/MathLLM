\documentclass[10pt]{article}
\usepackage[utf8]{inputenc}
\usepackage[T1]{fontenc}
\usepackage{amsmath}
\usepackage{amsfonts}
\usepackage{amssymb}
\usepackage[version=4]{mhchem}
\usepackage{stmaryrd}
\usepackage{bbold}

\title{QUALIFYING EXAMINATION }


\author{HARVARD UNIVERSITY}
\date{}


\begin{document}
\maketitle
Tuesday September 3, 2019 (Day 1)

\begin{enumerate}
  \item (AT) Suppose that $M$ is a compact connected manifold of dimension 3, and that the abelianization $\left(\pi_{1} M\right)_{\mathrm{ab}}$ is trivial. Determine the homology and cohomology groups of $M$ (with integer coefficients).
\end{enumerate}

Solution: Since the obstruction to orienting $M$ is a homomorphism from $\pi_{1} M$ to the abelian group $\{ \pm 1\}$ we know that $M$ is orientable, and we may avail ourselves of Poincaré duality. By Poincaré's theorem $H_{1}(M)=\pi_{1}(M)_{\mathrm{ab}}=0$. By Poincaré duality this gives us $H^{3}(M)=\mathbb{Z}$ and $H^{2}(M)=0$. We now use the universal coefficient sequence

$$
0 \rightarrow \operatorname{Ext}\left(H_{i-1} M, \mathbb{Z}\right) \rightarrow H^{i}(M) \rightarrow \operatorname{Hom}\left(H_{i} M ; \mathbb{Z}\right) \rightarrow 0
$$

to conclude that $H^{0}(M)=\mathbb{Z}$ and $H^{1}(M)=0$, from which, say, by Poincaré duality you can conclude that $H_{3}(M)=\mathbb{Z}$ and $H_{2}(M)=0$. In short $M$ has the homology and cohomology of $S^{3}$.

\begin{enumerate}
  \setcounter{enumi}{1}
  \item (A) Prove that for every finite group $G$ the number of groups homomorphisms $h: \mathbf{Z}^{2} \rightarrow G$ is $n|G|$ where $n$ is the number of conjugacy classes of $G$.
\end{enumerate}

Solution. $\mathbf{Z}^{2}=\langle a, b \mid a b=b a\rangle$, so the homomorphisms $\mathbf{Z}^{2} \rightarrow G$ are in bijection with pairs $(g, h) \in G$ such that $g h=h g$.

Given $g_{0} \in G$, the group elements $h$ such that $g_{0} h=h g_{0}$ constitute the stabilizer of $g_{0}$ under the action of $G$ on itself by conjugation. The orbit of $g_{0}$ is its conjugacy class $\left[g_{0}\right]$, so the number of solutions $\left(g_{0}, h\right)$ is $|G| /\left|\left[g_{0}\right]\right|$ (orbitstabilizer theorem). Thus for each conjugacy class $c$ the number of solutions with $g \in c$ is $|c| \cdot|G| /|c|=|G|$. Summing over the $n$ conjugacy classes gives $n|G|$, Q.E.D.

\begin{enumerate}
  \setcounter{enumi}{2}
  \item (AG) Let $X \subset \mathbb{P}^{n}$ be a projective variety over a field $K$, with ideal $I(X) \subset$ $K\left[Z_{0}, \ldots, Z_{n}\right]$ and homogeneous coordinate ring $S(X)=K\left[Z_{0}, \ldots, Z_{n}\right] / I(X)$. The Hilbert function $h_{X}(m)$ is defined to be the dimension of the $m$ th graded piece of $S(X)$ as a vector space over $K$.
\end{enumerate}

a. Define the Hilbert polynomial $p_{X}(m)$ of $\mathrm{X}$.
b. Prove that the degree of $p_{X}$ is equal to the dimension of $X$.

c. For each $m$, give an example of a variety $X \subset \mathbb{P}^{n}$ such that $h_{X}(m) \neq$ $p_{X}(m)$.

\section{Solution:}
a. The Hilbert polynomial $p_{X}(m)$ is the unique polynomial such that $p_{X}(m)=$ $h_{X}(m)$ for all sufficiently large integers $m$.

b. By induction on the dimension of $X$ : if $X$ is a finite set of $d$ points, then the Hilbert polynomial is the constant $d$; and in general the Hilbert polynomial $p_{X \cap H}$ of a general hyperplane section of a variety $X$ is the first difference of the Hilbert polynomial $p_{X}$

c. Let $X$ consist of any $k$ distinct points of $\mathbb{P}^{n}$. Then $X$ is a variety of dimension 0 and degree $k$, so by the previous part $p_{X}(m)=k$. But $h_{X}(m)$ is at most the dimension of the space of homogeneous degree $m$ polynomials in $n+1$ variables, so for sufficiently large $k, h_{X}(m)<k=$ $p_{X}(m)$.

\begin{enumerate}
  \setcounter{enumi}{3}
  \item (CA) Use contour integration to prove that for real numbers $a$ and $b$ with $a>b>0$,
\end{enumerate}

$$
\int_{0}^{\pi} \frac{d \theta}{a-b \cos \theta}=\frac{\pi}{\sqrt{a^{2}-b^{2}}}
$$

Solution: Since the integrand is symmetric about $\theta=\pi$, we can extend the integration interval to $[0,2 \pi], I=\frac{1}{2} \int_{0}^{2 \pi} \frac{d \theta}{a-b \cos \theta}$, which can be written as an integral around an unit circle in the complex plane

$$
\begin{aligned}
I & =\frac{1}{2} \oint \frac{1}{a-\frac{b}{2}\left(z+\frac{1}{z}\right)} \frac{d z}{i z}=\oint \frac{1}{2 a z-b z^{2}-b} \frac{d z}{i} \\
& =-\frac{1}{b i} \oint \frac{1}{z^{2}-\frac{2 a}{b} z+1} d z .
\end{aligned}
$$

Then the singular points are the zeros of the denominator, $z^{2}-\frac{2 a}{b} z+1=0$. The roots $z_{1}, z_{2}$ are

$$
z_{1}=\frac{1}{b}\left(a-\sqrt{a^{2}-b^{2}}\right), \quad z_{2}=\frac{1}{b}\left(a+\sqrt{a^{2}-b^{2}}\right) .
$$

Note that only $z_{1}$ lies inside the unit circle. Thus

$$
\begin{aligned}
\oint \frac{d z}{\left(z-z_{1}\right)\left(z-z_{2}\right)} & =\left.2 \pi i \operatorname{Res}\right|_{z=z_{1}}\left[\frac{1}{\left(z-z_{1}\right)\left(z-z_{2}\right)}\right] \\
& =2 \pi i \lim _{z \rightarrow z_{1}}\left(z-z_{1}\right) \frac{1}{\left(z-z_{1}\right)\left(z-z_{2}\right)} \\
& =2 \pi i \frac{1}{\left(z_{1}-z_{2}\right)} \\
& =-2 \pi i \frac{b}{2 \sqrt{a^{2}-b^{2}}} .
\end{aligned}
$$

Therefore

$$
I=-\frac{1}{b i} \cdot\left(-2 \pi i \frac{b}{2 \sqrt{a^{2}-b^{2}}}\right)=\frac{\pi}{\sqrt{a^{2}-b^{2}}}
$$

\begin{enumerate}
  \setcounter{enumi}{4}
  \item (RA) Dirichlet's function $D$ is the function on $[0,1] \subset \mathbb{R}$ that equals 1 at every rational number and equals 0 at every irrational number. Thomae's function $T$ is the function on $[0,1]$ whose value at irrational numbers is 0 and whose value at any given rational number $r$ is $1 / q$, where $r=p / q$ with $\mathrm{p}$ and q relatively prime integers, $q>0$.

  \item Prove that $D$ is nowhere continuous.

  \item Show that $T$ is continuous at the irrational numbers and discontinuous at the rational numbers.

  \item Show that $T$ is nowhere differentiable.

\end{enumerate}

Solution. For the first part: for any rational number $\alpha$ we can find a sequence of irrational numbers $\alpha_{n}$ converging to $\alpha$; since $\lim D\left(\alpha_{n}\right) \neq D(\alpha), D$ cannot be continuous at $\alpha$. Similarly, if $\alpha$ is irrational, we can find a sequence of rational numbers $\alpha_{n}$ converging to $\alpha$ to conclude that $D$ cannot be continuous at $\alpha$.

For the second, exactly the same argument shows that $T$ is discontinuous at rational numbers. But now if $\alpha$ is irrational and $\alpha_{n}=p_{n} / q_{n}$ is a sequence of rational numbers converging to $\alpha$ then we must have $\lim q_{n}=\infty$, so $\lim T\left(\alpha_{n}\right)=T(\alpha)$ and $T$ is continuous at $\alpha$.

Finally, suppose that $\alpha$ is irrational. We can certainly find a sequence of irrational numbers $\beta_{n} \neq \alpha$ converging to $\alpha$, from which we see that if the limit $T^{\prime}(\alpha)$ exists it must be 0 . But for any $n$ we can also find a rational number $\alpha_{n}=p_{n} / 2^{n}$ with denominator $2^{n}$ such that $\left|\alpha-\alpha_{n}\right| \leq 1 / 2^{n}$, from which we see that if the limit $T^{\prime}(\alpha)$ exists it must be $\geq 1$.

\begin{enumerate}
  \setcounter{enumi}{5}
  \item (DG)
\end{enumerate}

Consider the Riemannian manifold $(\mathbb{D}, g)$ with $\mathbb{D}$ the unit disk in $\mathbb{R}^{2}$ and

$$
g=\frac{1}{1-x^{2}-y^{2}}\left(d x^{2}+d y^{2}\right)
$$

Find the Riemann curvature tensor of $(\mathbb{D}, g)$. Use this to read off the Gaussian curvature of $(\mathbb{D}, g)$.

Solution. This is a straightforward computation. We have

$$
g^{-1}=\left(\begin{array}{cc}
1-x^{2}-y^{2} & 0 \\
0 & 1-x^{2}-y^{2}
\end{array}\right)
$$

We set $x^{1}=x$ and $x^{2}=y$ and compute the Christoffel symbols

$$
\begin{aligned}
& \Gamma_{11}^{1}=\Gamma_{12}^{2}=\Gamma_{21}^{2}=-\Gamma_{22}^{1}=\frac{x}{1-x^{2}-y^{2}} \\
& -\Gamma_{11}^{2}=\Gamma_{12}^{1}=\Gamma_{21}^{1}=\Gamma_{22}^{2}=\frac{y}{1-x^{2}-y^{2}}
\end{aligned}
$$

With these, we can compute the only independent component of the Riemann curvature tensor:

$$
R\left(\frac{\partial}{\partial x}, \frac{\partial}{\partial y}, \frac{\partial}{\partial x}, \frac{\partial}{\partial y}\right)=g\left(R\left(\frac{\partial}{\partial x}, \frac{\partial}{\partial y}\right) \frac{\partial}{\partial y}, \frac{\partial}{\partial x}\right)=\frac{-2}{\left(1-x^{2}-y^{2}\right)^{2}}
$$

(Indeed, by antisymmetry, all other components of the Riemann curvature tensor either vanish or are equal to this up to a sign.)

The Gaussian curvature $K$ satisfies

$$
R_{a b c d}=K\left(g_{a c} g_{b d}-g_{a d} g_{b c}\right)
$$

which implies $K=-2$.

\section*{QUALIFYING EXAMINATION }
Department of Mathematics

Wednesday September 4, 2019 (Day 2)

\begin{enumerate}
  \item (CA) Fix $a \in \mathbb{C}$ and an integer $n \geq 2$. Show that the equation $a z^{n}+z+1=0$ has a solution with $|z| \leq 2$.
\end{enumerate}

Solution: There are two cases. First, assume that $|a|<2^{-n}$. Let $D$ denote the disk where $|z| \leq 2$ and let $\partial D$ denote the circle $|z|=2$. Let $f(z)=$ $a z^{n}+z+1$ and let $g(z)=z+1$. On $\partial D$, the function $g-f$ obeys the inequality $|g(z)-f(z)|=|a||z|^{n}<1$. Since this is less than $|g(z)|$ for each $z \in \partial D$, and since $g$ has no zeros on $\partial D$, none of the members of the 1parameter family of functions $\left\{f_{t}=f+t(g-f)\right\}_{t \in[0,1]}$ has a zero on $\partial D$. Therefore $f$ (which is $f_{t}=0$ ) and $g$ (which is $f_{t}=1$ ) have the same number of zeros (counting multiplicity) in $D$ and the number is 1 .

Now assume that $|a| \geq 2^{-n}$. By the fundamental theorem of algebra, the function $f(z)=a z^{n}+z+1$ factors as $f(z)=a \prod_{k=1}^{n}\left(z-\alpha_{k}\right)$, where $\alpha_{k} \in \mathbb{C}$. In particular,

$$
(-1)^{n} a \prod_{k=1}^{n} \alpha_{k}=1
$$

Thus, $\prod\left|\alpha_{k}\right| \leq 2^{n}$. This implies at least one of $\left|\alpha_{k}\right| \leq 2$.

\begin{enumerate}
  \setcounter{enumi}{1}
  \item (AG) Let $\mathbb{P}^{N}$ be the space of nonzero homogeneous polynomials of degree $d$ in $n+1$ variables over a field $K$, modulo multiplication by nonzero scalars, and let $U \subset \mathbb{P}^{N}$ be the subset of irreducible polynomials $F$ such that the zero locus $V(F) \subset \mathbb{P}^{n}$ is smooth.
\end{enumerate}

(a) Show that $U$ is a Zariski open subset of $\mathbb{P}^{N}$.

(b) What is the dimension of the complement $D=\mathbb{P}^{N} \backslash U$ ?

(c) Show that $D$ is irreducible.

Solution: In the product $\mathbb{P}^{N} \times \mathbb{P}^{n}$, consider the incidence correspondence

$$
\Phi=\left\{(F, p) \mid F(p)=\frac{\partial F}{\partial X_{i}}(p)=0 \forall i\right\}
$$

This is a closed subvariety of $\mathbb{P}^{N} \times \mathbb{P}^{n}$ and hence projective, so that its image $D=\pi_{1}(\Phi) \subset \mathbb{P}^{N}$ is closed; hence $U$ is open. Moreover, since the projection map $\pi_{2}: \Phi \rightarrow \mathbb{P}^{n}$ is a projective bundle with fiber $\mathbb{P}^{N-n-1}$, we see that $\Phi$ is
irreducible of dimension $N-1$. Finally, since there exist hypersurfaces with just one singular point (e.g., cones), the general fiber of $\Phi$ over a point in its image $D \subset \mathbb{P}^{N}$ is 0-dimensional; it follows that $D$ is again irreducible of dimension $N-1$.

\begin{enumerate}
  \setcounter{enumi}{2}
  \item (RA) Let $B$ denote the Banach space of continuous, real valued functions on $[0,1] \subset \mathbb{R}$ with the sup norm.

  \item State the Arzela-Ascoli theorem in the context of $B$.

  \item Define what is meant by a compact operator between two Banach spaces.

  \item Prove that the operator $T: \mathcal{B} \rightarrow \mathcal{B}$ defined by

\end{enumerate}

$$
(T f)(x)=\int_{0}^{x} f(y) d y
$$

is compact.

Solution. If $\mathcal{B}, \mathcal{B}^{\prime}$ are Banach spaces, a linear operator $T: \mathcal{B} \rightarrow \mathcal{B}^{\prime}$ is said to be compact if the closure of $\{T v: v \in \mathcal{B},\|v\| \leq 1\}$ (that is, the closure of the image of the closed unit ball) is compact in $\mathcal{B}^{\prime}$.

For our $T$, the image of the closed unit ball is an equicontinuous family of functions on $[0,1]$. Indeed if $f \in \mathcal{B}$ with $\|f\| \leq 1$ then

$$
\left|(T f)\left(x^{\prime}\right)-(T f)(x)\right|=\left|\int_{x}^{x^{\prime}} f(x) d x\right| \leq\left|x^{\prime}-x\right|
$$

so that given $\epsilon>0$ the same $\delta$ (namely $\delta=\epsilon$ ) works uniformly for all such $T f$. Moreover this image is uniformly bounded: $(T f)(0)=0$ for all $f$, so $|(T f)(x)| \leq x \leq 1$ for all $x \in[0,1]$. Hence the closure of $\{T v: v \in \mathcal{B},\|v\| \leq 1\}$ is compact by the Arzelà-Ascoli theorem.

\begin{enumerate}
  \setcounter{enumi}{3}
  \item (A) Let $\mathbb{F}_{q}$ be the finite field with $q$ elements. Show that the number of $3 \times 3$ nilpotent matrices over $\mathbb{F}_{q}$ is $q^{6}$.
\end{enumerate}

Solution: Although $\mathbb{F}_{q}$ is not algebraically closed, there is still a Jordan normal form for nilpotent matrices; by Cayley-Hamilton, for example, we know $T^{3}$ is identically zero if $T$ denotes our endomorphism of our vector space $V$, and we may consider $0 \subset \operatorname{ker} T \subset \operatorname{ker} T^{2} \subset \operatorname{ker} T^{3}=V$. There are only three different possibilities for the list of nontrivial dimensions, namely $(3,3),(2,3),(1,2)$,
and choosing bases appropriately, we find that all nilpotent matrices are conjugate, over $\mathbb{F}_{q}$, to exactly one of

$$
0,\left(\begin{array}{lll}
0 & 1 & 0 \\
0 & 0 & 0 \\
0 & 0 & 0
\end{array}\right),\left(\begin{array}{lll}
0 & 1 & 0 \\
0 & 0 & 1 \\
0 & 0 & 0
\end{array}\right)
$$

There is exactly one matrix in the first case. For the latter two cases, we have only to compute the stabilizer subgroup under the conjugation action for each matrix above to find the size of the conjugacy orbit. First, recall $\left|\mathrm{GL}_{3}\left(\mathbb{F}_{q}\right)\right|=\left(q^{3}-1\right)\left(q^{3}-q\right)\left(q^{3}-q^{2}\right)$ by successively picking the images of the standard basis vectors; that a matrix $\left(\begin{array}{lll}a_{11} & a_{12} & a_{13} \\ a_{21} & a_{22} & a_{33} \\ a_{31} & a_{32} & a_{33}\end{array}\right)$ commutes with the second matrix above is equivalent to the conditions that $a_{11}=a_{22}, a_{21}=$ $a_{31}=a_{23}=0$ and that the determinant vanish is that $a_{11}=a_{22}, a_{33}$ be units, and so the size of this stabilizer group is $(q-1)^{2} q^{3}$. Similarly, finding the subgroup of matrices that commutes with the third matrix above is explicitly equivalent to $a_{11}=a_{22}=a_{33}, a_{12}=a_{23}, a_{21}=a_{31}=a_{32}=0$ of order $(q-1) q^{2}$. Hence, the total number of nilpotent matrices is

$$
\begin{aligned}
\left|\mathcal{N}_{3}\left(\mathbb{F}_{q}\right)\right| & =1+\frac{\left(q^{3}-1\right)\left(q^{3}-q\right)\left(q^{3}-q^{2}\right)}{(q-1)^{2} q^{3}}+\frac{\left(q^{3}-1\right)\left(q^{3}-q\right)\left(q^{3}-q^{2}\right)}{(q-1) q^{2}} \\
& =1+\left(q^{3}-1\right)(q+1)+\left(q^{3}-1\right)\left(q^{3}-q\right) \\
& =q^{6} .
\end{aligned}
$$

\begin{enumerate}
  \setcounter{enumi}{4}
  \item (AT) Let $\mathrm{Sym}^{n} X$ denote the $n$th symmetric power of a CW complex $X$, i.e. $X^{n} / S_{n}$, where the symmetric group $S_{n}$ acts by permuting coordinates. Show that for all $n \geq 2$, the fundamental group of $\operatorname{Sym}^{n} X$ is abelian.
\end{enumerate}

Solution: This is a version of the Eckmann-Hilton argument. If one takes as base-point some point in the (thin) diagonal $X \subset \operatorname{Sym}^{n} X$, then every based loop $S^{1} \rightarrow \operatorname{Sym}^{n} X$ lifts to a loop $S^{1} \rightarrow X^{n}$, and so one directly has $\pi_{1} X^{n} \rightarrow \pi_{1} \operatorname{Sym}^{n} X$, but $\pi_{1} X^{n} \simeq\left(\pi_{1} X\right)^{n}$, and

$$
\begin{aligned}
\left(\gamma_{1}, \cdots, \gamma_{n}\right) & =\left(\gamma_{1}, 1, \cdots 1\right) \circ\left(1, \gamma_{2}, 1, \cdots, 1\right) \circ \cdots \circ\left(1, \cdots, 1, \gamma_{n}\right) \\
& =\left(\gamma_{1}, 1, \cdots, 1\right) \circ\left(\gamma_{2}, 1, \cdots, 1\right) \circ \cdots \circ\left(\gamma_{n}, 1, \cdots, 1\right) \\
& =\left(\gamma_{1} \cdots \gamma_{n}, 1, \cdots, 1\right)
\end{aligned}
$$

in $\pi_{1} \operatorname{Sym}^{n} X$ by simply "waiting" to do each loop $\gamma_{i}$ in turn and then using that as we're in the symmetric product, it doesn't matter in which factor we're doing the loop. So it suffices to show $(\gamma, 1, \cdots, 1)$ and $(\sigma, 1, \cdots, 1)$ commute
but

$$
\begin{aligned}
(\gamma, 1, \cdots, 1)(\sigma, 1, \cdots, 1) & =(\gamma, 1, \cdots, 1)(1, \sigma, 1, \cdots, 1) \\
& =(\gamma, \sigma, 1, \cdots, 1) \\
& =(\sigma, \gamma, 1, \cdots, 1) \\
& =(\sigma, 1, \cdots, 1)(\gamma, 1, \cdots, 1)
\end{aligned}
$$

by using the above reasoning a few more times.

\begin{enumerate}
  \setcounter{enumi}{5}
  \item (DG) Let $S^{2} \subset \mathbb{R}^{3}$ be the unit 2 -sphere, with its usual orientation. Let $X$ be the vector field generating the flow given by
\end{enumerate}

$$
\left(\begin{array}{ccc}
\cos (t) & -\sin (t) & 0 \\
\sin (t) & \cos (t) & 0 \\
0 & 0 & 1
\end{array}\right) \cdot\left[\begin{array}{l}
x \\
y \\
z
\end{array}\right],
$$

and let $\omega$ be the volume form induced by the embedding in $\mathbb{R}^{3}$ (so the total "volume" is $4 \pi$ ). Find a function $f: S^{2} \rightarrow \mathbb{R}$ satisfying

$$
d f=\iota_{X} \omega
$$

where $\iota_{X} \omega$ is the contraction of $\omega$ by $X$.

Solution: The outward unit normal to $S^{2}$ is

$$
x \partial_{x}+y \partial_{y}+z \partial_{z}
$$

so the volume form of $S^{2}$ is

$$
\omega=\iota_{x} \partial_{x}+y \partial_{y}+z \partial_{z} d x \wedge d y \wedge d z=x d y \wedge d z-y d x \wedge d z+z d x \wedge d y
$$

The vector field $X$ is $-y \partial x+x \partial y$. So the contraction $\iota_{X} \omega$ is

$$
y^{2} d z-y z d y+x^{2} d z-x z d x=\left(x^{2}+y^{2}\right) d z-z(y d y+x d x)
$$

Using the fact that on $S^{2}$ one has

$$
x d x+y d y+z d z=0
$$

this becomes

$$
\left(x^{2}+y^{2}+z^{2}\right) d z=d z
$$

We may therefore take $f(x, y, z)=z$.

\section*{QUALIFYING EXAMINATION }
Department of Mathematics

Thursday September 5, 2019 (Day 3)

\begin{enumerate}
  \item (RA) Let $f:[0,1] \rightarrow \mathbb{R}$ be in the Sobolev space $H^{1}([0,1])$; that is, functions $f$ such that both $f$ and its derivative are $L^{2}$-integrable. Prove that
\end{enumerate}

$$
\lim _{n \rightarrow \infty}\left(n \int_{0}^{1} f(x) e^{-2 \pi i n x} d x\right)=0
$$

Solution. The quantity

$$
\int_{0}^{1} f(x) e^{-2 \pi i n x} \mathrm{~d} x=\hat{f}(n)
$$

is the $n$-th Fourier coefficient of $f$. Since $f \in H^{1}([0,1])$, it holds that $n \hat{f}(n)$ is square-summable and, in particular, forms a null-sequence.

\begin{enumerate}
  \setcounter{enumi}{1}
  \item (CA) Given that the sum
\end{enumerate}

$$
\sum_{n \in \mathbb{Z}} \frac{1}{(z-n)^{2}}
$$

converges uniformly on compact subsets of $\mathbb{C} \backslash \mathbb{Z}$ to a meromorphic function on the entire complex plane, prove the identity

$$
\frac{\pi^{2}}{\sin ^{2} \pi z}=\sum_{n \in \mathbb{Z}} \frac{1}{(z-n)^{2}}
$$

Solution: Both sides of the desired equality are entire meromorphic functions with double poles at the integers. Moreover, they have the same polar part $1 /(z-n)^{2}$ at each $n \in \mathbb{Z}$, so that the difference

$$
g(z)=\frac{\pi^{2}}{\sin ^{2} \pi z}-\sum_{n \in \mathbb{Z}} \frac{1}{(z-n)^{2}}
$$

is an entire holomorphic function.

We claim now that $g(z)$ is bounded, and hence by Liouville's theorem constant. By the periodicity $g(z+n)=g(z) \forall n \in \mathbb{Z}$, it suffices to prove that it's bounded in the strip $0 \leq \Re(z) \leq 1$, and hence that it's bounded in the region
$0 \leq \Re(z) \leq 1,|\Im(z)|>N$. But each of the two terms in the expression for $g(z)$ above has limit 0 as $|\Im(z)| \rightarrow \infty$; it follows that $g(z)$ is constant and hence 0 .

\begin{enumerate}
  \setcounter{enumi}{2}
  \item (AG) Let $C \subset \mathbb{P}^{3}$ be a smooth curve of degree 5 and genus 2 .
\end{enumerate}

(a) By considering the restriction map $\rho: H^{0}\left(\mathcal{O}_{\mathbb{P}^{3}}(2)\right) \rightarrow H^{0}\left(\mathcal{O}_{C}(2)\right)$, show that $C$ must lie on a quadric surface $Q$.

(b) Show that the quadric surface $Q$ is unique.

(c) Similarly, show that $C$ must lie on at least one cubic surface $S$ not containing $Q$.

(d) Finally, deduce that there exists a line $L \subset \mathbb{P}^{3}$ such that the union $C \cup L$ is a complete intersection of a quadric and a cubic.

Solution: For the first part, we know that $h^{0}\left(\mathcal{O}_{\mathbb{P}^{3}}(2)\right)=10$, while by RiemannRoch we have $h^{0}\left(\mathcal{O}_{C}(2)\right)=9$; thus the map $\rho$ must have a kernel. For the second, observe that $Q$ must be irreducible, since $C$ cannot lie in a plane (there are no smooth plane curves of degree 5). If there were a second quadric $Q^{\prime} \neq Q$ containing $C$, then, the intersection $Q \cap Q^{\prime}$ would be one-dimensional and so by Bezout of degree at most 4 .

Similarly, we have $h^{0}\left(\mathcal{O}_{\mathbb{P}^{3}}(3)\right)=20$, while by Riemann-Roch we have $h^{0}\left(\mathcal{O}_{C}(3)\right)=$ 14 , so there is at least a 6 -dimensional vector space of cubic polynomials vanishing on $C$; only a four-dimensional subspace of these can be multiples of the quadratic polynomial defining $Q$. Finally, if $S$ is any cubic surface containing $C$ but not containing $Q$, by Bezout the intersection $S \cap Q$ will have to consist of the union of $C$ and a line $L$.

\begin{enumerate}
  \setcounter{enumi}{3}
  \item (A) Show that if $p, q$ are distinct primes then the polynomial $\left(x^{p}-1\right) /(x-1)$ is irreducible $\bmod q$ if an only if $q$ is a primitive residue of $p$ (i.e. if every integer that is not a multiple of $p$ is congruent to $q^{e} \bmod p$ for some integer $e$ ).
\end{enumerate}

ii) Prove that $x^{6}+x^{5}+x^{4}+x^{3}+x^{2}+x+1$ factors mod 23 as the product of two irreducible cubics.

Solution. Since $p$ is not $0 \bmod q$, the zeros of $x^{p}-1 \bmod q$ are distinct (a multiple root would have $\left.p x^{p-1}=0\right)$. So the zeros of $f_{p}(x):=\left(x^{p}-1\right) /(x-1) \bmod q$ are the $p-1$ nontrivial $p$-th roots of unity in the splitting field, call it $F$, of $f_{p}$ over $k:=\mathbf{Z} / q \mathbf{Z}$. Now $f_{p}$ is irreducible if and only if the $\operatorname{Gal}(F / k)$ permutes those zeros transitively. For an extension of finite fields, $\operatorname{Gal}(F / k)$ is generated by the Frobenius automorphism $\phi: x \mapsto x^{q}$. If $\zeta$ is a root of $f_{p}$ then the
other roots are $\zeta^{n}$ with $n$ ranging over $(\mathbf{Z} / p \mathbf{Z})^{*}$, and $\phi^{e}$ takes $\zeta$ to $\zeta^{q^{e}}$, so the Galois orbit contains all the roots if and only if $q$ is a generator of $(\mathbf{Z} / p \mathbf{Z})^{*}$, Q.E.D.

ii) Here $(p, q)=(7,23)$ and $23 \equiv 2 \bmod 7$. The smallest $e>0$ such that $2^{e} \equiv 1 \bmod 7$ is $e=3$, so the Galois orbits on nontrivial 7 th roots of unity have size 3 , whence each Galois-stable factor of $f_{7}$ has degree 3, Q.E.D. [The factorization is $\left(x^{3}+10 x^{2}+9 x-1\right)\left(x^{3}-9 x^{2}-10 x-1\right)$.]

\begin{enumerate}
  \setcounter{enumi}{4}
  \item (DG) Suppose that $G$ is a Lie group.
\end{enumerate}

(a) Consider the map $\iota: G \rightarrow G$ defined by $\iota(g)=g^{-1}$. Show that the derivative of $\iota$ at the identity element is multiplication by -1 .

(b) For $g \in G$ define maps $L_{g}, R_{g}: G \rightarrow G$ by

$$
\begin{aligned}
& L_{g}(x)=g x \\
& R_{g}(x)=x g
\end{aligned}
$$

Show that if $\omega$ is a $k$-form which is bi-invariant in the sense that $L_{g}^{*} \omega=$ $R_{g}^{*} \omega$ then $\iota^{*} \omega=(-1)^{k} \omega$.

(c) Show that bi-invariant forms on $G$ are closed.

Solution: For the first part note that if $g(t)=\exp (\lambda t)$ then $g(t)^{-1}=\exp (-\lambda t)$. The claim follows by taking the derivative with respect to $t$ at $t=0$. For the second part note that

$$
\iota^{*} \omega_{g}=\iota^{*} L_{g^{-1}}^{*} \omega_{e}=R_{g}^{*} \iota^{*} \omega_{e}
$$

so it suffices to show that $\iota^{*} \omega_{e}=(-1)^{k} \omega_{e}$. But if $X_{1}, \ldots, X_{k}$ are tangent vectors at $e$ then

$$
\begin{aligned}
\iota^{*}(\omega)\left(X_{1}, \ldots, X_{k}\right) & =\omega\left(d \iota\left(X_{1}\right), \ldots, d_{\iota} X_{k}\right) \\
& =\omega\left(-X_{1}, \ldots,-X_{k}\right) \\
& =(-1)^{k} \omega\left(X_{1}, \ldots, X_{k}\right)
\end{aligned}
$$

by the first part. The third part follows from the equations

$$
(-1)^{(k+1)} d \omega=\iota^{*} d \omega=d \iota^{*} \omega=(-1)^{k} d \omega .
$$

\begin{enumerate}
  \setcounter{enumi}{5}
  \item (AT) Suppose that $m$ is odd. Show that if $n$ is odd there is a fixed point free action of $\mathbb{Z} / m$ on $S^{n}$. What happens if $n$ is even?
\end{enumerate}

Solution: For the first part write $n=2 k-1$ and regard $S^{n}$ as the unit sphere in $\mathbb{C}^{n}$. The formula

$$
\left(z_{1}, \ldots, z_{n}\right) \mapsto e^{2 \pi i / m}\left(z_{1}, \ldots, z_{n}\right)
$$

defines a free action of $\mathbb{Z} / m$ on $S^{n}$. (This part does not require $m$ to be odd). There can be no free action on $S^{2 k}$. This follows from the Lefschetz fixed point formula. Since the automorphism group of $\mathbb{Z}$ is cyclic of order 2 and $m$ is odd, there are no non-trivial actions of $\mathbb{Z} / m$ on $\mathbb{Z}$. It follows that the Lefschetz number of any action is 2 , so there must be a fixed point.


\end{document}