\documentclass[10pt]{article}
\usepackage[utf8]{inputenc}
\usepackage[T1]{fontenc}
\usepackage{amsmath}
\usepackage{amsfonts}
\usepackage{amssymb}
\usepackage[version=4]{mhchem}
\usepackage{stmaryrd}
\usepackage{bbold}
\usepackage{graphicx}
\usepackage[export]{adjustbox}
\graphicspath{ {./images/} }

\title{Solutions of Qualifying Exams I, 2013 Fall }

\author{}
\date{}


\begin{document}
\maketitle
\begin{enumerate}
  \item (Algebra) Consider the algebra $M_{2}(k)$ of $2 \times 2$ matrices over a field $k$. Recall that an idempotent in an algebra is an element $e$ such that $e^{2}=e$.
\end{enumerate}

(a) Show that an idempotent $e \in M_{2}(k)$ different from 0 and 1 is conjugate to

$$
e_{1}:=\left(\begin{array}{cc}
1 & 0 \\
0 & 0
\end{array}\right)
$$

by an element of $G L_{2}(k)$.

(b) Find the stabilizer in $G L_{2}(k)$ of $e_{1} \in M_{2}(k)$ under the conjugation action.

(c) In case $k=\mathbb{F}_{p}$ is the prime field with $p$ elements, compute the number of idempotents in $M_{2}(k)$. (Count 0 and 1 in.)

Solution. (a) Since $e \neq 0,1$, the image and the kernel of $e$ are both onedimensional. Let $v_{1}$ be a nonzero element in the image, so $v_{1}=e\left(v_{0}\right)$ for some $v_{0} \in k^{\oplus 2}$. Then

$$
e\left(v_{1}\right)=e\left(e\left(v_{0}\right)\right)=e^{2}\left(v_{0}\right)=e\left(v_{0}\right)=v_{1}
$$

Pick a nonzero element $v_{2}$ in the kernel of $e$, and we get a basis of $k^{\oplus 2}$ in which $e$ takes the form $e_{1}$.

(b) For a general element

$$
g=\left(\begin{array}{ll}
a & b \\
c & d
\end{array}\right)
$$

to be in the stabilizer, it must satisfy $g e_{1}=e_{1} g$. Writing the equation in four entries out, one sees that it means $b=c=0$ (and $a, d$ arbitrary). So the centralizer is the subgroup of diagonal matrices.

(c) By (a) and (b), the set of rank 1 idempotents is in bijection with $G L_{2}\left(\mathbb{F}_{p}\right) / T\left(\mathbb{F}_{p}\right)$, whose cardinality is

$$
\frac{\left(p^{2}-1\right)\left(p^{2}-p\right)}{(p-1)(p-1)}=(p+1) p
$$

So the total number of idempotents is equal to $p^{2}+p+2$.

\begin{enumerate}
  \setcounter{enumi}{1}
  \item (Algebraic Geometry) (a) Find an everywhere regular differential $n$-form on the affine $n$-space $\mathbb{A}^{n}$.
\end{enumerate}

(b) Prove that the canonical bundle of the projective $n$-dimensional space $\mathbb{P}^{n}$ is $\mathcal{O}(-n-1)$.

Solution (Sketch). Part (a) is really a hint for Part (b). Letting $x_{1}, x_{2}, \ldots, x_{n}$ be affine $\left(\mathbb{A}^{n}\right)$ coordinates, put $\omega:=d x_{1} \wedge d x_{2} \cdots \wedge d x_{n}$ giving (a). Denoting the corresponding homogenous $\mathbb{P}^{n}$ coordinates $t_{0}, t_{1}, \ldots, t_{n}$, with $x_{i}:=t_{i} / t_{0}$ for $i=1,2, \ldots, n$ extend $\omega$ to $\mathbb{P}^{n}$ writing $d x_{i}=d t_{i} / t_{0}-t_{i} / t_{0}^{2} d t_{0}$ and wedging to discover that the divisor of poles of $\omega$ is $(n+1) H$ where $H$ is the hyperplane at infinity $\left(t_{0}=0\right)$ and then conclude (appropriately).

\begin{enumerate}
  \setcounter{enumi}{2}
  \item (Complex Analysis) (Bol's Theorem of 1949). Let $\tilde{W}$ be a domain in $\mathbb{C}$ and $W$ be a relatively compact nonempty subdomain of $\tilde{W}$. Let $\varepsilon>0$ and $G_{\varepsilon}$ be the set of all $(a, b, c, d) \in \mathbb{C}$ such that $\max (|a-1|,|b|,|c|,|d-1|)<\varepsilon$. Assume that $c z+d \neq 0$ and $\frac{a z+b}{c z+d} \in \tilde{W}$ for $z \in W$ and $(a, b, c, d) \in G_{\varepsilon}$. Let $m \geq 2$ be an integer. Prove that there exists a positive integer $\ell$ (depending on $m$ ) with the property that for any holomorphic function $\varphi$ on $\tilde{W}$ such that
\end{enumerate}

$$
\varphi(z)=\varphi\left(\frac{a z+b}{c z+d}\right) \frac{(c z+d)^{2 m}}{(a d-b c)^{m}}
$$

for $z \in W$ and $(a, b, c, d) \in G_{\varepsilon}$, the $\ell$-th derivative $\psi(z)=\varphi^{(\ell)}(z)$ of $\varphi(z)$ on $\tilde{W}$ satisfies the equation

$$
\psi(z)=\psi\left(\frac{a z+b}{c z+d}\right) \frac{(a d-b c)^{\ell-m}}{(c z+d)^{2(\ell-m)}}
$$

for $z \in W$ and $(a, b, c, d) \in G_{\varepsilon}$. Express $\ell$ in terms of $m$.

Hint: Use Cauchy's integral formula for derivatives.

Solution. Let

$$
A z=\frac{a z+b}{c z+d}
$$

for $A \in G_{\varepsilon}$. We take a positive integer $\ell$ which we will determine later as a function of $n$. We use Cauchy's integral formula for derivatives to take the $\ell$-th derivative $\psi(z)$ of $\varphi(z)$. For $z \in \tilde{W}$ we use $U(z)$ to denote an open
neighborhood of $z$ in $\tilde{W}$ and use $\partial U(z)$ to denote its boundary. The $\ell$-th derivative $\psi$ of $\varphi$ at $z \in \tilde{W}$ is given by the formula

$$
\psi(z)=\frac{\ell !}{2 \pi \sqrt{-1}} \int_{\zeta \in \partial U(z)} \frac{\varphi(\zeta) d \zeta}{(\zeta-z)^{\ell+1}}
$$

and

$$
\psi(A z)=\frac{\ell !}{2 \pi \sqrt{-1}} \int_{\zeta \in \partial U(A z)} \frac{\varphi(\zeta) d \zeta}{(\zeta-A z)^{\ell+1}} \quad \text { when } A z \in \tilde{W}
$$

It follows from

$$
\begin{aligned}
\zeta \in U(z) & \Longleftrightarrow A \zeta \in U(A z), \\
\zeta \in \partial U(z) & \Longleftrightarrow A \zeta \in \partial U(A z)
\end{aligned}
$$

with the change of variable $\zeta \mapsto A \zeta$, that

$$
\int_{\zeta \in \partial U(A z)} \frac{\varphi(\zeta) d \zeta}{(\zeta-A z)^{\ell+1}}=\int_{A \zeta \in \partial U(A z)} \frac{\varphi(A \zeta) d(A \zeta)}{(A \zeta-A z)^{\ell+1}}
$$

From the following straightforward direct computation of the discrete version of the formula for the derivative of fractional linear transformation

$$
\begin{aligned}
A \zeta-A z & =\frac{a \zeta+b}{c \zeta+d}-\frac{a z+b}{c z+d} \\
& =\frac{(a \zeta+b)(c z+d)-(a z+b)(c \zeta+d)}{(c \zeta+d)(c z+d)} \\
& =\frac{(a c \zeta z+b c z+a d \zeta+b d)-(a c \zeta z+a d z+b c \zeta+b d)}{(c \zeta+d)(c z+d)} \\
& =\frac{(a d-b c)(\zeta-z)}{(c \zeta+d)(c z+d)}
\end{aligned}
$$

we obtain

$$
\begin{aligned}
\int_{A \zeta \in \partial U(A z)} \frac{\varphi(A \zeta) d(A \zeta)}{(A \zeta-A z)^{\ell+1}} & =\int_{\zeta \in \partial U(z)} \frac{\varphi\left(\frac{a \zeta+b}{c \zeta+d}\right) \frac{a d-b c}{(c \zeta+d)^{2}} d \zeta}{\frac{(a d-b c)^{\ell+1}(\zeta-z)^{\ell+1}}{(c \zeta+d)^{\ell+1}(c z+d)^{\ell+1}}} \\
& =\int_{\zeta \in \partial U(z)} \frac{\varphi(\zeta) \frac{(a d-b c)^{m}}{(c \zeta+d)^{2 m}} \frac{a d-b c}{(c \zeta+d)^{2}} d \zeta}{\frac{(a d-b c)^{\ell+1}\left(\zeta-z \ell^{\ell+1}\right.}{(c \zeta+d)^{\ell+1}(c z+d)^{\ell+1}}} \\
& =\frac{(c z+d)^{\ell+1}}{(a d-b c)^{\ell-m}} \int_{\zeta \in \partial U(z)} \frac{\varphi(\zeta) d \zeta}{(\zeta-z)^{\ell+1}}(c \zeta+d)^{\ell-1-2 m}
\end{aligned}
$$

The extra factor $(c \zeta+d)^{\ell-1-2 m}$ inside the integrand on the extreme righthand side becomes 1 and can be dropped if $\ell-1-2 m=0$, that is, if $\ell=2 m+1$. Thus, if $\ell=2 m+1$, then

$$
\psi(A z)=\frac{(c z+d)^{\ell+1}}{(a d-b c)^{\ell-m}} \psi(z)
$$

That is,

$$
\psi(z)=\psi\left(\frac{a z+b}{c z+d}\right) \frac{(a d-b c)^{\ell-m}}{(c z+d)^{2(\ell-m)}}
$$

because $\ell=2 m+1$ implies $\ell+1=2(\ell-m)$.

\begin{enumerate}
  \setcounter{enumi}{3}
  \item (Algebraic Topology) (a) Show that the Euler characteristic of any contractible space is 1.
\end{enumerate}

(b) Let $B$ be a connected $\mathrm{CW}$ complex made of finitely many cells so that its Euler characteristic is defined. Let $E \rightarrow B$ be a covering map whose fibers are discrete, finite sets of cardinality $N$. Show the Euler characteristic of $E$ is $N$ times the Euler characteristic of $B$.

(c) Let $G$ be a finite group with cardinality $>2$. Show that $B G$ (the classifying space of $G$ ) cannot have homology groups whose direct sum has finite rank.

Solution. (a) The homology of a point with coefficients in a field $k$ is $H_{0}=k$, $H_{i}=0$ for $i>0$. Hence its Euler characteristic is $\sum(-1)^{i} \operatorname{dim} H_{i}=1$. All contractible spaces are homotopy equivalent so their Euler characteristic is that of the point.

(b) For any open cover $\left\{U_{i}\right\}$, we know that the chain complex of singular chains living in $U_{i}$ for some $i$ has equivalent homology to the chain complex of all chains. Taking the cover of $B$ by trivializing neighborhoods $U_{i}$, the chain complex of chains living in $U_{i}$ receives a map from chains in $E$ living in $\pi^{-1}\left(U_{i}\right)$. The latter is simply $|G|$ direct sums of the former, and the chain map between them is the "add every component" map. This shows the ranks of homology of $E$ is $N$ times the rank of homology of $B$.

(c) Strictly speaking, this problem cannot be solved based on easy machinery (as far as I know). A much more reasonable problem would be: Prove $B G$ is not homotopy equivalent to anything made up of only finitely many cells. I did not take off points for people not distinguishing between this condition,
and the condition stated in the problem itself. We know $B G=E G / G$, but $E G$ is contractible. So $\chi(E G)=1$. If $B G$ has finite homology, $\chi(B G)=$ $1 /|G|$, which cannot be an integer unless $|G|=1$.

\begin{enumerate}
  \setcounter{enumi}{4}
  \item (Differential Geometry) Let $H=\left\{(x, y) \in \mathbb{R}^{2}: y>0\right\}$ be the upper half plane. Let $g$ be the Riemannian metric on $H$ given by
\end{enumerate}

$$
g=\frac{(\mathrm{d} x)^{2}+(\mathrm{d} y)^{2}}{y^{2}}
$$

$(H, g)$ is known as the half-plane model of the hyperbolic plane.

(a) Let $\gamma(\theta)=(\cos \theta, \sin \theta)$ and $\eta(\theta)=(\cos \theta+1, \sin \theta)$ for $\theta \in(0, \pi)$ be two paths in $H$. Compute the angle $A$ at their intersection point shown in Figure 1 , measured by the metric $g$.

\begin{center}
\includegraphics[max width=\textwidth]{2023_10_29_d1ae8b5466f5f6dcce3cg-05}
\end{center}

Figure 1: Angle $A$ between the two curves $\gamma$ and $\eta$ in the upper half plane $H$.

(b) By computing the Levi-Civita connection

$$
\nabla_{\frac{\partial}{\partial x_{i}}} \frac{\partial}{\partial x_{j}}=\sum_{k=1}^{2} \Gamma_{i j}^{k} \frac{\partial}{\partial x_{k}}
$$

of $g$ or otherwise (where $\left(x_{1}, x_{2}\right)=(x, y)$ ), show that the path $\gamma$, after arclength reparametrization, is a geodesic with respect to the metric $g$.

Solution. (a) The intersection point is $(1 / 2, \sqrt{3} / 2)$ : solving for

$$
\gamma(\theta)=(\cos \theta, \sin \theta)=(\cos \phi+1, \sin \phi)=\eta(\phi)
$$

we obtain $\theta=\pi / 3, \phi=2 \pi / 3$.

The angle $A$ satisfies

$$
\begin{aligned}
\cos A & =\frac{\left\langle\gamma^{\prime}(\pi / 3),-\eta^{\prime}(2 \pi / 3)\right\rangle_{g}}{\left\|\gamma^{\prime}(\pi / 3)\right\|_{g}\left\|-\eta^{\prime}(2 \pi / 3)\right\|_{g}} \\
& =\frac{\langle(-\sqrt{3} / 2,1 / 2),(\sqrt{3} / 2,1 / 2)\rangle_{g}}{\|(-\sqrt{3} / 2,1 / 2)\|_{g}\|(\sqrt{3} / 2,1 / 2)\|_{g}} \\
& =\frac{-\frac{1}{2} \frac{1}{y^{2}}}{\frac{1}{y^{2}}} \\
& =-\frac{1}{2}
\end{aligned}
$$

and so $A=2 \pi / 3$.

(b) Using the formula

$$
\Gamma_{j k}^{i}=\frac{1}{2} g^{i l}\left(g_{j l, k}+g_{j l, j}-g_{j k, l}\right)
$$

one obtains

$$
\Gamma_{j k}^{i}=\frac{-1}{y}\left(\delta_{i j} \delta_{k, 2}+\delta_{k i} \delta_{j, 2}-\delta_{j k} \delta_{i, 2}\right)
$$

After arc-length reparametrization, the tangent vectors of the path are

$$
v(\theta)=\frac{\gamma^{\prime}(\theta)}{\left\|\gamma^{\prime}(\theta)\right\|_{g}}=\left(-\sin ^{2} \theta, \sin \theta \cos \theta\right)
$$

Then

$$
\nabla_{v(\theta)} v(\theta)=v^{\prime}(\theta)+\left(\begin{array}{cc}
\Gamma_{1}^{1} & \Gamma_{2}^{1} \\
\Gamma_{1}^{2} & \Gamma_{2}^{2}
\end{array}\right) \cdot v(\theta)
$$

where

$$
\begin{aligned}
& \Gamma_{1}^{1}=(-\sin \theta) \Gamma_{11}^{1}+(\cos \theta) \Gamma_{21}^{1}=-\cot \theta \\
& \Gamma_{2}^{1}=(-\sin \theta) \Gamma_{12}^{1}+(\cos \theta) \Gamma_{22}^{1}=1 ; \\
& \Gamma_{1}^{2}=(-\sin \theta) \Gamma_{11}^{2}+(\cos \theta) \Gamma_{21}^{2}=-1 ; \\
& \Gamma_{2}^{2}=(-\sin \theta) \Gamma_{12}^{2}+(\cos \theta) \Gamma_{22}^{2}=-\cot \theta
\end{aligned}
$$

Thus one has $\nabla_{v(\theta)} v(\theta)=0$.

\begin{enumerate}
  \setcounter{enumi}{5}
  \item (Real Analysis) For any positive integer $n$ let $M_{n}$ be a positive number such that the series $\sum_{n=1}^{\infty} M_{n}$ of positive numbers is convergent and its limit is $M$. Let $a<b$ be real numbers and $f_{n}(x)$ be a real-valued continuous function on $[a, b]$ for any positive integer $n$ such that its derivative $f_{n}^{\prime}(x)$ exists for every $a<x<b$ with $\left|f_{n}^{\prime}(x)\right| \leq M_{n}$ for $a<x<b$. Assume that the series $\sum_{n=1}^{\infty} f_{n}(a)$ of real numbers converges. Prove that
\end{enumerate}

(a) the series $\sum_{n=1}^{\infty} f_{n}(x)$ converges to some real-valued function $f(x)$ for every $a \leq x \leq b$,

(b) $f^{\prime}(x)$ exists for every $a<x<b$, and

(c) $\left|f^{\prime}(x)\right| \leq M$ for $a<x<b$.

Hint for (b): For fixed $x \in(a, b)$ consider the series of functions

$$
\sum_{n=1}^{\infty} \frac{f_{n}(y)-f_{n}(x)}{y-x}
$$

of the variable $y$ and its uniform convergence.

Solution. (a) Fix $x \in(a, b]$. For $q>p \geq 1$, by the Mean Value Theorem applied to the function $\sum_{n=p}^{q} f_{n}$ on $[a, x]$ we can find $a<\xi_{p, q}<x$ such that

$$
\sum_{n=p}^{q} f_{n}(x)-\sum_{n=p}^{q} f_{n}(a)=(x-a) \sum_{n=p}^{q} f_{n}^{\prime}\left(\xi_{p, q}\right)
$$

which implies that

$$
\begin{gathered}
\left|\sum_{n=p}^{q} f_{n}(x)\right| \leq\left|\sum_{n=p}^{q} f_{n}(a)\right|+(x-a)\left|\sum_{n=p}^{q} f_{n}^{\prime}\left(\xi_{p, q}\right)\right| \\
\leq\left|\sum_{n=p}^{q} f_{n}(a)\right|+(x-a) \sum_{n=p}^{q} M_{n} .
\end{gathered}
$$

Since both series $\sum_{n=1}^{\infty} f_{n}(a)$ and $\sum_{n=1}^{\infty} M_{n}$ are convergent and therefore Cauchy, for any $\varepsilon>0$ we can find a positive integer $N_{1}$ such that

$$
\left|\sum_{n=p}^{q} f_{n}(a)\right|<\frac{\varepsilon}{2}
$$

for $q>p \geq N_{1}$ and we can find a positive integer $N_{2}$ such that

$$
\left|\sum_{n=p}^{q} M_{n}\right|<\frac{\varepsilon}{2(x-a)}
$$

for $q>p \geq N_{2}$. Thus for $n \geq \max \left(N_{1}, N_{2}\right)$ we have

$$
\left|\sum_{n=p}^{q} f_{n}(x)\right|<\varepsilon
$$

and the series $\sum_{n=1}^{\infty} f_{n}(x)$ is Cauchy. Hence the series $\sum_{n=1}^{\infty} f_{n}(x)$ converges to some real-valued function $f(x)$ for every $a \leq x \leq b$.

(b) Before the proof of the statement in (b), we would like to state that the uniform limit of continuous functions is continuous. That is, if $h_{n}(x)$ is a sequence of functions on a metric space $E$ which converges to a function $h(x)$ on $E$ uniformly on $E$ and if for some $x_{0} \in E$ and for every $n$ the function $h_{n}(x)$ is continuous at $x=x_{0}$, then $h(x)$ is continuous at $x_{0}$. This results from the so-called $3 \varepsilon$ argument as follows. Given any $\varepsilon>0$. The uniform convergence of $h_{n} \rightarrow h$ on $E$ implies that there exists some positive integer $N$ such that $\left|h_{N}(x)-h(x)\right|<\varepsilon$ for all $x \in E$. Since $h_{N}$ is continuous at $x=x_{0}$, there exists some $\delta>0$ such that $\left|h_{N}(x)-h_{N}\left(x_{0}\right)\right|<\varepsilon$ for $d_{E}\left(x, x_{0}\right)<\delta$ (where $d_{E}(\cdot, \cdot)$ is the metric of the metric space $E$ ). Thus for $d_{E}\left(x, x_{0}\right)<\delta$ we have

$\left|h(x)-h\left(x_{0}\right)\right| \leq\left|h(x)-h_{N}(x)\right|+\left|h_{N}(x)-h_{N}\left(x_{0}\right)\right|+\left|h_{N}\left(x_{0}\right)-h\left(x_{0}\right)\right|<3 \varepsilon$,

which implies the continuity of $h$ at $x=x_{0}$.

We now prove the statement in (b). Take $x_{0} \in(a, b)$. We introduce the function $g_{n, x_{0}}(x)$ on $[a, b]$ which is defined by

$$
\left\{\begin{array}{l}
g_{n, x_{0}}(x)=\frac{f_{n}(x)-f_{n}\left(x_{0}\right)}{x-x_{0}} \text { for } x \neq x_{0} \\
g_{n, x_{0}}\left(x_{0}\right)=f_{n}^{\prime}\left(x_{0}\right) .
\end{array}\right.
$$

It follows from the continuity of $f_{n}$ on $[a, b]$ and the existence of $f_{n}^{\prime}\left(x_{0}\right)$ that $g_{n, x_{0}}$ is a continuous function on $[a, b]$.

When $x \in[a, b]$ with $x \neq x_{0}$, by the Mean Value Theorem

$$
\frac{f_{n}(x)-f_{n}\left(x_{0}\right)}{x-x_{0}}=f_{n}^{\prime}\left(\xi_{x}\right)
$$

for some $\xi_{x}$ strictly between $x_{0}$ and $x$ and as a consequence

$$
\left|g_{n, x_{0}}(x)\right|=\left|f_{n}^{\prime}\left(x_{0}\right)\right| \leq M_{n}
$$

When $x=x_{0}$,

$$
\left|g_{n, x_{0}}(x)\right|=\left|f_{n}^{\prime}\left(x_{0}\right)\right| \leq M_{n}
$$

Thus $\left|g_{n, x_{0}}(x)\right| \leq M_{n}$ for $x \in[a, b]$. From $\sum_{n=1}^{\infty} M_{n} \leq M<\infty$ it follows that the series $\sum_{n=1}^{\infty} g_{n, x_{0}}$ is uniformly convergent on $[a, b]$. It follows that the uniform limit $\sum_{n=1}^{\infty} g_{n, x_{0}}$ is a continuous function on $[a, b]$ by the $3 \varepsilon$ argument given above. For $x \neq x_{0}$

$$
\sum_{n=1}^{\infty} g_{n, x_{0}}(x)=\sum_{n=1}^{\infty} \frac{f_{n}(x)-f_{n}\left(x_{0}\right)}{x-x_{0}}=\frac{f(x)-f\left(x_{0}\right)}{x-x_{0}}
$$

The continuity of $\sum_{n=1}^{\infty} g_{n, x_{0}}(x)$ at $x=x_{0}$ means that the limit of

$$
\frac{f(x)-f\left(x_{0}\right)}{x-x_{0}}
$$

exists as $x \rightarrow x_{0}$, which implies that $f^{\prime}\left(x_{0}\right)$ exists and is equal to

$$
\sum_{n=1}^{\infty} g_{n, x_{0}}\left(x_{0}\right)=\sum_{n=1}^{\infty} f_{n}^{\prime}\left(x_{0}\right)
$$

(c) From

$$
f^{\prime}\left(x_{0}\right)=\sum_{n=1}^{\infty} g_{n, x_{0}}\left(x_{0}\right)=\sum_{n=1}^{\infty} f_{n}^{\prime}\left(x_{0}\right)
$$

and $\left|f_{n}^{\prime}\left(x_{0}\right)\right| \leq M_{n}$, it follows that

$$
\left|f^{\prime}\left(x_{0}\right)\right| \leq \sum_{n=1}^{\infty} M_{n}=M
$$

\section{Solutions of Qualifying Exams II, 2013 Fall}
\begin{enumerate}
  \item (Algebra) Find all the field automorphisms of the real numbers $\mathbb{R}$.
\end{enumerate}

Hint: Show that any automorphism maps a positive number to a positive number, and deduce from this that it is continuous.

Solution. If $t>0$, there exists an element $s \neq 0$ such that $t=s^{2}$. If $\varphi$ is any field automorphism of $\mathbb{R}$, then

$$
\varphi(t)=\varphi\left(s^{2}\right)=(\varphi(s))^{2}>0
$$

It follows that $\varphi$ preserves the order on $\mathbb{R}$ : If $t<t^{\prime}$, then

$$
\varphi\left(t^{\prime}\right)=\varphi\left(t+\left(t^{\prime}-t\right)\right)=\varphi(t)+\varphi\left(t^{\prime}-t\right)>\varphi(t)
$$

Any real number $\alpha$ is determined by the set (Dedekind's cut) of rational numbers that are less than $\alpha$, and any field automorphism fixes each rational number. Therefore $\varphi$ is the identity automorphism.

\begin{enumerate}
  \setcounter{enumi}{1}
  \item (Algebraic Geometry) What is the maximum number of ramification points that a mapping of finite degree from one smooth projective curve over $\mathbb{C}$ of genus 1 to another (smooth projective curve of genus 1 ) can have? Give an explanation for your answer.
\end{enumerate}

Solution (Sketch). By the Riemann-Hurwitz formula, if we have a mapping $f$ of finite degree $d$ from one smooth projective (irreducible, say) curve onto another the Euler characteristic of the source curve is $d$ times the Euler characteristic of the target minus a certain nonnegative number $e$, and moreover $e$ is zero if and only if the mapping is unramified. Now compute: the Euler characterstic of our source and target curves is, by hypothesis, 0 and so this $e$ is zero, and therefore the mapping is unramified.

\begin{enumerate}
  \setcounter{enumi}{2}
  \item (COmplex Analysis) Let $\omega$ and $\eta$ be two complex numbers such that $\operatorname{Im}\left(\frac{\omega}{\eta}\right)>0$. Let $G$ be the closed parallelogram consisting of all $z \in \mathbb{C}$ such that $z=\lambda \omega+\rho \eta$ for some $0 \leq \lambda, \rho \leq 1$. Let $\partial G$ be the boundary of $G$ and Let $G^{0}=G-\partial G$ be the interior of $G$. Let $P_{1}, \cdots, P_{k}, Q_{1}, \cdots, Q_{\ell}$ be points in $G^{0}$ and let $m_{1}, \cdots, m_{k}, n_{1}, \cdots, n_{\ell}$ be positive integers. Let $f$ be a function on $G$ such that
\end{enumerate}

$$
\frac{f(z) \prod_{j=1}^{\ell}\left(z-Q_{j}\right)^{n_{j}}}{\prod_{p=1}^{k}\left(z-P_{p}\right)^{m_{p}}}
$$

is continuous and nowhere zero on $G$ and is holomorphic on $G^{0}$. Let $\varphi(z)$ and $\psi(z)$ be two polynomials on $\mathbb{C}$. Assume that $f(z+\omega)=e^{\varphi(z)} f(z)$ if both $z$ and $z+\omega$ are in $G$. Assume also that $f(z+\eta)=e^{\psi(z)} f(z)$ if both $z$ and $z+\eta$ are in $G$. Express $\sum_{p=1}^{k} m_{p}-\sum_{j=1}^{\ell} n_{j}$ in terms of $\omega$ and $\eta$ and the coefficients of $\varphi(z)$ and $\psi(z)$.

Solution. Let $A=0, B=\eta, C=\eta+\omega$, and $D=\omega$. Since $\operatorname{Im}\left(\frac{\omega}{\eta}\right)>0$, it follows that going from $A$ to $B$, to $C$, to $D$ and then back to $A$ is in the counterclockwise direction. By the argument principle

$$
\begin{aligned}
\sum_{p=1}^{k} m_{p}-\sum_{j=1}^{\ell} n_{j} & =\frac{1}{2 \pi \sqrt{-1}} \oint_{\partial G} d \log f \\
& =\frac{1}{2 \pi \sqrt{-1}}\left(\int_{\overrightarrow{A B}} d \log f+\int_{\overrightarrow{B C}} d \log f+\int_{\overrightarrow{C D}} d \log f+\int_{\overrightarrow{D A}} d \log f\right) \\
& =\frac{1}{2 \pi \sqrt{-1}}\left(\int_{\overrightarrow{A B}} d \log f-\int_{\overrightarrow{C D}} d \log f+\int_{\overrightarrow{B C}} d \log f-\int_{\overrightarrow{A D}} d \log f\right) \\
& =\frac{1}{2 \pi \sqrt{-1}}\left(-\int_{\overrightarrow{A B}} d \varphi(z)+\int_{\overrightarrow{A D}} d \psi(z)\right) \\
& =\frac{1}{2 \pi \sqrt{-1}}(-\varphi(\eta)+\varphi(0)+\psi(\omega)-\psi(0)) .
\end{aligned}
$$

Thus, the answer is

$$
\sum_{p=1}^{k} m_{p}-\sum_{j=1}^{\ell} n_{j}=\frac{1}{2 \pi \sqrt{-1}}(-\varphi(\eta)+\varphi(0)+\psi(\omega)-\psi(0)) .
$$

\begin{enumerate}
  \setcounter{enumi}{3}
  \item (Algebraic Topology) (a) Fix a basis for $H_{1}$ of the two-torus (with integer coefficients). Show that for every element $x \in S L(2, \mathbb{Z})$, there is an automorphism of the two-torus such that the induced map on $H_{1}$ acts by $x$.
\end{enumerate}

Hint: $S L(2, \mathbb{Z})$ also acts on the universal cover of the torus.

(b) Fix an embedding $j: D^{2} \times S^{1} \rightarrow S^{3}$. Remove its interior from $S^{3}$ to obtain a manifold $X$ with boundary $T^{2}$. Let $f$ be an automorphism of the two-torus and consider the glued space

$$
X_{f}:=\left(D^{2} \times S^{1}\right) \cup_{f} X
$$

If $X$ is homotopy equivalent to $D^{2} \times S^{1}$, compute the homology groups of $X_{f}$.

Solution. (a) Given $g \in S L(2, \mathbb{Z}) \subset S L(2, \mathbb{R})$ let $x: \mathbb{R}^{2} \rightarrow \mathbb{R}^{2}$ be the induced action. Since $g$ is in $S L(2, \mathbb{Z})$ it respects the relationship of whether two vectors in $\mathbb{R}^{2}$ differ by integer coordinates. So the map on the torus $\left[\left(x_{1}, x_{2}\right)\right] \mapsto\left[g\left(x_{1}, x_{2}\right)\right]$ is well-defined. This clearly sends a homology generating pair given by the curves $\left(x_{1}, 0\right)$ and $\left(0, x_{2}\right)$ to the expected images via $g$.

(b) There is an ambiguity in the problem about how $f$ glues $X$ and $D^{2} \times$ $S^{1}$ together; so I gave full credit regardless of whether you identified this ambiguity or not. Note $X_{f}=\left(D^{2} \times S^{1}\right) \cup_{S^{1} \times S^{1}} X$. Write $U=D^{2} \times S^{1}$ and $V=X$. The Mayer-Vietoris sequence gives

$$
\begin{array}{r}
\longrightarrow H_{0}(U \cap V) \longrightarrow H_{0}(U) \oplus H_{0}(V) \longrightarrow H_{0}(U \cup V) \\
\longrightarrow H_{1}(U \cap V) \longrightarrow H_{1}(U) \oplus H_{1}(V) \longrightarrow H_{1}(U \cup V) \\
\longrightarrow H_{2}(U \cap V) \longrightarrow H_{2}(U) \oplus H_{2}(V) \longrightarrow H_{2}(U \cup V) \\
\longrightarrow H_{3}(U \cap V) \longrightarrow H_{3}(U) \oplus H_{3}(V) \longrightarrow H_{3}(U \cup V)
\end{array}
$$

but because we know the homology of $D^{2} \times S^{1} \simeq S^{1}$ and $S^{1} \times S^{1}$, we can fill in various groups in the long exact sequence:

\begin{center}
\includegraphics[max width=\textwidth]{2023_10_29_d1ae8b5466f5f6dcce3cg-12}
\end{center}

Since $g$ is an isomorphism, we know $H_{1}$ must inject into $\mathbb{Z}$, but the inclusion map $H_{0}(U \cap V) \rightarrow H_{0}(U) \oplus H_{0}(V)$ is an injection, so $H_{1}(U \cup V)=0$.

We know $H_{0}$ is either equal to $\mathbb{Z}$ from the long exact sequence above, or by observing that $X_{f}$ is path-connected.

If $f$ induces an isomorphism, we see $H_{2}$ must be zero; this was the intent of the problem, but you can get a different answer based on how you interpreted the "gluing" by $f$.

Finally, $H_{3}$ is also isomorphic to $\mathbb{Z}$ by the exactness of the above sequence.

\begin{enumerate}
  \setcounter{enumi}{4}
  \item (Differential Geometry) Let $M=U(n) / O(n)$ for $n \geq 1$, where $U(n)$ is the group of $n \times n$ unitary matrices and $O(n)$ is the group of $n \times n$ orthogonal matrices. $M$ is a real manifold called the Lagrangian Grassmannian.
\end{enumerate}

(a) Compute and state the dimension of $M$.

(b) Construct a Riemannian metric which is invariant under the left action of $U(n)$ on $M$.

(c) Let $\nabla$ be the corresponding Levi-Civita connection on the tangent bundle $T M$, and $X, Y, Z$ be any $U(n)$-invariant vector fields on $M$. Using the given identity (which you are not required to prove)

$$
\nabla_{X} Y=\frac{1}{2}[X, Y]
$$

show that the Riemannian curvature tensor $R$ of $\nabla$ satisfies the formula

$$
R(X, Y) Z=\frac{1}{4}[Z,[X, Y]]
$$

Solution. (a)

$$
T_{[I]} M \cong \mathfrak{u}(n) / \mathfrak{o}(n) \cong \operatorname{Sym}^{2}\left(\mathbb{R}^{n}\right)
$$

where $\operatorname{Sym}^{2}\left(\mathbb{R}^{n}\right)$ denotes the space of real $n \times n$ symmetric matrices. Thus

$$
\operatorname{dim} M=\frac{n(n+1)}{2}
$$

(b) Define a metric on $\operatorname{Sym}^{2}\left(\mathbb{R}^{n}\right)$ by

$$
\langle A, B\rangle=\operatorname{tr}\left(A B^{t}\right)=\operatorname{tr}(A B)
$$

$g \in O(n)$ acts on $T_{[I]} M \cong \operatorname{Sym}^{2}\left(\mathbb{R}^{n}\right)$ by $g \cdot A=g A g^{-1}$. Then

$$
\langle g \cdot A, g \cdot B\rangle=\operatorname{tr}\left(g \cdot A B g^{-1}\right)=\langle A, B\rangle
$$

Hence this metric is invariant under the action of $O(n)$. By translating the metric to tangent spaces at other points by the action of $U(n)$, this gives a well-defined invariant metric on $U(n) / O(n)$.
(c)

$$
\nabla_{X} Y=\frac{1}{2}[X, Y]
$$

Then

$$
\begin{aligned}
R(X, Y) Z & =\nabla_{X} \nabla_{Y} Z-\nabla_{Y} \nabla_{X} Z-\nabla_{[X, Y]} Z \\
& =\frac{1}{4}([X,[Y, Z]]-[Y,[X, Z]])-\frac{1}{2}[[X, Y], Z] \\
& =\frac{1}{4}[Z,[X, Y]]
\end{aligned}
$$

where the last equality follows from Jacobi identity.

\begin{enumerate}
  \setcounter{enumi}{5}
  \item (Real Analysis) Show that there is no function $f: \mathbb{R} \rightarrow \mathbb{R}$ whose set of continuous points is precisely the set $\mathbb{Q}$ of all rational numbers.
\end{enumerate}

Solution. For fixed $\delta>0$ let $C(\delta)$ be the set of points $x \in \mathbb{R}$ such that for some $\varepsilon>0$ we have $\left|f\left(x^{\prime}\right)-f\left(x^{\prime \prime}\right)\right|<\delta$ for all $x^{\prime}, x^{\prime \prime} \in(x-\varepsilon, x+\varepsilon)$. Clearly $C(\delta)$ is open since for every $x \in C(\delta)$, we have $(x-\varepsilon, x+\varepsilon) \subset C(\delta)$. Now let $C$ denote the set of continuous points of $f$. From the definitions, we have that

$$
C=\bigcap_{n=1}^{\infty} C(1 / n)
$$

Now suppose that $C=\mathbb{Q}$. Then

$$
\mathbb{R}-\mathbb{Q}=\bigcup_{n=1}^{\infty} X_{n}
$$

where $X_{n}=\mathbb{R}-C(1 / n)$. Since $C(1 / n)$ is open, $X_{n}$ is closed. Also $\mathbb{Q}$ is countable, say $\mathbb{Q}=\left\{q_{1}, q_{2}, \ldots\right\}$. Let $Y_{n}=\left\{q_{n}\right\}$. Then

$$
\mathbb{R}=\left(\bigcup_{n=1}^{\infty} X_{n}\right) \cup\left(\bigcup_{n=1}^{\infty} Y_{n}\right)
$$

i.e. we have written $\mathbb{R}$ as a countable union of closed sets. Then by Baire's theorem, some $X_{n}$ or $Y_{n}$ has nonempty interior. Clearly it cannot be one of the $Y_{n}$. So there exists $X_{n}$ containing an interval $(a, b)$. But this is impossible because $X_{n} \subset \mathbb{R}-\mathbb{Q}$ and every interval contains a rational number. Thus, we obtain a contradiction, which shows that $C \neq \mathbb{Q}$.

\section{Solutions of Qualifying Exams III, 2013 Fall}
\begin{enumerate}
  \item (Algebra) Consider the function fields $K=\mathbb{C}(x)$ and $L=\mathbb{C}(y)$ of one variable, and regard $L$ as a finite extension of $K$ via the $\mathbb{C}$-algebra inclusion
\end{enumerate}

$$
x \mapsto \frac{-\left(y^{5}-1\right)^{2}}{4 y^{5}}
$$

Show that the extension $L / K$ is Galois and determine its Galois group.

Solution. Consider the intermediate extension $K^{\prime}=\mathbb{C}\left(y^{5}\right)$. Then clearly $\left[L: K^{\prime}\right]=5$ and $\left[K^{\prime}: K\right]=2$, therefore $[L: K]=10$.

Thus, to prove that $L / K$ is Galois it is enough to find 10 field automor-

phisms of $L$ over $K$. Choose a primitive 5 th root of 1 , say $\zeta=e^{2 \pi i / 5}$. For $i \in \mathbb{Z} / 5$ and $s \in\{ \pm 1\}$, the $\mathbb{C}$-algebra automorphism $\sigma_{i, s}$ of $L$ defined by

$$
y \mapsto \zeta^{i} y^{s}
$$

leaves $x$, hence $K$, fixed.

There can be many ways to determine the group, here's one.

Looking at the law of composition of these automorphisms, one sees that the subgroup $\operatorname{Gal}\left(L / K^{\prime}\right) \simeq \mathbb{Z} / 5$, (which is necessarily normal, being of index 2 ) is not central, for conjugation by $\sigma_{0,-1}$ acts as -1 on it.

So the group is the dihedral group of 10 elements.

\begin{enumerate}
  \setcounter{enumi}{1}
  \item (Algebraic GeOmetry) Is every smooth projective curve of genus 0 defined over the field of complex numbers isomorphic to a conic in the projective plane? Give an explanation for your answer.
\end{enumerate}

Solution (Sketch). Yes. Apply the Riemann-Roch theorem which guarantees the existence of a nonconstant meromorphic function with a simple pole at exactly one point. Argue that this meromorphic function identifies the curve with $\mathbb{P}^{1}$, and using that fact, embed the curve as a conic in the plane in any convenient way, e.g., If $t_{0}, t_{1}$ are projective $\left(\mathbb{P}^{1}\right)$ coordinates, let $z_{0}=t_{0}^{2}, z_{1}=t_{0} t_{1} z_{2}=t_{1}^{2}$ be the map to $\mathbb{P}^{2}$. The conic, then, would be $z_{0} z_{2}=z_{1}^{2}$. (Alternatively: one can consider the complete linear system attached to the anticanonical divisor.)

\begin{enumerate}
  \setcounter{enumi}{2}
  \item (Complex Analysis) Let $f(z)=z+e^{-z}$ for $z \in \mathbb{C}$ and let $\lambda \in \mathbb{R}, \lambda>1$. Prove or disprove the statement that $f(z)$ takes the value $\lambda$ exactly once in the open right half-plane $H_{r}=\{z \in \mathbb{C}: \operatorname{Re} z>0\}$.
\end{enumerate}

Solution. First, let us consider the real function $f(x)=x+e^{-x}$. Since $f$ is continuous, $f(0)=1$ and $\lim _{x \rightarrow \infty} f(x)=\infty$, by the intermediate value theorem, there exists $u \in \mathbb{R}$ such that $f(u)=\lambda$. Now let us show that such $u$ is unique. Let $R>2 \lambda$ and let $\Gamma$ be the closed right half disk of radius $R$ centered at the origin

$\{z=x+i y \in \mathbb{C}: x=0,|y| \leq R\} \cup\left\{z \in \mathbb{C}:|z|=R,-\frac{\pi}{2} \leq \arg (z) \leq \frac{\pi}{2}\right\}$.

Let $F(z)=\lambda-z$ and $G(z)=-e^{-z}$. Then for $z \in \Gamma$, we have $|G(z)|=$ $\left|e^{-\operatorname{Re} z}\right| \leq 1$ since $\operatorname{Re} z \geq 0$, while $|F(z)|>1$ by construction. Hence by Rouché's theorem, $\lambda-f(z)=F(z)+G(z)$ has the same number of zeros inside $\Gamma$ as $F(z)$, namely 1 . Since this is true for all $R$ large enough, we conclude that the point $u$ is unique.

\begin{enumerate}
  \setcounter{enumi}{3}
  \item (Algebraic Topology) (a) Let $X$ and $Y$ be locally contractible, connected spaces with fixed basepoints. Let $X \vee Y$ be the wedge sum at the basepoints. Show that $\pi_{1}(X \vee Y)$ is the free product of $\pi_{1} X$ with $\pi_{1} Y$.
\end{enumerate}

(b) Show that $\pi_{1}(X \times Y)$ is the direct product of $\pi_{1} X$ with $\pi_{1} Y$.

(c) Note the canonical inclusion $f: X \vee Y \rightarrow X \times Y$. Assume that $X$ and $Y$ have abelian fundamental groups. Show that the map $f_{*}$ on fundamental groups exhibits $\pi_{1}(X \times Y)$ as the abelianization of $\pi_{1}(X \vee Y)$.

Hint: The Hurewicz map is natural.

Solution. (a) This follows form the Van Kampen theorem: Writing $X \vee Y$ as the union

$$
X \cup_{*} Y
$$

we have that $\pi_{1}(X \vee Y) \cong \pi_{1}(X) *_{\pi_{1}(*)} \pi_{1}(Y)=\pi_{1}(X) * \pi_{1} Y$.

(b) There is the obvious continuous map

$$
\operatorname{Maps}_{*}\left(S^{1}, X\right) \times \operatorname{Maps}_{*}\left(S^{1}, Y\right) \rightarrow \operatorname{Maps}_{*}\left(S^{1}, X \times Y\right)
$$

given by sending $\left(t \mapsto \gamma_{X}(t), t \mapsto \gamma_{Y}(t)\right) \mapsto\left(t \mapsto\left(\gamma_{X}(t), \gamma_{Y}(t)\right)\right)$. This map is a continuous so it induces a map

$$
\pi_{0}\left(\operatorname{Maps}_{*}\left(S^{1}, X\right) \times \operatorname{Maps}_{*}\left(S^{1}, Y\right)\right) \rightarrow \pi_{0} \operatorname{Maps}_{*}\left(S^{1}, X \times Y\right)
$$

where the lefthand side is isomorphic to $\left.\pi_{0} \operatorname{Maps}_{*}\left(S^{1}, X\right) \times \pi_{0} \operatorname{Maps}_{*}\left(S^{1}, Y\right)\right)$. Further, the above map is clearly a bijection, so it induces an injection and a surjection on $\pi_{0}$.

(c) The Hurewicz map is natural so we have a commutative diagram

\begin{center}
\includegraphics[max width=\textwidth]{2023_10_29_d1ae8b5466f5f6dcce3cg-17}
\end{center}

where the vertical maps are abelianizations by the Hurewicz theorem. But the lower-right corner is equal to $H_{1}(X) \times H_{1}(Y)$ by the Kunneth theorem (since $X$ and $Y$ are connected), and the bottom copy of $f_{*}$ is the obvious isomorphism on $H_{1}$. Since $q$ is an abelianization by definition, but the bottom arrow and rightmost arrow are both isomorphisms, the top arrow must also be an abelianization.

\begin{enumerate}
  \setcounter{enumi}{4}
  \item (Differential Geometry) (a) Let $\mathbb{S}^{1}=\mathbb{R} / \mathbb{Z}$ be a circle and consider the connection
\end{enumerate}

$$
\nabla:=\mathrm{d}+\pi \sqrt{-1} \mathrm{~d} \theta
$$

defined on the trivial complex line bundle over $\mathbb{S}^{1}$, where $\theta$ is the standard coordinate on $\mathbb{S}^{1}=\mathbb{R} / \mathbb{Z}$ descended from $\mathbb{R}$. By solving the differential equation for flat sections $f(\theta)$

$$
\nabla f=\mathrm{d} f+\pi \sqrt{-1} f \mathrm{~d} \theta=0
$$

or otherwise, show that there does not exist global flat sections with respect to $\nabla$ over $\mathbb{S}^{1}$.

(b) Let $T=V / \Lambda$ be a torus, where $\Lambda$ is a lattice and $V=\Lambda \otimes \mathbb{R}$ is the real vector space containing $\Lambda$. Let $L$ be the trivial complex line bundle equipped with the standard Hermitian metric. By identifying flat $U(1)$ connections with $U(1)$ representations of the fundamental group $\pi_{1}(T)$ or otherwise, show that the space of flat unitary connections on $L$ is the dual torus $T^{*}=V^{*} / \Lambda^{*}$, where $\Lambda^{*}:=\operatorname{Hom}(\Lambda, \mathbb{Z})$ is the dual lattice and $V^{*}:=\operatorname{Hom}(V, \mathbb{R})$ is the dual vector space.

Solution. (a) The differential equation

$$
f^{\prime}(\theta)+\pi \sqrt{-1} f(\theta)=0
$$

has a unique solution

$$
f(\theta)=A \mathrm{e}^{-\pi \sqrt{-1} \theta}
$$

up to a constant $A \in \mathbb{C}$. This is not a well-defined function over $\mathbb{S}^{1}$ because $f(0) \neq f(1)$.

(b) The space of flat $G$-connections over $T$ can be identified as

$$
\operatorname{Hom}\left(\pi_{1}(T), G\right) / \operatorname{Ad} G
$$

Since $\pi_{1}(T)=\Lambda$ and for the abelian group $G=U(1)$ the adjoint action is trivial, we have

$$
\operatorname{Hom}\left(\pi_{1}(T), G\right) / \operatorname{Ad} G=\operatorname{Hom}(\Lambda, U(1))=T^{*}
$$

\begin{enumerate}
  \setcounter{enumi}{5}
  \item (Real Analysis) (Fundamental Solutions of Linear Partial Differential Equations with Constant Coefficients). Let $\Omega$ be an open interval $(-M, M)$ in $\mathbb{R}$ with $M>0$. Let $n$ be a positive integer and $L=\sum_{\nu=0}^{n} a_{\nu} \frac{d^{\nu}}{d x^{\nu}}$ be a linear differential operator of order $n$ on $\mathbb{R}$ with constant coefficients, where the coefficients $a_{0}, \cdots, a_{n-1}, a_{n} \neq 0$ are complex numbers and $x$ is the coordinate of $\mathbb{R}$. Let $L^{*}=\sum_{\nu=0}^{n}(-1)^{\nu} \overline{a_{\nu}} \frac{d^{\nu}}{d x^{\nu}}$. Prove, by using Plancherel's identity, that there exists a constant $c>0$ which depends only on $M$ and $a_{n}$ and is independent of $a_{0}, a_{1}, \cdots, a_{n-1}$ such that for any $f \in L^{2}(\Omega)$ a weak solution $u$ of $L u=f$ exists with $\|u\|_{L^{2}(\Omega)} \leq c\|f\|_{L^{2}(\Omega)}$. Give one explicit expression for $c$ as a function of $M$ and $a_{n}$.
\end{enumerate}

Hint: A weak solution $u$ of $L u=f$ means that $(f, \psi)_{L^{2}(\Omega)}=\left(u, L^{*} \psi\right)_{L^{2}(\Omega)}$ for every infinitely differentiable function $\psi$ on $\Omega$ with compact support. For the solution of this problem you can consider as known and given the following three statements.

(I) If there exists a positive number $c>0$ such that $\|\psi\|_{L^{2}(\Omega)} \leq c\left\|L^{*} \psi\right\|_{L^{2}(\Omega)}$ for all infinitely differentiable complex-valued functions $\psi$ on $\Omega$ with compact support, then for any $f \in L^{2}(\Omega)$ a weak solution $u$ of $L u=f$ exists with $\|u\|_{L^{2}(\Omega)} \leq c\|f\|_{L^{2}(\Omega)}$.

(II) Let $P(z)=z^{m}+\sum_{k=0}^{m-1} b_{k} z^{k}$ be a polynomial with leading coefficient 1. If $F$ is a holomorphic function on $\mathbb{C}$, then

$$
|F(0)|^{2} \leq \frac{1}{2 \pi} \int_{\theta=0}^{2 \pi}\left|P\left(e^{i \theta}\right) F\left(e^{i \theta}\right)\right|^{2} d \theta
$$

(III) For an $L^{2}$ function $f$ on $\mathbb{R}$ which is zero outside $\Omega=(-M, M)$ its Fourier transform

$$
\hat{f}(\xi)=\int_{-M}^{M} f(x) e^{-2 \pi i x \xi} d x
$$

as a function of $\xi \in \mathbb{R}$ can be extended to a holomorphic function

$$
\hat{f}(\xi+i \eta)=\int_{-M}^{M} f(x) e^{-2 \pi i x(\xi+i \eta)} d x
$$

on $\mathbb{C}$ as a function of $\xi+i \eta$.

Solution. This problem is to compute the constant $c$ in Lemma 3.3 on p.225 of the book of Stein and Shakarchi on Real Analysis by going over its arguments and keeping track of the constants involved in each step.

Introduce the polynomial

$$
Q(\zeta)=\sum_{k=0}^{n}(-1)^{k} \overline{a_{k}}(2 \pi \zeta)^{k}
$$

so that

$$
\left(\widehat{L^{*} \psi}\right)(\zeta)=Q(\zeta) \hat{\psi}(\zeta)
$$

any $\psi \in \mathcal{C}_{0}^{\infty}(\mathbb{R})$, where ${ }^{-}$denotes taking the Fourier transform. Consider first the special case where $a_{n}=\frac{1}{(2 \pi i)^{n}}$ so that the coefficient of $\xi^{n}$ in the polynomial $Q(\zeta)$ of degree $n$ in $\zeta$ is 1 . Writing $\zeta=\xi+\sqrt{-1} \eta$ (with both $\xi$ and $\eta$ real) and taking the $L^{2}$ of both sides of (\#) over $\mathbb{R}$ as functions of $\eta$. Then

$$
\int_{-\infty}^{\infty}|Q(\xi+i \eta) \hat{\psi}(\xi+i \eta)|^{2} d \xi=\int_{-\infty}^{\infty}\left|\left(\widehat{L^{*} \psi}\right)(\xi+i \eta)\right|^{2} d \xi
$$

Since from the definition of Fourier transform

$$
\left(\widehat{L^{*} \psi}\right)(\xi+i \eta)=\int_{x=-\infty}^{\infty}\left(L^{*} \psi\right)(x) e^{-2 \pi i(\xi+i \eta) x} d x=\int_{x=-\infty}^{\infty}\left(\left(L^{*} \psi\right)(x) e^{2 \pi \eta x}\right) e^{-2 \pi i \xi x} d x
$$

it follows that $\left(\widehat{L^{*} \psi}\right)(\xi+i \eta)$ is equal to the value at $\xi$ of the Fourier transform of the function $\left(L^{*} \psi\right)(x) e^{2 \pi \eta x}$. Thus, by applying Plancherel's identity to the function $\left(L^{*} \psi\right)(x) e^{2 \pi \eta x}$, we get

$$
\begin{gathered}
\int_{\xi=-\infty}^{\infty}\left|\left(\widehat{L^{*} \psi}\right)(\xi+i \eta)\right|^{2} d \xi \\
=\int_{x=-\infty}^{\infty}\left|\left(L^{*} \psi\right)(x) e^{2 \pi \eta x}\right|^{2} d x \leq e^{4 \pi|\eta| M} \int_{-\infty}^{\infty}\left|\left(L^{*} \psi\right)(x)\right|^{2} d x
\end{gathered}
$$

because the support of $\psi(x)$ (as well as the support of $\left(L^{*} \psi\right)(x)$ ) is in the interval $\Omega=(-M, M)$. Thus from (b) it follows that

$$
\int_{-\infty}^{\infty}|Q(\xi+i \eta) \hat{\psi}(\xi+i \eta)|^{2} d \xi \leq e^{4 \pi|\eta| M} \int_{-\infty}^{\infty}\left|\left(L^{*} \psi\right)(x)\right|^{2} d x
$$

Setting $\eta=\sin \theta$ in $(\sharp)$, we get from $|\eta| \leq 1$ that

$$
\int_{-\infty}^{\infty}|Q(\xi+i \sin \theta) \hat{\psi}(\xi+i \sin \theta)|^{2} d \xi \leq e^{4 \pi M} \int_{-\infty}^{\infty}\left|\left(L^{*} \psi\right)(x)\right|^{2} d x
$$

Replacing $\xi$ by $\xi+\cos \theta$ in the integrand on the left-hand side of $(\dagger)$, we get

$$
\begin{gathered}
\int_{-\infty}^{\infty}|Q(\xi+\cos \theta+i \sin \theta) \hat{\psi}(\xi+\cos \theta+i \sin \theta)|^{2} d \xi \\
\leq e^{4 \pi M} \int_{-\infty}^{\infty}\left|\left(L^{*} \psi\right)(x)\right|^{2} d x
\end{gathered}
$$

By Statement (III) given above the function $\hat{\psi}(\xi+i \eta)$ as a function of $\xi+i \eta \in$ $\mathbb{C}$ is holomorphic on $\mathbb{C}$. Since $Q(\xi+i \eta)$ as a function of $\xi+i \eta \in \mathbb{C}$ is a polynomial of degree $n$ with leading coefficient 1, it follows from Statement (II) applied to $F(z)=\hat{\psi}(\xi+z)$ and $P(z)=Q(\xi+z)$ that

$$
|\hat{\psi}(\xi)|^{2} \leq \frac{1}{2 \pi} \int_{\theta=0}^{2 \pi}|Q(\xi+\cos \theta+i \sin \theta) \hat{\psi}(\xi+\cos \theta+i \sin \theta)|^{2} d \theta
$$

Integrating both sides over $\xi \in(-\infty, \infty)$ and using $(\ddagger)$, we get

$$
\begin{aligned}
& \int_{\xi=-\infty}^{\infty}|\hat{\psi}(\xi)|^{2} \leq \int_{\xi=-\infty}^{\infty}\left(\frac{1}{2 \pi} \int_{\theta=0}^{2 \pi}|Q(\xi+\cos \theta+i \sin \theta) \hat{\psi}(\xi+\cos \theta+i \sin \theta)|^{2} d \theta\right) d \xi \\
& \quad=\frac{1}{2 \pi} \int_{\theta=0}^{2 \pi}\left(\int_{\xi=-\infty}^{\infty}|Q(\xi+\cos \theta+i \sin \theta) \hat{\psi}(\xi+\cos \theta+i \sin \theta)|^{2} d \xi\right) d \theta \\
& \quad \leq \frac{1}{2 \pi} \int_{\theta=0}^{2 \pi}\left(e^{4 \pi M} \int_{-\infty}^{\infty}\left|\left(L^{*} \psi\right)(x)\right|^{2} d x\right) d \theta=e^{4 \pi M} \int_{-\infty}^{\infty}\left|\left(L^{*} \psi\right)(x)\right|^{2} d x
\end{aligned}
$$

By applying Plancherel's formula to $\psi$, we conclude that

$$
\|\psi(\xi)\|_{L^{2}(\Omega)}^{2} \leq e^{4 \pi M}\left\|\left(L^{*} \psi\right)(x)\right\|_{L^{2}(\Omega)}^{2}
$$

under the additional assumption that $a_{n}=\frac{1}{(2 \pi i)^{n}}$. When this additional assumption is not satisfied, we can apply the argument for the special case to

$$
\frac{1}{a_{n}(2 \pi i)^{n}} L
$$

instead of to $L$ to conclude that

$$
\|\psi(\xi)\|_{L^{2}(\Omega)}^{2} \leq \frac{e^{4 \pi M}}{\left|a_{n}(2 \pi)^{n}\right|^{2}}\left\|\left(L^{*} \psi\right)(x)\right\|_{L^{2}(\Omega)}^{2}
$$

or

$$
\|\psi(\xi)\|_{L^{2}(\Omega)} \leq c\left\|\left(L^{*} \psi\right)(x)\right\|_{L^{2}(\Omega)}
$$

with

$$
c=\frac{e^{2 \pi M}}{\left|a_{n}\right|(2 \pi)^{n}}
$$

By Statement (I) given above, when we set

$$
c=\frac{e^{2 \pi M}}{\left|a_{n}\right|(2 \pi)^{n}}
$$

we can conclude that for any $f \in L^{2}(\Omega)$ a weak solution $u$ of $L u=f$ exists with $\|u\|_{L^{2}(\Omega)} \leq c\|f\|_{L^{2}(\Omega)}$.


\end{document}