\documentclass[10pt]{article}
\usepackage[utf8]{inputenc}
\usepackage[T1]{fontenc}
\usepackage{amsmath}
\usepackage{amsfonts}
\usepackage{amssymb}
\usepackage[version=4]{mhchem}
\usepackage{stmaryrd}
\usepackage{bbold}

\title{QUALIFYING EXAMINATION }


\author{HARVARD UNIVERSITY}
\date{}


\begin{document}
\maketitle
Department of Mathematics

Tuesday August 31, 2021 (Day 1)

\begin{enumerate}
  \item (A) Let $G$ be a finite group, let $V$ be a representation of $G$ on a finitedimensional vector space over $\mathbb{C}$, and let $W \subset V$ be a subrepresentation. Show that there is a subrepresentation $W^{\prime} \subset V$ such that
\end{enumerate}

$$
V=W \oplus W^{\prime}
$$

Solution: Choose any complementary subspace $U \subset V$ with $V=W \oplus U$, and let

$$
\pi: V \longrightarrow W
$$

be the corresponding projection onto the first component. Define a new linear map $\pi^{\prime}: V \longrightarrow W$ by averaging $\pi$ over the group $G$-that is,

$$
\pi^{\prime}(v):=\frac{1}{|G|} \sum_{g \in G} g \pi\left(g^{-1} v\right)
$$

This is a $G$-equivariant map such that, for any $w \in W$,

$$
\pi^{\prime}(w)=w
$$

Its kernel $W^{\prime}:=\operatorname{ker} \pi^{\prime}$ is $G$-invariant, of complementary dimension to $W$, and has the property that $W \cap W^{\prime}=0$. Therefore

$$
V=W \oplus W^{\prime}
$$

\begin{enumerate}
  \setcounter{enumi}{1}
  \item (AG) Consider the varieties in the affine plane $\mathbb{A}_{\mathbb{C}}^{2}$ with coordinates $(x, y)$ defined by the following polynomials:

  \item $X_{1}=V\left(x^{2}-1\right)$

  \item $X_{2}=V\left(x^{2}-y\right)$

  \item $X_{3}=V\left(x^{2}-y^{2}\right)$

  \item $X_{4}=V\left(x^{2}-y^{3}\right)$

  \item $X_{5}=V\left(x^{2}-y^{4}\right)$.

\end{enumerate}

Prove that no two of the varieties $X_{i}$ are isomorphic. (Note: we are not adopting the convention that varieties are assumed irreducible.)

Solution: First off, the varieties $X_{2}$ and $X_{4}$ are irreducible, whereas the other three are reducible. Since $X_{2}$ is nonsingular and $X_{4}$ is singular, they are not isomorphic to each other.

Among the varieties $X_{1}, X_{3}$ and $X_{5}$, the first is nonsingular whereas the other two are singular. And finally, in the case of $X_{3}$, the intersection of the two irreducible components is transverse, while in $X_{5}$ the two irreducible components are tangent at their point of intersection.

\begin{enumerate}
  \setcounter{enumi}{2}
  \item (AT) Let $D^{n}$ be a closed disc in $\mathbb{R}^{n}$ and $S^{n-1}=\partial D^{n}$ its boundary. For any topological space $X$ and map $\alpha: S^{n-1} \rightarrow X$, we define the space $Y$ obtained from $X$ by attaching an n-cell via the map $\alpha$ to be the quotient of the disjoint union $D^{n} \sqcup X$ by the equivalence relation generated by $p \sim \alpha(p)$ for all $p \in \partial D^{n}$. Assuming that the Betti numbers of $X$ are finite, show that one of the two following statements holds:

  \item the $n$th Betti number of $Y$ is 1 greater than the $n$th Betti number of $X$, and all other Betti numbers are equal; or

  \item the $(n-1)$ st Betti number of $Y$ is 1 less than the $(n-1)$ st Betti number of $X$, and all other Betti numbers are equal.

\end{enumerate}

Solution: We consider the covering of $Y$ by the two open sets $U$ and $V$, where $U=Y \backslash\{0\}$ is the complement in $Y$ of the image of the origin $0 \in D^{n}$, and $V$ is the image in $Y$ of the open disc $D^{n} \backslash S^{n-1}$. Here $V$ is contractible, so its reduced homology is 0 , and $V$ may be retracted back to $X$, so its reduced homology is the same as that of $X$. Finally, the intersection $U \cap V$ has the homotopy type of $S^{n-1}$, so its reduced homology is $\mathbb{Z}$ in degree $n-1$ and 0 otherwise. The relevant part of the Mayer-Vietoris sequence is thus

$$
0 \rightarrow H_{n}(X) \rightarrow H_{n}(Y) \rightarrow H_{n-1}\left(S^{n-1}\right) \cong \mathbb{Z} \rightarrow^{\alpha_{*}} H_{n-1}(X) \rightarrow H_{n-1}(Y) \rightarrow 0
$$

If the rank of the map $\alpha_{*}$ is zero - that is, if the image in $H_{n-1}(X)$ of the fundamental class of $S^{n-1}$ is torsion-then the first statement holds; if the rank of $\alpha_{*}$ is 1 , the second holds.

\begin{enumerate}
  \setcounter{enumi}{3}
  \item (CA) Evaluate the series
\end{enumerate}

$$
\sum_{n=-\infty}^{\infty} \frac{n^{2}+n+1}{n^{4}+1}
$$

by integrating $R(z) \cot \pi z$ for some appropriate rational function $R(z)$ over the boundary of the square $C_{n} \subset \mathbb{C}$ whose four vertices are $\left(n+\frac{1}{2}\right)( \pm 1 \pm i)$ and then letting $n \rightarrow \infty$.

Solution Since

$$
\cot \pi z=i \frac{e^{i \pi z}+e^{-i \pi z}}{e^{i \pi z}-e^{-i \pi z}}=i \frac{e^{-\pi y} e^{i \pi x}+e^{\pi y} e^{-i \pi x}}{e^{-\pi y} e^{i \pi x}-e^{\pi y} e^{-i \pi x}}
$$

by looking at $y \rightarrow \infty$ and $y \rightarrow-\infty$ separately, we conclude from

$$
\cot \pi(z+2)=\cot \pi z
$$

that $\cot \pi z$ is uniformly bounded on $C_{n}$ (independent of $n$ ). Let

$$
f(z)=\frac{z^{2}+z+1}{z^{4}+1} \pi \cot \pi z
$$

From

$$
\lim _{n \rightarrow \infty} \sup _{z \in C_{n}} \frac{z^{2}+z+1}{z^{4}+1}=O\left(\frac{1}{n^{2}}\right)
$$

and the length of $C_{n}$ of order $O(n)$, it follows that

$$
\lim _{n \rightarrow \infty} \int_{C_{n}} f(z) d z=0
$$

and the sum of the residues of $f(z)$ on $\mathbb{C}$ vanishes. The poles of $f$ are all simple poles and are at $z \in \mathbb{Z}$ and the four roots $e^{\frac{i k \pi}{4}}(k=1,3,5,7)$ of $z^{4}+1=0$. The residue at $z=n$ is $\frac{n^{2}+n+1}{n^{4}+1}$ and the residue at $e^{\frac{i k \pi}{4}}$ is

$$
\left(\frac{z^{2}+z+1}{4 z^{3}} \pi \cot \pi z\right)_{z=e^{\frac{i k \pi}{4}}} .
$$

The sum of the four residues at $e^{\frac{i k \pi}{4}}(k=1,3,5,7)$ is

$$
-\frac{i \pi}{\sqrt{2}}\left(\cot \left(\pi e^{\frac{i \pi}{4}}\right)+\cot \left(\pi e^{\frac{3 i \pi}{4}}\right)\right)
$$

Thus,

$$
\sum_{n=-\infty}^{\infty} \frac{n^{2}+n+1}{n^{4}+1}=\frac{i \pi}{\sqrt{2}}\left(\cot \left(\pi e^{\frac{i \pi}{4}}\right)+\cot \left(\pi e^{\frac{3 i \pi}{4}}\right)\right)
$$

\begin{enumerate}
  \setcounter{enumi}{4}
  \item (DG) Let $c>0$. Consider the catenary $C$ defined by
\end{enumerate}

$$
x=c \cosh \left(\frac{z}{c}\right)
$$

in the $x z$-plane. Let $S$ be the catenoid in the $x y z$-space obtained by rotating the catenary $C$ with respect to the $z$-axis. Use $\theta, z$ as coordinates for $S$, where $\theta$ is from the polar coordinates $(r, \theta)$ of the $x y$-plane. In terms of $(\theta, z)$, write down the first and second fundamental forms of $S$ and the mean curvature and Gaussian curvature of $S$.

Solution: The parametric equations for $S$ are

$$
\begin{aligned}
x & =c \cosh \left(\frac{z}{c}\right) \cos \theta \\
y & =c \cosh \left(\frac{z}{c}\right) \sin \theta \\
z & =z
\end{aligned}
$$

The first fundamental form $I=E d \theta^{2}+2 F d \theta d z+G d z^{2}$ is

$$
\begin{aligned}
d s^{2}= & d x^{2}+d y^{2}+d z^{2} \\
= & \left(-c \cosh \left(\frac{z}{c}\right) \sin \theta d \theta+\sinh \left(\frac{z}{c}\right) \cos \theta d z\right)^{2} \\
& +\left(c \cosh \left(\frac{z}{c}\right) \cos \theta d \theta+\sinh \left(\frac{z}{c}\right) \sin \theta d z\right)^{2}+d z^{2} \\
= & c^{2} \cosh ^{2}\left(\frac{z}{c}\right) d \theta^{2}+\cosh ^{2}\left(\frac{z}{c}\right) d z^{2}
\end{aligned}
$$

with

$$
\begin{aligned}
E & =c^{2} \cosh ^{2}\left(\frac{z}{c}\right), \\
F & =0 \\
G & =\cosh ^{2}\left(\frac{z}{c}\right) .
\end{aligned}
$$

To compute the unit normal vector $\vec{n}$, we compute the partial derivatives of the radius vector $\vec{r}$ with respect $\theta$ and $z$,

$$
\begin{aligned}
\vec{r}_{\theta} & =\left(-c \cosh \left(\frac{z}{c}\right) \sin \theta, c \cosh \left(\frac{z}{c}\right) \cos \theta, 0\right) \\
\vec{r}_{z} & =\left(\sinh \left(\frac{z}{c}\right) \cos \theta, \sinh \left(\frac{z}{c}\right) \sin \theta, 1\right)
\end{aligned}
$$

to form

$$
\vec{r}_{\theta} \times \vec{r}_{z}=\left(c \cosh \left(\frac{z}{c}\right) \cos \theta, c \cosh \left(\frac{z}{c}\right) \sin \theta,-c \sinh \left(\frac{z}{c}\right) \cosh \left(\frac{z}{c}\right)\right) .
$$

The length of $\vec{r}_{\theta} \times \vec{r}_{z}$ is equal to $\sqrt{E G-F^{2}}=c \cosh ^{2}\left(\frac{z}{c}\right)$ so that

$$
\vec{n}=\left(\cosh \left(\frac{z}{c}\right)^{-1} \cos \theta, \cosh \left(\frac{z}{c}\right)^{-1} \sin \theta,-\sinh \left(\frac{z}{c}\right) \cosh \left(\frac{z}{c}\right)^{-1}\right)
$$

To obtain the coefficients $L, M, N$ of the second fundamental form $I I=L d z^{2}+$ $2 M d z d \theta+N d \theta^{2}$, we compute the partial derivatives of the radius vector $\vec{r}$,

$$
\begin{aligned}
\vec{r}_{\theta \theta} & =\left(-c \cosh \left(\frac{z}{c}\right) \cos \theta,-c \cosh \left(\frac{z}{c}\right) \sin \theta, 0\right), \\
\vec{r}_{\theta z} & =\left(-\sinh \left(\frac{z}{c}\right) \sin \theta, \sinh \left(\frac{z}{c}\right) \cos \theta, 0\right), \\
\vec{r}_{z z} & =\left(\frac{1}{c} \cosh \left(\frac{z}{c}\right) \cos \theta, \frac{1}{c} \cosh \left(\frac{z}{c}\right) \sin \theta, 0\right) .
\end{aligned}
$$

The coefficients $L, M, N$ of the second fundamental form are given by

$$
\begin{aligned}
L & =\vec{n} \cdot \vec{r}_{\theta \theta}=-c \\
M & =\vec{n} \cdot r_{\theta z}=0 \\
N & =\vec{n} \cdot \vec{r}_{z z}=\frac{1}{c}
\end{aligned}
$$

The mean curvature of $S$ is

$$
\frac{1}{2} \frac{L G-2 M F+N E}{E G-F^{2}}=\frac{1}{2} \frac{(-c) \cosh ^{2}\left(\frac{z}{c}\right)+\frac{1}{c} c^{2} \cosh ^{2}\left(\frac{z}{c}\right)}{c^{2} \cosh ^{2}\left(\frac{z}{c}\right) \cosh ^{2}\left(\frac{z}{c}\right)}=0 .
$$

The Gaussian curvature of $S$ is

$$
\frac{L N-M^{2}}{E G-F^{2}}=\frac{(-c) \frac{1}{c}}{c^{2} \cosh ^{2}\left(\frac{z}{c}\right) \cosh ^{2}\left(\frac{z}{c}\right)}=\frac{-1}{c^{2} \cosh ^{4}\left(\frac{z}{c}\right)}
$$

\begin{enumerate}
  \setcounter{enumi}{5}
  \item (RA) Suppose $f:[-1,1] \rightarrow \mathbf{R}$ is a continuous function such that
\end{enumerate}

$$
\int_{-1}^{1} x^{2 n} f(x) d x=0
$$

for each $n=0,1,2,3, \ldots$ Prove that $f$ is an odd function (i.e., that $f(-x)=$ $-f(x)$ for all $x \in[-1,1])$.

Solution: Let $g:[-1,1] \rightarrow \mathbf{R}$ be the continuous function defined by $g(x)=$ $f(x)+f(-x)$. We prove that $g$ is the zero function, which is equivalent to the desired $f(-x)=-f(x)$.

First note that $\int_{-1}^{1} x^{m} g(x) d x=0$ for each $m=0,1,2,3, \ldots$; this is automatic for $m$ odd, and follows from the hypothesis for $m$ even. By linearity it follows that $\int_{-1}^{1} P(x) g(x) d x=0$ for all polynomials $P$. By the Weierstrass approximation theorem there exists a sequence $\left\{P_{k}\right\}_{k=1}^{\infty}$ of polynomials such that $P_{k}(x) \rightarrow g(x)$ uniformly for all $x \in[-1,1]$. Since $g$ is bounded (continuous function on a compact set), it follows that

$$
\int_{-1}^{1} g(x)^{2} d x=\lim _{k \rightarrow \infty} \int_{-1}^{1} P(x) g(x) d x
$$

Hence $\int_{-1}^{1} g(x)^{2} d x=0$. Since $g$ is continuous and real valued, it thus vanishes identically, and we are done.

\section*{QUALIFYING EXAMINATION }
Department of Mathematics

Wednesday September 1, 2021 (Day 2)

\begin{enumerate}
  \item (AT)
\end{enumerate}

(a) Let $X$ and $Y$ be compact, connected, oriented n-manifolds, and $f: X \rightarrow$ $Y$ a continuous map. Define the degree of the map $f$.

(b) Let $S^{n}$ be the unit sphere in $\mathbb{R}^{n+1}$, and let $r_{i}: S^{n} \rightarrow S^{n}$ be the reflection in the $i$ th axis; that is, the map

$$
\left(x_{0}, \ldots, x_{n}\right) \mapsto\left(x_{0}, \ldots, x_{i-1},-x_{i}, x_{i+1}, \ldots, x_{n}\right)
$$

What is the degree of $r_{i}$ ?

(c) Let $S^{n}$ be the unit sphere in $\mathbb{R}^{n+1}$, and let $a: S^{n} \rightarrow S^{n}$ be the antipodal map sending $x$ to $-x$. What is the degree of $a$ ?

Solution: For the first part, by hypothesis we have $H_{n}(X)=H_{n}(Y) \cong \mathbb{Z}$, where we choose the identification so that the generator $1 \in \mathbb{Z}$ corresponds to the fundamental class given by the orientation. The map $f_{*}: H_{n}(X) \rightarrow$ $H_{n}(Y)$ is then simply multiplication by an integer $d$; the degree of the map is defined to be this integer $d$.

As for the second part, the reflection $r_{i}$ is an orientation-reversing automorphism of $S^{n}$, so its degree is -1 . And for the third part, we note that $a$ is the composition of the $n+1$ reflections $r_{0}, \ldots, r_{n}$, so its degree is $(-1)^{n+1}$.

\begin{enumerate}
  \setcounter{enumi}{1}
  \item (CA) Suppose that $f:\{z: 0<|z|<1\} \rightarrow \mathbb{C}$ is holomorphic and $|f(z)| \leq$ $A|z|^{-3 / 2}$ for some constant $A$. Prove that there is a complex constant $\alpha$ such that $g(z):=f(z)-\alpha z^{-1}$ can be extended to a holomorphic function on $\{z:|z|<1\}$.
\end{enumerate}

Solution: For $0<a<|z|<b<1$, we can write

$$
2 \pi i f(z)=\int_{|w|=b} \frac{f(w)}{z-w} \mathrm{~d} w-\int_{|w|=a} \frac{f(w)}{z-w} \mathrm{~d} w
$$

Notice that

$$
\int_{|w|=a} \frac{f(w)}{z-w} \mathrm{~d} w=\frac{1}{z} \int_{|w|=a} f(w) \mathrm{d} w-\frac{1}{z} \int_{|w|=a} O(w / z) f(w) \mathrm{d} w
$$

By assumption, the last term can be estimated by

$$
\frac{1}{|z|} \int_{|w|=a} O(w / z)|f(w)||\mathrm{d} w| \leq \frac{A}{|z|^{2}} \sqrt{a}
$$

As $a \rightarrow 0$, the last term vanishes. Thus we have

$$
2 \pi i f(z)+\frac{c}{z}=\int_{|w|=b} \frac{f(w)}{z-w} \mathrm{~d} w, \quad c=\int_{|w|=a} f(w) \mathrm{d} w
$$

Notice that $c$ is independent of the choice of $a$. The right hand side defines a holomorphic function near $z=0$.

\begin{enumerate}
  \setcounter{enumi}{2}
  \item (DG) Which of the following smooth manifolds:

  \item $S^{2}$,

  \item $\mathbb{R P}^{2}$ and

  \item $S^{1} \times S^{1}$

\end{enumerate}

admit a closed, non-exact differential 1-form? In each case, either argue why such form does not exist or give an example.

Solution: By deRham's theorem, if $M$ is a manifold, then the real-valued singular cohomology groups $H^{*}(M, \mathbb{R})$ are isomorphic to the cohomology of the complex of differential forms. Thus, it follows that $M$ admits a closed, non-exact differential 1-form if and only if $H^{1}(M, \mathbb{R}) \neq 0$.

Since $H^{1}\left(S^{2}, \mathbb{R}\right)=0$ and $H^{1}\left(\mathbb{R P}^{2}, \mathbb{R}\right)=0$, these are no such forms in these cases.

In the last case, we have $H^{1}\left(S^{1} \times S^{1}, \mathbb{R}\right) \simeq \mathbb{R} \oplus \mathbb{R}$, so such forms exist. As an explicit example, let us choose a diffeomorphism

$$
(\theta, \rho): S^{1} \times S^{1} \rightarrow \mathbb{R} / \mathbb{Z} \times \mathbb{R} / \mathbb{Z}
$$

Then, $\theta$ can be considered as a real-valued function, well-defined up to adding an integral constant, so that $d \theta$ is a well-defined differential 1-form on $S^{1} \times S^{1}$. By definition, $d \theta$ is locally a differential of a real-valued function, so it is closed. On the other hand, it is not exact, as its integral around the loop corresponding to $\mathbb{R} / \mathbb{Z} \times\{e\}$ is not zero.

\begin{enumerate}
  \setcounter{enumi}{3}
  \item (RA) Let $\mathbf{T}$ be the torus $(\mathbf{R} / \mathbf{Z})^{2}$, and let $a: \mathbf{T} \rightarrow \mathbf{R}$ be any continuous function. Prove that the $\mathbf{R}$-vector space of solutions of the partial differential equation
\end{enumerate}

$$
\frac{\partial^{2} f}{\partial x^{2}}+\frac{\partial^{2} f}{\partial y^{2}}=a f
$$

in functions $f: \mathbf{T} \rightarrow \mathbf{R}$ is finite dimensional.

Solution: Call that vector space $V$, and write the differential equation as $(1-a) f=(1-\Delta) f$ where $\Delta$ is the Laplacian $\partial^{2} / \partial x^{2}+\partial^{2} / \partial y^{2}$. Let $A$ be the operator $(1-\Delta)^{-1}$ on $L^{2}(\mathbf{T})$, which is compact because it is diagonalized by the Fourier basis with eigenvalues $\left(1+4 \pi^{2}\left(m^{2}+n^{2}\right)\right)^{-1}$, only finitely many of which are outside $(0, \epsilon)$ for any $\epsilon>0$. Then $V$ is the fixed subspace of $A(1-a)$, which is also compact (composition of the compact operator $A$ with the bounded operator $1-a$ ). Hence $V$ is finite dimensional (for example, because the closure of its unit ball is compact), Q.E.D.

\begin{enumerate}
  \setcounter{enumi}{4}
  \item (A) Consider the polynomial $f(x)=x^{4}+1$.
\end{enumerate}

(a) Prove that the Galois group $G$ of $f$ over $\mathbb{Q}$ has order 4 .

(b) Show that $G$ is in fact isomorphic to the group $\mathbb{Z} / 2 \mathbb{Z} \times \mathbb{Z} / 2 \mathbb{Z}$.

(c) Is there any prime $p>2$ such that $f$ is irreducible over the finite field of order $p$ ?

Solution: Let $\alpha \in \mathbb{C}$ be a root of $f$. Then the full set of roots of $f$ is given by

$$
\{ \pm \alpha, \pm i \alpha\}
$$

Since $\alpha^{2}= \pm i$, it follows that $\mathbb{Q}[\alpha]$ is the splitting field of $f$ over $\mathbb{Q}$, and we have

$$
|G|=[\mathbb{Q}[\alpha]: \mathbb{Q}]=4
$$

For the second part, note that the Galois group $G$ acts transitively on the roots of $f$, so it contains elements $\sigma$ and $\tau$ such that

$$
\sigma(\alpha)=-\alpha \quad \text { and } \quad \tau(\alpha)=\alpha^{3}
$$

Then

$$
\begin{aligned}
& \sigma^{2}(\alpha)=\alpha, \text { and } \\
& \tau^{2}(\alpha)=\alpha^{9}=(-1)^{2} \cdot \alpha=\alpha
\end{aligned}
$$

Since $G$ is a group of order 4 which contains two elements of order 2 , it must be isomorphic to $\mathbb{Z} / 2 \mathbb{Z} \times \mathbb{Z} / 2 \mathbb{Z}$.

Finally, the arguments in parts (a) and (b) show that, if $\mathbb{F}$ is a field of characteristic not equal to 2 or 3 over which $f$ is irreducible, the Galois group of $f$ over $\mathbb{F}$ is isomorphic to $\mathbb{Z} / 2 \mathbb{Z} \times \mathbb{Z} / 2 \mathbb{Z}$. However, the Galois group of any finite extension of $\mathbb{F}_{p}$ is cyclic. Therefore, $f$ cannot be irreducible over $\mathbb{F}_{p}$.

Alternatively, we could argue that the cyclic group $\mathbb{F}_{p^{2}}^{\times}$has order $p^{2}-1$, which is always congruent to $0(\bmod 8)$. This implies that there is an element $\alpha \in \mathbb{F}_{p^{2}}^{\times}$ of order 8 . Then $\alpha^{4}=-1$, so $\alpha$ is a root of the polynomial $f(x) \in \mathbb{F}_{p}[x]$.

Suppose that $f(x)$ is irreducible over $\mathbb{F}_{p}$. Then $\mathbb{F}_{p}(\alpha)$ is the splitting field of $f$ over $\mathbb{F}_{p}$ and it has degree 4 . We get

$$
2=\left[\mathbb{F}_{p^{2}}: \mathbb{F}_{p}\right]=\left[\mathbb{F}_{p^{2}}: \mathbb{F}_{p}(\alpha)\right]\left[\mathbb{F}_{p}(\alpha): \mathbb{F}_{p}\right]=\left[\mathbb{F}_{p^{2}}: \mathbb{F}_{p}(\alpha)\right] \cdot 4
$$

-a contradiction!

\begin{enumerate}
  \setcounter{enumi}{5}
  \item (AG) Let $C \subset \mathbb{P}^{3}$ be a smooth, irreducible, nondegenerate curve of degree 4 .
\end{enumerate}

(a) If the genus of $C$ is 0 , show that $C$ is contained in a quadric surface.

(b) If the genus of $C$ is 1 , show that $C$ is equal to the intersection of two quadric surfaces.

(c) Show that the genus of $C$ cannot be greater than 1 .

Solution: For both (a) and (b), the key is to look at the restriction map $\rho: H^{0}\left(\mathcal{O}_{\mathbb{P}^{3}}(2)\right) \rightarrow H^{0}\left(\mathcal{O}_{C}(2)\right)$.

In either case, we know that $h^{0}\left(\mathcal{O}_{\mathbb{P}^{3}}(2)\right)=10$. As for $\mathcal{O}_{C}(2)$, this is a line bundle/invertible sheaf of degree $2 \times 4=8$ on $C$; if the genus of $C$ is 0 , then by Riemann-Roch we have $h^{0}\left(\mathcal{O}_{C}(2)\right)=9$, and so the map $\rho$ must have a kernel; thus $C$ lies on a quadric. Similarly, if the genus of $C$ is 1 , we have $h^{0}\left(\mathcal{O}_{C}(2)\right)=8$, so $C$ must lie on at least two linearly independent quadrics $Q$ and $Q^{\prime}$. Since $C$ is nondegenerate, these quadrics must be irreducible, and so by Bezout we must have $Q \cap Q^{\prime}=C$.

There are many ways to do part (c); probably the simplest would be to argue that for a general point $p \in C$, the projection map $\pi_{p}: C \rightarrow \mathbb{P}^{2}$ maps $C$ birationally onto a plane cubic curve, which will either have genus 1 (if it's smooth) or 0 (if it's singular).

\section*{QUALIFYING EXAMINATION }
Department of Mathematics

Thursday September 2, 2021 (Day 3)

\begin{enumerate}
  \item (DG) Let $a_{i j}$ for $1 \leq i \leq n-1$ and $1 \leq j \leq n$ be real constants. For $1 \leq i \leq n-1$ consider the vector field
\end{enumerate}

$$
X_{i}=(\underbrace{0, \cdots, 0,1,0 \cdots, 0}_{1 \text { in } i^{\text {th }} \text { position }}, \sum_{j=1}^{n} a_{i j} x_{j})
$$

on $\mathbb{R}^{n}$ (with coordinates $x_{1}, \cdots, x_{n}$ ). Let $\Pi$ be the distribution of the tangent subspace of dimension $n-1$ in $\mathbb{R}^{n}$ spanned by $X_{1}, \cdots, X_{n-1}$. Determine the necessary and sufficient condition for $\Pi$ to be integrable. Express the condition in terms of symmetry properties of the $(n-1) \times(n-1)$ matrix $\left(a_{i j}\right)_{1<i, j \leq n-1}$ and the relation among the ratios $\frac{a_{i k}}{a_{j k}}$ for $1 \leq i<j \leq n-1$ and $1 \leq k \leq n$.

Solution: Write

$$
X_{i}=\frac{\partial}{\partial x_{i}}+\left(\sum_{j=1}^{n} a_{i j} x_{j}\right) \frac{\partial}{\partial x_{n}}
$$

for $1 \leq i \leq n-1$. By Frobenius theorem, integrability of $\Pi$ is equivalent to $\left[X_{i}, X_{j}\right]$ being spanned by $X_{1}, \cdots, X_{n-1}$ for $1 \leq i<j \leq n-1$. Since

$$
\left[X_{i}, X_{j}\right]=\left(a_{j i}+\left(\sum_{k=1}^{n} a_{i k} x_{k}\right) a_{j n}-a_{i j}-\left(\sum_{k=1}^{n} a_{j k} x_{k}\right) a_{i n}\right) \frac{\partial}{\partial x_{n}}
$$

has zero coefficients for $\frac{\partial}{\partial x_{k}}$ for $1 \leq k \leq n-1$, the integrability condition can be rewritten as the vanishing of $\left[X_{i}, X_{j}\right]$ for $1 \leq i<j \leq n-1$, which means

$$
a_{j i}+\left(\sum_{k=1}^{n} a_{i k} x_{k}\right) a_{j n}=a_{i j}+\left(\sum_{k=1}^{n} a_{j k} x_{k}\right) a_{i n}
$$

Equating the coefficients, we obtain $a_{j i}=a_{i j}$ and $a_{i k} a_{j n}=a_{j k} a_{i n}$ for $1 \leq i<$ $j \leq n-1$ and $1 \leq k \leq n$. The necessary and sufficient condition is that the $(n-1) \times(n-1)$ matrix $\left(a_{i j}\right)_{1 \leq i, j \leq n-1}$ is symmetric and for $1 \leq i<j \leq n-1$ the $n$ ratios $\frac{a_{i k}}{a_{j k}}$ for $1 \leq k \leq n$ are equal in the sense of equality after crossmultiplication.

\begin{enumerate}
  \setcounter{enumi}{1}
  \item (RA) Suppose $U$ and $V$ are two random variables. We say that $U$ and $V$ are uncorrelated if $\operatorname{Cov}(U, V)=\mathbb{E}[U V]-\mathbb{E}[U] \mathbb{E}[V]=0$.
\end{enumerate}

(a) Is it true that if $U$ and $V$ are uncorrelated, then $U$ and $V$ are independent? Prove it or give a counter example.

(b) Suppose $\mathrm{X}$ and $\mathrm{Y}$ are distributed by the following bivariate normal distribution with density

$$
f(x, y)=\frac{1}{2 \pi} \frac{1}{\sqrt{1-\rho^{2}}} e^{-\frac{x^{2}-2 \rho x y+y^{2}}{2\left(1-\rho^{2}\right)}}
$$

where $0<\rho<1$ is a parameter. Let $U=X+a Y$ and $V=X+b Y$ with $a, b \neq 0$. Find the condition that $\operatorname{Cov}(U, V)=0$. In this case, prove that $U$ and $V$ are independent (you cannot just cite a theorem) .

Solution: Define the matrix

$$
A^{-1}=\frac{1}{\left(1-\rho^{2}\right)}\left[\begin{array}{cc}
1 & -\rho \\
-\rho & 1
\end{array}\right], \quad A=\left[\begin{array}{cc}
1 & \rho \\
\rho & 1
\end{array}\right]
$$

and denote the column vectors $\mathbf{x}=(x, y)^{t}$ and $\mathbf{s}=(s, t)^{t}$. Then the characteristic function

$$
\phi(s, t)=\mathbb{E} e^{i s X+i t Y}=\frac{1}{2 \pi} \frac{1}{\sqrt{1-\rho^{2}}} \int e^{i \mathbf{s} \cdot \mathbf{x}} e^{-\mathbf{x}^{t} A^{-1} \mathbf{x} / 2} \mathrm{~d} x \mathrm{~d} y=e^{-\mathbf{s}^{t} A \mathbf{s} / 2}
$$

This gives $\operatorname{Cov}(X, X)=\operatorname{Cov}(Y, Y)=1$ and $\operatorname{Cov}(X, Y)=\rho$. Now the characteristic function of $U, V$ can be computed from $\phi(s, t)$, i.e.,

$$
\mathbb{E} e^{i \alpha U+i \beta V}=\phi(\alpha+\beta, a \alpha+b \beta)
$$

The condition $\operatorname{Cov}(U, V)=0$ will imply that $\mathbb{E} e^{i \alpha U+i \beta V}=\mathbb{E} e^{i \alpha U} \mathbb{E} e^{i \beta V}$ and hence $U, V$ are independent.

\begin{enumerate}
  \setcounter{enumi}{2}
  \item (A) Suppose $R$ is a commutative ring with unit, $I$ an ideal in $R$, and $M$ a finitely-generated $R$-module. If $I M=M$, prove that there exists $r \in R$ such that $r-1 \in I$ and $r M=0$.
\end{enumerate}

Solution: Let $M$ be generated by $x_{1}, \ldots, x_{n}$. Then $I M$ consists of module elements of the form $\sum_{j=1}^{n} a_{j} x_{j}$ with each $a_{j} \in I$. Thus $M=I M$ means that each $x_{i}$ can be written as $\sum_{j=1}^{n} a_{i j} x_{j}$ for some $a_{i j} \in I$. Let $A$ be the $n \times n$ matrix $\left(a_{i j}\right)$, and $\vec{x}$ the column vector $\left(x_{i}\right)$; then we have $(\mathbf{1}-A) \vec{x}=0$. Multiplying from the left by $\operatorname{adj} A$, we $\operatorname{deduce}$ that $\operatorname{det}(\mathbf{1}-A) \cdot \vec{x}=0$, and
thus that $\operatorname{det}(\mathbf{1}-A) \cdot M=0$. But the ring element $\operatorname{det}(\mathbf{1}-A)$ is in $1+I$ because $\mathbf{1}-A \equiv \mathbf{1} \bmod I$. $\diamond$

Remark. This is not just a random trick: the result is (depending on preferred terminology) either Nakayama's lemma or a key step in the proof of Nakayama's lemma.

\begin{enumerate}
  \setcounter{enumi}{3}
  \item (AG) Let $\mathbb{P}^{n^{2}-1}$ be the variety of nonzero $n \times n$ complex matrices modulo scalars. Consider the set
\end{enumerate}

$$
X:=\left\{[A] \in \mathbb{P}^{n^{2}-1} \mid A \text { is nilpotent }\right\} .
$$

(a) Show that $X$ is a closed subvariety of $\mathbb{P}^{n^{2}-1}$.

(b) Show that $X$ is irreducible, and find its dimension.

Solution: For the first part, observe that for any $n$-by- $n$ matrix $A$, the characteristic polynomial of $A$ is

$$
\operatorname{char}_{A}(T)=T^{n}+p_{1}(A) T^{n-1}+\ldots+p_{n-1}(A) T+p_{n}(A)
$$

where the coefficients $p_{1}(A), \ldots, p_{n}(A)$ are homogeneous polynomials in the entries of $A$. The matrix $A$ is nilpotent if and only if $A^{n}=0$. In other words, the variety $X$ is cut out by the vanishing of the homogeneous polynomials $p_{1}, \ldots, p_{n}$.

For the second, let $\mathcal{F}$ be the variety of complete flags in $\mathbb{C}^{n}$-that is, let $\operatorname{Gr}(k, n)$ be the Grassmannian of $k$-dimensional subspaces of $\mathbb{C}^{n}$ and let

$$
\mathcal{F}:=\left\{V_{\bullet}=\left(V_{0}, V_{1}, \ldots, V_{n}\right) \mid V_{k} \in \operatorname{Gr}(k, n) \text { and } V_{k} \subset V_{k+1}\right\}
$$

Note that

$$
\operatorname{dim} \mathcal{F}=\frac{n(n-1)}{2}
$$

Define an incidence variety

$$
\Lambda:=\left\{\left(A, V_{\bullet}\right) \in X \times \mathcal{F} \mid A \cdot V_{\bullet} \subset V_{\bullet}\right\}
$$

which consists of pairs of a nilpotent element $A$ and a flag $V_{\bullet}$ such that $A$ preserves $V$. The fiber over the standard flag $E_{\bullet}$ defined by

$$
E_{k}=\left\{\left(x_{1}, \ldots, x_{k}, 0, \ldots, 0\right) \in \mathbb{C}^{n}\right\}
$$

consists exactly of the upper-triangular nilpotent matrices. Since any complete flag is conjugate to the standard flag, it follows that $\Lambda$ fibers over $\mathcal{F}$ with fiber the projective space of dimension

$$
\frac{n(n-1)}{2}-1
$$

Therefore $\Lambda$ is irreducible of dimension $n^{2}-n-1$.

The projection onto the first component

$$
\pi: \Lambda \longrightarrow X
$$

is surjective, because any nilpotent matrix is conjugate to an upper-triangular one and therefore stabilizes at least one flag. This implies that $X$ is irreducible.

Moreover, recall that any nilpotent matrix of rank $n-1$ is conjugate to the maximal nilpotent Jordan block, which stabilizes only the standard flag $E_{\bullet}$. Therefore $\pi$ is generically one-to-one, and it follows that

$$
\operatorname{dim} X=n^{2}-n-1
$$

\begin{enumerate}
  \setcounter{enumi}{4}
  \item (AT) Let $M$ be a connected closed 4-manifold such that $\pi_{1}(M)$ is perfect; that is, does not have any non-trivial abelian quotients. Determine the possible cohomology groups $H^{*}(M, \mathbb{Z})$.
\end{enumerate}

Solution: We first claim that $M$ is orientable. Let $p: \widetilde{M} \rightarrow M$ be the orientation cover of $M$. As $M$ is connected, this is completely classified as a covering by any fibre $p^{-1}(m)$ together with the action of $\pi_{1}(M, m)$.

As the orientation covering is 2-fold, this is the same as a homomorphism $\pi_{1}(M, m) \rightarrow \Sigma_{2} \simeq \mathbb{Z} / 2$. It this was non-trivial, then it would be surjective, which is impossible since $\pi_{1}(M)$ is assumed to be perfect. Thus, we deduce that the orientation covering is the trivial 2 -fold covering, so that $M$ is orientable.

We will now determine the possible homology groups. Orientability tells us that $H^{4}(M, \mathbb{Z}) \simeq \mathbb{Z}$ and similarly $H_{4}(M, \mathbb{Z}) \simeq \mathbb{Z}$.

By Hurewicz theorem, we have $H_{1}(M, \mathbb{Z}) \simeq \pi_{1}(M)^{a b}$. By the perfectness assumption, the latter vanishes, and hence so does the former.

By universal coefficient theorem combined with vanishing of $H_{1}$, we deduce that

$$
H^{2}(M, \mathbb{Z}) \simeq H o m\left(H_{2}(M, \mathbb{Z}), \mathbb{Z}\right)
$$

The latter group is torsion-free, and we deduce that $H^{2}(M, \mathbb{Z})$ is a torsion free abelian group, hence finite free rank as $M$ is compact. The Poincare duality isomorphism $H_{2}(M, \mathbb{Z}) \simeq H^{2}(M, \mathbb{Z})$ allows us to deduce that the second homology group is also free of finite rank.

Similarly, we have Poincare isomorphism $H_{3}(M, \mathbb{Z}) \simeq H^{1}(M, \mathbb{Z})$ and a universal coefficient isomorphism $H^{1}(M, \mathbb{Z}) \simeq \operatorname{Hom}\left(H_{1}(M, \mathbb{Z}), \mathbb{Z}\right)$. The last group vanishes, and we deduce the same is true for the third homology groups.

These shows that the possible homology groups of $M$ are respectively

$$
\mathbb{Z}, 0, A, 0, \mathbb{Z}
$$

where $A$ is free of finite rank. All of these can be realized by a connected sum of complex projective planes.

\begin{enumerate}
  \setcounter{enumi}{5}
  \item (CA) Let $a<b$ and $f(z)$ be a continuous function on the closed strip $\{a \leq$ $x \leq b\}$ which is holomorphic on its interior $\{a<x<b\}$, where $z=x+i y$, such that $|f(z)|=O\left(e^{\varepsilon|y|}\right)$ on $\{a \leq x \leq b\}$ for every $\varepsilon>0$ as $|y| \rightarrow \infty$. If $|f(z)| \leq M$ on the boundary $\{x=a$ or $x=b\}$ of the strip $\{a \leq x \leq b\}$ and on the interval $[a, b]$ for some positive number $M$, prove that $|f(z)| \leq M$ on the entire closed strip $\{a \leq x \leq b\}$.
\end{enumerate}

Hint: Consider

$$
g_{\varepsilon}(z)=e^{\varepsilon i z} f(z) \quad \text { and } h_{\varepsilon}(z)=e^{-\varepsilon i z} f(z) .
$$

Solution: Fix arbitrarily $\varepsilon>0$. Let $C_{\varepsilon}>0$ such that $|f(x+i y)| \leq C_{\varepsilon} e^{\frac{\varepsilon}{2}|y|}$ on $\{a \leq x \leq b\}$ for any $y \in \mathbb{R}$. Since

$$
\left|g_{\varepsilon}(a+i y)\right|=e^{-\varepsilon y}|f(a+i y)| \leq e^{-\varepsilon y} C_{\varepsilon} e^{\frac{\varepsilon}{2} y} \leq M
$$

and

$$
\left|g_{\varepsilon}(b+i y)\right|=e^{-\varepsilon y}|f(b+i y)| \leq e^{-\varepsilon y} C_{\varepsilon} e^{\frac{\varepsilon}{2} y} \leq M
$$

when $y \geq T_{\varepsilon}$ for some sufficiently large positive number $T_{\varepsilon}$. By the maximum modulus principle applied to $g_{\varepsilon}(z)$ on the rectangle with vertices

$$
a, b, b+i T, a+i T
$$

when $T \geq T_{\varepsilon}$, we conclude that $\left|g_{\varepsilon}(z)\right| \leq M$ on the half strip

$$
\{a \leq x \leq b\} \cap\{y \geq 0\}
$$

Passing to limit as $\varepsilon \rightarrow 0^{+}$, we obtain $|f(z)| \leq M$ on the half strip

$$
\{a \leq x \leq b\} \cap\{y \geq 0\}
$$

Repeat the same argument with $g_{\varepsilon}(z)$ replaced by $h_{\varepsilon}(z)$ and with the condition $y \geq T_{\varepsilon}$ replaced by $y \leq-S_{\varepsilon}$ for some sufficiently large positive number $S_{\varepsilon}$. Analogously we get the conclusion that $|f(z)| \leq M$ on the half strip

$$
\{a \leq x \leq b\} \cap\{y \leq 0\} .
$$


\end{document}