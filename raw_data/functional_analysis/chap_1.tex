TOPOLOGICAL VECTOR SPACES Introduction 1.1 Many problems that analysts study arc not primarily concerned with a single object such as a function, a measure, or an operator, but they deal instead with large classes of such objects. Most of the interesting classes that occur in this way turn out to be vector spaces, either with real scalars or with complcx oncs. Since limit processes play a role in every analytic problem (explicitly or implicitly), it should be no surprise that these vector spaces are supplied with metrics, or at least with topologies, that bear some natural relation to the objects of which the spaccs are made up. The simplest and most important way of doing this is to introduce a norm. The resulting structure (defined below) is called a normed vector space, or a normed linear space, or simply a normed space. Throughout this book, the term vector space will refer to a vector space over the complex field $\textstyle{\mathcal{C}}$ or over the real feld R. For the sake of completeness, detailed definitions are given in Section 1.4GENERAL THEORY 1.2 Normed spaces A vector space $\textstyle X$ is said to be a normed space if to every x,in such xe Xthere is associated a nonnegative real number lxl| alled the norm of x $X,$ a way that (a) $\|x+y\|\leq\|x\|+\|y\|$ if $x\in{\mathcal{X}}$ and $\textstyle{\mathcal{A}}$ is a scalar, in $X,$ (c) l $|x||>0$ if $x\neq0.$ for all x nd $\boldsymbol{y}$ (b) $\|\alpha x\|=|\alpha|\ \ \|x\|$ The word $\mathrm{\mathrm{}~^{*}n o r m}^{\prime}$ is also used to denote the function that maps $\scriptstyle{\mathcal{X}}$ to lxl Every normed space may be regarded as a metric space, in which the distance d( $\scriptstyle{\epsilon_{\circ,y}}$ betwccn $\mathbf{\Omega}\cdot{\boldsymbol{\Lambda}}$ and $\mathbf{y}$ is $\|x-y\|.$ The relevant properties of ${\boldsymbol{d}}$ are () $0\leq d(x,y)<\infty$ for all $\textstyle{\mathcal{X}}$ x and ${\mathit{y}},$ (i) $d(x,y)=0$ if and only if $x=y$ (iti) $d(x,\,y)=d(y,\,x)$ for all $\scriptstyle{\mathcal{X}}$ and ${\ {\mathit{y}}},$ (iD) $d(x,z)\leq d(x,y)+d(y,z)$ for all $X,y,\,z.$ In any metric space,the open ball with center at x and radius ${\mathbf{}}T$ is the set $$ B_{r}(x)=\{y\colon d(x,\,y)<r\}. $$ In particular, if T $\textstyle X$ is a normed space, the sets $$ B_{1}(0)=\{x:\|x\|<1\}\qquad{\mathrm{and}}\qquad{\bar{B}}_{1}(0)=\{x:\|x\|\leq1\} $$ are the open wnit ball and the closed uni ball o ${\mathcal{N}},$ respectively By declaring a subset of a metric space to be open if and only if itis a (possibly empty union of open balls, a topology is obtained.(See Section 1.5.)It is quite easy to verify that the vector space operations additon and scalar mutiplication) are continuous in this topology, if the metric is derived from a norm, as above A Banach space is a normed space which is complete in the metric defined by its norm; this means that every Cauchy sequence is required to converge 1.3 Many of the bes-known function spaces are Banach spaces. Let us mention just a few types: spaces of continuous functions on compact spaces; the familiar $L^{p}.$ -spaces that occur in integration theory; Hilbert spaces - the closest relatives o euclidean spaces; certain spaces of differentiable functions; spaces of continuous inear mappings from one Banach space into another; Banach algebras. All of these wil occur later on in the text But there are also many important spaces that do not ft into this famework. Here are some examples: euclidean space (a)C(QD), the space of ll continuous complex functions on some open set $\Omega$ 2 in a $\textstyle R^{n}$ $(b)\ \ H(\Omega),$ the space of all holomorphic functions in some open set Q in the complex plane.TOPoLOGICAL VECTOR SPACES (c) CR,the spaceofalinitel difetiable comple functons o with nonempty interior that vanish outside some fxed compact set $\textstyle R^{n}$ $\textstyle K$ (d) themselves The tet fnctonspaes usd theor o istiuton,a e distibution These spaces crry aural togistha cannot be induced by norms, as we shal elater Tey as ell te normed spaes aecxamies o oiaiec spaces, a concept that pervades all f functional alysis Aferthis riefatmt motivation he a tetaiedeifions olowe Gin Section 1.9) by a preview of some of the results of Chapter 1 ${\boldsymbol{R}}$ or 1.4 Vetor spaes Th lette $\boldsymbol{R}$ and $\textstyle{\mathcal{C}}$ wil alays denot e fel oreal umbers p stand for either $X,$ ${\boldsymbol{C}},$ andthe fld ofcomplex numbers respectively. For the moment, iet $\Phi$ is a set A scalar is a member of the scalar feld D.A nector space orer $\Phi$ whoe elensarcaled ctos and in which twopratins ditonana sca muliplcation, aredfined, withth following familiar agebraic propeitics (a)To every pair of vectors $\textstyle{\mathcal{X}}$ and $\mathbf{\nabla}y$ corresponds a vector $X+y,$ in such a way that x $$ \left.+\,y\,\equiv\,y\,+\,x\,\qquad\mathrm{and}\qquad x\,+\,\left(y\,+\,z\right)= (x\,+\,y\right)\,+\,{\mathrm{t.}}\,{\mathrm{i.}}\, $$ z; $\textstyle{\cal{X}}$ contains a unique vector ${\mathfrak{O}}$ (the zero rector or origin of X) such that $x+0=x$ for every $x\in X;$ and to each $x{\mathfrak{c}}{\mathfrak{c}}$ corresponds a unique vector $-{\mathfrak{X}}$ such that $x+(-x)=0.$ (b) To every par $\scriptstyle(x_{i},x)$ with $x\in\Phi$ and $x\in X$ corresponds a vector ${\mathcal{D}}X,$ in such a way that $$ 1x-x,\qquad\alpha(\beta x)=(\alpha\beta)x, $$ and such that the two distributive laws hold. $$ x(x+y)=\alpha x+\alpha y,\qquad(x+\beta)x=\alpha x+\beta x $$ The symbol ${\mathfrak{O}}$ will o ourse alo beused for the zero element o te scar fel_ $\mathbf{A}$ real tector space is one for which $\Phi=R;$ a complex tector space is one for If $\textstyle{\cal{X}}$ is a vector space, $A\subset X,\;B\subset X,\;x\in$ Any statement about vector spaces in which the scalar field is not $\lambda\in\Phi.$ the following notations which $\Phi=C$ explcitly mentioned isto be upderstoo o apply to both of these case will be used: X, and $$ \begin{array}{c}{{x+A=\{x+a;a\in A\},}}\\ {{x-A=\{x-a;\,a\in A\},}}\\ {{A+B=\{a+b;\,a\in A,\,\,b\in B\},}}\\ {{A+B=\{a+b;\,a\in A,\,\,b\in B\}.}}\end{array} $$6 GENERAL THEORY In particular (taking .= -1), -A denotes the set of all ditive inverses of members of $\textstyle A$ A word of warning: With Lhese conventions, it may happen that $2A\neq-A+A$ (Exercise 1). A set $Y\subset X$ is called a subspace of $X:\mathbb{F}\ Y$ is itself a vector space (with respect to the same operations, of course).One checks easily that this happens if and only if 0e ${\cal{Y}}$ and $$ \alpha Y+\beta Y c\ Y $$ for all scalars $\textstyle{\mathcal{Q}}$ and ${\boldsymbol{\beta}}.$ $\mathbf{A}$ set $C\subset{\mathcal{X}}$ is said to be convex if $$ t C+(1-t)C\subset C\qquad(0\leq t\leq1), $$ In other words, it is required that ${\boldsymbol{C}}$ should contain $t x+(1-t)y{\mathrm{~if~}}x\in C,y\in C,$ and O $\simeq t\leq1.$ $\mathrm{A}$ set $B\subset X$ is said to be balanced if c $\scriptstyle B\subset B$ for every $x\in\Phi$ with $|\alpha|\leq1.$ A vector space $\textstyle X$ has dimension $\;n$ (dim $X=n)$ if $\textstyle X$ has a basis $\{u_{1},\ldots,u_{n}\}.$ This means that every $X\in X$ has a unique representation of the form $$ x=\alpha_{1}u_{1}+\cdot\cdot\cdot+\alpha_{n}u_{n}\qquad(\alpha_{i}\in\Phi). $$ If dim $X=n$ for some ${\boldsymbol{n}}.$ X is said to have finite dimension、If $X=\{0\},$ then dim $X=0.$ Example If $X=C$ (a one-dimensional vector space over the scalar field C) the balanced sets are: ${\mathcal{C}},$ the empty set ${\mathcal{D}},$ ,and every circular disc (open or closed centered at O.If $X=R^{2}$ (a two-dimensional vector space over the scalar field $\textstyle R\backslash_{!}$ there are many more balanced sets; any line segment with midpoint at $\scriptstyle(0,\,0)$ wil o. The point is that in spite of the well-known and obvious idcntifica- tion of ${\mathcal{C}}$ with $R^{2}.$ these two are entirely different as far as thcir vector space structure is concerned $\mathbf{\hat{e}}_{\infty}$ 1.5 Topological spaces A topological space is a set $\mathbf{S}$ in which a collection t of subsets (called open sets) has been specified, with the following properties: $\boldsymbol{S}$ is open, C is open, the intersection of any two open sets is open, and the union of every collec tion of open sets is open. Such a collection tis called a topology on $\boldsymbol{\mathsf{S}}$ When clarity seems to demand it, the topological space corresponding to the topology r will b written (S, r) rather than ${\boldsymbol{S}}.$ Here is some of the standard vocabulary that will be used,if S and r are as above. A set $\scriptstyle{E\subset S}$ is closed if and onlyif its complement is open. The closure $\overline{{E}}$ E of $\boldsymbol{E}$ is the intersection of all closed sets that contain $E.$ The interior $E^{\circ}$ of E is the unionTOPoLoGICAL VECToR SPACEs of $\textstyle K$ that contains of all open sets that are subsets or $\textstyle{\boldsymbol{\sigma}}$ is thecoecio o llintesecton sct $K\hookrightarrow S$ is compac i evry open cover is any open se then o If l…olo $\overline{{\Gamma}}$ is ue y mti $\ d{\mathrm{~}}$ A neighborhood of a point $\rho\in S$ $V\in\tau,$ ama of a point $E\subset S$ are compatible with each other. $E.$ $\mathbf{A}$ is a base for f vey mber oi with mhsioms is a topology on $E,$ has a finite subcover. A collectio $\tau^{\prime}\subset\tau$ .5: . a4 ca an aisoiron sint A collction of neighborhood W”mn $d$ points o ${\boldsymbol{p}}.$ $\boldsymbol{\mathsf{S}}$ have disoineghborhoos $p\in{\mathcal{S}}$ tht is every open st s uno ofmers o ${\boldsymbol{\tau}}^{\prime}$ $E\cap V,$ ${\boldsymbol{E}}$ iand i i s eve igho minsme or as is iyv eie l thtoy ta Reseo v $x_{n}=x)$ A sequence $\scriptstyle\{x_{n}\}$ in a Hausdoff space $\textstyle X$ converges to a point xe $\textstyle X$ (or: lim $x_{n}$ it cv ehodo $\textstyle{\mathcal{X}}$ conasa init"imny epoin 1.6 Toloical etorsacesSups r olog vctor pc $\textstyle X$ such that (a)every point of $\textstyle X$ is a closed set, ad (の)he eo pce oeruis comusizh ect o topological vector space. Lner hsceonoions sid to bea etor oly o $X.$ ,and $\textstyle X$ is a Here is moe rcisewy o statng (o For ever $x\in N\colon$ , the set {xy which has x as its only member is a closed sct l manxtxi o it o deito ora ogcalvetor spac irossis isuseveisioms oacecomsrsie rsouso. tie typess sts ineie i esxmmieoem wi o t o nd o titmyit isissio To sytai socms aens em tateipn $$ (x,\,y)\to x+\,y $$ of the cartesian product $X\times X$ into $\textstyle X$ is continuous: If $x_{i}\in X$ for $i=1,$ 2, and if ${\mathbf{}}V$ is a neighborhood of $x_{1}+x_{2}\,,$ , there should cxst neighborhods $V_{i}$ of $X_{i}$ such that $$ \mathbf{\varepsilon}_{\circ}\cdot\quad V_{1}+V_{2}\subset V. $$ mapping inoul uheasumtionta caumuio ousmsuat $$ (x,x)\to\alpha x $$ of $\Phi\times X$ into $\textstyle X$ is continuous $\operatorname{If}x\in X,$ c is a scalar, and of ${\mathcal{X}}$ we have $\beta W<V$ whenever then for some $\scriptstyle r\gg0$ and some neighborhood ${\boldsymbol{V}}$ is a eighborhood of x $\left|\beta-\alpha\right|<r.$ $\mathcal{W}$8 GENERAL THEoRy hood A subse ${\boldsymbol{F}},$ of atological vectorsae said t be boudito ey nighbor for every $t>s$ ${\mathbf{}}V$ rof o in $\textstyle X$ corresponds a number $s>0$ such that $E\subset t V$ 1.7 Invariance Let $X$ be a topological vector space. Associate to each ae $X$ and to each scalar $\lambda{\mathcal{H}}^{0}$ the translation operator $\textstyle T_{a}$ and the multiplication operator $\begin{array}{l}{{M_{\lambda}}}\end{array}$ by the formulas $$ T_{a}(x)=a+x,\qquad M_{\lambda}(x)=\lambda x\qquad(x\in X). $$ The following simple proposition is very important Proposition $\textstyle T_{a}$ and $M_{\lambda}$ are homeomorphisms of $\textstyle X$ onto $X.$ that they map $\textstyle X$ Y onto X $X,$ PRoor The vector space axioms alone imply that $\textstyle T_{a}$ and $\varphi_{\lambda{\lambda}}$ are one-to-one. , and that their inverses are $T_{-x}$ and $\ M_{1/\lambda}\,,$ respectively. The assumed continuity of the vector space operations implies tat thesefor mappings are continuous. Hence each of them is a homeomorphism (a con tinuous mapping whosc inverse is also continuous). // invariant (or simply inariant,for brevity): One consequence of this proposition is that every vector topology ris translation- is open if and only if each of $\mathbf{A}$ set $E\subset X$ its translates $a+E$ is open. Thus ris completely determined by any local base In the vector space context, the term local base will always mean a local base of A mectric $\left(\begin{array}{l l}\end{array}\right)$ at O. A Iocal base of a topological vector space $\textstyle{\mathcal{N}}$ will be called invariant if is thus a collection W of neighbor- The open sets $\textstyle X$ $\textstyle X$ hoos of sch that every neighborhood ofo contains a member of ${\mathcal{Q}},$ ${\mathcal{R}},$ on a vector space are then precisely those that are unions of translates of members of $$ d(x+z,\;y+z)=d(x,\,y) $$ for all $x,y,$ z in $X.$ 1.8 Types of topological vector spaces In the following defnitions, ${\boldsymbol{X}}$ ’ always denotes a tological vector space, with topology r. (a)Xis locally conex if tere isa local base W whose members are convex (b)Xis locally ounded if O has a bounded neighborhood (c) $\textstyle X$ is locally compact if O has a neighborhood whose closure is compact. ${\mathit{d}}.$ (d) Xismetrizable if ris compatible with some metric d (e) $X$ is an F-space if its topology r is induced by a complcte invariant metric (Compare Section 1.25.) ( f) $\textstyle X$ is a Fréchet space if $\textstyle X$ is a locally convex $\textstyle F\cdot$ -space (g) Xis normable if a norm exists on $\textstyle X$ Y such that the metric induced by the norm is compatible with t.roroLGiCAL VECToR SPACEs 9 (i) $X$ (h)Normed spaces and Bunach spaces have already been defined (Section 1.2) $\textstyle X$ has the Heinc- Borel property if every closed and bounded subse o compact. ${\boldsymbol{F}}.$ The terminology of $\mathbf{\Psi}({\boldsymbol{e}})$ and $(f)$ is not universally agreed upon: In some texts, local convity s omtted fm he denion f aTFrchtspact weas tes s -space to describe what we havecalled Fréchet space space $X.$ 1.9 Here is is of ome rcatiosbewenthepropetis f a tologcal vecto 1.15]. (a) I Xis loally bounded, the $X$ has acountable loa as pat ce of Theore (6) $\textstyle X$ is metrizable if and only ir $\textstyle X$ is loally convex and lcally bounedCTheore (c) $\textstyle X$ is normable if and oniy i $X$ has a countable local base Theorem 1.240 1.39) d) $\textstyle X$ has finite dimension if and ony if $\textstyle X$ has the Heine-Boe property then $\textstyle X$ has finite e) lf lcally bounded spac $\textstyle X$ is loallycompact CTheorems 1.21,1.22 dimension (Theorem 1.23) false The spaces $H(\Omega)$ and ${C_{K}^{\omega}}$ mentioned in Section 1.3 are infinite-dimensiona Frche spaces iththe Heine-Bore pety Sectins .45,1.46. Thy areor not locally bounded. hence not normabe; they also show that the converse of Ga)is convex (Section 1.47). On the other hand, therexis locally bounded ${\boldsymbol{F}}{\boldsymbol{\cdot}}$ spacesta are not ocall Separation Properties 1.10 Theorem Suppose $\textstyle K$ anl ${\boldsymbol{C}}$ are subses of a topological vector space X, V such that $K$ is compact, ${\boldsymbol{C}}$ is closed, and ${\cal K}\,r\cap{\cal C}={\mathcal Q}.$ 5. Then $\mathbf{0}$ has a neighborhood ${\mathbf{}}V$ $$ (\mathbb{K}+V)\cap(\mathbb{C}+V)={\mathcal{D}}. $$ Note that $K+V\mathbf{i}$ is a union of translates $x+V$ of $V\left(x\in K\right)$ Thus $K+V$ is an that contain $\textstyle K$ and ${\cal{C}}_{,}$ open set that contains K. The theorem thus implies thexistence of disioint open set respectively contexts as well: rRoor We begin wih te following proposition, which wil b seful in other If Wis a neighborhood of O in $X,$ then there is a neighborhood $\scriptstyle U\,d f\,0$ which is symmetric (in the sense that $U=-U)$ and which satisfies $U+U\subset{\mathcal W}.$10 GENERAL THEORY To se this, note that $0+0=0,$ that addition is continuous, and that If $\mathbf{0}$ therefore has ncighborhoods $\gamma_{i},\,\nu_{2}$ such that $V_{1}+V_{2}\subset W.$ $$ U=V_{1}\cap_{\bf\gamma}V_{2}\cap(-V_{1})\cap(-V_{2}), $$ then $U$ has the required properties The proposition can now be applied to $U$ in place of $W$ and yields a new symmetric neighborhood $U_{\mathbf{\delta}}U$ of $\mathbf{0}$ such that $$ U+\,U+U-\,\mathcal{W}. $$ If It is clear how this can be continued $K\ +\ V=\mathcal{D},$ and the conclusion of the theorem is obvious $K=\varnothing,$ then We therefore assume that $K\neq\varnothing,$ and consider a point $x\in K,$ Since ${\boldsymbol{C}}$ 'is closed, since $\textstyle{\mathcal{X}}$ is not in ${\boldsymbol{C}}_{s}$ C, and since the topology of $X$ is invariant under translations, the preceding proposition shows that ( ${\boldsymbol{0}}$ ) has a symmetric neighborhood $V_{\boldsymbol{x}}$ such that $x+V_{x}+V_{x}+V_{x}$ does not intersect ${\mathsf{C}}\,;$ the symmetry of $V_{x}$ then shows that (1) $$ (x+\,V_{x}+\,V_{x})\,\cap\,(C+\,V_{x})=\,\mathcal{D} $$ Since ${\cal K}\,$ is compact, there are finitely many points $X_{1},\ \ x\ \cdot,$ $X_{n}$ in $\textstyle K$ such that $$ K\subset(x_{1}+\;V_{x_{1}})\;\cup\;\cdot\cdot\cdot\cup\;(x_{n}+\;V_{x_{n}}) $$ Put $V=V_{x_{i}}\cap\cdots\longleftrightarrow V_{x_{n}}$ Then $$ K+V\subset\bigcup_{i=1}^{n}(x_{i}+V_{x_{i}}+V)\subset\bigcup_{i=1}^{n}(x_{i}+V_{x_{i}}+V_{x_{i}}), $$ and no term in this last union intersects $C+V.$ by $(\,1\,).$ This completes the proof. $l/{\big/}{\big/}$ $C+V;$ Since $C+V$ is open, it is even true that the closure of $K+V$ does not intersect in particular, thc closure of $K+V$ does not intersect ${\boldsymbol{C}}$ The following special case of this, obtained by taking $\scriptstyle k\cdot\langle0\rangle$ is of considerable interest. 111 Thcorem f1M isa local base or a topologicalvecor space $X$ then every member of !% contains the closure of some member of !M So far we have not used the assumption that every point of $\textstyle X$ is a closed set We now use it and apply Theorem 1.10 to a pair of distinct points in place of ${\cal K}\,$ and ${\cal{C}}.$ The conclusion is that these points have disjoint neighborhoods. In other words,the Hausdorff separation axiom holds: 1.12 Theorem Every topological vector space is a Hausdorff space We now derive some simple properties of closures and interiors in a topological belongs to vector space. See Section 1.5 for the notations $\overline{{E}}$ and $E^{\circ}$ .Observe that a point $D\!\!\!\!/$ $\overline{{E}}$ if and only if every neighborhood of p intersects $E.$TrorouooucAL vrcrox srAcs 1 1.13 Theorem Lct $\textstyle X$ be a toloyial vcorspae (f) 1f ${\boldsymbol{E}}$ 1 Ae Xthen and $B\subset X,$ then ${\vec{A}}+{\vec{B}}<{\vec{A+B}}.$ rums througnall eghohods 0 is blanced (a) If A $\propto X$ $\vec{A}=\left(\begin{array}{c}{{\langle A+V\rangle,}}\end{array}\right.$ where ${\mathbf{}}V$ (b) (c) 1 Yis a subspace of $X,$ so is ${\overline{{Y}}}.$ and ${\cal C}^{\circ}\!\!.$ then $B^{\circ}$ (d) I C is a coex subset o $X,$ so are $\bar{C}$ if also $0\in B^{\circ}$ (e) I Bisa blaced subset $X,$ so is ${\overline{{B}}}\colon$ is a bounded subset of $X,$ so is ${\overline{{F}}}.$ of $0_{\mathrm{{s}}}$ PRoor(a) if ${\mathcal{x}}\neq0\,;$ if $\scriptstyle x\;=\;0$ . the s osyequl Hc to ${\boldsymbol{V}}.$ Since $-\,V\,\rfloor$ S $\alpha{\bar{Y}}=\alpha{\bar{Y}}$ (c) Suppose $\textstyle{\mathcal{Q}}$ and if ànd only if $(x+V)\cap A\neq\emptyset$ for every neighborhood ${\mathbf{}}V$ $x\in{\overline{{A}}}$ a neighborhood of and this happens if and ony i ;let $\bar{u}$ and $\boldsymbol{\partial}$ such that $\mathcal{W}_{1}+\mathcal{W}_{2}\subset\mathcal{W}$ There exist neighborhoods ${\mathcal{Y}}_{1}$ $\mathbf{0}$ if and only i $x\in A-V$ for every sc $a+b.$ There are (b) Take $a\in{\vec{A}},\ b\in{\vec{B}}.$ ${\mathcal{W}}$ ${\mathbf{}}V$ 'is one,the proof scople and ${\mathcal{N}}_{2}$ of be a neighborhood of $x\in A\cap\ M_{1}$ and $y\in B\cap W_{2}\,,$ $\beta$ are scalars. By the proposition in Seccion 17, $a+b\in{\overline{{A+B}}}.$ $(A\vdash B)\cap\ W,$ since $a\in{\overline{{A}}}$ and $b\in{\mathcal{B}}$ Then $x+y$ lies in so thtintetinsotempty.Consquien from (b) that $$ \cdot\bar{Y}_{+}\,\beta\bar{Y}=\vec{\alpha\,Y}\,+\,\bar{\beta\,Y}\,<\,\overline{{{\alpha\,Y}}}\,+\,\beta\qquad\qquad. $$ from $(d)$ the assumption that $(e).$ is a subspace was sed inth laticluso $\boldsymbol{\mathit{I}}$ $(d)$ Since $C^{\circ}<C$ and G ${\mathbf{C}}$ Toro oeeacewoesmemianeds and howeasonosos s sisiai sooi iwiasmi tne is convex, we have $$ t C^{\circ}+(1-t)C^{\circ}<C $$ if $0<t<1$ Te osone en raropn e s s er sm Sinc is convex . If $B^{\circ}$ For these every open subset of ${\boldsymbol{C}}$ is a subset of ${C}^{\circ},$ it follows that ${\boldsymbol{C}}^{\circ}$ $^{\alpha B^{\circ}\subset B^{\circ}}$ for somc ${\big/}{\big/}{\big/}{\big/}\qquad.$ Hence (e) If $0<|\alpha|\leq1,$ then ${\boldsymbol{B}}$ Bis banced Bat $\alpha B^{\circ}$ is open. So $W\subset V$ neighborhood ${\mathcal{W}}$ of O.Since ${\boldsymbol{E}}$ $\alpha B^{\circ}=(\alpha B)^{\circ},$ since $X\to$ cxis a homeomorphism. $({\mathcal{I}})$ ${\mathit{l}},$ $x B^{\circ}\subset\alpha B\subset B,$ since $x{\bar{B}}^{*}\subset B^{*}$ even for $\alpha=0.$ for all sfiently large t contains the origin, te hbe anighborood or O. By Theorem 1.11 $E\subset t W$ Let ${\mathbf{}}V$ we have $E\subset t W\subset t V.$ is bounded, 1.14 Theorem Lma tolylael eosu $X,$ 。 rewo, ousaulale iofo ma (の) uery ow eo omammezelody12 GENERAL THEORY PRoOF(a) Suppose $U_{\mathbf{\delta}}U$ is a neighborhood of $\mathbf{0}$ in $X.$ Since scalar multiplication is continuous, there is ${\mathfrak{a}}\;{\mathfrak{\mathfrak{a}}}>0$ and there is a neighborhood ${\mathbf{}}V$ of O in $\textstyle X$ such that $\alpha V\subset U$ whenever $|\alpha|<\delta$ Let $\mathcal{W}$ be the union of all these sets ${\mathcal{Q V}}.$ Then ${\mathcal{W}}$ is a neighborhood of O, ${\mathcal{W}}$ is balanced, and $W<U.$ (b) Suppose $U_{\mathbf{\delta}}U$ lis a convex neighborhood of O in $X.$ Let $A\,=\,\bigcap{\mathcal{A}}U,$ where implies that the interior α ranges over the scalars of absolute value 1. Choose $|\alpha|=1;$ hence $W<\alpha U.$ as in part (a). Since ${\mathcal{W}}$ is $W$ is balanced, an intersection of convex sets, $\scriptstyle A$ d is convex; hence so is $A^{\circ}$ Thus $W\subset{\mathcal{A}},$ which $A^{\circ}$ $\alpha^{-1}W=W$ when t s a neighborhood of O. Clearly $A^{\circ}\subset U.$ Being $A^{\circ}$ of $\textstyle{A}$ To prove that is a neighborhood with the desired properties, we have to show that ${\mathcal{A}}^{\prime}$ balanced; for this it sfices to prove that $\textstyle A$ l is balanced. Choose r and $\beta$ so that $0\leq r\leq1$ $|\beta|=1.$ Then $$ r\beta A=\bigcap_{|\alpha|=1}r\beta\alpha U=\bigcap_{|\alpha|=1}r\alpha U. $$ Since ${\mathfrak{x}}U$ is a convex sct that contains $0_{\mathrm{,}}$ we have $r x U\subset\alpha U.$ Thus $r\beta A\subset A,$ which completes the proof. // Theorem 1.4 can be restated in terms of local bases. Let us say that a local base W is balanced if its members are balanced sets, and let us call ${\mathcal{A}}$ convex if its members are convex sets Corollary (a) Every topological vector space has a balanced local base )Every locally comex space has abolanced covex local base Recalals that Thcorem 1.ll holds for each of these local base 1.15 Theorem Suppose ${\mathbf{}}V$ is a neighborhood of O in a topological vector space $X.$ (a) If 0 <八<r,<…and $r_{n}\to\infty$ as $n arrow\infty.$ , then $$ X=\bigcup_{n=1}^{\infty}r_{n}\,V. $$ (6) Every compact subset $\textstyle K$ of $\textstyle X$ is bounded , and if ${\mathbf{}}V$ is bounded, then the collection (c) ${\cal I}f\,\delta_{1}>\,\delta_{2}>\,\cdot\cdot\,a n d\,\delta_{n}\to0\ ,$ as $n arrow\infty.$ $$ \{\delta_{n}\,V;\,n=1,\,2,\,3,\,\ldots\} $$ is a local base for $X.$ PRoor(a) Fix xeX. Since α→αxis a continuous mapping of the scala hence contains field into X, the set of all c with xe V is open, contains $0,$ 1/r, for ll large n. Thus (l/r,)xe V, or xer,V, for large mToPoLoGICAL VECToR SPACEs 13 (b) Let ${\mathcal{W}}$ he a balanced neighborhood of ${\boldsymbol{0}}$ p such that $W\subset V.$ By (aの) $$ K\subset\bigcup_{n=1}^{\infty}n W. $$ Since $K\quad$ ris compact, the arc integcrs , $\scriptstyle n_{S}$ such that $$ K\subset n_{1}W\cup\cdot\cdot\cup\;n_{s}\;\dot{W}=n_{s}\,W. $$ $t/V\subset t\,V.$ The equality holds because $\mathcal{W}$ is balanced、 1f $t>n_{s}\,.$ ,it ollows tha $\kappa\,c$ $s>0$ (c) Let $U_{\mathit{l}}$ be $\bar{\mathbf{a}}$ neighborhood of $\mathbf{0}$ in $X.$ If ${\mathbf{}}V$ is bounded, there exists $\beta_{a}\nu$ such that $V\subset t U$ for all $t\geqslant s.$ If iso arge thal $s\delta_{n}<1,$ it follows that $V\subset(1/\delta_{n})U.$ Hence $U_{\mathbf{\delta}}U$ actul cotins al bt fitely many r tc se $J/{\big/}$ Linear Mappings 1.16 Definitons When $\textstyle X$ Y and $\boldsymbol{\mathit{I}}$ are sets,the symbol $$ f\colon X\to Y $$ willmen that fis a mappingor $X$ Y into ${\cal{Y}}.$ If $A\subset X$ and $B\subset Y,$ the image f(A)of A and te me rme -"(0))0 $\boldsymbol{B}$ are efine $$ f(A)=\{f(x);\,x\in A\},\qquad f^{-1}(B)=\{x;f(x)\in B\}. $$ ping Suppose now that $\textstyle X$ and $\boldsymbol{\mathit{I}}$ rare veto saces vr h asmsaf fil.A map $\Lambda\colon X\to Y\,\mathbf{i}$ is said to be linear if $$ \Lambda(\alpha x+\beta y)=\alpha\Lambda x+\beta\Lambda y $$ for all $\textstyle{\mathcal{X}}$ x and $\mathbf{\nabla}y$ in $\textstyle X$ ani a sca $\scriptstyle{\mathcal{X}}$ and ${\boldsymbol{\beta}}.$ Note tat e nwies x、 rate than $\Lambda(x)$ when $\Lambda$ is linear. Linear mappings of $X$ into safld ar aldina fmctiom of Section 1.7 are linear but the translation operators For xample themulicatnopcrat $\ M_{\alpha}$ whose proofs are so $T_{\alpha}$ are not, except when $a=0.$ $\operatorname{A}\colon X\to Y$ Here are some proprties oficer mpin and $B\subset Y\colon$ easy that we omithem it s ssucd th $d\subset X$ (a) AO = 0. (b)If $\scriptstyle A$ is a sbpae(or convexst or ane se nesm s treo $\Lambda(A)$ (c) If ${\boldsymbol{B}}$ Bisa spae a ove st oraneste smsitie $\Lambda^{-1}(B),$ (d)n particulat, the st $$ \Lambda^{-1}(\{0\})=\{x\in X\colon\Lambda x=0\}={\mathcal{W}}(\Lambda), $$ is a subspace of alld themul spce o A We now un onuypetienermpin14 GENERAL THEORY 1.17 Theorem Let $X$ and ${\cal{Y}}$ be topological vector spaces.I了人 $:X\to Y$ is linear and continuous at $\mathbf{0},$ then $\Lambda$ each neighborhood W o/O im ${\cal{Y}}$ is continuous.In fact, A is uniformly continuous, in the of O in following sense: $T_{O}$ corresponds a neighborhood ${\mathbf{}}V$ $\textstyle X$ such that $$ y-x\in V\,i m p l i e s\ \Lambda y-\Lambda x\in W. $$ some neighborhood ${\mathbf{}}V$ of ${\boldsymbol{0}}.$ is chosen, the continuity of $\Lambda$ at O shows that $\Lambda V<W$ for PROOFOnce ${\mathcal{W}}$ If now $y-x\in V,$ the linearity of $\Lambda$ shows that $\Lambda y-\Lambda x=\Lambda(y-x)\in W.$ Thus $\Lambda$ maps the neighborhood $x+y$ of xinto the preassigned neighborhood $\Lambda x+W$ of $\Lambda{\cal X}_{,}$ which says that $\Lambda$ is continuous at x. ${a\!\!\!/}{b\!\!\!/}{b\!\!\!/}$ 1.18 Theorem Let Λ be a linear functional on a topological vector space $X.$ Assume $\operatorname{A}\!x\neq0$ for some $x\in X;$ Then each of the following four properties implie the other three: (a)A is continuous (b) The null space ${\mathcal{N}}(\Lambda)$ is closed. (c) ${\mathcal{N}}(\Lambda)$ is not dense in $X.$ 0/ 0. (d)A is bounded in some neighborhoodl ${\mathbf{}}V$ pROOF Assume (a implies (b). By hypothesis, ${\mathcal{N}}(\Lambda)\neq X.$ Hence and {o} is a closed subset of the scalar feld implies (c) ${\mathcal{N}}(\Lambda)$ has non- ${\Phi}_{\!,}$ Since ${\mathcal W}(\Lambda)=\Lambda^{-1}(\{0\})$ empty interior. By Theorem 1.14, $\mathbf{\Psi}(b)$ $\left(c\right)$ holds; i., assume that the complemcnt of (1) $$ (x+V)\cap{\mathcal{M}}(\Lambda)={\mathcal{D}} $$ for some $x\in X$ and some balanced neighborhood $V_{\mathrm{of}}$ Then $\Lambda\,{\cal V}\,$ is a balanced subset of the field $\Phi.$ Thus either $\wedge V$ is bounded, in which case $(d)$ holds,or $\Lambda{\mathcal{V}}=\Phi$ In the latter case, there cxists $\scriptstyle y\in V$ such that $\Lambda y=\,-\,\Lambda x,$ and so $\xi,r>0$ and if $W=(r/M)V,$ in contradiction to (1). Thus (c) implies d) for all $\textstyle{\mathcal{X}}$ x in ${\mathbf{}}V$ and for some $M<\infty.$ ${\big/}{\big/}{\big/}{\big/}$ $x+y\in{\mathcal{N}}(\Lambda),$ $|\operatorname{A}x|<M$ for every $\textstyle{\mathcal{X}}$ in $W$ ’. Hence $\Lambda$ is con- Finally, if (d) holds then then |A $\textstyle|x|<r$ this implies (aJ. tinuous at the origin. By Theorem $1.{\boldsymbol{17}},$ Finite-dimensional Spaces 1.19 Among the simplest Banach spaces are $R^{n}$ and ${\mathcal{Q}}^{n},$ the standard ${\boldsymbol{R}}^{\ast}$ dimensional vector spaces over ${\boldsymbol{R}}$ and $\textstyle{\mathcal{C}},$ respectively, normed by means of thc usual uclidean metric: If, for example, $z=(z_{1},\ \cdot\cdot\cdot\cdot z_{n})\qquad(z_{i}\in C)$ToroLoGuCAL VECToR sPACEs 15 is a vector in ${\mathcal{C}}^{n},$ then $$ \|z\|=(|z_{1}|^{2}+\cdots+|z_{n}|^{2})^{1/2}. $$ Ouher norms can be defined on ${\d C}^{n}.$ For example $$ \|z\|=|z_{1}|\,+\cdots+|z_{n}|\qquad{\mathrm{or}}\qquad\|z\|=\operatorname*{max}\left(|z_{i}|:1\leq i\leq n\right). $$ is true: Thee om corsond, of ours, o difrntmetries $C^{n}$ (when $n>1)$ but one can vesiaty inice esmiougy o ${\mathcal{C}}^{n}.$ Actually more If $X$ induces an isomorphism of $\textstyle X$ is a toloical ector spaceover C,and dim ${\mathcal{C}}^{n}.$ Theorem $1.21$ willprove tht tisisomorphism ${\mathcal{C}}^{n}$ is the only $X$ complex ones. onto $X=n,$ then every basis of ……oh om. trors st etoy necto a n iesasisecimeoaauvome e A Ye sa s se titimisasestewavysos Fvrtin elenisi mnseweit rnasais pac we t w lm h ese homs . an . 1.20′Lemma Suppose ${\cal{Y}}$ is upece ( toloial eor spac $X,$ and $\boldsymbol{\mathit{I}}$ is loclbyoma n oymezio” $X,$ “Zmwsxa e $X.$ set PRoor There is a compact se ${\mathbf{}}V$ of o in $U$ of $\mathbf{\nabla}(\mathbf{\nabla})$ in $\textstyle X$ such that $U\cap V\subset K.$ ${\cal{Y}})$ contains O. Hence there s a nighborhoo $K\subset Y$ whose interior Grelative to Choose a symmetric neighborhood $\textstyle X$ such that $\overline{{{V}}}\,+\,\overline{{{V}}}\subseteq\,U.$ We claim that the (1) $$ Y\cap(x+{\overline{{V}}}) $$ is compact, for every $x\in X{\mathrm{:}}$ in (I), To see this, fix $y_{0}$ in (I). For any $\mathbf{\nabla}y$ $$ y\not{D-y_{0}}=(y-x)+(x-y_{0})\in\overline{{{y}}}\ _{+}\tilde{V}\subset U. $$ AIso, $y-y_{0}\in Y$ , since $\boldsymbol{\mathit{I}}$ is a subspace. Thus $$ \L_{\cdot}{\cdot}y\longrightarrow y_{0}\in U\subset Y\subset K, $$ $X.$ which implies tha since $x\ +{\overline{{V}}}$ is closed in $X$ and since $\boldsymbol{\mathit{I}}$ inheris its topology from $\textstyle X$ ' such that 0 ∈ W and $W<V,$ $\operatorname{\mathcal{(1)}}$ lies in the compact se $y_{o}+K.$ But $\operatorname{\mathcal{(1)}}$ is also a closed subset of ${\cal{Y}},$ is cs sbst o a cmpctset n ecompa the set rin Thus $\operatorname{\mathcal{}}(1)$ $\times\ x\in Y.$ Let Z be the collcton of all open sets $W\in{\mathcal{B}}$ $\mathcal{W}$ Now fi and associate with cach $$ {\cal E}_{W}=Y\cap(x+\bar{W}). $$16 GENERAL THEORY Since $W<V$ ,each $E_{W}$ is compact. Since $x\in Y,$ no $E_{W}$ is empty. Since inter- $W\in{\mathcal{B}}\}$ section ofinitely many members of W belong to ${\mathcal{B}}_{z}$ ,it follows that $(E_{\mathrm{w}}$ is a collection of compact sets with the finite intersection property. Therefore there exists $z\in\bigcap E_{W}.$ This z lies in ${\cal{Y}}.$ On the other hand, $z\in x+W$ for every $W\in{\mathcal{B}}$ Thus $z=x$ (Theorem 1.12).Hence $x\in\,Y.$ This proves that ${\bar{Y}}=Y,$ $J/f$ and so Yis closed. $X,$ 1.21 Thcorem Suppose $X$ is a complex topological tvector space, Y is a subspace of n is a positive integer, and dim $\textstyle Y=n$ .Then (a)every isomorphism of $Q^{n}$ onto Y is a homeomorphism, and (6) Y is closed. ${\mathcal{C}}^{n}$ The term “homeomorphism”refers, of course, to the euclidean topology of inherits from $\textstyle{X}$ on the other. Since ${\mathcal{C}}^{n}$ on the one hand, and to the topology that ${\cal{Y}}$ is ocally compact, Lemma 1.20 shows that (b) follows from $(\alpha).$ The proof that follows also ields the analogous theorem with real scalars in place of complex ones PROOF $\operatorname{Let}\,P_{n}$ be the theorem as stated. We first prove $\mathbf{\mathcal{P}}_{1}$ Let $\operatorname{A}\!:C\to Y$ be Then $\Lambda\mathcal{X}=\mathcal{\alpha}u.$ an isomorphism Gi.e., a one-to-one linear mapping of ${\boldsymbol{C}}$ onto ${\cal{Y}}\}$ . Put $u=\mathbb{A}|$ $\Lambda$ The continuity of the vector space operations in Yimplies that is continuous. Note that $\Lambda^{-\,1}$ is a linear functional on $\boldsymbol{\mathit{I}}$ with null space {O} a closed set. By Theorem $\mathsf{L.N s},$ $\Lambda^{-1}$ is continuous. This proves $P_{1}$ Assume next that $\scriptstyle n\;{\sim}\;1$ and $p_{n-1}$ is true. Let A: $C^{n}\to Y$ be an isomor- phism. ${\mathrm{Let~}}\{e_{1},\ldots,e_{n}\}$ be a basis of $\sqrt{\gamma}^{\gamma\gamma h}\ _{\bf\Pi}$ the kth coordinate of ${\mathcal{C}}_{k}$ is ${\mathfrak{I}}\,;$ the others are O. Put $u_{k}=\Lambda e_{k}\,.$ for $k=1,\dotsc,n.$ Then $$ \Lambda(\alpha_{1},\,\ldots,\,\alpha_{n})=\alpha_{1}u_{1}\,+\,\cdot\cdot\cdot\,+\,\alpha_{n}u_{n}, $$ and the continuity of the vector space operations in $\boldsymbol{\mathit{I}}$ implies again that $\Lambda$ is continuous. Since $\Lambda$ is an isomorphism, $\{u_{1},\ \cdot\cdot\ ,\ u_{n}\}$ is a basis of ${\boldsymbol{Y}},$ Hence there are linear functionals $\gamma_{1,}\ \cdot\cdot\cdot\ ,\ \gamma_{n}$ on $\boldsymbol{\mathit{I}}$ such that every $x\in Y$ has a unique representation of the form $$ x=\gamma_{1}(x)u_{1}\,+\,\cdot\,\cdot\,\cdot\,+\,\gamma_{n}(x)u_{n}\,. $$ Each y $\gamma_{i}$ assumed truth of $p_{s-1}.$ Hence $\gamma_{i}$ of dimcnsion $n-1,$ which is closed in $\textstyle Y,$ by the has a null space in ${\cal Y},$ is continuous, by Theorem 1.18. Since $$ \Lambda^{-1}x=(\gamma_{1}(x),\ldots,\gamma_{n}(x))\qquad(x\in Y), $$ it follows that $\Lambda^{-1}$ is continuous. Hence $\textstyle P_{n}$ is true, and the proof is complete $J/\slash$roroLoGcAL VECToR SPACEs 17 mension. 1-元 Theonm Ioy lo men ol co nv $X$ has finite di base for $X.$ rxoor The origin o $\textstyle{X}$ has a neihborhood ${\mathbf{}}V$ whose closure is compact. By form a local Theorem 15, ${\mathbf{}}V$ is bounded anie se $2^{-\,n}V\left(n=1,\,2,\,3,\,\ldots\right)$ The compactness of ${\cal{V}}$ / shows that there exis $X_{1},\,\cdot\,\cdot\,\cdot\,\cdot\,\times\,\prime\,X_{m}$ in $\textstyle{X}$ such that $$ {\overline{{V}}}\subset(x_{1}+{\textstyle{\frac{1}{2}}}V)\cup\cdot\cdot\cdot\cup(x_{m}+{\textstyle{\frac{1}{2}}}V). $$ Let $\boldsymbol{\mathit{I}}$ be he vector space spaned b ${\mathcal{X}}_{1},\,\ast\,\cdot\,,\,{\mathcal{X}}_{m}$ Then dim $Y\leq m.$ By Since Theorem 1.21, Yis a cose space and since $\lambda Y=Y|$ for every scalar ${\bar{\lambda}}\neq0_{s}$ it folows that $X.$ $V\subset\,Y+{\frac{1}{2}}V$ $$ .\qquad{\frac{1}{2}}{\mathcal{V}}_{C}\ C^{\prime}+{\frac{1}{4}}{\mathcal{V}}_{C}^{\prime} $$ so that $$ V\subset\,Y+{\textstyle{\frac{1}{2}}}\,V\subset\,Y+\,{\textstyle{\frac{1}{4}}}\,V=\,Y+{\textstyle{\frac{1}{4}}}\,V. $$ Ir we continue in this way, we e tha $$ V\subset_{n=1}^{\infty}(Y+2^{-n}V). $$ Since $\{2^{-n}V\}$ is ol ase noolwsfom G Theorm 1. a ${{\cal{Y}}}\,\mathrm{for}\,k\,=\,1,\,2,\,3,\,\ldots.$ // $V\subset{\bar{Y}}.$ But ${\bar{Y}}=Y.$ Thus $V\subset{\mathfrak{Y}}$ Y, which implies that kV。 $X\leq m.$ Hence $Y=X$ , hy do of Theorem 15, an onsequenty i 1.23 Theorem If $\textstyle X$ is。 allbowded oi etospe the e Borel poperty then X has mite imsi Statement PRoor By assumption, the origin o $\textstyle{\cal{X}}$ has a bounded neighborhood isco $V{\mathrm{:}}$ $(f)$ qf T hrem . sos th $\overline{{V}}$ is atosndea Tus $\overline{{V}}$ Bat…-on e-o poeg: s $\textstyle X$ Hocsalipae ne finitedimensiona, by Theorem 1.2 Metrization 中 $\textstyle X$ w resoo . n s $\textstyle X$ is s emnzefitec asmtr $\scriptstyle d$ 。 whisigmspwiaascs esmsu mer o ofous oxo i neessisnmoimiciawsisnir o logica et spacs,tns t to beasice18 cENERAL ruroRv 1.24 Theorem 1f $\textstyle{\mathcal{X}}$ is a topological vector space with a countable local base,then there is a mectric $\ d$ on $\textstyle X$ such that (a)d is compatible wih the topolgy of $X,$ (b)the open balls centered at O are balanced, and (c) d is invariant: $d(x+z,y+z)=d(x,y)\,j$ for $x,y,z\in X.$ and also ,in adition, Xis locally comvex, then $d$ can be chosen so as tosaisfy a),(6b),(e) (d)all open balls are convex. PRoOrBy Theorem $\scriptstyle{1.14}$ , X has a balanced local base $\langle V_{x}\rangle$ such that (1) $$ V_{n+1}+V_{n+1}\subset V_{n}\qquad(n=1,\,2,\,3,\,\ldots); $$ when ${\boldsymbol{X}}$ is locally convex, this local base can be chosen so that cach $V_{n}$ is also convex. Let ${\boldsymbol{D}}$ be the set of all rational numbers ${\mathbf{}}T$ of the form (2) $$ r=\sum_{n=1}^{\infty}c_{n}(r)2^{-n}, $$ where each of the “digits” $c_{i}(r)$ is O or l and only finitely many are I. Thus each $r\in D$ satisfies the inequalities $\ 0\leq r<1$ Put $A(r)=X$ if $r\geq1\,\cdot$ for any $\gamma\epsilon\,D$ define (3) $$ \bar{\iota}(r)=c_{1}(r)V_{1}+c_{2}(r)V_{2}+c_{3}(r)V_{3}+\cdot\cdot\cdot. $$ Note thal each of these sums is acully finite. Define (4) $$ f(x)=\operatorname*{inf}\left\{r:x\in A(r)\right\}\qquad(x\in X) $$ and (5) $$ d(x,y)=f(x-y)\qquad(x\in X,y\in X). $$ The prof that this $\ d$ has the desired properties depends on the inclusions (6) $$ A(r)+A(s)\subset A(r+s)\qquad(r\in D,s\in D). $$ every $A(s)$ contains 0,(G) implies Bcforc proving(G), let us see how the theorem follows from it. Since (7) $$ A(r)\subset A(r)\vdash A(t-r)\subset A(t)\qquad{\mathrm{iff}}\qquad r\ <t. $$ Thus $\langle{\mathcal{A}}(r)\rangle$ is tally ordered by set inclusion. We claim that (8) $$ f(x+y)\leq f(x)+f(y)\qquad(x\in X,y\in X). $$ In the proof of (8) we may, of course, assume that thc right sidc is $<1$ Fix $\scriptstyle t\;>\;0.$ There exist r and sin ${\boldsymbol{D}}$ such that $$ f(x)<r,\qquad f(y)<s,\qquad r+s<f(x)+f(y)+\varepsilon. $$1orouoGicAL vrcroR seAcrs 19 Thus $x\in{\mathcal{A}}(r),y\in{\mathcal{A}}(s),$ and(6) implies $x+y\in A(r+s).$ Now (8) follows because $$ f(x+y)\leq r+s<f(x)+f(y)+\varepsilon, $$ and $\scriptstyle{\mathcal{E}}$ was arbitrary is balanced, $f(x)=f(-x),$ and ${\mathfrak{s o}}\,f(x)\geq2^{-n}>0.$ $f(0)=0.$ If d on $X.$ Since each $A(r)$ for some ${\boldsymbol{n}},$ lt is obvious tha $x\neq0$ , then ${x}_{:}\not\in V_{n}=A(2^{-n})$ Thesepets o sow that eins a ato-nvant metr The open balls ented at re th open se (9) $$ B_{\vartheta}(0)=\{x:f(x)<\delta\}=\bigcup_{r<\delta}A(r). $$ $I\operatorname{f}\,\delta<2^{-n},$ then $B_{\partial}(0)\subset V_{n}$ Hence $\{B_{s}(0)\}$ is a local base for the topology of If each $V_{n}$ is $\;I f r+s<1$ The prof of G will be by induction $\operatorname{Let}P_{N}$ be the statement $X.$ This proves $(a).$ Since each $A(r)$ is balanced, so is each $\scriptstyle B_{k}(0).$ hence convex, so is each ${\mathit{A}}(r),\;{\mathrm{and}}\,(7)$ $\scriptstyle B_{d}(0)$ implies that the same is truc of each $\scriptstyle B_{d}(0),$ also of each translate of and $c_{n}(r)=c_{n}(s)=0\,f o r\;a l l\;n>$ N, then (10) $$ A(r)+A(s)\subset A(r+s). $$ $\mathbf{g}^{\prime}$ by $r\in D,s\in D,$ so that is true,by inspection. Assume and $c_{n}(r)=c_{n}(s)=0$ i $n>N.$ .and define ${\boldsymbol{r}}^{\prime}$ and $P_{1}$ $p_{\mathrm{v-1}}$ is true, for some $N>1.$ Choose $r+s<1$ (11 $$ r=r^{\prime}+c_{N}(r)2^{-N},\qquad s=s^{\prime}+c_{N}(s)2^{-N}. $$ Then (12)) $$ A(r)=A(r^{\prime})+c_{N}(r)V_{N},\qquad A(s)=A(s^{\prime})+c_{N}(s)V_{N}. $$ By $P_{N-1},A(r^{\prime})+A(s^{\prime})\subset A(r^{\prime}+s^{\prime})$ Hence (13) $$ A(r)+A(s)\subset A(r^{\prime}+s^{\prime})+c_{N}(r)V_{N}+c_{N}(s)V_{N}\,. $$ and If $c_{N}(r)=c_{N}(s)=0.$ then $r=r^{\prime},s=s^{\prime},$ and(13) gives(10). If $c_{s}(r)=0$ $c_{N}(s)=1,$ the rght side of (13) is $$ A(r^{\prime}+s^{\prime})+V_{N}=A(r^{\prime}+s^{\prime}+2^{-N})=A(r+s), $$ If so that (1O) holds again. The-case $c_{N}(r)=1,\,c_{N}(s)=0$ is handled the same way $c_{N}(r)-c_{N}(s)=1,$ the right side of(13) : A $$ \begin{array}{l c r}{{({{t^{\prime}}+{s^{\prime}})+{V_{N}}+{V_{N}}-{d({{r}^{\prime}}+{s^{\prime}})+{V_{N}}-1}}}\\ {{{}=\lambda({{t^{\prime}}+{s^{\prime}}})+{d({-}^{-N+1}})-{d({{r}^{\prime}+{s^{\prime}}+{2^{-}}^{-N+1}})}={d({r+s)}.}}}\end{array} $$ Thus The last inclusion depended on implies $P_{N}\,.$ Hence(G is corect, and the proot is complete $\scriptstyle P_{g-1}$ $P_{N-1}$ ${j}//{j}$20 GENERAL THEORY $X$ 1.25 Cauchy sequences(a) Suppose $d$ is a metric on a set $X.$ A sequence $\langle x_{n}\rangle$ in is a Cauchy sequence if to every $\scriptstyle{n>0}$ there corresponds an integer ${\cal N}$ such tha $d(x_{m},\,x_{n})<\varepsilon$ whenever $m>N$ and $n>N$ If every Cauchy sequence in $\textstyle X$ converges to a point of $X,$ then $\mathcal{A}$ is said to be a complete metric on $X.$ (b) Letr be the topology of a topological vector space $X.$ .The notion of Cauchy sequence can be defined in this setting without reference to any metric: Fix a loca base ${\mathcal{B}}$ for t.A sequence $\scriptstyle\{x_{a}\}$ in $\textstyle{X}$ is then said to bc a Cauchy sequence if to every $V\in{\mathcal{B}}$ corresponds an ${\boldsymbol{N}}$ such that $x_{n}-x_{m}\in V$ if $n>N$ and $m>N$ It is clear that different local bases for the same r give rise to the same class of Cauchy sequences. (c) Suppose now that $X$ Y is a topological vector space whose topology t is compatible with an invariant metric $d.$ Let us temporarily use the terms ${\mathcal{A}}{\mathcal{C}}$ Cauchy sequence and r-Cauchy sequence for the concepts defined in (a) and ${\mathfrak{(}}b{\mathfrak{)}},$ respectively Since $$ d(x_{n},\,x_{m})=d(x_{n}-x_{m},\,0), $$ and since the ${\mathcal{A}}$ balscentered at the origin form a local base for $\tau_{\circ}$ we conclude A sequence {x)) in $\textstyle X$ is a d-Cauchy sequenceifand only fitisat-Cauchy sequence Consequently, any two invariant metrics on $\textstyle X$ that are compatible with t havc the same Cauchy sequences. They clearly also have the same convergent sequences (namely, the r-convergent ones. These remarks prove te following theorem 1.26 Theorem $|g\,d_{i}$ and $d_{2}$ are invariant metrics on a vector space $\textstyle X$ which induce the same topology on ${\mathcal{X}},$ , then (a) $d_{\mathrm{i}}$ and $d_{2}$ have the same Cauchy sequences, and is complete (b)d, is complete if and only if $d_{2}$ lnvariance is needed in the hypothesis (Exercise 12) The ncxt theorem is an analogue of Lemma 1.20, with completeness in place of local compactness. Note that the two proofs are quite similar. 1.27 Theorem Suppose $\boldsymbol{\mathit{I}}$ is a subspace of a topological vector space $X,$ ,and ${\mathbf{}}Y$ Y is an ${\boldsymbol{F}}.$ space (in the topology inherited fom X).Then $\boldsymbol{\mathit{I}}$ is a closed subspace o $\textstyle{\bar{X}}$ PROOF Choose an invariant metric $\ d$ d on ${\boldsymbol{Y}},$ compatible with. its topology. Let $$ B_{1/n}=\left\{y\in Y\colon d(y,0)<\frac{1}{n}\right\}, $$ let $U_{n}$ be a neighborhood of $\mathbf{0}$ in $X$ such that $Y\cap U_{n}=B_{1/n}.$ and choose sym- metric neighborhoods $V_{n}$ V, of O in X such that $V_{n}+V_{n}\subset U_{n}.$TorouoocL vrcroe sexces 21 Sappose $x\in{\overline{{Y}}},$ and define in Let have exacty one poin be a neighborhood of ${\mathfrak{O}}$ in $$ E_{n}=Y\cap(x+V_{n})\qquad(n=1,2,3,\ldots). $$ and also in ${\cal{Y}}.$ -closures of the sets $E_{n}$ If $y_{1}\in E_{n}$ and $y_{2}\in E_{n},$ then $y_{1}-y_{2}$ lies in ${\cal{Y}}$ $V_{n}+\,V_{n}\subset\,U_{n}.$ hence is $B_{1,n}$ The diaes o the se $E_{n}$ therfore tend to O. Since eac $E_{n}$ nonempty and since $\displaystyle Y$ Yis complte fow that h and define ${\mathfrak{J}}{\mathfrak{I}}$ $\mathbf{\mathcal{S}}_{0}$ in common $X,$ flwsth $y_{0}$ Thus xe y. This roves th $$ \quad\qquad F_{n}=\Y\mathrm{{\Large~v~}}(x+\textstyle\mathrm{{\it{/}}}\cap V\mathrm{{\it{<}}}\ V_{n}). $$ Hencc $y_{W}=y_{0}$ Since $F_{n}\subset x+W,$ it common point ${\mathbf{}}{\mathbf{}}y_{W}$ But Theos anemes snat r-osus o he s ${\mathcal{W}}$ . This implies have one // les t $F_{n}\subset E_{n}$ $F_{n}$ $y_{0}=x.\qquad.$ $\textstyle X\!\cdot\!\supset$ cioueo ""w ${\bar{Y}}=\,Y.$ The fololwisipetas setie se 1.28 Theorem (O)U dis a nsioinimrie $\bar{a}$ rco ae $\textstyle{\cal{X}}$ then (の)f{8x) s Jor ery xe X and o $$ d(n x,\,0)\leq n d(x,\,0) $$ ,such that $\gamma_{n}\to\varnothing0$ and $\gamma_{n}\,x_{n} arrow0.$ $x_{n}\to0$ as $n arrow\infty,$ Statemcnt $\mathbf{\Psi}(a)$ $n=1,2,3,\ldots.$ $\gamma_{n}$ $\textstyle X$ and if $\textstyle{\mathcal{A}}$ isquce merzale oial ecor po lhen there ar osiesay PROOF follows rom that $d(x_{n},0)<k^{-2}\ i\Gamma\,n\geq n_{k}.$ Put $$ d(n x,0)\leq_{k=1}^{n}d(k x,(k-1)x)=n d(x,0). $$ if $n<n_{1};$ put $\gamma_{n}=k$ if m, $\cong n<n_{k+1}$ $X.$ Since T po 0.1 thres n neseosievmese ${\boldsymbol{n}}_{k}$ such $d(x_{n},0)\to0,$ $\mathcal{A}$ hc a ues n.osie utos For suchA $\gamma_{n}=1$ Hence $\gamma_{n}\,{\mathfrak{X}}_{n}\to0$ as $n arrow\Omega\cup\neq.$ $$ d(\gamma_{n}x_{n},\,0)=d(k x_{n},\,0)\le k d(x_{n},\,0)\,<k^{-1}. $$ // Boundedness and Continuit 1a Bgusdosssog n anme-ses oy oyuzienenon $X$ ,os…so o nsemeamsesein w $\textstyle{X}$ io moiosososismsnrsosese y ymn notion o boundesssme sae22 GENERAL THEORY If ${\boldsymbol{d}}$ is a metric on a set $X,$ a set $E\subset X$ is said to be ${\mathcal{A}}{\mathcal{C}}$ bounded if there is a number $M<\infty$ such that $d(x,y)\leq M$ for all x and $\mathbf{y}$ in $\boldsymbol{E}$ the bounded sets If $\textstyle X$ tis a topological vector space with a compatible metric $d,$ and the d-bounded ones need not be the same,even if ${\mathcal{A}}$ is invariant. For instance, i $^{*}d$ is a mctric such as the one constructed in Theorem $1.24,$ then $X$ itself is d-bounded (with $M=\Gamma.$ )but, as we shall see presently, $\textstyle X$ cannot be bounded,unless $X=\{0\}.$ If $\textstyle X$ is a normed space and ${\mathcal{A}}$ is the metric induced by the norm,then the two notions of boundedness coincide; but if $ .d\!\begin{array}{l}{\quad{}}\\ {\quad}\\ {\quad}\end{array}$ is rcplaccd by $d_{1}=d/(\mid+d)$ (an invariant metric which induces the same topology) they do not. Whenever bounded subsets of atopological vector space are discussed, t will understood that the definition is as in Section 1.6: $\mathbf{A}$ set $\boldsymbol{E}$ is bounded if, for every neighborhood ${\mathbf{}}V$ Vof C $0,$ O, we have $E\subset t V$ for all sufficiently large ${\mathit{I}}.$ We already saw(Theorem 1.15) that compact sets are bounded. To see another type of example, let us prove that Cauchy seunces are bounded (hence comvergent sequences are bounded): 1f with $V+V\subset W$ , then [part (b) of Section 1.25] there exists ${\mathbf{}}V$ ’and ${\mathcal{W}}$ are balanced $\boldsymbol{N}$ $\scriptstyle(x_{n})$ is a Cauchy sequence in $X_{\circ}$ and neighborhoods of ( $\mathbf{0}$ such that $x_{n}\in x_{N}+V$ for ${\operatorname{all}}\,n\geq N.$ Take $\operatorname{s>1}$ so that $x_{N}\in s V.$ Then $$ x_{n}\in s V+\ V\subset s V+s V\subset s W\quad\quad(n\geq N). $$ Hence $x_{n}\in t W$ for all $\scriptstyle n\geq1.$ if t is sufficiently large. Also, closures of bounded sets are bounded (Theorcm 1.13) On the other hand, if $x\not\equiv0$ and $E=\{n x\colon n=1,\,2,\,3,\,\ldots\,\}$ , then Eis not bounded, because there is a neighborhood ${\mathbf{}}V$ of O that does not contain x; hence nxis not in $n{\mathcal{V}}$ it follows that no $n\,{\mathcal{Y}}$ contains $\textstyle E.$ Consequently, no subspace of $\textstyle{\bar{X}}$ (other than {0}) can be bounded The next theorem characterizes boundedness in terms of sequences 1.30 Theorem The following two properties of a set E in a topological vector space are equivalent: (a) E is bounded. ${\boldsymbol{E}}$ and $\scriptstyle(x_{n}|$ is a sequence of scalars such that α。→0O as (b) ${\boldsymbol{J}}$ {x,} is a sequence in $\textstyle n\!\to\!\infty$ , then $\alpha_{n}\,\chi_{n} arrow0$ as $n arrow\infty0$ PROOF Suppose $\boldsymbol{E}$ is bounded. Let ${\mathbf{}}V$ be a balanced neighborhood of $\mathbf{0}$ in $X.$ if Then $E\subset t V$ for some ${\dot{t}}.$ If $x_{n}\in L$ and $\alpha_{n} arrow0,$ there exists $\boldsymbol{N}$ V such that $|\alpha_{n}|t<1$ $n>N$ Since $t^{-1}E\subset V$ and ${\mathbf{}}V$ is balanced, $\alpha_{n}\,x_{n}\in V$ for al $n>N.$ Thus $\alpha_{n}x_{n}\to0.$ Conversely,if ${\boldsymbol{E}}$ is not bounded, there is a neighborhood ${\mathbf{}}V$ of ${\boldsymbol{0}}$ and a Then no $r_{n}^{-1}x_{n}$ is in $V_{\mathrm{{J}}}$ such that no $\scriptstyle{}_{r,\,\!{\mathcal{V}}}$ 'contains E. Choose $\chi_{n}\in E$ such that $x_{n}\notin r_{n}\,V.$ ${a\!\!\!/}{b\!\!\!/}{b\!\!\!/}$ sequence $r_{n} arrow\infty$ so that $\{r_{n}^{-1}x_{n}\}$ does not converge to O.TorouooucAL vcrox srcs 23 above. spaces and A: $X\to Y$ 1.31 Doune iner tansfrmatiosSuppo $\boldsymbol{\mathit{I}}$ for every bounded set $E\subset X.$ bounded sets, ic., if $\lambda(E)$ $X$ ana $\boldsymbol{\mathit{I}}$ Y are topological vector is linear. A is said to be boumde if $\mathrm{\A}$ maps bounded sets into is a bounded subset or Tisinioncnitvt ht uaitno Sitoinenicn nas bin onwos aesis ond st tes o ineancioioe han solos s eoe u sti oincienmemimsisiastens disi s o urosot atieiosmmemso oeisaeas 1.32 Theorem Suppose $\textstyle X$ and $\boldsymbol{\mathit{I}}$ are topological vector spaces and $\Lambda\,;$ $X\to Y$ is lieun among e oio ponies XMwiizo $$ (a)\to(b)\to(c) $$ hold. I Xis metrzable,then alo $$ (c)\to(d)\to(a), $$ so tu alfour popeties are equialen (6) $\mathbf{(}a)\cdot\Lambda\ i$ is continuous $\Lambda$ is bounded. (c) ${\mathcal{I}}\,x_{n}\to0$ then $\{\Lambda x_{n};n=1,2,\,3,\dots.$ .} is bounded. (d) $r}$ then Ax,一→0 Exercise $1\,3$ contains an xample in which(b) holds bu $\mathbf{\Psi}(a)$ does not. PROOF Assume $(\alpha),$ let $\boldsymbol{\mathit{L}}$ E be a bounded set in ${\cal X},$ and let $W$ be a neighborhood $\textstyle X$ of Oin ${\cal{Y}},$ Since $\Lambda$ is continuous $(\operatorname{and}\Lambda0=0)$ there sa neighborhood V ofo in , so that such that $\Lambda(V)\subset W.$ Since $\boldsymbol{E}$ L is bounded $E\subset t V$ for all large ${\bar{t}},$ $$ \Lambda(E)\subset\Lambda(t V)=t\Lambda(V)\subset t W. $$ This shows that $\Lambda(E)$ is a bounded set in ${\cal Y}.$ satisfies $(c),$ and $\operatorname{flat}\,x_{n}\to0$ D.By Thus $(a)\to(b).$ Since convergent scquences are bounded, $\Lambda$ such that $\gamma_{n}\,x_{n}\to0$ Hence $\{\Lambda(\gamma_{n}.x_{n})\}$ is a bounded set in ${\mathbf{}}Y.$ Yis metrizable, that $(b)\to(c).$ Assume now that $\textstyle X$ $\gamma_{n} arrow\infty$ Theorem 1.28, there are positive scar and now Theorem 1.30 implis th $$ \Lambda x_{n}=\gamma_{n}^{-}{}^{1}\Lambda(\gamma_{n}x_{n})\to0\qquad\mathrm{~as~}\;\;n\to\sigma\mathrm{~} $$ such that $\Lambda^{-1}(W)$ Finalyssu tht a fails Thn the s a nihborhoo in $\textstyle X$ so that $x_{n}\to0$ but $\Lambda x_{n}\notin W.$ // Thus (d fails contains no neighborhood of oO in $X.$ If $\textstyle X$ $\mathcal{W}$ of O in ${}^{Y}$ local base the is therefor a sequcenc has a countable $\scriptstyle\{x_{n}\}$24 GENERAL THEORY Seminorms and Local Convexity 1.33 Definitions A seminorm on a vector space $\textstyle X$ is a real-valued function $D\!\!\!\!/$ p on $\textstyle{X}$ such that (a) $p(x+y)\leq p(x)+p(y)$ (6) $p(\alpha x)=\left|\alpha\right|p(x)$ and $\mathbf{\nabla}y$ in $\textstyle{\cal{X}}$ X and all scalars α for all $\textstyle{\mathcal{X}}$ Property $\mathbf{\Psi}(a)$ is called subadditivity. Theorem 1.34 will show that a seminorm p is a norm if it satisfies (c) $p(x)\neq0$ if $x\neq0.$ A family ${\mathcal{P}}$ of seminorms on $\textstyle X$ is said to be separating if to each $x\neq0$ corre- sponds at least one $p\in{\mathcal{P}}{\mathrm{~with~}}p(x)\neq0$ which is absorbing, in the sense that every Next, consider a convex set $A\subset{\mathcal{X}}$ xe X lies in $t A$ for some $l=\iota(x)>0.$ [For example, Ga) of Theorem 1.15 implies that every neighborhood ofC ${\boldsymbol{0}}$ in a topological vector space is absorbing. Every absorbing set obviously contains O. The Minkowski functional $\textstyle\mathit{l}_{A}$ q of $\scriptstyle A$ is defined by $$ \mu_{A}(x)=\operatorname*{inf}\left\{t>0:t^{-1}x\in A\right\}\qquad(x\in X). $$ Note that $\mu_{A}(x)<\infty$ for all $x\in X_{!}$ since $\scriptstyle A$ is absorbing. The seminorms on $\textstyle X$ will turn out to be precisely the Minkowski functionals of balanced convex absorbing sets. Seminorms are closely related to local convexity, in two ways: In every locally convex space there exists a separating family of continuous seminorms. Conversely, if ${\mathcal{P}}$ is a separating family of seminorms on a vector space $\textstyle{X}$ with the property that every $X,$ then ${\mathcal{P}}$ can be used to define a locally convex topology on $p\in{\mathcal{P}}$ is continuous This is a frequently used method of introducing a topology The details are contained in Theorems 1.36 and 1.37. 1.34 Theorem Suppose $\boldsymbol{\mathit{P}}$ is $\textstyle{\mathcal{A}}$ seminorm on $\bar{a}$ vector space $X.$ Then a) p(0) = 0. (b) $p(x)-p(y)|\leq p(x-y)$ (c) p(x)≥ 0. (d) $\{x\colon p(x)=0\}$ is a subspace of $X.$ $\ a n d\,p_{.}=\mu_{B}.$ (e) The set $B=\{x;p(x)<1\}$ is convex, balanced, absorbing, PROOF Statement $\mathbf{\Psi}(a)$ follows from $p(x x)=|\alpha|p(x),$ with $\alpha=0.$ 'The subaddi tivity of $D\!\!\!\!/$ shows that $p(x)=p(x-y+y)\leq p(x-y)+p(y)$ so that $p(x)-p(y)\leq p(x-y).$ This also holds with $\textstyle{\mathcal{X}}$ and $\mathbf{\nabla}y$ interchanged.TorotocucAL vrcron srAces $25$ Since $p(x-y)=p(y-x),$ (b) follows. With $y=0_{\mathrm{\bar{z}}}$ (b) implies ce) If 0x)= $p(y)=0$ and $x,\rho$ are scalars, (c) implies As to $(e),$ it is clear that ${\boldsymbol{B}}$ $$ 0\leq p(x x+\beta y)\leq|x|p(x)+|\beta|p(y)-0. $$ and $0<t<1,$ then This proves (D ris balanced. 1f $x\in B,y\in B,$ then $p(t^{-1}x)\geq1,$ and so $\scriptstyle{x^{-1}x}$ $$ p(t x+(1-t)y)\leq t p(x)+(1-t)p(y)<1. $$ ${\boldsymbol{B}}.$ This impie $\mu_{R}\leq p.$ But if $0<t\leq p(x)$ / Thus $\boldsymbol{B}$ Pis convex ${\mathrm{If}}\;x\in X{\mathrm{~and~}}S>p(x){\mathrm{~then~}}p(s^{-1}x)=s^{-1}p$ Hence px)<1. This shows that $\boldsymbol{B}$ is absorbing and also tha $\mu_{B}(x)\leq s.$ and com- pletes the proof. is not in $p(x)\leq\mu_{B}(x)$ (c) $\textstyle\mu_{A}$ $J f B=\{x:\mu_{A}(x)<1\}$ and TheormsSyo a coeaoose ncosc then Be Ae C and $\mu_{B}=\mu_{A}=\mu_{C}.$ 1.35 (a) Ax $X.$ Then (b) $x+y)\leq\mu_{A}(x)+\mu_{A}(y).$ $\mu_{A}(t x)=t\mu_{A}(x)\;i f\,t\geq0.$ (d) is a seminorm if $\scriptstyle{\mathcal{A}}$ is blanced. $C=\{x\colon\mu_{A}(x)\leq1\},$ PRoor sciat with each xe he se s e Suppose $t\in H_{A}(x)$ and $S>\iota$ Since $$ H_{A}(x)=\{t>0:t^{-1}x\in A\}. $$ and $\textstyle{\mathcal{A}}$ is convex,it follows that Since $\textstyle{\mathcal{A}}$ $H_{A}(x)$ Each $0\in{\mathcal{A}}$ Th $\mathrm{len}\,s^{-1}x\in A,\,t^{-1}y\in A.$ $\mu_{a}(x)$ Suppose $H_{A}(x)$ is a alf lne whose len en oint is convex $\mu_{A}(x)<s,\,\mu_{A}(y)<t,\,u=s+t$ lies in ${\mathcal{A}}.$ Hence $$ \mu^{-1}(x+y)={\binom{s}{u}}(s^{-1}x)+\left({\frac{t}{u}}({t}^{-1}y)\right) $$ are now obvious. $\mu_{A}(x+y)\leq u.$ This gives (a). Properties(b) and $\left(c\right)$ $x\in X,$ If $\mu_{A}(x)<1,$ then $1\in H_{A}(x),$ and so $x\in{\mathcal{A}}$ Likewise, if xe A, thcen , for every H.A(x) ≤1. Thus $B\ll A\subseteq C$ 八 $$ \mu_{\scriptscriptstyle C}(x)\leq\mu_{A}(x)\leq\mu_{B}(x). $$ $H_{B}(x)\subset H_{A}(x)\subset H_{C}(x).$ $s^{-1}x\in C,$ hence so that This implie $\mu_{A}(s^{-1}x)\leq1$ To prove that cquaity hols、supos $\mu_{C}(x)<s<t.$ Then , so that Thus $$ \mu_{A}(t^{-1}x)\leq{\frac{s}{t}}<1. $$ This completes he roo // $t^{-1}x\in B,\,\mu_{B}(t^{-1}x)\leq1,\,\mu_{B}(x)\leq t.$26 GHNERAL THEORY 1,36 Theorem Suppose W is a convex balanced local base ${\mathit{i}}n$ a topological vector space X. Associate to every Ve M its Minkowski functiona $\mu_{\mathrm{{F}}}$ Then $\langle\mu_{V};$ V∈ 9%} is a separating family of continuous seminorms on $X,$ PRO0F Since ${\mathbf{}}V$ is convex, balanced, and absorbing, ${\boldsymbol{\mu}}_{V}$ is a seminorm. $\operatorname{If}\,x\in X$ and $x\not\equiv0_{i}$ then $x\not\in V$ for some $V\in{\mathcal{B}}.$ For this ${\mathbf{}}V$ we have $h_{V}(x)\geq1$ Thus $\scriptstyle\{\mu_{Y}\}$ is a separating family.If $x\in V,$ , then $t x\in V$ for some $\textstyle t>1,$ since ${\mathbf{}}V$ is open. Hence $\mu_{V}<1$ in ${\mathit{V}}.$ If $\ r>0,$ it follows from Theorem 1.34 that $$ |\mu_{\nu}(x)-\mu_{\nu}(y)|\leq\mu_{\nu}(x-y)<r $$ if $x-y\in r V$ . This proves that each ${\boldsymbol{\mu}}{\boldsymbol{\nu}}$ is continuous. // 1.37 Theorem Suppose ${\mathcal{P}}$ P is a separating family of seminorms on a vector space $X.$ Associate to each pe JP and to each positive integer n the set $$ V(p,n)=\left\{x;p(x)<{\frac{1}{n}}\right\}. $$ Let W be the collection of all finite intersections of the sets $\mathcal{V}(p,n)$ .Then ${\mathcal{B}}$ is a convex balanced local base for a topology t on $X,$ which turns $X$ into $\bar{a}$ z locally convex space such that (a) every $p\in{\mathcal{P}}\,,$ is coninuous, amd (b) a set $E\subset X$ is bounded if and only if every $P\in{\mathcal{P}}$ is bounded on $\textstyle E.$ PROOF Declare a set $A\subset X$ to be open if and only if $\textstyle A$ 4 is a (possibly empty) union of translates of members of ${\mathcal{B}}.$ This clearly defines a translation-invariant topology $\overline{{\Gamma}}$ T on $X{\dot{\boldsymbol{r}}}:$ each member of ${\mathcal{A}}$ Bis convex and balanced, and ${\mathcal{A}}$ is a local base for r. Suppose $x\in N,\,x\ \tau=0$ Then $p(x)>0$ for some $P\in{\mathcal{P}},$ Since $x:{\mathfrak{s}}$ not in $V(p,n)$ if $n p(x)>1,$ we see that $\mathbf{0}$ is not in the neighborhood $x-V(p,n)$ of $X_{\textstyle}$ so that $\scriptstyle{\mathcal{X}}$ is not in the closure of {0}. Thus {y} is a closed set, and since t is translation-invariant, every point of $\textstyle{X}$ ’is a closed set Next we show that addition and scalar multiplication are continuous. Let $U$ be a neighborhood of Oin $X.$ Then (1) $$ U\ni V(p_{1},\,n_{1})\subset\cdot\cdot\cdot\,\frown\,V(p_{m},\,n_{m}) $$ for some $p_{1},\cdot\cdot\cdot,p_{m}\in P$ and some positive integers $n_{1,}\,\cdot\cdot\cdot,\;n_{m}\,.$ Put (2) $$ V=V(p_{1},2n_{1})\;\mathrm{c}\cdot\cdot\cdot\cdot\cdot\cdot\cap\;V(p_{m},2n_{m}). $$ Since every $~{\boldsymbol{P}}$ e 9 is subadditive, $V x+V\subset U.$ This proves that addition is continuousTorouooUCAL vEcron sPACrs $27$ $x\in s V$ for some Supposc now that xe $X,$ a is a scalar, and If $y\in x+t V$ are as above. Then then $s>0$ 、 Put $U_{\mathbf{\delta}}U$ and ${\mathbf{}}V$ and $\left|\beta-\alpha\right|\,<1/s,$ $t=s/(1+\vert\alpha\vert s).$ which lies in $$ \beta y-\alpha x=\beta(y-x)+(\beta-\alpha)x $$ $$ |\beta|\,t V+|\beta-\alpha|s V\mathrm{\!~}\subset V+\,V\subset U $$ since $|\beta|t\leq1$ and ${\mathbf{}}V$ is banced Thi rovs hat amuiation continuous. 1.34 Thus $\textstyle X$ is a cly covx pac. The defniono is conius $X,$ by (b) of Theorem every $p\in{\mathcal{P}}$ is coniu a O Henec $\mathcal{V}(p,n)$ shows that $D\!\!\!\!/$ 、If $n>M_{i}n_{i}$ Finally supose $E\subset X$ is bounded. Fix $E.$ $p\in{\mathcal{P}}.$ Since VY(p,I) is a neigh ${\mathcal{X}}\in E.$ $0_{\mathrm{{J}}}$ $\mathrm{{It}}$ borhood of O, $E\subset k V(p,1$ ) for some $k<\infty$ Hence $p(x)<k$ for cvery $J/{\big/}$ follows that every for $1\leq i\leq n$ n, it follows that $E\subset n U,$ so that ${\boldsymbol{E}}$ 上 is bounded. $(1\leq i\leq m).$ $P\in{\mathcal{P}}$ is bounded on is a neighborhood of Conversely, suppose ${\boldsymbol{E}}$ satifes hiscondition, $U_{\mathit{l}}$ on E and G) holds There are number $M_{i}<\infty$ such that $p_{i}<M_{i}$ 1.38 Remarks $X_{\mathrm{\bar{z}}}$ by the procscribd n Teore $X,$ as in Theorem 1.36. This and $\boldsymbol{P_{2}}$ defined b $p_{i}(x)=$ and that the set $V(p,n)$ of Theorem $1.37$ 0 , was ssyt tak efinitenetctioso hes ${\boldsymbol{p}}_{\boldsymbol{1}}$ ${\mathit{V}}(p,n)$ wna in Theorem 1.37 th st $\scriptstyle V(p,\iota)$ ”Dpesese ormacaSChey doPo is a convex balanced this, take $X=R^{2},$ and let ${\mathcal{P}}$ lsr samscseoiossarepo $1.37.$ Is $\tau=\tau_{1}\,?$ topology $\tau_{1}$ $x=(x_{1},\,x_{2}).$ consist of th seminorms ${\mathcal{A}}$ generates a separating lxl; here ${\mathcal{O}}$ of continuous seninormson $X,\operatorname{then}{\mathcal{B}}$ in turn inducesa family (b)Theorems $1.36$ Exercis elos scmn frn ${\mathcal{P}}$ and 1.3 ras aua probiem: T locseo toy ofaiiseviesp on $p=\mu_{1\nu}\,,$ Th awser siftiaiv T esns intitha are in r. Hence $\tau_{1}\subset\tau.$ Conversely, i We a $\operatorname{very}p\in{\mathcal{P}}\,\mathbf{i}$ s -continuous, so then $X,$ (c Thcorem 1.37 Shows tha ${\mathcal{P}}$ $$ {\mathcal W}=\{_{x}\colon\mu_{W}(x)<1\}=V(p,1) $$ $\tau\subset\tau_{1}.$ Define Thus $W\in\tau_{1}$ for every $W\in{\mathcal{B}};$ thisimpies tha $\{p_{1}\}.$ $\Pi\,{\mathcal{P}}=\{p_{i}\colon i=1,2,3,\dots\}$ is upte espainfaiyo osemnorms o invariant etic endidciy nems indcs aology with cunatecabases Toeam…e stiso emesmasmsosreas $(\mathbf{1})$ $$ d(x,\,y)=\sum_{i=1}^{\infty}{\frac{2^{-i}p_{i}(x-y)}{1+p_{i}(x-y)}} $$28 GENERAL THEoRx $\mathrm{\boldmath~It~}$ is easy to verify that $d$ is a metric on $X.$ To prove that $\ d$ l is compatible with r, we show that the ball (2) $$ B_{r}=\{x;d(x,0)<r\}\qquad(r>0) $$ of C form a local base for r is continuous(Theorem 1.37) and since the series (I) converges is a neighborhood $\mathbf{0}_{;}$ Since each $\boldsymbol{p_{i}}$ dis continuous; hence each $B_{r}$ , is open. If $\mathcal{W}$ uniformly on $X\times X,$ , then ${\mathcal{W}}$ contains the intersection of appropriately chosen sets (3) $$ V(p_{i},\,n_{i})=\left\{x;p_{i}(x)<\frac{1}{n_{i}}\right\}\qquad(1\leq i\leq k). $$ If $x\in B,$ , then (4) $$ \frac{2^{-i}p_{i}(x)}{1+p_{i}(x)}<r\qquad(i=1,2,3,\dots). $$ If ris small enough, (4) forces $p_{1}(x),\cdot\cdot,\,p_{k}(x)$ to be so small that $B_{r}$ lies in each of the sets (3); hence $B_{r}\subset W$ is compatible with . This proves that $\ d{\mathrm{~}}$ Formula(I) has considerable advantages over the more complicated constru tion of Theorem 1.24. Of course,(I) s applicable only in lcally cnvex spaces, and iü hasa faw even thcrc: Thebals whch it dfines need not be convex. An example o this is given in Exercise 18 1.39 Theorem A topological vector space Xisnormable if and onlyif its orgin ha a convex bounded neighborhood rxoor If $X$ is normable,an flis norm that s compatible with the tol is convex and bounded. ogy of $\textstyle X_{\mathrm{:}}$ then the open unit ball $\{x z\ \|x\|<1\}$ For the converse, assume ${\mathcal{V}}$ cis a convex bounded neighborhood of O. By Theorem 1.14, V contains a convex balanced neighborhood L ${\boldsymbol{U}}$ of O; of course, ${\boldsymbol{U}}$ is also bounded. Define (1) $$ \|x\|=\mu(x)\qquad(x\in X) $$ ogy of $X$ .If $\textstyle X\neq0,$ u s he Minkowski functional of ${\boldsymbol{U}},$ form a local base for the topol- where 山 $\boldsymbol{\mu}$ of Thcorem 1.15, the sets $r U\left(r\gg0\right)$ hence $\|x\|\geq r$ . It now follows By $\mathbf{\Psi}(c)$ then $x\not\in r U$ for some $r>0;$ functional, together with the fact that $U$ defines a norm. The definition of the Minkowski from Theorem 1.35 that $\operatorname{\left(1\right)}$ Vis open, implies that (2) $$ \{x;\ \|x\|<r\}=r U $$ for every $r>0.$ The norm topology coincides therefore with the given one $l/l/$ToroLoGICAL V 29 Quotient Spaces 1.40 Definitions Let $\textstyle N$ V that contains ${\mathcal{X}};$ ; thus $X.$ For every xe $X,$ let T(x) be the coset of ${\boldsymbol{N}}$ be a subspace ot vector spac $$ \pi(x)=x+N. $$ modulo $N_{\!\mathrm{)}}$ Thesecos a thens f avecto spa $X/N,$ calle the quotien space o $\textstyle{\mathcal{X}}$ in wh auin atnmiaoinicesie (1) $$ \pi(x).+\pi(y)=\pi(x+y),\qquad\alpha\pi(x)=\pi(x x). $$ This diffrs rom thc usul notion a $(1)$ are well (2) DNote that now $\alpha\pi(x)=N$ when $\scriptstyle x\;=\;0$ (that is $x^{\prime}-x\in N)$ and $\pi(y)=\pi(y^{\prime})$ then introdced i eton . Sic $\textstyle N$ is a vector space, the operation defined. This means that i $\pi(x)=\pi(x^{\prime})$ $$ \pi(x)+\pi(\dot{y})=\pi(x^{\prime})+\pi(y^{\prime}),\qquad\mathit{a}\pi(x^{\prime})=\alpha\pi(x). $$ $\textstyle{\mathcal{N}}$ open sets The origin of X/N is $\pi(0)=N$ By $(1)_{\mathrm{,}}\pi$ nisa inearmapping or Y onto $X/N.$ onto $X/N$ with $X.:\ L e t\ \tau_{N}$ be the colleton of al set $\textstyle X$ and that $\textstyle N$ $\pi^{-1}(E)\in{}^{\cdot}$ $\textstyle X$ turns out to be a topology on $X/N,$ as is nul space isotecall theqotnenmpo $E\subset X/N$ for which $\textstyle{X}$ is a cosed subspace o $\tau_{N}$ Suppose now tht vctortogyo T. Then colltheuoie tolgy Soe r spetes ioes e eteneme e a mm…smrsmpreseso 1. hsom,e $X$ and defin $\tau_{N}$ as aboe be the toloy o $\textstyle{\mathcal{N}}$ A .s suno olilno nex Ln (6) I C A tinuous, and open *mr uoy onX/v e uoinm "2, -X/ ino,。c with Ve W is a local (a) (d) $\textstyle X$ is an 上 ocal se or te he coletin all e $\pi(V)$ $\bar{a}$ (c) base for $\tau_{N}$ - spae o Frechet space, o a Bachspace,so is X/N. boundedess.mtrizabiliy normabiliy foyC/0y e X imete by X/: oacorxy lo ${\boldsymbol{F}}\cdot$ PROOF Sinc $\mathrm{\boldmath~\nabla~}\pi^{-1}(A\cap B)^{*}=\pi^{-1}(A)\cap\pi^{-1}(B)\ t$ and $$ \pi^{-1}(\emptyset/E_{\lambda})=\bigcup\pi^{-1}(E_{\lambda}), $$ $\tau_{N}$ is a topologyA set $F\subset X/N$ is $\tau_{N}.$ clsd if and only i $\pi^{-1}(F)$ is r-closed. ln particulat, very poin or X/N s losea sin $$ \pi^{-1}(\pi(x))=N+x $$ and $\textstyle N$ was assumed to be closed30 GENERAL THEORY The continuity of $\textstyle\pi$ follows directly from the definition of $\tau_{N}.$ .Next, suppose Ve. Since $$ \pi^{-1}(\pi(V))=N+V $$ and $N+V\in\tau,$ it follows that $\pi(V)\in\tau_{N}$ Thus $\textstyle\pi$ is an open mapping If now $W$ is a neighborhood of ${\boldsymbol{0}}$ in $X/N,$ there is a neighborhood ${\mathbf{}}V$ of O in $\textstyle X$ such that $$ V+V\subset\pi^{-1}(W). $$ Hence $\pi(V)+\pi(V)\subset W.$ Since r is open, $\scriptstyle\pi(V)$ is a neighborhood of Oin $X/N.$ Addition is therefore continuous in $X/N.$ The continuity of scalar multiplication in X/Nis proved in thesame manner. This establishes (a) $\mathrm{I}\mathbf{t}$ is clear that (a implies $\mathbf{\nabla}(b)$ With the aid of Theorems 1.32, 1.24, and 1.39,it is just as easy to see that $\mathbf{\nabla}(b)$ implies (c) $X,$ compatible with t. Suppose next that $\ d{\mathrm{~}}$ is an invariant metric on Define $~\rho$ p by $$ \rho(\pi(x),\,\pi(y))=\operatorname*{inf}\left\{d(x-y,\,z):z\in N\right\}. $$ This may be interpreted as the distance from $x-y$ to ${\cal N}$ We omit the veri- fications that are now needed to show that $~\rho$ is well defined and that it is an invariant metric on $X/N.$ Since $$ \pi(\{x;d(x,\,0)<r\})=\{u:\rho(u,\,0)<r\}, $$ it follows from $\mathbf{\nabla}(\mathbf{\nabla})$ that $~\rho$ is compatible with $\tau_{N}$ If $\textstyle X$ is normed, this definition of $~\rho$ specializes to yield what is usually called the quotient norm of $X/N\colon$ $$ \|\pi(x)\|=\operatorname*{inf}\left\{\|x-z\|\cdot z\in N\right\}. $$ complete To prove $(d)$ we have to show that $~\rho$ p is a complete metric whenever $\mathcal{A}$ is Suppose $\scriptstyle\{u_{a}\}$ is a Cauchy sequence in $X/N,$ relative to ${\boldsymbol{\rho}}$ There is a as subsequence $\scriptstyle\{w_{n}\}$ with $\rho(t_{n})$ $u_{n_{i+1}}\rangle<2^{-i}.$ One can then inductively choose $u_{n_{i}}\to\pi(x)$ $x_{i}\in X$ such that $\pi(x_{i})=u_{n_{i}}$ and $d(x_{i},\,x_{i+1})<2^{-\,t}.$ If dis complcte, the Cauchy sequence $\langle x_{i}\rangle$ converges to some xeX. The continuity of rimplies that $i arrow\infty.$ But if a Cauchy sequence has a convergent subsequence then the fu 1.41. sequence must converge. Hence ${\boldsymbol{\rho}}$ is complete, and so is the proof of Theorem 7roroLoCICAL VcroR sPACEs 31 Here isan easy application of these concepts $\textstyle N$ 1.42 Theorem Suppose ${\boldsymbol{N}}$ and ${\boldsymbol{F}}$ are subspaces of a topological vector space $X,$ is closed and ${\mathbf{}}F$ has finite dimension. Then $N+F$ is closed. PROOF Let $\textstyle\pi$ be the quotient map of $\textstyle{X}$ onto $X/N,$ and give X/N its quotient is a topology. Then $\,_{n(F)}$ is a finite-dimensional subspace of $X/N;$ since $X/N$ Since $N+F=\pi^{-1}(\pi(F))$ topological etor space, Theorem 1.2l implies that rGFDis closed i $X/N.$ // pare Exercise 20). and ris continuous, weconclude that $N+F i s$ closed. (Com- $\textstyle X$ and 1.43 Seminorms and quoticnt spaces Suppose $D\!\!\!\!/$ is a seminorm ona vector space $$ .^{\circ}\qquad\qquad\qquad\qquad\qquad\qquad\qquad\qquad\qquad\qquad\qquad\qquad\qquad\qquad\qquad\qquad\qquad\qquad\qquad\qquad\qquad\qquad\qquad\qquad\qquad\qquad\qquad\qquad\qquad\qquad\qquad\qquad\qquad\qquad(\qquad\qquad\qquad(1). $$ and define Then Nis a subspace of $\textstyle X$ (Theorem 1.34). Let r be the quotient map of $X$ onto $X/N,$ $$ \tilde{p}(\pi(x))=p(x). $$ If $\pi(x)=\pi(y)_{*}$ ). then $p(x-y)=0,$ and since $$ |p(x)-p(y)|\leq p(x-y) $$ it follows that ${\tilde{p}}(\pi(x))={\tilde{p}}(\pi(y))$ . Thus $\tilde{p}$ is well defined on $X/N,$ and it is now easy to verify that $\tilde{P}$ j is a norm on $X/N.$ $|0,1\rangle$ for which $r,\ 1\leq r<\omega;$ let ${\boldsymbol{L}}^{\prime}$ be the space of all Here is a familiar cxample of this. Fix Lebesgue measurable functions on $$ p(f)=\|f\|_{r}=\left\{\int_{0}^{1}|f(t)|^{r}\,d t\right\}^{1/r}<\infty\,. $$ $D\!\!\!\!/$ to ${\tilde{\mathcal{P}}}.$ This defines a seminorm on $\boldsymbol{N}$ be the set of these “null functions:.”Then U/IN is the Banach whenever ${\mathcal{I}}=0$ almost everywhere. Let $L^{\prime},$ not $\vec{\mathbf{a}}$ norm, since $\|f\|_{r}=0$ space that is usually called ${\boldsymbol{L}}.$ The norm of ${\bar{\boldsymbol{L}}}^{\prime}$ is obtained by the above passage from Examples $K_{n}$ valued continuous functions on $\Omega,$ 1.44 The spaces CQ)if is a nonemty open set insome cucican spac the which can be chosen so that lies in the interior of SQ is the union of countably many compact set $K_{n}\neq\varnothing$ $K_{n+1}\left(n=1,\,2,\,3,\,\ldots\,\right).$ C(QD)is the vector spaceof ll complex- topologized by the separating family of seminorms $\operatorname{\mathcal{(1)}}$ $$ p_{n}(f)=\operatorname*{sup}\left\{|f(x)|:x\in K_{n}\right\}, $$32 GENERAL THEoRY in accordance with Theorem 1.37. Since $p_{1}\leq p_{2}\leq\cdots,$ the sets (2) $$ V_{n}=\left\{f\in C(\Omega)\colon p_{n}(f)<\frac{1}{n}\right\}\qquad(n=1,\,2,\,3,\,\ldots) $$ topology of form a convex local base for $c(\Omega).$ According to remark(c) of Section 1.38, the $C(\Omega)$ is compatible with the metric (3) $$ d(f,g)=\sum_{n=1}^{\infty}\,\frac{2^{-n}p_{n}(f-g)}{1+p_{n}(f-g)}. $$ If proved hat $c(\mathbf{a})$ is a Cauchy sequence relative to this metric, then $p_{n}(f_{i}-f_{j})\to0$ for every n,as $^{\left\{f\right\}}$ $i,j\to\varnothing,$ so that $^{\left\{J\right\}}$ converges uniformly on $K_{n}\,$ . to a function fe C(Q).An easy computation then shows $d(f,f_{i})\to0$ Thus $\ d$ is a complete metric.We have now is a Fréchet space. By $\mathbf{\nabla}(b)$ of Theorem 1.37, a set $E\subseteq C(\Omega)$ is bounded if and only if there are numbers $M_{n}<\infty$ such that p,f) ≤ M,for all fé E;explicitly, (4) $$ |f(x)|\leq M_{n}\qquad\mathrm{if}f\in E{\mathrm{~and~}}x\in K_{n} $$ no $V_{n}$ Since every $V_{n}$ contains an f for which $p_{n+1}(f)$ is as large as we please,it follows that is bounded. Thus $C(\Omega)$ is not locally bounded, hence is not normable. 1.45 The spaces $H(\Omega)$ Let $\mathbb{Q}$ now be a nonempty open subset of the complex plane, define $c(\Omega)$ as in Section 1.44, and Iet H(Q) be the subspace of $C(\Omega)$ that con- sists of the holomorphic functions in $\mathbb{Q}$ 2. Since sequences of holomorphic functions that converge uniformly on compact sets have holomorphic limits, $\scriptstyle I(\Omega)$ is a closed subspace of C(Q2).Hence $H(\Omega)$ is a Freéchet space. We shall now prove that $I({\boldsymbol{\Omega}})$ has the Heine-Borel property It will then follow from Theorem 1.23 that $H(\Omega)$ is not locally bounded, hence is not normable. Let $\boldsymbol{E}$ be a closed and bounded subset of $H(\Omega).$ Then $\boldsymbol{E}$ satisfies inequalities such as $(\lambda)$ of Section 1.44. Montel's classical theorem about normal families (Th. 14.6 of [23]) implies therefore that every sequence $\{f_{i}\}\subset E$ has a subsequence that con- verges uniformly on compact subsets of Q [hcncc in thc topology of H(Q)] to somc fe H(D).Since ${\boldsymbol{E}}$ is closed, fe E. This proves that ${\boldsymbol{F}}$ is compact. 1.46 The spaces $C^{\infty}(\Omega)$ and ${\mathcal{D}}_{K}$ We begin this section by introducing some terminology that will be used in our later work with distributions In any discussion of functions of n variables, the term multi-index denotes an ordered ${\boldsymbol{\pi}}^{\prime}$ tuple (1) $$ {\mathcal{x}}=(\alpha_{1},\ \cdot\cdot,\ \sigma_{n}) $$ 'Numbes racetsrer ores idi he BibiographyrorouooucAL VrcroR sePAcs 3 operator of nonnegatve integer $\propto_{i}$ wWith each mult-inde $\textstyle{\mathcal{Q}}$ is associated te difentia (2) $$ D^{\alpha}=\left(\frac{\partial}{\partial x_{1}}\right)^{\alpha_{1}}\cdot\cdot\cdot\left(\frac{\hat{g}}{\hat{g}x_{n}}\right)^{\alpha_{n}} $$ whose order is (3) $$ |\alpha\,|=\alpha_{1}+\cdots+\alpha_{n}. $$ If $|x|=0$ ,,D* =/ ${\mathcal{G}}.$ my oulspeos soc is sa to belong to $C^{\infty}(\Omega)$ Aompeolfiso fne nsmenemtyopene $\Omega\subset R^{n}$ $\{x;f(x)\neq0\}.$ if Dye COD for evey mt-ne J" wy…o… settin $K_{i+1}$ and is a compact set i $\textstyle R^{n}\!,$ then ${\mathcal{D}}_{K}$ dcnots e spaeo ll / C(KR" whos whenever $K\subset\Omega.$ If $\textstyle K$ interior of support lies in K.(The lette ${\mathcal{D}}$ has e so tspsevesecsenwa into a Freche space wih lies in the subspace of $C^{\alpha}(\Omega).$ published his work on istonsS》 ${\mathcal{A}}_{K}$ is a clsed suhspace o $C^{\infty}(\Omega)$ may be identifed with a $K\subset\Omega$ 、then ${\mathcal{Q}}_{K}$ We mow define ology on $\Omega=\cup K_{i}$ Define seminorms $(i=1,\,2,\,3,\,*.\,)$ on $C^{\infty}(\Omega),$ $N=1,$ 2,3,.., b the Heine-Bore propery, snchiha $C^{\infty}(\Omega)$ which makes $C^{\infty}(\Omega)$ $K_{i}$ To do thi chse comat s $K_{i}$ such that ${\mathcal{P}}_{N}$ (4) $$ p_{N}(f)-\operatorname*{max}\,\{\,|D^{x}f(x)|\,;\,x\in K_{N},\,|x|\leq N\}. $$ remark this topology. Since ${\mathcal{D}}_{K}$ Theye euelell owx ogy $K_{\!\cdot\!}$ it follows that ${\mathcal{D}}_{K}$ is closed in $C^{\infty}(\Omega).$ $\left(c\right)$ of Section 1.38. For cac $C^{\alpha}(\Omega),$ see Theorem 1.37 and A local base s given y te set $x\in\Omega$ the functional $f\to f(x)$ is continuous in ranges over the complemento pitcoionusess deumns s (5) $$ V_{N}=\left\{f\in C^{\alpha}(\Omega)\colon p_{N}(f)<\frac{1}{N}\right\}\qquad(N=1,2,3,\ldots). $$ $\operatorname{rlr}(f_{i j})$ isa Cauchy sequence in $C^{\omega}(\Omega)$ (se ctin 1.25 and f Nisfx on $K_{N},$ if $|\alpha|\leq N.$ It that Thus $C^{\alpha}(\Omega)$ in th topology of $C^{\infty}(\Omega).$ lt is now evident that $g_{0}\in C^{\infty}(\Omega),$ that ${\mathrm{then}}f_{i}-f_{j}\in V_{N}$ ${\mathcal{G}}_{x}\,.$ if $\hat{\boldsymbol{l}}$ follows that each ${\mathfrak{p}}f_{i}$ are sfficty large. Thus $|D^{x}f_{i}-D^{x}f_{j}|<1/N$ $g_{\alpha}=D^{\alpha}g_{0}\,,$ and and ${}\dot{J}$ ln particular $f_{i}(x)\to g_{0}(x).$ cgnyvssioiomy oam ts i" $f_{i}\to g$ spaces .x. is ret s T ne smie t o co sedu34 GENERAL THEoRY on process that every sequence in $\boldsymbol{E}$ imply the equicontinuity of $\{D^{\beta}f\colon f\in E\}$ is closed and bounded. By Theorem 1.37, the $M_{N}<\infty$ such that con- $p_{N}(f)\leq M_{N}$ Suppose ncxt that $E\bumpeq C^{n}(\Omega)$ ,vaid boundedness of $\boldsymbol{E}$ is equivalent to the existence of numbers for $N=1,\,2,$ 3, .….and for all fe E. The inequalities $|D^{\alpha}f|\leq M_{N}$ $K_{N}$ when $|\alpha|\leq N,$ verges, uniformly on compact subsets of $\Omega,$ for each multi-index ${\boldsymbol{\beta}}.$ $K_{N-1},\mathrm{if}\ |\beta|\leq N-1$ con- on lt now follows from Ascois theorem (proved in Appendix A) and Cantor's diagonal $\langle D^{f}f_{h}\rangle$ contains a subsequence {f} for which Hence $(f_{i})$ whenever $\textstyle K$ verges in the topology of $C^{\infty}(\Omega).$ This proves that ${\boldsymbol{E}}$ is compact. ${\mathcal{D}}_{K}$ $C^{\infty}(\Omega)$ Hence Ci(QD has the Heine-Borel property t fllows from Theorem 1.23 that because dim ${\mathcal{D}}_{K}=\infty$ in that is not locally bounded, hence not normable. The same conclusion holds for < has nonempty interior (otherwise ${\mathcal{Q}}_{K}=\{0\},$ case. This last statement is a consequence of the following proposition: there exists $\phi\in C^{\infty}(R^{n})$ such that $\phi(x)=1$ for every $x\in B_{1}$ and in the interior of $\scriptstyle{B_{2}}$ , then $\scriptstyle{\mathcal{X}}$ 1f B, and $B_{2}$ are concentric closed balls n $R^{n},$ with $B_{1}$ for every $\phi(x)=0$ outside $B_{2}$ To find such a $\phi_{i}$ , we construct $g\in C^{\infty}(R^{1})$ such that G(x).= 0 for $x<a,g(x)=1$ for $x>b$ (where $0<a<b<\infty$ are preassigned) and put (6) $$ \phi(x_{1},\ldots,x_{n})=1-g(x_{1}^{2}+\cdot\cdot\cdot+\,x_{n}^{2}). $$ The following construction ${\mathfrak{o t}}{\mathfrak{g}}$ has the advantage that suitable choics of {,} can lead to functions with other desired properties. Suppose $0<a<b<\infty.$ Choose positive numbers $\delta_{0}$ $\partial_{1},$ $\delta_{2},\,\ldots,$ with $\exists_{i}:$ b-a; put (7) $$ m_{n}=\frac{2^{n}}{\delta_{1}\cdot\cdot\cdot\delta_{n}}\qquad(n=1,2,3,\cdot\cdot\cdot); $$ $\left|\mathrm{et}\,f_{0}\right.$ be a continuous monotonic function such $\mathrm{that}\,f_{0}(x)=0$ when $x<a,f_{0}(x)=1$ when $x>a+\delta_{0}{\mathrm{:}}$ and define (8) $$ f_{n}(x)=\frac{1}{\delta_{n}}\int_{x-\delta_{n}}^{x}f_{n-1}(t)\,d t\qquad(n=1,\,2,\,3,\,\cdot\cdot\cdot). $$ Diflerentiation of this integral shows,by induction, that /, has n continuous derivatives and that $\mid D^{n}f_{n}\mid\leq p n.$ If $n\succ r,$ then (9) $$ D^{r}f_{n}(x)=\frac{1}{\delta_{n}}\int_{0}^{\delta_{n}}(D^{r}f_{n-1})(x-t)\;d t, $$ so that (10) $\vert D^{\prime}f_{n}\vert\leq m_{r}\quad\quad(n\geq r),$TOroLoGICAL vECToR SrACEs 35 again b induction on n. The meanvalu theorem,apied to O) shows tha (11) $$ \vert\,D^{r}\!f_{\!n}-D^{r}\!f_{\!n-1}\,\vert\leq m_{\!r+1}\,\delta_{\!n}\qquad(n\geq r+2). $$ Since $\Sigma\delta_{n}<\infty,$ each $\scriptstyle\{U f_{k}\}$ converges, uniformly for $r=1_{\mathrm{\mathrm{\Lambda}}}$ 2,3,.. such that $g(x)=0$ for converges to a function ${\mathcal{G}},$ with $\mid D^{\prime}g\mid\leq m,$ $\mathbf{on}\,(-\infty,\,\infty),$ as n→00. Hence {) $x<a{\mathrm{~and~}}g(x)=1$ for $x>b.$ of 1.47 The spaces ${\mathcal{L}}^{p}$ with $\scriptstyle0\,<\,p\,<\,1$ Consider a fixed pin this range. The elements ${\boldsymbol{L}}^{p}$ are those Lebesgue measurable functions f on [0,1 for which (1) $$ \Delta(f)=\int_{0}^{1}|f(t)|^{p}\,d t<\infty, $$ $0<p<1$ , thc incquality wih he ual ientcition ofucionsthat concie alos veywe. Sne (2) $$ (a+b)^{p}\leq a^{p}+b^{p} $$ holds when $a\geq0$ and $b\geq0.$ This give (3) $$ \Delta(f+g)\leq\Delta(f)+\Delta(g), $$ so that (4) $$ d(f,g)=\Delta(f-g) $$ defines an invariant metric o $p\geq1$ The balls That this dis complene is proved in the same way as $L^{p}.$ in the familiar case (5) $$ B_{r}=\left\{f\in L^{p}\colon\Delta(f)<r\right\} $$ bounded form a local base for the topology of $L^{p}.$ Since $B_{1}=r^{-1/p}B_{r};$ , for all > 0, $B_{\mathrm{i}}$ is Thus ${\boldsymbol{L}}^{p}$ is a locally bounded -spac of $|{\mathcal{F}}|\;^{p}.$ positive integer $\;n$ such that $n^{p-1_{\circ}}\Delta(f)<r.$ contains no conuex open sets, other tham O and $L^{p}.$ without $\downarrow\bar{\downarrow}\rangle\leq$ claim that ${\boldsymbol{L}}^{p}$ To prove this, suppose $V\neq{\mathcal{D}}$ is open and convex in ${\mathcal{L}}^{p}$ P.Assume $0\in V,$ there is a loss of generality. Then $V\simeq B_{r}\,.$ for some $\scriptstyle r\gg0$ ${\mathrm{Pick}}\,f\in L^{p}.$ Since $p<1.$ , there are points By the continuity of the indefinite integral $$ 0=x_{0}\prec x_{1}\prec\cdot\cdot\cdot<x_{n}=1 $$ such that $(6)$ 1Jf0)lP dt=n"公(O (1≤i≤ n).36 GENERAL THEORv Define $g_{i}(t)=n J(t)$ if $x_{i-1}<t\leq x_{i},g_{i}(t)=0$ otherwise. Then $g_{i}\in V_{i}$ since (6) shows (7) $$ \Delta(g_{i})=n^{p-1}\,\,\Delta(f)\,<p\;\;\;\;\;\;\;\;\;(1\,\leq\,i\,\leq\,n) $$ and $V\to B_{r}$ Since ${\mathbf{}}V$ Vis convex and (8) $$ f={\frac{1}{n}}\left(g_{1}+\cdot\cdot\cdot+g_{n}\right), $$ it follows that f∈ ${\cal{V}}.$ Hence $V=L^{p}.$ This lack of convex open sets has a curious consequence. Suppose $\Lambda:I^{p}\to Y$ is a continuous linear mapping of ${\boldsymbol{J}}^{p}$ P into some locally con- vex space Y. Let ${\mathcal{A}}$ be a convex local base for ${\cal{Y}}.$ If $W\in{\mathcal{B}},$ then $\Lambda^{-1}(W)$ is convex open, not empty.Hence $\Lambda^{-1}(W)=L^{p}.$ Consequently, $\Lambda(L)\subset W$ for cvery $W\in{\mathcal{B}}$ We conclude that $\Lambda f=0\;\mathrm{for\;every}f\in L^{p}.$ into any locally convex space ${\boldsymbol{\gamma}},$ Thus O is the only continuous linear mapping of T ${\boldsymbol{L}}^{p}$ if O $<p<1$ In particular, ${\boldsymbol{0}}$ is the only continuous linear functional on these $p\geq1$ ${\boldsymbol{L}}^{p}$ P-spaces This is, of course, in violent contrast to the familiar case Exercises Z Suppose $\textstyle X$ Yis a vector space.Al sets mentioned below are understood to be subsets of $X.$ Prove the following statements from thc axioms as given in Section 1.4.(Some of thcsc are tacitly used in the text.) (a) Ifxe $\textstyle X$ ’ and ye 入 $\textstyle X$ there isa unique $z\in X$ such that $x+z=y.$ $\mathbf{\nabla}(b)$ 0x = 0 = c0O if $\scriptstyle x\in X$ and αis a scalar (c) $2A arrow A+A;$ it may happen that $2A\neq A+A$ $\langle{\mathcal{Q}}\rangle\ .$ is convex if and only if $(s+t)A=s A+t A$ for all positive scalars s and t (e) Every union (and intersection) of balanced sets is balanced. C) Every intersection of convex sets is convex $(g)$ $\mathbf{\hat{I}}$ 'is a collection of convex sets that is totally ordered by set inclusion, then the union of all mcmbers of ${\bf I}$ is convex (h) If $\scriptstyle A$ and $\boldsymbol{B}$ are convex, so is $A+B.$ (i)If $\scriptstyle A$ and $\boldsymbol{B}$ are balanccd, so is $A+b.$ $(\,j\,\!j\,\!j\,\!j\,\!$ Show that parts $(f)_{\circ}$ $(g),$ and (h) hold with subspaces in place of convex sets. 2'The convex hull of a set ${\mathcal A}$ in a vector space $\textstyle X$ is the set of all convex combinations of members of $A_{\cdot}$ t, that is, the set f all sums $$ \left.\ell_{1}{\mathcal{X}}_{1}\ \ |\ \ *\ \ast\ -|-\ \ell_{n}{\mathcal{X}}_{n}\right. $$ 3 in which $x_{t}\in A,\,t_{t}\geq0,\,\sum\,t_{t}=1;n\,\mathrm{is}$ arbitrary. Prove that the convex hull of A is convex and that it is the interscction of all convex scts that contain ${\cal A}.$ Let Xbe a topological vector space. Al ets mentioned below are understood to be sub- sets of $X.$ Prove the following statements a) The convex hull of every open set is open.TorouooucAL VECTo seACes 37 5 7 Let 。 ropteps is compact and $\boldsymbol{B}$ such that $E<t V^{\prime}$ ts olosos ogog eyeo oda s one Cris is closed. of $\mathbf{\nabla}\quad\mathbf{\nabla}$ (b) If X $\textstyle X$ Let subset of ${\boldsymbol{E}}$ and $\boldsymbol{B}$ ${\boldsymbol{E}}$ false tu oa conviysesctio .4.9 $A+B\,$ $A+B$ is bancd butat is iteroris o ${\mathbf{}}V$ $\mathbf{\Psi}(c)$ $\operatorname{F}{\boldsymbol{A}}$ and $\boldsymbol{B}$ are bounded, so is (d) If $\scriptstyle A$ are compact, so is $A+B.$ (e) If $\scriptstyle{\mathcal{A}}$ is closed, then 1.13 may therefor be strict CO Ttu sgos ia o oesS e enuson o Theore $\textstyle X$ $B=\{(z_{1},z_{2})\in{\bf C}^{2}\colon|z_{1}|\leq|z_{2}|\}$ Show th $\boldsymbol{B}$ corresponds some ICompare with e f Theorem 113.1 by the family of seminorms Goegoroso oess en cion 1. wua eonen $\scriptstyle t\gg0$ eoisionigonse eseeresasesvevy eo E is bounded m,oescs oude n s ouna e owososoalmeoe nto n ua . oi $$ p_{x}(f)=|f(x)|\qquad(0\leq x\leq1). $$ constant $M<\infty$ Show thate i sequc $\{f_{n}\}$ lin Ths oyseset o oin.x oc Justi seinoy such that(a ${\mathcal{O}}$ 'and is closed under max.[ This $\mathbf{0}$ O as $n\to\varnothing\quad$ (の) Suppose $\textstyle{\mathcal{Q}}$ 2 is as in part (a) and $\Lambda$ $\pm p\in\mathcal{A}$ $p=\operatorname*{max}\left(p_{1},p_{2}\right),$ then $\scriptstyle\{i,i\}$ converges to C $X.$ Let $\overline{{\mathcal{L}}}$ be the means: If Theorem .3 s appied t ${\mathcal{P}}$ and to ${\mathcal{O}},$ $\textstyle X$ does no converge 8(a) Suppose ${\mathcal{D}}$ hu ay seueceoeocasasuc th $\gamma_{n} arrow\infty\mathrm{~then~}\{\gamma_{n}f_{n}\}$ Pismos s so oesfsgou oiecioa co Y that contains same cardinality as [0, 11) i. ainaiy o stminms om x ctors psc T sua tisi cno bomte o heoem 1.2 smalles family or seminorms on $\textstyle X$ $P\i\in{\mathcal{D}},\,P\i\in{\mathcal{D}},$ and $\mathcal{A}$ leuadict basrine anahosba lf te construction o ISee Remark (a) of Sction 1.38. $p\in{\mathcal{R}}_{\cdot}]$ cide. The main difference is that show tathe toresuin toiesc uous if andonly if hexis such that sa inear functional on X. Show itha for all $x\in X$ and some $\Lambda$ is contin- $|\operatorname{A}x|\leq M\rho(x)$ 9 Suppose (a) $\textstyle X{\mathrm{~}}$ and ${\bf{}}Y$ are tological vetor spaces ${\bar{X}},$ (c) (b) A: $\boldsymbol{N}$ is a closed subspace o $X\to Y$ is inea (d)r: $X\to X/N$ is the qoticnt map, and (e) Ax - for e $x\in N$ rove te a uniqu : X/V一Y whichsase $\Lambda{}=f{\boldsymbol{\epsilon}}$ , thtis, Mx=/(r() forus xRoetASineanamnaAiosisusuaianuiyaSzlcon uous. Also,A sopen if and oly if is open38 GENERAL THEoRY $\left|\begin{array}{c c c c c c c c c c c c c c}{{|\bar{\star}\,}}&{{}}&{{}}&{{}}&{{}}&{{}}&{{}}&{{}}&{{}}&{{}}&{{}}&{{}}&{{}}&{{}}\\ {{\overline{{{\bf j}}}\,}}&{{}}&{{}}&{{}}&{{}}&{{}}&{{}}&{{}}\end{array}\right|$ $I O$ Suppose $X$ and ${\cal{Y}}$ are topological vector spaces, dim $Y<\infty,$ A: $X{\xrightarrow{}}Y$ is linear, and $\mathbf{\nabla}(b)$ $\Lambda(X)=Y$ is closed, and prove that $\mathbf{\hat{A}}$ is then (a) Prove that A is an open mapping. $\mathbf{A}$ Assume, in addition, that the null space of ${\mathcal{U}}$ If continuous the codimension of $\boldsymbol{N}$ V in Xis, by definition, the ${\boldsymbol{N}}$ is a subspace of a vector space $X,$ 12 that $d_{1}$ and $d_{2}$ dimension of the quotient space $X/N.$ where $\phi(x)=x/(1+|x|).$ Prove ${\boldsymbol{L}}^{p}$ ${\mathrm{suppose~}}0<p<1$ and prove that every subspace ofnite codimension is dense i GSee Section 1.47.) Suppos , are metrics on ${}\propto\,d_{1}(x,y)=|x-y|,\ d_{2}(x,y)=|\phi(x)-\phi(y)|,$ is complete R R which induce the same topology, although ${\mathcal{A}}_{1}$ $\textstyle{\mathcal{R}}$ ${\it1}{\it3}$ and $d_{2}$ is not. $\scriptstyle[0,\,1]$ Define Let ${\boldsymbol{C}}$ bethe vector space ofall omplex continuous functions on $$ d(f,g)=\int_{0}^{1}{\frac{|f(x)-g(x)|}{1+|f(x)-g(x)|}}\,d x. $$ Let $(C,\,\sigma)$ be ${\boldsymbol{C}}$ P with the topology induced by this metric. Let $\scriptstyle(\mathbf{c},\tau)$ be the topological vector space defined by the seminorms $$ p_{x}(f)=|f(x)|\qquad(0\leq x\leq1), $$ id: in accordance with Theorem 1.37. 'is also o-bounded and that the identity map $\scriptstyle(\mathbf{c},\,\tau)$ i (の) Prove that id: Go) Prove that every r-bounded set in ${\boldsymbol{C}}$ or Theorem 1.32.)Show also directly tha not metrizable.(See Appendix ${\mathrm{As}}.$ therefor caris bounded sesinto bounded sets $(C,\tau)\to(C,\sigma)$ is nevertheless not continuous, although it is sequen- $(C,\tau)\to(C,\,\sigma)$ tialycotinuous by Lebesgue's dominated convergenc theorem). Hence $\scriptstyle(C,\tau)$ has no countable local base is of the form (eo Prove that every continus inear functional on $(C,\,\tau)$ $$ f\to{\frac{x}{z-1}}c_{i}f(x) $$ for some choice of $x_{1},\ \cdot\cdot\cdot\cdot\cdot\cdot x_{n}$ in [0,1J and some $c_{i}\in C.$ and ${\boldsymbol{C}}.$ (d) Prove that (C, o) contains no convex open sets other han $\widetilde{{\mathcal D}}$ 14 Put (e) Prove that id $\cdot(C,\sigma)\cdots(C,\tau)$ is not continuous if $D=d/d x\colon$ $K=[0,1]$ and define ${\mathcal{D}}_{K}$ as in Section 1.46. Show that the following treefamilie ${\mathcal{O}}_{K}\,,$ of seminorms (where n =0,1,2,... define the same topology on (a) $\|D^{*}f\|_{x}=\operatorname*{sup}\left\{|D^{*}f(x)|:-\infty<x\right\}.$ < oo}. ( $\begin{array}{l}{{\vphantom[]D^{n}f||_{1}=\int_{0}^{1}\left|D^{n}f(x)\right|\ .}}\end{array}$ 女. ${\boldsymbol{I}}{\boldsymbol{G}}$ (c) $\|D^{n}f\|_{2}=\left\{\int_{0}^{1}|D^{n}f(x)|^{2}\,d x\right\}^{1/2}.$ $(\mathbf{k}_{\mathrm{{B}}},$ as ${\boldsymbol{\mathit{1}}}\o{\widehat{S}}$ Prove that the spaces C(D) (Section 144 do not have the Heine-Borel property Prove that the topology of C(OD) does not depend on the particular choice of iongas hiseuenceatsfes theonitiospcified nSection 1.4. Do the same for $C^{\infty}(\Omega)$ (Section 1.46)ToPoLoOICAL vEcroR SPACEs 39 $I{\mathcal{S}}$ $C^{\infty}(\Omega)$ and also of ${\mathcal{D}}_{K}$ into ${\mathcal{D}}_{K},$ f-tostit io scton .4. oethat - D/is onusmpping $C^{\infty}(\Omega)$ into $I{\bar{Y}}$ The seminorms for every multi-index . $$ p_{n}(f)=\operatorname*{sup}\left\{|f(x)|:-n\leq x\leq n\right\} $$ induce the metric $$ .\qquad\qquad\ d(f,g)=\sum_{n=1}^{\infty}\,{\frac{2^{-n}p_{n}(f-g)}{1+p_{n}(f-g)}} $$ in the spac $c(R);$ compare Section 1.46 and remark $\mathbf{\Psi}(c)$ of Section 1.38.Define $$ f(x)=\operatorname*{max}\,(0,\,1-|x|),\qquad\theta(x)=\log(x-2),\qquad2h=f+g, $$ and compute that $$ d(f,0)=\frac{1}{2},~~~~~d(g,0)=\frac{50}{101},~~~~~~d(\dot{h},0)=\frac{1}{6}+\frac{50}{102}. $$ $2\theta$ A: fined, that The balls with radius $\textstyle{\frac{1}{x}}$ are therfore not convex, alihough ${\mathbf{}}r$ are convex? is compatible with the space, and 19 Suppose For eac real number ${\mathbf{}}T$ and each integer ${\boldsymbol{n}}_{\mathrm{{J}}}$ deine $e_{n}(t)=e^{i n t},$ ${\mathcal A}_{d}$ $x,\,Y$ is an ${\boldsymbol{F}}_{-}$ and linear $M\to Y\ i$ usual ocaly cnvex topology of ${\mathcal{C}}(R),$ $\tilde{\Lambda}$ is linear and continuous. and $x_{n}\in(x+\ {\mathcal{V}}_{n})\cap{\mathcal{M}},$ $X)$ show that $\mathcal{M}$ Is there any $\scriptstyle r\leq1$ for which the balls of radius ${\mathrm{k}}\colon X\to Y$ such that $V_{n}+\,V_{n}\subset\,V_{n-}$ (xo)is a Cauchy seuence i $\textstyle Y,$ is a dense subspace of a tological vector spac $\operatorname{ff}x\in X$ $\textstyle X$ is well de- Suggestion: Let ${\mathit{V}}_{n}$ is conus ctit tuetogy ihat M neitstro and such that d(O, Prove that has a continus inear xenso . b balanced neighborhoods of o in $\mathrm{A}x)<2^{-n}\operatorname*{if}x\in M\cap V_{n}\ .$ $\Lambda x=\Lambda x$ if and define ${\bar{\Lambda}}x$ to be its limit. Show that $\bar{\Lambda}$ $x\in M,$ and that and define $$ f_{n}=e_{-n}+n e_{n}\qquad(n=1,\,2,\,3,\,\ldots). $$ $L^{2}$ space of $L^{2}$ Regard thcse funcions s members o $L^{\geq}(-\pi,\pi).$ Let $X_{1}$ be the smallest closed sub that contains that contains $\varphi_{0}\;,\;e_{1},\;e_{2}\;,\;.\;.\;.$ and let $X_{2}$ bo the smallcst closed subspace o instance, the vector $f_{1},f_{2}\,,f_{3}\,,\,\ldots.$ Show that $X_{1}+X_{2}$ is dense in $L^{2}$ but not closed. For $$ {\bf\nabla\cdot\nabla}\quad x=\sum_{n=1}^{\infty}n^{-{\bf1}}{\varrho_{-n}} $$ 21 is in $L^{2}$ but not in $X_{1}+X_{2}.$ (Compare with Theorcm 1.42.5 and $f(x)=1$ $X.$ Prove that there is a real $\textstyle X$ is a Let ${\mathbf{}}V$ be a neighborhood $\boldsymbol{f}$ on $X$ such that $f(0)=0$ and that Construct f as in he proof of (Thus of O such that $V_{1}+V_{1}<V$ and n a tologcal vetorspac $\mathbf{0}$ outside ${\mathit{V}}.$ continuous function $\mathrm{OI}\mid$ complerely reula toloica space.)Snestion: Let $V_{n+1}+V_{n+1}\subset V_{n}\,.$ be balanced ncighborhood $V_{n}$ Theorem 1.24. Show hat fis cotnuous a $$ |f(x)-f(y)|\leq f(x-y). $$40 GENERAL THEORY 22 1 fis a complex functondfined on te compact intera $I=[0,1]\in R,$ define $$ \omega_{o}(f)=\operatorname*{sup}\;\{\;|f(x)-f(y)|:|x-y|\leq\delta\,,\,x\in I,\,y\in I\}. $$ If $0<\alpha\leq1,$ the coresondn Lischt space Lp cosists ${\mathfrak{a l}}f{\tilde{Y}}$ for which $$ |f||=|f(0)|+\operatorname*{sup}\left\{\delta^{-x}\omega_{\delta}(f);\,\delta>0\right\} $$ is finite. Define $$ {\mathrm{ip~}}\alpha=\{f\epsilon\,{\mathrm{Lip~}}\alpha:\operatorname*{lim}_{\delta\,\ldots\,\delta}\delta^{-\,x}\omega_{\sigma}(f)=0\}. $$ $2{\ 3}$ $\scriptstyle\epsilon\in X$ Let r be the topology on $\textstyle X$ Provetat Lip xisaBah spae an tht i z sa loed sbspac o Lp z $|g(x)-f(x)|<r\operatorname{for}a|1\,x\in(0,\operatorname{I})$ Let and $r>0,$ let $\scriptstyle\nu(J,r)$ that these sets be kectosis oricinus ftitns n t os emn .2,Tr enerae Show tha adion is- $\textstyle X$ ocons ofallg e Xsuch that $V(f,r)$ 24 that $\textstyle A$ Show that the set continuous but scalar multiplication is not. is convex. $\mathcal{W}$ fhacsm e ior herm 14 e o covexa need not be balanced unless $U$