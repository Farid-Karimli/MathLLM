1 COMMUTATIVE BANACH ALGEBRAS as will the fact that the scalar ficld is ${\mathcal{C}}.$ This chapter deals primarily with the Gelfand theory of commutative Banach algebras alilhough some of the results of ths theory wilbe applied to noncommutative situa tions. The terminology of the preceding chapter wil be used without change. In particular, Banach algebras will not be assumed to be commutative unless ths is explicily tated, but the presence of a unit will be assumed without special mention Tdeals and Homomorphisms an ideal if 11.1 Definition A subst $\*{*}_{J}^{*}$ of a commutative complex algebra $\scriptstyle A$ l is said to be (a)J is a subspace of $\textstyle{\mathcal{A}}$ (in the vector space sense), and $(b)\quad x y\in J$ whenever xe A and yeJ. contained in any larger proper ideal IfJ≠4, Jis a proper idcal. Maximal ideals are proper ideals which are no264 BANACH ALGEBRAS AND sPrCTRAL THroRv 11.2 Proposition (b) (a)No proper ideal of A contains any invertible element of A ,then its closure J is also an J J is an ideal in a commutatine Banach algebra ${\bar{A}},$ ideal. The proofs are so simple that they re left as an excrcise 11.3 Theorem (a) I/ A is a commutative complex algebra with unit, then every proper ideal of A s contained in a maximal ideal of $A.$ (b) If A is a commutative Banach algebra, then every maximal ideal of A is closed PROOra) Let $\boldsymbol{J}$ be a proper ideal of ${\bar{A}}.$ Let ${\mathcal{P}}$ be the collection of all proper bea maximal ideals of $\textstyle{A}$ that contain J. Partially order ${\mathcal{P}}$ by set inclusion, let $\textstyle{\mathcal{Q}}$ totally ordered subcollection of ${\mathcal{P}}$ (the existence of ${\mathcal{Q}}$ is assured by Hausdorff's maximality theorem), and let $\bar{M}$ T be the union of all members of ${\mathcal{Q}}.$ Being the union of a toally ordered collcton of idcals, $\ M$ 1 is an ideal. Obviously $J\subset M,$ and $M\neq A$ since no member of ${\mathcal{O}}$ contains the unit of $\textstyle{\mathcal{A}}$ The maximality of $\mathcal{Q}$ implies that $\bar{M}$ is a maximal ideal of $\textstyle{\mathcal{A}}$ element of (6) Supposc $\bar{M}$ is a maximal ideal in $A.$ Since $\bar{M}$ contains no invertible contains $\textstyle A$ and since the set of ll nvertible elements is open, $\overline{{M}}$ no invertible element either. Thus $\overline{{M}}$ is a proper ideal of $A.$ and the maximality of $\bar{M}$ shows therefore that $M={\widetilde{M}}.$ // 11.4 null space or kernel of $\phi~~~~~~~~~~~~~~~~~~~~~~~~~~~~~~~~~~~~~~~~~~~~~~~~~~~~~~~~~~~~~~~~~~~~~~~~~~~~~~~~~~~~~~~~~~~~~~~~~~~~~~~~~~~~~~~~~~~~~~~~~~~~~~~~~~~~~~~~~~~~~~~~~~~~~~~~~~~~~~~~~~~~~~~~~~~~~~~~~~~~~~~~~~~~~~~~~~~~~~~~~~~~~~~~~~~~~~~~~~~~~~~~~~~~~~~~~~~~~~~$ Homomorphisms and quotient algebras $A/J$ is a Banach space, with respect to the $\pi\colon{\mathcal{A}}\to A/J$ is the Banach algcbras and $\phi$ is obviously an ideal in ${\cal A},$ If $\textstyle A$ and ${\boldsymbol{B}}$ are commutative is a homomorphism o $\dot{A}$ into $\boldsymbol{B}$ (see Section 10.4) then the Conversely, suppose $\boldsymbol{\jmath}$ is a proper closed ideal in , which is closed if d and $\phi$ is continuous. $\scriptstyle A$ quoticnt map, as in Definition 1.40. Then quotient norm (Theorem 1.41).We will show that $A/J$ is actually a Banach algebra and that r is a homomorphism. If $x^{r}-x\in J$ and $y^{\prime}-y\in J,$ uhe identity (1) $$ x^{\prime}y^{\prime}-x y=(x^{\prime}-x)y^{\prime}+x(y^{\prime}-y) $$ shows that $x^{\prime}y^{\prime}-x y\in J;$ hence $\pi(x^{\prime}y^{\prime})=\pi(x y).$ Multiplication can therefore be unambiguously defined in $A/J$ by (2) $$ \pi(x)\pi(y)=\pi(x y)\qquad(x\in A,y\in A). $$ $\mathrm{{It}}$ is then easily verified that $A/J$ is a complex algebra and that r is a homomorphism Since $\|\pi(x)\|\leq\|x\|,$ by the definition of the quotient norm, r is continuous.CoMMUTATVE BANACH ALGEBRAS 265 Suppose $x_{i}\in C$ A $(i=1,2)$ and $\delta>0$ . Then (3) $$ \|x_{i}+y_{i}\|\leq\|\pi(x_{i})\|+\delta\qquad(i=1,2) $$ for some $y_{i}\in J_{i}$ by he definition of the quotient norm. Since $$ (x_{1}+y_{1})(x_{2}+y_{2})\in x_{1}x_{2}+J, $$ we have (4) $$ \|\pi(x_{1}x_{2})\|\leq\|(x_{1}+y_{1})(x_{2}+y_{2})\|\leq\|x_{1}+y_{1}\|\,\|x_{2}+y_{2}\|, $$ so that O) implies the multiplicative inequality (5) $$ \|\pi(x_{1})\pi(x_{2})\|\le\|\pi(x_{1})\|\,\|\pi(x_{2})\|\,. $$ x∈ A, Finally,ife is the unit element of ${\bar{A}},$ ,then (2) shows that Since $\|\pi(x)\|\ \leq\ \|x\|$ for every $A/J,$ and since $\pi(e)\neq0,\;($ S) shows that $\|\pi(e)\|\geq1=\|e\|$ $\scriptstyle\pi(\epsilon)$ is th unit o $\|\pi(e)\|=1$ This completes the proof. Part(o) of the next therem is one of the key facts of the whole theory The sct A hat appers in wilater be given a compact Hausdof tology CTheore 11.9. The study of commtative Banach algebraswi hent a arextent b reduced o he study of more famiar and more specan bct. namey, aigebra of continuous complex fuctions on . with pointwise adtonand muiticaton Howcver. horm 1. as nesigceiseieemwiou h introduction of ths topology Secions 1.6 and 1.7 ustateths pon all complex homomorphisms of A. 11.5 Theorem Let A be a commuaie ach alebr, amd le $\hat{\boldsymbol{\Delta}}$ be the set o (c) (d)An elemen $\scriptstyle x\in{\mathcal{A}}$ Q Exery maximl iul is herel o som $h(x)=\lambda f o r$ some $h\in\Delta.$ $h\in\Delta.$ ${\mathit{h}}\subset\Delta$ (e) (6b)Jf heA, the kernel o his a maximal ideal of A 入e o(ox) if and only if An element xe A is iertible im Aif and onyif MAx) + 0 for cvery is iwerible m Aif and only f x lies in opoper ideal of A. PROOF (a) Let $\bar{M}$ r be a maximal ideal of $A.$ Then $\ M$ is closed (Theorem 11.3 and $z\mathbb{I} \langle M \rangle$ is therefore a Banach algebra. Choose $x\in A,\;x\notin M,$ and put (1) $$ \cdot J=\{a x+y\colon a\in A,y\in M\}. $$ Then $\boldsymbol{J}$ is an ideal in $\scriptstyle{\mathcal{A}}$ which is largcr than $M,$ since xe J.(Take $a=e,y=0.$ $\mathcal{M}$ Thus $J=A,$ ,and ax + 9 = e for some of $A/M$ onto ${\bar{C}}.$ Put $h=j\circ\pi.$ Then $h\in\Delta,$ and algebra there is an isomorphism $j$ $a\in A,y\in M$ If ${\mathfrak{:}}\pi\colon A\to A/M$ is the quotient $A/M$ is therefore invertible in $A/M.$ Every nonzero element r(xy of the Banach map,it follows that $\pi(a)\pi(x)=\pi(e).$ By the Gelfand-Mazur theorem, is the null space of h266 BANACH ALGEBRAS AND SPECTRAL THEORY (b) If $h\in\Lambda_{!}$ then $h^{-1}(0)$ is an ideal in $\scriptstyle{A}$ which is maximal because it has codimension ${\textbf{I}},$ (c) If xis invertible in $\textstyle A$ and $h\in\Delta,$ then $$ h(x)h(x^{-1})=h(x x^{-1})=h(e)=1, $$ so that $h(x)\neq0.$ If xis not invertible, then the set {ax: $a\in A\}$ does not contain e, hence is a proper ideal which lies in a maximal one (Theorem 11.3) and which is therefore annihilated by some $h\in\Delta,$ because of $(a).$ (d) No invertible element lies in any proper ideal. The converse was proved in the proof of $(c).$ in place of $\scriptstyle{\mathcal{X}}$ // (e) Apply (c) to $\lambda e-x$ Our first application concerns functions on $\textstyle R^{n}$ that are sums of absolutely convergent trigonometric series. The notation is as in Exercise $22$ of Chapter $7.\qquad\qquad\qquad\qquad$ 11.6 Wiener's lemma Suppose fis a function on $R^{n},$ and (1) $$ f(x)=\sum a_{m}e^{i m\cdot x},\qquad\sum|a_{m}|<\infty,\qquad\qquad\qquad. $$ where both sums are extended over $a l l\;m\in Z^{n}.$ If f(x) ≠ 0 for every xe $\textstyle R^{n},$ then (2) $$ \frac{1}{f(x)}=\sum b_{m}e^{i m\cdot x}\;\;\;\;\;\;\;\;w i t h\;\;\;\Sigma\;|b_{m}|<\infty. $$ PRooF Let $\textstyle A$ be the set of functions of the form (1), normed by $\|f\|=\sum|a_{m}|.$ One checks easily that $\textstyle A$ is a commutative Banach algebra, with respect to pointwise multiplication.Its unit is the constant function I. For each $X,$ the evaluation $f\to f(x)$ is a complex homomorphism of $A.$ The assumption about the given function $\boldsymbol{\mathit{f}}$ is that no evaluation annihilates it. If we can prove that $\dot{A}$ has no other complex homomorphisms, Cc of Theorem ll.5 will imply that fis invertible in ${\bar{A}},$ which is exactly the desired conclusion For $r=1.$ ….,n, put $g_{r}(x)=\exp\,(i x_{r}),$ where $x_{r}$ is the rth coordinate of $X.$ Then ${\mathcal{O}}_{r}$ , and $\textstyle{\sum}\not\varrho_{r}$ are in $\scriptstyle A$ and have norm 1. If $h\in\Delta,$ it follows from (c) of Theorem 10.7 that $$ |h(g_{r})|\le1\qquad\mathrm{and}\qquad\left|\frac{1}{h(g_{r})}\right|=\left|\left.h (\frac{1}{g_{r}}\right)\right|\le1. $$ Hence there are real numbers ${\mathbf{}}y_{r}$ , such that (3) h(g,)= exp (iy,) = 9,0) (1 ≤r≤n) where y = 0,…,y). If Pis a trigonometric polynomial (which means, bycowsurArIVE DANACu ALCDxAs 267 of the functions ${\mathcal{G}}_{r}$ and $\scriptstyle16\rho,$ then (3) implies Pis a finite linear combination of products of integral powers definition, that ${\boldsymbol{P}}$ (4) $$ h(P)=P(y), $$ Thus $\dot{h}$ because his linear and multiplicative. Since $\dot{\boldsymbol{h}}$ is continuous on $\textstyle{\mathcal{A}}$ (Theorem $10.7)$ is evaluation at ${\mathbf{}}y_{*}$ and since the set of all trigonometric polynomials is dense in $\textstyle A$ (as is obvious // from the definition of the norm),(4) implies that $h(f)=f(y)$ for every fe A. and the proof is complete. This lemma was used(with $n=1\mathrm{)}$ in the original proof of the tauberian theorem "as being embedded 9.7 se the conecion let reinteret the emma. Regard $\textstyle{Z^{n}}$ in $R^{n}$ in the obvious way. The given coeficients $\alpha_{m}$ define then a measure $\boldsymbol{\mathit{I}}$ on $R^{n},$ concentrated on $\textstyle{Z^{n}},$ which assigns mass ${\mathcal{Q}}_{m}$ to each $\mathbb{Z}^{n},$ , such that the convolu- $m\in Z^{n}$ Consider the problem of finding a complex measure ${\mathcal{O}}_{},$ , concentrated on tion $\textstyle\mu*G$ is the Dirac measure ${\bar{\delta}}.$ Wiener's lemma states that this problem can be solved if (and trivially only if) the Fourier transform of $\boldsymbol{\mathit{l}}$ has no zero on $\scriptstyle R^{n}\,\mathrm{:}$ this is precisely the tauberian hypothesis in Thcorem 9.7. product of $\;n$ For our next application, let $1\leq i\leq n.$ In other words, this polydisc We define ${\mathcal{A}}(U^{n})$ $U^{n}$ is the cartesian in ${\mathcal{C}}^{n}$ such that $|z_{i}|<1$ copies of thc open unit disc $U_{\mathit{l}}$ l in ${\boldsymbol{C}}$ ” be the set of all points $z=(z_{1},\ldots,z_{n})$ $U^{n}$ for to be the set of all functions fthat are holomorphic in $U^{n}$ (see Definition 7.20) and that are continuous on its closure ${\overline{{U}}}^{n}$ 11.7 Theorem Suppose $f_{1},\,\cdot\,\cdot\,\cdot\,\cdot\,\mathcal{f}_{k}\in\mathcal{A}(U^{n})\mathcal{,}$ and suppose that to each $z\in{\bar{U}}^{n}$ there $\phi_{k}\in{\mathcal{A}}(U^{n})$ corresponds at least one $\hat{\boldsymbol{l}}$ such that $f_{i}(z)\neq0$ Then there exist functions $\phi_{1},\ldots,$ such that (1) $$ f_{1}(z)\phi_{1}(z)+\cdot\cdot\cdot+f_{k}(z)\phi_{k}(z)=1\;\;\;\;\;\;\;(z\in\bar{U}^{n}). $$ PROOF $A=A(U^{n})$ is a commutativc Banach algcbra, with pointwisc multi- $\phi_{i}\in A$ Then $\boldsymbol{J}$ plication and the supremum norm. Let $\boldsymbol{J}$ be the set of all sums $\textstyle\sum f_{i}\phi_{i},$ with is an ideal. If the conclusion is false, then $J\neq A;$ hence $\boldsymbol{J}$ lies in some maximal ideal of $\textstyle A$ (Theorcm 11.3), and some $h\in\Delta$ annihilates ${\boldsymbol{J}},$ by $\left|w_{r}\right|\leq1.$ (a) of Theorem 11.5 $w=(w_{1}\,,\,\cdot\,\cdot\,\cdot\,\cdot\,w_{n}).$ Then $w\in{\overline{{U}}}^{n},$ and $h(g_{r})=g_{r}(w).$ It follows with For $1\leq r\leq n,\;\;\mathrm{put}^{*}g_{r}(z)=z_{r}.$ Then $\|g_{r}\|=1\,;$ hence $h(g_{r})=w_{r},$ Put that $h(P)=P(w)$ for every polynomial ${\boldsymbol{P}},$ since $\ \ \!\left(\begin{array}{l l}{h\!}&{}\end{array}\right)$ is a homomorphism. The poly- pothesis nomials are dense in ${\mathit{A}}(U^{n})$ (Exercise 4).Hence $h(f)=f(w)$ for every $f\in{\mathcal{A}},$ Since $\boldsymbol{\mathit{h}}$ annihilaes by essetilyth same argument that was used in the proof of Theorem 1.6 71 $J,\;f_{i}(w)=0$ for l≤i≤k. This contradicts the hy-268 wANAcH ALOrnRAs AND sPECTRAL THEonv Gelfand Transforms 11.8 Definitions Let $A.$ The formula tive Banach algebra $\Lambda$ be the se f all complex homomorphisms of a commuta (1) $$ \hat{x}(h)=h(x)\qquad(h\in\Delta) $$ assigns to each $\hat{\mathbf{A}}$ be the set of al $A_{\cdot}$ 4, denoted by rad ${\hat{x}}\colon\Delta arrow C;$ we call $\hat{X}$ the Gelfand transform of x is the weak $\Lambda.$ $\textstyle A$ $x\in A$ a function Let ${\mathfrak{b y}}{\tilde{A}},$ that is thewakest topology that makes every $\hat{\mathbf{x}}$ continuous topology induced ${\hat{x}},$ for $x\in A$ The Gelfand topology of $\Delta$ Then obviously ${\bar{A}}\subset C(\Delta),$ the alger o allcmplex continuous functions on $x\to{\bar{x}}$ of $\scriptstyle A$ onto A. of A. If rad and the members of A(Theorem 11.5), Sice e i 。 ont-oeocitievewememaial' ualy caled the maximal ideal space of $\textstyle\Delta_{\mathrm{{,}}}$ equipped with its Gelfand topology is $A=\{0\},$ ${\cal A}.$ , is the intersection of all maximal ideals The radical of The term"“ Gelfand transform "is alsoapled to the mappi ${\bar{A}},$ A is called semisimple 11.9 Theorem Let L be hemaximalidelsace ola commtative ach algebra A () only if A $\dot{\boldsymbol{\jmath}}$ (a)Ais a compact Hfausdorf space $\textstyle{\mathcal{A}}$ 4. The Gefand tranform is therefore an isomorphsm i una $\hat{\mathbf{A}}$ 01 $C(\partial),$ whose kermel is rad The Gelfamd Iransform is a homomorphism of $\textstyle A$ onto a subalgebra semisimple (c)For each $x\in A.$ the range of ${\hat{X}}$ is the spectrum $\sigma(x)$ Hence $$ \|{\hat{x}}\|_{\infty}=\rho(x)\leq\|x\|, $$ where $\left\|{\hat{X}}\right\|_{\infty}$ is he maximum of|xuh)| on ${\boldsymbol{\Delta}},$ and xe rad $\scriptstyle A$ if and only $~j f\rho(x)=0.$ PROOF We first prove (b) and $(c).$ Suppose $x\in A,\,y\in A,\,\alpha\in C,\,h\in\Delta.$ Then (ax)"Ch = (αx) = αA(x) = (αx)(0) (x+ J)0)= Mx +)一 A(x) + 6y) = *4) + 9(6) = (8 + 9)(1) and $$ (x y)^{\times}(h)=h(x y)=h(x)h(y)={\hat{x}}(h){\hat{y}}(h)=({\hat{x}}{\hat{y}})(h). $$ $\longrightarrow\overbrace{\mathbb{Z}}$ Thus $x\to{\hat{X}}$ is a homomorphism.Its kernel consists of those $\scriptstyle1,5.$ this is the intersection of all maximal which satisfy $h(x)=0$ for every $h\in\Delta\colon$ by Theorem $x\in{\mathcal{A}}$ ideals of ${\bar{A}}_{;}$ that is, rad ${\cal A}\,.$ (b) and (c) To say that $\lambda$ is in the range of Q means that $\lambda={\mathcal{R}}(h)=h(x)$ for some $h\in\Delta.$ By (e) of Theorem 1.5, ths hapnsifand only if $\lambda\in\sigma(x).$ This provescoMMUTATIVE BANACH ALOEBRAS 269 of $\Delta$ Let To prove $(a),$ is evidently terestriction t $\hat{\boldsymbol{\Delta}}$ be the dual space of $A^{\ast}.$ By the Banach-Alaoglu theorem It is therefore and let $\textstyle K$ let $A^{\star}$ $\textstyle{\mathcal{A}}$ 1 (regarded as a Banach space) be the norm-closed unit ball f $\textstyle K$ $\mathrm{\A}_{0}$ is weak*-compact. By (c) of Theorem $10.7,$ $\Delta<K$ The Gelfand topology enough to show that be in the weak*-closure o $\Delta.$ of the weak*-topology of $A^{\cong}.$ $A^{\textbf{x}}$ $\Delta$ is a weak*-closed subset of We have to show that (2) $$ \begin{array}{c c c}{{\circ}}&{{\quad\Lambda_{0}(x y)=\Lambda_{0}\,x\Lambda_{0}y\qquad(x\in A,y\in A)}}\\ {{\qquad.}}&{{\qquad\qquad\Lambda_{a}e=1.}}\end{array} $$ (1 and which is not in A.] [INote that (2) is necessary; otherwise $\Lambda_{0}$ would be the zero homomorphism ${\mathrm{Fix~}}x\in A,\,y\in A,\,:\,k$ : > 0. Put (3) $$ W=\{\Lambda\in A^{*}\!:\,|\Lambda z_{i}-\Lambda_{0}\,z_{i}\,|\,<\varepsilon\,{\mathrm{for~}}1\,\leq i\leq4\}, $$ where $z_{1}=e,$ $z_{2}=x,$ $z_{3}=y$ $z_{4}=x y$ Then ${\mathcal{W}}$ is a weak*-neighborhood of $\mathbb{A}_{0}$ which therefore contains an $h_{i}\in\Delta.$ For this ${\dot{h}}_{\circ}$ (4) $$ |1-\Lambda_{\mathrm{o}}\,e|\,=\,|\,h(e)-\Lambda_{\mathrm{o}}\,e|<\varepsilon, $$ which gives (2), and $$ \begin{array}{c}{{\Lambda_{0}(x y)-\Lambda_{0}\,x\Lambda_{0}y=[\Lambda_{0}(x y)-h(x y)]+[h(x)h(y)-\Lambda_{0}\,x\Lambda_{0}y]}}\\ {{=[\Lambda_{0}(x y)-h(x y)]+[h(y)-\Lambda_{0},y]h(x)+[h(x)-\Lambda_{0}\,x]\Lambda_{0}y,}}\end{array} $$ which gives (5) $$ |\Lambda_{0}(x y)-\Lambda_{0}\,{x\Lambda_{0}}\,y|<(1\,+\,\|x\|\ +\,\|\Lambda_{0}\,y\,\|)\rangle\varepsilon. $$ Since (S) implies (I), the prof s complete Semisimlealgebras ave an importnt property which was arir provcd fo ${\cal{C}};$ $\boldsymbol{B}$ 11.10 Thcorem If y: $B\to A$ is a homomorphism of commutative Banach algebra into a semisimplecommutative Bamach algebra ${\cal A},$ then $\psi$ is continuows PROoF Suppose $x_{n}\to X$ in B and Wx)→yin A. By the closedgraph theorem $h\in\Delta_{A}{\mathrm{~;}}$ put Let it is enough to show tha and $\textstyle\Delta_{B}$ be the respective maximal ideal spaces. Fix $y=\psi(x).$ $\textstyle\Delta_{A}$ $\phi=h\circ\psi.$ Then $\phi\in\Delta_{B}$ By Theorem 10.7, $\boldsymbol{\hbar}$ and $\phi$ are continuous. Hence $$ h(y)=\mathrm{Iim}\ h(\psi(x_{n}))=\mathrm{Iim}\ \phi(x_{n})=\phi(x)=h(\psi(x)), $$ for every he $\Delta_{A}.$ Hence y -wGx)erad A. Since rad 4=0}, =Wr 7270 BANACH ALGEBRAS AND SPtCTRAL THroxy Corollary Every isomorphism between two semisimple commutative Banach algebras is a homeomorphism. In particular, this is true of every automorphism of a semisimple commutative Banach algebra.The topology of such an algebra is therefore completely determined by its algebraic structure In Theorem 1.9, the algebra $\hat{\cal A}$ may or may not be closcd in $C(\partial)_{i}$ with rcspect to the supremum norm. Which of these cases occurs can be decided by comparing $\|x^{2}\|$ with $\|{\boldsymbol{X}}\|^{2};$ , for all $x\in A$ .Recall that $\|x^{2}\|\leq\|x\|^{2}$ is always true. 11.11 Lemma If A is $\overline{{a}}$ commutative Banach algebra and (1) $$ r=\mathrm{inf}\frac{\|x^{2}\|}{\|x\|^{2}},\qquad s=\mathrm{inf}\frac{\||\hat{x}|\|_{\infty}}{\|x\|}\qquad(x\in A,\,x\neq0), $$ then. $s^{2}\leq r\leq s$ PROOF Since ISllo≥s llxH (2) $$ \|x^{2}\|\geq\|{\hat{x}}^{2}\|_{\infty}=\|{\hat{x}}\|_{\infty}^{2}\geq s^{2}\|x\|^{2} $$ for every $x\in A$ Thus $s^{2}\leq r$ Since $\|x^{2}\|\geq r$ lxl for every $x\in A,$ induction on $\;n$ shows that (3) $$ \|x^{m}\|\geq r^{m-1}\,\|x\|^{m}\qquad(m=2^{n},\,n=1,\,2,\,3,\,\ldots,). $$ Take mth roots in 3) and let $m arrow\infty\,.$ By the spectral radius formula and (c) of Theorem 11.9, (4) $$ \|{\hat{x}}\|_{\infty}=\rho(x)\geq r\|x\|\ \ \ \ \ \ (x\in A). $$ Hence $r\leq s$ 11.12 Theorem Suppose $\scriptstyle A$ is ${\mathcal{A}}_{}$ i commutative Banach algebra (a) The Gelfand transform is an isometry (that $i{\mathcal{S}},$ $\|x\|=\|{\hat{x}}\|_{x}$ for every $x\in A)$ $i f$ and only $i f\left\|x^{2}\right\|=\left\|x\right\|^{2}$ for every $x\in{\mathcal{A}}$ (b) A is semisimple and $\bar{A}$ is closed in C(A) if and only if there exists K<Oo such that llxll $<K\Vert x^{2}\Vert$ for every x∈ A. PROOF (a) In the terminology of Lemma 11.11, the Gelfand transform is an isometry if and only if $s=1$ , which happens (by the lemma) if and only if $\scriptstyle\gamma=1$ (6)The existence of $\textstyle K$ is equivalent to $\scriptstyle r\gg0$ D, hence $\mathbf{to}\ s>0.$ by the lemma $\Pi~s>0,$ then ${\bar{x}}\to{\bar{x}}$ is one-to-one and has a continuous inverse,so that $\hat{A}$ is complete hence closed) in ${\mathrm{C}}(\partial)$ Conversely, if $x\to{\mathcal{E}}$ is one-to-one and if $\widetilde{A}$ is closed in CA), the open mapping theorem implies that s > 0 7CoMMUTATIVE BANACH ALGEBRAS $\stackrel{\circ}{\doublebarwedge}\bigcap_{\Delta}$ 11.13 Examples In some case, the maximal ideal space of a given commutative Banach algebra can easily be described explicily.、 In others, extreme pathologies occur. We shall now give some cxamples to illustrate this. (a) Let $X$ be a compact Hausdorf space, put $A=C(X),$ , with the supremum of $\textstyle X$ norm. For $\operatorname{each}x\in X,f\to f(x)$ is a complex homomorphism $h_{x}\neq h_{y}.$ Thus $x\to h_{x}$ cmbeds $\textstyle X$ in $\Delta$ in ${\mathsf{C}}(X)$ which contains, for each $p\in X,$ I contains finitely many functions ${\mathcal{f}}_{1},$ $h_{x}\,.$ Since C(X) separates $\bar{M}$ points on X (Urysohn's lemma), $x\neq y$ implies We claim that each $h\in\Delta$ is an $h_{x}\,.$ ..If this is false, there is a maximal ideal implies then that $\bar{M}$ a function $f\operatorname{with}f(p)\neq0.$ The compactness $\ast\cdot\cdot J_{n}$ such that at least one of them is ${\overline{{\gamma}}}(0)$ at each point of $X.$ Put $$ g=f_{1}\bar{f}_{1}+\cdot\cdot\cdot+f_{n}\bar{f}_{n}. $$ Then $g\in M,$ since $\bar{M}$ is an ideal; $\scriptstyle g\;>\;0$ at every point of $X{\mathrm{:}}$ ; hence $\scriptstyle{\mathcal{G}}$ is invertible in C(X) But proper ideals contain no invertible elements Thus $x rightarrow h_{x}$ is a one-to-one correspondence between $\textstyle X$ and $\Delta$ and can be used to identify $\Delta$ A with $\textstyle X$ This identification is also correct in terms of the two topologies ${\mathsf{C}}(X)$ that are involved: The Gelfand topology y of $\textstyle X$ is the weak topology induced by $\scriptstyle\gamma$ is a Hausorf topology: yand is therefore weaker than t, the original one, but $\scriptstyle\gamma$ hence $\gamma=\tau.$ $\mathrm{[See~}(a)$ of Section 3.8.] and the Gelfand trans- summing up X“is" the maximal dal space o $C(X),$ form is the identity mapping on C(X). (b) Let $\scriptstyle A$ be the algcbra of all absolutely convergcnt trigonomtric series, as in Section 11.6. We found there that the complex homomorphisms are the evalua- tions at points of $\textstyle R^{n}.$ Since the members of $\textstyle A$ are 2zr-periodic in each variable, $\Delta$ is the torus $T^{n}$ obtained from $\textstyle R^{n}$ by the mapping $$ \big(x_{1},\ \cdot\cdot\cdot\ ,\ x_{n}\big) arrow\big(e^{i x_{1}},\ \cdot\cdot\cdot\cdot\cdot\,\rlap/e^{i x_{n}}\big). $$ This is an example in which $\hat{\cal A}$ is dense in $C(\Delta),$ although ${\hat{\lambda}}\neq C(\Delta)$ (c) In the same way, the proof of Theorem 11.7 contains the result that ${\bar{U}}^{n}$ is the maximal ideal space of $A(U^{n}).$ The argument used at the end of (a) shows that ; the the natural topology of $U^{n}$ is the same as the Gelfand topology induced by ${\mathit{d}}({\widehat{U}}^{n})$ same remark applies to (b). (d) The preceding example has interesting generalizations. Let A now be a means that $x_{i}\in A\left(1\leq i\leq n\right)$ commutative Banach algebra with a finite set of generators, say ${\mathcal X}_{1},\ \ \circ\ ,\ \ \chi_{n}\ .$ This and that the set of all polynomials in $X_{\texttt{t a c}*\ *\ *\ *\ldots\times}Y_{\downarrow{n}}\ .$ is dense in $A\,.$ Define (1) $\phi(h)=(\hat{x}_{1}(h),\cdot\cdot\cdot\hat{x}_{n}(h))\qquad(h\in\Delta)$272 BANACH ALC AND SPEC TRAL THEORY $$ \begin{array}{l l l l l}{{\begin{array}{l l l l l}{{\begin{array}{l}{{\begin{array}{l}{{\begin{array}{{\begin{array}{{\begin{array}{{\ ({\epsilon}}}}&{{}}}&{{}}&{{}}&{{}}&{{}}\\ {{\begin{array}{{{\frac{{\epsilon}}{\epsilon}}}}&{{{\epsilon}}}}\end{array}}}&{{}}\end{array}}}\\ {{\begin{array}{l}{{\left\begin{array}{l}{{ [{\begin{array}{l}{{\qquad}}}\end{array}}}\end{array}}}&{{}}\end{array}}}\end{array}}}\end{array}}}\end{array}\right) $$ Then $\phi$ is a homeomorphism of $\Delta$ onto a compact set $K\subset C^{n}$ Indeed $\phi~~~~~~~~~~~~~~~~~~~~~~~~~~~~~~~~~~~~~~~~~~~~~~~~~~~~~~~~~~~~~~~~~~~~~~~~~~~~~~~~~~~~~~~~~~~~~~~~~~~~~~~~~~~~~~~~~~~~~~~~~~~~~~~~~~~~~~~~~~~~~~~~~~~~~~~~~~~~~~~~~~~~~~~~~~~~~~~~~~~~~~~~~~~~~~~~~~~~~~~~~~~~~~~~~~~~~~~~~~~~~~~~~~~~~~~~~~~~~~~~~~~~~~~~~~~~~~~~$ is continuous since ${\bar{\lambda}}\in C(\Delta).$ If $\phi(h_{1})=\phi(h_{2}),$ then $h_{1}(x_{i})=h_{2}(x_{i})$ for all ${\dot{t}};$ hence $h_{1}(x)=h_{2}(x)$ ${\cal A},$ whenever $\textstyle{\mathcal{X}}$ is a polynomial in $X_{\mathbf{I}},\ \cdot\cdot\cdot,\cdot$ x, and since these polynomils are dense in $h_{1}=h_{2}$ Thus $\varnothing\;$ is one-to-onc. $\hat{\cal A}$ from $\Lambda$ O $\textstyle K$ and may thus regard ${\cal K}\,\,$ as the maximal We can now transfer ideal space of $A.$ To make this precise, define (2) $$ \psi(x)=\hat{x}\circ\phi^{-1}\qquad(x\in A). $$ algebra ${\mathcal{Y}}(A)$ of is a homomorphism (an isomorphism if $\textstyle A$ is semisimple) of $\textstyle A$ onto a sub- Then $\vartheta$ ${\mathcal{C}}(K)$ One verifies easily that (3) $$ \psi(x_{i})(z)=z_{i}\qquad{\mathrm{if~}}z=(z_{1},\dots,z_{n})\in K, $$ and therefore (4) $$ \psi(P(x_{1},\dots,x_{n}))(z)=P(z)\qquad(z\in K) $$ for every polynomial ${\boldsymbol{\mathit{P}}}$ in $\scriptstyle{n}$ variables. is a uiform limit of polynomials on $K.$ The sets lt follows that every member of ${\mathcal{Y}}(A)$ $K\subset{\mathit{C}}^{n}$ which arise in this fashion as maximal ideal spaces have a property known as polynomial convexity: $I f w\in C^{n}$ and $w\not\in K,$ there exists a polynomial ${\boldsymbol{P}}$ such that $|P(z)|\leq1$ for every z ∈ K, but $|P(w)|>1$ To prove ths, assume there is no such polynomial. The norm-decreasing property of the Gelfand transform implies then that (5) $$ |P(w)|\leq\|P(x_{1},\ \cdot\cdot\cdot,\,x_{n})\| $$ of for every polynomial ${\cal P};$ the norm is that of $\textstyle A$ Since $\{X_{1},\ \circ\ ,\ X_{n}\}$ is a set of generators $v\in K,$ ${\bar{A}},$ it follows from(5)thal Lhere is an $h\in\Delta$ such tha $\phi(h)=w$ But then and we have a contradiction The compact polynomially convex subsets of ${\boldsymbol{C}}$ are simply those whose com plement is connected; this is an easy consequence of Runge's theorem. In ${\mathcal{C}}^{n},$ the structur of the polynomially convex sets is by no means fully understood (e Our next example shows that the Gelfand transform is a generalization of the Fourier transform, at least in the ${\mathcal{L}}^{1}$ l-context. members of $\scriptstyle{\dot{A}}$ are of the form $f+\alpha\delta,$ with a unit atached, as described in $(d)$ of Section 10.3. The $\cdots{\begin{array}{l}{r_{+}}\end{array}}\,,}\,$ Let $\textstyle{\mathcal{A}}$ be $L^{1}(R^{n})$ $f\in L^{1}(R^{n}),\ \alpha\in\mathbb{C},$ and 0 $\delta$ is the Dirac where neasure on $R^{n}{}_{;}$ multiplication in A is convolution: $$ (f+\alpha\delta)*(g+\beta\delta)=(f*g+\beta f+\alpha g)+\alpha\beta\delta. $$ For each $\iota\in R^{*}.$ the formula (6) $$ h_{t}(f+\alpha\delta)=\hat{f}(t)+\alpha $$comMUTATiVE BANACH ALOEBRAs 273 addition, defines a complex homomorphism of ${\mathcal{A}}\colon$ here $\hat{f}$ is the Fourier transform of ${\f}.$ In (7) $$ h_{\alpha}(f+\alpha\delta)=\alpha $$ also defines a complex homomorphism. There are no others. (A prof willbesketched presently.)Thus $\Delta,$ as a set, is $R^{n}\cup\{\infty\}.$ as $|t|\to\infty,$ for every $f\in L^{1}(R^{n}).$ $(7).$ lf $h(f)=0$ Give △ the topology of the one-point compactification of $R^{n}$ Since ${\hat{f}}(t)\to0$ $\hat{A}$ separates points on $\Delta,$ , the weak topology induced on $\operatorname{\sinhy}{\hat{A}}$ lt remains to be proved that every $h\in\Delta$ it follows from (6) and(T) that ${\tilde{A}}\subset C(\Delta)$ Since is the same as the one that we just chose. is of the form (6) or for cvery $f\in L^{1}(R^{n})$ then $h=h_{\alpha}\,.$ Assume $h(J)\neq0$ for some $f\in L^{1}(R^{n})$ Then $h(f)=\textstyle\int f\beta\,d m_{n},$ for some $\beta\in L^{\infty}(R^{n}).$ Since $h(f*g)=h(f)h(g),$ one can prove that $\beta$ coincides almost everywhere with a continuous function ${\boldsymbol{b}}$ which satisfies (8) $$ b(x\div y)=b(x)b(y)\qquad(x,y_{*}\in R^{n}) $$ Finally, every bounded solution of $\left(\aleph\right)$ is of the form (9) $$ b(x)=e^{-i x\cdot t}\qquad(x\in R^{n}) $$ for some $\iota\in R^{n}$ Thus $h(f)={\hat{f}}(t)$ ), and $\boldsymbol{\mathit{h}}$ has the form (6) For $n\,=\,1.$ , the details that complete the preceding sketch may be found in Sec. 9.22 of [231. The case $\scriptstyle n\;\!>\;\!$ is quite similar. () Our final example is $L^{\infty}(m)$ is the usual Ranach space of equivalence classes (modulo $L^{\infty}(m)$ Here ${\boldsymbol{m}}$ is Lebesgue measure on the unit interval [O,1], and sets of measure O) of complex bounded measurable functions on [0,1], normed by the essential supremum. Under pointwise multiplication, this is obviously a commuta tive Banach algebra $\mathbb{F}\in L^{\infty}(m)$ and ${\cal G}_{f}$ is the union of all open sets ${\bar{0}}\subset{\mathcal{Q}}$ with $m({f}^{-1}(G))=0,$ then the complement of $\sigma(J)$ o ${\mathfrak{f}}_{f},$ hence with the range of its Gelfand transform.、It follows ${}^{'}\!f\!\!\!\in\!\rangle$ is easily seen to coincide $G_{f}$ (called the essential range of with the spectrum tha $\operatorname{it}{\hat{f}}$ is real if fis real. Hence $L^{\infty}(m)^{\times}$ is closed under complex conjugation. By the is the maximal is Sione-Weierstrass thcorcm, $\scriptstyle I\cdot\sigma(m)$ lt also follows that $f\to{\hat{f}}$ is an isometry,so that $\Delta$ $L^{\infty}(m)^{\times}$ ideal space of $L^{\infty}(m)^{\times}$ is therefore dense in ${\mathsf{C}}(\Delta)$ where closed in ${\mathsf{C}}(\Delta)$ We conclude that f→fis an isometry o $L^{\infty}(m)$ onto $C(\Delta).$ Next, ${\widehat{f}}\to\int f\,d m$ is a bounded linear functional on ${\mathsf{C}}(\Delta)$ By the Riesz repre- sentation theorem, there is therefore a regular Borel probability measure $t{\hat{\boldsymbol{\mu}}}$ on $\underline{{\land}}$ that satisfies (10) $\int_{\Delta}^{}{\hat{J}}\,d{\hat{m}}=\int_{0}^{1}\!f\,d m\,-\,\left[\,f\in L^{\infty}(m)\,\right].$274 BANACH ALGEBRAS AND SPECTRAL THEORY If $\underline{{\mathbf{Q}}}$ is a nonempty open set in ${\bar{f}}=0$ outside $\mathbb{Q}$ 2, and ${\hat{f}}(p)=1$ at some $p\in\Omega$ Hence fis no $\textstyle\Delta,$ Urysohn's lemma implies that there exists ${\hat{f}}\in C(\Delta),{\hat{f}}\geq$ 0, such that the zero element of $L^{\omega}(m)$ and the integrals (10) are positive. Thus m(Q) > 0if Q is open and nonempty Since Assume next that $\phi~~~~~~~~~~~~~~~~~~~~~~~~~~~~~~~~~~~~~~~~~~~~~~~~~~~~~~~~~~~~~~~~~~~~~~~~~~~~~~~~~~~~~~~~~~~~~~~~~~~~~~~~~~~~~~~~~~~~~~~~~~~~~~~~~~~~~~~~~~~~~~~~~~~~~~~~~~~~~~~~~~~~~~~~~~~~~~~~~~~~~~~~~~~~~~~~~~~~~~~~~~~$ is a Borel function on $\Delta,$ $|\phi|\leq1.$ By Lusin's theorem $L^{2}({\hat{m}})$ $f{\boldsymbol{\mapsto}}f$ [23] there are functions ${\tilde{f}}_{n}\in C(\Delta)$ $|\hat{f}_{n}|\leq1,$ that converge to $\phi$ in the norm of preserves complex conjugation (as we saw above) and is a homomorphism, it follows from (10), applied to $(f_{i}-f_{j})({\bar{f}}_{i}-{\bar{f}}_{j}),$ that (11) $$ \int_{\Delta}|\hat{f}_{i}-\hat{f}_{j}|^{2}\,d\hat{m}=\int_{0}^{1}|f_{i}-f_{j}|^{2}\,\,d m. $$ Thus conclusion is that $\phi=f$ $\mathrm{that}\,f_{n}{\to}f\mathrm{in~}L^{2}(m),$ and now (11) implies $\mathrm{that}\,\hat{f}_{n} arrow\hat{f}$ a.e. [m] Hence there exists $L^{2}({\hat{m}})$ The $\scriptstyle\{f_{n}\}$ is a Cauchy sequence in $L^{2}(m).$ AIso, $|f_{n}|\leq1$ $f\in L^{\infty}(m)$ a.e.[m] in such Every bounded Borel function $\phi$ on $\Delta$ coincides with some $f\in C(\Delta)$ a.e.[m] Thus This means, by definition, that $t h e$ are identical as Banach spaces! is extremally disconnected. ${\mathrm{c}}(s)$ and $L^{\infty}(\hbar)$ Another consequence of the last result is that $\Delta$ closure vy every open set is open.GHence disjoin ${\hat{f}}=\phi$ the continuity open sets have disjoint closures.) $\hat{\boldsymbol{f}}$ is neither O nor lis open and has measure let $\Omega_{1}$ be the complement of the closure ${\hat{f}}\in C(\Delta)$ so that $\mathbf{To}$ prove this, let $\Omega_{0}$ be open in $\Delta,$ ${\overline{{\boldsymbol{\Sigma}}}}_{0}$ of $\mathbb{Q}_{0}$ $\operatorname{off}$ shows tha be the characteristic function of at every $p\in\Omega_{0}$ Likewise $:j\left(p\right)=1{\mathrm{~if~}}p\in\Omega_{1}.$ a.e. [m] Since $\phi=0\,\mathrm{in}\,\Omega_{0}$ $\Omega_{1},$ and choose ,let $\phi$ and since nonempty open sets have positive measure, $\textstyle{\mathfrak{I}}\left(p\right)=0$ hence it is empty. Let $K_{i}=\{p\in\Delta\colon f(p)=i\},$ $i=0.$ 1. Then $0,$ , sinc $\mathrm{e}f=\iota$ p a.e. [m]; The set on which $K_{0}$ and $K_{1}$ are disjoin compact sets whose union is A.They are therefore open; also $\Omega_{0}\subset K_{0}\,,\,\Omega_{1}\subset K_{1}$ It follows that $\Omega_{\mathrm{o}}=K_{\mathrm{o}}$ , and the proof is complete. $0,$ We have also proved, incidentally, that boundaries of open sets have measure since $\hat{m}(\Omega_{0})=\hat{m}(K_{0})$ almost conuains ${\boldsymbol{E}}$ if ${\boldsymbol{\mathit{f}}}$ contains ${\boldsymbol{E}}$ except for a set of measure O, sets, let us say that We conclude with an application to measure theory. If ${\boldsymbol{E}}$ and ${\mathbf{}}F$ F are measurable ${\mathbf{}}F$ that is, if $m(E-F)=0$ The union of an uncountable collection of measurable sets is not always measurable. However, the following is true: $\mathbf{\Pi}_{\mathbf{n}=-\mathbf{\Sigma}}.$ set $I f\vee S_{\alpha}\}$ is am arbitrary olleciof measurable sets i 0, 1], hrc is ameasurable $E\subset[0]$ ,1 wih the Jollowing two properties: G) E almost contains every E almost contains every $E_{x},$ then ${\mathbf{}}F$ almost contains ${\boldsymbol{E}}.$ (iü)If Fis measurable and ${\mathbf{}}F$coMMUTATVE BANACH ALOEBRAs 275 $\bar{f}$ which Thus $\boldsymbol{E}$ is the least upper bound of ${\widetilde\Omega}.$ The desired set $\boldsymbol{E}$ is the set of al $x\in[0,1]$ at these $\scriptstyle\Omega_{n}$ Then is the characteristic finction of $\langle E_{n}\rangle$ The existence of $0)_{\vec{0}}$ is complete. ${\hat{f}}_{\alpha}$ is then the $\underline{{\otimes}}$ characterisi function of an open (and closebt st $\Omega_{\alpha}\subset\Delta$ Let ( $\Omega$ ${\boldsymbol{E}}$ implies that the Boles ler o meauabesmoduo s s mse Its Gelfand transform With the mchincry now at our ispsa theroo svey smp $E_{\alpha}$ Let f. be the characteristic function of is open, so is its closure ${\widetilde\Omega}.$ L,and there exis be the union of al $f\in L^{\infty}(m)$ such that $f(x)=1.$ Involutions 11.14 Definition A mapping and $\lambda\in C.$ of a complex (not necessariy commutative algebra $\textstyle A$ d into $x\to x^{*}$ if it has tefollowing four properties for all $\textstyle A$ is called an involution on $\scriptstyle{\dot{A}}$ $x\in A,\,y\in A,$ (1) $$ (x+y)^{\ast}=x^{\ast}+y^{\ast}. $$ (2) (2x)* = 兀x*. (3) $$ (x y)^{*}=y^{*}x^{*}. $$ (4)) $$ x^{**}=x. $$ Any $x\in A$ for which $x^{*}=x$ ln oue rs a noin nuneieiaiuomorhis o riod_ ). The onc that we will be most For example, $f\to{\overline{{f}}}$ is an involution on is called hermitian, or self-adjoin. $C(X)$ concerned wit latr s h pssage rom noperator on Hilber spac o sajoin 11.15 Theorem I Ais a Bamach algebra wtha iwoluiouy, u $\;{\hat{f}}\,x\in A,$ then a $x+x^{*},\,i(x-x^{*})\mathrm{,}$ ,and $x x^{n}$ are hermitian, with u∈ A, $v\in A,$ and both u and v (b) x has a wnique rcpresentation $x=u+i v,$ hermitian, (c) the uni $\scriptstyle{\mathcal{C}}$ is hermitian, (d) and xis inerible i if umd only f x* is inerible,in which cae $(x^{\kappa})^{-1}=(x^{-1})^{\kappa},$ e 入e o(x if and only if Xe ocxr*) Then both $\lambda$ Statemcnt (a) is obvious. ${\mathrm{If}}\,2u=x+x^{*}.$ ${}^{*},2v=i(x^{*}-x),$ then $x=u+i v$ PROOF is arepresentation as in (b) Suppose $x=u^{\prime}+i v^{\prime}$ is another one. Put $w=v^{\prime}-v.$ j and iw are hermitan,so that $$ i w=(i w)^{\ast}=-i w^{\ast}=-i w. $$ Hence w = $0_{\!_{J}}$ -and the uniqueness follows.276 BANACH ALGEBRAS AND SPFCTRAL THFORY Finally, e) follows if $(d)$ ,a) implies co;(d follows from $\left(c\right)$ and $(x y)^{*}=y^{*}x^{*}.$ // Since $e^{*}\underline{{\bigcup{-e}}}e^{\kappa}.$ is applied to $\lambda e-x$ in place of $\textstyle{\mathcal{X}}$ 11.16 Theorem If the Banach algebra $\textstyle A$ is commutative and semisimple, then every involution on A is continuous. phism. Hence $\varnothing\;$ be a complex homomorphism of $x_{n} arrow X$ and $x_{n}^{\ast}\to y$ in $\textstyle A$ , Then PROOF Let $\ \boldsymbol{h}$ ${\bar{A}}_{s}$ 4,and deline $\phi(x)=\bar{h}(x^{\ast}).$ Properties $(1)$ to (3) of Definition 11.14 show that $\phi$ is a complex homomor- is continuous. Suppose $$ \bar{h}(x^{*})=\phi(x)=\operatorname*{lim}\;\phi(x_{n})=\operatorname*{lim}\;\bar{h}(x_{n}^{*})=\bar{h}(y). $$ Since $\textstyle A$ is semisimple $y=x^{s}$ Hence $x\to X^{\mathbb{N}}$ is continuous, by the closed graph theorem ${\Big/}{\Big/}{\Big/}{\Big/}{\Big/}$ 11.17 Definition A Banach algebra $\textstyle A$ with an involution $x\to X^{\mathbb{R}}$ that satisfie (1) $$ \|x x^{*}\|=\|x\|^{2} $$ for every $x\in A$ is called a $B^{\mathrm{st}}.$ -algebra. implies $\left\|\,\chi\,\right\|\begin{array}{l}{{<}}\\ {{\bf\chi}}\end{array} \|\,\chi^{\star}\,\P$ l, hence also Note that $\|x\|^{2}=\|x x^{*}\|\leq\|x\|\ \|x^{*}\|$ $$ \|{\mathcal{X}}^{\star}\|_{\mathbf{\psi}}\leq\|_{X^{\star\star}}\|\ =\|_{\mathbf{\psi}}\chi\|. $$ Thus (2) $$ \|x^{*}\|=\|x\| $$ in every $B^{\ddagger}$ -algebra.It also follows that (3) $$ \|x x^{\textrm{s}}\|=\|x\|\ \|x^{\check{n}}\|. $$ be given in Chapter 12 Conversely,(2) and (3) obviously imply (1) Th foowingtcoremis thekey to the roof of he specta therm that wi 11.18 Theorem (Gelfand-Naimark)Suppose A is a commutative B*-algebra, with maximal ideal space A. The Gefand transform is then an isometric isomorphism of A onto C(A), which has the additional property that (1) $$ h(x^{*})={\overline{{h(x)}}}\qquad(x\in A,\,h\in\Delta),\qquad\qquad. $$ or, equivalently, that (2) (x*) =元 (x∈ A) $\mathit{I\!n}$ particular, xis hermitiamif ad only ifXis a real-valued fuction.comaUTATTVr DANACH ALGEBRAs 277 tion on $\scriptstyle A$ The interpretation of $\mathbf{\Psi}(2)$ is that the Gelfand transform carries the iven involu- isomorphis to the natural involution on ${\mathsf{C}}(\Delta)$ , which is conjugation. 1somorphisms that preserve involutions in this manner are oftncalle ${\mathfrak{S l}}_{k_{n}}$ real. Put PROor Assume first that $u\in{\mathcal{A}}$ $u=u^{*}$ $h\in\Delta.$ We have to prove that $\scriptstyle b(u)$ is $z=u+i e,$ for real t. If $^{*}h(u)=\alpha+i I$ β, with $\scriptstyle{\mathcal{A}}$ and $\beta$ real, then $$ {\it\Psi_{2}}(z)=\alpha\L+i(\l_{1}+t),~~~~~~z z^{\star}=u^{2}\L+t^{2}e, $$ so that $$ \begin{array}{l}{{x^{2}+(\beta^{\cdot}+t)^{2}=|h(z)|^{2}\leq||z||^{2}=||z||^{2}=\|z||\leq||u||^{2}+t^{2},}}\end{array} $$ or By (3), $\beta=0\,;$ ience hu) is real $$ \begin{array}{c c}{{\displaystyle{\mathcal{J}_{\cdot}\cdot^{2}\cdot^{2}+\beta^{2}+2\beta t}\leq||u||^{2}\qquad(-\infty<t<\infty).}}\end{array} $$ (3) If $x\in A$ , then $x.{\overset{\frown}{=}}u+i v,$ with $u=u^{*},$ $v=v^{*},$ Hence $x^{\mathrm{s}}=u-i v.$ Since $\hat{\mathcal{U}}$ and $\hat{U}$ 龙 are real,(2) is proved $\mathrm{\hat{l}}_{\star_{\mathrm{L}}}$ so that $\|y^{2}\|=\|y\|^{2}$ lt follows, by Thus $\hat{\mathbf{A}}$ is closed under complex conjugation. By the Stone-Weierstras :theorem, $\hat{\mathbf{A}}$ is therefore dense in ${\mathsf{C}}(\mathbb{A})$ $\operatorname{If}\ x\in A$ and $y=x x^{*}$ ,then $y=y^{*}$ that induction on ${\boldsymbol{n}},$ that $\|y^{m}\|=\|y\|^{m}$ for $m=2^{n}$ Hence $\|{\hat{y}}\|_{\infty}=\|y\|_{$ ,by the ${\hat{y}}=|{\hat{x}}|^{2}$ Hence spcctral radius formula and (e) of Theorem 11.9. Since $y=\,x x^{\kappa},$ (2) implies $$ \|{\hat{x}}\|_{\infty}^{2}=\|{\hat{y}}\|_{\infty}=\|y\|=\|x x^{*}\|=\|x\|^{2}, $$ or $\|{\hat{x}}\|_{\infty}=\|x\|$ Thus $x\to{\mathfrak{X}}$ is an isometry. Hence ${\bar{\lambda}}=C(\Delta)$ This completes the proof ${\mathsf{C}}(\Delta)$ Since $\hat{\mathbf{A}}$ is aiso dense in $C(\Delta),$ we conclude that $\hat{\cal A}$ is closed in $i j j$ with the symbolic calculus. The next theorem is a special case of the one just proved. We shal ste it i a form that involvcs th inverse of the Gelfand transform,in order lo make contact 11.19 Theorem I/ A is a commutative $B^{\rtimes}.$ *-algebra which contains an element x such that the polynomials in x and $X^{\infty}$ * are dense in ${\bar{A}},$ then the formula (1) $$ (\Psi f)^{\sim}=J^{\circ}\:{\hat{x}} $$ defines an isometric isomorphism $\mathbf{\hat{P}}$ of $C({\boldsymbol{\sigma}}({\boldsymbol{x}}))$ onto $\scriptstyle{\mathcal{A}}$ A which satisfies (2) $\Psi{\hat{f}}=(\Psi f)^{*}$ for every fe C(ox)).Moreover, $i f f(\lambda)=\lambda\ o n\ \sigma(x),\,l h e n\ \Psi f=3$ x278 BANACH ALGEBRAS AND srtCTRAL THroRr PROOF Let A $\Delta$ be the maximal ideal space of ${\cal A}\,.$ Then $\hat{\mathbf{x}}$ is a continuous function on $\Delta$ whose range is $\sigma(x).$ Suppose $h_{1}\in\Delta_{1}$ $h_{2}\in\Delta$ and ${\hat{x}}(h_{1})={\hat{x}}(h_{2});$ that is, $h_{1}(x)=h_{2}(x).$ Theorem 11.18 implies then that $h_{1}(x^{\star})=h_{2}(x^{\star})$ If $\boldsymbol{P}$ P is any polynomial in two variables, it folows that $$ h_{1}(P(x,\,x^{\ast}))=h_{2}(P(x,\,x^{\ast})), $$ $P(x,x^{*})$ The mapping $f\to f\circ X$ Sis compact, itfollows that $\hat{X}$ are homomorphisms. By hypothesis, elements of the form and $h_{2}$ implies therefore that is one- since $h_{1}$ and $h_{2}$ to-one. Since are dense in $A.$ .The continuity of $h_{1}$ We have proved that $\hat{\mathbf{x}}$ $\sigma(x)$ $h_{1}(y)=h_{2}(y)$ $\Delta$ for every $\nu\in{\mathcal{A}}$ . Hence $h_{1}=h_{2}$ is a homeomorphism of $\Delta$ A onto $C({\boldsymbol{\sigma}}({\boldsymbol{x}}))$ is therefore an isometric isomorphism of onto ${\cal{C}}(\Delta)$ tion (2) comes from $\left(2\right)$ which also preserves complex conjugation. $\|\mathbf{w}f\|=\|f\|_{\alpha}$ Asser- element of $\textstyle A$ of Thcorem 11.18.If is thus (by Theorem 11.18) the Gelfand transform of a unique ther $1f\circ{\hat{x}}={\hat{x}},$ so that $\operatorname{Each}f\circ{\hat{x}}$ which we denote by Yf and which satisfies $f(\lambda)=\lambda,$ $\operatorname{\mathcal{(1)}}$ gives $\Psi f=x$ $J/I$ sense to write $f(x)$ for the element of Remark In the situation described by Theorem 11.19,it makes pefectly good whose Gelfand transform is f。全. Ths $\chi_{\nu_{1,\underline{{{3}}}}}^{\star}$ $\textstyle{\mathcal{A}}$ notation is indeed frequently used. It extends the symbolicalclus for the particular algebras) to arbitrary continuous functions on the spectrum of whether they are holomorphic or not. The existence of squre roots is often of special intcrest, and in algebras wit involution one may ask under what conditions hermitian elements have hermitian square roots. 11.20 Theorem Suppose A is a commutative Banach algebra with an involuion, X $\epsilon\:A,\;x=x^{*}$ ,and o(x) contains no real $\boldsymbol{\lambda}$ with $\scriptstyle\lambda<0$ . Then there exists $y\in{\mathcal{A}}$ with $\textstyle y=y^{*}\,a n d\,y^{2}=x$ Note that the given involution is not assumed to be continuous. We shall se later (Theorem 1126 that commutativity can be dropped from the hypothesis PRoor Let Q be the complement (in ${\mathcal{C}})$ Z) of the set of all nonpositive real numbers $\sigma(x)\subset\Omega$ we can. There exists fe H(2) such that $f^{2}(\lambda)=\lambda,\;\mathrm{and}f(1)=1$ Since define $\scriptstyle y\in{\mathcal{A}}$ by (1) $$ y={\hat{f}}(x), $$ as in Definition .10.26. Then $y^{2}=x,$ by Theorem 10.27. Wewil prove that y* =ycoMMUTATITVE BANACH ALGEBRAS 279 Since $\underline{{\mathbf{Q}}}$ is simply conncted, Runge's theorem furnishes polynomial $\Omega.$ Define ${\mathcal{Q}}_{n}$ by ${\mathbf{}}P_{n}$ that converge to f, uniformly on compact subsets of (2) $$ 2{\cal Q}_{n}(\lambda)={\cal P}_{n}(\lambda)+\overline{{{{\cal P}_{n}(\lambda)}}}. $$ Since $f(\lambda)={\overline{{f(\lambda)}}},$ the polynomials $Q_{n}$ converge to fin the same manner. Define (3) $$ y_{n}=Q_{n}(x)\qquad(n=1,2,3,\ldots). $$ $y_{n}=y_{n}^{*}\,.$ By (2), the polynomials $Q_{n}$ have real coeIicients. Since $x=x^{*}.$ , it follows that By Theorem 10.27, (4) $$ y=\operatorname*{lim}_{n\to\infty}y_{n}, $$ since $Q_{n}\to f,$ so that $Q_{n}(x)\to{\hat{J}}(x).$ If the involution were assumed to be contin- $y^{*}=y$ would follow uous, the set of hermitan elements would be closed, and directly from (4). Let ${\boldsymbol{R}}$ be the radical of ${\bar{A}}.$ Let r: $A\to A/R$ be the quotient map.Define an involution in $A/R$ by (5) $$ [\pi(a)]^{\star}=\pi(a^{\star})\qquad(a\in A). $$ is isomorphic to .If ae A is hermitian, so s r(a).Snce z is continuous $A/R$ is semisimple, and therefore every in- $A/R$ $\hat{\cal A}$ (Theorem 11.9), $\pi(y_{n}) arrow\pi(y).$ Since volution in $A/R$ is continuous (Theorem 11.16) 1t follows that rGy)is hermitian. Hence $\pi(y)=\pi(y^{*}).$ lies in the radical of ${\bar{A}}.$ . We just proved that $v\in R.$ We conclude that $y^{*}-y$ where $u=u^{*}$ and $v\,-\,v^{*}$ By Theorem 11.15, $y=u+i v,$ Since $x=\gamma^{2}$ , we have (6) $$ x=u^{2}-v^{2}+2i u v. $$ $\mathrm{{let}}\,h$ be any complex homomorphism of $0\not\in\sigma(x)$ Thus $h(x)\neq0;$ hence $b(u)\neq0.$ By Theorem $[h(u)]^{2}.$ By hypothesis ${\bar{A}}.$ Sincc $v\in R,h(v)=0.$ Hence $h(x)=$ 11.5,u is invertible in $\textstyle{\mathcal{A}}$ Since $x=x^{*},$ (6) implies that $u v=0.$ Since $v=$ u (w), we conclude that $v=0$ This complctes the proof. $l/l/$ Remark f $\sigma(x)\subset(0,\,\infty_{.}^{\circ}$ ),then also $\sigma(y)\subset(0,\,\infty).$ This follows from(1) (the defintion of y) and the spectal mapping theorem Applications to Noncommutative Algebras Noncommutative algebras always contain commutative ones. Their presence can sometimes be exploited to extend certain theorems from the commutative situation to the noncommutative one, On a trivial level, we have already done this: In the ele mentary discussion of spectra, our atention was usually fixed on one element xe A;280 BANACH ALCEDRAs ANpD SFECTRAL THEoRr the (closed) subalgebra ${\mathcal{A}}_{0}$ of $\textstyle A$ that $\scriptstyle{\mathcal{X}}$ generates is commutative, and much of the dis- cussion took placc within ${\mathcal{A}}_{0}$ . Onc possible diffculty was that xmight have differen spectra with respect to $\textstyle{\mathcal{A}}$ and ${\mathcal{A}}_{0}.$ There is a simple construction(Theorem 11.22) has an that circumvents this. Another device (Theorem 11.25) can be used when $\textstyle A$ involution. 11.21 Centralizers If $\boldsymbol{\mathsf{S}}$ is a subset of a Banach algebra A, the centralizer of $\boldsymbol{\mathsf{S}}$ S is the set $$ \Gamma(S)=\{x\in A\colon x s=s x\;\mathrm{for\;every\;}s\in S\}. $$ We say that $\boldsymbol{\mathsf{S}}$ commutes if any two elements of $\mathbf{S}$ commute with each other. We shall use uhe following simple properties of centralizers. (a)T(S) is a closed subalgebra of A (6) Sc T(T(S (c)I/ S commutes, then $\Gamma(\Gamma(S))$ commutes. Indeed, if x and $\mathbf{\nabla}y$ commute with every se S,so do .x, $x+y,$ and xy;since multi- plication is continuous in $A,\Gamma(S)$ is closed. This proves $(a).$ Since every se Scommutes with every xe T(S),(b) holds. If $\mathbf{\SS}$ commutes, then ${\mathbf{S}}\subset\Gamma({\mathbf{}}S).$ hence $\Gamma(S)\equiv\,\Gamma(\Gamma(S)).$ which proves (Cc), since T(E) obviously commutes whenever $\Gamma(E)\subset E.$ 11.22 Theorem Suppose Ais a Banach algebra $S\subset A.$ $\boldsymbol{\mathsf{S}}$ commutes, and B = for erery T(F(S)) Then Bis a commuiauive Banach algebra, ${\mathcal{S}}\subset B$ , and $\sigma_{B}(x)=\sigma_{A}(x)$ $x\in B$ PROOF Since $e\in B_{s}$ Section 11.21 shows that $\boldsymbol{B}$ is a commutative Banach algebra that contains S. Suppose $x\in B$ and $\mathbf{\Omega}\lambda_{1}^{\lambda}$ is invcrtiblc in ${\mathcal A}$ Wc havc to show that $x^{-1}\in B.$ Since $x\in B,$ $x y=y x$ for every $y\in\Gamma(S).$ ; hence $y=x^{-1}y x,$ // $y x^{-1}=x^{-1}y$ . This says that $x^{-1}\in\Gamma(\Gamma(S))=B.$ 11.23 Theorem Suppose $\dot{A}$ is a Banach algebra, $x\in A,\;y\in A.$ ,and $x y=y x$ Then $$ \tau(x+y)\subset\sigma(x)+\sigma(y)\qquad a n d\qquad\sigma(x y)\subset\sigma(x)\sigma(y). $$ PROOF Put $S=\{x,y\};$ put $B=\Gamma(\Gamma(S))$ Then $x+y\in B,\,x y\in B,$ and Theorem 11.2 sows that we have to prove that $$ \sigma_{B}(x+y)\subset\sigma_{B}(x)+\sigma_{B}(y)\qquad\mathrm{and}\qquad\sigma_{B}(x y)\subset\sigma_{B}(x)\sigma_{B}(y). $$ Since $\boldsymbol{B}$ is commutative, $\sigma_{a}(z)$ is the range of the Gelfand transform ${\hat{\mathbb{Z}}},$ for every $\scriptstyle{\varepsilon\in G}$ (The Gelfand transforms are now functions on the maximal ideal space of B.) Since $$ cdot(x+y){}^{\star}={\hat{x}}+{\hat{y}}\qquad{\mathrm{and}}\qquad(x y)^{\star}={\hat{x}}{\hat{y}}, $$ we have the desired conclusion. $J/f$coMMUTATTVE BANACH ALCEBRAs 281 Corollay $J{\cal f}\,C_{x}=R_{x}-L_{x},$ aS ${\mathit{i n}}\,$ Section 10.37, then $\sigma(C_{x})\subset\sigma(x)-\sigma(x)$ the algebra ${\mathcal{R}}(A),$ PRoor If the theorem is appied to the commuting clements $\textstyle R_{x}$ and $-L_{\alpha}$ of the conclusion is $$ \sigma(C_{x})\subset\sigma(R_{x})-\sigma(L_{x}). $$ But $\sigma(R_{x})=\sigma(x)=\sigma(L_{x}).$ / then $\scriptstyle{\mathcal{X}}$ 11.24 Definition Let A be an algebra with an involution.1 $\operatorname{f}x\in A$ and $x x^{*}=x^{*}x_{;}$ is said to be normal. A set ${\mathcal{S}}\subset{\mathcal{A}}$ is said to be normal if $\boldsymbol{\mathsf{S}}$ commutes and if $x^{*}\u S$ whenever $x\in S.$ 11.25 Theorem Suppose A isa Banach algebra withan involution, and $\boldsymbol{B}$ is a normal subset of Athat is maximal wihrespect to being normal. Then (a) B is a closed commutatitre subalgebra of A, and (b) $\sigma_{B}(x)=\sigma_{A}(x)f o r\;e v e r y\;x\in B.$ .Note that the involution is not assumed to be continuous but that $\boldsymbol{B}$ neverthe less turns out to be closed. PROOF We begin with a simple criterion for membership in ${\boldsymbol{B}}\colon$ B::Ifx∈ A,if xx* $=x^{*}x,$ and if xy = yx for every $y\in B,$ , then $x\in B$ for all $y\in B_{i}$ For if $\scriptstyle{\mathcal{X}}$ x satisfies these conditions, we also have $x y^{*}=y^{*}x$ since $\boldsymbol{B}$ is normal, and therefore $x^{*}y=y x^{*}$ .t follows that $B\cup\{x,\,x^{\bullet}\}$ is normal. Hence $x\in B,$ since $\boldsymbol{B}$ is maximal This criterion makes it clear that sums and products of members of ${\boldsymbol{B}}$ B are in ${\boldsymbol{B}}.$ Thus $\boldsymbol{B}$ is a commutative algebra_ Suppose $x_{n}\in B$ and $x_{n}\to x.$ Since $x_{n}y=y x_{n}\operatorname{for}\operatorname{all}y\in B,$ and multiplica tion is continuous, we have $x y=$ yx and thcrcfore also $$ x^{\ast}y=(y^{\ast}x)^{\ast}=(x y^{\ast})^{\ast}=y x^{\ast}. $$ In particular, $\chi^{\ast}x_{n}=\chi_{n}\,\chi^{\ast}$ for all n, which leads to $\chi^{\mathrm{st}}\chi=\chi\chi^{\mathrm{st}}.$ Hence $x\in b.$ by the above criterion. This proves that $\boldsymbol{B}$ is closed and completes $(a).$ normal, so is Note also that e∈ B. To prove (b), assume $x\in B,$ $x^{-1}\in A$ Since xis ${\mathcal{X}}^{-1},$ and since $\scriptstyle{\mathcal{X}}$ commutes with every $y\in B,$ so does $x^{-1}$ Hence $x^{-1}\in B$ Our frst application of thsis a generalization of Theorem $1\mid.20;$ 11.26 Theorem The word $\mathbb{Z}\mathbb{C}$ commutatie” may be dropped from the hypothesis of Theorem 11.20.282 BANACH ALOEBRAs AND srCTRAL THEoR、 PROOF By Hausdorff's maximality theorem. the given hermitian hence normaD $J/I\!f$ xe A lies in some maximal normal set in place of $A.$ By Theorem 11.25 we can apply ${\boldsymbol{B}}.$ Theorem 11.20 with $\boldsymbol{B}$ Our next application of Thcorem 11.25 will extend some consequences o -algebras Theorem 11.18 to arbitrary Gnot neessrily commutative $B^{\ddagger}.$ 11.27 Definition In a Banach algebra with involution,the statement $^*x\geq0^{\prime\prime}$ means that $x=x^{*}$ and that $\sigma(x)\subset[0$ , cO). 11.28 Theorem Every B*-algebra $\textstyle A$ has the following properies (a) 1lermitian elements have real spectra. (e) $I f y\in A.$ If x∈ A is normal, then $\rho(x)=\|x\|.$ ,then $u+v\geq0.$ (5) (c) If y A, then $(y y^{*})=\|y\|^{2}$ and $v\geq0$ (d) $I f u\in A,$ ve A, $u\geq0,$ ,then $y y^{*}\geq0$ (() $\;U f y\in A,$ then $e+y y^{\mathrm{is}}$ is mvertible in $A.$ PRO0F 11.18 and 11.25, $\boldsymbol{B}$ is a commutative $B^{\kappa}.$ lies in a maximal normal set ${\mathcal{B}}\subset{\mathcal{A}}$ By Theorems Every normal $x\in{\mathcal{A}}$ -algebra which is isometrically isomor phic to its Gelfand transform ${\bar{B}}=C(\Delta)$ and which has the property that (1) $$ \sigma(z)=\hat{z}(\Delta)\qquad(z\in B). $$ Here $\sigma(t)$ is the spectrum of z relative to $A_{\cdot}$ $\Delta$ is the maximal ideal space of ${\mathcal{B}},$ If Hence (I) implies (a and 2(A) s the range of the Gelfand transform of z, regarded as an element of is a real-valued function on $\Delta.$ $x arrow x^{\star},$ , Theorem 11.18 shows Lhat ${\boldsymbol{B}}.$ $\hat{\mathbb{X}}$ and For any normal x,(I) implies $\rho(x)=\|{\hat{x}}\|_{\infty}\,.$ Also, $\|{\hat{x}}\|_{x}=\|x\|,$ since $\boldsymbol{B}$ $\hat{B}$ are isometric. This proves (b) $\operatorname{If}y\in A,$ then y is hermitian. Hcncc (e follows from $(b),$ since $\rho(y y^{*})=$ $\|y^{\ast}\|=\|y\|^{2}$ Suppose now that u and $\boldsymbol{\mathit{v}}$ are as in (d).Put $x=\left\|u\right\|$ $\beta=\|v\|.$ $w=u+v,$ $\gamma=\alpha+\beta$ Then $\sigma(u)\subset[0,\alpha],$ so that (2) $$ \sigma(\alpha e-u)\subset[0,\alpha] $$ Hence and Gb) imples therefore tha $\|\alpha e-u\|\leq\alpha.$ For the same reason, $\|\beta e-v\|\leq\beta.$ (3) $$ \|\gamma e-w\|\leq\gamma. $$ Since $w=\,w^{\star}.$ $\mathbf{\Psi}(a)$ implies that $\sigma(\gamma e-w)$ is real. Thus (4) $\sigma(\gamma e-w)\subset[-\gamma,\gamma],$ because of O))But (4)implies that ov) I0, 2y] Thusw ≥0,and d is provedcowmMrATV BANACH ALGEBRAS 283 on Since ${\tilde{B}}=C(\Delta),$ we turn to the proof of (e). Put $x=y y^{*}$ . Then $\hat{\mathbf{x}}$ is are-vie uincti $\boldsymbol{B}$ $\Delta$ is chen as n h frt praraph o ths pro. he $\scriptstyle{\mathcal{X}}$ is hermitian, and if By (I), we have to show that ${\hat{x}}\geq0$ on $\hat{\boldsymbol{\Delta}}$ there exists $z\in B$ such that (5) $$ \hat{z}=|\hat{x}|-\hat{x}\qquad\mathrm{on}\;\Delta. $$ Then $z=z^{*}.$ because $\hat{\mathbb{Z}}$ is real CTheorem 11.18) Put (6) $$ z y=w=u+i v, $$ where ${\mathcal{U}}$ and $\boldsymbol{\mathit{v}}$ are hermitian elements of $\textstyle A$ 4. Then (7) $$ w w^{*}=z y y^{*}z^{*}=z x z=z^{2}x $$ and therefore (8) $$ w^{\star}w=2u^{2}+2v^{2}-w w^{\star}=2u^{2}+2v^{2}-z^{2}x. $$ Since $u=u^{*}$ $\sigma(u)$ is real, by $\scriptstyle(a),$ hence $u^{2}\geq0,$ by the spectral mapping theorem Likewise $v^{2}\geq0$ By (5) $\hat{z}^{2}\hat{x}\leq0$ on $\Delta$ Since $z^{2}x\in B,$ it follows from (1) that $-z^{2}x>0$ Now (8) and (d) imply tha $w^{\star}w\geq0.$ when But $\sigma(w\nu^{*})\subset\sigma(w^{*}w)\cup\{0\}$ $\Xi^{2}{\hat{x}}\geq0$ on $\Delta$ A, By (5), this last inequality holds only $w w^{\star}>0$ // By (7), this means that (Exercise 2, Chapter 10). Hence ${\hat{x}}_{*}=|{\hat{x}}|$ Thus ${\hat{x}}\geq0,$ and (e) is proved Finally $\left(f\right)$ is a corollary of (e. mutativity plays no role Fqualiy of speta can now be proved in yet another situation, in which com 11.29 and xt e B for every $\mathrm{rer}\,B\,$ Then Suppose A is a B*-algebra, B is a closed subalgebra of ${\bar{A}}_{\mathrm{,}}$ e ∈ B, Theorem $\sigma_{A}(x)=\sigma_{B}(x)f o r\ e v e r y\ x\in B.$ PRoor Suppose xe $\boldsymbol{B}$ and xhas an inverse n A. We have to show tha Hence $(x x^{*})^{-1}\in B_{i}$ and finally ${\mathcal{C}},$ Since xis invertiblc in ${\cal A}\,,$ so is $\lambda_{\mathrm{\Lambda}}^{a^{\frac{3(n)^{2}}{2(n+1)}}}.$ hence also xx*. Thus $x^{-1}\in B.$ by (e) of Theorem 11.28. Since $\sigma_{A}(x x^{\star})\subset(0,\,\infty),$ Theorem 10.18 shows that $\sigma_{A}(x x^{*})$ has connected complement in $\sigma_{B}(x x^{*})=\sigma_{4}(x x^{*}).$ $x^{-1}=x^{*}(x x^{\star})^{-1}\in B$ // Positive Functionals bra 11.30 Definition A positive functional is a linear functional ${\mathbf{}}F$ on a Banach alge $\scriptstyle A$ with an involution, that satisfes $F(x x^{\star})\geq0$284 BANACH ALGEBRAS AND srECTRAL THFORY for every $x\in{\mathcal{A}}$ Note that $\textstyle{\mathcal{A}}$ is not assumed to be commutative and that continuity of ${\mathbf{}}F$ is not postulated.(The meaning of the term“positive”depends of course on the particular involution that is under consideration.) 11.31 Theorem Every positive functional ${\mathbf{}}F$ 午 on a Banach algebra A with involution has the following properties: (a $F(x^{*})={\overline{{F(x)}}}.$ (6) $|F(x y^{*})|^{2}\leq F(x x^{*})F(y y^{*}).$ (c) $|F(x)|^{2}\leq F(e)F(x x^{*})\leq F(e)^{2}\rho(x x^{*}).$ (d) $|F(x)|\leq F(e)\rho(x)$ for every normal $x\in A.$ if Ais commutative, e) Fis a bounded linear functional on A. Moreover, $\|F\|=F(e)$ $a n d\left\|F\right\|\leq\beta^{1/2}F(e)$ if the involution satisfies $\left\|x^{\geq}\right\|\leq\beta\left\|x\right\|_{.}$ for every xe A PROOF If $x\in{\mathcal{A}}$ and $y\in A.$ , pu (1) $$ p=F(x x^{\ast}),\,q=F(y y^{\ast}),\,r=F(x y^{\ast}),\,s=F(y x^{\ast}). $$ Since $F[(x+\alpha y)(x^{\ast}+\bar{\alpha}y^{\ast})]\geq0$ for every ${\mathfrak{x}}\in C,$ (2) $$ p+\bar{\sigma}r+\infty s+|\alpha|^{2}q\geq0. $$ With $\scriptstyle x\,=\,1$ and $x=i,$ (2) shows that $\textstyle s+\prime$ and $i(s-r)$ are real. Hence $S={\bar{T}}.$ With $y=e,$ this gives (a) D, take ${\mathcal{x}}=t r/|r|$ in (2), where t is real. If $r=0,$ (b) is obvious.I1f $\scriptstyle{r\wedge v}$ Then (2) becomes (3) $$ p+2|r|t+q t^{2}\geq0\qquad(-\infty<t<\infty), $$ so that $|r|^{2}\leq p q.$ This provcs (b) is a special case of $(\not b).$ For the second Since ee* = e, the first half o $\left(c\right)$ lies in the open right half-plane. By half, pick 1> p(xx*). Then $\sigma(t e-x x^{*})$ Theorem 11.26,there exists $u\in A,$ with $u=u^{*}$ , such that $u^{2}=t e-x x^{*}$ Hencc (4) $$ t F(e)-F(x x^{*})=F(u^{2})\geq0. $$ It follows that (5) $$ F(x x^{*})\leq F(e)\rho(x x^{*}). $$ This completes part (c). If $\scriptstyle{\mathcal{X}}$ is normal, i.e.,if $x\backslash x^{\mathrm{s}}=\chi^{\mathrm{s}}x,$ Theorem 11.23 implies that $\scriptstyle\sigma(x x^{*})\in^{-}$ o(x)o(x*), so that $\mathbf{\tau}(6)$ $$ \rho(x x^{*})\le\rho(x)\rho(x^{*})=\rho(x)^{2}. $$ Clearly, d follows from (6) and (ccowwUTATIVE BANACH ALCEBRAs 283 ff If $\|x^{\star}\|\leq\beta\|x\|,$ $\left(c\right)$ A is commutative, then $(d)$ holds for every $x\in A,$ so that $\|F\|=F(e)$ T his $\mathcal{A}$ $F(x)=0$ implies $\begin{array}{r}{|F(x)|}\end{array}\leq F(e)\beta^{1/2}\|x\|,\;\mathrm{since}\;\rho(x x^{\ast})\leq\|x\|,\|x^{\ast}\|.$ $F(e)\geq0$ and that for every disposes of the special cas of part e) if $F(e)=0;$ thi folos from Ge) In the remainder o Before turning to te general case、 we observe tha $x\in A$ ths rof hal teor assme,wthot fs o eait, th (7) $$ F(e)=1. $$ ${\cal H}$ and Let $H$ be the closure of $H.$ . the setof al hrmitian elements of $\scriptstyle{\mathcal{A}}$ Note that $i H$ arc real vector spaces and that $A=H+i H,$ by Theorem 11.15 By (d),the restriction of $F\tan H$ is a rel-inear funcional of norm I, which there fore extends to a real-linear functional $\Phi$ b on ${\bar{H}},$ also of norm 1. We claim that (8) $$ \Phi(y)=0\qquad\mathrm{if}\,y\in\overline{{{I}}}\overline{{{I}}}\,\cap\,i\overline{{{I}}}, $$ for if y = lim $u_{n}=\operatorname*{lim}\,(i v_{n})$ ), where $u_{n}\in H$ and $v_{n}\in H,$ then $u_{n}^{2}\to y^{2};$ $v_{n}^{2}\to-y^{2},$ so that $\left(c\right)$ and (d imply (9) $$ |F(u_{n})|^{2}\leq F(u_{n}^{2})\leq F(u_{n}^{2}+v_{n}^{2})\leq\|u_{n}^{2}+v_{n}^{2}\|\to0. $$ Since $\Phi(y)=\operatorname*{lim}\;F(u_{n}),$ (8) is proved such that every $x\in{\mathcal{A}}$ has a By Theorem 5.20, there is a constan $\gamma<\infty$ representation (10) $$ .\qquad x=x_{1}+i x_{2}\,,\,x_{1}\in\widehat{\cal H},\,x_{2}\in\widehat{\cal H},\,\|x_{1}\|\,+\,\|x_{2}\|\leq\gamma\|x\|. $$ 1f $x=u+i v,$ with $\nu\in H_{\circ}$ t ${\mathfrak{e}}\,H.$ then $x_{1}-u$ and $x_{2} arrow v$ lic in $H\cap i{\bar{H}}$ Hence (B)yields (11) $$ F(x)=F(\imath)+i F(v)=\Phi(x_{1})\mathrm{\boldmath~\mid~}i\Phi(x_{2}), $$ so that (12) $$ |F(x)|\leq|\Phi(x_{1})|+|\Phi(x_{2})|\leq\|x_{1}\|+\|x_{2}\|\leq\gamma\|x\|. $$ This completes the proof // Exercise l3 contains further information about part (e) Examples of psitive fnctioals-and a relation between them and posii measures-are furnished by thenext theorm l onains Bochescsiaiteore aboutpositvefie functions sa very special case.The ientiatos ta ea from one to the other are indicated in Exercise 14 11.32 Theorem Suppose A is a commutative Banach algcbra, with maximal idea space , and wih a iwolutio that is symmeric he sense tha (1) $$ L(x^{\star})=\overline{{{h(x)}}}\qquad(x\in A,\,h\in\Delta). $$286 BANACH ALGEBRAS AND SPECTRAL THEORY Let K bethe st of ll ositive functionals F on Athat satisf $F(e)\leq1.$ Ler M be the set of all positive regular Borel measures $\boldsymbol{\mathit{l}}$ on $\Delta$ that satisfy $\mu(\Delta)\leq1$ Then the formula (2) $$ F(x)=\int_{\Lambda}{\hat{x}}\;d\mu\qquad(x\in A) $$ establishes a one-to-one correspondence between the convex sets ${\cal K}$ and $\bar{M},$ which carries extreme points to extreme points Consequently, the multiplicative linear functionals on $\textstyle{A}$ are precisely the extreme points of K. rRoor If $\mu\in M$ and ${\mathbf{}}F$ is defined by (2), then ${\mathbf{}}F$ is obviously inear, and $F(x x^{*})$ $={\int}\left|{\hat{x}}\right|^{2}d\mu\geq0$ ,because(1) implies that $(x x^{*})^{\times}=|\hat{x}|^{2}.$ Since $F(e)=\mu(\Delta)_{i}$ $F\in K,$ If $F\in K,$ then ${\mathbf{}}F$ vanishes on the radical on $A.$ by $(d)$ of Theorem 11.31 $x\in A$ In fact, Hence there is a functional $\hat{F}$ on $\hat{\cal A}$ that satisfies ${\hat{F}}({\hat{x}})=F(x)$ for al (3) $$ \big\vert\,\hat{F}(\hat{x})\big\vert=\big\vert F(x)\big\vert\leq F(e)\rho(x)=F(e)\big\vert\hat{x}\big\vert_{\infty}\qquad(x\in A), $$ ${\mathfrak{b y}}\left(d\right)$ on the subspace $\hat{A}$ of of Theorem 11.31. It follows that $\hat{F}$ is a linear functional of norm $c(\partial)$ with the same $F(e)$ $C(\Delta).$ This extends to a functional on norm, and now the Riesz representation theorem furnishes a regular Borel measure ${\boldsymbol{\mu}},$ with $\|{\boldsymbol{u}}\|=F({\boldsymbol{e}}).$ that satisfies (2). Since (4) $$ \mu(\Delta)=\int_{\Delta}\hat{e}\,d\mu=F(e)=\|\mu\|, $$ we see that $\mu\geq0$ Thus $l\,\rlap{/}{=}\;M.$ $\mathrm{By}\left(1\right),\,{\bar{A}}$ satisfies the hypotheses of the Stone-Weierstasstheorem and s One extreme point of $\mathcal{M}$ . This implics that $0\,;$ the others are uni masses concentrated a ${\boldsymbol{F}}.$ thcrcforc dcnsc in $C(\Delta).$ is $\boldsymbol{\mu}$ is uniquely determined by points $h\in{\bar{\Delta}}$ Since every complex homomorphism of $\textstyle{\mathcal{A}}$ has the form $x arrow\hat{x}(h),$ for some $h\in\Delta_{\circ}$ the proof is completc ${a\!\!\!/}{b\!\!\!/}{b\!\!\!/}$ We conclude by showing that the extreme points of $K$ K are mutiplicative even if (1) is not satisficd 11.33 Theorem Let $\textstyle K$ be the ser of all positive fumctionals ${\mathbf{}}F$ 071 $\bar{\boldsymbol{a}}$ commutative Banach algebra A with an involution, that satisfy $F(e)\leq1$ .If F∈ K, then each of $\scriptstyle h e_{\circ}$ following three properties implies the other two: (a) $F(x y)=F(x)F(y)f o r\ a l l\ x\ a n d\ y\in.$ A $(b)\ \ \ F(x x^{\ast})=F(x)F(x^{\ast})f o,$ r every xe A (cF is an extreme point of K.COMMUTATTVE BANACH ALGEBRAS 287 Clearly, PRoor It is trivial that $\mathbf{\Psi}(a)$ implies $(B).$ Suppose(b) holds. With When $F(c)=0,$ then $F(e)=1$ $F_{l}(e)=1=F(e)$ . If $x\in{\mathcal{A}}$ is such that $F(x)=0,$ $x\equiv e$ ,(b) shows that $F(e)=F(\sigma)^{2},$ and so $F(e)=0$ or $F(c)=1$ then $K.$ Assume $F=0,$ by $\mathbf{\Psi}(c)$ of Theorem 11.31, and so ${\mathbf{}}F$ is an extreme point ot ,and $2{\cal F}={\cal F}_{1}+{\cal F}_{2}\,,\;{\cal F}_{1}$ e K, $F_{2}\in K.$ We have to show that $F_{1}=F$ (1) $$ \left|F_{1}(x)\right|^{2}\leq F_{1}(x x^{*})\leq2F(x x^{*})=2F(x)F(x^{*})=0, $$ by (b) and Theorem 11.31. Thus $F_{1}=F.$ Hence (b) implies (c) 'on the null space of ${\mathbf{}}F$ 'and at e. It follows that $F_{1}$ coincides with ${\mathbf{}}F$ $F(e)=0.$ To show that (c) implies (a), let ${\mathbf{}}F$ be an extreme point of $K.$ Either in which case there is nothing to prove, or $F(e)=1$ We shall frs prove a special case of (a), namely (2) $$ F(x x^{*}y)=F(x x^{*})F(y)\qquad(x\in A,y\in A). $$ Choose $\textstyle{\mathcal{X}}$ so that $\|x x^{*}\|<1.$ By Theorem 11.20, there exists $Z\in{\cal{A}},$ $z=z^{*},$ such that $z^{2}=e-x x^{\pm}$ Define (3) $$ \Phi(y)=F(x x^{*}y)\qquad(y\in{\cal A}). $$ Then (4) $$ \Phi(y y^{*})=F(x x^{*}y y^{*})=F[(x y)(x y)^{*}]\geq0, $$ and also (5) (F - D)Gy*) = F[(e - x*)*1= F(2)y*) - FI(yrD)02*12 0 Since (6) $$ 0\leq\Phi(e)=F(x x^{*})\leq F(e)\|x x^{*}\|<1, $$ $(\lambda)$ and (5) show that both $\bar{\Phi}$ and $F_{-}\oplus$ are in $K.$ If $\Phi(e)=0,$ then $\scriptstyle{\mathbb{P}}=0,$ If $\Phi(e)>0,$ (6) shows that (7) $$ F=\Phi(e)\cdot{\frac{\Phi}{\Phi(e)}}+(F-\Phi)(e)\cdot{\frac{F-\Phi}{F(e)-\Phi(e)}}. $$ a convex combination of members of $K.$ Since ${\boldsymbol{F}}$ is cxtreme, we conclude that (8) $$ {}^{*}\qquad\Phi=\Phi(e)F. $$ Now (2) follows from (8) and (3) Finally, th passage from (2) to Ga is accomplished by any of he ollowing identities, which are satisfied by every involution: If n = 3,4, 5,..if o = exp (2riln), ifxe A,and $i f z_{p}=e+\omega^{-p}x,\,t h e n$ (9) X $:={\frac{1}{n}}\sum_{p=1}^{n}\omega^{p}z_{p}z_{p}^{*}$288 BANACH ALGERRAS AND sPECTRAL THroRy The proof of (9) is a straightforward computation which uses the fac that (10) $$ \sum_{p=1}^{n}\omega^{p}=\sum_{p=1}^{n}\omega^{2p}=0. $$ 1 Exercises Prove Proposition 11.2 P State and prove an analogue of Wiener's Iemma 11.6 for power series that converge absolutely on the closed unit disc If Xisa compact Hausdorff space,show that there is a natural one-to-one correspondence between closed subsets of $\textstyle X$ Y and closed ideals of ${\mathsf{c}}(X)$ ${\cal A}(U^{n})$ (See Theorem 11.7.) Prove that the polynomials are dense in the polydisc algebra Suggestion: If $f\in A(U^{n}),\,0<r<1,$ and $f_{r}$ is defined by $f_{r}(z)=f(r z),$ then f, is the sum 5 Suppose $\textstyle A$ of an absolutely (hence uniformly) convergent multiple power series on $x\in A,$ and $\boldsymbol{f}$ "is holomorphic in some ${\overline{{U}}}^{n}.$ is a commutative Banach algebra, open set ${\mathfrak{G}}{\subset}{\mathit{C}}$ that contains the range of ${\tilde{x}},$ Prove that there exists $y\in A$ such that ${\hat{\gamma}}=f\circ{\hat{x}},$ that is, such that $h(y)=f(h(x))$ for every complex homomorphism $\boldsymbol{\mathit{h}}$ of ${\bar{A}}.$ Prove that $\mathbf{\nabla}y$ is uniquely determined by $\scriptstyle{\mathcal{X}}$ and fif A is semisimple. Suppose $\textstyle A$ and $\boldsymbol{B}$ are commutative Banach algebras, $\boldsymbol{B}$ is semisimple, $\textstyle\psi:A arrow B$ is a homomorphism whose range is dense in ${\boldsymbol{B}},$ and ${\mathfrak{X}}\,;$ $\Delta_{B} arrow\Delta_{A}$ is defined by $$ (\alpha h)(x)=h(\slash{\psi}(x))\qquad(x\in A,\,h\in\Delta_{D}). $$ Prove that α is a homeomorphism of $\Delta_{B}$ onto a compact subset of $\textstyle\Delta_{A}$ [The fact that $\scriptstyle{i(\lambda)}$ is dense in $\boldsymbol{B}$ implies that $\scriptstyle{\mathcal{\mathbf{x}}}$ is one-to-one and that the topology of $\Delta_{R}$ is the weak topology induced by the Gelfand transforms of the clements $\scriptstyle y(x),$ for $x\in A.$ let $\psi$ proper subset of $\textstyle\Delta_{A}:$ be the disc algebra, let $\psi$ is one-to-one ${\boldsymbol{B}}.$ This example shows that $\scriptstyle a(\Delta_{n})$ may bc a Let $\scriptstyle A$ $B=C(K),$ where $\boldsymbol{K}$ is an arc in the unit disc, and be the restriction mapping of ${\mathcal{A}}$ into even if Q Find an examplc in which $\psi(A)=B$ but ${\bf x}(\Delta_{B})\neq\Delta_{A}\,.$ Can Lebesguc In Example 11.13(6) it was asserted that ${\dot{A}}\neq C(\Delta)$ Find several proofs of this. Which properties of Lebesgue mcasurc are used in Example $11.13(f)^{\gamma}$ measure be replaced by any positive measure, without changing any of the results ? Borel set was proved for open sets.) Supply the details for the last paragraph in Example $11.13(f).$ for every $S\subset\Delta$ Using the notation in Example 11.13(/), prove that ${\hat{m}}(S)=m(S)$ Hence boundaries or Borel sets have measure O. Gn the tex, thi 9 Let ${\boldsymbol{C}}^{\prime}$ be the algebra of all continuously differentiable complex functions on the unit interval $|0,1\rangle,$ with pointwise multiplication, normed by $$ \|f\|=\|f\|_{\infty}+\|f^{\prime}\|_{\alpha}\,. $$ (a) Show that ${\boldsymbol{C}}^{\prime}$ is a semisimple commutative Banach algebra. Find its maximal ideal spacecowwrATV BANACH ALCEBRAS 289 $I{\mathcal{O}}$ Let in Exercise 14 of Chapter 10 is ${\boldsymbol{C}}^{\prime}/J$ let J be the set ofalf C for which $f(p)=f^{\prime}(p)=0.$ Show that (b) Fix p,G $3\leq p\leq1;$ ${\mathbf{}}J$ is a closed ideal in ${\boldsymbol{C}}^{\prime}$ D’and that ${\boldsymbol{C}}^{\prime}/J$ is a two-dimensional algebra which has a one-dimensional radical. CThis gives an example of a semisimple algebra with a quoticnt algebra that is not semisimple.)To which or the two algebras describe $\textstyle A$ isomorphic? be the is alebra sAssciato cach fe Aafunction *e A by the formula $$ f^{*}(z)={\overline{{f({\overline{{z}}})}}}. $$ Then $f\to f^{*}$ is an involution on $\textstyle A$ -algebra ? (a) Does this involuton turn $\scriptstyle{A}$ l into a $B^{\mathbf{k}}.$ (b)Does $\sigma(\beta/\ ^{*})$ always lie in the real axis are positive funcionals、with respect to ths involution? (e) Which complex homomorphisms o $\scriptstyle{\mathcal{A}}$ (d) If ${\boldsymbol{\mu}}$ is a positive finite Borel measure on $\mathbb{F}{\boldsymbol{-}}1.$ 1], then $$ f\to\int_{-1}^{1}f(t)\,d\mu(t)\quad. $$ $I I$ $x y=y x$ is a positive functional on c and $\mathbf{\vec{y}}$ in a Banach algebra, then eithe $\scriptstyle>1$ Explicitly, if or $\|x-y\|\geq1.$ Show thiat this may fail if $x y\neq y x.$ ${\cal A},$ Arc thcrc any others? $x=y$ $x^{2}=x,\ y^{2}=y,$ Show that commuting idempotents have distance for some $\scriptstyle{\mathcal{X}}$ ${\mathit{I}}{\mathit{Z}}$ If $x y=y x\ 1$ for some $\scriptstyle{\mathcal{X}}$ and $\mathbf{y}$ in a Banach algebra, prove that $$ \rho(x y)\leq\rho(x)\rho(y)\qquad\mathrm{and}\qquad\rho(x+y)\leq\rho(x)+\rho(y). $$ ${\it\it1}{\it3}$ Let t be a large positve number, and define a norm on ${\mathcal{C}}^{2}$ by $$ ||w||=|w_{1}|\ +t|w_{2}|\qquad\mathrm{if\w=(w_{1},w_{2}).} $$ Let $\scriptstyle{\mathcal{A}}$ be the algebr f all comlex 2-by-2 matrices,with thecorrespondig operato norm: $$ |y||=\operatorname*{max}\{|y|\upsilon\langle\upsilon\rangle||:\|\psi|=1\}\qquad(y\in{\cal A}). $$ For y∈ A, let $y^{\star}$ be the coniugate transpose of ${\boldsymbol{y}}.$ Consider a fixed $x\in A,$ namely, (a)ILx(w) Prove the folloing statements $$ x={\binom{0}{1}}\quad r^{2}{\underset{0}{\sqrt{}}{}}. $$ $|=t||w||;$ hence $\|x\|=t.$ 6) $\sigma(x)=\{t,-t\}=\sigma(x^{\ast}).\qquad\d t^{\ast}$ (c) $\tau(x x^{\kappa})=\{1,\,t^{4}\}=\sigma(x^{\kappa}x).$ (d) $\sigma(x+x^{*})=\{1+t^{2},\,-1-t^{2}\}.$ , for $y\in A,$ then ${\mathbf{}}F$ is a positive functional $(f)$ If (e) Therefore commutativity s required in Theorem 11.23 and in Exercise 12 ${\boldsymbol{y}},$ $\scriptstyle{\vec{F}}(y)$ is the sum of the four entries in on AL $(g)$ The equality $|F||=F(e)$ Isee (e) of Theorem 11.11 does not hold, because $F(e)=2$ and FKx)=1+1,so that |FI>1290 BANACH ALGEBRAS AND SPECTRAL THroRv functional on $\textstyle A$ (h) If K is the set of all positive functionals $\boldsymbol{\mathit{f}}$ on $\textstyle A$ that satisfy $f(c)\leq1$ (as in Theorem of 11.33), then $\textstyle K$ K has many extreme poins alhough $\mathbf{0}$ is the only multipicative lincar $(c)\to(a)$ Commutativity is therefore required in the implication Theorem 11.33. 14 A complex function ${\boldsymbol{\phi}}_{*}$ defined on ${\textstyle\mathcal{R}}^{n},$ is said to be positive-definite i $$ \_{i\geq j=1}^{r}c_{i}\,\bar{c}_{j}\,\phi(x_{i}-x_{j})\geq0 $$ for every choice of $X_{1},\ \cdot\cdot\cdot,\ X_{r}$ in ${\boldsymbol{R}}^{n}$ Ph n and for every choice of complex numbers $c_{1},\ \cdot\ \cdot\cdot\ ,\ C_{r}$ is (a) Show that $|\phi(x)|\leq\phi(0)$ for every $x\in R^{n}.$ ${\boldsymbol{R}}^{n}$ (6) Show that the Fourier transform of every finite positive Borel measure on positive-definite. Borel measure. (c Complete the following outline of the converse of (b)(Bochner's theorem: ${\mathcal{I}}\,\phi$ is continuous and positie-deinite, then p is the Fouier transform of a finite positive Let $\scriptstyle{A}$ be the convolution algebra $L^{1}(R^{n})$ ), with a unit attached, as described in (d) of Section 10.3 and $\mathbf{\Psi}({\boldsymbol{e}})$ of Section 11.13. De $\operatorname{fine}{\tilde{f}}(x)={\overline{{f(-x)}}}.$ Show that $$ f+\alpha\delta arrow\tilde{f}+\bar{\alpha}\delta $$ is an involution on $\scriptstyle{\dot{A}}$ 4 and that $$ f+\alpha\delta arrow\int_{\boldsymbol{R^{n}}}f\phi\ d m_{n}+\alpha\phi(0) $$ is a positive functional on $A.$ By Theorem 1132 and $\mathbf{\Psi}({\boldsymbol{e}})$ of Section 11.13, there is a positive measure $\boldsymbol{\mu}$ on the one-point compactification $\underline{{\land}}$ of $\textstyle R^{n}\!,$ ”,such that $$ \int_{{\cal R}^{n}}f\phi\,d m_{n}+\alpha\phi(0)=\int_{\Lambda}(\dot{f}+\alpha)\,d\mu. $$ If o is the restriction of $\boldsymbol{\mathit{l}}$ Lto $K^{n},$ ", it follows that $$ \int_{R^{n}}f\phi\ d m_{n}=\int_{R^{n}}f^{g}d\sigma $$ for cvcry f $\in L^{1}(R^{*}).$ Hence $\phi={\dot{G}}.$ .(Actually, ${\boldsymbol{\mu}}$ is already concentrated on ${\mathcal{R}}^{n},$ SO that $\sigma=\mu.$ .) ${\boldsymbol{J}}{\boldsymbol{S}}$ (d) Let ${\boldsymbol{P}}$ be the set o ${\mathrm{all}}$ continuous positive-definite functions $\phi$ on ${\dot{R}}^{n}$ that satisfy $\beta<\Delta$ xe A.(Trivially, $\Delta$ S is an $A\!\!\!/.$ Find all xteme points of this convcx sct Call a cosed set $\phi(0)\leq1.$ $A.$ ${\boldsymbol{\beta}},$ for every an Let L be the maximal ideal space of commutative Banach algebra of all A-boundarics is an A-boundary. $A\!\cdot\!$ Prove that the intersection -boundary.) a-boundr if the maximum of |xl on L equals is maximum on ${\hat{\sigma}}_{A}$ then ${\tilde{\sigma}}_{A}$ ${\tilde{O}}_{A}$ is called the Shilov boundary of A. The terminology is suggested by the maximum is the iscalgebra, modulus proprty of holomorphic unctions. For instance, when $\scriptstyle{A}$ is the unit circle, which is the topological boundary of L,the closed unit disc Ouline of progf: Show first that there is an A-boundary β $\beta_{0}$ o which is minimal in the sense that no proper subset of B。is an A-boundary. OPatilly rder the collection ot A-coMMUTATIVE BANACH ALOEBRAs 291 and put bouaris s st insion etc )The pi $h_{0}\in\beta_{0}$ , pick $x_{i},\,\cdot\cdot\cdot,\,x_{n}\in A{\mathrm{~with~}}{\bar{x}}_{i}(h_{0})=0$ $$ V=\{h\in\Delta\colon|{\dot{x}}_{i}(h)|<1{\mathrm{~for~}}1\leq i\leq n\}. $$ Since A-boundary ${\boldsymbol{\beta}},$ is minimal, there exis $x\in{\mathcal{A}}$ with $|{\hat{X}}||_{\infty}=1$ and $|{\hat{x}}(b)|<1$ on $\beta_{0}-V.$ 1f $\beta_{0}$ $y\div x^{m}$ and $\mathbf{\nabla}m$ nis sufinlylargc, then $|{\hat{A}}_{i}{\hat{J}}|<1$ on $\beta_{0}\,.$ for all ${\bar{l}}.$ Hence $\|{\hat{x}}_{i}{\hat{y}}\|_{i\cdot p}<1.$ 16 Suppose $\scriptstyle{\mathcal{A}}$ Conclude from this frs tha $\left|{\dot{f}}(h)\right|=\left|{\dot{y}}\right||_{x}$ only in ${\mathit{V}},$ hence that and intersects every is a Banach algebra, ${\mathfrak{p}}n$ is an integer $m\geq2,\,K<\infty,$ $\mathcal{V}$ , and finally that $h_{0}\in\beta.$ Thus $\beta_{0}\in\beta,$ and $\beta_{0}=\partial_{A}$ $$ .\qquad\quad\ \|{\boldsymbol{x}}\,\|^{m}\ {\underline{{<}}}\,K\,||{\boldsymbol{x}}^{m}\,\|^{n}\ {\underline{{<}}}\,K\,||\ . $$ for every $x\in A.$ Show that thercxis onstants $K_{n}<\infty,$ for $n=1,$ 2,3,.…, such that $$ \|x\|^{n}\cong K_{n}\left||x^{n}|\right|\qquad(x\in{\mathcal{A}}). $$ ${\mathit{I}}{\mathit{I}}$ Suppose fo (This extends Theore 11.12. are positive numbers such tha $\scriptstyle w_{\mathrm{s}}=1$ and $(-\,\infty\,<n<\infty)$ $$ \omega_{m+n}\leq\omega_{m}\,\omega_{n} $$ foralinegers m and n Let $A=A\{\omega_{n}\}$ bethet ofall comle functions on the integer for which the norm · $$ \|f\|=\sum_{-\alpha}^{\infty}\left|f(n)\right|\omega_{n} $$ is finie Define multiplicationin $\textstyle{\mathcal{A}}$ 4 by $$ (f*g)(n)=\sum_{k=-\infty}^{\infty}f(n-k)g(k). $$ (d) Put (a) Show that each ${\mathcal{A}}\{\omega_{n}\}$ is a commutative Banach algebra exists and that $R_{-}\leq R\,,$ $\scriptstyle{R_{\circ}}$ (b) Show that $R_{+}\,=\,\vert\mathrm{i}\mathrm{Im}_{n arrow\infty,\mathsf{F}}\,(\omega_{n})^{1/n}$ exists and is finite, by showing that $\mathrm{i}\Omega f_{n\ge0}\;(\omega_{n})^{1/n}.$ $R_{-}=\mathrm{lim}_{n arrow\infty}(\omega_{-n})^{1/n}$ (c) Show similarly that ideal space of $\Delta-\{\lambda\in C\colon R_{-}\subseteq|\lambda|\leq R_{+}\}$ Show that $\bar{\Delta}$ can be identifid with the maximal ${\mathcal{A}}\{\omega_{n}\}$ and that the Gellan rnsorsare abotey civie Laurent series on △ Ge) Consider thefllowing choices for fo) () $\omega_{n}=1.$ (i) $\omega_{n}=2^{n}$ (ii)o $\;_{n}\doteq2^{n}\;\mathrm{i}\mathrm{f}\;n\geq0,\;\omega_{n}=1\;\mathrm{i}\mathrm{f}\;n\ll0.$ (iv) $\mathbf{\Delta}_{v_{n}}=1+2n^{2}.$ (v)w $\iota_{n}=1+2n^{2}{\mathrm{~if~}}n\geq0,$ 0,=1 ifn<0. that $\hat{A}$ For which of these i $\underline{{\land}}$ a circle"? For which choices is ${\hat{A}}\{\omega_{n}\}$ selfadjoint in the sense is closed under complex conjugation? $(g)$ Is there an $\scriptstyle A(\omega_{i})$ always semisimple ? the unit cicle, such that $\bar{A}$ consis ntirely of infntel $(f)\operatorname{Is}A\{\omega_{n}\}$ , with $\underline{{\Delta}}$ diferentiable functions?