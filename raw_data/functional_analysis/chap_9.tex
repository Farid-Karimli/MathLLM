9 TAUBERIAN THEORY Wiener's Theorem 9.1 Introduction A tauberian theorem is one in which the asymptotic behavior of a scqucncc or of a function is deduced from the behavior of som of its avcragcs Tauberian theorems are often converses of fairly obvious results, but usually these converses depend on some additional assumption,called a tauberian condition. To see an example of this, consider the following three properties of a sequence of complex numbers $s_{n}=a_{0}+\cdot\cdot\cdot+\,a_{n}$ (a) $\operatorname*{lim}_{x\to c}s_{n}=s.$ (b)If f(r) ,C,r r",0<r<1, then lim f(r) = s. (c)lim $n a_{n}=0.$ n→o $\mathrm{Since}f(r)=(1-r)\sum s_{n}r^{n}$ and $(1-r)\sum r^{n}=1,f(r)$ is, for each $r\in(0,1),$ an average of the sequence $\scriptstyle\{s_{a}\}$ It is extrcmely casy to prove that $\mathbf{\Psi}(a)$ implics (b). The converse is not true, but (b) and Cc) together imply Ga); this is also quite easy and was proved by Tauber. The tauberian condition Cc) can be replaced by the weaker assumption thatTAUBLKIAN TuEoRv 209 of lates of its taslates hece aso i $^cdot(K*\phi)(x)\to0,$ SAsosguol nmoaohomusm mn aeun iswek as $x\to+\infty,$ then it is alost tivial tha may be in $L^{1}.$ $K.$ ing of c) makes the proo as ${\cal X} arrow+\,\alpha\qquad$ for every $K\in L^{1}(R)$ The convolution $K\ast\phi$ of trans- $\textstyle K$ $L^{1}$ on te real line. 1f $\phi\in L^{\infty}(R)$ and if $\phi(x)\to0$ eirsnrsgsoses o amaersn einonin Wieners onverse iao Theore ${\mathcal{G}}$ ns aen $(K*\phi)(x)\to0$ for one $K\in L^{1}(R)$ and if the rurer trisormo ti iopoe clusion that When regarded s avrags o t east when $\phi\rangle(x) arrow0$ then $(f*\phi)(x)\to0$ for cery LR ester e enters the proof vanishes at no point o ${\boldsymbol{R}},$ $K=1.$ 9.7 statesthat if (K $\hat{K}$ $K$ neoq oulne e oes i isoso ioeso $\mathcal{G}$ $\phi(x)\to0$ $\phi$ le ${\hat{K}}{\boldsymbol{\Sigma}}{\boldsymbol{\Sigma}}{\boldsymbol{\Sigma}}$ in thefllowing manner:1 ou eus…oiss…smnasmsae the same is tre if $K$ is replaced by any or Thsusetuin oimmi e mionsn is red an inear iombihato h so sis aet se mincu CTho . i o esesasin 9.2 Lemma Suppose ${}^{t}\in L^{1}(R^{n}),\,t\in R^{n},$ and $\varepsilon>0$ Then thexis $h\in L^{*}(R),$ with $\|{\hat{h}}\|_{1}<\varepsilon,$ such that (1) $$ {\hat{h}}(s)={\hat{f}}(t)-{\hat{f}}(s) $$ for alls in some neighohood o , For The lemma state put $g\in L^{1}(R^{n})$ so that $\scriptstyle{\theta\to1}$ in some neighborhood of the orign $f+h$ PROOF Chooe $\operatorname{that}f{\mathrm{is}}$ approximated, in thc ${\boldsymbol{L}}^{1}$ '-norn, by a function wbosertsmosanmamisiesioifof eoI ${\boldsymbol{\lambda}}>0,$ (2) $$ g_{\lambda}(x)=e^{i t\cdot\,x}\lambda^{-n}g(x/\lambda)\qquad(x\in R^{n}, $$ and defne (3) with $h_{\lambda}$ in place of ${\mathit{j}}_{2}$ NexL, $V_{\lambda}$ of ,(3) shows hat(U olds fo Since ${\hat{g}}_{\lambda}(s)=1$ $$ h_{\lambda}(x)=\hat{f}(\iota)g_{\lambda}(x)-(f*g_{\lambda})(x). $$ in some neighborhood $s\in V_{\lambda},$ (4) $$ h_{\lambda}(x)=\int_{{\cal R}^{n}}\!f(y)[e^{-i t\cdot\cdot y}g_{\lambda}(x)-g_{\lambda}(\dot{x}-y)]\,d m_{n}(y). $$ Th about vle o te exerson brackes $({\cal5})$ $$ \left|\lambda^{-n}g(\lambda^{-1}x)-\lambda^{-n}g(\lambda^{-1}(x-y))\right|. $$210 DISTRrBvrToNS AND FoURIER TRANSroRMs lt follow tha (6) $$ \left\|h_{\lambda}\right\|_{1}\leq\int_{R^{n}}\left|f(y)\right|\,d m_{n}(y)\int_{R^{n}}\left|g(\zeta)-g(\zeta-\lambda^{-1}y)\right|\,d m_{n}(\zeta), $$ by the change of variables $x=\lambda\xi.$ The inner integral in(6) is at most 2lgll and it tends to O for every $y\in R^{n},$ as $\lambda\to\alpha\quad$ Hence $|h_{2}|_{1}\cdots0,$ as 入→Oo, by // the dominated convergence theorem 9.3 Theorem $I f\phi\in L^{\infty}(R^{n})$ , Y is a subspace of $L^{1}(R^{n}),$ , and (1) $$ f*\phi=0\,f o r\;e v e r y\,f\in Y, $$ then the set (2) $$ Z(Y)=\bigcap_{f\in Y}\{s\in R^{n}:{\hat{f}}(s)=0\} $$ contains the support of the tempered distribution ${\hat{\phi}}.$ PROOF Fix a point t in the complement of $\mathbb{Z}(Y).$ Then ${\hat{f}}(t)=1$ for a certain $f\in Y.$ Lemma 9.2 furnishes $h\in L^{1}(R^{n});$ , with $\|h\|_{1}<1,$ such that ${\hat{h}}(s)=1-{\hat{f}}(s)$ in some neighborhood ${\mathbf{}}V$ of t To prove the theorem, it suffices to show that ${\hat{\phi}}=0$ in ${\mathbf{}}V$ , or, equivalently, that ${\hat{\phi}}({\hat{\psi}})=0$ for every $\psi\in{\mathcal{G}}_{n}$ whose Fourier transform $\hat{\psi}$ has its support in ${\mathit{V}}.$ Since (3) $$ \hat{\phi}(\hat{\psi})=\phi(\hat{\psi})=(\phi\ast\psi)(0), $$ and since it suffices to show that $\phi*\psi=0$ rm ≥. Then $\|g_{m}\|_{1}\leq\|h\|_{1}^{m}\|\not v_{i}\|_{1},$ Fix suc ${\textrm{l a}}\psi.{\textrm{P u t}}g_{0}=\psi,g_{m}=h*g_{m-1}\mathrm{fo}I$ $\|h\|_{1}<1,$ the function $G-\geq g_{m}$ is in $L^{1}(R^{n}).$ Since $\bar{h}(s)=1\,-\bar{f}(s)$ on the support of ${\widehat{\mathcal{V}}}_{s}$ we have (4) $$ (1-\hat{h}(s))\hat{\psi}(s)=\hat{\psi}(s)\hat{f}(s)\qquad(s\in R^{n}), $$ or (5) $$ \hat{\psi}=\sum_{m=0}^{\infty}\hat{h}^{m}\hat{\psi}\hat{J}=\hat{G}\hat{J}. $$ Thus ${\boldsymbol{\psi}}=G*f,$ and $(1)$ implies (6) $$ \psi\star\phi=G\star f\star\phi=0. $$ 9.4 Wiener's theorem $\textstyle Y i s$ a closed translation-invariant subspace o $f L^{1}(R^{n})$ and i $\mathbb{Z}(Y)$ is empty, then $Y=L(R^{n}).$TAUBERIAN THEoRY 211 element of (Theorem 7.15), t fllows tha $\phi=0$ is translation-invariant means that $\tau_{x}f\in Y\mathrm{iff}\,f\in Y$ and PROOF To say that ${\cal{Y}}$ $x\in R^{n},$ lf $\phi\in L^{\infty}(R^{n})$ is such that $\textstyle{\lceil f\rceil\phi=0}$ for every fe ${\cal{Y}},$ the translation- invariance of $\boldsymbol{\mathit{I}}$ implies t $\operatorname{lat}f*\phi=0$ for every to ${\mathcal{P}}_{n}^{\prime}$ in a one-to-onc fashion is the zero and since the Fourier transform maps ${\mathcal{P}}_{n}$ $\scriptstyle f\in Y.$ By Theorem 9.3, the support of the distribution $\hat{\phi}$ is therefore empty hence ${\tilde{\phi}}=0$ (Theorem 6.24), as a distribution. Hence $\phi$ Thus $L^{\infty}(R^{n}).$ By the Hahn-Banach theorem, ths implies tha $Y=L(R^{n}).$ $Y^{\perp}=\{0\}.$ // 9.5 Theorem Suppose $K\in L^{1}(R^{n})$ $K,$ Then $Y=L^{1}(R^{n})$ if and only if $K(t)\neq0\,j$ for every subspace of $L^{1}(R^{n})$ that contains and $\mathbf{\Sigma}Y$ is the smalles closed translation-imarian $\scriptstyle t\in K^{\circ}$ PROOF Note that $Z(Y)=\{t\in R^{n}$ ${\hat{K}}(t)=0\}$ The thcorem ths asserts that the $Y=L^{1}(R^{n})$ if anoy f Ziyis empty. oehar fnis ihcor $\mathfrak{o}A\!\!\!\geq$ $J/J$ other half is trivial. 9.6 Definition A function $\phi\in L^{\infty}(R^{n})$ is said to be slowl ocilaing if to every $\scriptstyle a>0$ correspond an $A<\infty$ and a $\delta>0$ such that (1) $$ |\phi(x)-\phi(y)|<\varepsilon\qquad{\mathrm{if~}}|x|>A,|y|>A,|x-y|<\delta. $$ lf $\scriptstyle n=1.$ one can also define what it means for $\phi$ to be slowly oscillati $a t+\infty{\mathrm{:}}$ the requirement (l) is replaced by (2) $$ \vert\phi(x)-\phi(y)\vert<\varepsilon\qquad\mathrm{if~}x>A,\,y>A,\,\vert x-y\vert<\delta. $$ The same definition can of course be made at $\textstyle{\ --\alpha0}$ Note that every uniformly coninuous bounded function is sloly oscilating bu that some slowly oscillatin functions are not continuous We now come to Wiener's taubrian heorem; part(b) was add by Pit 9.7 Theorem (a)Suppose $\phi\in L^{\infty}(R^{n}).$ Ke L(R"), K()≠ 0 for every $\iota\in K^{*}$ ", and (1) $$ \operatorname*{lim}_{|x| arrow\infty}(K*\phi)(x)=a\hat{K}(0). $$ Then $\left(2\right)$ lx→α lim ( p)x) (0). for every fe $L^{1}(R^{n}).$212 DISTRIsurroNS AND roURIER TRANSFORMs (b)I/, in addition, $\phi$ is slowly oscillating, then (3) $$ \operatorname*{lim}_{|x| arrow\infty}\phi(x)=a. $$ PRoor Put $\psi(x)=\phi(x)-a.$ Let I ${\cal{Y}}$ be the set of $\mathrm{all}\,f\in L^{1}(R^{n})$ for which (4) $$ \operatorname*{lim}_{x\mid\to\infty}(f*\psi)(x)=0. $$ $f_{i}\in Y_{\cdot}\;\|f-f_{i}\|_{1}\to0.$ Since is a vector space. Also, $\boldsymbol{\mathit{I}}$ is closed. To see this, suppose lt is clear that $\mathbf{\Sigma}Y$ (5) $$ \|f*\psi-f_{i}*\psi\|_{\infty}\leq\|f-f_{i}\|_{1}\|\psi\|_{\infty}\,, $$ $f_{i}*\vartheta\vartheta arrow f*\vartheta$ uniformly on $R^{n}{\dot{\cdot}}$ hence (4) holds. Since (6) $$ ((\tau_{y}f)*\psi)(x)=(\tau_{y}(f*\psi))(x)=(f*\psi)(x-y), $$ $\boldsymbol{\mathit{I}}$ Theorem $\mathfrak{O}.{\mathfrak{S}}$ is translation-invariant. Finally $K\in Y.$ by(I), since $K*a=a\tilde{K}(0).$ $L^{1}(R^{n})$ nowapplies and shows that $Y=L^{1}(R^{n})$ Thus every /c and choose satisfies 4), which is the same as so $\operatorname{that}f\geq0,f(0)=1_{!}$ T his proves part $(a).$ as in Definition 9.6, By (2), $(2).$ If $\phi~~~~~~~~~~~~~~~~~~~~~~~~~~~~~~~~~~~~~~~~~~~~~~~~~~~~~~~~~~~~~~~~~~~~~~~~~~~~~~~~~~~~~~~~~~~~~~~~~~~~~~~~~~~~~~~~~~~~~~~~~~~~~~~~~~~~~~~~~~~~~~~~~~~~~~~~~~~~~~~~~~~~~~~~~~~~~~~~~~~~~~~~~~~$ $f\in L^{1}(R^{n})$ is slowly oscillating and if $\scriptstyle{\varepsilon>0}$ choose $\scriptstyle A$ and $\bar{\partial}$ if $|x|\geq\delta$ anc $1f(x)=0$ (7) $$ \operatorname*{lim}_{|x| arrow\infty}(f*\phi)(x)=a. $$ Also, (8) $$ \phi(x)-(f*\phi)(x)=\int_{|y|<\delta}[\phi(x)-\phi(x-y)]f(y)\,d m_{n}(y). $$ 1f $\mid x\mid>A+\delta$ , our choice of ${\cal A},$ ${\bar{\partial}},$ and $\boldsymbol{\mathit{f}}$ shows that (9) $$ \left|\,\phi(x)-(f*\phi)(x)\,\right|\,<\,\epsilon. $$ Now $(3)$ follows from (T) and (9) / This completes thc proof. (b) that 9.8 Remark If $n=1,$ Theorem ${\mathfrak{O}}{\mathcal{T}}$ can be modified in an obvious fashion, by writing $\varnothing\;$ $x\to+\infty$ in place of $|x|\to\infty$ wherever the latte occurs and by assuming in is slowly oscillating at ${}+\infty.$ . The proof remains unchanged. The Prime Number Theorem $D\!\!\!\!/$ 9.9 Introduction For any positive number $x,\,\pi(x)$ dcnotes the number of primes that satisfy $p\leq x.$ The prime number theorem is the statement that (1) x→0 X ${\underline{{\pi(x)}}}\log x=$ 1. limTAUBERIAN TuEoRY213 Wospros sn s ns a tuaenoeoe o ngm asos hat ${\boldsymbol{F}}.$ emno wiontore esop etaeiiee nctn iy in ${\boldsymbol{F}}$ wo ayohswoisieasisiscse iaoseaeuern draw a concusion about r fromknowledgeo $[x]\leq x;$ 2.10ratn. e e $\scriptstyle{\mathcal{X}}$ wil eposiv ume BS ietitea ${\mathfrak{p}}$ ana 1 $r_{\mathit{l}}$ will be the symbol $\scriptstyle d|n$ means h ${\mathcal{A}}$ wul nwas dete aprim amne $x-1<z$ Poive $D\!\!\!\!/$ and yj eistivetieDen (1) $$ \begin{array}{l l}{{\Lambda(n)=\displaystyle\left\{\log p\right.\qquad\mathrm{if}\not=p,\,p^{2},\,p^{3},\ldots,}}\\ {{\psi(x)=\displaystyle\sum_{n\leq x}\Lambda(n),\qquad\qquad\qquad\qquad\qquad\qquad\qquad\qquad\qquad\qquad\qquad\qquad\qquad\qquad\qquad\qquad\qquad\qquad\qquad\qquad\qquad\qquad\qquad\qquad\qquad\qquad\qquad\qquad\qquad\qquad\qquad\qquad\quad\quad\quad\quad\quad\quad\quad\quad\quad\quad\quad\quad\quad\quad\quad\quad\quad\quad\quad\quad\quad\quad\quad\quad\quad\quad\quad\quad\quad\quad\quad\quad\quad\quad\quad\quad\quad\quad\quad\quad\quad\quad\quad\quad\quad\quad\quad\quad\quad\quad\quad\quad\quad\quad\quad\quad\quad\quad\quad\quad\quad\quad\quad\quad\quad\quad\quad\quad\quad\quad\quad\quad\quad\quad\quad\quad\quad\quad\quad\quad\quad\quad\quad\quad\quad0\;0\;1}\quad\quad\quad\quad\quad\quad\quad\quad\quad\quad\quad\quad\quad\quad\quad\quad\quad\quad\quad\quad\quad\quad\quad\quad0\;0\;\quad\quad\quad\quad\quad\quad\quad\quad\quad\quad\quad0}&{\quad\quad\quad\quad\quad\quad\quad\quad\quad\quad0}&{\quad\quad\quad\quad\quad\quad\quad\quad\quad\quad\quad\quad\quad\quad\quad\quad\quad\quad\quad\quad\quad\quad\quad\quad\quad\quad\quad\quad\quad\quad\quad\quad\quad\quad\quad\quad\quad\quad\quad\quad\quad\quad $$ (2) (3) The following propertis o $\vartheta$ and ${\mathbf{}}F$ 'will be used : (4) $$ {\frac{\psi(x)}{x}}\leq{\frac{\pi(x)\log x}{x}}<{\frac{1}{\log x}}\div{\frac{\psi(x)\log x}{x\;\log\left(x/\log^{2}x\right)}} $$ if x>e,and (5) $$ F(x)=x\log x-x+b(x)\log x, $$ where $\scriptstyle b(x)$ remains bounded as ${\mathcal{X}} arrow G O.$ By (A) the rime mber theorm s consequc o elai (6) $$ \operatorname*{lim}_{x\to\infty}{\frac{\psi(x)}{x}}=1, $$ which i broed fom ) and S by turnteore rnoor or 4)Iog xlog ls heumer o owes $\boldsymbol{\mathit{P}}$ tha do oted 、. Henc $$ \textstyle y(x)=\sum_{p\leq x}{\sqrt{\log\,p}}\mathbf{\sqrt{log}}\,p\leq\sum_{p\leq x}1\cos x=\pi(x)\log x. $$ This givsthe frstinquality in 4).1 $1<y<x,$ then $$ \pi(x)-\pi(y)=\sum_{y<p\leq x}1\leq\sum_{y<p\leq x}{\frac{\log p}{\log y}}\leq{\frac{\not{\textstyle\not{l}}(x)}{\log y}}. $$ Hle nk. <+veo/logy Wt - /oyxhis e setsnali e214 Dsrnorios AND rouRER TRANsroxws PROOF OF(5) 1f $\scriptstyle n\;\!>1.$ then $$ F(n)-F(n-1)=\sum_{m=1}^{\infty}\,\left\{\vartheta\!\left({\frac{n}{m}}\right)-\vartheta\!\left({\frac{n-1}{m}}\right)\right\}\!. $$ The mth summand is O except when $\scriptstyle n/m$ is an integer, in which case $\mathrm{if}$ t iS $\Lambda(n/m),$ Hence $$ F(n)-F(n-1)=\sum_{m|n}\Lambda{\biggl(}{\frac{n}{m}}{\biggr)}=\sum_{d|n}\Lambda(d)=\log n. $$ primes. Since The last quality dends on the factorization f into a product of powers of disti $F(1)=0.$ ,we have computed tha (7) $$ F(n)=\sum_{m=1}^{n}\log m=\log\left(n!\right)\qquad(n=1,\,2,\,3,\,\dots), $$ which suggests comparison of $F(x)$ with the integral (8) $$ J(x)=\int_{1}^{x}\!\log\,t\,d t=x\log\,x-x+1. $$ 1 $n\leq x\leq n+1$ then (9) $$ J(n)<F(n)\leq F(x)\leq F(n+1)<J(n+2) $$ so that (10) $$ |F(x)-J(x)|\ <2\log\left(x+2\right). $$ Now(5) follows from (B8) and (10) 9.11 The Riemann zeta tunction As is the custom in analyuic number theory, it. In the half-plane ${\boldsymbol{\sigma}}>1$ complex variabls will now be writen in the form $s=\sigma\mid$ the zeta function is defined by the series (1) $$ \zeta(s)=\sum_{n=1}^{\infty}n^{-s} $$ $\mathrm{Since}\,|\,n^{-s}|=n^{-\sigma}.$ the series cnveres uniformy on every compact subset of this half-plane,and Kis holomorphic therc A simple computation gives $$ s\int_{1}^{N+1}[x]x^{-1-s}\,d x=s\sum_{n=1}^{N}n\int_{n}^{n+1}x^{-1-s}\,d x=\sum_{n=1}^{N}n^{-s}-N(N+1)^{-s}. $$ When $\sigma>1,\,N(N+1)^{-s} arrow0$ as $N arrow\mathbf{\nabla}\circ\mathbf{\nabla}$ Hence (2) $$ \zeta(s)=s\int_{1}^{\infty}[x]x^{-1-s}\,d x-{-\ (\sigma>1)}. $$TAUBERIAN THEoRY 215 If $b(x)\doteq[x]-x,$ it folow from (2) that (3) $$ \zeta(s)={\frac{<s}{s-1}}+s\int_{1}^{\infty}b(x)x^{-1-s}\,d x\qquad(\sigma>1). $$ sims $\boldsymbol{\partial}$ i oe s enans ore inhal pil $\operatorname{of}(S$ to $\sigma>0.$ which is holomorphic $\scriptstyle\sigma>0$ Thus 3) furnishs analytic contnuaio ,wit eiu Th most iram rpeywesh exceptfora simple poleats $\textstyle{\mathsf{\uparrow}}=1.$ $\sigma=1;$ nced s that has o zeros n the in (4) $$ \zeta(1+i t)\neq0\qquad(-\infty<t<\infty). $$ The prof of (4) depends on te identit (5) $$ \zeta(s)=\prod_{p}{(1-p^{-s})^{-1}}\qquad(\sigma>1). $$ Since $(1-p^{-s})^{-1}=1+p^{-s}+p^{-2s}+\cdots,$ and that $\sum p^{-\sigma}<\infty$ if $\scriptstyle\sigma>1.$ (5) shows that $\zeta(s)\neq0$ if $\scriptstyle\sigma>1$ the fac that the product (S) cquals the srio n nmdic csucn o intiataieseysoitveia uniqu actciainto roduc o pwes irpmes Sine (6) $$ \log\,\zeta(s)=\sum_{p}\sum_{m=1}^{\infty}m^{-1}p^{-m s}\qquad(\sigma>1). $$ Fix a real $\scriptstyle t\neq0$ ${\mathrm{ff}}\,\sigma>1.$ (6) implies that (7) $$ \log\,|\zeta^{3}(\sigma)\zeta^{*}(\sigma+i t)\zeta(\sigma+2i t)|\,-\sum_{p,m}m^{-1}p^{-m\sigma}\,\mathrm{Re}\,\{3+4p^{-i n t}+p^{-2i m t}\}\geq0, $$ Hence bccause $\mathrm{Re}\,\{3+4e^{i\theta}+e^{2i\theta}\}-2(1+\cos\theta)^{2}$ for all real ${\boldsymbol{\theta}},$ (8) $$ \left|(\sigma-1){\zeta(\sigma)}\right|^{3}\left|\frac{\zeta(\sigma+i t)}{\sigma-1}\right|^{4}\left|\zeta(\sigma+2i t)\right|\geq\frac{1}{\sigma-1}. $$ 1f $\zeta(1+i t)$ were $0,$ the left side of $(\mathbf{8})$ would converge to a limit, namly $|\zeta^{\prime}(1+i t)|^{4}|\zeta(1+2i t)|\,,$ as $\scriptstyle{\mathcal{O}}$ decreases to 1. Since the right side of(8) tends to infiniyths simossible, and 4s proved 9.12 Ingham'stauberian theorem Suypose $\mathcal{G}$ T is a real nondecreasing function on (0, $\circ\circ),\ g(x)=0\ i f x<1,$ (1) $$ G(x)\stackrel{x\ ^{\sim}\sim}\sum_{n=1}^{\infty}g\left(\frac{x}{n}\right)\qquad(0<x<\infty), $$ and (2) $$ G(x)=a x\log x+b x+x s(x), $$ where a $\underline{{b}}$ are constants and $s(x)\to0$ as $x\to\infty.\setminus T h e t$ (3) $\operatorname*{lim}\,x^{-1}g(x)=a.$216 DsruorroNs AND Fourtuex TRAsroxs If g is the function $\textstyle\psi$ defined in Section 9.10, Ingham's theorem implies, in view of Equations (3) and S) of Section 9.10 that(O) of Section 9.10 holds and this, as we saw there, gives the prime number theorem PRoor We first show that $x^{-1}g(x)$ is bounded.Since a $\mathbf{\Omega}\cdot\mathbf{\Omega}$ is nondecreasing, $$ \begin{array}{r}{g(x)-g\left({\frac{x}{2}}\right)\leq\sum_{n=1}^{\infty}(-1)^{n+1}g\left({\frac{x}{n}}\right)=G(x)-2G\left({\frac{x}{2}}\right)}\\ {=x\left(a\,\log\,2+\varepsilon(x)-\varepsilon(x)-\varepsilon\left({\frac{x}{2}}\right)\right)<A x,}\end{array} $$ where $\textstyle A$ is some constant. Sincc $$ g(x)=g(x)-g{\biggl(}{\frac{x}{2}}{\biggr)}+g{\biggl(}{\frac{x}{2}}{\biggr)}-g{\biggl(}{\frac{x}{4}}{\biggr)}+\cdots. $$ it follows that (4) $$ g(x)<A\left(x+{\frac{x}{2}}+{\frac{x}{4}}+\cdots\right)=2A x.~~~~~~~~~~~~~~~~~. $$ transforms in a familiar settin. For We now mak acange of vabes tht wil ealeus to use Fouri define $-\infty<x<\infty.$ (5) $$ h(x)=g(e^{x}),\qquad H(x)=\sum_{n=1}^{\infty}h(x-\log n). $$ Then $h(x)=0$ if x<0 $\d\L{\Delta}_{\L\L}H(x)=G(e^{x})$ ;hence $\left(2\right)$ becomes (6) $$ H(x)=e^{x}(a x+b+\varepsilon_{1}(x)) $$ where $\varepsilon_{1}(x)\to0$ as $x\to\infty.$ $\mathrm{If}$ (7) $$ \phi(x)=e^{-x}h(x)\qquad(-\infty<x<\infty), $$ then $\phi$ is bounded, by (4). We have to prove tha (8) $$ \operatorname*{lim}_{x\to\infty}\phi(x)=a. $$ P $\operatorname{ut}k(x)=[e^{x}]e^{-x},$ let be a positive irrational number, and define (9) $$ K(x)=2k(x)-k(x-1)-k(x-\lambda)\qquad(-\infty<x<\infty). $$ Then $K\in L^{1}(-\infty,\,\infty);$ in fact $e^{x}K(x)$ is bounded.See Exercise 8) $\mathbb{F}\ s-\mathbb{F}$ $\sigma+i t,\sigma>0,$ then formula $\left(\mathbf{2}\right)$ of Section 9.11 shows that $$ \stackrel{\circ\circ}{\h\left(x\right)e^{-x s}\,d x=\int_{o}^{\circ}[e^{x}]e^{-x(s+1)}\,d x=\int_{1}^{\circ}[y]y^{-2-s}\,d y=\frac{\zeta(1+s)}{1+s}. $$TAUBERIAN THEoRx 217 $\sigma\to0$ Repeat this with $k(x-1)$ and $k(x-\lambda)$ in place of $k(x),$ use (9), and then let The result is (10) $$ \int_{-\infty}^{\infty}K(x)e^{-i t x}\,d x=(2-e^{-i t}-e^{-i\lambda t}){\frac{\zeta(1+i t)}{1+i t}}. $$ Since $\zeta(1\ +i t)\neq0$ and since $\lambda$ is irrational, ${\mathcal{K}}(t)\neq0$ if $\scriptstyle t\neq0$ Since $\zeta\quad\zeta$ has a pole with residue $\mathbf{\hat{l}}$ at $s=1,$ the rght side of (1O) tends to $\mathbf{\tau}\vdash\lambda$ as $t\to0$ Thus $\bar{K}(0)\ne0.$ $u(x)=[e^{x}],$ To apply Wiener's theorem, we have to estimat . AIso, $u=v\ast\mu.$ Hlence and whose support is this set. By(S5) $H=h*\mu.$ $K\ast\phi.$ To do this, put let b te hatiti fction f O, o) n e be temasure $\log n\!:\ n=1,\,2,\,3,\,.\,.,$ hatassignsmassl to each point of the set (11) $$ (h*u)(x)=(h*v*\mu)(x)=(H*v)(x)=\int_{0}^{x}H(y)\,d y. $$ with respect to he normalized measure QNote that we now take convolutions with respect to Lcbesgue measure, not Since $m_{\mathrm{t}},\mathrm{)}$ (G)) and(1) imply that $$ (\phi*k)(x)=\int_{-\infty}^{\infty}e^{y-x}h(x-y)[e^{y}]e^{-y}\,d y=e^{-x}(h*u)(x), $$ (12 $$ (\phi*k)(x)=e^{-x}\int_{0}^{x}H(y)\,d y=a x+b-a+\varepsilon_{2}(x), $$ where $s_{2}(x)\to0$ as $x\to\infty.$ By (12) and (9), (13) $$ \operatorname*{lim}_{x arrow\infty}\,(K rightarrow\phi)(x)=(1+\lambda)a=a\,\int_{-\infty}^{\infty}K(y)\,d y. $$ Therefore Winer's theorem 9.7 e also Rcmark 9.8 implies tha (14) $$ \operatorname*{lim}_{x arrow\infty}(f\ast\phi)(x)=a\int_{-\infty}^{\infty}f(y)\,d y $$ Thus for every fe ${\mathfrak{4}}f_{\mathfrak{2}}$ and $f_{2}$ be nonngative functions whose integral is l and whose is nondecreasig $L^{1}(-\infty,$ cO). Le supports lie in $\scriptstyle[0,\,c]$ and $[-\varepsilon,0],$ respectively. By $(7),\,e^{x}\phi(x)$ if $x\leq y\leq x+s$ $\phi(y)\leq e^{s}\phi(x)$ if $x-\varepsilon\leq y\leq x,$ and $\phi(y)\geq e^{-s}\phi(x)$ Consequenty, (15) $$ e^{-\varepsilon}(f_{1}*\phi)(x)\leq\phi(x)\leq e^{\varepsilon}(f_{2}*\phi)(x). $$ lie between $\scriptstyle a e^{-{\frac{a}{b}}}$ and ae'.Since 1 folws fom(14 and (5 thath uperand lowe is o Ge).a $x arrow\infty,$ $1{\mathrm{S}}$ complete. $\scriptstyle{\pi\gg0}$ was arbitrary,(8) holds, and uhe proof /218 DISTRIBUriONS AND roURIER TRANSFORMS The Renewal Equation As anotheraplication of Wieners taubrian theorem weshl now give a bref di of the integral equation cussion of the behavior of bounded solutions $\phi~~~~~~~~~~~~~~~~~~~~~~~~~~~~~~~~~~~~~~~~~~~~~~~~~~~~~~~~~~~~~~~~~~~~~~~~~~~~~~~~~~~~~~~~~~~~~~~~~~~~~~~~~~~~~~~~~~~~~~~~~~~~~~~~~~~~~~~~~~~~~~~~~~~~~~~~~~~~~~~~~~~~~~~~~~~~~~~~~~~~~~~~~~~~~~~~~~~~~~~~~~~~~~~~~~~~~~~~~~~~~~~~~~~~~~~~~~~~~~~$ $$ \phi(x)-\int_{-\infty}^{\infty}\phi(x-t)\,d\mu(t)=f(x) $$ givenctionan which ocus in probabiliy theory. Ilere $\boldsymbol{\mu}$ is a givcn Borel probability measure,fis a exists for every $x\in R$ $\phi$ is asmed to be a bounded Borel function so that te integr : The equation can be written in the form $$ \phi-\phi\star\mu=f, $$ for brevity We begin with a uniqueness theorem. 9.13 Theorem If u is $\overline{{a}}$ 1 Borel probabiliy measure on ${\boldsymbol{R}}$ whose suport does not $l i e$ in any cyclic subgroup of ${\boldsymbol{R}},$ ,ad if pis a bounded Borel Junction that satisfies the homo- geneous equation (1) $$ \phi(x)-(\phi*\mu)(x)=0 $$ for every $x\in R.$ then there is a constant $\scriptstyle A$ such that $\phi(x)=A$ except possily n $\overline{{\alpha}}$ set of Lebesgue measure $\mathbf{0}$ PROOF Since $\boldsymbol{\mu}$ is a probability measure, ${\hat{\mu}}(0)=1.$ Suppose that ${\hat{m}}(t)=1$ for some $\scriptstyle t\neq0$ Since $$ {\hat{\mu}}(t)=\int_{-\infty}^{\infty}e^{-i x t}\,d\mu(x), $$ (2) lf it follows that $\boldsymbol{\mu}$ must be concentrated on the set of all $2\pi/t.$ /t. But this is ruled out by the $e^{-i x t}=1,$ $\overline{{\chi}}$ at which if and only if $\scriptstyle t\,=\,0$ that is, on the set of all inegral mutiples o ${\hat{\sigma}}=1-{\hat{\mu}}.$ Hence $\delta(t)=0$ hypothesis of the theorem $\sigma=\delta-\mu,$ where is the Dirac measure, then , and (I) can be written in the form (3) $$ \phi*\sigma=0. $$ so that space generated by $K$ in place of $\textstyle Y_{\cdot}$ and sincec $\phi~~~~~~~~~~~~~~~~~~~~~~~~~~~~~~~~~~~~~~~~~~~~~~~~~~~~~~~~~~~~~~~~~~~~~~~~~~~~~~~~~~~~~~~~~~~~~~~~~~~~~~~~~~~~~~~~~~~~~~~~~~~~~~~~~~~~~~~~~~~~~~~~~~~~~~~~~~~~~~~~~~~~~~~~~~~~~~~~~~~~~~~~~~~~~~~~~~~~~~~~~~~~~~~~~~~~~~~~~~~~~~~~~~~~~~~~~~~~~~~~~~~~~~~~~~~~~~~~~~~~~~~~~~~~~~~~~~~~~~~~~~~~$ )the distribution $K\in L^{1},\,K(t)=0$ only "if $\frac{1}{j}$ Hence Put $\phi$ is a polynomial, in the distribution sense. p is assumed to be bounded, we have $\hat{\phi}$ $g(x)=\exp\left(-x^{2}\right)\!\cdot\!$ put $K=g*\sigma.$ Then (with the one-dimensiona $t=0,\;{\mathrm{and}}\;(3)$ shows that $K*\phi=0$ .By Theorem ${\mathfrak{g}}_{.{\bar{3}}}$ has its support in {0} is a fnite linear combination of $\hat{\phi}$ $\bar{\partial}$ and its dcrvatives (Theorem 6.25) Since nonconstant poly- nomials are not bounded on ${\boldsymbol{R}},$ reached the desired conclusion. $J/I$TAunEuAN THEoxv219 then 2.14 Convolutions of measures Ir $\boldsymbol{\mu}$ and are complex Borcl measures $\textstyle{R^{n}},$ (1) $$ f\to\int_{\boldsymbol{R^{n}}}\int_{\boldsymbol{R^{n}}}f(x+y)\,d\mu(x)\,d\lambda(y) $$ mw is a bounded linear functional on $C_{0}(R^{n}).$ the sace ofal coninus functions o $R^{n}$ measure $\mu*{\scriptstyle\lambda}$ on ${\textstyle R}^{n}$ that satisfies Ms e iasmesmesisosn (2) $$ \int_{R^{n}}f\,d(\mu*\lambda)=\int_{R^{n}}\int_{R^{n}}f(x+y)\,d\mu(x)\,d\lambda(y)\qquad[f\in C_{0}(R^{n})]. $$ Aotomnspsisingmn swa n oasorve ouea Borel functon . prticulaw seta (3) $$ {\bf\nabla}^{\prime}\qquad\bigl(\mu\star\lambda\bigr)^{\times}=\hat{\mu}\hat{\lambda}.\qquad. $$ almost obvious inequality Tooses o use n ne teoem one isn (4) $$ \|\mu*\lambda_{\|}\,\leq\,\|\mu\|\,\|\lambda\|, $$ where hnom deota varaion Theote h at ta if thsis true of $\mu;$ for in that case continuous Gelative to Lebesguemasue $\mu*\lambda$ is absolutely $m_{a})$ (5) $$ \int_{R^{n}}f(x+y)\;d\mu(x)=0 $$ tion for every $y\in R^{n},{\mathrm{if}}f{\mathrm{is}}$ the chati fucton ofaBorel s $\boldsymbol{\mathit{I}}$ i has a unique Lebesgue decomposi- and (2) shows that $(\mu*\lambda)(E)=0.$ $\boldsymbol{E}$ with $m_{n}(E)=0,$ Recal tateverycomplex Bore mcasue (6) $$ \mu=\mu_{a}+\mu_{s}, $$ where $\textstyle H_{a}$ .is absolutely continuous rclative to $T l_{n}$ , and $\textstyle\mu_{s}$ L is singular The next therm s due to Karin 9.15 Theorem Suppose uis Bore pobliy meawe onR,such tha (1) $$ \mu_{a}\neq0, $$ (2) $$ \textstyle{\int_{-\infty}^{\infty}|x|\,d\mu(x)<\infty,} $$ (3) M xdu(x)≠ 0220 DIsTRuDUrIoNS AND roUnIER TRANSroRMs Suppose that f∈ $\dot{L}^{1}(R),\,t h a t{f(x)} arrow0\;a s\;x arrow\pm\infty,$ and that $\varnothing\;$ is a bounded function that satisfies (4) $$ \phi(x)-(\phi*\mu)(x)=f(x)\qquad(-\infty0<x<\infty). $$ Then the limis (5) $$ \phi(\infty)=\operatorname*{lim}_{x\to\infty}\phi(x),\qquad\phi(-\infty)=\operatorname*{lim}_{x\to-\infty}\phi(x) $$ exist, and (6) $$ \phi(\infty)-\phi(-\infty)={\frac{1}{M}}\int_{-\infty}^{\infty}\!f(y)\,d y. $$ PRoor Put $\sigma=\delta-\mu,$ . as in the proof of Theorem 9.13. Define (T) $$ K(x)=\sigma((-\infty,x))={\binom{-\mu((-\infty,x))}{\mu([x,\infty))}}\qquad{\mathrm{if~}}x\leq0, $$ The assumption (2) guarantees that $K\in L^{*}(R),$ A straightforward computation, whose details we omit, shows that (8) $$ \begin{array}{r l}{\int_{-\infty}^{\infty}K(x)e^{-i x t}\,d x={\binom{\hat{\sigma}(t)/i t}{M}}}&{{}~{\mathrm{~if~}}t\neq0}\\ {-\infty}\end{array} $$ and that (9) $$ \left.\int_{r}^{s}\!f(x)\,d x=(K*\phi)(s)-(K*\phi)(r)\qquad(-\infty<r<s<\infty),\qquad\right. $$ since $f=\phi*\sigma.$ is not singular. The argument used at the beginngof the proo By (1), $\boldsymbol{\mu}$ of Theorem 9.13 shows therefore that ${\hat{\sigma}}(t)\neq0$ if t $\scriptstyle{\gamma^{a}=0}$ Hence(8) and (3) imply that $\hat{\mathbf{K}}$ has no zero in $\textstyle{\mathcal{R}}$ Since $f\in L^{1}(R),\,(5$ D implies that $K\ast\phi$ has limits at ±o,whose difference is $\textstyle\int_{-\infty}^{\omega}f.$ bis slowly oscilating. Once this is done,(5) and (6) We shall show that q $\phi~~~~~~~~~~~~~~~~~~~~~~~~~~~~~~~~~~~~~~~~~~~~~~~~~~~~~~~~~~~~~~~~~~~~~~~~~~~~~~~~~~~~~~~~~~~~~~~~~~~~~~~~~~~~~~~~~~~~~~~~~~~~~~~~~~~~~~~~~~~~~~~~~~~~~~~~~~~~~~~~~~~~~~~~~~~~~~~~~~~~~~~~~~~~~~~~~~~~~~~~~~~~~~~~~~~~~~~~~~~~~~~~~~~~~~~~~~~~~~~$ 办 and K p that we just proved, by Pit's theorem follow from the properties of $\textstyle K$ (b) of 9.7 Repeated substitution of $\phi=f+\phi*\mu$ into its right-hand sidc givcs (10) $$ \phi=f+f*\,p+\,\cdots\,+f*\,\mu^{n-1}\,+\,\phi*\,\mu^{n} $$ $$ \mathbf{\Phi}=f_{n}+g_{n}+h_{n}\qquad(n=2,\,3,\,4,\,\dots), $$ where $\mu^{1}=\mu,\mu^{n}=\mu*\mu^{n-1},f_{n}=f+\cdots+f*\mu^{n-1},\,\mathrm{and}\,$ (11 $g_{n}=\phi*(\mu^{n})_{a},~~~~~~h_{n}=\phi*(\mu^{n})_{s}.$TAURERMN THEoRy 221 $f_{n}+g_{n}$ For cach n,/,x)→0a $x\mapsto\pm\infty.$ and $\scriptstyle{\theta_{a}}_{\mathrm{t}}$ fiformy continuous. Htence (12) is lo osi sic o anosoanan we have $$ \|(\mu^{n})_{s}\|\leq\|(\mu_{s})^{n}\|\leq\|\mu_{s}\|^{n}, $$ (13) $$ \bigl|h_{n}(x)\bigr|\leq\ \|\phi\|\cdot\|h_{s}\|^{n}\qquad(-\infty<x<\infty), $$ ${\boldsymbol{R}}.$ By(D) $\|\mu_{\mathrm{el}}\|<1,$ oio $h_{n}\to0,$ proof where Il s the supemum or $\left|\,\phi\,\right|$ on Hence // functions mo ",A coern Thi ipes th $\phi$ xu uo $f_{n}+g_{n}$ $\phi~~~~~~~~~~~~~~~~~~~~~~~~~~~~~~~~~~~~~~~~~~~~~~~~~~~~~~~~~~~~~~~~~~~~~~~~~~~~~~~~~~~~~~~~~~~~~~~~~~~~~~~~~~~~~~~~~~~~~~~~~~~~~~~~~~~~~~~~~~~~~~~~~~~~~~~~~~~~~~~~~$ is slowy ocilting,and ompeest Exercises 2 Suppose $\phi\in L^{\alpha}(R^{n})$ roveteorcm o Tauestae n scti $\phi$ matimomme such that $\widehat{\phi\,}$ consists of $\left\{\begin{array}{l l}{K}&{\quad}\end{array}\right.$ distinc pon $\{g_{j}\}$ $S_{1,\ \epsilon}\ast\cdot\cdot,\cdot S_{k}$ Construct suiabl fnction ${\sqrt{1}}_{1},\ \circ\circ\circ\cdot\psi$ $9.1.$ has the singeto an th supot o e itiutio a sunascocun $(\phi*\psi_{j})^{*}$ m $$ \phi(x)=a_{i}e^{i{\mathbf{_{i}}}\cdot{\mathbf{x}}}+\cdots+a_{k}e^{i{\mathbf{_{k}}}\cdot{\mathbf{x}}}\qquad({\mathbf{a}}{\cdot}{\mathbf{x}}). $$ 6 If of 化 (The case $\scriptstyle{k=1}$ is on n t roT neorm .1. RD such tha $\mathbb{Z}(Y)$ consists as 5 $\boldsymbol{k}$ Suppose $\boldsymbol{\mathit{I}}$ ,i sasainaisae $\mathbf{\Psi}(a)$ of Thcorcm 9.7: 1 ${\boldsymbol{R}}^{n}$ Show that thcn uhere exis to prove that 中 $f\in I,^{1}(R^{n})$ Y has coiension $\boldsymbol{k}$ and $\bar{\mathbb{X}}$ dioios o…s m romem9 ${\mathcal{D}}(R$ e ece $\mathbf{\nabla}2$ $t\in R^{n}$ conclusion of $\mathbf{\Psi}(a)$ $\ ^{*}\phi(x)=\sin\left(x^{2}\right),\;-\infty<x<\infty,\,{\mathrm{s}}\,{\mathrm{s}}\,{\mathrm{s}}\,{\mathrm{s}}\,{\mathrm{s}}\,{\mathrm{s}}\,{\mathrm{s}}\,{\mathrm{s}}\,{\mathrm{s}}\,{\mathrm{s}}\,{\mathrm{s}}\,{\mathrm{s}}\,{\mathrm{s}}\,{\mathrm{s}}\,{\mathrm{s}}\,{\mathrm{s}}\,{\mathrm{s}}\,{\mathrm{s}}\,{\mathrm{s}}\,{\mathrm{s}}\,{\mathrm{s}}\,{\mathrm{s}}\,{\mathrm{s}}\,{\mathrm{s}}\,{\mathrm{s}}\,{\mathrm{s}}\,{\mathrm{s}}\,{\mathrm{s}}\,{\mathrm{s}}\,{\mathrm{s}}\,{\mathrm{s}}\,{\mathrm{s}}\,{\mathrm{s}}\,{\mathrm{s}}\,{\mathrm{s}}\,{\mathrm{s}}\,{\mathrm{s}}\,{\mathrm{s}}\,{\mathrm{s}}$ show that and concude frmt itist $x\in R^{n},$ atithough cosis f exactly those $\phi\in L^{\infty}(R^{n})$ such that k in $L^{1}(R^{n}),$ as $|x|\to\infty,$ for $\operatorname{cvcry}f\in L^{*}(R^{*}).$ ${\cal{Y}}$ and $(K_{t}*\phi)(x)\to0$ Assume $K\in L^{1}(R^{r})$ whose Fouricr ransfoms ar ${\mathfrak{U}}$ hat every point of $\scriptstyle{Z(Y)}$ and if to ever $|x|\to\infty,$ then Frove e oowinanie sch tha $K_{i}(t)\neq0$ $\phi\in L^{\infty}(R^{n}),$ corresponds a function $K_{t}\in L^{1}(R^{n})$ $(f*\phi)(x)\to0$ $(K*\phi)(x)=0$ has at least one zeroTn does not satisfy th or Theorem 9.7 for ever $\phi$ $$ \operatorname*{lim}_{|x|\to\infty}(f*\phi)(x)=0 $$ 8 If lates of g is dense i $L^{1}(R):{\mathfrak{f}}$ frecxy L"A0. uh n ecsno o Thcoe ${\mathfrak{g}}.{\mathcal{Y}}$ ai Sn $f_{\theta}$ f。 in the 7 For same way; put $g=f_{x}+f_{\beta}$ 。 e heetitotonmce does not hold Tian $\scriptstyle\alpha>0,$ let $f_{\alpha}$ $\scriptstyle\alpha>0$ and $\alpha x=I,$ Frot st t ninint omsnaus prove that and onyir l s rationa $1-\alpha<\alpha|x\}\leq1,$ ma edeon uaceoaunae nenorom mn222 DISTRsurIoNS AND FoURIER TRANSFORMS $I O$ (c) $\operatorname{If}\mu$ is concentrated on ${\cal Q},$ and let $\phi$ rational numbers. ${\mathrm{Lef}}^{*}\mu$ be a probability measure on Show that $\scriptstyle\theta(x)=$ 9 (6) If $\scriptstyle(\phi_{\gamma})$ and $\lambda$ for every $x\in R.$ $k\in L^{1}(R^{n}),$ is a sequence of slowy oscilatng functions on ${\boldsymbol{R}}^{n}$ ${\boldsymbol{R}}$ that Let ${\cal{Q}}$ denote the set of and ${\mathrm{all}}$ s be the characteristic function of ${\cal Q}.$ (a) $(\partial*\mu)(x)$ although $\phi$ is not constant.(Compare with Theorem 9.13. Prove them $\operatorname{If}\,\phi\in L^{\infty}(R^{n})$ What other sets could be used in plac of then $k\ast\phi$ is uniformly continuous that converges uniformly ${\boldsymbol{Q}}$ to achicvc the same effect ? Special cas of he followingfacts were used in Theorem 9.15. to a function $\phi,$ , then $\boldsymbol{\phi}$ is slowly oscillating then are complex Borel measures on $\textstyle R^{n}\!,$ $$ \|(\mu*\lambda)_{s}\|\leq\|\mu_{s}\|\,\|\lambda_{s}\|. $$ ${\cal I}_{\llap{I}}$ Put $\vartheta(x)=\cos\left(|x|^{1/3}\right)$ and define $$ f(x)=\psi(x)-\frac{1}{2}\int_{-1}^{1}\psi(x-y)\,d y\qquad(-\cos\,<x<\infty). $$ Prove t $\operatorname{hat}f\in(L^{1}\cap C_{0})(R)$ but that no bounded solution of the cquation $$ \phi(x)\,-{\frac{1}{2}}\int_{-1}^{1}\phi(x-y)\,d y=f(x) $$ has limits at +oo or at $-\infty.$ GThis ilustates the relevanc of the conditio $M\neq0$ in ${\mathit{1}}{\mathit{2}}$ $\phi$ on Theorem 9.15.) which is periodic with period 1 satisfies $\phi-\phi*\mu=0.$ (This is relevant to Let ${\mathit{l}}^{2}$ be a probability measure concentrated on the integers. Prove that every functin $\textstyle{\mathcal{K}}$ Theorems 9.13 and 9.15.) ${\boldsymbol{\mathit{1}}}{\boldsymbol{\bar{3}}}$ Assume $\phi\in L^{\infty}(0,$ co), $$ \begin{array}{l l}{{\int_{0}^{\infty}|K(x)|\;\displaystyle{\frac{d x}{x}}<\infty,}}&{{\;\;\int_{0}^{\infty}|H(x)|\;\displaystyle{\frac{d x}{x}}<\infty,}}\\ {{\int_{0}^{\infty}K(x)x^{-i t}{\frac{d x}{x}}\not{\neq}0}}&{{\;\;\mathrm{for}-\infty<t<\infty,}}\end{array} $$ and Prove that $$ \operatorname*{lim}_{x\to o}\int_{0}^{\infty}K\biggl({\frac{x}{u}}\biggr)\phi(u)\,{\frac{d u}{u}}=0.\qquad\qquad\qquad\qquad\qquad\qquad.\qquad.\qquad.\qquad. $$ $I{\mathcal{M}}$ Assume to be defined to obtain thecorresponding analog of $\mathbf{(}b\mathbf{)}$ of Theorem How would “slowly oscillating” have it is This is an analoguc of $\mathbf{\Psi}(a)$ of Thcorem ${\mathfrak{g}}.{\mathcal{T}}.$ $9.7\uparrow$ $|\,n a_{n}|\leq1,\;f(r)=\sum_{\cdots}^{\infty}\,a_{n}r^{n},$ and Complete the detais n th following outline of Wicncr' proof f Ltleod's theorem as r→1. $s_{n}=a_{0}+\cdot\cdot\cdot+a_{n}\,,$ $f(r)\to0$ $\mathrm{tO}$ be proved that Sn→>0 as $n\to\infty.$TAUBEuAN THronv 22 (a) $\left|s_{n}-f(1-1/n)\right|<2.$ Hence $\{g_{n}\}{\mathrm{~i}},$ s bounded. then (の) If 4ax)=s。 on , $n+1)\quad$ and $0<x<y,$ $$ \left|\phi(y)-\phi(x)\right|\leq\frac{(y+1-x)}{x}\,. $$ (e) Js xe-"pu d =/(e")→0a $x\to0$ Hence $$ \operatorname*{lim}_{x\to c}\,\int_{0}^{x}K\biggl(\frac{x}{u}\biggr)\phi(u)\,\frac{d u}{u}=0 $$ if $$ K(x)=\left({\frac{1}{x}}\right)\exp\left(-{\frac{1}{x}}\right). $$ a K(x dx - 7(1 in + Oif is ea (e) Put $H(x)=1/(\varepsilon x)$ ir + 0-'<x<1, M0 = otrwise Cocnodeth $$ \operatorname*{lim}_{x\to\infty}{\frac{1}{e x}}\int_{x}^{(1)}\!\!\!_{\left({1\right)}x}^{\frac{1}{4}}\phi(y)\,d y=0. $$ () By $\mathbf{\nabla}(b)$ and Ge), Ii $\phi(x)=0$ ${\boldsymbol{J}}{\boldsymbol{S}}$ f* 9 $\gamma\in\mathfrak{I}$ Y whenever $\scriptstyle{f\in Y}$ and $g\in L^{1}(R^{n}).$ is sme olt.then modiaio o se $L^{1}(R^{n})$ are thus exacty the same as is all that is Let Note: If $n a_{n}\to0$ $L^{1}(R^{n}).$ Rrovethat Yis trationinan i an ony i $\mathbf{\Psi}(a)$ $\mathbf{\Sigma}Y$ needed for the proof Y bea closed subspace o The cose ationivarantsaces o $L^{1}(R^{n})$ the cosedal n th convouionangecbPART THREE Banach Algebras anc Spectral Theory