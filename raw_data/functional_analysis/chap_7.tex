\documentclass[10pt]{article}
\usepackage[utf8]{inputenc}
\usepackage[T1]{fontenc}
\usepackage{amsmath}
\usepackage{amsfonts}
\usepackage{amssymb}
\usepackage[version=4]{mhchem}
\usepackage{stmaryrd}
\usepackage{mathrsfs}
\usepackage{bbold}

\begin{document}
\section{FOURIER TRANSFORMS}
\section{Basic Properties}
7.1 Notations (a) The normalized Lebesgue measure on $R^{n}$ is the measure $m_{n}$ defined by

$$
d m_{n}(x)=(2 \pi)^{-n / 2} d x
$$

The factor $(2 \pi)^{-n / 2}$ simplifies the appearance of the inversion theorem $\bar{T} . \bar{T}$ and the Plancherel theorem 7.9. The usual Lebesgue spaces $L^{p}$, or $L^{p}\left(R^{n}\right)$, will be normed by means of $m_{n}$ :

$$
\|f\|_{p}=\left\{\int_{R^{n}}|f|^{p} d m_{n}\right\}^{1 / p} \quad(1 \leq p<\infty)
$$

It is also convenient to redefine the convolution of two functions on $R^{n}$ by

$$
(f * g)(x)=\int_{R^{n}} f(x-y) g(y) d m_{n}(y)
$$

whenever the integral exists.

(b) For each $t \in R^{n}$, the character $e_{t}$ is the function defined by

$$
e_{t}(x)=e^{i t \cdot x}=\exp \left\{i\left(t_{1} x_{1}+\cdots+t_{n} x_{n}\right)\right\} \quad\left(x \in R^{n}\right)
$$

Each $^{\prime} e_{t}$. satisfies the functional equation

$$
e_{t}(x+y)=e_{t}(x) e_{t}(y)
$$

Thus $e_{t}$ is a homomorphism of the additive group $R^{n}$ into the multiplicative group of the complex numbers of absolute value 1 .

(c) The Fourier transform of a function $f \in L^{1}\left(R^{n}\right)$ is the function $\hat{f}$ defined by

$$
\hat{f}(t)=\int_{R^{n}} f e_{-t} d m_{n} \quad\left(t \in R^{n}\right)
$$

The term "Fourier transform" is often also used for the mapping that takes $f$ to $\hat{f}$. Note that

$$
\hat{f}(t)=\left(f * e_{t}\right)(0)
$$

(d) If $\alpha$ is a multi-index, then

$$
D_{\alpha}=(i)^{-|\alpha|} D^{\alpha}=\left(\frac{1}{i} \frac{\partial}{\partial x_{1}}\right)^{\alpha_{1}} \cdots\left(\frac{1}{i} \frac{\partial}{\partial x_{n}}\right)^{\alpha_{n}}
$$

The use of $D_{\tilde{x}}$ in place of $D^{\alpha}$ simplifies some of the formaiism. Note that

$$
D_{\alpha} e_{t}=t^{\alpha} e_{t}
$$

where, as before, $t^{\alpha}=t_{1}^{\alpha_{1}} \cdots t_{n}^{\alpha_{n}}$. If $P$ is a polynomial of $n$ variables, with complex coefficients, say

$$
P(\xi)=\sum c_{\alpha} \xi^{\alpha}=\sum c_{\alpha} \xi_{1}^{\alpha_{1}} \cdots \xi_{n}^{\alpha_{n}}
$$

the differential operators $P(D)$ and $P(-D)$ are defined by

$$
P(D)=\sum c_{\alpha} D_{\alpha}, \quad P(-D)=\sum(-1)^{|\alpha|} c_{\alpha} D_{\alpha} .
$$

It follows that

$$
P(D) e_{t}=P(t) e_{t} \quad\left(t \in R^{n}\right) .
$$

(e) The translation operators $\tau_{x}$ are defined, as before, by

$$
\left(\tau_{x} f\right)(y)=f\left(y_{*}-x\right) \quad\left(x, y \in R^{n}\right) .
$$

7.2 Theorem Suppose $f, g \in L^{1}\left(R^{n}\right), x \in R^{n}$. Then

(a) $\left(\tau_{x} f\right)^{\wedge}=e_{-x} \hat{f}$

(b) $\left(e_{x} f\right)^{\wedge}=\tau_{x} \hat{f}$

(c) $(f * g)^{\wedge}=\hat{f} \hat{g}$.

(d) If $\lambda>0$ and $h(x)=f(x / \lambda)$, then $\hat{h}(t)=\lambda^{n} \hat{f}(\lambda t)$.

PROOF It follows from the definitions that

$$
\left(\tau_{x} f\right)^{\wedge}(t)=\int\left(\tau_{x} f\right) \cdot e_{-t}=\int f \cdot \tau_{-x} e_{-t}=\int f \cdot e_{-t}(x) e_{-t}=e_{-x}(t) \hat{f}(t)
$$

and

$$
\left(e_{x} f\right)^{\wedge}(t)=\int e_{x} f e_{-t}=\int f e_{-(t-x)}=\left(\tau_{x} \hat{f}\right)(t)
$$

An application of Fubini's theorem gives $(c) ;(d)$ is obtained by a linear change of variables in the definition of $\hat{f}$.

7.3 Rapidily decreasing functions This name is sometimes given to those $f \in C^{\infty}\left(R^{\prime \prime}\right)$ for which

$$
\sup _{|\alpha| \leq N} \sup _{x \in R^{n}}\left(1+|x|^{2}\right)^{N}\left|\left(D_{\alpha} f\right)(x)\right|<\infty
$$

for $N=0,1,2, \ldots$ (Recall that $|x|^{2}=\sum x_{i}^{2}$.) In other words, the requirement is that $P \cdot D_{\alpha} f$ is a bounded function on $R^{n}$, for every polynomial $P$ and for every multiindex $\alpha$. Since this is true with $\left(1+|x|^{2}\right)^{N} P(x)$ in place of $P(x)$, it follows that every $P \cdot D_{\alpha} f$ lies in $L^{1}\left(R^{n}\right)$.

These functions form a vector space, denoted by $\mathscr{S}_{n}$, in which the countable collection of norms (1) defines a locally convex topology, as described in Theorem 1.37.

It is clear that $\mathscr{P}\left(R^{n}\right) \subset \mathscr{S}_{n}$.

\subsection{Theorem}
(a) $\mathscr{S}_{n}$ is a Fréchet space.

(b) If $P$ is a polynomial, $g \in \mathscr{S}_{n}$, and $\alpha$ is a multi-index, then each of the three mappings

$$
f \rightarrow P f, \quad f \rightarrow g f, \quad f \rightarrow D_{\alpha} f
$$

is a continuous linear mapping of $\mathscr{S}_{n}$ into $\mathscr{S}_{n}$.

(c) If $f \in \mathscr{S}_{n}$ and $P$ is a polynomial, then

$$
(P(D) f)^{\wedge}=P \hat{f} \quad \text { and } \quad(P f)^{\wedge}=P(-D) \hat{f}
$$

(d) The Fourier transform is a continuous linear mapping of $\mathscr{S}_{n}$ into $\mathscr{S}_{n}$.

[Part $(d)$ will be strengthened in Theorem 7.7.]

PROOF (a) Suppose $\left\{f_{i}\right\}$ is a Cauchy sequence in $\mathscr{S}_{n}$. For every pair of multiindices $\alpha$ and $\beta$ the functions $x^{\beta} D^{\alpha} f_{i}(x)$ converge then (uniformly on $R^{n}$ ) to a bounded function $g_{\alpha \beta}$, as $i \rightarrow \infty$. It follows that

$$
g_{\alpha \beta}(x)=x^{\beta} D^{\alpha} g_{00}(x)
$$

and hence that $f_{i} \rightarrow g_{00}$ in $\mathscr{S}_{n}$. Thus $\mathscr{S}_{n}$ is complete.
(b) If $f \in \mathscr{S}_{n}$, it is obvious that $D_{\alpha} f \in \mathscr{S}_{n}$, and the Leibniz formula implies that $P f$ and $g f$ are also in $\mathscr{S}_{n}$. The continuity of the three mappings is now an easy consequence of the closed graph theorem.

(c) If $f \in \mathscr{S}_{n}$, so is $P(D) f$, by $(b)$, and

$$
(P(D) f) * e_{t}=f * P(D) e_{t}=f * P(t) e_{t}=P(t)\left[f * e_{t}\right]
$$

Evaluation of these functions at the origin of $R^{n}$ gives the first part of (c), namely,

$$
(P(D) f)^{\wedge}(t)=P(t) \hat{f}(t)
$$

If $t=\left(t_{1} \ldots, t_{n}\right)$ and $t^{\prime}=\left(t_{1}+\varepsilon, t_{2}, \ldots, t_{n}\right), \varepsilon \neq 0$, then

$$
\frac{\hat{f}\left(t^{\prime}\right)-\hat{f}(t)}{i \varepsilon}=\int_{R^{n}} x_{1} f(x) \frac{e^{-i x_{1} \varepsilon}-1}{i x_{1} \varepsilon} e^{-i x \cdot t} d m_{n}(x) .
$$

The dominated convergence theorem can be applied, since $x_{1} f \in L^{1}$, and yields

$$
-\frac{1}{i} \frac{\partial}{\partial t_{1}} \hat{f}(t)=\int_{R^{n}} x_{1} f(x) e^{-i x \cdot t} d m_{n}(x)
$$

This is the case $P(x)=x_{1}$ of the second part of $(c)$; the general case follows by iteration.

(d) Suppose $f \in \mathscr{S}_{n}$ and $g(x)=(-1)^{|x|} x^{\alpha} f(x)$. Then $g \in \mathscr{S}_{n}$; now . (c) implies that $\hat{g}=D_{\alpha} \hat{f}$ and $P \cdot D_{\alpha} \hat{f}=P \cdot \hat{g}=(P(D) g)^{\wedge}$, which is a bounded function, since $P(D) g \in L^{1}\left(R^{n}\right)$. This proves that $\hat{f} \in \mathscr{S}_{n}$. If $f_{i} \rightarrow f$ in $\mathscr{S}_{n}$, then $f_{i} \rightarrow f$ in $L^{1}\left(R^{n}\right)$. Therefore $\hat{f}_{i}(t) \rightarrow \hat{f}(t)$ for all $t \in R^{n}$. That $f \rightarrow \hat{f}$ is a continuous mapping of $\mathscr{S}_{n}$ into $\mathscr{S}_{n}$ follows now from the closed graph theorem.

7.5 Theorem If $f \in L^{1}\left(R^{n}\right)$, then $\hat{f} \in C_{0}\left(R^{n}\right)$, and $\|\hat{f}\|_{\infty} \leq\|f\|_{1}$.

Here $C_{0}\left(R^{n}\right)$ is the supremum-normed Banach space of all complex continuous functions on $R^{n}$ that vanish at infinity.

PROOF Since $\left|e_{t}(x)\right|=1$, it is clear that

$$
|\hat{f}(t)| \leq\|f\|_{1} \quad\left(f \in L^{1}, t \in R^{n}\right) .
$$

Since $\mathscr{D}\left(R^{n}\right) \subset \mathscr{S}_{n}, \mathscr{S}_{n}$ is dense in $L^{1}\left(R^{n}\right)$. To each $f \in L^{1}\left(R^{n}\right)$ correspond functions $f_{i} \in \mathscr{S}_{n}$ such that $\left\|f-f_{i}\right\|_{1} \rightarrow 0$. Since $\hat{f}_{i} \in \mathscr{S}_{n} \subset C_{0}\left(R^{n}\right)$ and since (1) implies that $\hat{f}_{i} \rightarrow \hat{f}$ uniformly on $R^{n}$, the proof is complete.

The following lemma will be used in the proof of the inversion theorem. It depends on the particular normalization that was chosen for $m_{n}$.

7.6 Lemma If $\phi_{n}$ is defined on $R^{n}$ by

$$
\phi_{n}(x)=\exp \left\{-\frac{1}{2}|x|^{2}\right\}
$$

then $\phi_{n} \in \mathscr{S}_{n}, \hat{\phi}_{n}=\phi_{n}$, and

$$
\phi_{n}(0)=\int_{R^{n}} \hat{\phi}_{n} d m_{n}
$$

PROOF It is clear that $\phi_{n} \in \mathscr{S}_{n}$. Since $\phi_{1}$ satisfies the differential equation

$$
y^{\prime}+x y=0
$$

a short computation, or an appeal to (c) of Theorem 7.4, shows that $\hat{\phi}_{1}$ also satisfies (3). Hence $\hat{\phi}_{1} / \phi_{1}$ is a constant. Since $\phi_{1}(0)=1$ and

$$
\hat{\phi}_{1}(0)=\int_{R} \phi_{1} d m_{1}=(2 \pi)^{-1 / 2} \int_{-\infty}^{\infty} \exp \left\{-\frac{1}{2} x^{2}\right\} d x=1,
$$

we conclude that $\hat{\phi}_{1}=\phi_{1}$. Next,

$$
\phi_{n}(x)=\phi_{1}\left(x_{1}\right) \cdots \phi_{1}\left(x_{n}\right) \quad\left(x \in R^{n}\right)
$$

so that

$$
\hat{\phi}_{n}(t)=\hat{\phi}_{1}\left(t_{1}\right) \cdots \hat{\phi}_{1}\left(t_{n}\right) \quad\left(t \in R^{n}\right)
$$

It follows that $\hat{\phi}_{n}=\phi_{n}$ for all $n$. Since $\hat{\phi}_{n}(0)=\int \phi_{n} d m_{n}$, by definition, and since $\hat{\phi}_{n}=\phi_{n}$, we obtain (2).

\subsection{The inversion theorem}
(a) If $g \in \mathscr{S}_{n}$, then

$$
g(x)=\int_{R^{n}} \hat{g} e_{x} d m_{n} \quad\left(x \in R^{n}\right)
$$

(b) The Fourier transform is a continuous, linear, one-to-one mapping of $\mathscr{S}_{n}$ onto $\mathscr{S}_{n}$, of period 4 , whose inverse is also continuous.

(c) If $f \in L^{1}\left(R^{n}\right), \hat{f} \in L^{1}\left(R^{n}\right)$, and

$$
f_{0}(x)=\int_{R^{n}} \hat{f} e_{x} d m_{n} \quad\left(x \in R^{n}\right)
$$

then $f(x)=f_{0}(x)$ for almost every $x \in R^{n}$.

PROOF If $f$ and $g$ are in $L^{1}\left(R^{n}\right)$, Fubini's theorem can be applied to the double integral

to yield the identity

$$
\int_{R^{n}} \int_{R^{n}} f(x) g(y) e^{-i x \cdot y} d m_{n}(x) d m_{n}(y)
$$

$$
\int_{R^{n}} \hat{f} g d m_{n}=\int_{R^{n}} f \hat{g} d m_{n}
$$

To prove part (a), take $g \in \mathscr{S}_{n}, \phi \in \mathscr{S}_{n}, f(x)=\phi(x / \lambda)$, where $\lambda>0$. By $(d)$ of Theorem 7.2, (3) becomes

or

$$
\int_{R^{n}} g(t) \lambda^{n} \hat{\phi}(\lambda t) d m_{n}(t)=\int_{R^{n}} \phi\left(\frac{y}{\lambda}\right) \hat{g}(y) d m_{n}(y),
$$

$$
\int_{R^{n}}-g\left(\frac{t}{\lambda}\right) \hat{\phi}(t) d m_{n}(t)=\int_{R^{n}} \phi\left(\frac{y}{\lambda}\right) \hat{g}(y) d m_{n}(y) .
$$

As $\lambda \rightarrow \infty, g(t / \lambda) \rightarrow g(0)$ and $\phi(y / \lambda) \rightarrow \phi(0)$, boundedly, so that the dominated convergence theorem can be applied to the two integrals in (4). The result is

$$
g(0) \int_{R^{n}} \hat{\phi} d m_{n}=\phi(0) \int_{R^{n}} \hat{g} d m_{n} \quad\left(g, \phi \in \mathscr{S}_{n}\right) .
$$

If we specialize $\phi$ to be the function $\phi_{n}$ of Lemma 7.6, (5) gives the case $x=0$ of the inversion formula (1). The general case follows from this, since $(a)$ of
Theorem 7.2 yields

This completes part $(a)$.

$$
g(x)=\left(\tau_{-x} g\right)(0)=\int_{R^{n}}\left(\tau_{-x} g\right)^{\wedge} d m_{n}=\int_{R^{n}} \hat{g} e_{x} d m_{n}
$$

To prove part $(b)$, we introduce the temporary notation $\Phi g=\hat{g}$. The inversion formula (1) shows that $\Phi$ is one-to-one on $\mathscr{S}_{n}$, since $\hat{g}=0$ obviously implies $g=0$. It also shows that

$$
\Phi^{2} g=\check{g}
$$

where, we recall, $\check{g}(x)=g(-x)$, and hence that $\Phi^{4} g=g$. It follows that $\Phi$ maps $\mathscr{S}_{n}$ onto $\mathscr{S}_{n}$. The continuity of $\Phi$ has already been proved in Theorem 7.4. To prove the continuity of $\Phi^{-1}$, one can now either refer to the open mapping theorem or to the fact that $\Phi^{-1}=\Phi^{3}$.

To prove (c), we return to the identity (3), with $g \in \mathscr{S}_{n}$. Insert the inversion formula (1) into (3) and use Fubini's theorem, to obtain

$$
\int_{R^{n}} f_{0} \hat{g} d m_{n}=\int_{R^{n}} f \hat{g} d m_{n} \quad\left(g \in \mathscr{S}_{n}\right) .
$$

By $(b)$, the functions $\hat{g}$ cover all of $\mathscr{S}_{n}$. Since $\mathscr{D}\left(R^{n}\right) \subset \mathscr{S}_{n}$, (7) implies that

$$
\int_{R^{n}}\left(f_{0}-f\right) \phi d m_{n}=0
$$

for every $\phi \in \mathscr{D}\left(R^{n}\right)$, hence (by a uniform approximation described in Exercise 1 of Chapter 6 ) for every continuous $\phi$ with compact support. It follows that $f_{0}-f=0$ a.e.

7.8 Theorem If $f \in \mathscr{S}_{n}$ and $g \in \mathscr{S}_{n}$, then

(a) $f * g \in \mathscr{S}_{n}$, and

(b) $(f g)^{\wedge}=\hat{f} * \hat{g}$.

PROOF By $(c)$ of Theorem 7.2, $(f * g)^{\wedge}=\hat{f} \hat{g}$, or

$$
\Phi(f * g)=\Phi f \cdot \Phi g
$$

in the notation used in the proof of $(b)$ of Theorem 7.7. With $\hat{f}$ and $\hat{g}$ in place of $f$ and $g$, (1) becomes

$$
\Phi(\hat{f} * \hat{g})=\Phi^{2} f \cdot \Phi^{2} \tilde{g}=\check{f} \check{g}=(f g)^{2}=\Phi^{2}(f g)
$$

Now apply $\Phi^{-1}$ to both sides of (2) to obtain (b). Note that $f g \in \mathscr{S}_{n}$; hence (b) implies that $\hat{f} * \hat{g} \in \mathscr{S}_{n}$, and this gives (a), since the Fourier transform maps $\mathscr{S}_{n}$ onto $\mathscr{S}_{n}$.

IIII

7.9 The Plancherel theorem There is a linear isometry $\Psi$ of $L^{2}\left(R^{n}\right)$ onto $L^{2}\left(R^{n}\right)$ which is uniquely determined by the requirement that

$$
\Psi \hat{\Psi}=\hat{f} \quad \text { for every } \hat{f} \in \mathscr{S}_{n} \text {. }
$$

Observe that the equality $\Psi f=\hat{f}$ extends from $\mathscr{S}_{n}$ to $L^{1} \cap L^{2}$, since $\mathscr{S}_{n}$ is dense in $L^{2}$ as well as in $L^{1}$. This gives consistency: The domain of $\Psi$ is $L^{2}, \hat{f}$ was defined in Section 7.1 for all $f \in L^{1}$, and $\Psi f=\hat{f}$ whenever both definitions are applicable. Thus $\Psi$ extends the Fouricr transform from $L^{1} \cap L^{2}$ to $L^{2}$. This extension $\Psi$ is still called the Fourier transform (sometimes the Fourier-Plancherel transform), and the notation $\hat{f}$ will continue to be used in place of $\Psi f$, for any $f \in L^{2}\left(R^{n}\right)$.

PROOF If $f$ and $g$ are in $\mathscr{S}_{n}$, the inversion theorem yields

$$
\begin{aligned}
\int_{R^{n}} f \bar{g} d m_{n} & =\int_{R^{n}} \bar{g}(x) d m_{n}(x) \int_{R^{n}} \hat{f}(t) e^{i x \cdot t} d m_{n}(t) \\
& =\int_{R^{n}} \hat{f}(t) d m_{n}(t) \int_{R^{n}} \bar{g}(x) e^{i x \cdot t} d m_{n}(x) .
\end{aligned}
$$

The last inner integral is the complex conjugate of $\hat{g}(t)$. We thus get the Parseval formula

$$
\int_{R^{n}} f \bar{g} d m_{n}=\int_{R^{n}} \hat{f} \overline{\hat{g}} d m_{n} \quad\left(f, g \in \mathscr{S}_{n}\right)
$$

If $g=f$, (1) specializes to

$$
\|f\|_{2}=\|\hat{f}\|_{2} \quad\left(f \in \mathscr{S}_{n}\right)
$$

Note that $\mathscr{S}_{n}$ is dense in $L^{2}\left(R^{n}\right)$, for the same reason that $\mathscr{S}_{n}$ is dense in $L^{1}\left(R^{n}\right)$. Thus (2) shows that $f \rightarrow \hat{f}$ is an isometry (relative to the $L^{2}$-metric) of the dense subspace $\mathscr{S}_{n}$ of $L^{2}\left(R^{n}\right)$ onto $\mathscr{S}_{n}$. (The mapping is onto by the inversion theorem.) It follows, by elementary metric space arguments, that $f \rightarrow \hat{f}$ has a unique continuous extension $\Psi: L^{2}\left(R^{n}\right) \rightarrow L^{2}\left(R^{n}\right)$ and that this $\Psi$ is a linear isometry onto $L^{2}\left(R^{n}\right)$. Some details of this are given in Exercise 13

It should be noted that the Parseval formula (1) remains true for arbitrary $f$ and $g$ in $L^{2}\left(R^{n}\right)$.

That the Fourier transform is an $L^{2}$-isometry is one of the most important features of the whole subject.

\section{Tempered Distributions}
Before we define these, we establish the following relation between $\mathscr{S}_{n}$ and $\mathscr{D}\left(R^{n}\right)$.

\subsection{Theorem}
(a) $\mathscr{D}\left(R^{\dot{n}}\right)$ is dense in $\mathscr{S}_{n}$.

(b) The identity mapping of $\mathscr{D}\left(R^{n}\right)$ into $\mathscr{S}_{n}$ is continuous.

These statements refer, of course, to the usual topologies of $\mathscr{D}\left(R^{n}\right)$ and $\mathscr{S}_{n}$, as defined in Sections 6.3 and 7.3

PROOF (a) Choose $f \in \mathscr{S}_{n}, \psi \in \mathscr{D}\left(R^{n}\right)$ so that $\psi=1$ on the unit ball of $R^{n}$, and put

$$
f_{r}(x)=f(x) \psi(r x) \quad\left(x \in R^{n}, r>0\right)
$$

Then $f_{r} \in \mathscr{D}\left(R^{n}\right)$. If $P$ is a polynomial and $\alpha$ is a multi-index, then

$$
P(x) D^{\alpha}\left(f-f_{r}\right)(x)=P(x) \sum_{\beta \leq \alpha} c_{\alpha \beta}\left(D^{\alpha-\beta} f\right)(x) r^{|\beta|} D^{\beta}[1-\psi(r x)]
$$

Our choice of $\psi$ shows that $D^{\beta}[1-\psi(r x)]=0$ for every multi-index $\beta$ when $|x| \leq 1 / r$. Since $f \in \mathscr{S}_{n}$, we have $P \cdot D^{\alpha-\beta} f \in C_{0}\left(R^{n}\right)$ for all $\beta \leq \alpha$. It follows that the above sum tends to 0 , uniformly on $R^{n}$, when $r \rightarrow 0$. Thus $f_{r} \rightarrow f$ in $\mathscr{S}_{n}$, and $(a)$ is proved.

(b) If $K$ is a compact set in $R^{n}$, the topology induced on $\mathscr{D}_{K}$ by $\mathscr{S}_{n}$ is clearly the same as its usual one (as defined in Section 1.46), since each $\left(1+|x|^{2}\right)^{N}$ is bounded on $K$. The identity mapping of $\mathscr{D}_{K}$ into $\mathscr{S}_{n}$ is therefore continuous (actually, a homeomorphism), and now (b) follows from Theorem 6.6

7.11 Definition If $i: \mathscr{D}\left(R^{n}\right) \rightarrow \mathscr{S}_{n}$ is the identity mapping, if $L$ is a continuous linear functional on $\mathscr{S}_{n}$, and if

$$
u_{L}=L \circ i
$$

then the continuity of $i$ (Theorem 7.10 ) shows that $u_{L} \in \mathscr{D}^{\prime}\left(R^{n}\right)$; the denseness of $\mathscr{D}\left(R^{n}\right)$ in $\mathscr{S}_{n}$ shows that two distinct $L$ 's cannot give rise to the same $u$. Thus (1) describes a vector space isomorphism between the dual space $\mathscr{S}_{n}^{\prime}$ of $\mathscr{S}_{n}$, on the one hand, and a certain space of distributions on the other. The distributions that arise in this way are called tempered:

The tempered distributions are precisely those $u \in \mathscr{D}^{\prime}\left(R^{n}\right)$ that have continuous extensions to $\mathscr{S}_{n}$.

In view of the preceding remarks, it is customary and natural to identify $u_{L}$ with $L$. The tempered distributions on $R^{n}$ are then precisely the members of $\mathscr{S}_{n}^{\prime}$.

The following examples will explain the use of the word "tempered" in this connection; it indicates a growth restriction at infinity. (See also Exercise 3.)

7.12 Examples (a) Every distribution with compact support is tempered. Suppose $K$ is the compact support of some $u \in \mathscr{D}^{\prime}\left(R^{n}\right)$, fix $\psi \in \mathscr{D}\left(R^{n}\right)$ so that $\psi=1$ in some open set containing $K$, and define

$$
\tilde{u}(f)=u(\psi f) \quad\left(f \in \mathscr{S}_{n}\right)
$$

If $f_{i} \rightarrow 0$ in $\mathscr{S}_{n}$, then all $D^{\alpha} f_{i} \rightarrow 0$ uniformly on $R^{n}$, hence all $D^{\alpha}\left(\psi f_{i}\right) \rightarrow 0$ uniformly on $R^{n}$, so that $\psi f_{i} \rightarrow 0$ in $\mathscr{D}\left(R^{n}\right)$. It follows that $\tilde{u}$ is continuous on $\mathscr{S}_{n}$. Since $\tilde{u}(\phi)=u(\phi)$ for $\phi \in \mathscr{D}\left(R^{n}\right), \tilde{u}$ is an extension of $u$.

(b) Suppose $\mu$ is a positive Borel measure on $R^{n}$ such that

$$
\int_{R^{n}}\left(1+|x|^{2}\right)^{-k} d \mu(x)<\infty
$$

for some positive integer $k$. Then $\mu$ is a tempered distribution. The assertion is, more explicitly, that the formula

$$
\Lambda f=\int_{R^{n}} f d \mu
$$

defines a continuous linear functional on $\mathscr{S}_{n}$.

Tó see this, suppose $f_{i} \rightarrow 0$ in $\mathscr{S}_{n}$. Then

$$
\varepsilon_{i}=\sup _{x \in R^{n}}\left(1+|x|^{2}\right)^{k}\left|f_{i}(x)\right| \rightarrow 0
$$

Since $\left|\Lambda f_{i}\right|$ is at most $\varepsilon_{i}$ times the integral in (2), $\Lambda f_{i} \rightarrow 0$. This proves the continuity of $\Lambda$.
(c) Suppose $1 \leq p<\infty, N>0$, and $g$ is a measurable function on $R^{n}$ such that

$$
\int_{R^{n}}\left|\left(1+|x|^{2}\right)^{-N} g(x)\right|^{p} d m_{n}(x)=C<\infty .
$$

Then $g$ is a tempered distribution.

As in $(b)$, define

$$
\Lambda f=\int_{R^{n}} f g d m_{n}
$$

Assume first that $p>1$; let $q$ be the conjugate exponent. Then Hölder's incquality gives

$$
\begin{aligned}
|\Lambda f| & \leq C^{1 / p}\left\{\int_{R^{n}}\left|\left(1+|x|^{2}\right)^{N} f(x)\right|^{q} d m_{n}(x)\right\}^{1 / q} \\
& \leq C^{1 / p} B^{1 / q} \sup _{x \in R^{n}}\left|\left(1+|x|^{2}\right)^{M} f(x)\right|,-
\end{aligned}
$$

where $M$ is taken so large that

$$
\int_{R^{n}}\left(1+|x|^{2}\right)^{(N-M) a} d m_{n}(x)=B<\infty
$$

The inequality (7) proves that $\Lambda$ is continuous on $\mathscr{S}_{n}$. The case $p=1$ is even easier.

(d) It follows from $(c)$ that every $g \in L^{p}\left(R^{n}\right)(1 \leq \tilde{p} \leq \infty)$ is a tempered distribution. So is every polynomial and, more generally, every measurable function whose absolute value is majorized by some polynomial.

7.13 Theorem If $\alpha$ is a multi-index, $P$ is a polynomial, $g \in \mathscr{S}_{n}$, and $u$ is a tempered distribution, then the distributions $D^{x} u, P u$, and gu are also tempered.

PROOF This follows directly from $(b)$ of Theorem 7.4 and the definitions

$$
\begin{aligned}
\left(D^{x} u\right)(f) & =(-1)^{|x|} u\left(D^{x} f\right) \\
(P u)(f) & =u(P f) \\
(g u)(f) & =u(g f)
\end{aligned}
$$

7.14 Definition For $u \in \mathscr{S}_{n}^{\prime}$, definè

$$
\hat{u}(\phi)=u(\hat{\phi}) \quad\left(\phi \in \mathscr{S}_{n}\right)
$$

Since $\phi \rightarrow \hat{\phi}$ is a continuous linear mapping of $\mathscr{S}_{n}$ into $\mathscr{S}_{n}[(d)$ of Theorem 7.4], and since $u$ is continuous on $\mathscr{S}_{n}$, it follows that $\hat{u} \in \mathscr{S}_{n}^{\prime}$.

We have thus associated with each tempered distribution $u$ its Fourier transform $\hat{u}$, which is again a tempered distribution. Our next theorem will show that the formal
properties of Fourier transforms of rapidly decreasing functions are preserved in the larger setting of tempered distributions.

But first there arises a consistency question that ought to be settled. If $f \in L^{1}\left(R^{n}\right)$, then $f$ may also be regarded as a tempered distribution, say $u_{f}$, so that two definitions of the Fourier transform are available, namely, $(c)$ of Section 7.1 and Definition 7.14. The question is whether they agree, i.e., whether the distribution $\left(u_{f}\right)^{\wedge}$ corresponds to the function $\hat{f}$. The answer is affirmative, because

$$
\left(u_{f}\right)^{\wedge}(\phi)=u_{f}(\hat{\phi})=\int f \hat{\phi}=\int \hat{f} \phi=\left(u_{\hat{f}}\right)(\phi)
$$

for every $\phi \in \mathscr{S}_{n}$. The third of these equalities is the identity (3) of Section 7.7 ; the others are definitions.

Since $L^{2}\left(R^{n}\right) \subset \mathscr{S}_{n}^{\prime}$, the same question arises for the Fourier-Plancherel transform. The answer is again affirmative, by the same proof, since the identity $\int f \hat{\phi}=\int \hat{f} \phi$ persists for $f \in L^{2}\left(R^{n}\right)$ and $\phi \in \mathscr{S}_{n}$.

\subsection{Theorem}
(a) The Fourier transform is a continuous, linear, one-to-one mapping of $\mathscr{S}_{n}^{\prime}$ onto $\mathscr{S}_{n}^{\prime}$, of period 4 , whose inverse is also continuous.

(b) If $u \in \mathscr{S}_{n}^{\prime}$ and $P$ is a polynomial, then

$$
(P(D) u)^{\wedge}=P \hat{u} \quad \text { and } \quad(P u)^{\wedge}=P(-D) \hat{u} \text {. }
$$

Note that these are the analogues of $(b)$ of Theorem 7.7 and $(c)$ of Theorem 7.4. The topology to which $(a)$ refers is the weak*-topology that $\mathscr{S}_{n}$ induces on $\mathscr{S}_{n}^{\prime}$. Note also that the differential operators $P(D)$ and $P(-D)$ are defined in terms of $D_{\alpha}$, not $D^{\alpha}$; see $(d)$ of Section 7.1.

PROOF Let $W$ be a neighborhood of 0 in $\mathscr{S}_{n}^{\prime}$. Then there exist functions $\phi_{1}, \ldots, \phi_{k} \in \mathscr{S}_{n}$ such that

$$
\left\{u \in \mathscr{S}_{n}^{\prime}:\left|u\left(\phi_{i}\right)\right|<1 \quad \text { for } \quad 1 \leq i \leq k\right\} \subset W .
$$

Define

$$
V=\left\{u \in \mathscr{S}_{n}^{\prime}:\left|u\left(\hat{\phi}_{i}\right)\right|<1 \quad \text { for } \quad 1 \leq i \leq k\right\}
$$

Then $V$ is a neighborhood of 0 in $\mathscr{S}_{n}^{\prime}$, and since

$$
\hat{u}(\phi)=u(\hat{\phi}) \quad\left(\phi \in \mathscr{S}_{n}, u \in \mathscr{S}_{n}^{\prime}\right)
$$

we see that $\hat{u} \in W$ whenever $u \in V$. This proves the continuity of $\Phi$, where we write $\Phi u=\hat{u}$. Since $\Phi$ has period 4 on $\mathscr{S}_{n}$, (3) shows that $\Phi$ has period 4 on $\mathscr{S}_{n}^{\prime}$, that is, that $\Phi^{4} u=u$ for every $u \in \mathscr{S}_{i n}^{\prime}$. Hence $\Phi$ is one-to-one and onto, and since $\Phi^{-1}=\Phi^{3}, \Phi^{-1}$ is continuous.
by the computations

Statement $(b)$ follows from $(c)$ of Theorem 7.4 and from Theorem 7.13,

and

$$
\begin{aligned}
(P(D) u)^{\wedge}(\phi) & =(P(D) u)(\hat{\phi})=u(P(-D) \hat{\phi}) \\
& =u\left((P \phi)^{\wedge}\right)=\hat{u}(P \phi)=(P \hat{u})(\phi)
\end{aligned}
$$

$$
\begin{aligned}
(P(-D) \hat{u})(\phi) & =\hat{u}(P(D) \phi)=u\left((P(D) \phi)^{\wedge}\right) \\
& =u(P \hat{\phi})=(P u)(\hat{\phi})=(P u)^{\wedge}(\phi)
\end{aligned}
$$

where $\phi$ is an arbitrary function in $\mathscr{S}_{n}$.

7.16 Examples We saw in $(d)$ of Section 7.12 that polynomials are tempered distributions. Their Fourier transforms are easily computed. We begin with the polynomial 1; regarded as a distribution on $R^{n}, 1$ acts on test functions $\phi$ by the formula

Hence

$$
1(\phi)=\int_{R^{n}} 1 \phi d m_{n}=\int_{R^{n}} \phi d m_{n} .
$$

$$
\hat{\mathrm{i}}(\phi)=1(\hat{\phi})=\int_{R^{n}} \hat{\phi} d m_{n}=\phi(0)=\delta(\phi),
$$

where $\delta$ is the Dirac measure on $R^{n}$. Likewise,

$$
\hat{\delta}(\phi)=\delta(\hat{\phi})=\hat{\phi}(0)=\int_{R^{n}} \phi d m_{n}=1(\phi)
$$

Thus (2) and (3) give the results

$$
\hat{1}=\delta \quad \text { and } \quad \hat{\delta}=1
$$

If $P$ is now an arbitrary polynomial on $R^{n}$, and if we apply $(b)$ of Theorem 7.15 with $u=\delta$ and with $u=1$, the results in (4) show that

$$
(P(D) \delta)^{\wedge}=P \quad \text { and } \quad \hat{P}=P(-D) \delta
$$

The two formulas in (4) [as well as those in (5)] can also be derived from each other by the inversion theorem, which may be stated for tempered distributions in the following way:

$$
\text { If } u \in \mathscr{S}_{n}^{\prime} \text {, then }(\hat{u})^{\wedge}=\check{u} \text {, where } \check{u} \text { is defined by }
$$

$$
\check{u}(\phi)=u(\check{\phi}) \quad\left(\phi \in \mathscr{S}_{n}\right)
$$

The proof is trivial, since $(\hat{\phi})^{\wedge}=\check{\phi}$, by $(a)$ of Theorem 7.7:

$$
(\hat{u})^{\wedge}(\phi)=\hat{u}(\hat{\phi})=u\left((\hat{\phi})^{\wedge}\right)=u(\check{\phi})=\check{u}(\phi) .
$$

Note that $\check{\delta}=\delta$.

If we combine (5) with Theorem 6.25 , we find that a distribution is the Fourier transform of a polynomial if and only if its support is the origin (or the empty set).

The following lemma will be used in the proof of Theorem 7.19. Its analogue, with $\mathscr{D}\left(R^{n}\right)$ in place of $\mathscr{S}_{n}$, is much easier and was used without comment in the proof of Theorem 6.30.

7.17 Lemma If $w=(1,0, \ldots, 0) \in R^{n}$, if $\phi \in \mathscr{S}_{n}$, and if

$$
\phi_{\varepsilon}(x)=\frac{\phi(x+\varepsilon w)-\phi(x)}{\varepsilon} \quad\left(x \in R^{n}, \varepsilon>0\right)
$$

then $\dot{\psi}_{\varepsilon} \rightarrow \partial \phi / \partial x_{1}$ in the topology of $\mathscr{S}_{n}$, as $\varepsilon \rightarrow 0$.

PROOF The conclusion can be obtained by showing that the Fourier transform of $\phi_{\varepsilon}-\partial \phi / \partial x_{1}$ tends to 0 in $\mathscr{S}_{n}$, that is, by showing that

$$
\psi_{\varepsilon} \hat{\phi} \rightarrow 0 \text { in } \mathscr{S}_{n}, \quad \text { as } \varepsilon \rightarrow 0
$$

where

$$
\psi_{\varepsilon}(y)=\frac{\exp \left(i \varepsilon y_{1}\right)-1}{\varepsilon}-i y_{1} \quad\left(y \in R^{n}, \varepsilon>0\right)
$$

If $P$ is a polynomial and $\alpha$ is a multi-index, then

$$
P \cdot D^{\alpha}\left(\psi_{\varepsilon} \hat{\phi}\right)=\sum_{\beta \leq \alpha} c_{\alpha \beta} P \cdot\left(D^{\alpha-\beta} \hat{\phi}\right) \cdot\left(D^{\beta} \psi_{\varepsilon}\right)
$$

A simple computation shows that

$$
\left|D^{\beta} \psi_{\varepsilon}(y)\right| \leq \begin{cases}\varepsilon y_{1}^{2} & \text { if }|\beta|=0 \\ \varepsilon\left|y_{1}\right| & \text { if }|\beta|=1 \\ \varepsilon^{|\beta|-1} & \text { if }|\beta|>1\end{cases}
$$

The left side of (4) tends therefore to 0 , uniformly on $R^{n}$, as $\varepsilon \rightarrow 0$. The definition of the topology of $\mathscr{F}_{n}$ (Section 7.3) shows now that (2) holds.

7.18 Definition If $u \in \mathscr{S}_{n}^{\prime}$ and $\phi \in \mathscr{S}_{n}$, then

$$
(u * \phi)(x)=u\left(\tau_{x} \check{\phi}\right) \quad\left(\dot{x} \in R^{n}\right)
$$

Note that this is well defined, since $\tau_{x} \check{\phi} \in \mathscr{S}_{n}$ for every $x \in R^{n}$.

7.19 Theorem Suppose $\phi \in \mathscr{S}_{n}$ and $u$ is a tempered distribution. Then

(a) $u * \phi \in C^{\infty}\left(R^{n}\right)$, and

$$
D^{\alpha}(u * \phi)=\left(D^{\alpha} u\right) * \phi=u *\left(D^{\alpha} \phi\right)
$$

for every multi-index $\alpha$,
(b) $u * \phi$ has polynomial growth, hence is a tempered distribution,
(c) $(u * \phi)^{\wedge}=\hat{\phi} \hat{u}$,

(d) $(u * \phi) * \psi=u *(\phi * \psi)$, for every $\psi \in \mathscr{S}_{n}$,

(e) $\hat{u} * \hat{\phi}=(\phi u)^{\wedge}$.

PROOF. The second equality in $(a)$ is proved exactly as in Theorem 6.30, since convolution obviously still commutes with translations. This also shows that

$$
\left(\frac{\tau_{-\varepsilon w}-\tau_{0}}{\varepsilon}\right)(u * \phi)=u *\left(\frac{\tau_{-\varepsilon w}-\tau_{0}}{\varepsilon}\right) \phi
$$

Lemma 7.17 now gives $D^{\alpha}(u * \phi)=u *\left(D^{\alpha} \phi\right)$ if $\alpha=(1,0, \ldots, 0)$. Iteration of this special case gives $(a)$.

Let $p_{N}(f)$ denote the norm (1) of Section 7.3, for $f \in \mathscr{S}_{n}$. The inequality

shows that

$$
1+|x+y|^{2} \leq 2\left(1+|x|^{2}\right)\left(1+|y|^{2}\right) \quad \therefore \quad\left(x, y \in R^{n}\right)
$$

$$
p_{N}\left(\tau_{x} f\right) \leq 2^{N}\left(1+|x|^{2}\right)^{N} p_{N}(f) \quad\left(x \in R^{n}, f \in \mathscr{S}_{n}\right)
$$

Since $u$ is a continuous linear functional on $\mathscr{S}_{n}$ and since the norms $p_{N}$ determine the topology of $\mathscr{S}_{n}$, there is an $N$ and a $C<\infty$ such that

$$
|u(f)| \leq C p_{N}(f) \quad\left(f \in \mathscr{S}_{n}\right)
$$

see Chapter 1, Exercise 8. By (3) and (4),

$$
|(u * \phi)(x)|=\left|u\left(\tau_{x} \phi\right)\right| \leq 2^{N} C p_{N}(\phi)\left(1+|x|^{2}\right)^{N},
$$

which proves $(b)$. then

Thus $u * \phi$ has a Fourier transform, in $\mathscr{S}_{n}^{\prime}$. If $\psi \in \mathscr{D}\left(R^{n}\right)$, with support $K$,

so that

$$
\begin{aligned}
(u * \phi)^{\wedge}(\hat{\psi}) & =(u * \phi)(\check{\psi})=\int_{R^{n}}(u * \phi)(x) \psi(-x) d m_{n}(x) \\
& =\int_{-K} u\left[\psi(-x) \tau_{x} \check{\phi}\right] d m_{n}(x)=u \int_{-K} \psi(-x) \tau_{x} \check{\phi} d m_{n}(x) \\
& =u\left((\phi * \psi)^{\vee}\right)=\hat{u}\left((\phi * \psi)^{\wedge}\right)=\hat{u}(\hat{\phi} \hat{\psi})
\end{aligned}
$$

$$
(u * \phi)^{\wedge}(\hat{\psi})=(\hat{\phi} \hat{u})(\hat{\psi})
$$

In the preceding calculation, Theorem 3.27 was applied to an $\mathscr{S}_{n}$-valued integral, when $u$ was moved across the integral sign. So far, (6) has been proved for $\psi \in \mathscr{D}\left(R^{n}\right)$. Since $\mathscr{D}\left(R^{n}\right)$ is dense in $\mathscr{S}_{n}$, the Fourier transforms of members of $\mathscr{D}\left(R^{n}\right)$ are also dense in $\mathscr{S}_{n}$, by $(b)$ of Theorem 7.7. Hence (6) holds for every $\hat{\psi} \in \mathscr{S}_{n}$. The distributions $(u * \phi)^{\wedge}$ and $\hat{\phi} \hat{u}$ are therefore equal. This proves $(c)$.

In the computation that precedes (6), the two end terms are now seen to be equal for any $\psi \in \mathscr{S}_{n}$. Hence

$$
(u * \phi)(\breve{\psi})=u\left((\phi * \psi)^{\vee}\right)
$$

which is the same as

$$
((u * \phi) * \psi)(0)=(u *(\phi * \psi))(0)
$$

If we replace $\psi$ by $\tau_{x} \psi$ in (8), we obtain (d).

Finally, $(\hat{u} * \hat{\phi})^{\wedge}=\check{\phi} \check{u}=(\phi u)^{\vee}$, by $(c)$ above and (6) of Section 7.16; this gives $(e)$, since $(\phi u)^{\wedge}=\left((\phi u)^{\wedge}\right)^{\wedge}$.

\section{Paley-Wiener Theorems}
One of the classical theorems of Paley and Wiener characterizes the entire functions of exponential type (of one complex variable), whose restriction to the real axis is in $L^{2}$, as being exactly the Fourier transforms of $L^{2}$-functions with compact support; see, for instance, Theorem 19.3 of [23]. We shall give two analogues of this (in several variables), one for $C^{\infty}$-functions with compact support, and one for distributions with compact support.

7.20 Definitions If $\Omega$ is an open set in $\ell^{n}$, and if $f$ is a continuous complex function in $\Omega$, then $f$ is said to be holomorphic in $\Omega$ if it is holomorphic in each variable separately. This means that if $\left(a_{1}, \ldots, a_{n}\right) \in \Omega$ and if

$$
g_{i}(\lambda)=f\left(a_{1}, \ldots, a_{i-1}, a_{i}+\lambda, a_{i+1}, \ldots, a_{n}\right)
$$

each of the functions $g_{1}, \ldots, g_{n}$ is to be holomorphic in some neighborhood of 0 in C. A function that is holomorphic in all of $\mathscr{C}^{n}$ is said to be entire.

Points of $\ell^{n}$ will be denoted by $z=\left(z_{1}, \ldots, z_{n}\right)$, where $z_{k} \in \mathscr{C}$. If $z_{k}=$ $x_{k}+i y_{k}, x=\left(x_{1}, \ldots, x_{n}\right), y=\left(y_{1}, \ldots, y_{n}\right)$, then we write $z=x+i y$. The vectors

$$
x=\operatorname{Re} z \quad \text { and } \quad y=\operatorname{Im} z
$$

are the real and imaginary parts of $z$, respectively; $R^{n}$ will be thought of as the set of all $z \in \mathbb{C}^{n}$ with $\operatorname{Im} z=0$. The notations

$$
\begin{aligned}
|z| & =\left(\left|z_{1}\right|^{2}+\cdots+\left|z_{n}\right|^{2}\right)^{1 / 2} \\
|\operatorname{Im} z| & =\left(y_{1}^{2}+\cdots+y_{n}^{2}\right)^{1 / 2} \\
z^{\alpha} & =z_{1}^{\alpha_{1}} \cdots z_{n}^{\alpha_{n}} \\
z \cdot t & =z_{1} t_{1}+\cdots+z_{n} t_{n} \\
e_{z}(t) & =\exp (i z \cdot t)
\end{aligned}
$$

will be used, for any multi-index $\alpha$ and any $t \in R^{n}$.

7.21. Lemma Iff is an entire function in $\ell^{n}$ that vanishes on $R^{n}$, then $f=0$.

PROOF We consider the case $n=1$ as known. Let $P_{k}$ be the following property of $f:$ If $z \in \mathbb{C}^{n}$ has at least $k$ real coordinates, then $f(z)=0 . P_{n}$ is given; $P_{0}$ is to be proved. Assume $1 \leq i \leq n$ and $P_{i}$ is true. Take $a_{1}, \ldots, a_{i}$ real. The function $g_{i}$ considered in Section 7.20 is then 0 on the real axis, hence is 0 for all $\lambda \in \mathbb{C}$. It follows that $P_{i=1}$ is true.

In the following two theorems,

$$
r B=\left\{x \in R^{n}:|x| \leq r\right\}
$$

\subsection{Theorem}
(a) If $\phi \in \mathscr{D}\left(R^{n}\right)$ has its support in $r B$, and if

$$
f(z)=\int_{R^{n}} \phi(t) e^{-i z \cdot t} d m_{n}(t) \quad\left(z \in \mathbb{C}^{n}\right),
$$

then $f$ is entire, and there are constants $\gamma_{N}<\infty$ such that (2) $\vdots$

$$
|f(z)| \leq \gamma_{N}(1+|z|)^{-N} e^{r|\operatorname{lm} z|} \quad\left(z \in \mathbb{C}^{n}, N=0,1,2, \ldots\right)
$$

(b) Conversely, if an crtire function $f$ satisfies the conditions (2), then there exists $\phi \in \mathscr{D}\left(R^{n}\right)$, with support in $r B$, such that (1) holds.

PROOF (a) If $t \in r B$ then

$$
\left|e^{-i z \cdot t}\right|=e^{y \cdot t} \leq e^{|y||t|} \leq e^{r|\operatorname{lm} z|} .
$$

The integrand in (1) is therefore in $L^{1}\left(R^{n}\right)$, for every $z \in \mathbb{C}^{n}$, and $f$ is well defined on $\ell^{n}$. The continuity of $f$ is trivial, and an application of Morera's theorem, to each variable separately, shows that $f$ is entire. Integrations by part give

$$
z^{\alpha} f(z)=\int_{R^{n}}\left(D_{\alpha} \phi\right)(t) e^{-i z \cdot t} d m_{n}(t)
$$

Hence

$$
\left|z^{\alpha}\right||f(z)| \leq\left\|D_{\alpha} \phi\right\|_{1} e^{r|\operatorname{Im} z|}
$$

and (2) follows from the inequalities (3).

(b) Suppose $f$ is an entire function that satisfies (2), and define

$$
\phi(t)=\int_{R^{n}} f(x) e^{i t \cdot x} d m_{n}(x) \quad\left(t \in R^{n}\right)
$$

Note first that $(1+|x|)^{N} f(x)$ is in $L^{1}\left(R^{n}\right)$ for every $N$, by (2). Hence $\phi \in C^{\infty}\left(R^{n}\right)$, by the argument that proved $(c)$ of Theorem 7.4.

Next, we claim that the integral

$$
\int_{-\infty}^{\infty} f\left(\xi+i \eta, z_{2}, \ldots, z_{n}\right) \exp \left\{i\left[t_{1}(\xi+i \eta)+t_{2} z_{2}+\cdots+t_{n} z_{n}\right]\right\} d \xi
$$

is independent of $\eta$, for arbitrary real $t_{1}, \ldots, t_{n}$ and complex $z_{2}, \ldots, z_{n}$. To see this, let $\Gamma$ be a rectangular path in the $(\xi+i \eta)$-plane, with one edge on the real axis, one on the line $\eta=\eta_{1}$, whose vertical edges move off to infinity. By Cauchy's theorem, the integral of the integrand (5) over $\Gamma$ is 0 . By (2), the contributions of the vertical edges to this integral tend to 0 . It follows that (5) is the same for $\eta=0$ as for $\eta=\eta_{1}$. This establishes our claim.

The same can be done for the other coordinates. Hence we conciude from (4) that

$$
\phi(t)=\int_{R^{n}} f(x+i y) e^{i t \cdot(x+i y)} d m_{n}(x)
$$

for every $y \in R^{n}$.

Given $t \in R^{n}, t \neq 0$, choose $y=\lambda t /|t|$, where $\lambda>0$. Then $t \cdot y=\lambda|t|$, $|y|=\lambda$

$$
\left|f(x+i y) e^{i t \cdot(x+i y)}\right| \leq \gamma_{N}(1+|x|)^{-N} e^{(r-|t|) \lambda}
$$

and therefore

$$
|\phi(t)| \leq \gamma_{N} e^{(r-|t|) \lambda} \int_{R^{n}}(1+|x|)^{-N} d m_{n}(x)
$$

where $N$ is chosen so large that the last integral is finite. Now let $\lambda \rightarrow \infty$. If $|t|>r$, (7) shows that $\phi(t)=0$. Thus $\phi$ has its support in $r B$.

Now (1) follows, for real $z$, from (4) and the inversion theorem. Since both sides of (1) are entire functions, they coincide on $\mathscr{C}^{n}$, by Lemma 7.21. This completes the proof.

The following remarks will motivate the next theorem.

Let $u$ be a distribution in $R^{n}$, with compact support. Then $\hat{u}$ is defined, as a tempered distribution, by $\hat{u}(\phi)=u(\hat{\phi})$. However, the definition $\hat{f}(x)=\int f e_{-x} d m_{n}$, made for $f \in L^{1}\left(R^{n}\right)$, suggests that $\hat{u}$ ought to be a function, namely,

$$
\hat{u}(x)=u\left(e_{-x}\right) \quad\left(x \in R^{n}\right)
$$

because $e_{-x} \in C^{\infty}\left(R^{n}\right)$ and $u(\phi)$ makes sense for every $\phi \in C^{\infty}\left(R^{n}\right)$, as shown by $(d)$ of Theorem 6.24. Moreover, $e_{-z} \in C^{\infty}\left(R^{n}\right)$ for every $z \in \mathbb{C}^{n}$, and $u\left(e_{-z}\right)$ therefore looks like an entire function, whose restriction to $R^{n}$ is $\hat{u}$.

That all this is correct is part of the content of the next theorem, which also characterizes the resulting entire functions by certain growth conditions.

\subsection{Theorem}
(a) If $u \in \mathscr{D}^{\prime}\left(R^{n}\right)$ has its support in $r B$, if $u$ has order $N$, and if

$$
f(z)=u\left(e_{-z}\right) \quad\left(z \in \ell^{n}\right)
$$

then $f$ is entire, the restriction of $f$ to $R^{n}$ is the Fourier transform of $u$, and there is a constant $\gamma<\infty$ such thait

$$
|f(z)| \leq \gamma(1+|z|)^{N} e^{r|\operatorname{Im} z|} \quad\left(z \in \mathbb{C}^{n}\right)
$$

(b) Conversely, if $f$ is an entire function in $\ell^{n}$ which satisfies (2) for some $N$ and some $\gamma$, then there exists $u \in \mathscr{D}^{\prime}\left(R^{n}\right)$, with support in $r B$, such that (1) holds.

Note: The notation $\hat{u}$ will sometimes be used to denote the extension to $C^{n}$ given by (1). Thus

$$
\hat{u}(z)=u\left(e_{-z}\right)
$$

for $z \in \mathscr{C}^{n}$. This extension is sometimes called the Fourier-Laplace transform of $u$.

PROOF (a) Suppose $u \in \mathscr{D}^{\prime}\left(R^{n}\right)$ has its support in $r B$. Pick $\psi \in \mathscr{D}\left(R^{n}\right)$ so that $\psi=1$ on $(r+1) B$. Then $u=\psi u$, and $(e)$ of Theorem 7.19 shows that

$$
\hat{u}=(\psi u)^{\wedge}=\hat{u} * \hat{\psi} .
$$

Thus $\hat{u} \in C^{\infty}\left(R^{n}\right)$. Pick $\phi \in \mathscr{S}_{n}$ so that $\hat{\phi}=\psi$. Then

so that (3) gives

$$
\begin{aligned}
(\hat{u} * \hat{\psi})(x) & =(\hat{u} * \check{\phi})(x)=\hat{u}\left(\tau_{x} \phi\right)=u\left(\left(\tau_{x} \phi\right)^{\wedge}\right) \\
& =u\left(e_{-x} \hat{\phi}\right)=u\left(\psi e_{-x}\right)=u\left(e_{-x}\right)
\end{aligned}
$$

$$
\hat{u}(x)=u\left(e_{-x}\right) \quad\left(x \in R^{n}\right) .
$$

Our next aim is to show that the function $f$ defined by (1) is entire. Choose $a \in \mathscr{C}^{n}, b \in \mathbb{C}^{n}$, and put

$$
g(\lambda)=f(a+\lambda b)=u\left(e_{-a-\lambda b}\right) \quad(\lambda \in \mathscr{C}) .
$$

The continuity of $f$ poses no problem: If $w \rightarrow z$ in $\ell^{n}$, then $e_{-w} \rightarrow e_{-z}$ in $C^{\infty}\left(R^{n}\right)$, and $u$ is continuous on $C^{\infty}\left(R^{n}\right)$. To prove that $f$ is entire it is therefore enough to show that each of the functions $g$ defined by (5) is entire.

Let $\Gamma$ be a rectangular path in $C$. Since $\lambda \rightarrow e_{-a-\lambda b}$ is continuous, from $\varnothing$ to $C^{\infty}\left(R^{n}\right)$, the $C^{\infty}\left(R^{n}\right)$-valued integral

$$
F=\int_{\Gamma} e_{-a-\lambda b} d \lambda
$$

is well defined. Evaluation at any $t \in R^{n}$ is a continuous linear functional on $C^{\infty}\left(R^{n}\right)$. It therefore commutes with the integral sign. Hence

$$
F(t)=\int_{\Gamma} e_{-a-\lambda b}(t) d \lambda=\int_{\Gamma} e^{-i a \cdot t} e^{-i(b \cdot t) \lambda} d \lambda=0
$$

Thus $F=0$, and (6) gives

$$
0=u(F)=\int_{\Gamma} u\left(e_{-a-\lambda b}\right) d \lambda=\int_{\Gamma} g(\lambda) d \lambda
$$

By Morera's theorem, $g$ is entire.

The proof of part $(a)$ will be completed by proving (2). Choose an auxiliary function $h$ on the real line, infinitely differentiable, such that $h(s)=1$ when $s<1$ and $h(s)=0$ when $s>2$, and associate with each $z \in \ell^{n}(z \neq 0)$ the function

$$
\phi_{z}(t)=e^{-i z \cdot t} h(|t||z|-r|z|) \quad\left(t \in R^{n}\right)
$$

Then $\phi_{z} \in \mathscr{D}\left(R^{n}\right)$. Since the support of $u$ is in $r B$ and $h(|t||z|-r|z|)=1$ if $|t| \leq|z|^{-1}+r$, comparison of (1) and (7) shows that

$$
f(z)=u\left(\phi_{z}\right)
$$

Since $u$ has order $N$, there is a $\gamma_{0}<\infty$ such that $|u(\phi)| \leq \gamma_{0}\|\phi\|_{N}$ for all $\phi \in \mathscr{D}\left(R^{n}\right)$, where $\|\phi\|_{N}$ is as in (1) of Section 6.2; see (d) of Theorem 6.24. Hence (8) gives

$$
|f(z)| \leq \gamma_{0}\left\|\phi_{z}\right\|_{N}
$$

On the support of $\phi_{z},|t| \leq r+2 /|z|$, so that

$$
\left|e^{-i z \cdot t}\right|=e^{y \cdot t} \leq e^{2+r|\operatorname{Im} z|}
$$

If we now apply the Leibniz formula to the product (7) and use (10), (9) implies (2).

This completes the proof of part $(a)$.

(b) Since $f$ now satisfies (2), we have

$$
|f(x)| \leq \gamma(1+|x|)^{N} \quad\left(x \in R^{n}\right)
$$

The restriction of $f$ to $R^{n}$ is therefore in $\mathscr{S}_{n}^{\prime}$ and is the Fourier transform of some tempered distribution $u$.

Pick a function $h \in \mathscr{D}\left(R^{n}\right)$, with support in $B$, such that $\int h=1$, define $h_{\varepsilon}(t)=\varepsilon^{-n} h(t / \varepsilon)$, for $\varepsilon>0$, and put

$$
f_{\varepsilon}(z)=f(z) \hat{h}_{\varepsilon}(z) \quad\left(z \in C^{n}\right)
$$

where $\hat{h}_{\varepsilon}$ now denotes the entire function whose restriction to $R^{n}$ is the Fourier transform of $h_{\varepsilon}$. Statement (a) of Theorem 7.22, applied to $h_{\varepsilon}$, leads to the conclusion that $f_{\varepsilon}$ satisfies (2) of Theorem 7.22 with $r+\varepsilon$ in place of $r$. Therefore
(b) of Theorem 7.22 implies that $f_{\varepsilon}=\hat{\phi}_{\varepsilon}$ for some $\phi_{\varepsilon} \in \mathscr{D}\left(R^{n}\right)$ whose support
lies in $(r+\varepsilon) B$.

Consider some $\psi \in \mathscr{S}_{n}$ such that the support of $\hat{\psi}$ does not intersect $r B$. Then $\hat{\psi} \phi_{\varepsilon}=0$ for all sufficiently small $\varepsilon>0$. Since $f \psi \in L^{1}\left(R^{n}\right)$ and $\hat{h}_{s}(x)=$ $\hat{h}(\varepsilon x) \rightarrow 1$ boundedly on $R^{n}$, we conclude that

$$
\begin{aligned}
\therefore u(\hat{\psi}) & =\hat{u}(\psi)=\int f \psi d m_{n}=\lim _{\varepsilon \rightarrow 0} \int f_{\varepsilon} \psi d m_{n} \\
& =\lim _{\varepsilon \rightarrow 0} \int \hat{\phi}_{\varepsilon} \psi d m_{n}=\int \hat{\psi} \phi_{\varepsilon} d m_{n}=0 .
\end{aligned}
$$

Hence $u$ has its support in $r B$.

Now we see that $z \rightarrow u\left(e_{-z}\right)$ is an entire function, and since (1) holds for $z \in R^{n}$ (by the choice of $u$ ), Lemma 7.21 completes the proof of $(b)$.

\section{Sobolev's Lemma}
If $\Omega$ is a proper open subset of $R^{n}$, no Fourier transform has been defined for functions whose domain is $\Omega$ or for distributions in $\Omega$. Nevertheless, Fourier transform techniques cán sometimes be used to attack local problems. Theorem 7.25, known as Sobolev's lemma, is an example of this.

7.24 Definitions A complex measurable function $f$, defined in an open set $\Omega \subset R^{n}$, is said to be locally $L^{2}$ in $\Omega$ if $\int_{K}|f|^{2} d m_{n}<\infty$ for every compact $K \subset \Omega$.

Similarly, a distribution $u \in \mathscr{D}^{\prime}(\Omega)$ is locally $L^{2}$ if there is a function $g$, locally $L^{2}$ in $\Omega$, such that $u(\phi)=\int_{\Omega} g \phi d m_{n}$ for every $\phi \in \mathscr{D}(\Omega)$. To say that a function $f$ has a distribution derivative $D^{\alpha} f$ which is locally $L^{2}$ refers to the distribution $D^{\alpha} f$ and means, explicitly, that there is a function $g$, locally $L^{2}$, such that

$$
\int_{\Omega} g \phi d m_{n}=(-1)^{|\alpha|} \int_{\Omega} f D^{\alpha} \phi d m_{n}
$$

for every $\phi \in \mathscr{D}(\Omega)$. A priori, this says nothing about the existence of $D^{\alpha} f$ in the classical sense, in terms of limits of quotients.

On the other hand, the class $C^{(p)}(\Omega)$ consists, for each nonnegative integer $p$, of those complex functions $f$ in $\Omega$ whose derivatives $D^{\alpha} f$ exist in the classical sense, for each multi-index $\alpha$ with $|\alpha| \leq p$, and are continuous functions in $\Omega$.

We shall write $D_{i}^{k}$ for the differential operator $\left(\partial / \partial x_{i}\right)^{k}$.

7.25 Theorem Suppose $n, p$, $r$ are integers, $n>0, p \geq 0$, and

$$
r>p+\frac{n}{2}
$$

Suppose $f$ is a function in an open set $\Omega \subset R^{n}$ whose distribution derivatives $D_{i}^{k} f$ are locally $L^{2}$ in $\Omega$, for $1 \leq i \leq n, 0 \leq k \leq r$.

Then there is a function $f_{0} \in C^{(p)}(\Omega)$ such that $f_{0}(x)=f(x)$ for almost every $x \in \Omega$.

Note that the hypothesis involves no mixed derivatives, i.e., no terms like $D_{1} D_{2} f$. The conclusion is that $f$ can be "corrected" so as to be in $C^{(p)}(\Omega)$, by redefining it on a set of measure 0 .

Note also, as a corollary, that if all distribution derivatives of $f$ are locally $L^{2}$ in $\Omega$, then $f_{0} \in C^{\infty}(\Omega)$.

PROOF By hypothesis, there are functions $g_{i k}$, locally $L^{2}$ in $\Omega$, that satisfy

$$
\int_{\Omega} g_{i k} \phi d m_{n}=(-1)^{k} \int_{\Omega} f D_{i}^{k} \phi d m_{n} \quad[\phi \in \mathscr{D}(\Omega)]
$$

for $1 \leq i \leq n, 0 \leq k \leq r$.

Let $\omega$ be an open set whose closure $K$ is a compact subset of $\Omega$. Choose $\psi \in \mathscr{D}(\Omega)$ so that $\psi=1$ on $K$, and define $F$ on $R^{n}$ by

$$
F(x)= \begin{cases}\psi(x) f(x) & \text { if } x \in \Omega \\ 0 & \text { if } x \notin \Omega\end{cases}
$$

Then $F \in\left(L^{2} \cap L^{1}\right)\left(R^{n}\right)$.

In $\Omega$, the Leibniz formula gives

$$
D_{i}^{r} F=\sum_{s=0}^{r}\left(\begin{array}{l}
r \\
s
\end{array}\right)\left(D_{i}^{r-s} \psi\right)\left(D_{i}^{s} f\right)=\sum_{s=0}^{r}\left(\begin{array}{l}
r \\
s
\end{array}\right)\left(D_{i}^{r-s} \psi\right) g_{i s} .
$$

In the complement $\Omega_{0}$ of the support of $\psi, D_{i}^{r} F=0$. These two distributions coincide in $\Omega \cap \Omega_{0}$. Hence $D_{i}^{r} F$, originally defined as a distribution in $R^{n}$, is actually in $L^{2}\left(R^{n}\right)$, for $1 \leq i \leq n$, because the functions $\left(D_{i}^{r-s} \psi\right) g_{i s}$ are in $L^{2}(\Omega)$. [Having compact support, $D_{i}^{r} F$ is therefore also in $L^{\mathbf{1}}\left(\bar{R}^{i n}\right)$.]

The Plancherel theorem, applied to $F$ and to $D_{1}^{r} F, \ldots, D_{n}^{r} F$, shows now that

$$
\int_{R^{n}}|\hat{F}|^{2} d m_{n}<\infty
$$

and

$$
\int_{R^{n}} y_{i}^{2 r}|\hat{F}(y)|^{2} d m_{n}(y)<\infty \quad(1 \leq i \leq n)
$$

Since

$$
(1+|y|)^{2 r}<(2 n+2)^{r}\left(1+y_{1}^{2 r}+\cdots+y_{n}^{2 r}\right)
$$

where $|y|=\left(y_{1}^{2}+\cdots+y_{n}^{2}\right)^{1 / 2},(4)$ and (5) imply

$$
\int_{R^{n}}(1+|y|)^{2 r}|\hat{F}(y)|^{2} d m_{i n}(y)<\infty .
$$

If $J$ denotes the integral (7), and if $\sigma_{i n}$ is the $(n-1)$-dimensional volume of the unit sphere in $R^{n}$, the Schwarz inequality gives

$$
\begin{aligned}
\left\{\int_{R^{n}}(1+|y|)^{p}|\hat{F}(y)| d m_{n}(y)\right\}^{2} & \leq J \int_{R^{n}}(1+|y|)^{2 p-2 r} d m_{n}(y) \\
& =J \sigma_{n} \int_{0}^{\infty}(1+t)^{2 p-2 r} t^{n-1} d t<\infty
\end{aligned}
$$

since $2 p-2 r+n-1<-1$. We have thus proved that

$$
\int_{R^{n}}(1+|y|)^{p}|\hat{F}(y)| d m_{n}(y)<\infty .
$$

Define

$$
F_{\omega}(x)=\int_{R^{n}} \hat{F}(y) e^{i x \cdot y} d m_{n}(y) \quad\left(x \in R^{n}\right) .
$$

By $(c)$ of the inversion theorem 7.7, $F_{\omega}=F$ a.e. on $R^{n}$. Moreover, (8) implies that $y^{\alpha} \widehat{F}(y)$ is in $L^{1}$ whenever $|\alpha| \leq p$. Iteration of the proof of $(c)$ of Theorem 7.4 leads therefore to the conclusion

$$
F_{\omega} \in C^{(p)}\left(\bar{R}^{n}\right)
$$

Our given function $f$ coincides with $F$ in $\omega$. Hence $f=F_{\omega}$ a.e. in $\omega$.

If $\omega^{\prime}$ is another set like $\omega$, the preceding proof gives a function $F_{\omega^{\prime}} \in$ $C^{(p)}\left(R^{n}\right)$, which coincides with $f$ a.c. in $\omega^{\prime}$. Hence $F_{\omega^{\prime}}=F_{\omega}$ in $\omega^{\prime} \cap \omega$. The desired function $f_{0}$ can therefore be defined in $\Omega$ by setting $f_{0}(x)=F_{\omega}(x)$ if $x \in \omega$.


\end{document}