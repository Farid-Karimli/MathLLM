7 FOURIER TRANSFORMS Basic Properties 7.1 Notations (a) Thc normalized Lcbesgue measure on $R^{n}$ is the measure $\gamma i_{n}$ defined by $$ d m_{n}(x)=(2i)^{-n/2}\ d x. $$ The factor (2元) $-n/2$ simplifies the appearance of the inversion theorem ${\mathcal{V}}.{\overline{{\cal{I}}}}$ and the Plancherel theorem ${\mathcal{V}}.{\mathfrak{g}}$ . The usual Lebesgue spaces $L^{p},$ or $L^{p}(R^{n})$ ), illbe normed by means of $p i_{n}$ $$ \left||f||_{p}=\left\{\int_{R^{n}}|f|^{p}\;d m_{n}\right\}^{1/p}\qquad (1\leq p<\infty\right). $$ $\mathbf{\hat{t}}$ is also convenient to redefine the convolution of two functions on $\textstyle R^{n}$ by $$ (f*g)(x)=\int_{\cal R^{n}}f(x-y)g(y)\,d m_{n}(y)\,\qquad\qquad. $$ $\mathbf{\nabla}(b)$ whenever the integral exists. $\scriptstyle t\in R^{n},$ the character ${\mathcal{C}}_{t}$ is the function defined by For each $$ e_{t}(x)=e^{i t\cdot x}=\exp\left\{i(t_{1}x_{1}+\cdot\cdot\cdot\cdot+t_{n}x_{n})\right\}\qquad(x\in R^{n}). $$roururR TRAsroxws 167 Each ${\mathcal{C}}_{t}.$ .sats th fnctonal cquato $$ e_{t}(x+y)=e_{t}(x)e_{t}(y). $$ Thuw ${\mathcal{E}}_{t}$ is a homorphsm of th dive grou ${\boldsymbol{\mathsf{I}}}\,.$ into te mulilicative group of thecoplex nubers o aboute va $R^{n}$ is the function defined by (e) The Fourier ransform o afunction e $L^{1}(R^{n})$ $$ .\quad\hat{f}(t)\doteq\int_{\cal R}\!\!\!\!f e_{-t}\,d m_{n}\qquad(t\in{\cal R}^{n}). $$ Note th Toecm…ourinan ${\mathfrak{Y}}\setminus\mathfrak{Y}$ A oneseonsmuymuae ${\hat{f}}.$ $$ {\hat{f}}(t)=(f*e_{t})(0). $$ (dO If is a multi-index then $$ D_{\alpha}=(i)^{-1\alpha1}D^{\alpha}=\left(\frac{1}{i}\frac{\partial}{\partial x_{1}}\right)^{\alpha_{1}}\cdot\cdot\cdot\left(\frac{1}{i}\frac{\partial}{\partial x_{n}}\right)^{\alpha_{n}}. $$ The use o $\textstyle{D_{x}}$ in place of $D^{x}$ simpliesm o th ormalism Noe ta $$ D_{\alpha}\,e_{t}=t^{x}e_{t} $$ where, as beforc $t^{x}=t_{1}^{x_{1}}\cdot\cdot\cdot t_{n}^{x_{n}}.$ If ${\boldsymbol{P}}$ is a polynomial of $r_{\mathit{l}}$ variables, with complex coeficints sa $$ P(\xi)=\sum c_{\alpha}\xi^{\alpha}=\sum c_{\alpha}\xi_{1}^{\alpha_{1}}\cdot\cdot\cdot\xi_{n}^{\alpha_{n}}, $$ the diffrential operator $P(D){\mathrm{~and~}}P(-D)$ are defned by lt folow tha $$ P(D)=\sum c_{x}\,D_{x},\qquad P(-\,D)=\sum(-1)^{\left|x\right|}c_{x}\,D_{x}. $$ $$ P(D)e_{t}=P(t)e_{t}\qquad(t\in R^{n}). $$ (e) Th translation operators $\tau_{x}$ are defined, as before, by $$ (\tau_{x}f)(y)=f(y{\cdot}\!\,\!\cdot\!\lnot x)\qquad(x,\,y\in R^{n}). $$ 7.2 Theorem Suppose $f\circ g\in L^{1}(R^{n}),\,x\in R^{n}$ Then (a) ( $\tau_{x}f)^{\times}=e_{-x}\,\hat{f};$ (6の)e, = +,F $(c)\ \ \ (f*g)^{\times}=f\dot{f}$ 10. $(d)\ \ U\lambda>0$ and $h(x)=f(x/\lambda),\;t h e n\;\hat{h}(t)=\lambda^{n}\hat{f}(\lambda t).$168 DISrRuwrioNsANp rouER TRANsronws PROOF Ii follows from the definitions that $$ (\tau_{x}f)^{\star}(t)=\int(\tau_{x}f)\cdot e_{-t}=\int f\cdot\tau_{-x}e_{-t}=\int f\cdot e_{-t}(x)e_{-t}=e_{-x}(t){\hat{f}}(t) $$ and $$ (e_{x}f)^{\star}(t)=\int e_{x}f e_{-t}=\int\!f e_{-(t-x)}=(\tau_{x}f)(t). $$ An application of Fubini's theorern gives GCc)(d is obtained by a linear change of variables in the definition of ${\hat{f}}.$ // ${\mathcal{V}}_{\cdot}{\mathcal{S}}$ Rapidly dccrcasing functions This name is sometimes given to those $f\in C^{\infty}(R^{n})$ for which (1) $$ \operatorname*{sup}_{\left|x\right|\le N\textrm{}\,x\in R^{n}}(1+\left|\,x\,\right|^{2})^{N} |(D_{\alpha}\,f)(x) |\,<\infty $$ index ${\mathcal{Q}}.$ $N=0,\;1,2,\ldots.$ (Recall that $|x|^{2}=\sum x_{i}^{2},)$ In other words, the requirement is for that $P\cdot D_{\alpha}f{\mathrm{i}}$ s a bounded function on $R^{n}$ ", for every polynomial $\boldsymbol{P}$ and for every multi . Since this is true with $(1\,+\,\vert\,x\,\vert^{2})^{N}P(x)$ in place of $P(x),$ it follows that every $P\cdot D_{\alpha}J$ lies in $L^{1}(R^{n})$ ${\mathcal{P}}_{n}\,,$ in which the countable Thesefunctions form a vector space, denoted by collection of norms (I) defines a locally cnvex topology, as described in Theorem 1.37. It is clear that $\mathcal{D}(R^{n})\subset\mathcal{P}_{n}.$ 7.4 Theorem (a) 9,is a Frichet space (6) If P is a polynomial, $g\in{\mathcal{F}}_{n},$ andαis au muli-index, then each of ihe uhree mappings $$ f arrow P f,\qquad f arrow g f,\qquad f arrow D_{\alpha}f $$ is a continuous linear mapping of SP,into ${\mathcal{G}}_{n}$ (c) If fe ${\mathcal{P}}_{n}$ and I ${\boldsymbol{P}}$ is a polynomial, then $$ (P(D)f)^{\times}=P f\qquad a n d\qquad(P f)^{\sim}=P(-D){\tilde{f}}. $$ (d)The Fourier transform is a conius linear mapping o ${\mathcal{S}}_{n}$ in1o ${\mathcal{F}}_{n}$ TPart (d) will be strengthened in Theorem 7.7.1 indices $\textstyle{\mathcal{A}}$ rRoor(a) Suppose $\beta$ B the functions $x^{\beta}D^{x}f_{i}(x)$ converge then(uniformly on $R^{n})$ to a $(f_{i})$ is a Cauchy sequence in ${\mathcal{G}}_{n}.$ For every pair of multi- and bounded function $\scriptstyle g_{\alpha\beta}\,,$ as $i arrow\infty.$ It Tollows that $g_{\alpha\beta}(x)=x^{\beta}D^{x}g_{00}(x)$ and hence that f→9oo i ${\mathcal{I}}_{n}.$ Thus ${\mathcal{P}}_{n}$ is completeroutR TRANSroaws 169 (b) $|{\mathrm{fr}}f\in{\mathcal{P}}_{n},{\mathrm{it}}$ is obvious that $D_{\alpha}f\in{\mathcal{F}}_{n},$ and the Leibniz formula implies that Pf and ${\mathcal{G}}{\mathcal{f}}$ are also in ${\mathcal{P}}_{n}$ The continuity of the three mappings is now an easy consequence of the closed graph theorem. (c) $|\Gamma f\in{\mathcal{F}}_{n}\,,$ so is $P(D)f,$ by (b), and $$ \qquad(P(D)/)*e_{t}=f*P(D)e_{t}=f*P(t)e_{t}=P(t)[f*e_{t}]. $$ Evaluation of these functions at the origin of $R^{n}$ gives the first part of (c), namcly, $$ \mathbf{\nabla}(P(D)f)^{\times}(t)=P(t){\hat{f}}(t). $$ ${\textrm{I f}}t=(t_{1+}\cdot\cdot\cdot,\;t_{n})$ and $t^{\prime}=(t_{1}+e,\ t_{2},\dots,t_{n}),\varepsilon\neq0$ ), then $$ \frac{\hat{f}(t^{\prime})-\hat{f}(t)}{i\varepsilon\ \cdot\ }.=\int_{\scriptscriptstyle{R^{n}}}x_{1}f(x)\frac{e^{-i x_{1}\varepsilon}-1}{i x_{1}\varepsilon}\,e^{-i x\cdot t}\,d m_{n}(x). $$ The dominated convergence theorem can be applied, since $x_{1}f\in L^{1}$ ,and yields $$ -\,\frac{1}{i}\frac{\partial}{\partial t_{1}}{\hat{f}}(t)=\int_{R^{n}}x_{1}f(x)e^{-i x\cdot t}\ d m_{n}(x). $$ This is the case $P(x)=x_{1}$ of the second part of (c); the general case follows by iteration (d) Suppose $f\in{\mathcal{F}}_{n}$ and $g(x)=(-1)^{|x|}x^{x}f(x)$ Then $g\in{\mathcal{F}}_{n}\,;$ now(c) implies that ${\hat{g}}=D_{x}f$ and $P\cdot D_{\alpha}f=P\cdot\hat{g}=(P(D)g)^{\sim},$ which is a bounded function, since $P(D)g\in L^{1}(R^{n})$ This proves that $f\in{\mathcal{F}}_{n}.$ $\mathbb{F}f_{i}\to f$ in ${\mathcal{G}}_{n},$ then $f_{i}\to f$ in $L^{1}(R^{n}).$ Therefore ${\hat{f}}_{i}(t)\to{\hat{f}}(t)$ for all $\ :R^{\prime},$ That $f\to f$ is a continuous $J/J$ mapping of ${\mathcal{P}}_{n}$ into ${\mathcal{P}}_{n}$ follows now from the closed graph thcorem. 7.5 Thcorcm If fe L'(R"), then fe Co(R"), and $\|{\hat{f}}\|_{\infty}\leq\|f\|_{1}$ Here $C_{0}(R^{n})$ is the supremum-normed Banach space of all complex continuous functions on $R^{n}$ that vanish at infinity. PROOF Since le,(x) $$ \begin{array}{l}{{=\;1,\;1\mathrm{t}\;\mathrm{1S}\;\mathrm{clextr}\;\mathrm{that}}}\\ {{\;|\hat{f}(t)\mid\leq\;||f||_{1}}}\end{array}(f\in{\cal L}^{1},\,t\in{\cal R}^{n}). $$ (1) Since ${\mathcal{D}}(R^{n})\subset{\mathcal{P}}_{n}$ ${\mathcal{P}}_{n}$ is dense in $L^{1}(R^{n})$ To each $f\in L^{1}(R^{n})$ correspond func- tions $f_{i}\in{\mathcal{T}}_{n}$ such that $\|f-f_{i}\|_{1}\to0$ Since ${\hat{f_{i}}}\in{\mathcal{F}}_{n}\subset C_{0}(R^{n})$ and since(1) implies that ${\hat{f}}_{i}\to{\hat{f}}$ uniformly on $R^{n},$ the proof is complctc. ${a\!\!\!/}{b\!\!\!/}{b\!\!\!/}$ The following lemma will be used in the proof of the inversion theorem. $\mathbf{{\hat{t}}t}$ depends on the particular normalization that was chosen for $p n_{n}.$170 Dusruvrros AND routR TxAsronus 7.6 Lemma $I f\emptyset_{n}$ is defined on ${\boldsymbol{R}}^{n}$ by (1) $$ \phi_{n}(x)=\exp{\{-{\frac{1}{2}}|x|^{2}\}} $$ then $\phi_{n}\subset\mathcal{P}_{n},\;\hat{\phi}_{n}=\phi_{n},$ and (2) $$ \phi_{n}(0)=\int_{{\cal R}^{n}}\hat{\phi}_{n}\,d m_{n}\,. $$ PROOF It is clear that $\phi_{n}\in{\mathcal{F}}_{n}.$ Since $\phi_{1}$ satisfes the diferential equation (3) $$ y^{\prime}+x y=0, $$ satisfies (3). Hence a short computation, or an appeal to ${\hat{\phi}}_{1}/\phi_{1}$ is a constant. Since $\phi_{1}(0)=1$ and ${\hat{\phi}}_{1}$ also $\left(c\right)$ of Theorem 7.4, shows that $$ \hat{\phi}_{1}(0)=\int_{\cal R}\phi_{1}\;d m_{1}=(2\pi)^{-1/2}\;\int_{-\infty}^{\infty}\exp\left\{-\textstyle{\frac{1}{2}}x^{2}\right\}d x=1, $$ we conclude that ${\dot{\phi}}_{1}=\phi_{1}$ Next, (4) $$ \phi_{n}(x)=\phi_{1}(x_{1})\cdot\cdot\cdot\phi_{1}(x_{n})\qquad(x\in R^{n})\qquad. $$ so that (5) $$ \hat{\phi}_{n}(t)=\hat{\phi}_{1}(t_{1})\cdot\cdot\cdot\hat{\phi}_{1}(t_{n})\qquad(t\in R^{n}). $$ since It follows that ${\dot{\phi}}_{n}=\phi_{n}$ for al ${\boldsymbol{\eta}}.$ Since $\phi_{n}(0)=\textstyle\int\phi_{n}\,d m_{n}.$ by definition, and ${j}/f$ ${\dot{\phi}}_{n}=\phi_{n},$ we obtain (2) 7.7 The inversion theorem (a) $I f g\in{\mathcal{G}}_{n},$ then (D $$ g(x)=\int_{x^{n}}\rlap{d}\!\epsilon_{x}\,d m_{n}\qquad(x\in R^{n}). $$ ${\mathcal{I}}_{n},$ of period 4, whose inverse $\dot{\boldsymbol{k}}$ (6b)The Fourier transform is a continuous, linear, one-to-one mapping of ${\mathcal{S}}_{n}$ onto also continvous. (c)If fe $:I^{1}(R^{n}),f\in I^{1}(R^{n}).$ and (2) $$ f_{0}(x)=\int_{R^{n}}{\hat{f}}e_{x}\,d m_{n}\qquad(x\in R^{n}), $$ then $f(x)=\int_{0}(x)\,J_{O},$ r almost every $x\in K^{n}$ PROOF lf f and $\scriptstyle{\mathcal{G}}$ are in $L^{1}(R^{n}),$ Fubini's theorem can beappled to the double integral to yield the identity $$ \int_{R^{n}}\int_{R^{n}}f(x)g(y)e^{-i x\cdot y}\,d m_{n}(x)\,d m_{n}(y) $$ $(\,3)$ $\int_{R^{n}}\widehat{f}g\ d m_{n}=\int_{R^{n}}f\widehat{g}\ d m_{n}\,,$rounrun rsrows 171 To provepart(o) tak $g\in{\mathcal{F}}_{n},\;\;\phi\in{\mathcal{F}}_{n},\;\;\emptyset$ /(x) = 4(r/2)。where .>0 By (d) of Theorem ${\mathcal{V}}.{\mathcal{Z}},$ (3) becomes or $$ \int_{R^{n}}\!g(t)\lambda^{n}\hat{\phi}(\lambda t)\,d m_{n}(t)=\int_{R^{n}}\!\phi\!\left(\frac{y}{\lambda}\right)\hat{g}(y)\,d m_{n}(y), $$ (4) $$ \begin{array}{c}{{\int_{R^{n}}\hat{g}\left(\frac{t}{\hat{\lambda}}\right)\hat{\phi}(t)\,d m_{n}(t)=\int_{R^{n}}\phi\left(\frac{y}{\lambda}\right)\hat{g}(y)\,d m_{n}(y).}}\end{array} $$ As $\lambda arrow\infty,\ g(t/\lambda) arrow g(0)$ and $\phi(y/\lambda) arrow\phi(0),$ boundedly, so that the dominated comves eneai s a awmeiasa s eun (5) $$ g(0)\int_{R^{n}}\hat{\phi}\;d m_{n}=\phi(0)\int_{R^{n}}\hat{\diamondsuit}\;d m_{n}\qquad(g,\;\phi\in\mathcal{G}_{n}). $$ $\mathbf{I}f$ we spcializ $\phi$ to be the fnction $\phi_{n}$ Amof Lemma $7.6,(5)$ gives the case $x=0$ fr uoer ${\mathcal{I}}_{\cdot}2$ on oua D Tn enase m s se Theorem yields implies inverion frmul Showstha $\bar{\Phi}$ $$ g(x)=(\tau_{-x}g)(0)=\int_{R^{n}}(\tau_{-x}g)^{\times}\,d m_{n}=\int_{R^{n}}\!\!\!\hat{g}e_{x}\,d m_{n}. $$ ${\mathcal{I}}_{n},$ since $\scriptstyle{\hat{g}}=0$ obviously This completes part (o lt also shows tha p is one-to-one on $\Phi_{\mathcal{G}}={\hat{g}}$ The To popPon. toe teoray otati ${\mathfrak{g}}=0.$ (6) $$ \Phi^{2}g={\dot{\theta}} $$ maps where, we recal onto ${\mathcal{P}}_{n}\,.$ The coninuityo and hence that $\Phi^{4}g=g$ t follows that $\Phi$ ${\bar{g}}(x)=g(-x),$ ${\mathcal{P}}_{n}$ $\Phi$ has aleadybee nroved inTheore 7.4. To prove the continuiyo $\Phi^{-1}.$ one can now either refer to the ope mapping theorem or to the fact tha $\Phi^{-1}=\Phi^{3}.$ $\mathrm{wlth}\ g\in{\mathcal{S}}_{n}$ Insert thinversion To prove e). eu t tedenty 3 formu D ino a s Puisisemodoobva (7) $$ \int_{R^{n}}f_{0}\hat{g}\;d m_{n}=\int_{R^{n}}f\hat{g}\;d m_{n}\quad\quad(g\in{\mathcal{F}}_{n}). $$ By Gb), the functions $\hat{\mathcal{G}}$ cover all of $\scriptstyle{\mathcal{G}}_{\circ}$ . Since ${\mathcal{P}}(R^{n})\subset{\mathcal{P}}_{n}\,,$ (T) implies that (8) $$ \int_{R^{n}}(f_{0} arrow f)\phi\;d m_{n}=0 $$ $f_{0}-f=0$ a.e. of Chapter O or every continious $\phi~~~~~~~~~~~~~~~~~~~~~~~~~~~~~~~~~~~~~~~~~~~~~~~~~~~~~~~~~~~~~~~~~~~~~~~~~~~~~~~~~~~~~~~~~~~~~~~~~~~~~~~~~~~~~~~~~~~~~~~~~~~~~~~~~~~~~~~~~~~~~~~~~~~~~~~~~~~~~~~~~~~~~~~~~~~~~~~~~~~~~~~~~~~~~~~~~~~~~~~~~~~$ sres Geo soo os onsmominanei eio // vih copat uport fow h172 DISTRUBUrIoNS AND FoURIER TRANSroRMs 7.8 Theorem If fe ${\mathcal{G}}_{n}$ and $g\in{\mathcal{F}}_{n},$ then (a) $f*g\in{\mathcal{F}}_{n}$ . and (6) $(f g)^{\times}={\dot{f}}^{\ast}\wedge\vartheta$ PROOF By $\scriptstyle\langle c\rangle$ of Theorem $7.2,(f*g)^{\times}=f\hat{\theta},$ or (1) $$ \Phi(f*g)=\Phi f\cdot\Phi g, $$ in the notation ued in the proof of O) of Theorem 7.7. With $\hat{f}$ and $\hat{\mathcal{G}}$ G in place of f and g,(1) becomes (2) $$ \Phi(\hat{f}*\hat{g})=\Phi^{2}f\cdot\Phi^{2}g=\tilde{f}\hat{g}=(f g)^{\vee}=\Phi^{2}(f g). $$ ${\mathcal{P}}_{n}$ Now apply $\Phi^{-1}$ to both sides of (2) to obtain $(\partial).$ Note that $f g\in{\mathcal{F}}_{n};$ hence (b) implies $\mathrm{that}\,\hat{J}\ast\hat{g}\in{\mathcal I}_{n}\,,$ and this gives a) since the Fourier transform maps $J/j$ onto ${\mathcal{G}},$ 7.9 The Plancherel theorem There is a linear isometry $\Psi$ of ${\cal L}^{2}(R^{n})$ onto LZ(R") which is uniquely determined by the requirement that $$ \Psi f=\hat{f}\qquad f o r\;e v e r y\,f\in{\mathcal I}_{n}. $$ $\mathbf{\hat{w}}$ Obscrvc that the equality $\Psi f={\hat{f}}$ extends from ${\mathcal{P}}_{n}$ to $L^{1}\cap L^{2},$ since ${\mathcal{P}}_{n}$ is dense in Section 7.1 for ${\mathrm{all}}\,f\in L^{*}$ , and This gives consistency: The domain of to $L^{2}.$ . This extension was defined in ${\cal L}^{2}$ as well as in ${\boldsymbol{L}}^{1}$ extends the Fourier transform from $L^{1}\cap L^{2}$ $\mathbf{\hat{P}}$ is $L^{2},{\hat{f}}$ is still called ${\mathcal{V}}f={\hat{f}}$ whenever both definitions arc appicable. Thus $\Psi$ the Fourer tansform (sometimes the Fourier-Plancerel transform, and the notation $\bar{f}$ will continue to be used in place of Y,for any $f\in L^{*}(R^{n}).$ PROOF If f and g are in ${\mathcal{I}}_{n}$ , the inversion theorem yields $$ \begin{array}{c}{{\int_{R^{n}}f\bar{g}\ d m_{n}=\int_{R^{n}}\bar{g}(x)\ d m_{n}(x)\int_{R^{n}}\bar{f}(t)e^{i x\cdot t}\,d m_{n}(t)}}\\ {{}}\\ {{}}\end{array} $$ The lastiner iteral is the complex conjugate o ${\tilde{g}}(t)$ We thus get the Parseval formula (1) $$ \dagger_{\scriptscriptstyle R^{n}}\!f\bar{g}\ d m_{n}=\int_{\scriptscriptstyle R^{n}}\!\!\!\!\!\!\!f\bar{\hat{g}}\ d m_{n}\;\;\;\;\;\;\;\;\;(f,\,g\in{\mathcal P}_{n}). $$ $\operatorname{If}g=f,(1)$ specializcs t $\left(2\right)$ $\|f\|_{2}=\|{\tilde{f}}\|_{2}\qquad(f\in{\mathcal{F}}_{n}).$rouie TRAsrows 173 $L^{1}(R^{n}).$ Note thiat ${\mathcal{P}}_{n}$ is dense in $L^{2}(R^{n}),$ for the same reason tha ${\mathcal{P}}_{n}$ is dense in ${i j}/$ isometry onto Thus C2) shows that $f\to{\hat{f}}$ is an isomctry (relative to the $L^{2}.$ -metric) of thc dense subspace ${\mathcal{F}}_{n}\circ\mathbb{F}L^{2}(R^{n})\ o n t o\,{\mathcal{F}}_{n}$ (The mappingis ont y teivesio has a unique continuous extension heormto ema ietesieamemsg a and that this $\mathbf{\hat{P}}$ ’ is a linear $L^{2}(R^{n})$ $\Psi\!:L^{2}(R^{n}) arrow L^{2}(R^{n})$ $f arrow{\hat{f}}$ Some deais i s reiven Eeice 13 and $\mathcal{G}$ in $L^{2}(R^{n}).$ That the Fourir transorm is an $L^{2}.$ " po oa esolrma ms e fnania $\boldsymbol{f}$ features of the whole subject P-isometry is one of the most important Tempered Distributions $\left.\right.\circ .\qquad\qquad\qquad\qquad\qquad\qquad\qquad\qquad\qquad\qquad\qquad$ Befoe etes saih fowng aiontewe ${\mathcal{P}}_{n}$ and 9(R”") 7.10 Theorem (a)9(R" is dense in J9, is continuous. 0の)7ne ieumimimi 40")m ${\mathcal{F}}_{n}$ as defined in Sections 6.3 and ${\mathcal{V}}.{\underline{{3}}}$ Thecsemensece os eua ogis ${\mathcal{D}}(R^{n})$ and ${\mathcal{G}}_{s}$ PRO0F (a) Choose $J\in\mathcal{P}_{n},\ \psi\in\mathcal{D}(R^{n})$ so that ${\mathcal{Y}}=1$ on the unit ball of $\textstyle R^{n},$ and put (1) $f_{r}\in{\mathcal{D}}(R^{n}).$ If $\boldsymbol{\mathit{P}}$ is a polynomial and $\textstyle{\mathcal{A}}$ is a multi-index, then Then $$ f_{r}(x)=f(x)\psi(r x)\qquad(x\in R^{n},\,r>0). $$ ${\mathcal{G}}_{n},$ and (b)If $\boldsymbol{K}$ is a compact set in $\textstyle R^{n}\!,$ $$ P(x)D^{x}(f-f_{r})(x)=P(x)\sum_{\beta\leq x}c_{x\beta}(D^{x-\beta}f)(x)r^{|\beta|}D^{\beta}[1-\psi(r x)]. $$ for every multi-index $\beta\leq\alpha.$ t follow in Our choice of $\mathcal{Y}$ shows that $D^{\theta}[1-\psi(r x)]=0$ for all Thus $f_{r} arrow f$ when $|x|\leq1/r$ $\mathbf{\Psi}(a)$ is proved. that te above sum tens to tniormy o $R^{n},$ when $r\to0.$ $\boldsymbol{\beta}$ Since $f\in{\mathcal{F}}_{n}$ we have $P\cdot D^{x-\beta}f\in C_{0}(R^{n})$ 6.6 $(1+|x|^{2})^{N}$ is bounded on $K.$ th topology induced on into ${\mathcal{P}}_{n}$ is therefore 中 clcausae sais ul ne asietea S ecio" Aa ${\mathcal{D}}_{K}$ by ${\mathcal{P}}_{n}$ is The identity mapping of me ${\mathcal{D}}_{K}$ contiusualya omoismna nw ilowrom iteoe $\,|j\rangle\!\rangle$174 DISTRIBUTIONS AND FOUR IER TRANSFOR MS 7.11 Definition If $i:{\mathcal{D}}(R^{n})-{\mathcal{P}}_{n}$ is the identity mapping, if ${\boldsymbol{L}}$ is a continuous linear functional on ${\mathcal{P}}_{n}\,;$ , and if (1) $$ u_{L}=L\circ i $$ ${\mathcal{D}}(R^{n})$ in then the continuity of $\dot{\boldsymbol{l}}$ (Theorem 7.10) shows that $u_{L}\in\mathcal{D}^{\prime}(R^{n});$ the denseness of ${\mathcal{P}}_{n}$ shows that two distinct L's cannot give rise to the same ${\mathcal{U}}.$ Thus (D describes a vector space isomorphism between the dual space ${\mathcal{P}}_{n}$ 、of ${\mathcal{G}}_{n},$ on the one hand, and a certain space of distributions on the other. The distributions that arise in this way are called tempered: The tempered distributions are precisely those ue 9'(R") that have continuous extensions to ${\mathcal{I}}_{s}$ In view of the preceding remarks, it is customary and natural to identify $\textstyle u_{L}$ with $\boldsymbol{\ L}$ The tempered distributions on ${\boldsymbol{R}}^{n}$ are then precisely the members o ${}^{c}{\mathcal{P}}_{n}^{\prime}\,.$ The following cxamplcs will explain the use of the word“tempered”in this connection; it indicates a growth restriction at infinity.(See also Exercise 3.) 7.12 Examples(a) Every distribution wih compact suppor $\dot{\boldsymbol{k}}$ tempered. Suppose $\textstyle K$ is the compact support of some $u\in{\mathcal{D}}^{\prime}(R^{n}),\operatorname{fix}\psi\in{\mathcal{D}}(R^{n}):$ so that ${\mathcal{Y}}=1$ in some open set containing $K_{\!\cdot\!}$ and define (1) $$ \tilde{u}(f)=u(\psi f)\qquad(f\in{\mathcal P}_{n}). $$ If $\mathbf{f}f_{i} arrow0$ in ${\mathcal{G}}_{\bullet}.$ then all $D^{\alpha}f_{i} arrow0$ uniformly on $R^{n}$ ”, hence all $D^{*}(\psi f_{i})\to0$ uniformly on $R^{n}$ ,so that $\psi f_{i}\lnot\cup0$ in ${\mathcal{A}}(R^{n}).$ It follows that $\widetilde{\cal H}$ is continuous on ${\mathcal{F}},$ Since $\tilde{u}(\phi)=u(\phi)$ for $\phi\in{\mathcal{D}}(R^{n}),$ i is an extension of $\boldsymbol{\mathit{l}}$ (b)Suppose $\boldsymbol{\mu}$ is a positive Borel measure on $R^{\ n}$ such that (2) $$ \int_{R^{n}}(1\,+\,\vert\,x\,\vert^{2})^{-k}\,d\mu(x)\,<\infty $$ for some positive integer $\boldsymbol{K}$ Then $\boldsymbol{\mathit{H}}$ is $\overline{{a}}$ tempered distribution. The assertion is, more explicitly, that the formula (3) $$ \Lambda f=\int_{R^{n}}f d\mu $$ defines a continuous linear functional on in ${\mathcal{F}}_{n}.$ Then To see this, suppose ${\mathcal{P}}_{n}\,.$ ${\mathcal{f}}_{i}\to0$ (4) $$ \varepsilon_{i}=\operatorname*{sup}_{x\in R^{n}}(1+|x|^{2})^{k}|f_{i}(x)|\to0. $$ Since |A/l is at most $G_{i}$ times the integral in (2), $\Lambda f_{i} arrow0.$ This proves the continuity of A.FouRIER TRANSFORMs 175 (c)Suppose $|^{\cdot\leq}p<c o,~N>0,$ and $\scriptstyle{\mathcal{G}}$ is a measurable function on $R^{n}$ such thal (5) $$ \int_{R^{n}}\left\vert(1\,+\,\vert x\vert^{2})^{-N}g(x)\right\vert^{p}\,d m_{n}(x)=C<\infty. $$ Then g is a temped dstributon As in $(b)_{\!\cdot\!}$ define (6) $$ .^{\cdot\cdot\cdot}\qquad.\qquad\Lambda f=\int_{{\cal R}^{n}}f g\,d m_{n}\,. $$ gives Assume first that $p>1$ ; let c $\boldsymbol{\mathit{d}}$ be he coniugate exponent. Then Hilders inequalit ( $$ \begin{array}{l l}{{7)}}&{{\mathrm{\Large~\left.~\right.~\left[\Lambdaf\right|\leqC_{\sum_{\cdot}}^{1/p}\left\{\int_{R^{n}}^{1}\left[(1+\vert x\vert^{2})^{N}f(x)\vert^{q}\,d m_{n}(x)\right\}^{1/e}}}}\\ {{}}&{{\mathrm{\footnotesize\leq~- .~-1/p_{B}^{1/q}{\scriptstyle\stackrel{\cdot~}{x\in R^{n}}} [(1+\vert x\vert^{2})^{M}f(x)\right],\cdot\cdot}}}\end{array} $$ where $\mathcal{M}$ is taken so large that $$ {\qquad}\quad\quad\quad\int_{R^{n}}(1\,+\,|x|^{2})^{(N-M)q}\,d m_{n}(x)=B<\infty. $$ The inquality TO proves tha is coninuous o $g\in L^{p}(R^{n})$ $1\leq p\leq\infty)$ is a tempered distribu- (dO t follows from c that ever ${\mathcal{P}}_{n}$ The case $\scriptstyle{p=1}$ is even easier in., severy poyomial and. moeaye masaie uncoiwho absolute value is majorized by some polynomial 7.13 Theorem 1f ais a mult-index ${\boldsymbol{P}}$ is a polynomia, $g\in{\mathcal{P}}_{n}$ , and $\boldsymbol{\mathit{l}}$ is a temperel distriution, hen he distribuions Du, $\textstyle P_{}^{a}$ and gu are aso tempered PROOF This follows directly from $\mathbf{\phi}(\mathbf{\phi})$ of Theorem ${\mathcal{V}}_{.}A$ and the definitions $$ \begin{array}{l c r}{{(D^{\prime}u)(f)=(-1)^{|x|}u(D^{x}f),}}\\ {{(P u)(f)=u(P f),}}\\ {{(g u)(f)=u(g f).}}\end{array} $$ // 7.14 Definition For ue ${\mathcal{I}}_{n}^{*},$ define (1) $$ \hat{u}(\phi)=u(\hat{\phi})\qquad(\phi\in{\mathcal P}_{n}). $$ ${\hat{u}},$ Since $\phi\circ\gamma~{\hat{\phi}}$ is a continuous linear mupping or it follows that ${\hat{u}}\in{\mathcal{G}}_{n}^{\prime}.$ into ${\mathcal{T}}_{n}$ [(d) oI Theorem 7.4, and since u $\boldsymbol{\mathit{u}}$ is continuous on ${\mathcal{P}}_{n}$ ${\mathcal{P}}_{n}$ We havetuscaevitech temedisrbtion is ouic amsfor which s aina tepediution. Ournex teremi sw ththeform176 Disriusorios AND roRiuER TRANSroxws properties of Fourier transforms of rapidly decreasing functions are preserved in the larger setting of tempered distributions to the function J Butfrst there arises a consistency quesion that ought to be setle $1{\bf f}f{\bf c}=L^{1}(R^{n}),$ ${\hat{f}}.$ then f may also be regarded as a temperd distribution, say $u_{f}\,.$ ,so that two definitions of the Fourier transform are available namely,(c) of Section 7.1 and Definition 7.14 corresponds The quesion is whether they agee, i., whether the distribution $\scriptstyle(u_{j})^{\times}$ Thc answer is affrative, because $$ (u_{J})^{\times}(\phi)=u_{J}(\hat{\phi})=\textstyle\int f\hat{\phi}=\textstyle\int f\phi=(u_{\tilde{J}})(\phi) $$ for every $\phi\in{\mathcal{G}}_{n}$ The third of these eqalitis s the identity 3) of Scctin 7.7; thc others are definitions. persists $\mathrm{for}f\in L^{z}(R^{*})$ and $\phi\in{\mathcal{F}}_{n}$ the same question arises for the Fourier-Plancherel trans $\textstyle{\int}{\hat{f}}{\hat{\phi}}=\int{\hat{f}}{\hat{\phi}}$ form. Since $L^{2}(R^{n})\subset{\mathcal{F}}_{n}^{\prime}.$ Theanswer isaganaffrative, bythe same proof since th identit 7.15 Theorem (a) o/ period The Fourier transform is a continuous linear, one-to-one mapping of ${\mathcal{P}}_{n}^{\prime}$ onto ${\mathcal{F}}_{n}^{\prime}$ ${\mathrm{d}}_{},$ whose inverse is also coninuous. (b) $I f\,u\in{\mathcal{F}}_{n}$ and $\boldsymbol{P}$ is a polynomial, then $$ (P(D)u)^{\star}=P{\hat{u}}\qquad a n d\qquad(P u)^{\star}=P(-D){\hat{u}}. $$ Note that these are the analogues of(() of Theorem and $P(-D)$ are defined in terms of of Theorem ${\mathcal{V}}_{\cdot}{\mathcal{A}},$ The topology to which (a) refers is the weak*-topology that ${\mathcal{V}}.{\mathcal{I}}$ and $(c)$ .Note also that the differential operators ${\mathcal{P}}_{n}$ induces on ${\mathcal{F}}_{n}^{\prime}$ not $P(D)$ $D_{x},$ $D^{\sharp}$ see $(d)$ of Section 7.1. PR00F Let ${\mathcal{W}}$ be a neighborhood of O in ${\mathcal{P}}_{n}$ Then there exist function $\phi_{1},\,.\,.\,.\,,\,\phi_{k}\in{\mathcal{P}}_{n}$ such that (I) $$ \{u\in\mathcal{G^{\prime}}_{n}^{\prime}\colon|u(\phi_{i})|\ <1\quad\mathrm{for}\quad1\leq i\leq k\}\subset\mathcal{W}. $$ Define (2) $$ V=\{u\in{\mathcal{P}}_{n}^{\prime}\colon\,|u(\hat{\phi}_{i})|\,<1\quad\mathrm{for}\quad|\,\leq i\leq k\}. $$ Then $\mathcal{V}$ is a neighborhood of $\mathbf{\nabla}(\mathbf{\nabla})$ in ${\mathcal{I}}_{*}$ ,and since (3) $$ \hat{u}(\phi)=u(\hat{\phi})\qquad(\phi\in{\mathcal{F}}_{n},\,u\in{\mathcal{F}}_{n}^{\prime}), $$ we see that $\hat{\boldsymbol{u}}$ $\textstyle{\varepsilon\,W}$ whenever uE $V.$ This proves the continuity of d $\Phi$ . where we write $\Phi_{W}={\hat{n}}.$ Since $\Phi$ has period $\scriptstyle A$ on ${\mathcal{I}}_{n},(3)$ shows that $\Phi$ has pcriod $\ 4$ 4 on ${\mathcal{F}}_{n}^{\prime}\,;$ that is, that $\Phi^{4}\iota=u$ for every $u\in{\mathcal{F}}_{n}^{\prime}.$ Hence $\Phi$ is one-to-one and onto, and sincc $\Phi^{-1}=\Phi^{3},\,\Phi^{-1}\;;$ is continuousrouR RAsroxws 177 by the computations Statement の)foflows from Go)of Theore ${\mathcal{V}}.{\mathcal{A}}$ and from Theorem 7.13, and $$ \begin{array}{l l}{{(P(D)u)^{\sim}(\phi)=(P(D)u)(\hat{\phi})=u(P(-D)\hat{\phi})}}&{{\qquad\qquad\qquad}}\\ {{=u(P\phi)^{\times})=\hat{u}(P\phi)=(P\hat{u})(\phi)}}&{{\qquad.}}\end{array}\qquad. $$ $$ (P(\cdot\!-D)\partial)(\phi)=\hat{a}(P(D)\phi)=u((P(D)\phi)^{\times}) $$ =u(P4) =(PD0(6) =(Prの~(6) where $\phi~~~~~~~~~~~~~~~~~~~~~~~~~~~~~~~~~~~~~~~~~~~~~~~~~~~~~~~~~~~~~~~~~~~~~~~~~~~~~~~~~~~~~~~~~~~~~~~~~~~~~~~~~~~~~~~~~~~~~~~~~~~~~~~~~~~~~~~~~~~~~~~~~~~~~~~~~~~~~~~~~~~~~~~~~~~~~~~~~~~~~~~~~~~~~~~~~~~~~~~~~~~~~~~~~~~~~~$ b is an arbitrary function in ${\mathcal{G}}_{x}.$ // 7.16 .Rxanpls oesw seon,. at pynoma a ed d by the formula tributions. Thi uortnsomareasicmutaiwe beni itepo $\phi~~~~~~~~~~~~~~~~~~~~~~~~~~~~~~~~~~~~~~~~~~~~~~~~~~~~~~~~~~~~~~~~~~~~~~~~~~~~~~~~~~~~~~~~~~~~~~~~~~~~~~~~~~~~~~~~~~~~~~~~~~~~~~~~~~~~~~~~~~~~~~~~~~~~~~~~~~~~~~~~~~~~~~~~~~~~~~~~~~~~~~~~~~~~~~~~~~~~~~~~~~~~~~~~~~~~~~~~~~~~~~~~~~~~~~~~~~~~~~~$ nomial l; regarded as a dstitiono $\scriptstyle\mathbf{R}\cdot\mathbf{i}$ acts on test functions (1) $$ 1(\phi)=\int_{{\cal R}^{n}}1\phi\ d m_{n}=\int_{{\cal R}^{n}}\phi\ d m_{n}. $$ Hence (2) · $$ {\hat{\Pi}}(\phi)=1({\hat{\phi}})=\int_{R^{n}}{\hat{\phi}}\;d m_{n}=\phi(0)=\delta(\phi), $$ where $\delta$ is the Dirac measure on $R^{n}.$ Lkewise (3) $$ \hat{\delta}(\phi)=\delta(\hat{\phi})=\hat{\phi}(0)=\int_{\scriptscriptstyle{R^{n}}}\phi\;d m_{n}=1(\phi). $$ Thus (2) and $\mathbf{\nabla}(3)$ give thc results (4) $$ \hat{\mathbb{I}}=\delta\qquad\mathrm{and}\qquad\hat{\delta}=1. $$ with $u={\bar{\delta}}$ and with is now an arbitary polynomial on $R^{n},$ and i we ppy(b of Theorem 7.15 If $\boldsymbol{P}$ $u=1.$ the resulsin 4) show that (5) $$ (P(D)\delta)^{\wedge}=P\qquad\mathrm{and}\qquad\hat{P}=P(-D)\delta. $$ $I J\textstyle u\in{\mathcal{F}}_{n}$ , then (0) = i, where $\dot{\mathcal{U}}$ $\dot{\boldsymbol{\imath}}S$ The oruma A s e s soe GO) a so eve om eac followng way $\scriptstyle{\boldsymbol{\tau}}^{\prime}\qquad$ eho u enisnioeoen ihn my eaeo emesisoms defined by (6) $$ \tilde{u}(\phi)=u(\tilde{\phi})\qquad(\phi\in{\mathcal G}_{n}). $$ The pof is trivial, since $(\bar{\phi})^{\star}=\bar{\phi},$ by (a) of Theorem ${\mathcal{V}}_{*}{\mathcal{V}}\colon$ Note that $$ (\hat{\mu})^{\star}(\phi)=\hat{u}(\hat{\phi})=u((\hat{\phi})^{\star})=u(\hat{\phi})=\check{u}(\phi). $$ ${\bar{\delta}}={\bar{\delta}}.$178 DIsrRustroNs AND FouRIFR TRANSFORMS with ${\mathcal{D}}(R^{n})$ in place of ${\mathcal{I}}_{n},$ If we combine (5) with Theorem 6.25, we find that a distribution is the Fourier trasform of a polynomialif ad only if itsuport is the orin or the empty set) The following lemma will be used in the proof of Theorem 7.19. Its analogue is much casicr and was used without comment inthe proof of Theorem 6.30. 7.17 Lemma_1 ${}^{F}w=(1,0,\ldots,0)\in R^{n},\,i\!f\,\phi\in{\mathcal{P}}_{n},$ and $i f$ (1) $$ \phi_{\varepsilon}(x)={\frac{\phi(x+\varepsilon w)-\phi(x)}{\varepsilon}}\qquad(x\in R^{n},\,\varepsilon>0), $$ then $\phi_{s}\to\partial\phi/\tilde{Q}x_{1}$ in the topology $\mathcal{I}\,\mathcal{P}_{n},\,a s\,\varepsilon\to0.$ of $\phi_{\varepsilon}-\partial\phi/\partial x_{1}$ tends to $\mathbf{0}$ PRor The conclusion can be obtained by showing that the Fourier transform in ${\mathcal{G}}_{n}$ 一 ht is, by showing that (2) $$ \begin{array}{c c c}{{\psi_{\varepsilon}\,\hat{\phi} arrow0\;\mathrm{in}\;{\mathcal G}_{n}\,,~~~}}&{{\mathrm{as}\;\varepsilon arrow0,}}\end{array} $$ where (3) $$ \psi_{\varepsilon}(y)=\frac{\exp\left(i\varepsilon y_{1}\right)-1}{s}-i y_{1}\qquad(y\in R^{n},\,\varepsilon>0).\qquad $$ If $\boldsymbol{P}$ is a polynomial and $\textstyle{\mathcal{A}}$ is a multi-index, then (4) $$ P\cdot D^{\alpha}(\psi_{\varepsilon}\,\hat{\phi})=\sum_{\beta\leq x}c_{\alpha\beta}P\cdot(D^{x-\beta}\hat{\phi})\cdot(D^{\beta}\psi_{\varepsilon}). $$ A simple computation shows that (5) $$ |D^{\beta}\psi_{e}(y)|\leq{\binom{\varepsilon y_{1}^{2}}{\langle{\varepsilon|y_{1}|}\atop{\varepsilon|}y_{1}|}}\quad{\mathrm{~if~|}\beta|=0,}}\\ {{|{\beta|-1}\quad}}&{{\mathrm{~if~|}\beta|>1.}}\end{array} $$ of the topology of ${\mathcal{F}}_{n}$ tends therefore to Q, uniformly on $R^{n},\,\mathrm{as}\,s\to0,$ The definition // The left side o $\mathbb{F}(4)$ (Section 7.3) shows now that (2) holds 7.18 Definition If $u\in{\mathcal{F}}_{n}^{\prime}$ and $\phi\in{\mathcal{G}}_{n},$ then $$ (u*\phi)(x)=u(\tau_{x}\tilde{\phi})\qquad(\dot{x}\in R^{n}). $$ Note that this is well defned, since $\tau_{x}\,\tilde{\phi}\in\mathcal{P}_{n}$ for every $x\in R^{n},$ 7.19 Theorem Suppose $\phi\in{\mathcal{G}}_{n}$ and uis a tempered distibution. Then $(a)\quad u*\phi\in C^{**}(R^{n}),$ and D(u * 4)=(Dfu) * 4 =ut(D*4) for every multi-index α,roune xsows 179 (C $\begin{array}{l l}{{\\!{\epsilon}}}&{{(u*\phi)^{\times}=\hat{\phi}\hat{\bar{u}}}}\end{array}$ womnho ool nm (a)u* 6) $**\psi=u*(\phi*\psi),f o r\;e v e r y\;\psi\in\mathcal{G}_{n},$ (e)a+6=(pun广 (2) shows that rxoor Th secoduaiy $(a)\leq b$ provexaty a n Theorem .0 sin lteration of (1) $\operatorname{Let}p_{N}(f)$ denote the norm $\operatorname{\mathcal{(1)}}$ eouo ysesm…s mans…5 sas if $x=(1,0,\dots,0)$ The inequlit Lema . now giv $$ \left(\frac{\tau_{-\,\varepsilon w}-\tau_{0}}{\varepsilon\sp{\circ}}\right)(u\ast\phi)=u\ast\left(\frac{\tau_{-\,\varepsilon w}-\tau_{0}}{\varepsilon}\right)\phi. $$ thspecia as ives a $D^{x}(u\ast\phi)=u\ast(D^{x}\phi)$ of Section 7.3, for fe sy, $$ 1\,+\,\vert x+y\vert^{2}\leq2(1\,+\,\vert\,x\vert^{2})(1\,+\,\vert\,y\vert^{2})^{\ '\, arrow\,-\,+\,-\,+\,-\,\dots\,\iota\,\dots\,\iota}_{\iota\star\,\iota} $$ (3) $$ \begin{array}{c l c r}{{p_{N}(\tau_{x}f)\leq2^{N}(1+\vert x\vert^{2})^{N}p_{N}(f)\quad}}&{{(x\in R^{n},f\in{\mathcal G}_{n}).}}\end{array} $$ and sic the nors ${\mathcal{P}}_{N}$ determine (4) the topology of ${\mathcal{I}}_{s}$ Sincgso ous ninrcioian anda $C<\infty$ such that , he is a ${\mathcal{G}}_{n}$ $\hat{N}$ $$ \left|u(f)\right|\leq C p_{N}(f)\qquad(f\in{\mathcal I}_{n}); $$ sce aer Execie s B OJ am aA (5) $$ \vert(u*\phi)(x)\vert\,=\,\vert\,u(\tau_{x}\phi)\vert\,\leq2^{N}C p_{N}(\phi)(1\,+\,\vert\,x\vert^{2})^{N}, $$ which proves (の) then Thus $u*\phi$ has a ournom : K".w usupo $K_{\!_{J}}$ so that (u $$ \begin{array}{r l}{\ast\ \phi)^{\displaystyle\sim(u\ast\phi)(\tilde{\psi})=\int_{\scriptscriptstyle{R^{n}}}(u\ast\phi)(x)\psi(-x)\;d m_{n}(x)}\\ {\displaystyle=\int_{-\kappa}u[\psi(-x)\tau_{x}\tilde{\phi}]\;d m_{n}(x)=u\int_{-\kappa}\hat{\psi}(-x)\tau_{x}\tilde{\phi}\;d m_{n}(x)}\\ {\displaystyle=u((\phi\ast\psi)^{\vee})=\hat{u}((\phi\ast\psi)^{\vee})=\hat{u}((\phi\ast\psi)^{\star})=\hat{u}((\phi\ast\psi)^{\star})}\end{array} $$ (6) $$ (u\ast\phi)^{\times}({\hat{\psi}})=_{}({\hat{\phi}}{\hat{u}})({\hat{\psi}}). $$ W for $u_{\textbf{t h}}$ The distribution $(u*\phi)^{*}$ 如 sosoualn .em 3 s piei o ${\mathcal{G}}_{s}$ the Furir trasforms o mbes .-valued $\hat{\psi}\in{\mathcal{S}}_{n}\,.$ ${\mathcal{G}}_{x.}$ by is dense in mg…ssmts ${\mathcal{I}}_{n}.$ $\psi\in{\mathcal{D}}(R^{n}).$ Since ${\mathcal{D}}(R^{h})$ of Theorem 7. ene O oi fireve 9f 90(R" re as ense $\mathbf{\nabla}(\partial)$ and an ereaYnspos180 Disrunoros ANp rouRuER TRAsrows ln the computation thtprcedes (O) the two end terms are now seen to be equal for any $\psi\in{\mathcal{G}}_{n}.$ Hence (T) $$ (u*\phi)(\bar{\psi})=u((\phi*\psi)^{\times}), $$ which is the same as (8) $$ ((u*\phi)*\psi)(0)=(u*(\phi*\psi))(0). $$ If we replacc $\textstyle\psi$ by $\tau_{x}\not\psi$ in (8), we obtain (d). , by (co above and (6) of Section 7.16; this Finally $(\hat{\mu}\ast\hat{\phi})^{\times}=\tilde{\phi}\sp{\times}=(\phi u)\sp{\times}$ gives Ge), sinc $(\phi u)^{\sim}=((\phi u)^{\times})^{\times}.$ Paley-Wiener Theorems $L^{2}.$ variables), onc for $C^{\infty}$ One or teclasialtheorems of Palyand Wiener charctries thectie unction functions with compact suppot; see of expontia ype of one complex variable). whse restiton o the realaxis isi $L^{2}.$ , as being exactly the Fourier transforms of for instance, Teorem 19.3 f 23] Weshall give two analogus f his in severa -functions with compact suport, and one for distriutions wit compact support tion in $\Omega_{\mathrm{{,}}}$ separately. This means that i $\left(a_{1},\,\ldots,\,a_{n}\right)\in\mathbf{C}$ and $\mathbf{i}\mathbf{f}$ and if fis a continuous complex func 7.20 Defnitions If $\underline{{\mathbf{Q}}}$ is an open set in ${\mathcal{C}}^{n};$ if tis holomorphic in each variable then $\boldsymbol{f}$ is said to be holomorphic in $\mathbb{Q}$ $$ g_{i}(\lambda)=f(a_{1},\ldots,\,a_{i-1},\,a_{i}+\lambda,\,a_{i+1},\,\ldots,\,a_{n}), $$ ${\mathcal{C}}.$ Points of ${\mathcal{C}}^{n}$ A function that is holomorphic in all o ${\mathcal{C}}^{n}$ is to be holomorphic in some neighborhood of where $Z_{k}\in\nabla.$ 1 z在一 ${\boldsymbol{0}}$ O in each of the functions $g_{1},\cdot\cdot\cdot,g_{n}$ is said to be entire. will be denoted by $z=(z_{1},\dots,z_{n}),$ $z=x+i y.$ The vectors $x_{k}+i y_{k},\,x=(x_{1},\,\ldots,\,x_{n}),\,y=(y_{1},\,\ldots,\,y_{n}),$ then we write $$ x=\mathrm{Re}\,z\qquad{\mathrm{and}}\qquad y=\mathrm{Im~}z $$ are the real and imaginary parts of $\xrightarrow{\xrightarrow{\varphi}}_{3}$ respectively $R^{n}$ will bethought of as the set of all $z\in C^{n}$ with Im $z=0.$ The notations $$ \mid z\mid=(\mid z_{1}\mid^{2}+\cdot\cdot\cdot+\mid z_{n}\mid^{2})^{1/2} $$ $|\operatorname{Im}z|=(y_{1}^{2}+\cdot\cdot\cdot+y_{n}^{2})^{1/2}$ $z^{\alpha}=z_{1}^{\alpha_{1}\cdot\cdot\cdot}z_{n}^{\alpha_{n}}$ $$ z\cdot t=z_{1}t_{1}+\cdot\cdot\cdot+z_{n}t_{n} $$ $$ e_{z}(t)=\exp\left(i z\cdot t\right) $$ will be used, for any multi-index x and any $\iota\in R^{*}.$roue RxAsrows 1 ${\mathcal{G}}_{i}$ 7.21 Lemma 7/s ameiei mimn " z amice as known $\mathrm{Let}\,P_{k}$ be te eowg propert is to PROOF be proved. Assum $1\leq i\leq n$ and $\textstyle P_{i}$ $R^{n}.$ $t h e n f=0$ $P_{0}$ $\lambda\in C.$ $\mathrm{of}f\colon1f:\in{\mathit{C}}^{n}$ we consd the cas $n=1$ $\mathrm{then}\,f(z)=0.$ $\textstyle P_{n}$ is given; // hsa lea $\left\{\begin{array}{l l}{K}&{\quad}\end{array}\right.$ real oni 1 flows tha conie eton 7.0* s he is true. Take $Q_{1},\ \cdot\cdot\cdot,\ d_{i}$ real. The functio $P_{i-1}$ is true ${\boldsymbol{0}}$ on the real axis, hence is $\mathbf{0}$ for all l tefolowngtwo theorem $$ r B=\{x\in R^{n}\colon|x|\leq r\}. $$ 7.22 Theorem (a) $I f\phi\in{\mathcal{D}}(R^{n})$ has is sunoni Ra $i f$ (1) $$ f(z)=\int_{R^{n}}\phi(t)e^{-i z\cdot t}\,d m_{n}(t)\qquad(z\in C^{n}), $$ ” $\phi\in{\mathcal{D}}(R^{n}),$ then senune,ad e recosnan $\boldsymbol{\mathit{f}}$ m/mm…n (2) with suport im $r B_{\mathrm{i}}$ $\gamma_{\mathrm{N}}\leqslant\infty\ s u c h\ i$ thal 一M/…m… $$ |f(z)|\le\gamma_{N}(1+|z|)^{-N}e^{r|\mathrm{Im}.z|}\qquad(z\in C^{n},\,N=0,1,2,...,). $$ rRoor (a) Ifie rB te , uchai() old on The intcgrand in $\operatorname{\left(\,1\,\right)}$ $$ [\,e^{-i z\cdot t}\,]\,=\,e^{\nu\cdot x}\leq e^{|\nu||t|}\leq\,e^{r|l m\cdot z}]. $$ for every $\mathbb{Z}\in{\mathit{C}}^{n}.$ and fis well defined $C^{n}.$ is therefore i $I^{1}(R^{n}),$ Siu…aco-… Tolsorss………m snato es as…"ssenseinsSsli $$ z^{\alpha}f(z)=\int_{\boldsymbol{R^{n}}}(D_{\alpha}\,\phi)(t)e^{-i z\cdot\,t}\,d m_{n}(t). $$ Hence (3) supo $\boldsymbol{\mathit{J}}$ isa s neitnas .an cin $(b)$ $$ |z^{\alpha}|\;|f_{\zeta_{c}}\rangle|\;\leq\;\|D_{\alpha}\,\phi||_{1}e^{r\|L\mid m\;z}|, $$ anu foo o einulis (4) $$ \phi(t)=\int_{R^{n}}f(x)e^{i t\cdot x}\,d m_{n}(x)\qquad(t\in R^{n}). $$ Notefrs tha $({\mathrm{I}}+\vert x\vert)^{N}f(x)$ is in $L^{1}(R^{n})$ for every $N,$ by (2). Hence $\phi\in C^{\infty}(R^{n}),$ by te rm aroeN"6 oen182 DisrRuaorTos ANp rouER TRANSsrows Next, we claim that the integral (5) $$ \int_{-\infty}^{\infty}\!f(\xi+i\eta,\,z_{2},\,\ldots,\,z_{n})\exp\left\{i[t_{1}(\xi+i\eta)+t_{2}\,z_{2}+\cdots+t_{n}z_{n}]\right\}d\xi $$ see this, let real axis, one on the line tributions of the verticaledges o thsintegral tend to $0.$ and complex $z_{2}\;,\;.\;.\;.\;,\;z_{n}\,.$ To is independent of ${\boldsymbol{\eta}}.$ for arbitrary real $t_{1},\,\cdot\,\cdot\,\cdot\,\cdot\,,\,\,l_{n}$ -plane, with one edge on the ${\boldsymbol{\Gamma}}$ be a rectangular path in the $(\xi+i\eta).$ is O By C).he co $\eta=\eta_{1}$ 、whse vetical edges move oft o infiniy、 BS ${\boldsymbol{\Gamma}}$ Caucny cem.ieneral o inern 6) oer It follows that() is the same for $\eta=0$ as for $\eta=\eta_{1}$ This establishes our claim. The same can be done for the oter cordinates. Hence we conclude fro (4) that (6) $$ \phi(t)=\int_{{\cal R}^{n}}f(x+i y)e^{i t\cdot(x+i y)}\,d m_{n}(x) $$ for every $y\in R^{n}$ choose $y=\lambda t/|\,t|\,,$ where $\lambda>0.$ Then $t\cdot y=\lambda|t|.$ Given $t\in R^{n},\,t\neq0,$ $|y|=\lambda.$ $$ |f(x+i y)e^{i t\cdot(x+i y)}|\leq\gamma_{N}(1+\vert x\vert)^{-N}e^{(r-\vert t\vert)\lambda}, $$ and therefore (7) $$ \big|\,\phi(t)\big|\,\leq\gamma_{N}e^{(r-|t|)\lambda}\,\int_{R^{n}}(\,1\,+\,|\,x\,|\,)^{-N}\,d m_{n}(x), $$ $|t|>r,(7)$ shows that $\phi(\iota)=0$ is choen so age that the lstitera s fnie Now le has is sport i $\sqrt{\gamma^{*}/h}_{\geq}$ by Lemma ${\mathcal{T}}.{\mathcal{Z}}]\,.$ This where ${\boldsymbol{N}}$ $\lambda\to\infty.$ If completes the proof. Thus $\phi~~~~~~~~~~~~~~~~~~~~~~~~~~~~~~~~~~~~~~~~~~~~~~~~~~~~~~~~~~~~~~~~~~~~~~~~~~~~~~~~~~~~~~~~~~~~~~~~~~~~~~~~~~~~~~~~~~~~~~~~~~~~~~~~~~~~~~~~~~~~~~~~~~~~~~~~~~~~~~~~~~~~~~~~~~~~~~~~~~~~~~~~~~~~~~~~~~~~~~~~~~~~~~~~~~~~~~~~~~~~~~~~~~~~~~~~~~~~~~~~~~~~~~~~~~~~~~~~~~~~~~~~~~$ $r B.$ $\;:\!|J\rangle\!\rangle$ Now(UD foows. oral z from a) and he inversion theorem. Sinc both sides of(I) are entire functions, they coincide on The following remarks will motivate the next theorem made for f∈ $I^{1}(R^{n}),$ suggests that t ought $\mathrm{tO}$ ,, with compact support. Then $\hat{\boldsymbol{u}}$ is defined, as a Let u be a distribution in $\textstyle R^{n}\!\!.$ However, the definition ${\hat{f}}(x)=\textstyle{\int}{\mathcal{E}}_{-x}\,d m_{n}.$ tempered distribution, by $\hat{u}(\phi)=u(\hat{\phi}).$ p be a function, namely. $$ \hat{u}(x)=u(e_{-x})\qquad(x\in R^{n}), $$ becausc $e_{-x}\in C^{\infty}(R^{n})$ and up) makes sense for every $\phi\in C^{\infty}(R^{n}),$ as shown by (d) like an entire function, whose restiction to $R^{n}$ for every zéC", and u(e-) herefore looks of Theorem 6.24.Moreover $e_{-z}\in C^{\infty}(R^{n})$ is t.roue RxAsrows 183 Tnengisomeon a enemn nunsomaieha shunes esunc coeuosremt onsane 7.23 Theorem (1) “… $\gamma<\alpha$ (0)Uve V(K" us sunori R1( aorc $N,$ and if a constant O such thait $$ f(z)=u(e_{-z})\qquad(z\in C^{n}), $$ wo-…/ 人" m -…moo uM (b) (2) $$ |f(z)|\leq\gamma(1+|z|)^{N}e^{r|\mathrm{Im}z|}\qquad(z\in C^{n}). $$ which atsfes 2) or some $\textstyle N$ V and some by (1). Thus Compey. s aneie fucioni $C^{n}$ ${\mathcal{C}}^{n}$ given y,then there exists $u\in{\mathcal{D}}^{\prime}(R^{n}),$ wihsupor m,B。sucn taNY mo AHn nonon mnsnses odenetinsono Thus ${\hat{u}}\in C^{\infty}(R^{n})$ Pick $\phi\in{\mathcal{G}}_{n}$ $$ \hat{u}(z)=u(e_{-z}) $$ ${\tilde{\phi}}=\psi$ Then $r B$ Pick $\psi\in{\mathcal{D}}(R^{n})$ so that $\psi=1$ on (r + D)B. Then $u=\psi v,$ tonx cr ns ensnismsleas ore-tieuenormo (3) Pnoor(a) Suppose $u\in{\mathcal{D}}^{\prime}(R^{n})$ has itsuport i and e of Theorem 7.9 sows th $$ {\hat{u}}=(\psi u)^{\times}={\hat{u}}*{\hat{y}}, $$ so tht so tatO) gives $$ \begin{array}{c}{{(\hat{u}\ast\hat{\psi})(x)=(\hat{u}\ast\hat{\phi})(x)=\hat{u}(\tau_{x}\phi)=u((\tau_{x}\phi)^{\times})}}\\ {{{}=u(e_{-x}\hat{\phi})=u(\psi e_{-x})=u(e_{-x}),}}\end{array} $$ (4) $$ \tilde{u}(x)=u(e_{-x})\qquad(x\in R^{n}). $$ a e $C^{n},\,b\in C^{n},$ and put our nxai osoh at fncto eine b $\operatorname{\rho}(\mathbf{I})$ is entie. Choose to (5) Let ${\boldsymbol{\Gamma}}$ be arctangular path in $C^{\infty}(R^{n})$ T poe at s eiet teite oh $e_{-w}\to e_{-z}$ in $C^{\infty}(R^{n}),$ ${\boldsymbol{C}}$ $C^{\infty}(R^{n}),$ the $C^{\infty}(R^{n}).$ -valued integra $$ g(\lambda)=f(a+\lambda b)=u(e_{-a-\lambda b})\qquad(\lambda\in C). $$ $w\to z\,\operatorname{in}\,C^{n}$ $\left(S\right)$ is entire. is continuous from The coinuiy o ps problem r dehtneu then and uis continuous on $\scriptstyle{\mathcal{G}}$ to showtaeco t ticto ${\boldsymbol{C}},$ Since $\lambda\to e_{-a-\lambda b}$ $\mathbf{\Psi}(6)$ $F=\int_{\cal C}e_{-a-lambda b}\,d\lambda$184 DsrRisurioNs AND rouRIER TRANSroRxMs is well defined. Evaluation at any $t\in R^{n}$ is a continuous linear functional on $C^{\omega}(R^{n})$ It therefore commutes with the integral sign. Hence $$ F(t)=\int_{\Gamma}e_{-a-\lambda b}(t)\,d\lambda=\int_{\Gamma}e^{-i a\cdot t}e^{-i(b\cdot t)\lambda}\,d\lambda=0. $$ Thus $F=0.$ , and (6) gives $$ 0=u(F)=\int_{\Gamma}u(e_{-\,a-\lambda b})\,d\lambda=\int_{\Gamma}g(\lambda)\,d\lambda. $$ function $\boldsymbol{\hbar}$ By Morera's theorem, $\textstyle{\mathcal{G}}$ is entire $h(s)=1$ when The proof of part $\mathbf{\Psi}(a)$ wil be completed by proving(2). Choose an auxiliary on the real line, infinitely difereniable, such that $s<1$ and $h(s)=0$ when $S>2,$ and associate with eachz $z\in C^{n}(z\neq0)$ the function (7) 中 $$ b_{z}(t)=e^{-i z\cdot{\bar{t}}}h(|t|\ |z|-r|z|)\qquad(t\in R^{n}). $$ Then $\phi_{z}\in{\mathcal{D}}(R^{n}).$ Since the support of $\boldsymbol{u}$ . is in $r B$ and $h(\left|t\right|\left|z\right|-r\left|z\right|)=1$ if $|t|\,\leq\,|z|^{-1}+r,$ comparison of $\operatorname{\mathcal{(1)}}$ and $(7)$ shows that (8) $$ f(z)=u(\phi_{z}). $$ Since L $\boldsymbol{u}$ has order $N.$ ,there is a $\gamma_{0}\times\infty$ such that $|u(\phi)|\leq\gamma_{0}||\phi||_{N}$ for all $\phi\in{\mathcal{D}}(R^{n}).$ where $\|\phi\|_{N}$ is as in $\operatorname{\mathcal{(1)}}$ of Section 6.2; see $(d)$ of Theorem 6.24.Hence (8) gives (9) $$ |f(z)|\ \leq\gamma_{0}\|\phi_{z}\|_{N}. $$ Oon the suport o $\phi_{z}\,,$ $|\,t\,|\,\leq r+2/|z\,|\,,$ so that (10) $$ \lfloor e^{-i z\cdot t}\rfloor=e^{y\cdot t}\leq e^{2+r\left[\mathrm{Im}\,z\right]}. $$ lf we nowapply he Libiz formula to th product (T) and use (10),(9) implies (2) This completes the proof of part (a) (6))Since f now satisfies 2), we have (11) $$ |f(x)|\leq\gamma(1+|x|)^{N}\qquad(x\in R^{n}). $$ Pick a function $h\in{\mathcal{D}}(R^{n}),$ with support in ${\boldsymbol{B}},$ and is the Fourier transform of some define The restriction $\operatorname{of}f\tan R^{n}$ is therefore in ${\mathcal{F}}_{n}^{\prime}$ such that $\textstyle{\int}h=1,$ tempered distribution w $h_{\varepsilon}(t)=\varepsilon^{-n}h(t/\varepsilon),$ for $\varepsilon>0,$ and put (12) $$ f_{r}(z)=f(z)\tilde{h}_{s}(z)\qquad(z\in C^{n}), $$ where transform of $h_{\boldsymbol{\varepsilon}}\,.$ now denotes the entire function whose restriction to $R^{n}$ is the Fourier ${\hat{h}}_{\varepsilon}$ Statement (a) of Theorem 7.22, applied to $h_{\varepsilon}\,,$ leads to the conclusion that f stisfies 2)of Theorem 7.22 with r +tin place of . ThereforerOURIER TRANSFoRMs 185 lies in $(r+\varepsilon)B$ of Theorem 7.2 mpie $\operatorname{that}J_{\varepsilon}={\hat{\phi}}_{\varepsilon}$ for some $\phi_{\varepsilon}\in{\mathcal{D}}(R^{n})$ whose support $(\partial)$ Then Consider some $\psi\in{\mathcal{P}}_{n}$ suchuhat the suport o Since $f\psi\in L^{1}(R^{n})$ and ${\tilde{h}}_{s}(x)=$ $r B.$ $\hbar(s x) arrow1$ boundedly on $\textstyle R^{n},$ we conclude that $\hat{\mathcal{\psi}}$ does not interset ${\tilde{\psi}}\phi_{\kappa}=0$ for al uienuy sma $\scriptstyle{a>0}$ $$ \begin{array}{r l}{\circ\ u(\hat{\psi})=\hat{u}(\hat{\psi})=\int\!J\!\!\!/\psi\,d m_{n}=\operatorname*{lim}_{\varepsilon arrow0}\int\!f_{\varepsilon}\,\psi\,d m_{n}}\\ {\cong\operatorname*{lim}_{\varepsilon arrow0}\int\hat{\phi}_{\varepsilon}\,\psi\,d m_{n}=\int\!\hat{\psi}\phi_{\varepsilon}\,d m_{n}=0.}\end{array} $$ Hence has sppor $r B$ is anie ucton an since Dbols f $(\partial).$ // Nowe see th $z arrow u(e_{-z})$ z r oy tce .m mises o Sobolev's Lemma $L^{2}$ in $\mathbb{Q}_{*}$ ir isaproproen sbseto $L^{2}$ in $\mathbb{Q}$ if $\textstyle\int_{K}\left|f\right|^{2}d m_{n}<\infty{\mathrm{~fo~}}$ $\bigwedge_{\mathbf{x}}^{r}$ r ery compact $K\subset\Omega$ $\Omega\subset{\mathcal{R}}^{n},$ $\textstyle R^{n}\!,$ n0.Frurtsom ha e en f fnctin Sobolev em nxame e ts wog soo as……msainseslnsms er Sioswesosmuous…u sreeisenisz soos 7.24 explicily tht thcre is inctio ${\mathcal{G}},$ potioy,essn netenonstni nr s if the is a functio $D^{\alpha}\!_{J}$ rand means, is sai e oay ${\mathcal{D}}^{\circ}$ f whichis locally $L^{2}$ rfers t isrin locall Similarly,a distribuion $u\in{\mathcal{D}}\left(\Omega\right)$ is locall ${\mathit{L}}^{2}$ ${\mathcal{I}},$ such that $u(\phi)=\textstyle\int_{\Omega}g\phi\;d m_{n}$ for every $\phi\in{\mathcal{D}}(\Omega)$ To say that afiuntion ha a distibution drivative locally $L^{2}{}_{\!;}$ , such that On the other hand, the cla $C_{\,\,\,\,\,,\,\,\,}^{(p)}(\Omega)$ $$ \int_{\Omega}g\phi\;d m_{n}=(-1)^{|x|}\int_{\Omega}f{\cal D}^{x}\phi\;d m_{n} $$ $D^{\alpha}f$ exst in the lasicalsense, fo ${\boldsymbol{\rho}},$ of fn r 6.A ninisoniasu ene $D^{\alpha}j$ in the lassi cal ses n term is qotien we shall writ $D_{i}^{k}$ for te difential operato $(\partial/\partial x_{i})^{k}$ cons chnngati ne $\mathbb{Q}$ those complex functions in $\mathbb{Q}$ whosé dervatives each mut-index c with $|\alpha|\leq p,$ and are continuousfnctons n 7.25 Theorem Suppose $n,\,p,\,P,\,P$ are inters $n>0,\,p\geq0,$ and $\left(1\right)$ $r>p+{\frac{n}{2}}.$186 DsrRuioroNs AND routu rRANsroxws Suppose $\boldsymbol{f}$ is a function in an open se $\Omega\subset R^{n}$ whose distribution derivatives Df are locally $L^{2}$ in Q, for $1\leq i\leq n,$ $0\leq k\leq r.$ such that fox) = f(x) for almost every Then there is a function $f_{0}\in C^{(p)}(\Omega)$ xe Q Note that the hypothesis inolves no mixed dlerivatives, .e., no terms like $D_{1}D_{2}f.$ The conclusion is that fcan be“corrected”so as to be in $C^{(p)}(\Omega),$ by redefining iton a set of measure $0.$ Note also, as a corollary that if $\ a l l$ distribution derivatives of f are locally $L^{2}$ in $\Omega,$ $\operatorname{then}f_{0}\in C^{\alpha}(\Omega)$ PRoor By hypothesis, there are functions $g_{\mathrm{a}}$ locally $L^{2}$ in $\Omega,$ that satisfy (2) $$ \int_{\Omega}g_{i k}\phi\ d m_{n}=(-1)^{k}\int_{\Omega}f D_{i}^{k}\phi\ d m_{n}\qquad[\phi\in{\mathcal D}(\Omega)], $$ for $1\leq i\leq n,\,0\leq k\leq r$ is a compact subset of $\Omega.$ Choose Let o be an open set whose closure ${\cal K}$ $y\in{\mathcal{D}}(\Omega)$ so that $\psi=1$ on $K,$ and define ${\mathbf{}}F$ on $R^{n}$ by $$ F(x)={\Bigl\{\psi(x)}f(x)\qquad{\textrm{i f}}x\in\Omega, $$ Then $F\in(L^{2}\cap L^{1})(R^{n})$ In $\Omega,$ the Leibniz formula gives $$ D_{i}^{r}F=\sum_{s=0}^{r}\left({}_{s}^{r}\!\right)(D_{i}^{r-s}\psi)(D_{i}^{s}f)=\sum_{s=0}^{r}{\binom{r}{s}\!\!\!\binom{r}{s}\!\!\!\slash(D_{i}^{r-s}\!\!\psi)g_{i s}. $$ In the complement ${\mathfrak{C}}_{0}$ of the support of ${\boldsymbol{\psi}}.$ $D_{i}^{\prime}F=0$ These two distributions coincide in $\Omega:\cap\Omega_{0}$ Hence $v_{i}c_{j}$ originally defined as a distribution in are in $L^{i}(\Omega)$ is $R^{n},$ actually in $L^{2}(R^{n}),$ for $1\leq i\leq n,$ because the functions $(D_{i}^{r-s}\psi)g_{i s}$ [Having compact support $\scriptstyle{\mathcal{I}}_{t}F$ is therefore also in $L^{1}(R^{n}).!$ shows now The Plancherel theorem, applied to ${\mathbf{}}F$ F and to $D_{1}^{r}F,\ldots\cdot D_{n}^{r}F,$ that (4) $$ \Bigr[_{R^{n}}\vert{\hat{F}}\vert^{2}\,d m_{n}<\infty $$ and (5) $$ \int_{\cal R}y_{i}^{2}t\left|\widetilde F(y)\right|^{2}d m_{n}(y)<\infty\qquad(1\leq i\leq n). $$ Since $(6)$ $$ (1\ +\ |y|)^{2r}<(2n+2)^{r}(1+y_{1}^{2r}+\cdots+y_{n}^{2r}), $$rouane rxsrows 18 where $|y|=(y_{1}^{2}+\cdot\cdot\cdot+y_{n}^{2})^{1/2},$ (4) and(S)imply (7) $$ \int_{R^{n}}(1+\vert y\vert)^{2r}\vert{\hat{F}}(y)\vert^{2}\,d m_{n}(y)<\infty. $$ If $\boldsymbol{J}$ unit sphere in $\textstyle R^{n}\!\!$ denots the integral(7). and ir $\sigma_{n}$ , is the $(n-1).$ -dimensional volume of the ” the Shwarz nqualiy gv since $$ \begin{array}{c}{{\left\{\int_{R^{n}}(1+\sqrt{\textstyle{\hat{y}}}\sp{-\cdot\hat{z}})^{p}\right\}\hat{F}(y)|~d m_{n}(y)\ \! \}^{2}\leq J\int_{{\hat{\cal R}}^{n}}(1+\ |y|)^{2p-2r}\,d m_{n}(y)}}\\ {{ .~~~~~~~~~~~~~~~~~~~~~~~~~~~~~~~~~~~~~~~~~~~~~~~~~~~~~~~~~~~~~~~~~~~~~~~~~~~~~~~~~~~~~~~~~~~~~~~~~~~~~~~~~~~~~~~~~~~~~~~~~~~~~~~~~~~~~~~~~~~~~~~~~~~~~~~~~}}\\ {{=J\sigma_{n}\displaystyle\int_{0}^{\infty}(1+t)^{2p-2r}(x-\infty)}}\end{array} $$ $2p-2r+n-1<-1.$ We have thus proved tha (8) $$ \int_{R^{n}}^{\ ^{\ast}}(1\stackrel{\cdot}{\+}|y|)^{p}|\hat{F}(y)|~d m_{n}(y)<...... $$ Define (9) $$ F_{\omega}(x)=\int_{R^{n}}\!\!\!{\widehat{F}}(y)e^{i x\cdot y}\,d m_{n}(y)\,\qquad(x\in R^{n}). $$ that $y^{a}F(y)$ By (e)of th inverin theorem ${\mathcal{L}}^{1}$ whenever $|\alpha|\leq p.$ lieration of the proof o $R^{n}.$ Morcovcr, 8) implie is in $7.7,\,F_{\omega}=F\l x.\L<.$ on of Theorem 7.4 leads therefore to thecocliuso $(c)$ (10) $$ F_{\omega}\in C^{(p)}(R^{n}). $$ x∈ o0. If ${\boldsymbol{\omega}}^{\prime}$ Qur given functio $f_{\mathrm{0}}$ can therefore be defined in $\omega^{\prime}{}_{*}$ Hence $\Omega$ by seting $f_{0}(x)=F_{a}(x)$ if E $C^{(p)}(R^{n}).$ desired function $\boldsymbol{\mathit{f}}$ coincides wit $F_{\mathrm{in}}$ 0. Hencc $f=F_{\omega}$ a.e.in o. $F_{\omega},$ is another set ik ${\boldsymbol{\omega}},$ uhe preceingproo gives a functo in ${\boldsymbol{\sigma}}^{\prime}\cap\omega.$ The which coincides wth a.c. $F_{\omega^{\prime}}=F_{\omega}$ // Exercises 3 to ${\mathcal{I}}_{\cdot}$ 2 Is the topology or ${\mathcal{P}}_{n}$ Suppose is anvtleineroperato ${\mathcal{G}}_{n}$ onto ${\mathcal{S}}_{n}{\boldsymbol{\gamma}}$ ${\boldsymbol{R}}^{n}$ 、wh not ted sc dituin Expres $\hat{\mathcal{G}}$ in tem senaiz S herem ,2 $R^{n},f\in L^{n}(R^{n}),$ and $g(x)=f(A x).$ Suppose are continuous linear functonais o ${\mathcal{D}}(R^{n})$ ;iogo osianmcetiwhn nse rur tr isa ured a form t ioety o cos n e lns show t $\scriptstyle{\mathcal{Q}}$ $f(x)=e^{x},\;g(x)-e^{x}$ tribution but tha Fs no By Exeris tesxs itons i whihaeno oniuos inexesn Eplin vwstsonaeaesianaeosn188 psTRupurioNs AND roURIER TRANSroRMs a) Construct a sequence in ${\mathcal{D}}(R^{n})$ which converges to $\mathbf{0}$ in the topology of ${\mathcal{P}},$ , but not in that of 9(R") ${\mathcal{D}}^{\prime}(R^{1})$ but (6) Constucta sequece of polynomials wich converes nthe topology o not in that of ${\mathcal{P}}_{1}^{\prime}$ into 6 Pove te oprtonsised in Theorem 7.13 are couiousmapings o ${\mathcal{P}}_{n}$ ${\mathcal{P}}_{n}^{\prime}.$ If ue 9';, prove that $$ (\tau_{x}\,u)^{\times}=e_{-x}\,{\hat{u}}\qquad{\mathrm{and}}\qquad(e_{x}\,u)^{\sim}=\tau_{x}\,{\hat{u}} $$ $I O$ for every $x\in R^{n}.$ $7.8$ directy(wihout using Fourier transforms) ${\hat{f}}=\lambda f.$ What can you say about ${\boldsymbol{\lambda}}\,{\boldsymbol{\gamma}}$ $\left\{_{*}\right.$ 8 Suppose fe $L^{1}(R^{n}),f\neq0,\lambda$ is a complex number, and Tis cusomarily defined tob 9 Prove $\mathbf{\Psi}(a)$ of Theorem ${\boldsymbol{R}}^{n}$ The Fourier transform of a complex Borel mcasurc ${\mu}$ L On the function ${\hat{m}}$ i given by $$ {\hat{\mu}}(x)=\int_{{\cal R}^{n}}e^{-i x\cdot\cdot\cdot t}\,d\mu(t)\qquad(x\in{\cal R}^{n}). $$ Of course, $\boldsymbol{\mathit{I}}$ is als a temperedistibution,an asch itsFourier transform was de hnedin Secton 7.14. Show hte tw definitinsare constent. Prove that each A ${\boldsymbol{\mathit{1}}}\mathbf{\widehat{/}}$ Supposc is bounded and uniformly continuous continuous, linear, and $\tau_{x}\wedge=\Lambda\tau_{x}$ for every $x\in R^{n}$ .Does it $\Lambda\colon{\mathcal{P}}_{n}\to C(R^{n}){\mathrm{~is~}}\zeta$ follow that there exists $i\,u\in{\mathcal{P}}_{*}^{\prime}$ such that $$ \Lambda\phi=u*\phi $$ ${\mathit{I}}{\mathit{2}}$ 1f for cvcry is approximate identiy as Definition 6.31. and ${\mathcal{P}}_{n}$ . does it follow that ${\cal{L}}{\cal{L}}$ u Suppose $\textstyle X$ $\phi\in{\mathcal{P}}_{n}$ in thc wcak*-topology of $u\in{\mathcal{P}}_{n}$ is uniformly $\{h_{j}\}$ $*:h_{j}\to u$ as $j\to\operatorname{c}\sigma,$ are complete metric spaces, $\scriptstyle A$ is dense in $x,y$ ${\boldsymbol{4}}\to Y$ and ${\mathbf{}}Y$ continuous. (6) (a) Prove that f has a unique continuous extension $F{\dot{z}}$ $X\to Y.$ is closed $\mathrm{If}$ fis an isometry, prove that the same is true o $F_{\mathrm{{J}}}$ and prove that $\scriptstyle{F(X)}$ 14 in ${\boldsymbol{Y}}.$ an entire function in ${\mathcal{C}}^{n},$ and suppose that to cach $\scriptstyle{n\gg0}$ there correspond an (This ased inteproote Pane iheorem e alo Exeris 19. Chapter I. Suppose $P_{\mathrm{fs}}$ integer $\scriptstyle N(\sigma)$ and a constant $\gamma(\varepsilon)<\infty$ such that $$ |F(z)|\leq\gamma(\varepsilon)(1+|z|)^{\otimes c(t)}e^{\varepsilon|1|m\;z|}\;\;\;(z\in C^{n}). $$ ${\boldsymbol{\jmath}}{\boldsymbol{S}}$ Prove that ${\mathbf{}}F$ is a polynomial is a positive integer,, $r\geq0.$ and Suppose fis an entire function in ${\mathcal{C}}^{n},\,N$ 1/(x)I≤1 $$ \left.\begin{array}{l}{{f(z)\right|\leq(1+\mid z\mid)^{N}e^{r\mid\mathrm{Im~}z\mid}}}\\ {{\cdot\cdot\cdot\cdot\cdot}}end{array} .\qquad [\operatorname{Or~all}\;z\in C^{n},\quad $$ for all xe R". Prove that then 1/G)l≤ eln for all z ∈ C"rourr RxAsrows 18 Suggestion: Fix $z=x+i y\in C^{n};$ define 16 In sphere Let $\textstyle\mu$ of Theore $7{.}23$ $$ g_{*}(\lambda)=(1-i s\lambda)^{-N-i}e^{i\nu t y t}f(x+\lambda y) $$ $|g_{a}(z)|\ll1$ Y $S^{2}.$ Comput oy spec for $\lambda\in C,\ s>0,$ “……………。na umue Let $\delta{\sim}0.$ 路 $\mathbf{\nabla}(\boldsymbol{b})$ i oosaeseaesunug ${\boldsymbol{R}}^{3}$ vti cnea o teu ${\boldsymbol{S}}^{2}$ show tat s tas ie it ionsese ah $N.$ The foing exampl coordinates) tha he oeo oiumesure eouw whiamarnis Put $u=D_{1\mu}.$ Then $$ {\hat{\mu}}(x)={\frac{\sin|x|}{|x|}}\qquad(x\in R^{3}). $$ Dedce fom xecie ha $$ |{\hat{a}}(x)|=|x_{1}{\hat{\mu}}(x)|\leq1\qquad(x\in R^{3}). $$ $I{\mathcal{T}}$ on hence also $(a).$ C0 um X2"9206./ o $K,$ $$ \vert u(e_{-z})\vert\le\gamma e^{\vert\mathrm{Im}\bar{z}\vert}\qquad(z\in C^{3})^{.} $$ $\bar{u}$ w ${\boldsymbol{P}},$ in twvriables t $\operatorname{Inat} \langle\!\left(b\right)\!,$ that is, 一…… $u(e_{-x}),\,z\in C^{3}.$ …osmn esaua… $\phi\in C^{\infty}(R^{*})$ that vanishes where the entire function or $n-2,$ and prove tha ${\sqrt{u}}=0$ for every whosr ouranofom a (a))Assume suppos s stitionh $\textstyle{\mathcal{R}}^{n}.$ wihcomat spor $K,$ a kowie fnciono ${\mathcal{R}}^{n},$ $n=1$ $K.$ Hooposuge ae ou mo paneee (の) Assume $n=2,$ $K.$ Prove tha $P_{i\ell}\equiv0$ oewu $\textstyle K$ semcse vanishes on equation $P(-D){\dot{a}}=0.$ ror oaope w $$ {\hat{\boldsymbol{x}}}+\Delta{\hat{a}}=0, $$ ${\underline{{I}}}-x_{i}^{2}-x_{2}^{2}-x_{3}^{2}$ $\Delta=\mathcal{C}^{2}/\partial_{X_{1}^{2}}+\bar{\sigma}^{2}/\partial_{X_{2}^{2}}$ is th Lalaca in place o $n=2.$ 2 …gous…easom ${\mathit{n}}=3$ becomes ase wi $|f(t)-f(s)|\leq C|t-s|^{1/2}$ Proe ha he Fgsts- oistesuor from $\textstyle{\bar{\cal K}}$ s les tha $\scriptstyle{\varepsilon\;S\;0$ Let $\{h_{c}\}$ $$ \int_{-\infty}^{\infty}\!f(x)\hat{n}(x)\,d x=0. $$ whose distance $\mathbf{\nabla}(b)$ of S"eno: Fr any ${\mathcal{n}}_{!}$ let $\textstyle{H_{e}}$ be e o l onsous $\textstyle K$ pto Tneor n ea u hoscamn eouwns and sho tat terefo $$ \|{\boldsymbol{u}}*h_{e}\|_{2}\le\|{\hat{u}}\|_{\circ}\ s^{-n/2}\|h_{1}\|_{2}\,, $$ for any $\phi\in{\mathcal{D}}(R^{n})$ t $$ \begin{array}{l}{{|\left.\eta(\phi)\right|\leq||\hat{\bar{a}}||_{\Phi}||\vphantom{h}_{1}\left|\vphantom{h}_{1}\right||\vphantom{h}_{1}\left|\vphantom{h}_{1}\left|\vphantom{h}_{2}\right.\left|\vphantom{h}_{\neq}\right.\kern\underbrace{h}_{\Big({\ominus}_{1}\right)} |\vphantom{h}_{\emptyset}_{\Big({\ominus}_{1} )}} \}^{1/2}}}\\ {{\mathrm{int~vanishese~on~K^{-}~{\xrightarrow{~~~c o}}{\1{\vphantom{h}_{\Big({\bot}}_{1}\bot\bot} )}}\end{array} $$ Tuao onsusasn:5 n190 DiSTRIBUrIoNS AND FoURIER TRANSFORMS 18 Was it necessary to introduce the function $\psi$ into the proof of Thcorcm $7.25^{\circ}$ Could the 19 proof have bcen simplified by settn $F(x)=f(x)$ on $K_{\circ}$ $F(x)=0$ off $K{\mathfrak{X}}$ for every multi- 'Showthat the yotheses of Theorem 7.5 imply that Dr is local $L^{2}$ 20 index α with $|\alpha|\leq r$ $\operatorname{Let}f\in L^{2}(R^{2})$ be th continuous function whose Fourer tantorm i Since er conclusion fe $C^{\alpha2}(R^{2})$ $$ f(y)=(1+\vert y\vert)^{-4}\langle\log{(2+\vert y\vert)}\rangle^{-1}\qquad(y\in R^{2}). $$ $|y|^{3}f(y)$ is in $L^{2}(R^{2}),$ Theorem 7. implies hat e C"(R*). Showthat the ston- is false, by proving that $$ {\frac{f(h,0)+f(-h,0)-2f(0,0)}{h^{2}}}\to-\infty{\mathrm{~as~}}h\to0. $$ 21 Suppose Prove that $\mathbf{}\!\cdot\!u$ cannot be replaced by ${\boldsymbol{R}}^{n}$ , whose first derivatives in( $\mathbf{(}1)$ 1) of Theorem 7.25. $L^{2}.$ 'Show that “locally” $L^{2}(R^{\prime}).$ This shows that $\mathbf{\sum_{d}}\mathbf{\sum_{d}}$ $\Sigma$ are functions in ${\boldsymbol{u}}$ u is a distribution in $D_{1}u,\cdot\cdot\cdot J_{n}u$ -function and an is also a function and that uis locall $L^{2}.$ canno be omitted in he concusion.) Hint: u s in fac the sum or an entire function. show that uis actully acotinuous function Show that ths stronger When $n=1,$ $n=2.$ For example, consider the function conclusion is false when $$ f(x)={\frac{|\log|x||^{1/4}}{1+|x|^{2}}}\qquad(x\in R^{2}). $$ 22 ness of ${\boldsymbol{T}}^{n};$ ": Every distribution on $T^{n}$ Se Exerie 1, Chapter B, for the same resul under weaker hypothese , have Fowrier series whose thcory Perodic distributions, or distributions on a torus $T^{n},$ is swatsmiet tht o orertnsorms. This maily ue t thcompat has compat support. I particua,,tempere distrihutions are nothing special Frove tevaious aertosmad n th flowing bsc ouline $$ T^{a}=\{(e^{i x_{1}},\cdot\cdot\cdot,e^{i x_{n}});x_{j}\,\mathrm{real}\}. $$ Functions $\phi$ on $T^{n}$ can be identified with functions ${\tilde{\phi}}\;\mathrm{on}\;R^{*}$ that are zm-periodic in cach variable, by setting $$ \tilde{\phi}(x_{1},\ldots,x_{n})=\phi(e^{i x_{1}},\ldots,e^{i x_{n}}). $$ $\mathbb{Z}^{n}$ is the set (or additive group) of n-tuples $k=(k_{1},\cdot\cdot\cdot,k_{n})$ of integers $k_{J}.$ For ke Z" the function e ${\boldsymbol{e}}_{k}$ is defined on $T^{n}$ ” by $$ e_{k}(e^{i x_{1}},\cdot\cdot\cdot,e^{i x_{n}})=e^{i k\cdot\cdot x}=\exp{\{i(k_{1}x_{1}+\cdot\cdot\cdot\cdot\cdot\mid{k_{n}x_{n}})\}}. $$ ${\boldsymbol{\sigma}}_{n}$ is the Haar measure of $T^{n}.$ If $\phi\in L^{1}(\sigma_{n}),$ the Fourier coeficients of $\phi$ are $$ \hat{\phi}(k)=\int_{{\cal T}^{n}}e_{-k}\phi\,d\sigma_{n}\qquad(k\in Z^{n}). $$ ${\mathcal{D}}(T^{n})$ is the space of all functions $\phi$ on $T^{n}$ such that $\phi\in C^{\infty}(R^{n})$ ${\bf I}\ \{\phi\in{\mathcal D}(T^{n})$ then $$ \left\{\sum_{k\mathrm{e}Z^{n}}(1+k\cdot k)^{N}|\hat{\phi}(k)|^{2}\right\}^{1/2}<\cdots $$rouR RANsrows 191 for $N=0,\;1,\;2,\;\ldots.$ Tncowsgose eresoy $\operatorname{\mathcal{D}}(T^{n}),$ which coin eides witeon ive ty te norm $$ \begin{array}{c c c}{{\displaystyle\Pi_{\mathrm{ax}}\times\mathrm{\sup}}}&{{[(D^{x}\widehat\phi)(x)]\hfill}}&{{\qquad(N=0,1,2,\ldots).}}\end{array} $$ ${\mathcal{D}}^{\prime}(T^{n})$ distributions on $T^{n}.$ yisos eo ounioinais and a ${\mathbf{C}}$ such that tsmes are t To eac ${}_{\nu}\quad\l_{m}$ The roeeceseray ${\mathcal{D}}(T^{n}).$ $u\in{\mathcal{D}}^{\prime}(T^{n})$ are dCfinc b $$ \dot{u}(k)=u(e_{-k})\qquad(k\in Z^{n}). $$ $u\in{\mathcal{D}}^{\prime}(T^{*})$ correspond an ${\boldsymbol{N}}$ ${\boldsymbol{C}}$ and $N,$ then $\scriptstyle{\theta={\hat{a}}}$ for some $u\in{\mathcal{D}}^{\prime}(T^{*})$ $$ \left|{\hat{a}}(k)\right|\geq C(1+\left|k\right|)^{\vee}\qquad(k\in Z^{n}). $$ $|g(k)|\leq C(1+|k|)^{\vee}$ for some $T^{n}{}_{,}$ Couors: g i ape ion ae “ partial sums” Tn…rs……-o…setesaouo $\mathbb{Z}^{n}$ on the other and if $u\in{\mathcal{D}}^{\prime}(T^{n}),$ the If on onpangae unionomtaon are finie setswhose unon s $\mathbb{Z}^{n},$ $E_{1}\subset E_{2}\subset E_{3}\subset\cdots$ 24 Put transforms, y replacing ${\mathbf{}}F$ $\mathrm{For~}j=1,2,3,\ldots,$ define $$ \sum_{\alpha\in J}d(k)e_{k} $$ ${\mathcal{D}}^{\prime}(T^{*}).$ is mstasine s avin are true; the 23 converge to as $j\to\alpha o_{\mathrm{\,,}}$ in the ek*-toloey o and $v\in{\mathcal{D}}^{\prime}(T^{n}):$ smo $6.37$ The convoluo ${\boldsymbol{u}}\ast v$ of ue ${\mathcal{D}}^{\prime}(T^{n})$ on the ea ine b $c=(2/\pi)^{1/2}.$ rooss6*: auo otous sesesea p lce Frour proos aremch simpie $7.25$ Modify t po f Theor y asuialeric fnicti ${\mathcal{G}}_{J}$ $H/\in L^{2},$ This s thimyom $\mathfrak{o f};$ $$ g_{j}(l)={\binom{c/t}{0}}\qquad{\mathrm{if~}}l/j<|t|<j $$ convcrges, in the $L^{\geq}$ -mtrc to function B- $\textstyle f\in L^{*}(R^{i}).$ mm…snems…soso as j→ 0.If it follows that $f\ast g_{j}$ formally, trivial.)Prove tha $$ (H f)(x)={\frac{1}{\pi}}\int_{-\infty}^{\infty}{\frac{f(t)}{x-t}}\,d t. $$ m…… $\mathbf{i}\mathbf{S}$ Snt sosesicosnsnina lessianon o ${\mathcal{X}},$ but this is not so So…sssmsmsaee…… $$ \|H f\|_{2}=\|f\|_{2}\qquad{\mathrm{and}}\qquad H(H f)=-f, $$ for every ${\mathsf{F}}\in L^{2}(R^{1}),$ Thus ${\boldsymbol{H}}$ is an $L^{2.}$ -isomety o perod $4.$ ls itretha $H f\in{\mathcal{P}}_{1}~\mathrm{if}f\in{\mathcal{P}}_{1}$ ?