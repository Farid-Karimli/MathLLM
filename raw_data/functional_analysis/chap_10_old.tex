\documentclass[10pt]{article}
\usepackage[utf8]{inputenc}
\usepackage[T1]{fontenc}
\usepackage{amsmath}
\usepackage{amsfonts}
\usepackage{amssymb}
\usepackage[version=4]{mhchem}
\usepackage{stmaryrd}
\usepackage{mathrsfs}
\usepackage{bbold}

\begin{document}
\section{BANACH ALGEBRAS}
\section{Introduction}
10.1 Definition A complex algebra is a vector space $A$ over the complex field $C$ in which a multiplication is defined that satisfies

$$
\begin{gathered}
x(y z)=(x y) z \\
(x+y) z=x z+y z, \quad x(y+z)=x y+x z
\end{gathered}
$$

and

$$
\alpha(x y)=(\alpha x) y=x(\alpha y)
$$

for all $x, y$, and $z$ in $A$ and for all scalars $\alpha$.

If, in addition, $A$ is a Banach space with respect to a norm that satisfies the multiplicative inequality

$$
\|x y\| \leq\|x\|\|y\| \quad(x \in A, y \in A)
$$

and if $A$ contains a unit element $e$ such that

$$
x e=e x=x \quad(x \in A)
$$

and

$$
\|e\|=1 \text {, }
$$

then $A$ is called a Banach algebra.

Note that we have not required that $A$ be commutative, i.e., that $x y=y x$ for all $x$ and $y$ in $A$, and we shall not do so except when explicitly stated.

It is clear that there is at most one $e \in A$ that satisfies (5), for if $e^{\prime}$ also satisfies (5), then $e^{\prime}=e^{\prime} e=e$.

The presence of a unit is very often omitted from the definition of a Banach algebra. - However, when there is a unit it makes sense to talk about inverses, so that the spectrum of an element of $A$ can be defined in a more natural way than is otherwise possible. This leads to a more intuitive development of the basic theory. Moreover, the resulting loss of generality is small, because many naturally occurring Banach algebras have a unit, and because the others can be supplied with one in the following canonical fashion.

Suppose $A$ satisfies conditions (1) to (4), but $A$ has no unit element. Let $A_{1}$ consist of all ordered pairs $(x, \alpha)$, where $x \in A$ and $\alpha \in C$. Define the vector space operations in $A_{1}$ componentwise, define multiplication in $A_{1}$ by

$$
(x, \alpha)(y, \beta)=(x y+\alpha y+\beta x, \alpha \beta)
$$

and define

$$
\|(x, \alpha)\|=\|x\|+|\alpha|, \quad e=(0,1) .
$$

Then $A_{1}$ satisfies properties (1) to (6), and the mapping $x \rightarrow(x, 0)$ is an isometric isomorphism of $A$ onto a subspace of $A_{1}$ (in fact, onto a closed two-sided ideal of $A_{1}$ ) whose codimension is 1 . If $x$ is identified with $(x, 0)$, then $A_{1}$ is simply $A$ plus the one-dimensional vector space generated by $e$. See Examples $10.3(d)$ and $11.13(e)$.

The inequality (4) makes multiplication a continuous operation in $A$. This means that if $x_{n} \rightarrow x$ and $y_{n} \rightarrow y$ then $x_{n} y_{n} \rightarrow x y$, which follows from the identity

$$
x_{n} y_{n}-x y=\left(x_{n}-x\right) y_{n}+x\left(y_{n}-y\right)
$$

In particular, multiplication is left-continuous and right-continuous:

$$
x_{n} y \rightarrow x y \quad \text { and } \quad x y_{n} \rightarrow x y
$$

if $x_{n} \rightarrow x$ and $y_{n} \rightarrow y$.

It is interesting that (4) can be replaced by the (apparently) weaker requirement (10) and that (6) can be dropped without enlarging the class of algebras under consideration.

10.2 Theorem Assume that $A$ is a Banach space as well as a complex algebra with unit element $e \neq 0$, in which multiplication is left-continuous and right-continuous.

Then there is a norm on $A$ which induces the same topology as the given one and which makes $A$ into a Banach algebra.

(The assumption $e \neq 0$ rules out the uninteresting case $A=\{0\}$.)

PROOF Assign to each $x \in A$ the left-multiplication operator $M_{x}$ defined by

$$
M_{x}(z)=x z \quad(z \in A)
$$

Let $\tilde{A}$ be the set of all $M_{x}$. Since right multiplication is assumed to be continuous, $\tilde{A} \subset \mathscr{B}(A)$, the Banach space of all bounded linear operators on $A$.

It is clear that $x \rightarrow M_{x}$ is linear. The associative law implies that $M_{x y}=$ $M_{x} M_{y}$. If $x \in A$, then

$$
\|x\|=\|x e\|=\left\|M_{x} e\right\| \leq\left\|M_{x}\right\|\|e\| .
$$

These facts can be summarized by saying that $x \rightarrow M_{x}$ is an isomorphism of $A$ onto the algebra $\tilde{A}$, whose inverse is continuous. Since

$$
\left\|M_{x} M_{y}\right\| \leq\left\|M_{x}\right\|\left\|M_{y}\right\| \quad \text { and } \quad\left\|M_{e}\right\|=\|I\|=1
$$

$\tilde{A}$ is a Banach algebra, provided it is complete, i.e., provided it is a closed subspace of $\mathscr{B}(A)$, relative to the topology given by the operator norm. (See Theorem 4.1.) Once this is done, the open mapping theorem implies that $x \rightarrow M_{x}$ is also continuous. Hence $\|x\|$ and $\left\|M_{x}\right\|$ are equivalent norms on $A$.

Suppose $T \in \mathscr{B}(A), T_{i} \in \tilde{A}$, and $T_{i} \rightarrow T$ in the topology of $\mathscr{B}(A)$. If $T_{i}$ is left multiplication by $x_{i} \in A$, then

$$
T_{i}(y)=x_{i} y=\left(x_{i} e\right) y=T_{i}(e) y
$$

As $i \rightarrow \infty$, the first term in (4) tends to $T(y)$, and $T_{i}(e) \rightarrow T(e)$. Since multiplication is assumed to be left-continuous in $A$, it follows that the last term of (4) tends to $T(e) y$. Put $x=T(e)$. Then

$$
T(y)=T(e) y=x y=M_{x}(y) \quad(y \in A)
$$

so that $T=M_{x} \in \tilde{A}$, and $\tilde{A}$ is closed.

10.3 Examples (a) Let $C(K)$ be the Banach space of all complex continuous functions on a nonempty compact Hausdorff space $K$, with the supremum norm. Define multiplication in the usual way: $(f g)(p)=f(p) g(p)$. This makes $C(K)$ into a commutative Banach algebra; the constant function 1 is the unit element.

If $K$ is a finite set, consisting of, say, $n$ points, then $C(K)$ is simply $\ell^{n}$, with coordinatewise multiplication.

In particular, when $n=1$, we obtain the simplest Banach algebra, namely $\varnothing$, with the absolute value as norm.
(b) Let $X$ be a Banach space. Then $\mathscr{B}(X)$, the algebra of all bounded linear operators on $X$, is a Banach algebra, with respect to the usual operator norm. The identity operator $I$ is its unit element. If $\operatorname{dim} X=n<\infty$, then $\mathscr{B}(X)$ is (isomorphic to) the algebra of all complex $n$-by-n matrices. If $\operatorname{dim} X>1$, then $\mathscr{B}(X)$ is not commutative. (The trivial space $X=\{0\}$ must be excluded.)

Every closed subalgebra of $\mathscr{B}(X)$ that contains $I$ is also a Banach algebra. The proof of Theorem 10.2 shows, in fact, that every Banach algebra is isomorphic to one of these.

(c) If $K$ is a nonempty compact subset of $\mathscr{C}$, or of $\mathscr{C}^{n}$, and if $A$ is the subalgebra of $C(K)$ that consists of those $f \in C(K)$ that are holomorphic in the interior of $K$, then $A$ is complete (relative to the supremum norm) and is therefore a Banach algebra.

When $K$ is the closed unit disc in $\ell$, then $A$ is called the disc algebra.

(d) $L^{1}\left(R^{n}\right)$, with convolution as multiplication, satisfies all requirements of Definition 10.1, except that it lacks a unit. One can adjoin one by the abstract procedure outlined in Section 10.1 or one can do it more concretely by enlarging $L^{1}\left(R^{n}\right)$ to the algebra of all complex Borel measures $\mu$ on $R^{n}$ of the form

$$
d \mu=f d m_{n}+\lambda d \delta
$$

where $f \in L^{1}\left(R^{n}\right), \delta$ is the Dirac measure on $R^{n}$, and $\lambda$ is a scalar.

(e) Let $M\left(R^{n}\right)$ be the algebra of all complex Borel measures on $R^{n}$, with convolution as multiplication, normed by the total variation. This is a commutative Banach algebra, with unit $\delta$, which contains $(d)$ as a closed subalgebra.

10.4 Remarks There are several reasons for restricting our attention to Banach algebras over the complex field, although real Banach algebras (whose definition should be obvious) have also been studied.

One reason is that certain elementary facts about holomorphic functions play an important role in the foundations of the subject. This may be observed in Theorems 10.9 and 10.13 and becomes even more obvious in the symbolic calculus.

Another reason-one whose implications are not quite so obvious-is that $\varnothing$ has a natural nontrivial involution (see Definition 11.14), namely, conjugation, and that many of the deeper properties of certain types of Banach algebras depend on the presence of an involution. (For the same reason, the theory of complex Hilbert spaces is richer than that of real ones.)

At one point (Theorem 10.44) a topological difference between $\ell$ and $R$ will even play a role.

Among the important mappings from one Banach algebra into another are the homomorphisms. These are linear mappings $h$ that are also multiplicative:

$$
h(x y)=h(x) h(y)
$$

Of particular interest is the case in which the range is the simplest of all Banach algebras, namely, $\mathscr{C}$ itself. Many of the significant features of the commutative theory depend crucially on a sufficient supply of homomorphisms onto $\mathscr{C}$.

\section{Complex Homomorphisms}
10.5 Definition. Suppose $A$ is a complex algebra and $\phi$ is a linear functional on $A$ which is not identically 0 . If

$$
\phi(x y)=\phi(x) \phi(y)
$$

for all $x \in A$ and $y \in A$, then $\phi$ is called a complex homomorphism on $A$.

(The exclusion of $\phi \equiv 0$ is, of course, just a matter of convenience.)

An element $x \in A$ is said to be invertible if it has an inverse in $A$, that is, if there exists an element $x^{-1} \in A$ such that

$$
x^{-1} x=x x^{-1}=e
$$

where $e$ is the unit element of $A$.

Note that no $x \in A$ has more than one inverse, for if $y x=e=x z$ then

$$
y=y e=y(x z)=(y x) z=e z=z .
$$

10.6 Proposition If $\phi$ is a complex homomorphism on a complex algebra $A$ with unit $e$, then $\phi(e)=1$, and $\phi(x) \neq 0$ for every invertible $x \in A$.

PROOF For some $y \in A, \phi(y) \neq 0$. Since

$$
\phi(y)=\phi(y e)=\phi(y) \phi(e)
$$

it follows that $\phi(e)=1$. If $x$ is invertible, then

$$
\phi(x) \phi\left(x^{-1}\right)=\phi\left(x x^{-1}\right)=\phi(e)=1
$$

so that $\phi(x) \neq 0$.

Parts $(a)$ and $(c)$ of the following theorem are perhaps the most widely used facts in the theory of Banach algebras; in particular, $(c)$ implies that all complex homomorphisms of Banach algebras are continuous.

10.7 Theorem Suppose $A$ is a Banach algebra, $x \in A,\|x\|<1$. Then

(a) $e-x$ is invertible,

(b) $\left\|(e-x)^{-1}-e-x\right\| \leq \frac{\|x\|^{2}}{1-\|x\|}$,

(c) $|\phi(x)|<1$ for every complex homomorphism $\phi$ on $A$.

PRoOF Since $\left\|x^{n}\right\| \leq\|x\|^{n}$ and $\|x\|<1$, the elements

$$
s_{n}=e+x+x^{2}+\cdots+x^{n}
$$

form a Cauchy sequence in $A$. Since $A$ is complete, there exists $s \in A$ such that $s_{n} \rightarrow s$. Since $x^{n} \rightarrow 0$ and

$$
s_{n} \cdot(e-x)=e-x^{n+1}=(e-x) \cdot s_{n},
$$

the continuity of multiplication implies that $s$ is the inverse of $e-x$. Next, (1) shows that

$$
\|s-e-x\|=\left\|x^{2}+x^{3}+\cdots\right\| \leq \sum_{n=2}^{\infty}\|x\|^{n}=\frac{\|x\|^{2}}{1-\|x\|} .
$$

Finally, suppose $\lambda \in \mathbb{C},|\lambda| \geq 1$. By $(a), e-\lambda^{-1} x$ is invertible. By Proposition 10.6,

$$
1-\lambda^{-1} \phi(x)=\phi\left(e-\lambda^{-1} x\right) \neq 0 .
$$

Hence $\phi(x) \neq \lambda$. This completes the proof.

We now interrupt the main line of development and insert a theorem which shows, for Banach algebras, that Proposition 10.6 actually characterizes the complex homomorphisms among the linear functionals. This striking result has apparently found no interesting applications as yet.

10.8 Lemma Suppose $f$ is an entire function of one complex variable, $f(0)=1$, $f^{\prime}(0)=0$, and

$$
0<|f(\lambda)| \leq e^{|\lambda|} \quad(\lambda \in \mathscr{C}) .
$$

Then $f(\lambda)=1$ for all $\lambda \in \varnothing$.

PROOF Since $f$ has no zero, there is an entire function $g$ such that $f=\exp \{g\}$, $g(0)=g^{\prime}(0)=0$, and $\operatorname{Re}[g(\lambda)] \leq|\lambda|$. This inequality implies

$$
|g(\lambda)| \leq|2 r-g(\lambda)| \quad(|\lambda| \leq r)
$$

The function

$$
h_{r}(\lambda)=\frac{r^{2} g(\lambda)}{\lambda^{2}[2 r-g(\lambda)]}
$$

is holomorphic in $\{\lambda:|\lambda|<2 r\}$, and $\left|h_{r}(\lambda)\right| \leq 1$ if $|\lambda|=r$. By the maximum modulus theorem,

$$
\left|h_{r}(\lambda)\right| \leq 1 \quad(|\lambda| \leq r)
$$

Fix $\lambda$ and let $r \rightarrow \infty .^{\vee}$ Then (3) and (4) imply that $g(\lambda)=0$.

10.9 Theorém (Gleason, Kahane, Zelazko) If $\phi$ is a linear functional on a Banach algebra $A$, such that $\phi(e)=1$ and $\phi(x) \neq 0$ for every invertible $x \in A$, then

$$
\phi(x y)=\phi(x) \phi(y) \quad(x \in A, y \in A) .
$$

Note that the continuity of $\phi$ is not part of the hypothesis.

PROOF Let $N$ be the null space of $\phi$. If $x \in A$ and $y \in A$, the assumption $\phi(e)=1$ shows that

$$
x=a+\phi(x) e, \quad y=b+\phi(y) e,
$$

where $a \in N, b \in N$. If $\phi$ is applied to the product of the equations (2), one obtains

$$
\phi(x y)=\phi(a b)+\phi(x) \phi(y) .
$$

The desired conclusion (1) is therefore equivalent to the assertion that

$$
a b \in N \quad \text { if } a \in N \text { and } b \in N \text {. }
$$

Suppose we had proved a special case of (4), namely,

$$
a^{2} \in N \quad \text { if } a \in N
$$

Then (3), with $x=y$, implies

$$
\phi\left(x^{2}\right)=[\phi(x)]^{2} \quad(x \in A) .
$$

Replacement of $x$ by $x+y$ in (6) results in

$$
\dot{\phi}(x y+y x)=2 \phi(x) \phi(y) \quad(x \in A, y \in A)
$$

Hence

$$
x y+y x \in N \quad \text { if } x \in N, y \in A \text {. }
$$

Consider the identity

$$
(x y-y x)^{2}+(x y+y x)^{2}=2[x(y x y)+(y x y) x] .
$$

If $x \in N$, the right side of (9) is in $N$, by (8), and so is $(x y+y x)^{2}$, by (8) and (6). Hence $(x y-y x)^{2}$ is in $N$, and another application of (6) yields

$$
x y-y x \in N \quad \text { if } x \in N, y \in A \text {. }
$$

Addition of (8) and (10) gives (4), hence (1).

Thus (5) implies (1), for purely algebraic reasons. The proof of (5) uses analytic methods.

By hypothesis, $N$ contains no invertible element of $A$. Thus $\|e-x\| \geq 1$ for every $x \in N$, by $(a)$ of Theorem 10.7. Hence

$$
\|\lambda e-x\| \geq|\lambda|=|\dot{\phi}(\lambda e-x)| \quad(x \in N, \lambda \in \mathscr{C}) .
$$

We conclude that $\phi$ is a continuous linear functional on $A$, of norm 1 .

To prove (5), fix $a \in N$, assume $\|a\|=1$ without loss of generality, and define

$$
f(\lambda)=\sum_{n=0}^{\infty} \frac{\phi\left(a^{n}\right)}{n !} \lambda^{n} \quad(\lambda \in \mathscr{C})
$$

Since $\left|\phi\left(a^{n}\right)\right| \leq\left\|a^{n}\right\| \leq\|a\|^{n}=1, f$ is entire and satisfies $|f(\lambda)| \leq \exp |\lambda|$ for all $\lambda \in \ell$. Also, $f(0)=\phi(e)=1$, and $f^{\prime}(0)=\phi(a)=0$.

If we can prove that $f(\lambda) \neq 0$ for every $\lambda \in \mathscr{C}$, Lemma 10.8 will imply that $f^{\prime \prime}(0)=0$; hence $\phi\left(a^{2}\right)=0$, which proves (5).

The series

$$
E(\lambda)=\sum_{n=0}^{\infty} \frac{\lambda^{n}}{n !} a^{n}
$$

converges in the norm of $A$, for every $\lambda \in \mathscr{C}$. The continuity of $\phi$ shows that

$$
f(\lambda)=\phi(E(\lambda)) \quad(\lambda \in \mathbb{C})
$$

The functional equation $E(\lambda+\mu)=E(\lambda) E(\mu)$ follows from (13) exactly as in the scalar case. In particular,

$$
E(\lambda) E(-\lambda)=E(0)=e \quad(\lambda \in \mathscr{C}) .
$$

Hence $E(\lambda)$ is an invertible element of $A$, for every $\lambda \in \ell$. This implies, by hypothesis, that $\phi(E(\lambda)) \neq 0$, and therefore $f(\lambda) \neq 0$, by (14). This completes the proof.

\section{Basic Properties of Spectra}
10.10 Definitions Let $A$ be a Banach algebra; let $G=G(A)$ be the set of all invertible elcments of $A$. If $x \in G$ and $y \in G$, then $y^{-1} x$ is the inverse of $x^{-1} y$; thus $x^{-1} y \in G$, and $G$ is a group.

If $x \in A$, the spectrum $\sigma(x)$ of $x$ is the set of all complex numbers $\lambda$ such that $\lambda e-x$ is not invertible. The complement of $\sigma(x)$ is the resolvent set of $x$; it consists of all $\lambda \in \mathscr{C}$ for which $(\lambda e-x)^{-1}$ exists.

The spectral radius of $x$ is the number

$$
\rho(x)=\sup \{|\lambda|: \lambda \in \sigma(x)\}
$$

It is the radius of the smallest closed circular disc in $\mathscr{C}$, with center at 0 , which contains $\sigma(x)$. Of course, (1) makes no sense if $\sigma(x)$ is empty. But this never happens, as we shall see.

10.11 Theorem Suppose $A$ is a Banach algebra, $x \in G(A), h \in A,\|h\|<\frac{1}{2}\left\|x^{-1}\right\|^{-1}$.
Then $x+h \in G(A)$, and

$$
\left\|(x+h)^{-1}-x^{-1}+x^{-1} h x^{-1}\right\| \leq 2\left\|x^{-1}\right\|^{3}\|h\|^{2} .
$$

PROOF Since $x+h=x\left(e+x^{-1} h\right)$ and $\left\|x^{-1} h\right\|<\frac{1}{2}$, Theorem 10.7 implies that $x+h \in G(A)$ and that the norm of the right member of the identity

$$
(x+h)^{-1}-x^{-1}+x^{-1} h x^{-1}=\left[\left(e+x^{-1} h\right)^{-1}-e+x^{-1} h\right] x^{-1}
$$

is at most $2\left\|x^{-1} h\right\|^{2}\left\|x^{-1}\right\|$.

10.12 Theorem If $A$ is a Banach algebra, then $G(A)$ is an open subset of $A$, and the mapping $x \rightarrow x^{-1}$ is a homeomorphism of $G(A)$ onto $G(A)$.

PROOF That $G(A)$ is open and that $x \rightarrow x^{-1}$ is continuous follows from Theorem 10.11. Since $x \rightarrow x^{-1}$ maps $G(A)$ onto $G(A)$ and since it is its own inverse, it is a homeomorphism.

\subsection{Theorem If $A$ is a Banach algebra and $x \in A$, then}
(a) the spectrum $\sigma(x)$ of $x$ is compact and nonempty, and

(b) the spectral radius $\rho(x)$ of $x$ satisfies

$$
\rho(x)=\lim _{n \rightarrow \infty}\left\|x^{n}\right\|^{1 / n}=\inf _{n \geq 1}\left\|x^{n}\right\|^{1 / n}
$$

inequality

Note that the existence of the limit in (1) is part of the conclusion and that the

$$
\rho(x) \leq\|x\|
$$

is contained in the spectral radius formula (1).

PROOF If $|\lambda|>\|x\|$ then $e-\lambda^{-1} x$ lies in $G(A)$, by Theorem 10.7, and so does $\lambda e-x$. Thus $\lambda \notin \sigma(x)$. This proves (2). In particular, $\sigma(x)$ is a bounded set.

To prove that $\sigma(x)$ is closed, define $g: \varnothing \rightarrow A$ by $g(\lambda)=\lambda e-x$. Then $g$ is continuous, and the complement $\Omega$ of $\sigma(x)$ is $g^{-1}(G(A))$, which is open, by Theorem 10.12. Thus $\sigma(x)$ is compact.

Now define $f: \Omega \rightarrow G(A)$ by

$$
f(\lambda)=(\lambda e-x)^{-1} \quad(\lambda \in \Omega) .
$$

Replace $x$ by $\lambda e-x$ and $h$ by $(\mu-\lambda) e$ in Theorem 10.11. If $\lambda \in \Omega$ and $\mu$ is sufficiently close to $\lambda$, the result of this substitution is

$$
\left\|f(\mu)-f(\lambda)+(\mu-\lambda) f^{2}(\lambda)\right\| \leq 2\|f(\lambda)\|^{3}|\mu-\lambda|^{2}
$$

so that

$$
\lim _{\mu \rightarrow \lambda} \frac{f(\mu)-f(\lambda)}{\mu-\lambda}=-f^{2}(\lambda) \quad(\lambda \in \Omega)
$$

Thus $f$ is a strongly holomorphic $A$-valued function in $\Omega$.

If $|\lambda|>\|x\|$, the argument used in Theorem 10.7 shows that

$$
f(\lambda)=\sum_{n=0}^{\infty} \lambda^{-n-1} x^{n}=\lambda^{-1} e+\lambda^{-2} x+\cdots
$$

This series converges uniformly on every circle $\Gamma_{r}$ with center at 0 and radius $r>\|x\|$. By Theorem 3.29, term-by-term integration is therefore legitimate. Hence

$$
x^{n}=\frac{1}{2 \pi i} \int_{\Gamma_{r}} \lambda^{n} f(\lambda) d \lambda \quad(r>\|x\|, n=0,1,2, \ldots)
$$

If $\sigma(x)$ were empty, $\Omega$ would be $\mathscr{C}$, and the Cauchy theorem 3.31 would imply that all integrals in (7) are 0 . But when $n=0$, the left-hand side of (7) is $e \neq 0$. This contradiction shows that $\sigma(x)$ is not empty.

Since $\Omega$ contains all $\lambda$ with $|\lambda|>\rho(x)$, an application of (3) of the Cauchy theorem 3.31 shows that the condition $r>\|x\|$ can be replaced in (7) by $r>\rho(x)$. If

$$
M(r)=\max _{\theta}\left\|f\left(r e^{i \theta}\right)\right\| \quad(r>\rho(x))
$$

the continuity of $f$ implies that $M(r)<\infty$. Since (7) now gives

$$
\left\|x^{n}\right\| \leq r^{n+1} M(r)
$$

we obtain

$$
\limsup _{n \rightarrow \infty}\left\|x^{n}\right\|^{1 / n} \leq r \quad(r>\rho(x))
$$

so that

$$
\limsup _{n \rightarrow \infty}\left\|x^{n}\right\|^{1 / n} \leq \rho(x)
$$

On the other hand, if $\lambda \in \sigma(x)$, the factorization

$$
\lambda^{n} e-x^{n}=(\lambda e-x)\left(\lambda^{n-1} e+\cdots+x^{n-1}\right)
$$

shows that $\lambda^{n} e-x^{n}$ is not invertible. Thus $\lambda^{n} \in \sigma\left(x^{n}\right)$. By (2), $\left|\lambda^{n}\right| \leq\left\|x^{n}\right\|$ for $n=1,2,3, \ldots$. Hence

$$
\rho(x) \leq \inf _{n \geq 1}\left\|x^{n}\right\|^{1 / n}
$$

and (I) is an immediate consequence of (11) and (13).

The nonemptiness of $\sigma(x)$ leads to an easy characterization of those Banach algebras that are division algebras.

10.14 Theorem (Gelfand-Mazur) If $A$ is a Banach algebra in which every nonzero element is invertible, then $A$ is (isometrically isomorphic to) the complex field. PROOF If $x \in A$ and $\lambda_{1} \neq \lambda_{2}$, then at most one of the elements $\lambda_{1} e-x$ and $\lambda_{2} e-x$ is 0 ; hence at least one of them is invertible. Since $\sigma(x)$ is not empty, it follows that $\sigma(x)$ consists of exactly one point, say $\lambda(x)$, for each $x \in A$. Since $\lambda(x) e-x$ is not invertible, it is 0 . Hence $x=\lambda(x) e$. The mapping $x \rightarrow \lambda(x)$ is therefore an isomorphism of $A$ onto $\mathscr{C}$, which is also an isometry, since $|\lambda(x)|=$ $\|\lambda(x) e\|=\|x\|$ for every $x \in A$.

Theorems 10.13 and 10.14 are among the key results of this chapter. Much of the content of Chapters 11 to 13 is independent of the remainder of Chapter 10.

10.15 Remarks (a) Whether an element of $A$ is or is not invertible in $A$ is a purely algcbraic property. The spectrum and the spectral radius of an $x \in A$ are thus defined in terms of the algebraic structure of $A$, regardless of any metric (or topological) considerations. On the other hand, $\lim \left\|x^{n}\right\|^{1 / n}$ depends obviously on metric properties of $A$. This is one of the remarkable features of the spectral radius formula: It asserts the equality of certain quantities which arise in entirely different ways.

(b) Our algebra $A$ may be a subalgebra of a larger Banach algebra $B$, and it may then very well happen that some $x \in A$ is not invertible in $A$ but is invertible in $B$. The spectrum of $x$ depends therefore on the algebra. The inclusion $\sigma_{A}(x) \supset \sigma_{B}(x)$ holds (the notation is self-explanatory); the two spectra can be different. The spectral radius is, however, unaffected by the passage from $A$ to $B$, since the spectral radius formula expresses it in terms of metric properties of powers of $x$, and these are independent of anything that happens outside $A$. detail.

Theorem 10.18 will describe the relation between $\sigma_{A}(x)$ and $\sigma_{B}(x)$ in greater

10.16 Lemma Suppose $V$ and $W$ are open sets in some topological space $X$, $V \subset W$, and $W$ contains no boundary point of $V$. Then $V$ is a union of components of $W$.

Recall that a component of $W$ is, by definition, a maximal connected subset of $W$.

PROOF Let $\Omega$ be a component of $W$ that intersects $V$. Let $U$ be the complement of $\bar{V}$. Since $W$ contains no boundary point of $V, \Omega$ is the union of the two disjoint open sets $\Omega \cap V$ and $\Omega \cap U$. Since $\Omega$ is connected, $\Omega \cap U$ is empty. Thus $\Omega \subset V$.

10.17 Lemma Suppose $A$ is a Banach algebra, $x_{n} \in G(A)$ for $n=1,2,3, \ldots, x$ is a boundary point of $G(A)$, and $x_{n} \rightarrow x$ as $n \rightarrow \infty$.

Then $\left\|x_{n}^{-1}\right\| \rightarrow \infty$ as $n \rightarrow \infty$.

PRoOF If the conclusion is false, there exists $M<\infty$ such that $\left\|x_{n}^{-1}\right\|<M$ for infinitely many $n$. For one of these, $\left\|x_{n}-x\right\|<1 / M$. For this $n$,

$$
\left\|e-x_{n}^{-1} x\right\|=\left\|x_{n}^{-1}\left(x_{n}-x\right)\right\|<1
$$

so that $x_{n}^{-1} x \in G(A)$. Since $x=x_{n}\left(x_{n}^{-1} x\right)$ and $G(A)$ is a group, it follows that $x \in G(A)$. This contradicts the hypothesis, since $G(A)$ is open.

\subsection{Theorem}
(a) If $A$ is a closed subalgebra of a Banach algebra $B$, and if $A$ coniains the unit element of $B$, then $G(A)$ is a union of components of $A \cap G(B)$.

(b) Under these conditions, if $x \in A$, then $\sigma_{A}(x)$ is the union of $\sigma_{B}(x)$ and a (possibly empty) collection of bounded components of the complement of $\sigma_{B}(x)$. In particular, the boundary of $\sigma_{A}(x)$ lies in $\sigma_{B}(x)$.

PROOF (a) Every member of $A$ that has an inverse in $A$ has the same inverse in $B$. Thus $G(A) \subset G(B)$. Both $G(A)$ and $A \cap G(B)$ are open subsets of $A$. By Lemma 10.16 , it is sufficient to prove that $G(B)$ contains no boundary point $\dot{y}$ of $G(A)$.

Any such $y$ is the limit of a sequence $\left\{x_{n}\right\}$ in $G(A)$. By Lemma 10.17, $\left\|x_{n}^{-1}\right\| \rightarrow \infty$. If $y$ were in $G(B)$, the continuity of inversion in $G(B)$ (Theorem 10.12) would force $x_{n}^{-1}$ to converge to $y^{-1}$. In particular $\left\{\left\|x_{n}^{-1}\right\|\right\}$ would be bounded. Hence $y \notin G(B)$, and $(a)$ is proved.

(b) Let $\Omega_{A}$ and $\Omega_{B}$ be the complements of $\sigma_{A}(x)$ and of $\sigma_{B}(x)$, relative to $\ell$. The inclusion $\Omega_{A} \subset \Omega_{B}$ is obvious, since $\lambda \in \Omega_{A}$ if and only if $\lambda e-x \in$ $G(A)$. Let $\lambda_{0}$ be a boundary point of $\Omega_{A}$. Then $\lambda_{0} e-x$ is a boundary point of $G(A)$. By $(a), \lambda_{0} e-x \notin G(B)$. Hence $\lambda_{0} \notin \Omega_{B}$. Lemma 10.16 implies now that $\Omega_{A}$ is the union of certain components of $\Omega_{B}$. The other components of $\Omega_{B}$ are therefore subsets of $\sigma_{A}(x)$. This proves $(b)$.

Corollary If $\sigma_{B}(x)$ does not separate $\mathcal{C}$, that is, if its complement $\Omega_{B}$ is connected, then $\sigma_{A}(x)=\sigma_{B}(x)$.

For then $\Omega_{b}$ has no bounded components.

The most important application of this corollary occurs when $\sigma_{B}(x)$ contains only real numbers.

As another application of Lemma 10.17 we now prove a theorem whose conclusion is the same as that of the Gelfand-Mazur theorem, although its consequences are not nearly so important.

10.19 Theorem If $A$ is a Banach algebra and if there exists $M<\infty$ such that

$$
\|\dot{x}\|\|y\| \leq M\|x y\| \quad(x \in A, y \in A),
$$

then $A$ is (isometrically isomorphic to) $C$.

PROOF Let $y$ be a boundary point of $G(A)$. Then $y=\lim y_{n}$ for some sequence $\left\{y_{n}\right\}$ in $G(A)$. By Lemma 10.17, $\left\|y_{n}^{-1}\right\| \rightarrow \infty$. By hypothesis,

$$
\left\|y_{n}\right\|\left\|y_{n}^{-1}\right\| \leqslant M\|e\| \quad(n=1,2,3, \ldots)
$$

Hence $\left\|y_{n}\right\| \rightarrow 0$ and therefore $y=0$.

If $x \in A$, each boundary point $\lambda$ of $\sigma(x)$ gives rise to a boundary point $\lambda e-x$ of $G(A)$. Thus $x=\lambda e$. In other words, $A=\{\lambda e: \lambda \in \mathscr{C})$.

IIII

It is natural to ask whether the spectra of two elements $x$ and $y$ of $A$ are close together, in some suitably defined sense, if $x$ and $y$ are close to each other. The next theorem gives a very simple answer.

10.20 Theorem Suppose $A$ is a Banach algebra, $x \in A, \Omega$ is an open set in $\bar{C}$, and $\sigma(x) \subset \Omega$. Then there exists $\delta>0$ such that $\sigma(x+y) \subset \Omega$ for every $y \in A$ with $\|y\|<\delta$. PROOF Since $\left\|(\lambda e-x)^{-1}\right\|$ is a continuous function of $\lambda$ in the complement of $\sigma(x)$, and since this norm tends to 0 as $\lambda \rightarrow \infty$, there is a number $M<\infty$ such that

$$
\left\|(\lambda e-x)^{-1}\right\|<M
$$

for all $\lambda$ outside $\Omega$. If $y \in A,\|y\|<1 / M$, and $\lambda \notin \Omega$, it follows that

$$
\lambda e-(x+y)=(\lambda e-x)\left[e-(\lambda e-x)^{-1} y\right]
$$

is invertible in $A$, since $\left\|(\lambda e-x)^{-1} y\right\|<1$; hence $\lambda \notin \sigma(x+y)$. This gives the desired conclusion, with $\delta=1 / M$.

\section{Symbolic Calculus}
10.21 Introduction If $x$ is an element of a Banach algebra $A$ and if $f(\lambda)=$ $\alpha_{0}+\cdots+\alpha_{n} \lambda^{n}$ is a polynomial with complex coefficients $\alpha_{i}$, there can be no doubt about the meaning of the symbol $f(x)$; it obviously denotes the element of $A$ defined by

$$
f(x)=\alpha_{0} e+\alpha_{1} x+\cdots+\alpha_{n} x^{n} .
$$

The question arises whether $f(x)$ can be defined in a meaningful way for other functions $f$. We have already encountered some examples of this. For instance, during the proof of Theorem 10.9 we came very close to defining the exponential function in $A$. In fact, if $f(\lambda)=\sum \alpha_{k} \lambda^{k}$ is any entire function in $\ell$, it is natural to define $f(x) \in A$ by $f(x)=\sum \alpha_{k} x^{k}$; this series always converges. Another example is given by the meromorphic functions

$$
f(\lambda)=\frac{1}{\alpha-\lambda}
$$

In this case, the natural definition of $f(x)$ is

$$
f(x)=(\alpha e-x)^{-1}
$$

which makes sense for all $x$ whose spectrum does not contain $\alpha$.

One is thus led to the conjecture that $f(x)$ should be definable, within $A$, whenever $f$ is holomorphic in an open set that contains $\sigma(x)$. This turns out to be correct and can be accomplished by a version of the Cauchy formula that converts complex functions defined in open subsets of $C$ to $A$-valued ones defined in certain open subsets of $A$. (Just as in classical analysis, the Cauchy formula is a much more adaptable tool than the power series representation.) Moreover, the entities $f(x)$ so defined (see Definition 10.26) turn out to have interesting properties. The most important of these are summarized in Theorems 10.27 to 10.29 .

In certain algebras one can go further. For instance, if $x$ is a bounded nomal operator on a Hilbert space $H$, the symbol $f(x)$ can be interpreted as a bounded normal operator on $H$ when $f$ is any continuous complex function on $\sigma(x)$, and even when $f$ is any complex bounded Borel function on $\sigma(x)$. In Chapter 12 we shall see how this leads to an efficient proof of a very general form of the spectral theorem.

10.22 Integration of A-valued functions If $A$ is a Banach algebra and $f$ is a continuous $A$-valued function on some compact Hausdorff space $Q$ on which a complex Borel measure $\mu$ is defined, then $\int f d \mu$ exists and has all the properties that were discussed in Chapter 3 , simply because $A$ is a Banach space. However, an addi-
tional property can be added to these and will be used in the sequel, namely: If $x \in A$,
then

and

$$
x \int_{\underline{Q}} f d \mu=\int_{\underline{Q}} x f(p) d \mu(p)
$$

$$
\left(\int_{Q} f d \mu\right) x=\int_{Q} f(p) x d \mu(p)
$$

To prove (1), let $M_{x}$ be left multiplication by $x$, as in the proof of Theorem 10.2 , and let $\Lambda$ be a bounded linear functional on $A$. Then $\Lambda \bar{M}_{x}$ is a bounded linear functional. Definition 3.26 implies therefore that

$$
\Lambda M_{x} \int_{Q} f d \mu=\int_{Q}\left(\Lambda M_{x} f\right) d \mu=\Lambda \int_{Q}\left(M_{x} f\right) d \mu
$$

for every $\Lambda$, so that

$$
M_{x} \int_{Q} f d \mu=\int_{Q}\left(M_{x} f\right) d \mu
$$

which is just another way of writing (1). To prove (2), interpret $M_{x}$ to be right multiplication by $x$.

10.23 Contours Suppose $K$ is a compact subset of an open $\Omega \subset \varnothing$, and $\Gamma$ is a collection of finitely many oriented line intervals $\gamma_{1}, \ldots, \gamma_{n}$ in $\Omega$, none of which intersects $K$. In this situation, integration over $\Gamma$ is defined by

$$
\int_{\Gamma} \phi(\lambda) d \lambda=\sum_{j=1}^{n} \int_{\gamma_{j}} \phi(\lambda) d \lambda
$$

It is well known that $\Gamma$ can be so chosen that

$$
\operatorname{Ind}_{\Gamma}(\zeta)=\frac{1}{2 \pi i} \int_{\Gamma} \frac{d \lambda}{\lambda-\zeta}= \begin{cases}1 & \text { if } \zeta \in K \\ 0 & \text { if } \zeta \notin \Omega\end{cases}
$$

and that the Cauchy formula

$$
f(\zeta)=\frac{1}{2 \pi i} \int_{\Gamma}(\lambda-\zeta)^{-1} f(\lambda) d \lambda
$$

then holds for every holomorphic function $f$ in $\Omega$ and for every $\zeta \in K$. See, for instance, Theorem 13.5 of [23]. $K$ in $\Omega$.

We shall describe the situation (2) briefly by saying that the contour $\Gamma$ surrounds

Note that neither $K$ nor $\Omega$ nor the union of the intervals $\gamma_{i}$ has been assumed to be connected.

10.24 Lemma Suppose $A$ is a Banach algebra, $x \in A, \alpha \in \mathscr{C}, \alpha \notin \sigma(x), \Omega$ is the complement of $\alpha$ in $\ell$, and $\Gamma$ surrounds $\sigma(x)$ in $\Omega$. Then

$$
\frac{1}{2 \pi i} \int_{\Gamma}(\alpha-\lambda)^{n}(\lambda e-x)^{-1} d \lambda=(\alpha e-x)^{n} \quad(n=0, \pm 1, \pm 2, \ldots)
$$

PROOF Denote the integral by $y_{n}$. When $\lambda \notin \sigma(x)$, then

$$
(\lambda e-x)^{-1}=(\alpha e-x)^{-1}+(\alpha-\lambda)(\alpha e-x)^{-1}(\lambda e-x)^{-1}
$$

By Section 10.22, $y_{n}$ is therefore the sum of

$$
(\alpha e-x)^{-1} \cdot \frac{1}{2 \pi i} \int_{\Gamma}(\alpha-\lambda)^{n} d \lambda=0
$$

since $\operatorname{Ind}_{\Gamma}(\alpha)=0$, and

$$
(\alpha e-x)^{-1} \cdot \frac{1}{2 \pi i} \int_{\Gamma}(\alpha-\lambda)^{n+1}(\lambda e-x)^{-1} d \lambda
$$

Hence

$$
(\alpha e-x) y_{n}=y_{n+1} \quad(n=0, \pm 1, \pm 2, \ldots)
$$

This recursion formula shows that (1) follows from the case $n=0$. We thus have to prove that

$$
\frac{1}{2 \pi i} \int_{\Gamma}(\lambda e-x)^{-1} d \lambda=e
$$

Let $\Gamma_{r}$ be a positively oriented circle, centered at 0 , with radius $r>\|x\|$. On $\Gamma_{r},(\lambda e-x)^{-1}=\sum \lambda^{-n-1} x^{n}$. Termwise integration of this series gives (5), with $\Gamma_{r}$ in place of $\Gamma$. Since the integrand in (5) is a holomorphic $A$-valued function in the complement of $\sigma(x)$ (see the proof of Theorem 10.13), and since

$$
\operatorname{Ind}_{\Gamma_{\mathbf{r}}}(\zeta)=1=\operatorname{Ind}_{\Gamma}(\zeta)
$$

for every $\zeta \in \sigma(x)$, the Cauchy theorem 3.31 shows that the integral (5) is unaffected if $\Gamma$ is replaced by $\Gamma_{r}$. This completes the proof.

\subsection{Theorem Suppose}
$$
R(\lambda)=P(\lambda)+\sum_{m, k} c_{m, k}\left(\lambda-\alpha_{m}\right)^{-k}
$$

is a rational function with poles at the points $\alpha_{m} .[P$ is a polynomial, and the sum in (1) has only finitely many terms.] If $x \in A$ and if $\sigma(x)$ contains no pole of $R$, define

$$
R(x)=P(x)+\sum_{m, k} c_{m, k}\left(x-\alpha_{m} e\right)^{-k}
$$

If $\Omega$ is an open set in $C$ that contains $\sigma(x)$ and in which $R$ is holomorphic, and if $\Gamma$, surrounds $\sigma(x)$ in $\Omega$, then

$$
R(x)=\frac{1}{2 \pi i} \int_{\Gamma} R(\lambda)(\lambda e-x)^{-1} d \lambda
$$

PROOF Apply Lemma 10.24.

Note that (2) is certainly the most natural definition of a rational function of $x \in A$. The conclusion (3) shows that the Cauchy formula achieves the same result. This motivates the following definition.

10.26 Definition Suppose $A$ is a Banach algebra, $\Omega$ is an open set in $\ell$, and $H(\Omega)$ is the algebra of all complex holomorphic functions in $\Omega$. By Theorem 10.20,

$$
A_{\Omega}=\{x \in A: \sigma(x) \subset \Omega\}
$$

is an open subset of $A$.

We define $\tilde{H}\left(A_{\Omega}\right)$ to be the set of all $A$-valued functions $\tilde{f}$, with domain $A_{\Omega}$, that arise from an $f \in H(\Omega)$ by the formula

$$
\tilde{f}(x)=\frac{1}{2 \pi i} \int_{\Gamma} f(\lambda)(\lambda e-x)^{-1} d \lambda
$$

where $\Gamma$ is any contour that surrounds $\sigma(x)$ in $\Omega$.

This definition calls for some comments.

(a) Since $\Gamma$ stays away from $\sigma(x)$ and since inversion is continuous in $A$, the integrand is continuous in (2), so that the integral exists and defines $f^{f}(x)$ as an element of $A$.

(b) The integrand is actually a holomorphic $A$-valued function in the complement of $\sigma(x)$. (This was observed in the proof of Theorem 10.13. See Exercise 3.) The Cauchy theorem 3.31 implies therefore that $f(x)$ is independent of the choice of $\Gamma$, provided only that $\Gamma$ surrounds $\sigma(x)$ in $\Omega$.

(c) If $x=\alpha e$ and $\alpha \in \Omega$, (2) becomes

$$
\because \tilde{f}(\alpha e)=f(\alpha) e .
$$

Note that $\alpha e \in A_{\Omega}$ if and only if $\alpha \in \Omega$. If we identify $\lambda \in \mathscr{C}$ with $\lambda e \in A$, every $f \in H(\Omega)$ may be regarded as mapping a certain subset of $A_{\Omega}$ (namely, the intersection of $A_{\Omega}$ with the one-dimensional subspace of $A$ generated by $e$ ) into $A$, and then (3) shows that $\tilde{f}$ may be regarded as an extension of $f$.

In most treatments of this topic, $f(x)$ is written in place of our $\tilde{f}(x)$. The notation $\tilde{f}$ is used here because it avoids certain ambiguities that might cause misunderstandings.
(d) If $S$ is any set and $A$ is any algebra, the collection of all $A$-valued functions on $S$ is an algebra, if scalar multiplication, addition, and multiplication are defined pointwise. For instance, if $u$ and $v$ map $S$ in to $A$, then

$$
(u v)(s)=u(s) v(s) \quad(s \in S)
$$

This will be applied to $A$-valued functions defined in $A_{\Omega}$.

10.27 Theorem Suppose $A, H(\Omega)$, and $\tilde{H}\left(A_{\Omega}\right)$ are as in Definition 10.26. Then $\tilde{H}\left(A_{\Omega}\right)$ is a complex algebra. The mapping $f \rightarrow \widetilde{f}$ is an algebra isomorphism of $H(\Omega)$ onto $\widetilde{H}\left(A_{\Omega}\right)$ which is continuous in the following sense:

If $f_{n} \in H(\Omega)(n=1,2,3, \ldots)$ and $f_{n} \rightarrow f$ uniformly on compact subsets of $\Omega$, then

$$
\tilde{f}(x)=\lim _{n \rightarrow \infty} \tilde{f}_{n}(x) \quad\left(x \in A_{\Omega}\right)
$$

If $u(\lambda)=\lambda$ and $v(\lambda)=1$ in $\Omega$, then $\tilde{u}(x)=x$ and $\tilde{v}(x)=e$ for every $x \in A_{\Omega}$.

PROOF The last sentence follows from Theorem 10.25. The integral representation (2) in Section 10.26 makes it obvious that $f \rightarrow \tilde{f}$ is linear. If $\tilde{f}=0$, then

$$
f(\alpha) e=\tilde{f}(\alpha e)=0 \quad(\alpha \in \Omega)
$$

so that $f=0$. Thus $f \rightarrow \tilde{f}$ is one-to-one.

The asserted continuity follows directly from the integral (2) in Section 10.26 , since $\left\|(\lambda e-x)^{-1}\right\|$ is bounded on $\Gamma$. (Use the same $\Gamma$ for all $f_{n}$, and apply Theorem 3.29.)

It remains to be proved that $f \rightarrow \tilde{f}$ is multiplicative. Explicitly, if $f \in H(\Omega)$, $g \in H(\Omega)$, and $h(\lambda)=f(\lambda) g(\lambda)$ for all $\lambda \in \Omega$, it has to be shown that

$$
\tilde{h}(x)=\tilde{f}(x) \tilde{g}(x) \quad\left(x \in A_{\Omega}\right) .
$$

If $f$ and $g$ are rational functions without poles in $\Omega$, and if $h=f g$, then $h(x)=f(x) g(x)$, and since Theorem 10.25 asserts that $R(x)=\widetilde{R}(x),(3)$ holds. In the general case, Runge's theorem (Th. 13.9 of [23]) allows us to approximate $f$ and $g$ by rational functions $f_{n}$ and $g_{n}$, uniformly on compact subsets of $\Omega$. Then $f_{n} g_{n}$ converges to $h$ in the same manner, and (3) follows from the continuity of the mapping $f \rightarrow \tilde{f}$.

Note that $\widetilde{H}\left(A_{\Omega}\right)$ is a commutative algebra, since $H(\Omega)$ is obviously commutative.

10.28 Theorem Suppose $x \in A_{\Omega}$ and $f \in H(\Omega)$.

(a) $\tilde{f}(x)$ is invertible in $A$ if and only if $f(\lambda) \neq 0$ for every $\lambda \in \sigma(x)$.

(b) $\sigma(\tilde{f}(x))=f(\sigma(x))$.

Part $(b)$ is called the spectral mapping theorem.

PROOF (a) If $f$ has no zero on $\sigma(x)$, then $g=1 / f$ is holomorphic in an open set $\Omega_{1}$ such that $\sigma(x) \subset \Omega_{1} \subset \Omega$. Since $f g=1$ in $\Omega_{1}$, Theorem 10.27 (with $\Omega_{1}$ in place of $\Omega$ ) shows that $\tilde{f}(x) \tilde{g}(x)=e$, and thus $\tilde{f}(x)$ is invertible. Conversely, if $f(\alpha)=0$ for some $\alpha \in \sigma(x)$ then there exists $h \in H(\Omega)$ such that

$$
(\lambda-\alpha) h(\lambda)=f(\lambda) \quad(\lambda \in \Omega),
$$

which implies

$$
(x-\alpha e) \tilde{h}(x)=\tilde{f}(x)=\tilde{h}(x)(x-\alpha e)
$$

by Theorem 10.27. Since $x-\alpha e$ is not invertible in $A$, neither is $\tilde{f}(x)$, by (2).

(b) Fix $\beta \in C$. By definition, $\beta \in \sigma(\tilde{f}(x))$ if and only if $\tilde{f}(x)-\beta e$ is not invertible in $A$. By $(a)$, applied to $f-\beta$ is place of $f$, this happens if and only if $f-\beta$ has a zero in $\sigma(x)$, that is, if and only if $\beta \in f(\sigma(x))$.

The spectral mapping theorem makes it possible to include composition of functions among the operations of the symbolic calculus.

10.29 Theorem Suppose $x \in A_{\Omega}, f \in H(\Omega), \Omega_{1}$ is an open set containing $f(\sigma(x))$, $g \in H\left(\Omega_{1}\right)$, and $h(\lambda)=g(f(\lambda))$ in $\Omega_{0}$, the set of all $\lambda \in \Omega$ with $f(\lambda) \in \Omega_{1}$.

Then $\tilde{f}(x) \in A_{\Omega_{1}}$ and $\tilde{h}(x)=\tilde{g}(\tilde{f}(x))$.

Briefly, $\tilde{h}=\tilde{g} \circ \tilde{f}$ if $h=g \circ f$.

PROOF By $(b)$ of Theorem 10.28, $\sigma(\tilde{f}(x)) \subset \Omega_{1}$, and therefore $\tilde{g}(\tilde{f}(x))$ is defined.

Fix a contour $\Gamma_{1}$ that surrounds $f(\sigma(x))$ in $\Omega_{1}$. There is an open set $W$, with $\sigma(x) \subset W \subset \Omega_{0}$, so small that

$$
\operatorname{Ind}_{\Gamma_{1}}(f(\lambda))=1 \quad(\lambda \in W) .
$$

Fix a contour $\Gamma_{0}$ that surrounds $\sigma(x)$ in $W$. If $\zeta \in \Gamma_{1}$, then $1 /(\zeta-f) \in H(W)$. Hence Theorem 10.27, with $W$ in place of $\Omega$, shows that

$$
[\zeta e-\tilde{f}(x)]^{-1}=\frac{1}{2 \pi i} \int_{\Gamma_{0}}[\zeta-f(\lambda)]^{-1}(\lambda e-x)^{-1} d \lambda \quad\left(\zeta \in \Gamma_{1}\right)
$$

Since $\Gamma_{1}$ surrounds $\sigma(\tilde{f}(x))$ in $\Omega_{1}$, (1) and (2) imply

$$
\begin{aligned}
\tilde{g}(\tilde{f}(x)) & =\frac{1}{2 \pi i} \int_{\Gamma_{1}} g(\zeta)[\zeta e-\tilde{f}(x)]^{-1} d \zeta \\
& =\frac{1}{2 \pi i} \int_{\Gamma_{0}} \frac{1}{2 \pi i} \int_{\Gamma_{1}} g(\zeta)[\zeta-f(\lambda)]^{-1} d \zeta(\lambda e-x)^{-1} d \lambda \\
& =\frac{1}{2 \pi i} \int_{\Gamma_{0}} g(f(\lambda))(\lambda e-x)^{-1} d \lambda=\frac{1}{2 \pi i} \int_{\Gamma_{0}} h(\lambda)(\lambda e-x)^{-1} d \lambda=\tilde{h}(x)
\end{aligned}
$$

We shall now give some applications of this symbolic calculus. The first one deals with the existence of roots and logarithms. To say that an element $x \in A$ has an $n$th root in $A$ means that $x=y^{n}$ for some $y \in A$. If $x=\exp (y)$ for some $y \in A$, then $y$ is a logarithm of $x$.

Note that $\exp (y)=\sum_{0}^{\infty} y^{n} / n$ ! but that the exponential function can also be defined by contour integration, as in Definition 10.26. The continuity assertion of Theorem 10.27 shows that these definitions coincide (as they do for every entire function).

10.30 Theorem Suppose $A$ is a Banach algebra, $x \in A$, and the spectrum $\sigma(x)$ of $x$ does not separate 0 from $\infty$. Then

(a) $x$ has roots of all orders in $A$,

(b) $x$ has a logarithm in $A$, and

(c) if $\varepsilon>0$, there is a polynomial $P$ such that $\left\|x^{-1}-P(x)\right\|<\varepsilon$.

Moreover, if $\sigma(x)$ lies in the positive real axis, the roots in (a) can be chosen so as to satisfy the same condition.

PROOF By hypothesis, 0 lies in the unbounded component of the complement of $\sigma(x)$. Hence there is a function $f$, holomorphic in a simply connected open set $\Omega \supset \sigma(x)$, which satisfies

$$
\exp (f(\lambda))=\lambda
$$

It follows from Theorem 10.29 that

$$
\exp (\tilde{f}(x))=x
$$

so that $y=\tilde{f}(x)$ is a logarithm of $x$. If $0<\lambda<\infty$ for every $\lambda \in \sigma(x), f$ can be chosen so as to be real on $\sigma(x)$, so that $\sigma(y)$ lies in the real axis, by the spectral mapping theorem. If $z=\exp (y / n)$, then $z^{n}=x$, and another application of the spectral mapping theorem shows that $\sigma(z) \subset(0, \infty)$ if $\sigma(y) \subset(-\infty, \infty)$. This proves $(a)$ and $(b)$; of course $(a)$ could have been proved directly, without passing through $(b)$.

To prove $(c)$, note that $1 / \lambda$ can be approximated by polynomials, uniformly on some open set containing $\sigma(x)$ (Runge's theorem), and use the continuity assertion of Theorem 10.27.

These results are not quite trivial even when $A$ is a finite-dimensional algebra. For example, it is a special case of $(b)$ that a complex $n$-by- $n$ matrix $M$ is the exponential of some matrix if and only if 0 is not an eigenvalue of $M$, that is, if and only if $M$ is invertible. To deduce this from $(b)$, let $A$ be the algebra of all complex $n$-by- $n$ matrices (or the algebra of all bounded linear operators on $\mathscr{C}^{n}$ ).

\subsection{Theorem}
(a) Suppose $A$ is a Banach algebra, $x \in A, P$ is a polynomial in one variable, and $P(x)=0$. Then $\sigma(x)$ lies in the set of zeros of $P$.

(b) In particular, if $x$ is idempotent, i.e., if $x^{2}=x$, then $\sigma(x) \subset\{0,1\}$.

(c) If the spectrum of some element of $A$ is not connected, then $A$ contains a nontrivial idempotent.

The trivial idempotents are 0 and $e$, of course.

PROOF By the spectral mapping theorem,

$$
P(\sigma(x))=\sigma(P(x))=\sigma(0)=\{0\}
$$

This gives $(a)$ and $(b)$. If $\sigma(x)$ is not connected, there are disjoint open sets $\Omega_{0}$ and $\Omega_{1}$, both of which intersect $\sigma(x)$ and whose union $\Omega$ covers $\sigma(x)$. Put $f(\lambda)=0$ in $\Omega_{0}, f(\lambda)=1$ in $\Omega_{1}$. Then $f \in H(\Omega)$. Put $y=\tilde{f}(x)$. Since $f^{2}=f$, Theorem 10.27 implies that $y^{2}=y$, and so $y$ is idempotent. By the spectral mapping theorem,

$$
\sigma(y)=f(\sigma(x))=\{0,1\} .
$$

Hence $y$ is nontrivial, since 0 and $e$ have one-point spectra.

10.32 Definition Let $\mathscr{B}(X)$ be the Banach algebra of all bounded linear operators on the Banach space $X$. The point spectrum $\sigma_{p}(T)$ of an operator $T \in \mathscr{B}(X)$ is the set of all eigenvalues of $T$. Thus $\lambda \in \sigma_{p}(T)$ if and only if the null space $\mathcal{N}(T-\lambda I)$ of $T-\lambda I$ has positive dimension.

When $A=\mathscr{B}(X)$, the spectral mapping theorem can be refined in the following way.

10.33 Theorem Suppose $T \in \mathscr{B}(X), \Omega$ is open in $\mathscr{C}, \sigma(T) \subset \Omega$, and $f \in H(\Omega)$.

(a) If $x \in X, \alpha \in \Omega$, and $T x=\alpha x$, then $\tilde{f}(T) x=f(\alpha) x$.

(b) $f\left(\sigma_{p}(T)\right) \subset \sigma_{p}(\tilde{f}(T))$.

(c) If $\alpha \in \sigma_{p}(\tilde{f}(T))$ and $f-\alpha$ does not vanish identically in any component of $\Omega$, then $\alpha \in f\left(\sigma_{p}(T)\right)$.

(d) If $f$ is not constant in any component of $\Omega$, then $f\left(\sigma_{p}(T)\right)=\sigma_{p}(\tilde{f}(T))$.

Part $(a)$ states that every eigenvector of $T$, with eigenvalue $\alpha$, is also an eigenvector of $\tilde{f}(T)$, with eigenvalue $f(\alpha)$.

PROOF (a) If $x=0$ there is nothing to be proved. Assume $x \neq 0$ and $T x=\alpha x$. Then $\alpha \in \sigma(T)$, and there exists $g \in I(\Omega)$ such that

$$
f(\lambda)-f(\alpha)=g(\lambda)(\lambda-\alpha)
$$

By Theorem 10.27, (1) implies

$$
\tilde{f}(T)-f(\alpha) I=\tilde{g}(T)(T-\alpha I)
$$

Since $(T-\alpha I) x=0,(2)$ proves $(a)$.

Thus $f(\alpha)$ is an eigenvalue of $f(T)$ whenever $\alpha$ is an eigenvalue of $T$. It follows that $(a)$ implies $(b)$.

Under the hypotheses of $(c)$,

$$
\alpha \in \sigma_{p}(\tilde{f}(T)) \subset \sigma(\tilde{f}(T))=f(\sigma(T))
$$

so that

$$
f^{-1}(\alpha) \cap \sigma(T) \neq \varnothing
$$

Moreover, the set (4) is finite, because $\sigma(T)$ is a compact subset of $\Omega$ and $f-\alpha$ does not vanish identically in any component of $\Omega$. Let $\zeta_{1}, \ldots, \zeta_{n}$ be the zeros of $f-\alpha$ in $\sigma(T)$, counted according to their multiplicities. Then

$$
f(\lambda)-\alpha=g(\lambda)\left(\lambda-\zeta_{1}\right) \cdots\left(\lambda-\zeta_{n}\right)
$$

where $g \in H(\Omega)$ and $g$ has no zero on $\sigma(T)$, so that

$$
\tilde{f}(T)-\alpha I=\tilde{g}(T)\left(T-\zeta_{1} I\right) \cdots\left(T-\zeta_{n} I\right)
$$

By $(a)$ of Theorem $10.28, \tilde{g}(T)$ is invertible in $\mathscr{B}(X)$. Since $\alpha$ is an eigenvalue of $f(T), f(T)-\alpha I$ is not one-to-one on $X$. Hence (6) implies that at least one of the operators $T-\zeta_{i} I$ must fail to be one-to-one. The corresponding $\zeta_{i}$ is in $\sigma_{p}(T)$, and since $f\left(\zeta_{i}\right)=\alpha$ the proof of $(c)$ is complete.

Finally, $(d)$ is an immediate consequence of $(b)$ and $(c)$.

\section{Differentiation}
We shall now investigate the extent to which the members of $\tilde{H}\left(A_{\Omega}\right)$ (see Definition 10.26) behave like holomorphic functions, as far as differentiability, power series representation, and the open mapping property are concerned. As might be expected, some of the results are more similar to the classical ones when $A$ is commutative than when it is not.

10.34 Definition Suppose $X$ and $Y$ are Banach spaces, $\Omega$ is an open subset of $X$, $F$ maps $\Omega$ into $Y$, and $a \in \Omega$. If there exists $\Lambda \in \mathscr{B}(X, Y)$ (the Banach space of all bounded linear mappings of $X$ into $Y$ ) such that

$$
\lim _{x \rightarrow 0} \frac{\|F(a+x)-F(a)-\Lambda x\|}{\|x\|}=0
$$

then $\Lambda$ is called the Frèchet derivative of $F$ at $a$. (The uniqueness of $\Lambda$ is trivial.)

The notation $(D F)_{a}$ will be used for the Fréchet derivative of $F$ at $a$. If $(D F)_{a}$ exists for every $a \in \Omega$, and if

$$
a \rightarrow(D F)_{a}
$$

is a continuous mapping of $\Omega$ into $\mathscr{B}(X, Y)$, then $F$ is said to be continuously differentiable in $\Omega$.

10.35 Difference Quotients If $A$ is a Banach algebra, $x \in A$, and $x+h \in A$, and if both sides of the identity

$$
(\lambda e-x)-(\lambda e-x-h)=h
$$

are multiplied by $(\lambda e-x-h)^{-1}$ on the left and by $(\lambda e-x)^{-1}$ on the right, one obtains

$$
(\lambda e-x-h)^{-1}-(\lambda e-x)^{-1}=(\lambda e-x-h)^{-1} h(\lambda e-x)^{-1},
$$

provided, of course, that these inverses exist.

Suppose now that $\Omega$ is open in $\ell, x \in A_{\Omega}, x+h \in A_{\Omega}$, and $f \in H(\Omega)$. Choose $\Gamma$ so that it surrounds $\sigma(x) \cup \sigma(x+h)$ in $\Omega$. Then (1) leads to

$$
\tilde{f}(x+h)-\tilde{f}(x)=\frac{1}{2 \pi i} \int_{\Gamma} f(\lambda)(\lambda e-x-h)^{-1} h(\lambda e-x)^{-1} d \lambda
$$

If $h$ and $x$ commute, i.e., if $x h=h x$, then $h$ can be moved outside the integral (2), as we saw in Section 10.22. This motivates the definition of the difference quotient

$$
(Q \tilde{f})(x ; h)=\frac{1}{2 \pi i} \int_{\Gamma} f(\lambda)(\lambda e-x-h)^{-1}(\lambda e-x)^{-1} d \lambda,
$$

which satisfies

$$
\tilde{f}(x+h)-\tilde{f}(x)=h(Q \tilde{f})(x ; h),
$$

provided that $x h=h x$, an assumption which applies to the remainder of this section. If $\|h\|<1 / M$, where $M>\left\|(\lambda e-x)^{-1}\right\|$ for every $\lambda$ on $\Gamma$, then the series

$$
\left(\lambda e-x_{\star}-h\right)^{-1}=\sum_{n=1}^{\infty}(\lambda e-x)^{-n} h^{n-1}
$$

converges, in the norm topology of $A$, uniformly for $\lambda$ on $\Gamma$. Hence (3) becomes

$$
\begin{aligned}
(Q \tilde{f})(x ; h) & =\sum_{n=1}^{\infty} \frac{1}{2 \pi i} \int_{\Gamma} f(\lambda)(\lambda e-x)^{-n-1} d \lambda h^{n-1} \\
& =\sum_{n=1}^{\infty} \frac{1}{n !} \frac{1}{2 \pi i} \int_{\Gamma} f^{(n)}(\lambda)(\lambda e-x)^{-1} d \lambda h^{n-1}=\sum_{n=1}^{\infty} \frac{\tilde{f}^{(n)}(x)}{n !} h^{n-1},
\end{aligned}
$$

where $f^{(n)}$ is the $n$th derivative of $f$, and $\tilde{f}^{(n)}$ is an abbreviation for $\left[f^{(n)}\right]^{\sim}$. The norm of the coefficient of $h^{n-1}$ in the last power series is dominated by a constant (depending on $f$ and $\Gamma$ ) times $M^{n}$. The series converges, therefore, in norm. By (4) and (6), the power series representation

$$
\tilde{f}(x+h)=\sum_{n=0}^{\infty} \frac{1}{n !} \tilde{f}^{(n)}(x) h^{n}
$$

holds if $x h=h x$ and $\|h\|$ is sufficiently small. (See Exercise 9.)

The following facts have now been proved:

10.36 Theorem Suppose $A$ is a commutative Banach algebra, $\Omega \subset \mathbb{C}$ is open, $x \in A_{\Omega}$, and $f \in H(\Omega)$. Then there exists $\delta>0$ such that

$$
\tilde{f}(x+h)=\sum_{n=0}^{\infty} \frac{1}{n !} \tilde{f}^{(n)}(x) h^{n}
$$

for all $h \in A$ with $\|h\|<\delta$. Consequently,

$$
(D \tilde{f})_{x}(h)=\tilde{f}^{\prime}(x) h \quad(h \in A)
$$

In other words, the operator $(D \tilde{f})_{x} \in \mathscr{B}(A)$ is multiplication by $\tilde{f}^{\prime}(x)$.

This is, of course, exactly as in the classical case $A=\ell$. We now consider the noncommutative situation.

10.37 Commutators Left and right multiplication by an element $x$ of a Banach algebra $A$ will now be denoted by $L_{x}$ and $R_{x}$, respectively. Since the associative law $y(x z)=(y x) z$ holds in $A$, every left multiplier $L_{y}$ commutes with every right multiplier $R_{z}$. In particular, $L_{x}$ and $R_{x}$ commute with each other and with the operator

$$
C_{x}=R_{x}-L_{x}
$$

Note that $C_{x}(y)=y x-x y$, the so-called commutator of $y$ and $x$.

Of course, $L_{x}, R_{x}$, and $C_{x}$ are members of $\mathscr{B}(A)$. It is easily seen that

$$
\sigma\left(L_{x}\right)=\sigma(x)=\sigma\left(R_{x}\right)
$$

and that $\left\|C_{x}\right\| \leq 2\|x\|$. Some further information about $\sigma\left(C_{x}\right)$ will be obtained in the corollary to Theorem 11.23.

10.38 Theorem Suppose $A$ is a Banach algebra, $\Omega$ is open in $\varnothing, x \in A_{\Omega}$, and $f \in$ $H(\Omega)$. Then $\tilde{f}$ is a continuously differentiable mapping of $A_{\Omega}$ into $A$; and

$$
(D \tilde{f})_{x}(y)=\frac{1}{2 \pi i} \int_{\Gamma} f(\lambda)(\lambda e-x)^{-1} y(\lambda e-x)^{-1} d \lambda \quad(y \in A)
$$

if $\Gamma$ is any contour that surrounds $\sigma(x)$ in $\Omega$.

The operator $(D \tilde{f})_{x}$ can also be represented by the $\mathscr{B}(A)$-valued integral

$$
(D \tilde{f})_{x}=\frac{1}{2 \pi i} \int_{\Gamma} f(\lambda)\left(\lambda I-R_{x}\right)^{-1}\left(\lambda I-L_{x}\right)^{-1} d \lambda
$$

and by the difference quotient

$$
(D \tilde{f})_{x}=(Q \tilde{f})\left(L_{x} ; C_{x}\right)
$$

If $\Omega$ contains all $\lambda$ with $|\lambda| \leq 3\|x\|$, then

$$
(D \tilde{f})_{x}=\sum_{m=1}^{\infty} \frac{1}{m !} \tilde{f}^{(m)}(x) C_{x}^{m-1}
$$

The notation used in (3) is perhaps not explicit enough. On the left side of (3), $\tilde{f}$ is a function from $A_{\Omega}$ into $A$; on the right side, $\tilde{f}$ stands for a function from $\mathscr{B}(A)_{\Omega}$ into $\mathscr{B}(A)$; both sides of (3) represent members of $\mathscr{B}(A)$.

PROOF If $M>\left\|(\lambda e-x)^{-1}\right\|$ for all $\lambda$ on $\Gamma$ and if $2 M\|y\|<1$, Theorem 10.11 shows that the norm of the difference between $\tilde{f}(x+y)-\tilde{f}(x)$ and the right side of (1) is at most $2 M^{3}\|y\|^{2}$, multiplied by the length of $\Gamma$ and the maximum of $|f|$ on $\Gamma$. This proves the formula (1).

Let $g(\lambda) \in \mathscr{B}(A)$ be the integrand in (2). Since inversion is continuous in every Banach algebra, and hence in $\mathscr{B}(A), g$ is continuous on $\Gamma$, and $T \in \mathscr{B}(A)$ can be defined by

$$
T=\int_{\Gamma} g(\lambda) d \lambda
$$

But (5) implies

$$
T y=\int_{\Gamma} g(\lambda) y d \lambda \quad(y \in A)
$$

and since $g(\lambda) y$ is exactly the integrand in (1), (6) shows that $T=2 \pi i(D \tilde{f})_{x}$. This proves (2).

It may be worthwhile to indicate in detail how (6) follows from (5) and Definition 3.26: If $F \in A^{*}$ (the dual space of $A$ ), if $y \in A$, and if $F_{1} S=F(S y)$ for $S \in \mathscr{B}(A)$, then $F_{1} \in \mathscr{B}(A)^{*}$, and

$$
F(T y)=F_{1} T=\int_{\Gamma} F_{1} g(\lambda) d \lambda=\int_{\Gamma} F[g(\lambda) y] d \lambda=F \int_{\Gamma} g(\lambda) y d \lambda .
$$

Let us return to (2). If $x_{n} \rightarrow x$, the contour $\Gamma$ used in (2) will surround $\sigma\left(x_{n}\right)$ in $\Omega$, for all but finitely many $n$. Discard these. Then $(D \tilde{f})_{x_{n}}$ is given by (2), if $x$ is replaced by $x_{n}$ in the integrand. Since

$$
\left(\lambda e-x_{i n}\right)^{-1} \rightarrow(\lambda e-x)^{-1} \quad \text { as } n \rightarrow \infty \text {, }
$$

uniformly on $\Gamma$, the integrands in (2) converge uniformly. We conclude that $x \rightarrow(D \tilde{f})_{x}$ is continuous. Thus $\tilde{f}$ is continuously differentiable.

Since $R_{x}=L_{x}+C_{x}$, (3) is just another way of writing (2).

If $\Gamma$ can be chosen in (2) so as to be a circle with radius $r>3\|x\|$ and center at 0 , then

$$
\left\|\left(\lambda I-L_{x}\right)^{-1}\right\|=\left\|\sum_{n=0}^{\infty} \lambda^{-n-1} L_{x}^{n}\right\| \leq \sum_{n=0}^{\infty} r^{-n-1}\|x\|^{n}=\frac{1}{r-\|x\|}
$$

for every $\lambda$ on $\Gamma$, so that

$$
\left\|\left(\lambda I-L_{x}\right)^{-1}\right\|\left\|C_{x}\right\| \leq \frac{2\|x\|}{r-\|x\|}<1
$$

By (8), the computation (6) of Section 10.35 can now be applied to $(Q \tilde{f})\left(L_{x} ; C_{x}\right)$. It yields

$$
(Q \tilde{f})\left(L_{x} ; C_{x}\right)=\sum_{m=1}^{\infty} \frac{1}{m !} \tilde{f}^{(m)}\left(L_{x}\right) C_{x}^{m-1}
$$

Finally, (4) follows from (3) and (9), because

$$
\begin{aligned}
\tilde{g}\left(L_{x}\right) y & =\frac{1}{2 \pi i} \int_{\Gamma} g(\lambda)\left(\lambda I-L_{x}\right)^{-1} y d \lambda \\
& =\frac{1}{2 \pi i} \int_{\Gamma} g(\lambda)(\lambda e-x)^{-1} y d \lambda=\tilde{g}(x) y
\end{aligned}
$$

for every $y \in A$ and every $g \in H(\Omega)$, hence in particular for every $f^{(m)}$.

This completes the proof.

The series (4) may actually fail to converge if $f$ has a singularity at distance $3\|x\|$ from the origin. An example of this is described in Exercise 22. The constant 3 that occurs in the last part of Theorem 10.38 is therefore the correct one.

If $A$ is commutative, then $C_{x}=0$. The term with $m=1$ is then the only one that remains in the series (4). This agrees with Theorem 10.36.

The following theorem will allow us to extract information about local mapping properties of the functions $\tilde{f}$ from Theorem 10.36 .

\subsection{The inverse function theorem Suppose}
(a) $W$ is an open subset of a Banach space $X$,

(b) $F: W \rightarrow X$ is continuously differentiable,

(c) for every $x \in W,(D F)_{x}$ is an invertible member of $\mathscr{B}(X)$.

Every point $a \in W$ has then a neighborhood $U$ such that

(i) $F$ is one-to-one in $U$,

(ii) $F(U)=V$ is an open subset of $X$,

(iii) $F^{-1}: V \rightarrow U$ is continuously differentiable.

The conclusion may be stated briefly by saying that $F$ is a local diffeomorphism. PROOF If $a \in W$, if $T=(D F)_{a}$, and if

$$
f(x)=T^{-1}[F(a+x)-F(a)] \quad(x \in W-a),
$$

then $f$ satisfies the hypotheses of the theorem, with $W-a$ in place of $W$. If $f$ satisfies the conclusion, the same will be true of $F$. We may therefore replace $F$ by $f$. In other words, we assume, without loss of generality, that

$$
0 \in W, \quad F(0)=0, \quad(D F)_{0}=I
$$

and we have to prove that 0 has a neighborhood $U$ that satisfies (i), (ii), and (iii).

Fix $\alpha, 0<\alpha<1$. Define

$$
\phi(x)=x-F(x) \quad(x \in W)
$$

$\therefore$ Then $(D \phi)_{0}=0$, and since $\phi$ is continuously differentiable in $W$, there is an open ball $B \subset W$, centered at 0 , such that

$$
\left\|(D \phi)_{x}\right\|<\alpha \quad \text { if } \quad x \in B
$$

Suppose $x^{\prime} \in B, x^{\prime \prime} \in \bar{B}, x_{t}=(1-t) x^{\prime}+t x^{\prime \prime}$, and

$$
\psi(t)=\phi\left(x_{t}\right) \quad(0 \leq t \leq 1)
$$

Then $\psi:[0,1] \rightarrow X$ is continuously differentiable,

$$
\psi^{\prime}(t)=(D \phi)_{x}\left(x^{\prime \prime}-x^{\prime}\right) \quad\left(x=x_{t}\right)
$$

by the chain rule, and hence (4) implies

$$
\left\|\psi^{\prime}(t)\right\| \leq \alpha\left\|x^{\prime \prime}-x^{\prime}\right\|
$$

note that $x_{t} \in B$, by the convexity of $B$. (See Exercise 10.) Since

$$
\phi\left(x^{\prime \prime}\right) \rightarrow \phi\left(x^{\prime}\right)=\psi(1)-\psi(0)=\int_{0}^{1} \psi^{\prime}(t) d t
$$

we conclude from (7) that $\phi$ satisfies the Lipschitz condition

$$
\left\|\phi\left(x^{\prime \prime}\right)-\phi\left(x^{\prime}\right)\right\| \leq \alpha\left\|x^{\prime \prime}-x^{\prime}\right\| \quad\left(x^{\prime} \in B, x^{\prime \prime} \in B\right)
$$

It now follows from (3) that

$$
\left\|F\left(x^{\prime \prime}\right)-F\left(x^{\prime}\right)\right\| \geq(1-\alpha)\left\|x^{\prime \prime}-x^{\prime}\right\| \quad\left(x^{\prime} \in B, x^{\prime \prime} \in B\right)
$$

This implies that $F$ is one-to-one in $B$. Also, if $G: F(B) \rightarrow B$ is defined by $G(F(x))=x$, then (10) shows that $G$ is continuous.

Our next aim is to show that $F(B) \supset(1-\alpha) B$.

Fix $y \in(1-\alpha) B$. Put $x_{0}=0, x_{1}=y$. Suppose $n \geq 1$, and $x_{0}, \ldots, x_{n}$ exist so that

$$
x_{i}=y+\phi\left(x_{i-1}\right) \quad(1 \leq i \leq n)
$$

and

$$
\left\|x_{i}-x_{i-1}\right\| \leq \alpha^{i-1}\|y\| \quad(1 \leq i \leq n)
$$

(These conditions hold when $n=1$.) By (12),

$$
\left\|x_{n}\right\| \leq \sum_{i=1}^{n}\left\|x_{i}-x_{i-1}\right\| \leq \sum_{i=1}^{n} \alpha^{i-1}\|y\| \leq(1-\alpha)^{-1}\|y\|
$$

so that $x_{n} \in B, \phi\left(x_{n}\right)$ exists, and one can define

$$
x_{n+1}=y+\phi\left(x_{n}\right)
$$

It follows from (14), (11), and (9) that

$$
\left\|x_{n+1}-x_{n}\right\|=\left\|\phi\left(x_{n}\right)-\phi\left(x_{n-1}\right)\right\| \leq \alpha\left\|x_{n}-x_{n-1}\right\| .
$$

Our induction hypotheses hold now with $n+1$ in place of $n$, and the construction can proceed, to yield a sequence $\left\{x_{n}\right\}$ that satisfies (11) and (12) for all $n$. Since $\alpha<1,(12)$ shows that $\left\{x_{n}\right\}$ is a Cauchy sequence, which converges to some $x \in B$, by (13). Now (11) and (3) imply that $F(x)=y$.

If $V=(1-\alpha) B$ and $U=G(V)=B \cap F^{-1}(V)$, then conclusions $(i)$ and (ii) hold. To complete the proof, we now show that $G$ is continuously differentiable in $V$.

Suppose $y \in V, \quad y+k \in V, \quad k \neq 0, \quad x=G(y), \quad x+h=G(y+k) ; \quad$ put $S=(D F)_{x}$. Then

$$
\begin{aligned}
G(y+k)-G(y)-S^{-1} k & =h-S^{-1} k \\
& =S^{-1}(S h-k)=-S^{-1}[F(x+h)-F(x)-S h]
\end{aligned}
$$

By $(10),(1-\alpha)\|h\| \leq\|k\|$. Hence

$$
\frac{\left\|G(y+k)-G(y)-S^{-1} k\right\|}{\|k\|} \leq\left\|S^{-1}\right\| \frac{\|F(x+h)-F(x)-S h\|}{(1-\alpha)\|h\|} .
$$

As $k \rightarrow 0$, (10) implies that $h \rightarrow 0$, and since $S=(D F)_{x}$, the last inequality shows that $S^{-1}=(D G)_{y}$. In other words,

$$
(D G)_{y}=\left[(D F)_{G(y)}\right]^{-1} \quad(y \in V) .
$$

Since $G$ maps $V$ continuously into $\mathscr{B}(X)$, and since inversion is continuous in $\mathscr{B}(X)$ (Theorem 10.12), (16) shows that $y \rightarrow(D G)_{y}$ is a continuous mapping of $V$ into $\mathscr{B}(X)$.

This completes the proof.

10.40 Theorem Suppose $A$ is a commutative Banach algebra, $\Omega$ is open in $\ell$, $x \in A_{\Omega}$, and the derivative $f^{\prime}$ of some $f \in H(\Omega)$ has no zero on $\sigma(x)$. Then $x$ has a neighborhood $U \subset A_{\Omega}$ such that the restriction of $\tilde{f}$ to $U$ is a diffeomorphism whose range is an open subset of $A$.

PROOF By Theorem 10.28, $\tilde{f}^{\prime}(x)$ is invertible in $A$. By Theorem 10.36, $(D \tilde{f})_{x}$ is therefore invertible in $\mathscr{B}(A)$. Since $y \rightarrow(D \tilde{f})_{y}$ maps $A_{\Omega}$ continuousiy into $\mathscr{B}(A)$, and since the invertible members of $\mathscr{B}(A)$ form an open set, $x$ has a neighborhood in which $(D \tilde{f})_{y}$ is invertible. The conclusion follows therefore ,from Theorem 10.39.

Note that the theorem just proved does not assert what one might expect to be true, namely, that $\tilde{f}$ is an open mapping of $A_{\Omega}$ into $A$ whenever $f \in H(\Omega)$ is not constant in any component of $\Omega$. This is actualily false; see Exercise 13 for an example. The theorem does prove that $\tilde{f}$ is open near points $x$ at which $(D \tilde{f})_{x}$ is invertible.

The hypothesis that $f^{\prime}$ has no zero on $\sigma(x)$ means that $f$ is locally one-to-one in some open set that contains $\sigma(x)$. Theorem 10.42 will show that this local condition does not imply that $\tilde{f}$ is open at $x$ if commutativity of $A$ is dropped from the assumptions. An analogous global theorem does, however, turn out to be true:

10.41 Theorem Suppose $A$ is a Banach algebra, $\Omega$ is open in $\ell, f \in H(\Omega)$, and $f$ is one-to-one in $\Omega$. Then $\tilde{f}$ is a diffeomorphism of $A_{\Omega}$ onto $A_{f(\Omega)}$.

PROOF Let $g: f(\Omega) \rightarrow \Omega$ be the inverse of $f$. Since $g \circ f$ and $f \circ g$ are the identity mappings in $\Omega$ and in $f(\Omega)$, respectively, Theorem 10.29 shows that $\tilde{g} \circ \tilde{f}$ is the identity in $A_{\Omega}$, so that $\tilde{f}$ is one-to-one, and $\tilde{f} \circ \tilde{g}$ is the identity in $A_{f(\Omega)}$, so that $A_{f(\Omega)}$ is the range of $\tilde{f}$. Since both $\tilde{f}$ and its inverse $\tilde{g}$ are continuously differentiable (Theorem 10.38), the proof is complete.

10.42 Theorem Suppose $A=\mathscr{B}(X)$, where $X$ is a complex Banach space and $\operatorname{dim} X>1$. If $\Omega$ is open in $\mathbb{C}$, if $f \in H(\Omega)$, and if $f$ is not one-io-one in $\Omega$, then some $T_{0} \in A_{\Omega}$ has a neighborhood $U$ such that $\tilde{f}(U)$ contains no neighborhood of $\tilde{f}\left(T_{0}\right)$.

Thus $\tilde{f}$ is not open at $T_{0}$.

PROOF By assumption, $\Omega$ contains two points $\alpha \neq \beta$ at which $f(\alpha)=f(\beta)=c$, say. Let $Y$ be a closed subspace of $X$, of codimension 1 ; choose $x_{1}, x_{2} \in X$, $x_{i} \neq 0$, such that $x_{2}$ is in $Y$ but $x_{1}$ is not; and define $T_{0} \in A$ by

$$
T_{0} x_{1}=\alpha x_{1}, \quad T_{0} y=\beta y \quad \text { if } y \in Y .
$$

If $\lambda \neq \alpha$ and $\lambda \neq \beta$, multiplication of $x_{1}$ by $(\lambda-\alpha)^{-1}$ and of $y$ by $(\lambda-\beta)^{-1}$ defines $\left(\lambda I-T_{0}\right)^{-1}$. Thus $\sigma\left(T_{0}\right)=\{\alpha, \beta\}$, and $T_{0} \in A_{\Omega}$. By $(a)$ of Theorem 10.33 ,

$$
\tilde{f}\left(T_{0}\right)=c I
$$

Put $x_{3}=x_{1}+x_{2}$, let $M$ be the one-dimensional subspace of $X$ generated by $x_{3}$, and let $\delta$ be the distance from $T_{0} x_{3}$ to $M$. Then $\delta>0$, since $T_{0} x_{3}=$ $\alpha x_{1}+\beta x_{2}$ and $\alpha \neq \beta$. Let $\Omega_{0}$ be the union of the components of $\Omega$ that contain $\alpha$ and $\beta$. (There are either one or two of these components.) Let $U$ be the set of all $T \in A$ such that

$$
\left\|T-T_{0}\right\|<\frac{\delta}{\left\|x_{3}\right\|} \quad \text { and } \quad \sigma(T) \subset \Omega_{0}
$$

Then $U$ is a neighborhood of $T_{0}$. We shall prove that $\tilde{f}(U)$ does not contain $\tilde{f}\left(T_{0}\right)+\eta S$ if $\eta \neq 0$ and if $S \in A$ is defined by

$$
S x_{1}=x_{3}, \quad S y=0 \quad \text { for } \quad y \in Y .
$$

We argue by contradiction. Suppose $\sigma(T) \subset \Omega_{0}, \eta \neq 0$, and

$$
\tilde{f}(T)=\tilde{f}\left(T_{0}\right)+\eta S=c I+\eta S .
$$

Then

$$
\tilde{f}(T) x_{3}=(c+\eta) x_{3}, \quad \tilde{f}(T) y=c y \quad \text { for } \quad y \in Y \text {. }
$$

Thus $c+\eta$ is an eigenvalue of $\tilde{f}(T)$, with eigenspace $M$. Since $f-(c+\eta)$ vanishes neither at $\alpha$ nor at $\beta$, it does not vanish identically in any component of $\Omega_{0}$, and $(c)$ of Theorem 10.33 implies that $c+\eta=f(\gamma)$ for some eigenvalue $\gamma$ of $T$. By $(a)$ of Theorem 10.33, the corresponding eigenspace lies in $M$, hence equals $M$, since $\operatorname{dim} M=1$. Thus $T x_{3} \in M$. Our choice of $\delta$ implies now that

$$
\delta \leq\left\|T x_{3}-T_{0} x_{3}\right\| \leq\left\|T-T_{0}\right\|\left\|x_{3}\right\|
$$

Hence $T$ is not in $U$.

10.43 The exponential function To illustrate the preceding results, let us see what they tell about the exponential function, defined by the power series

$$
\exp (x)=\sum_{n=0}^{\infty} \frac{1}{n !} x^{n}
$$

in every Banach algebrà $A$. (See also Theorem 10.30.)
(a) If $\sigma(x)$ contains no two points whose difference is an integral multiple of $2 \pi i$, the compactness of $\sigma(x)$ shows that exp is one-to-one in some open set $\Omega \supset \sigma(x)$; hence exp is a diffeomorphism of the neighborhood $A_{\Omega}$ of $x$ into $A$, by Theorem 10.41.

(b) The Fréchet derivative of exp at $x$ is

$$
(D \exp )_{x}=\exp (x) \widetilde{\Phi}\left(C_{x}\right)
$$

where $\Phi$ is the entire function defined by

$$
\Phi(\lambda)=\frac{\exp (\lambda)-1}{\lambda}
$$

This follows from the last part of Theorem 10.38 , since $f^{(m)}=f$ for $m \geq 1$ when $f(\lambda)=\exp (\lambda)$.

The zeros of $\Phi$ are at $2 k \pi i, k= \pm 1, \pm 2, \ldots$. If none of these lies in $\sigma\left(C_{x}\right)$, then $\widetilde{\Phi}\left(C_{x}\right)$ is invertible, by the spectral mapping theorem, and so is $(D \exp )_{x}$, and exp is again a diffeomorphism near $x$.

(c) We shall see later (Theorem 11.23) that

$$
\sigma\left(C_{x}\right) \subset \sigma(x)-\sigma(x)
$$

This provides a link between the preceding paragraphs $(a)$ and $(b)$.

(d) If $A$ is commutative, then $(D \exp )_{x}$ is invertible, for every $x \in A$, since it is simply multiplication by $\exp (x)$, an invertible member of $A$. (Theorem 10.36.) Hence exp is a local diffeomorphism, as in the familiar case $A=\ell$. However, if $A=\mathscr{B}(X)$, as in Theorem 10.42, then exp is not an open mapping of $A$ into $A$.

\section{The Group of Invertible Elements}
We shall now take a closer look at the structure of $G=G(A)$, the multiplicative group of all invertible elements of a Banach algebra $A$.

$G_{1}$ will denote the component of $G$ that contains $e$, the identity element of $G$. Sometimes $G_{1}$ is called the principal component of $G$. By the definition of component, $G_{1}$ is the union of all connected subsets of $G$ that contain $e$.

The group $G$ contains the set

$$
\exp (A)=\{\exp (x): x \in A\}
$$

the range of the exponential function in $A$, simply because $\exp (-x)$ is the inverse of $\exp (x)$. In fact, the power series definition of $\exp (x)$ (see Section 10.43) yields the functional equation

$$
\exp (x+y)=\exp (x) \exp (y)
$$

provided that $x y=y x$; also, $\exp (0)=e$.

Note also that $G$ is a topological group (see Section 5.12) since multiplication and inversion are continuous in $G$.

\subsection{Theorem}
(a) $G_{1}$ is an open normal subgroup of $G$.

(b) $G_{1}$ is the group generated by $\exp (A)$.

(c) If $A$ is commutative, then $G_{1}=\exp (A)$.

(d) If $A$ is commutative, the quotient group $G / G_{1}$ contains no element of finite order (except for the identity).

PROOF (a) Theorem 10.11 shows that every $x \in G_{1}$ is the center of an open bail $U \subset G$. Since $U$ intersects $G_{1}$ and $U$ is connected, $U \subset G_{1}$. Therefore $G_{1}$ is open.

If $x \in G_{1}$ then $x^{-1} G_{1}$ is a connected subset of $G$ which contains $x^{-1} x=e$. Hence $x^{-1} G_{1} \subset G_{1}$, for every $x \in G_{1}$. This proves that $G_{1}$ is a subgroup of $G$. Also, $y^{-1} G_{1} y$ is homeomorphic to $G_{1}$, hence connected, for every $y \in G$, and contains $e$. Thus $y^{-1} G_{1} y \subset G_{1}$. By definition, this says that $G_{1}$ is a normal subgroup of $G$.

(b) Let $\Gamma$ be the group generated by $\exp (A)$. For $n=1,2,3, \ldots$, let $E_{n}$ be the set of all products of $n$ members of $\exp (A)$. Since $y^{-1} \in \exp (A)$ whenever $y \in \exp (A), \Gamma$ is the union of the sets $E_{n}$. Since the product of any two connected sets is connected, induction shows that each $E_{n}$ is connected. Each $E_{n}$ contains $e$, and so $E_{n} \subset G_{1}$. Hence $\Gamma$ is a subgroup of $G_{1}$.

Next, $\exp (A)$ has nonempty interior, relative to $G$ (see Theorem 10.30); hence so has $\Gamma$. Since $\Gamma$ is a group and since multiplication by any $x \in G$ is a homeomorphism of $G$ onto $G, \Gamma$ is open.

Each coset of $\Gamma$ in $G_{1}$ is therefore open, and so is any union of these cosets. Since $\Gamma$ is the complement of a union of its cosets, $\Gamma$ is closed, relative to $G_{1}$.

Thus $\Gamma$ is an open and closed subset of $G_{1}$. Since $G_{1}$ is connected, $\Gamma=G_{1}$.

(c) If $A$ is commutative, the functional equation satisfied by exp shows that $\exp (A)$ is a group. Hence $(b)$ implies $(c)$.

(d) We have to prove the following proposition:

If $A$ is commutative, if $x \in G$, and if $x^{n} \in G_{1}$ for some positive integer $n$, then $x \in G_{1}$.

Under these conditions, $x^{n}=\exp (a)$ for some $a \in A$, by (c). Put $y=\exp$ $\left(n^{-1} a\right)$ and $z=x y^{-1}$. Since $y \in G_{1}$, it suffices to prove that $z \in G_{1}$.

The commutativity of $A$ shows that

$$
z^{n}=x^{n} y^{-n}=\exp (a) \exp (-a)=e .
$$

Put $f(\lambda) \doteq \lambda z-(\lambda-1) e$, and let $E=\{\lambda \in \mathscr{C}: f(\lambda) \in G\}$. If $\alpha \in \sigma(z)$, then $\alpha^{n} \in \sigma\left(z^{n}\right)=\sigma(e)=\{1\}$. If $\lambda \notin E$, it follows that $(\lambda-1)^{n}=\lambda^{n}$. This equation has only $n-1$ solutions in $\mathscr{C}$. Hence $E$ is connected. Consequently, $f(E)$ is a connected subset of $G$ which contains $f(0)=e$. Thus $f(E) \subset G_{1}$. In particular, $z=f(1) \in G_{1}$.

This completes the proof.

Theorem 12.38 will show that $\exp (A)$ is not always a group.


\end{document}