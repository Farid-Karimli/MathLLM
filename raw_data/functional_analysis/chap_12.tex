12 BOUNDED OPERATORS ON A HILBERT SPACE Basic Facts following rules hold: number x,J), cld the inner product or sclar produet of $\textstyle{\mathcal{X}}$ is called an inner product space(or such that the 12.1 Definitions A complex vector spacc ${\cal{H}}$ and $\mathbf{\nabla}y$ in $H$ is associated a complex uitary space if to each ordered pair of vectors $\scriptstyle{\mathcal{X}}$ and ${\mathit{V}},$ (a) $(y,x)=(x,y).$ (The bar denotes complex conjugation.) (b) $(x+y,z)=(x,z)+(y,z).$ if xe H, ye H,ae C. (c) ( $\alpha x,y)=\alpha(x,y)\,\mathbf{i}$ for all $x\in H$ (d) ( $x,x)\geq0$ only if $\scriptstyle x\;=\;0$ (e) $(x,x)=0$ $F\subset H,$ the notation For fixed y,(G×,Jy)is therefore alinear function of x. For fxed whenever $E.$ is a conjugate and used. Since $(x,y)=0$ $E\perp F$ means that $x\perp y$ $x\in E$ $x,{\hat{\mathbf{n}}}$ is sometimes is $\mathbf{f}\left(x,y\right)=0,$ implies $(0,x)=0,$ linear function of y. Such funcions o two variables are sometims aled squilinea $x\perp y$ $F\subset H$ $E^{\perp}$ xis said to be orthogonal to y,and the notation the relaion lisymmetric. f and $y\in F\colon$ Also, the set f all ye H that are orthogonal to every xepouvprb orexos o x mxe sAcr 293 Fvery nerout saca normed by dctin $$ \|x\|=(x,\,x)^{1/2}. $$ HIlbert space Tno i iai e unmes smle aille 12.2 Theorem If $x\in H$ and $y\in H,$ where $\textstyle H$ is animer product space, then () $|(x,y)|\leq\|x\|\Vert y\|$ and $(2)$ $$ \left\|{\mathcal{X}}\ +\ y\right\|\ \leq\left\|{\mathcal{X}}\right\|\ +\ \|y\|. $$ Moreover (3) $$ \|y\|\le\|\lambda x+y\|\qquad f o r\;e v e r y\;\lambda\in C $$ if and only $i^{f}x\perp y$ PROOF Put $\alpha=(x,y).$ A simple comutation gives (4) $$ 0\leq\|\lambda x+y\|^{2}=\left|\lambda|^{2}\|x\|^{2}+2\mathrm{\bf~Re} (\sigma\lambda\right)+\|y\|^{2}. $$ Hence G)) olds i $\scriptstyle x\;=\;0.$ If $x=0$ $\operatorname{\mathcal{(1)}}$ and (3) are obvious. If $x\ 7\ 0,$ take $\lambda=-\bar{\alpha}/\|x\|^{2}.$ With this 。,(4 beome (5) $$ 0\leq\|\lambda x+y\|^{2}=\|\gamma\|^{2}-{\frac{\|\alpha\|^{2}}{\|x\|^{2}}}. $$ This roves an sows tat ) fase wh $\scriptstyle x\,+\,0.$ By squaring both sides $J/\slash$ of())oness tat 2) s conseneo ) Note: Cunles cotay xpilt e ete $\textstyle{H}$ will fom now on denote a Hilbertspac 123 Theorem Every onemty closed cowex se $E\subset H$ contains a unique x of minimal norm. PROOF Thc parallelogram law (1) $$ \|x+y\|^{2}+\|x-y\|^{2}=2\|x\|^{2}+2\|y\|^{2}\qquad(x\in H,\,y\in H) $$ follow diectyfrom te deini $$ \|x\|^{2}=(x,\,x) $$ . Put (2) $$ d=\operatorname*{inf}\left\{\|x\|\colon x\in E\right\}. $$ Choose $\mathbf{\nabla}y$ ar rae y xa x n n.e isais eas A” enc $x_{m}\|^{2}\geq4d^{2}$ If x and $x_{n}\in E$ so that $\|x_{n}\|\to d.$ Since (x, + )e E,Ix。+294 BANACH ALGEBRAS AND sprCTRAL THronv (1) implies that $\scriptstyle\{x_{n}\}$ is a Cauchy sequence in $\textstyle H_{\cdot}$ I, which therefore converges to some $x\in E,$ with $\|x\|=d.$ If $y\in E$ and $\|y\|=d,$ the sequence $\{x,y,x,y,\ldots\}$ must converge, as we just saw. Hence $y\equiv x.$ ${a\!\!\!/}{b\!\!\!/}{b\!\!\!/}$ 12.4 Theorem If M is a closed subspace of $\textstyle H,$ then $$ {\cal H}=M\oplus M^{\perp}. $$ The conclusion is, more explicitly, that $\mathcal{M}$ and ${\cal{M}}^{i}$ are closed subspaces of $\textstyle H$ whose intersection is {O} and whose sum is $H.$ The space ${\mathcal{M}}^{i}$ is clled the orthogonal complement of $\bar{M}.$ PRO0Ff $E\subset H,$ the linearity of $(x,y)$ as a function of $\scriptstyle{\mathcal{X}}$ shows that $E^{\perp}$ is a subspace of $\textstyle H_{\cdot}$ , and the Schwarz inequality (1) of Theorem $\scriptstyle{12.2}$ implies thcn that $E^{\perp}$ is closed. If $x\in M$ and $x\in M^{1}.$ then $(x,x)=0;$ hence $\scriptstyle x\;=\;0.$ Thus $M\cap M^{\perp}=\{0\}$ If $x\in H,$ apply Theorem 12.3 o the set $x-M\tan0$ conclude that there exists $x_{1}\in M$ that minimize $\|x-x_{1}\|.$ Put $x_{2}=x-x_{1}$ Then $\|x_{2}\|\leq\|x_{2}+y\|$ for that ${\mathrm{all~}}y\in M$ Hence $\chi_{2}\in M^{\perp}$ , by Theorem 12.2. Since $x=x_{1}+x_{2}\,,$ we have shown / $M+M^{\perp}=H.$ Corollary If $\mathcal{M}$ is a closed subspace of $\textstyle H,$ then $$ (M^{\bot})^{\bot}=M. $$ PROOr The inclusion $M\subset(M^{\perp})^{\perp}$ is obvious. Since $$ {\cal M}\oplus{\cal M}^{\perp}={\cal H}=M^{\perp}\oplus(M^{\perp})^{\perp}, $$ $\bar{M}$ cannot be a proper subspace of $(M^{\bot})^{\bot}$ / We now describe the dual space $H^{\star}$ of $H.$ 12.5 Theorem There is a conjugate-linear isometry $y\to\Lambda$ of ${\boldsymbol{H}}$ onto H*, given by (1) $$ \Lambda x=(x,y)\qquad(x\in H). $$ rRoor If $\gamma_{y}\in H$ and $\Lambda$ is dcfined by (1), the Schwarz inequality Since of Theorem $v_{s-x}$ $\operatorname{\mathcal{(1)}}$ 12.2 shows that $\Lambda\in H^{*}$ and that $\|\Lambda\|\leq\|y\|.$ (2) $$ \|y\|^{2}=(y,y)=\Lambda y\leq\|\Lambda\|\Vert y\|, $$ it follows that $\|\mathbf{A}\|=\|y\|$ has the form (D It remains to be shown that every $\Lambda\in H^{*}$poUNDED OPIRAToRs ON A HLBRT SPACE 295 If $\scriptstyle\Lambda=0$ take $y=0.$ .If $\Lambda\not\equiv0,$ let ${\mathcal{M}}(\Lambda)$ be the null space of $\Lambda.$ By Theorem 12.4 there exists $z\in{\mathcal{N}}(\Lambda)^{\perp},\,z\neq0.$ Since (3) (Ax)z -(Az)x∈ .W(A) (xe H), it follows that(Ax)(z, $z)-(\Lambda z)(x,z)=0.$ Hence (I) holds with $y=(z,z)^{-1}({\overline{{\Lambda z}}})z$ $l/l/$ 12,6 Theorem ${\hat{I}}{\hat{f}}\left\{x_{n}\right\}$ is a sequence of pairwise orthogonal vectors in $\textstyle H,$ then each of the following three statements implies the other two. $$ \begin{array}{l}{{\frac{1}{\epsilon_{1}^{\prime}}|\Phi|\Phi|\Phi|\Phi|}}\\ {{\frac{1}{\epsilon_{1}^{\prime}}}}\\ {{\frac{1}{\epsilon_{2}^{\prime}}}}\end{array} $$ Thus strong convergence (a) and weak convergence $\left(c\right)$ are equivalent for series of orthogonal vectors. PROOF Since $(x_{i},\,x_{j})=0\;{\mathrm{if}}$ i≠, the equality (1) $$ \|{\boldsymbol{x}}_{n}+\cdot\cdot\cdot+x_{m}\|^{2}=\|{\boldsymbol{x}}_{n}\|^{2}+\cdot\cdot\cdot\cdot+\ \|{\boldsymbol{x}}_{m}\|^{2} $$ holds whenever $n\leq m.$ Hence $\mathbf{\nabla}(\partial)$ implies that the partial sums of $\textstyle\sum x_{n}$ form a Cauchy sequence in $H.$ Since $\textstyle H$ is complete, $\mathbf{\nabla}(b)$ implies (a). The Schwarz inequality shows that (a) implies Cc. Finally, assume that (c) holds. Define $\Lambda_{n}\in H^{\mathfrak{s}}$ by (2) $$ \Lambda_{n}y=\sum_{i=1}^{n}(y,\,x_{i})\qquad(y\in H,\,n=1,\,2,\,3,\,\ldots). $$ By Cc),{A,y} converges for every $\operatorname{v\inH}\colon$ hence {| ,} is bounded, by the Banach- Steinhaus theorem. But (3) $$ ||\Lambda_{n}||=\|x_{1}+\cdot\cdot\cdot+x_{n}\|=\{\|x_{1}\|^{2}+\cdot\cdot\cdot\cdot+\|x_{n}\|^{2}\}^{1/2}. $$ Hence (e) implies (b) Bounded Operators In conformity with notations used earlier, ${\mathcal{B}}(H)$ will now denote the Banach algebra of all bounded inear operators ${\boldsymbol{T}}$ on a Hilbert space $H\neq\{0\},$ normed by |IT"| = sup {ITxllixe H, |xl≤1}.296 BANACH ALGEBRAS AND S PECTRAL THEOR We shall see that ${\mathcal{A}}(H)$ has an involution which makcs it into a $B^{\ast}.$ -algebra. We begin with a simple but useful uniqueness theorem. 12.7 Theorem $$ J/T\in{\mathcal{B}}(H)\;a n d\;i f(T x,x)=0\,j o r\;e v e r y\;x\in H,\;i h e n\;T=0. $$ = 0. PRoor Since $(T(x+y),\;x+y)=0,$ we see that (1) $$ (T x,y)+(T y,x)=0\qquad(x\in H,y\in H) $$ If y is replaced by iy in(1), the result is (2) $$ -i(T x,y)+i(T y,x)=0\qquad(x\in H,y\in H). $$ Multiply (2) by iand add to (1), to obtain (3) $$ (T x,y)=0\qquad(x\in H,y\in H). $$ With $y=T x,(3)$ gives $\|T x\|^{2}=C$ D. Hence $T x=0.$ // Corollary f $S\in{\mathcal{B}}(H),\,T\in{\mathcal{B}}(H),$ and $$ (S x,\,x)=(T x,\,x) $$ for every xe H, then $S=T.$ PROO Apply the theorem to $S-T.$ Note that Theorem 12.7 would fail if the scalar field were ${\boldsymbol{R}}$ To see this, consider rotations in $R^{2}$ // 12.8 Theorem $\;U f\colon H\times H\to C$ is sesquilinear and bounded, in the sense that (1) $$ {\cal W}=\mathrm{sup} \{|f(x,y)|;\|x\|=\|y\|=1\}<\infty, $$ then there exists a unique $S\in{\mathcal{B}}(H)$ that satisfies (2) $$ f(x,y)=(x,S y)\qquad(x\in H,y\in H). $$ Moreover, $\|{\cal S}\|={\cal M}.$ PROOF Since $|f(x,y)|\leq M\|x\|\|y\|$ , the mapping $$ x\to f(x,y) $$ is, for each $y\in H,$ a bounded linear functional on $\textstyle H,$ of norm at most $M|y$ It now follows from Theorem 12.5 that to each $y\in H$ corresponds a unique element $S y\in H$ such that (2) holds; also, $\|S y\|\leq M\|y\|$ It is clear that $S\colon H\to H$ is additive. I $\textsf{f o r}c,$ then $$ (x,\,S(\alpha y))=f(x,\,\alpha y)={\tilde{\alpha}}f(x,\,y)={\tilde{\alpha}}(x,\,S y)=(x,\,\alpha S y) $$ for all x and $\mathbf{y}$ in $H.$ If follows that $\boldsymbol{S}$ 'is linear. Hence $S\in{\mathcal{P}}(H),\,\mathrm{and}\,\|S\|\leq M.$BoUNDED OPERATORs ON A HILBERT SPACE 297 But we also have $$ |f(x,y)|\,=\,|(x,S y)|\,\leq\,\|x\|\,\|S y\|\leq\,\|x\||S\|\,\|\,y\|, $$ which gives the opposite inequalit $M\leq\parallel S\parallel.$ $J/I J$ 12.9 Adjoints If $T\in{\mathcal{B}}(H),$ then $(T x,y)$ is linear in x, conjugate-linear in ${\mathit{y}},$ and bounded. Theorem 12.8 shows therefore that there exists a unique $T^{\bullet}\in{\mathcal{B}}(H)$ for which .(1) $$ (T x,y)=(x,T^{*}y)\qquad(x\in H,y\in H) $$ and also that (2) $$ \|T^{*}\|=\|T\|. $$ We claim that $T\to T^{*}$ is an involution on ${\mathcal{R}}(H).$ , that is, that the following four properties hold: (3) $$ (T+S)^{*}=T^{*}+S^{*}. $$ (4) $$ (x T)^{*}=\tilde{\alpha}{\cal I}^{r*}. $$ (5) $$ (S T)^{*}=T^{*}S^{*}. $$ (6) $$ T^{s s}=T. $$ Of these, $(3)$ is obvious. The computations $$ \begin{array}{l c r}{{(\mathcal{H}x,\,y)=\alpha(T x,\,y)=\alpha(x,\,T^{**}y)-(x,\,\bar{\alpha}T^{*}y),}}\\ {{(S T x,\,y)=(T x,\,S^{*}y)=(x,\,T^{**}S^{*}y),}}\\ {{(T x,\,y)=\displaystyle{\frac{1}{(T^{***}x)}}=(T^{***}x,\,y)=(T^{***}x,\,y)}}\end{array} $$ give(4),(5), and (6). Since $$ \|T x\|^{2}=(T x,T x)=(T^{*}T x,x)<\|T^{*}T\|\,\|x\|^{2} $$ for cvery $x\in H.$ we have $\|T\|^{2}\leq\|T^{*}T^{*}\|$ On the other hand, (2) gives $$ \|T^{\mathrm{s}}T\|\leq\|T^{\mathrm{s}}\|\|T\|=\|T\|^{2}. $$ Hence the equality (7) $$ \|T^{*}T\|=\|T\|^{2} $$ holds for every $T\in{\mathcal{B}}(H).$ ${\mathcal{R}}(H)$ is a $B^{\mathbf{x}}.$ 3*-algebra, relative to the involution $T\to T^{s}$ We have thus proved that defined by (ID298 BANACH ALGEBRAS AND SPECTRAL THEORY Note: In the preceding setting. $T^{\ast\ast}$ is sometimes called the Hilbert space adjoint of ${\boldsymbol{T}}.$ to distinguish it from the Banach space adjoint that was discussed in Chaptcr ${\mathrm{d}}.$ The only differcncc between the two is that in the Hilbert space setting $T\to T^{*}$ is conjugate-linear instead of linear. This is due to the conjugate-linear nature of the isometry described in Theorem 12. 1f $T^{\mathrm{{sk}}}$ * were regarded as an operator on $H^{\star}$ rather than on $\textstyle H,$ we would be exactly in the situation of Chapter 4. 12.10 Theorem $$ \begin{array}{l l}{{I f\,{\cal T}\in{\mathcal R}(H),\,t h e n}}&{{}}\\ {{\mathcal{A}(T^{*})=\mathcal{R}(T)^{\bot}}}&{{}}\\ {{\mathcal{A}(T^{*})=\mathcal{R}(T)^{\bot}}}&{{}}\end{array}\begin{array}{c}{{}}&{{}}\\ {{\mathcal{A}(T)=\mathcal{R}(T^{*})^{\bot}.}}\end{array} $$ We recall that ${\mathcal{N}}(T)$ and ${\mathcal{B}}(T)$ denote the null space and range of ${\cal T}_{\circ}$ respectively PROOF Each of the following four statements is clearly equivalent to the one that follows and/or precedes it (4) $$ \begin{array}{l}{{T^{*}y=0.}}\\ {{(x,\,T^{*}y)=0.}}\\ {{(T x,\,y)=0.5\mathrm{or\;every\;}x\in H.}}\\ {{y\in\mathcal{D}(T)^{\perp}.}}\end{array} $$ (1) (2) (3) Thus $\mathcal{A}(T^{*})=\mathcal{R}(T)^{\perp}.$ Since $T^{\bullet\varepsilon}=T,$ the second assertion follows from the first if ${\boldsymbol{T}}$ is replaced by $T^{\mathrm{sk}}$ // 12.11 Definition An operato $T\in{\mathcal{B}}(H)$ is said to be (a) normal i $T T^{*}=T^{*}T_{*}$ $T^{*}=T,$ (b)self-adjoint (or hermitin i (c) unitary it $T^{*}T=I=T T^{*}.$ where $\boldsymbol{\mathit{I}}$ is the identity operator on ${\boldsymbol{H}}$ Gd) a projecion if $T^{2}=T.$ lt is clear that self-adjoint operators and unitary operators are normal. Most of the theorems obtained in this chapter will be about normal operators. 12.12 Theorem Suppose $T\in{\mathcal{B}}(H).$ (a)T is normal if and only i $\|T x\|=\|T^{*}x\|\,.$ for cvery $x\in H$ (d)If ${\boldsymbol{T}}$ (b)If T is normal, then $\mathcal{N}(T^{\prime})=\mathcal{N}(T^{*})=\mathcal{B}(T)^{\perp}$ and $\alpha\in{\mathfrak{C}}.$ then $T^{*}x=\bar{\alpha}x.$ (c) If T' is normal and $T x=\textstyle{\frac{\alpha x}{\sqrt{0}}}T$ some $x\in H$ is normal and if α and $\beta$ are distict eigenvalues of T, then the corresponding eigenspaces are orthogonal to each other PROOF To see (a), combine the equalities $$ \|T x\|^{2}=(T x,\,T x)=(T^{*}T x,\,x) $$ $\|T^{*}x\|^{2}=(T^{*}x,T^{*}x)=(T T^{*}x,x)$poUNDED OPrRATORs ON A HLBERT SPACE 299 and 12.10.If (b)is aplied io $T\-\alpha I\operatorname{in}$ yihcolay horm . Obiousy (0 owstfo and Thecor $T x=\alpha x$ $T y=\beta y,$ an appication of ce givs place of T,(Co is obtained. Finally, i α(X $$ \displaystyle\cdot\!\cdot\!y\big)=(\alpha x,y)=(T x,y)=(x,T^{*}y)=(x,\bar{\beta}y)=\displaystyle\theta(x,y) $$ Since ${\mathfrak{A}}\neq\beta$ , we conclude that ${\mathfrak{X}}\perp$ y $J/\slash$ 12.13 Theorem 1 $U\in{\mathcal{B}}(H),$ the following three statements are equitvalent (a) U is unitary (b) ${\mathcal{R}}(U)=H$ and (Ux, Uy) = (×,J) orall x e H,ye H (c) ${\mathcal{R}}(U)=H$ and llUx|| IXllfor every xe H. PROOF If ${\boldsymbol{U}}$ is unitary,then ${\mathcal{R}}(U)=H$ because $U U^{\star}=I.$ AIso, $U^{*}U=I,$ so that $$ (U x,\,U y)=(x,\,U^{**}U y)=(x,\,y). $$ Thus (o) implies $\mathbf{\nabla}(b)$ ). It is obvious that(b) implics Gc) $\operatorname{F}\left(c\right)$ holds, then $$ (U^{**}U x,\,x)=(U x,\,U x)=\|\,U x\|^{2}=\|x\|^{2}=(x,\,x) $$ of for every $x\in H,$ so that $U^{\ast\ast}U=I.$ But (c implies also that Since $U^{*}U=I,\ U^{-1}=U^{*}$ , and $\textstyle H$ onto $\textstyle H,$ so that $U$ is invertible in ${\mathcal{O}}(H).$ $U$ is a linear isometry ${a\!\!\!/}/{j\!\!\!/}/$ therefore $U$ is unitar Note: The equivalence of $\mathbf{\Phi}(a)$ and(b) shows that the unitary operators are precisely those linear isomorphisms of ${\boldsymbol{H}}$ that also preserve the inner product. They are therefore the Hilbert space automorphisms The equivalenceor(O) and Cc is also a collary of Exercise2 12.14 Theorem Each o he followimg four properties f proiecti $\boldsymbol{P}$ e M(H) implies the other three: (a) Pis self-adjoin (b)P is normal. (c) $\mathcal{A}(P)=\mathcal{A}(P)^{\perp}.$ for every $x\in H.$ (d)(Px $x)=\|P x\|^{2}$ Property (c) is usually expressed by saying that $\boldsymbol{P}$ is an orthogonal projection implies Ge) rxoor t s trivial that Go mplies (b). Statement (b) o Theorem 12.12 shows is a rojection $\mathcal{A}(P)=\mathcal{A}(I-P),$ SO that .WC(P)= /2(PJ if Pis noral sinc ${\boldsymbol{P}}$ that M(P)is closd It now follows from the ollaryto Theorem 12.4 that (b)300 BANACH ALGEBRAS AND SPECTRAL THEoRY $P z=z$ If $\left(c\right)$ holds, every $x\in H$ has the form $x=y+z_{}$ ,with $\scriptstyle{\nu}\,{\boldsymbol{\cdot}}$ Z. $P y=0,$ Hence $p x=z,$ and $(P x,\,x)=(z,\,z).$ This proves (d) Finally, assume $(d)$ holds. Then $$ \|P x\|^{2}=(P x,x)=(x,P^{*}x)=(P^{*}x,\,x). $$ (d) implies (a) The last equality holds because $|P x|^{2}$ is real and $(x,P^{*}x)=\|P x\|^{2}$ Thus $(P x,\,x)=(P^{*}x,\,x)\d x)\d x=$ ,for every $x\in H,$ so that $P=P^{\sharp},$ by Theorem 12.7. Hence // 12.15 Theorem Suppose $S\in{\mathcal{B}}(H),\,a n d\,S$ is self-adjoint. Then $S T=0$ if and only if ${\mathcal{R}}(S)\perp{\mathcal{R}}(T)$ PROOF (Sx $T y)=(x,\,S T y).$ // This result will be most frequently used when both $\boldsymbol{S}$ and ${\mathbf{}}T$ are orthogonal projections A Commutativity Theorem and Let xand $\mathbf{\nabla}y$ be commuting elents in some Banch alebra with an involutin ti $x^{*}y^{*}=(y x)^{*}.$ Does it follow ${\mathcal{B}}(H)$ that $X$ then obvious that $\mathcal{A}_{\nu}^{\stackrel{ arrow\tilde{\nu}\tilde{\nu}}{\tilde{\nu}^{6}}}$ and $y^{\geq}$ commute,simply because and $\boldsymbol{y}$ are normal (Exercise 28) $y=x$ commutes with $y^{\mathbb{R}}\not\vdash$ Of course,the answer is negative whenever $\textstyle{\mathcal{X}}$ is not norma $X\mid$ But it can be negative even when both It istherfor anintersting fact that the answer s afirmatie if xisnormal) in rclativc to the involution furnished by the Hilbert space adjoint: If N∈ B(H) is normal, if T∈ gW(H), and if NT = TN, then $N^{\star}T=T N^{\star}.$ In fact, a more general result is true: 12.16 Theorem (Fuglede-Putnam-Roscnblum) Assume that $\varphi_{\alpha}K=1.$ N, $T\in{\mathcal{B}}(H),$ M and $\textstyle N$ V are normal, and (1) $$ W T=T N. $$ Then Λ $M^{\star}T=T N^{\star}$ PROOF Suppose first that $S\in{\mathcal{O}}(H).$ Put ${\cal V}=\,S\,-\,S^{\mathrm{i}},$ and define (2) $$ Q=\exp\,(V)=\sum_{n=0}^{\infty}\,\left(\frac{1}{n!}\right)V^{n}.~~~~~~~~~~~~. $$ Then $V^{\star}=-V.$ , and therefore (3) $$ {\cal Q}^{*}=\exp\left(V^{*}\right)=\exp\left(-V\right)={\cal Q}^{-1}. $$ Hence ${\boldsymbol{Q}}$ is unitary. The consequence we need is that (4) $$ \left\|\exp\left(S-S^{*}\right)\right\|=1\qquad\operatorname{for}\operatorname{every}S\in{\mathcal{B}}(H). $$BoUNDED OPFRATos ON A HLDERT SPACE 301 If (() hols, then $M^{k}T=T N^{k}$ for k = 1,2,3,..., by induction. Hence (5) $$ \exp\left(M\right)T=T\exp\left(N\right)\!, $$ or (6) $$ T=\exp\left(-M\right)T\exp\left(N\right). $$ Put $U_{1}=\exp{(M^{*}-M)},$ $U_{2}=\exp\left(N-N^{*}\right).$ Since ${\mathcal{M}}$ and ${\boldsymbol{N}}$ are normal, it follows from (G) that (7) $$ \exp\,(M^{*})T\exp\,(-N^{*})=U_{1}T U_{2}\,. $$ By (4) $\|U_{\mathfrak{p l}}\|=\|U_{2}\|=1$ 、So that $(7)$ implies (8) llexp(M*)7 exp(-V/T3TIT」 (9) We now def $$ \begin{array}{l}{{\mathrm{~ne~}}}\\ {{f(\cdot)=\mathrm{exp}\,(\lambda,M^{*})T\mathrm{exp}\,(-\lambda N^{*}).\quad(\lambda\in C).}}\end{array}\quad $$ every $\lambda\in C.$ The hyptheses of the theorem hold with $^{\mathrm{t}}\!\!M$ and ${\bar{\lambda}}N$ in place of $\mathcal{M}$ and $N.$ entire ${\mathcal{R}}(H)$ Therefore (8) implies that $\|f(\lambda)\|\leq\|T\|$ for every $\lambda\in V.$ Thus f is a bounded for Hence (9) becomes -valued function. By Liovile'steorem $3.32,f(\lambda)=f(0)=T,$ (10) $$ \exp\,(\lambda M^{*})T-T\exp\,(\lambda N^{*})~~~~~~(\lambda\in{\mathcal{C}}). $$ If we quate the coeficins of in (0)、 we obtan $M^{\ast}T=T N^{\ast}.$ // which are not shared by every Remark Inspctio of tisroof shows that it uses no properties ${\mathcal{R}}(H)$ $B^{\dagger*}$ algebra. This observation does not lead to a eneralization of the theorem, however,becausc of Thcorem 12.41 Resolutions of the Identity 12.17 Definition Let OD be a oaler in a se $\mathbb{Q}$ and le ${\boldsymbol{H}}$ be a Hilbert space ln tistin,aresoluion o h idniy o D) a mapi $$ E\colon\mathfrak{M}\to{\mathcal{B}}(H) $$ with the fowing properties (a $E({\mathcal{D}})=0$ 、E(Q2) = . (6) Each E(o)isa self-adjoint projection. (c) $E(\omega^{\prime}\cap\omega^{\prime\prime})=E(\omega^{\prime})E(\omega^{\prime\prime})$ , the $E(\omega^{\prime}\cup\omega^{\prime})=E(\omega^{\prime})+E(\omega^{\prime\prime})$ defined by (d) If o' $\omega^{*}={\mathcal{D}}$ and $y\in H,$ the set function $\scriptstyle{E_{x},\gamma}$ (e) For every $x\in H$ $E_{x,y}(\omega)=(E(\omega)x,y)$ is a complex mcasure on 90n$$ \qquad\qquad\qquad\qquad\qquad\qquad\qquad\qquad\qquad\qquad\qquad\qquad\qquad\qquad\qquad\qquad\qquad\qquad\qquad $$ 302 BANACH ALGEBRAS AND SPECTRAL THEORY When D is the o-algebra of all Borel sets on a compact or locally compact Hausdorff space,it is customary to add another requirement to $|e\rangle.$ Each $E_{x,y}$ should be a regular Borel measure.(This is automatically satisfied on compact metric spaces for instance. See [23].) HIerc are some immediate consequences of thesc propcrtics Since each $\scriptstyle{E(\omega)}$ is a self-adjoint projection, we have (1) $$ E_{x,\,x}(\omega)=(E(\omega)x,\,x)=\|L(\omega)x\|^{2}\qquad(x\in H) $$ so that each $E_{n,x}$ is a positive measure on Di whose total variation is (2) $$ \|E_{x,\,x}\|=E_{x,\,x}(\Omega)=\|x\|^{2}. $$ By (c), any two of the projections $\scriptstyle{E(\omega)}$ commute with each othe If $\omega^{\prime}\cap\omega^{\prime\prime}=\mathcal{D},\,(a)$ and $\left(c\right)$ show that the ranges of $E(\omega^{\prime})$ and E(o")are orthog- onal to each other (Theorem 12.15) By (d), $\boldsymbol{E}$ is finitely additive. The question arises whether $\boldsymbol{E}$ is countably additive, i.e., whether the series (3) $$ \*{\underbrace{\sum_{n=1}^{\infty}}}E(\omega_{n}) $$ converges, in the norm topology of ${\mathcal{R}}(H),$ to $E(\omega),$ whenever ${\boldsymbol{\sigma}}$ is the union of the disjoint sets $\omega_{n}\in\mathfrak{M}.$ Since the norm of any projection is either O or at least l, the partial sums of the serices (3) cannot form a Cauchy sequence, unless all bt finitely many of the $E(\omega_{n})$ are O. Thus $\boldsymbol{E}$ is not countably additive, except in some trivial situations Howcvcr, let $\scriptstyle[\omega_{k}]$ bc as abovc, and fix $x\in H.$ Sinc $E(\omega_{n})E(\omega_{m})=0$ when $n\not\in p n$ the vectors F(o,)x and $E(\omega_{m})x$ are orthogonal to each other (Theorem 12.15) By (e), (4) $$ \sum_{n=1}^{\infty}(E(\omega_{n})x,\,y)=(F(\omega)x,\,y)\, $$ for every p $\epsilon\,\l\,H\,$ It now follows from Theorem 12.6 that (5) $$ \sum_{n=1}^{\infty}E(\omega_{n})x=E(\omega)x. $$ The series (5) converges in the norm topology of ${\cal I}_{\O}$ We summarizc the result just proved: 12.18 Proposition I E is a resolution of the idlentiy, and $i f\,x\in H,$ then $$ \omega arrow E(\omega)x $$ is a countably additive H-valued measure on ${\mathfrak{I}}{\mathfrak{I}}.$noUNDED OPERATORs ON A HLBERT SPACE 303 Moreover, sets of measure zero can be handled in the usual way: 12.19 Proposition Suppose ${\boldsymbol{E}}$ is a resolution of the identity.IfomG D and $E(\omega_{a})=0$ for $n=1,2,3,\ldots.$ and if $\begin{array}{l}{{\mathcal{D}=\left(\begin{array}{l}{{\alpha}}\\ {{n=1}}\end{array}\right)_{n}^{\alpha},\ t h e n\ E(\omega)=0}}\end{array}$ PRoor Since $E(\omega_{n})=0$ $E_{x,x}(\omega_{n})=0$ for every $x\in H.$ Since $\scriptstyle n_{x,x}$ is countably additive, it follows that $E_{x,x}(\omega)=0$ But $\|E(\omega)x\|^{2}=E_{x,\,x}(\omega).$ Hence $E(\omega)=0.$ // $\boldsymbol{f}$ for which 12.20 The algebra $L^{\alpha}(E)^{*}\ \operatorname{Let}E$ be a resolution of thc identity on D,as above. Let of open subset of( $\textstyle{\mathcal{C}}$ be a complex Ut-measurable function on 2. There is a countable collectin Let ${\mathbf{}}V$ be the union of those $D_{i}$ open discs which forms a base for the topology of $\textstyle[D_{i}]$ ${\bar{C}}.$ $E(f^{-1}(D_{i}))=0.$ By Proposition 12.19,L $\zeta(f^{-1}(V))=0.$ Also, Vis the largest with this property The essential range of f is, by definition,the complement of ${\mathit{V}}.$ It is the smallest closed subset of 《 ${\boldsymbol{C}}$ that contains ${\mathcal{I}}(p)$ for almost all ${\mathfrak{p}}\in\Omega$ ,that is, for all $p\in\Omega$ except those that lie in some set $\omega\in\Re$ with $E(\omega)=0.$ We say that $\boldsymbol{\mathit{f}}$ compact. In that case, the largest value of $\left|{\boldsymbol{\lambda}}\right|$ s essenially bounded if its essential range is bounded, hence ,as Aruns through the essential range of Let $\boldsymbol{B}$ is called the essential supremum $\left\|f\right\|_{\infty}$ of f $\Omega{\mathrm{:}}$ $f,$ be the algebra of all bounded complex UR-measurable functions on with the norm $$ \|f\|=\operatorname*{sup}\left\{|f(p)|:p\in\Omega\right\}, $$ onc sccs casily that $\boldsymbol{\mathcal{B}}$ is a Banach algebra and that $$ N=\{f\in B\colon\|f\|_{\infty}=0\} $$ is an ideal of $\boldsymbol{B}$ which is closed, by Proposition 12.19. Hence $L^{\infty}(F).$ is a Banach algebra, ,and its spectrum $\sigma([f])$ distinction between f and its equivalence class $[f]$ ${\mathcal{B}}/N$ $|f|_{\alpha}$ which we denote (in the usual manner) by $[f]=f+N$ of $L^{\infty}(E)$ is then equal to The norm of any cose is thec csscntial range of f $\mathrm{A}\mathrm{s}$ is usually donc in mcasurc theory, the will be ignored. 12.21 Theorem If E is a resblution of the identity, as above, then the formula (1) $$ (\Psi(f)x,y)=\int_{\Omega}f d E_{x,y}\ \ \ \ \ (x\in H,\,y\in H) $$ defines an isometric isomorphism $\mathbf{\hat{P}}$ of the Banach algebra $L^{\infty}(E)$ onto a closed normal subalgebra A of %(H). This isomorphism ulso satisfies $\left(2\right)$ $$ \Psi(\bar{f})=\Psi(f)^{*}\;\;\;\;\;\;\;(f\in L^{\infty}(E)) $$304BANACH ALGEBRAS AND sPECTRAL THEORY and (3) $$ \|\Psi(f)x\|^{2}=\int_{\Omega}|f|^{2}\,d E_{x},x\mathrm{~\\\\\\(}x\in H,f\in L^{\infty}(E)). $$ Moreover, an operator $Q\in{\mathcal{B}}(H)$ commutes with every E(o) $i f$ and only if ${\mathcal{Q}}$ commutes with every $\Psi(f).$ Formula $\operatorname{\mathcal{(1)}}$ will sometimes be abbreviated to (4) $$ \Psi(f)=\int_{\Omega}f\,d E. $$ We recall that a normal subalgebra $\scriptstyle A$ 1 of ${\mathcal{A}}(H)$ is a commutative one which has the property that $T^{\bullet}\in{\mathcal{A}}$ whenever $T\in{\mathcal{A}}$ ; see Definition 11.24 PRO0F To begin with, le $\{o_{1},\cdot\cdot\cdot,\ o_{n}\}$ be a partition of $\Omega_{\circ}$ with $\omega_{i}\in{\mathfrak{M}},$ by and let s be a simple function, such that $s=\alpha_{i}$ on $\lbrack O_{i}\,.$ Define $\Psi(s)\in{\mathcal{D}}(H)$ (5) $$ \Psi(s)=\sum_{i=1}^{n}\alpha_{i}\,E(\omega_{i}). $$ Since each $E({\boldsymbol{\omega}}_{i})$ is self-adjoint, (6) $$ \Psi(s)^{*}=\sum_{i=1}^{n}\bar{\alpha}_{i}E(\omega_{i})=\Psi(\bar{s}). $$ 1f $\{\omega_{1}^{\prime},\,\ldots,\,\omega_{m}^{\prime}\}$ is another partition of this kind, and if $t=\beta_{j}$ on $\omega_{j}^{\prime}.$ ,then $$ \Psi(s)\Psi(t)=\sum_{i,j}\alpha_{i}\,\beta_{i}E(\omega_{i})E(\omega_{j}^{\prime})=\sum_{i,j}\alpha_{i}\,\beta_{j}\,E(\omega_{i}\cap\omega_{j}^{\prime}). $$ Since ${\mathbf{S}}I$ is the simple function that equals $x_{i}{\boldsymbol{\beta}}_{j}$ on $\omega_{i}\cap\omega_{j}$ ,it follows that (7) $$ \Psi(s)\Psi(r)=\Psi(s r). $$ An entirely analogous argument shows that (8) $$ \Psi(o s+\beta t)=\alpha\Psi(s)+\beta\Psi(t). $$ If $x\in H$ and $y\in H,$ (S) leads to (9) $$ ({\mit\Psi}^{\prime}(s)x,\,y)=\sum_{i=1}^{n}\alpha_{i}(E(\omega_{i})x,\,y)=\sum_{i=1}^{n}\alpha_{i}\,E_{x,\,y}(\omega_{i})-\int_{\Omega}s\,d E_{x,\,y}\,. $$ By (6) and (T), (10) $$ \Psi(s)^{*}\Psi(s)=\Psi(\bar{s})\Psi(s)-\Psi(\bar{s}s)=\Psi(|s|^{2}). $$ Hence (O) yields $\scriptstyle(11)$ $$ \|\Psi(s)x\|^{2}=(\Psi(s)^{*}\Psi(s)x,\,x)=(\Psi(|s|^{2})x,\,x)=\int_{\Omega}|s|^{2}\,d E_{x,\,x}, $$$$ \begin{array}{r l}{{\bar{\Psi}}}&{{}\qquad}\\ {{\bar{\Psi}}}&{{}\qquad}\\ {{\qquad.}}\end{array} $$ so that BoUNDED OPERAToRs oN A HLBERT SPACE 305 (12) $$ \|\Psi(s)x\|\leq\|s\|_{x},\|x\|, $$ by formula (2) of Section 12.17. On the other hand, it xe ${\mathcal{R}}(E(\omega_{j})).$ ). then (13) $$ \Psi(s)x=\alpha_{j}E(\omega_{j})x=\alpha_{j}x, $$ that since the projections $E(s_{i})$ have mually orthogonal ranges ${\mathfrak{I}}{\mathfrak{I}}{\mathfrak{I}}$ is chosen so $\left|\sigma_{j}\right|\,=\,\left|\left|s\right|\right|_{\infty},$ it folows from (12 and (13) that (14) $$ \|\Psi(s)\|_{\ }=\|s\|_{\infty}. $$ tions. Now suppose $f\in L^{\infty}(E).$ There is a sequence of simple measurable func ${\boldsymbol{S}}_{k}$ . that converges to f in the norm of $L^{\infty}(E)$ By (14), the corresponding operators $\Psi(s_{k})$ form a Cauchy sequence in $\Psi(f);$ it is easy to see hat $\Psi(f)$ does not convergent to an opcrator that we cal ${\mathcal{R}}(H)$ which is therefore norm- depend on the particular choice of $\langle i_{n}\rangle$ Obviously (14) leads to (15) $$ \|\Psi(f)\|=\|f\|_{\infty}\qquad[f\in L^{\infty}(E)]. $$ measure; Now (I) follows from (9)(with $\mathbf{S}_{k}$ in place of ${\boldsymbol{S}}),$ since each $E_{x,y}$ is a finite $\left(2\right)$ and (3) follow from (6) and (11); and if bounded measurable func tions $\boldsymbol{\mathit{f}}$ and $\scriptstyle{\mathcal{G}}$ are approximated, in the norm of $L^{\infty}(E)$ 、by simple measurable and ${\mathit{l}}.$ that ${\mathcal{O}}$ functions s and $\Psi$ is an isometric isomorphism of $L^{\infty}(E)$ into ${\mathcal{R}}(H).$ Since $L^{\infty}(E)$ ${\bar{t}},$ we see that $(7)$ and (8) hold with $\boldsymbol{f}$ and $\textstyle{\mathcal{G}}$ in place of $\boldsymbol{\mathsf{S}}$ Thus is complete, its image ${\mathcal{Q}}$ commutes with every $\textstyle{\mathcal{A}}$ ${\mathcal{A}}(H),$ because of (15) $\Psi_{(s)}$ $A=\Psi(L^{\infty}(E))$ is closed in Finally,if $E({\boldsymbol{\omega}}),$ ,then $\textstyle{\mathcal{Q}}$ commutes with // whenever sis simple, and therefore the approximation process sd above shows commutes with every member of The Spectral Theorem The principal assertion of the spectral theorem is that every bounded normal operator ${\boldsymbol{T}}$ 'on a Hilbert space induces (in a canonical way) a resolution ${\boldsymbol{E}}$ of the identity on the Borel subsets of its spectrum o(T) and that ${\boldsymbol{T}}$ can be reconstructed from ${\boldsymbol{E}}$ by an integral of the type discussed in Theorem 12.21. Alarge part of the theory of normal operators depends on this fact only if closed subalgebras $\textstyle A$ of ${\mathcal{B}}(H)$ lt should perhaps be stated explicitly that the spectrum ${\mathcal{B}}(H).$ In other words, $\sigma(T)$ of an operator if and $T-\lambda I$ T∈ .3(H) will always refer to the full algebra ${\mathcal{B}}(H).$ Sometimes we shall also be concerned with $\lambda\in\sigma(T)$ $T^{*}\in{\mathcal{A}}$ has no inverse in pwhich have the additional property that $\scriptstyle I\in A$ and is a whenever TeA.(Such algebras are sometims clld -agebras)Since ${\mathcal{R}}(H)$ $B^{\pm}.$ -algebra, Theorem 11.29 tells us, in this situation, that $\sigma(T)=\sigma_{A}(T)$ for every ${\boldsymbol{T}}:$ e A.306 BANACH ALCEBRAS AND SpECTRAL THEoRy Thus ${\boldsymbol{T}}$ has the same spectrum relative to all closed *-algebras in ${\mathcal{B}}(H)$ that contain ${\overline{{T}}}.$ Theorem 12.23 will be obtained as a special case of the following result, which deals with normal algebras of operators rather than with individual ones 12.22 Theorem 1f A is a closed normal subalgebra of !%(H)which contains the identity operator ${\mathit{I}},$ and if A is the maximal ideal space of A,then the following assertions are true: (a)There exists ${\boldsymbol{a}}$ unigue resolution of the identity $\boldsymbol{E}$ on the Borel subsets of A, that satisfies (1) $$ T=\int_{\Lambda}{\hat{T}}\,d E $$ for every ${\boldsymbol{T}}$ e A,where $\hat{T}$ is the Gelfand transform of T. (b) $E(\omega)\neq0$ for every nonempty open set $\omega\subset\Delta$ if and only if $\boldsymbol{\mathsf{S}}$ commutes (c) An operator $S\in{\mathcal{B}}(H)$ commutes with every $I\in{\mathcal{A}}$ with every projection $E(\omega).$ As in Theorem 12.21, formula (l) means that (2) $$ (T x,y)=\int_{\Delta}\hat{T}d E_{x,y}\qquad(x\in H,y\in H,T\in A).\qquad. $$ pRoor Since ${\mathcal{R}}(H)$ is a $B^{\cong}$ -algebra (Section 12.9),our given algebra $\textstyle A$ is a commutative $B^{\sharp}$ -algebra. The Gelfand-Naimark theorem 11.18 asserts therefore that $T\to{\hat{T}}\,{\hat{1}}$ is an isometric *-isomorphism of $\scriptstyle A$ onto $\ C(\partial)$ $E.$ Suppose ${\boldsymbol{E}}$ satisfies This Ieads to an easy proof of the uniqueness of (2). Since $\hat{T}$ T ranges over all of $C(\Delta).$ ), the assumed regularity of the complex Borel measures $E_{x,\;y}$ shows that cach $\scriptstyle{E_{x},r}$ is uniquely determined by (2); this follows from the uniqueness assertion that is part of the Riesz representation theorem ([23], Th. 6.19). Since, by definition (3) $$ (E(\omega)x,y)=E_{x,\,y}(\omega), $$ cach projcction $E(\omega)$ is also uniquely determined by (2) This uniqueness proof motivates the following proof of the existence of E. If $x\in H$ and $y\in H.$ , Theorem 11.18 shows that (4) T一→(Tx, y is a bounded linear functional on $C(\partial),$ of norm ≤lllyl, since $\|{\tilde{T}}\|_{\infty}=\|T\|.$nouNDrD OPERAToRs ON A HLBERT SrACE 307 complex Borel measures $\mu_{\alpha},$ on $\Lambda,$ The Rie ,rentin horm upis u eore withuniu egula such that (5) $$ (T x,y)=\int_{A}{\hat{T}}\,d\mu_{x,y}\qquad(x\in H,y\in H,\,T\in{\mathcal{A}}). $$ When ${\hat{T}}$ is rea ${\boldsymbol{T}}$ is ser-adjon, so tha $(T x,y)$ and $(T y,\,x)$ are complex conjugates of each other. Hcnce (6) $$ \mu_{x,y}={\widetilde\mu}_{y,x}\qquad(x\in H,\,y\in H).\qquad. $$ Borel se For fixéd $T\in A,$ he e se inar $\textstyle{\mathcal{X}}$ and couatiearin is, for every The uniqueness o th mcasurec $\textstyle\mu_{x},y$ implies therefore that $\mu_{x,y}(\omega)$ it folows that $\omega\subset\Delta,$ a sesquilinear functional. Since $\|\mu_{x,y\|}\leq\|x\|\,\|y\|,$ (の $$ \coprod_{a}f d\mu_{x,y} $$ is a bounded squilinear functional on $\textstyle H,$ for every bounded Borel function fon $\Phi(f)\in{\mathcal{B}}(H)$ A By Theorem . tecoresons ey sc aniopeat (8) $$ (\Phi(f)x,y)=\int_{\Lambda}f d\mu_{x,y}\ \ \ \ \ (x\in H,y\in H). $$ Comparison with (S) shows that (9) $$ \Phi(\hat{T})=T\qquad(T\in{\cal A}). $$ Thus $\bar{\Phi}$ is an extension of the mapping $\mathbf{i}\mathbf{S}$ the equalit is self-adioint is the complex conjugate of ${\bar{A}}.$ I $\boldsymbol{\mathit{f}}$ is real, then(6) shows tha ${\hat{T}}\to T{\mathrm{~that}}$ takes $C(\Delta)$ onto $(\Phi(f)y,\,x).$ This implies that $(\Phi(f)x,y)$ Our next objective i $\Phi(J)$ (10) $$ \Phi(J g)=\Phi(J)\Phi(g), $$ If for bounded Borel functions f and $T\in A,$ then $(S T)^{\times}=\tilde{S}\hat{T},$ and $\left(5\right)$ implies ${\mathcal{G}}.$ $S\in{\mathcal{A}}$ and (11) $$ \int_{\mathop{\Delta}{\delta}}\hat{S}\hat{T}\,d\mu_{x,y}=(S T x,\,y)=\int_{\mathop{\Delta}}\hat{S}\ d\mu_{T x,y}\,. $$ Since ${\hat{A}}=C(\Delta),$ it follows that (12) ${\big.}{\mathcal{f}}.$ Hence $X,\,y_{!}$ and ${\boldsymbol{T}}.$ The integrals (Il) remain therefore equal if $\hat{\boldsymbol{S}}$ is for ever coic $$ ,\quad\tilde{T}d\mu_{x,y}=d\mu_{T x,y} $$ replaced by (13) rdn. rdlur. = (0C/)Tx,)=(T7,2)= Tdn,308 BANAC CTRAL THEOR ${\mathcal{G}}.$ where $z=\Phi(f)^{*}y$ The same reasoning as above shows that the first and las Consequently integrals in(13) remain equal when $\hat{T}$ is relaccd by any bounded Borel function (14) $$ \begin{array}{c}{{(\Phi(f g)x,y)=\int_{\Delta}f g\,d\mu_{x,y}=\int_{\Delta}g\,d\mu_{x,z}}}\\ {{=(\Phi(g)x,z)=(\Phi(f)\Phi(g)x,y)}}\end{array} $$ which proves (10). We are ready to detfine $E.$ $[\![\ M]_{x},[\:]\!\![\frac{}{}_{..\!}$ is a Borel subset of $\Delta,$ , let f be the character- $\mathrm{{It}}$ istic function of ${\cal O}.$ and put $E(\omega)=\Phi\,(f).$ With $\omega^{\prime}=\omega,$ this shows that each $\scriptstyle{E(\omega)}$ By (10), E(o $\cap\ \omega^{\prime})=E(\omega)E(\omega^{\prime}).$ That $E(\Delta)=I$ follows from(9). The finite is self-adjoint. is a projection. Since O(f) is self-adjoint when fis real, each $\scriptstyle t i(\omega)$ is clear that $E({\mathcal{G}})=\Phi(0)=0$ additivity of ${\boldsymbol{E}}$ is a conscqucncc of(8), as is the relation (15) $$ (E(\omega)x,y)=\mu_{x,y}(\omega). $$ Hence ${\boldsymbol{E}}$ is a resolution of the identity. is open and $E(\omega)=0.$ Ir $T\in{\mathcal{A}}$ and ${\hat{T}}$ 'has its suppor The proof of part (a) is now complete sinc (2) follows from (S) and (15) Suppose next that c ${\boldsymbol{\omega}}$ in o,(1) implies that $T=0$ ; hence ${\hat{T}}=0$ Since ${\dot{A}}=C(\Delta)$ Urysohn's lemma implies now that $\omega={\mathcal{D}}.$ This proves (b). , and put $z=S^{*}y.$ For any To prove Cc), choose $S\in{\mathcal{B}}(H),\ x\in H,\ y\in H,$ $\scriptstyle T\in{\mathcal{A}}$ and any Borel sct $\omega\subset\Delta$ we then have (16) $$ (S T x,y)=(T x,z)=\int_{A}{\hat{T}}\,d E_{x,z}, $$ (17) $$ (T S x,y)=\int_{\Lambda}{\hat{T}}\,d E_{S x,y}, $$ (18) $$ (S E(\omega)x,y)=(E(\omega)x,z)=E_{x,x}(\omega), $$ (19) $$ (E(\omega)S x,y)=E_{S x,y}(\omega). $$ that If $S T=T S$ for every $T\epsilon\ d.$ the measures in(16) and (17) are equal, so $S E(\omega)=E(\omega)S$ The same argument estabishesthe converse. This completes the proof. / We now specialize this theorem to a single operator 12.23 Theorem 1 $T\in{\mathcal{B}}(H)$ and ${\cal T}\,$ tis normal, then there exisis a unique resolution of the identity ${\boldsymbol{F}}$ on the Borel subsets of $\scriptstyle{\sigma(T)}$ which satisfies (1) $T=\int_{e(T)}\lambda\,d E(\lambda).$BOUNDED OPERATOR SPACE 309 mutes with Furhermore, every projection E(o) commutes ith e $S\in{\mathcal{P}}(H)$ which com- ${\boldsymbol{T}}.$ We shall refer to this $\boldsymbol{E}$ E as the spectral decomposition of T. as being defined orallBoel sets n ${\cal{C}};$ to achieve this, put Sometims it is onvenient to thinko if o $\vdash\sigma(T)=\emptyset.$ $\boldsymbol{E}$ $E(\omega)=0$ $T^{\setminus{\mathfrak{X}}}.$ for ever maximal ideal space of $\textstyle{\mathcal{A}}$ bc the smallst closed subalgebra ot $\boldsymbol{E}$ E follows now from Theorem 12.22 that contains ${\mathit{I}},$ ${\cal T},$ and PROOF Let $\scriptstyle{\dot{A}}$ ${\mathcal{B}}(H)$ By Theorem 11.19, the Since ${\cal T}\,$ is normal, Theorem 12.22 applies to $\scriptstyle{\mathcal{A}}$ $\vee\lambda\in\sigma(T)$ d can be identified with $\sigma(T)$ in such a way that ${\hat{T}}(\lambda)=\lambda$ The existence of On the other hand, if ${\boldsymbol{E}}$ xists so that (I) holds, Thcorem 12.21 shows that (2) $$ p(T,T^{*})=\int_{\sigma(T)}p(\lambda,\bar{\lambda})\,d E(\lambda), $$ $T,$ where $\boldsymbol{\mathit{P}}$ is any polynomial in two variables (with complex cefients)By the jections $\scriptstyle{E(\omega)}$ Stone-Werstasthcorem, these polnomials are dense in CoT). The pr are thereforeuniquely determined by the integrais (2). hence by jut as in the uniqueness proof in Theorem 12.22 If $S T=T S.$ then also ${\bar{A}}.$ By (c) of Theorem 12.22 $S E(\omega)=E(\omega)S$ S commutes // Borel set $\omega\subset\sigma(T).$ $S T^{*}=T^{*}S,$ by Theorem 12.16; hence $\boldsymbol{\mathsf{S}}$ for every with every member of 12.24 The symbolic calculus for normal operators r ${\boldsymbol{E}}$ is the spectral dccom- position of a normal operator $T\in{\mathcal{P}}(H),$ and if fis a bounded Borel function on c(T) it is customary to denote the operator (1) $$ \Psi(f)=\int_{\sigma(T)}f d E $$ by $J(T).$ Using this notation, part of the contcnt of Theorems 12.21 to 12.23 an be sum- marized as follows: The mapping $f\to f(T)$ is a homomorphism of the algebra of all bounded Borel which carries the iden- functions on o(T) into !3((H), which crries the function 1 to ${\boldsymbol{J}},$ tity function on $\sigma(T)/\sigma\;T$ , and which satisfies (2) $$ f(T)=f(T)^{*} $$ and 3) l F(T)H≤ sup $\{|f(\lambda)|;$ $\lambda\in\sigma(T)\}$ lffe C(o(T)), hen equality holds in 3)310 BANACH ALGEBRAS ANpD sPECTRAL THEoRY 1/ /→f umiformly, then $\|f_{n}(T)-f(T)\|\to0,$ as $n\to\Omega0.$ If S∈ .38(H) and $S T=T S.$ then $S f(T)=f(T)s$ S for every bounded Borel function f Since the identity function can be uniformly approximated, on o(T")、 by simple Borel functions, it followis that ${\boldsymbol{T}}$ is a limi, in the norm topology of $\mathcal{B}(H),$ of fnite linear coinbinations of projections $\scriptstyle{E(\omega)}$ The following proof contains our first application of this symbolic calculus 12.25 Theorem $I f~T\in{\mathcal{B}}(H)$ is normal, then $$ \|T\|=\operatorname*{sup}\left\{|(T x,x)|:x\in H,\|x\|\leq1\right\}. $$ PROOF Choose $\scriptstyle\pi>0.$ lt is clearly enough to show that (1) $$ |(T x_{0}\,,\,x_{0})|>\|T\|-\varepsilon $$ for some $x_{0}\in H$ vith $\|x_{0}\|=1.$ ${\mathit{l}}_{\cdot}$ then $\mathbf{(}b\mathbf{)}$ of Theorem $12.22$ implies that $E(\omega)\ne0.$ Sincc $\|T\|=\|{\hat{T}}\|_{\infty}=\rho(T)($ (Theorem 11.18), thcre exists $\lambda_{0}\in\sigma(T)$ such that $|\lambda_{0}|=\|T\|.$ Let o be the set of al $\cdot\in\sigma(T)$ for which $|\lambda-\lambda_{0}|<\varepsilon$ If ${\boldsymbol{E}}$ is the spectral decomposition of Therefore there exists $x_{0}\in H$ with $\|x_{0}\|=1$ and $E(\omega)x_{0}=x_{0}\,.$ 、Then Define $:f(\lambda)=\lambda-\lambda_{0}$ for $\lambda\in\omega\,;$ put $f(\lambda)=0$ for all othe $\lambda\in\sigma(T)$ $$ f(T)=(T-\lambda_{0}I)E(\omega), $$ so that $$ f(T)x_{0}=T x_{0}-\lambda_{0}\,x_{0}\,. $$ Hence $$ |(T x_{0}\,,\,x_{0})-\lambda_{0}|=|(f(T)x_{0}\,,\,x_{0})|\leq\|f(T)\|\leq\varepsilon, $$ since $|f(\lambda)|<\varepsilon$ for al $\lambda\in\sigma(T)$ This implies(1), because $|\lambda_{0}|=|T|.$ // 12.26 Thcorcm A normal T∈ ${\mathcal{B}}(H)$ is (a)self-adjoint if and only if G(T) lies in the real axis, (b)unitary if and only i $\scriptstyle{\sigma(T)}$ lies on ihe uni circle. PRO0F Choose $\textstyle{\mathcal{A}}$ as in the proof of Theorem 12.23. Then $T(\cdot)=\lambda$ and $(T^{*})^{\sim}(\lambda)=\bar{\lambda}$ on o(T),Hence $\sigma(T)$ if and only if $\lambda=\bar{\lambda}$ on $\sigma({\boldsymbol{r}})$ and $T T^{*}=I\,\mathbf{i}\Gamma$ /// $T=T^{*}$ and only if $\lambda\bar{\lambda}=1$ on 12.27 Invariant subspaces A closcd subspacc $M.$ For example, every eigenspace ot ${\boldsymbol{T}}$ $\bar{M}$ of $\textstyle H$ is an invariant subspace of a set $\textstyle\sum\subset{\mathcal{B}}(H)$ if every $T\epsilon\geq{\frac{1}{2}}$ maps $\bar{M}$ into the spectral theorem implies that is an invariant subspace of ${\boldsymbol{T}}.$ When dim $H<\infty.$BoUNDED OPERAToRs oN $\mathrm{\boldmath~A~}$ HLBERT SPACE 311 1 ln fact.,e $\textstyle A$ the cigenspaces of every normal operator $\sigma(T)$ corresponds to a projection in $H.$ The sum of these $\textstyle A$ istic function of each point in ${\boldsymbol{T}}$ span $H.$ ISketch of proof: The character ${\mathbf{\nabla}}_{i}\mathbf{\mathbf{M}}$ projections is $\scriptstyle\phi({\mathfrak{o}})$ and $\scriptstyle{\varphi H}$ If dim $H=\infty$ , it can happen that ${\cal{M}}$ is invariant under $A.$ (that is, $E(\sigma(T))=I.]$ ${\boldsymbol{T}}$ has no eigenvalues CExercise 2)、 But normal opratsil avinvaintsbsacs tatare ntivi and $\varphi^{\prime}$ be the ranges of $F(w)$ and be a normal alebra as nTheorem12 and le $\Delta$ consists of a singepont the $\boldsymbol{E}$ beis esolutio $T\in{\mathcal{A}}$ $\hat{\Gamma}\ X\in M,$ of the icntiy onthe Borel sbses o' L. 1r and every subspaceo for every Suppose that consis of th scar multiples o ${\boldsymbol{J}},$ are nonempty disjoint Borel sets. Let $\Delta=\omega\cup\omega^{\prime}$ where o and ${\boldsymbol{\omega}}^{\prime}$ it folows that $E(a)$ . Then $T E(\omega)=E(\omega)T$ $$ T x=T E(\omega)x=E(\omega)T x, $$ so that $T x\in M$ The same holds for $\ {\mathcal{M}}^{\prime}$ Moreover, $M^{\prime}=M^{\perp},$ and are invarian suspaces o ${\mathcal{A}}.$ into pairwise orthogonal Hence $\mathcal{M}$ T and $\ M^{\prime}$ Decompsitionso ${\cal H}=M\oplus M^{\prime}.$ ${\cal{M}}$ has a nontrivial invariant subspace if $\textstyle H$ $\Delta$ into fity ay (or even countably many isoint ore $T\in{\mathcal{B}}(H)$ invariant subspaces of sets induc, inthe same manner, decmpoitions O ${\cal A}.$ lt is an open problem whether every (nonnormaD is an infinit-dimensional separable Hibert spac Eigenvalues of Normal Operators If $T\in{\mathcal{B}}(H)$ is noral it eienvalues bara simle rlaton to is spetral deco- symbolic calculus: poitin Thorm 22).Tis i deiverformie oingapiaio t 12.28 Theorem Suppose $T\in{\mathcal{B}}(H)$ is normal and $\boldsymbol{E}$ is its spectral decomposition I/厂∈ $C({\boldsymbol{\sigma}}(T))$ and $i f\,\omega_{0}=f^{-1}(0),$ then (1) $$ \mathcal{A}(f(T))=\mathcal{R}(E(\omega_{0})). $$ PROOF Put $g(\cdot)=1$ on $\omega_{0}\,,\,g(\lambda)=0$ at all other points of $\sigma(T)$ . Then $f g=0_{i}$ so that $f(T)g(T)=0.$ Sin $\operatorname{lcc}g(T)=E(\omega_{0})_{!}$ ,it follows uhat (2) $$ {\mathcal{R}}(E(\omega_{0}))\subset{\mathcal{N}}(f(T)). $$ compact set $\scriptstyle m_{0}$ is thc complement of ${\mathcal{O}}_{0}$ , relative to $\sigma(T)_{i}$ then $\tilde{\cal O}$ is the union of dis- If $\tilde{\omega}$ joint Borel sets $\omega_{n}$ $(n=1,2,3,\ldots),$ each of which has positive distance from the .Define $({\mathcal{3}})$ $$ f_{n}(\lambda)= \{{\!0}^{1/f(\lambda)}\quad\quad\textrm{o n}_{\omega_{n},}\quad\quad{\!\!-} $$312 nANACH ALGEBRAS AND sPrCTRAL THroRv Each $\textstyle{\iint_{n}}$ is a bounded Borel function on $\sigma(T)_{*}$ and (4) $$ f_{n}(T)f(T)=E(\omega_{n})\qquad(n=1,\,2,\,3,\,\ldots). $$ If $E(\vec{\omega})+E(\omega_{0})=I.$ Hence it follows that $E(\omega_{n})x=0$ . The countable additivity of the $E({\bar{\omega}})x=0.$ But mapping $\textstyle f(T)x=0,$ x (Proposition 12.18) shows therefore that $\omega arrow E(\omega)>$ $E(\omega_{0})x=x.$ we have now proved that (5) $$ \mathcal{N}(f(T))\subset\mathcal{B}(E(\omega_{0})), $$ and (I) follows from $\mathbf{\Psi}(2)$ and (5). // 12.29 Theorem Suppose ${\boldsymbol{E}}$ is the spectral decomposition of a normal $T\in{\mathcal{B}}(H),$ $\lambda_{0}\in\sigma(T).$ and $E_{0}=E(\{\lambda_{0}\})$ Then (b) $\lambda_{0}$ $\mathcal{N}(T-\lambda_{0}I)=\mathcal{B}(E_{0}),$ if and only if $E_{0}\neq0,$ and has a (a) is an eigenvalue of ${\mathbf{}}T$ is a countable set, then every $x\in H$ (c) every isolated pon of OGTD is an ienvlue o/ T (u) Moreover, if $\sigma(T)=\{\lambda_{1},\lambda_{2},\lambda_{3}\,,\ldots\}$ unique expansion of the form $$ x=\sum_{i=1}^{\infty}x_{i}, $$ where $T x_{i}=\lambda_{i}x_{i}$ Aiso. $x_{i}\perp x$ y whemever $i\neq j.$ Statements (b) and ce explain the term point spectrum o ${\boldsymbol{T}}$ T for the set of all eigen- values of ${\boldsymbol{T}}.$ PROOF a noncmpty open subset of $\sigma(T);$ is an immediate corollary of Theorem 12.28, $\mathrm{with}f(\lambda)=\lambda-\lambda_{0}\,.$ is Part $\mathbf{\Psi}(a)$ o is an isolated point of $\sigma(T),$ then $\scriptstyle \langle{\mathfrak{x}}_{0} \rangle$ It is clar that (b) follows from (a). I $\lambda_{0}$ by (b) of Theorem 12.22. There hence $E_{0}\neq0,$ fore $\left(c\right)$ follows from (b) limit points $\lambda_{i}$ of $\sigma(T),$ $\textstyle E_{i}$ ranges. To prove $\langle{\hat{a}}\rangle,$ put $E_{i}=E(\{\lambda_{i}\}),\,i=1,2,3,\ldots\cdot\mathbf{A}[$ have pairwise orthogonal may or may not be O. In any case, the projections $\textstyle E_{i}$ .The countable additivity of $\omega arrow E(\omega)x$ (Proposition 12.18) shows that $$ \begin{array}{r l r l}{{\frac{\sigma}{2}}E_{i}x=E(\sigma(T))x=x\,}&{{}\quad(x\in H).}\end{array} $$ $\lambda_{\nu}^{\star}{\mathrm{\frac{*}{k}}}\underline{{{y}}}$ and The series converges, in the norm of $\textstyle H.$ This gives the desired reprcscntation of $J/{\big/}/$ ${\mathcal{X}},$ if $x_{i}=E_{i}x.$ The uniqenssflowsfrom te orthonality of the ectors $T x_{i}=\lambda_{i}x_{i}$ follows from (a) 12.30 Theorem A.normal operator $T\in{\mathcal{B}}(H)$ is compact famd onlyif it satis festhe following two conditionsBoUNDED OPERATORs ON A HILBERT SPACE 313 (a) $\sigma(T)$ has no limit poin cxcept possiby O (b) $I/\lambda\neq0$ ,then dim ${\mathcal{N}}(T-\lambda I)<\infty.$ $\lambda=\lambda_{i}$ PRoor For thencessty, se and $\mathrm{put}\,f_{n}(\cdot)=0$ $(d)$ of Theorem $4.18,$ and Theorem 4.25. define $f_{n}(\cdot)=\lambda$ if and of the nonzero points of To prove the sufficiency, assume (o) and (b) hold,let $\scriptstyle \langle{\lambda_{a}} \rangle$ be an enumeration as in Theorem 12.29, then $\sigma(I)$ such that $\vert\lambda_{1}\vert\geq\vert\lambda_{2}\vert\geq\cdots,$ $\sigma(T).$ If $E_{i}=E(\{i_{i}\}),$ $i\leq n.$ at the other points of $$ \begin{array}{r l}{\quad,{\quad}f_{n}(T)=\lambda_{1}E_{1}+\cdots+\lambda_{n}E_{n}\,.}\end{array} $$ Since dim $\mathcal{A}(F_{i})=\mathrm{dim}\;\mathcal{A}(F-\lambda_{i}I)<\infty,$ each $f_{s}(T)$ is a compact operator. Since $|\lambda-f_{n}(\lambda)|\leq|\lambda_{n}|$ for all $\lambda\in\sigma(I),$ we have $$ \|T-f_{n}(T)\|\leq\left|\lambda_{n}\right|\to0\quad\mathrm{as}\quad n\to\infty. $$ It now follows from (c) of Theorem 4.18 that ${\boldsymbol{T}}$ is compact // 12.31 Theorem Suppose $T\in{\mathcal{B}}(H)$ is normal and compact. Then (a) T has an eigenvale ${\boldsymbol{\bar{\lambda}}}$ with $|\lambda|=|T|,$ ,and (b) ${\mathcal{I}}(t)$ is compact $i f f\in C(\sigma(T))\;a n d f(0)=0.$ PROOF Since $\underline{{T}}$ tis normal, Theorem $\scriptstyle1{\mathrm{~}}18$ shows that there exists $\lambda\in\sigma(T)$ with $|\lambda|=\|T\|$ Since $\sigma(T)$ is an at most countable compact set in $\|T\|=0,$ (a) is obvious. $\mathbf{I}\mathbf{f}$ $\|T\|>0.$ this $\lambda$ is an isolated point of $\sigma(T)$ (Theorem 12.30), hence an eigenvalue of ${\boldsymbol{T}}$ (Theorem 12.29). If $Q_{\mathrm{{,}}}$ its complement is connected. Mergclyan's theorem(Ssee 【23) shows therefore that there are polynomials $p_{n}:$ ,with $p_{n}(0)=0$ which converge to $f,$ uniformly on $\sigma(T)$ The operators $p_{a}(T)$ converge therefore, in the norm of ${\mathcal{B}}(H),\operatorname{to}f(T).$ Since $p_{n}(0)=0,$ (f) of Thcorcm 4.18 shows that $\cosh p_{n}(T)$ is compact. Hence $\textstyle f(T)$ is compact, $J i J$ by (c) of Theorem 4.18. This proof of $\mathbf{\nabla}(b)$ could also have been based on thc classical approximation theorem of Runge rather than on the more diffcult one of Mergelyan Positive Operators and Square Roots 12.32 Theorem Suppose $T\in{\mathcal{P}}(H)$ Then (a) $(T x,x)\geq0$ for every $x\in H$ if and only if (b) $T=T^{*}$ and o(T)c [0, o) If Te &B(H) satisfies (a), we call ${\boldsymbol{T}}$ a positive operator and write $T\geq0.$ The theorem asserts that this terminology agrees with Definition 11.27314 BANACH ALGEBRAS AND SPECTRAL THEORY PROOF In general, $(T x,x)$ and $(x,T x)$ are complex conjugates of each other But if (a) holds, then $(T x,\,x)$ is real, so that $$ (x,T^{*}x)=(T x,x)=(x,T x) $$ for every $x\in H.$ By Theorem 12.7, $T=T^{\bullet},$ and thus $\sigma(I)$ lies in the real axis (Theorem 12.26).If ${\boldsymbol{\lambda}}>0.$ (a) implies that $$ \lambda\|x\|^{2}=(\lambda x,x)\leq((T+\lambda I)x,\,x)\leq\|(T+\lambda I)x\|\Vert x\|, $$ so that $$ \|(T+\lambda I)x\|\geq\lambda\|x\|. $$ Hence $T+\lambda I$ is invertible in ${\mathcal{B}}(H)$ , and $\mathbf{\Psi}\cdot\mathbf{\Psi}\cdot\bigwedge$ is not in ${\boldsymbol{\sigma}}({\boldsymbol{I}})$ It follows tha $\mathbf{\Psi}(a)$ implies (b) Assume now that (b) holds, and let $\boldsymbol{E}$ be the spectral decomposition of $T,$ so that $$ (T x,x)=\int_{\sigma(T)}\lambda\,d E_{x,\,x}(\lambda)\qquad(x\in H). $$ Since each $E_{\alpha,\alpha}$ is a positive measure, and since $\scriptstyle\lambda\geq0$ on $\sigma(T)$ )、 we have // $(T x,x)\geq0.$ Thus $\mathbf{\nabla}(b)$ implies (a). 12.33 Theorem Every positive T∈ 9(H) has a unique positive square roo1 $S\in{\mathcal{P}}(H).$ I/ T is invertible, so is ${\boldsymbol{S}}.$ PROOF Let $\scriptstyle A$ be any closed normal subalgebra of ${\mathcal{B}}(H)$ that contains $\boldsymbol{\mathit{I}}$ and $T,$ and Iet $\Delta$ be the maximal ideal space-of A. By Theorem l1.18, ${\hat{A}}=C(\Delta).$ Since ${\mathbf{}}T$ satisfies condition $\mathbf{\nabla}(D)$ of Theorem 12.32, and since $\sigma(T)={\hat{T}}(\Delta).$ we see that $T\geq0,$ Since every nonnegative continuous function has a unique nonnegative continuous square root, it follows that there is a unique $\mathbf{S}\in{\mathcal{A}}$ that satisfies $S^{2}=T$ and ${\bar{S}}\geq0;$ by Theorem 12.32, ${\bar{\S}}\geq0$ is equivalent to $S\geq0.$ Then there In particular, let ${\bar{A}}_{0}$ be the smallest of these algebras ${\bar{A}}.$ exists $S_{0}\in A_{0}$ such that $S_{0}^{2}=T$ and ${\bf S_{0}}\geq0.\;\;\mathrm{If}\;{\cal S}\in{\mathcal B}(H)$ is any positive square Thcn $T\in{\mathcal{A}},$ since $T-S^{2},$ t be the smallest closed subalgebra of ${\mathcal{R}}(H)$ that contains land ${\boldsymbol{S}}.$ root of T, let $\scriptstyle{\dot{A}}$ Hencc $A_{0}\subset A$ , so that $S_{0}\in A.$ The conclusion of the preceding paragraph shows now that $S=S_{\mathrm{o}}$ Finally, if ${\boldsymbol{\mathit{1}}}^{\prime}$ is invertible, then $S^{-1}=T^{-1}S$ since $\mathbf{S}$ ' and ${\cal T}\,$ commute ${}^{\kappa}{}_{\lambda}$ $I/I/$ 12.34 Theorem If Te 8(II), then the positive square root of T*T is the only posi- tive operator Pe W(H) that satisfies |IPxl =|Txl for every xe $\textstyle H.$roUNDED OrtxAros oN A Hu BtRr SPAce 315 PRo0r Note first tha (1) $$ (T^{*}T x,\,x)=(T x,\,T x)=||T x||^{2}\geq0\qquad(x\in H), $$ so that $T^{*}T\geq0.$ (In the more abstact seting of Theorem 11.28 this was much harder to prove!) Next, if $P\in{\mathcal{B}}(H)$ and $P=P^{\kappa}.$ , the (2) $$ \cdot\quad(P^{2}x,\,x)=(P_{X},\,P_{X})=||P x||^{2}\qquad(x\in H) $$ By Theorem $\,\cdot{12}{\mathcal{I}},$ it follows that $\|P x\|=\|I x\|$ for ever $x\in H$ if and only if ${\boldsymbol{P}}^{2}={\boldsymbol{T}}^{*}{\boldsymbol{T}}.$ This completes the proof // of where The facthat every complex number $\bar{\lambda}$ can be factored in the form $\lambda=\alpha|{\boldsymbol{\lambda}}|,$ with $U$ $|\alpha|=1$ , sugests the problem of trying to facto $T\in{\mathcal{B}}(H)$ in thc form $T\simeq U P,$ ${\boldsymbol{T}}.$ unitary and $P\geq0.$ When this is possible, we call ${\boldsymbol{U}}{\boldsymbol{P}}$ a polar decomposition ${\boldsymbol{P}}$ is uniquely determined by ${\boldsymbol{T}}.$ beingunitry is nisometry. Thorem 1.34 shows thefore ha Note that $U,$ 12.35 Theorem (a) 1/ Te ${\mathfrak{s}}{\mathcal{B}}(H)$ is invetible, then ${\mathbf{}}T$ has a unigue polar decomposii $T=U P.$ and ${\boldsymbol{\mathit{P}}}$ (b) If T'∈ ${\mathcal{R}}(H)$ is ormal,then ${\boldsymbol{T}}$ has a polar decompositio $T=U P i n$ which ${\boldsymbol{U}}$ commute with each other and with ${\boldsymbol{T}}.$ invertible, and rRoor(a)If T's nvetble, so are o $T^{\ast}T$ is also invertible. Put $U=T P^{-1}$ Then ${\boldsymbol{U}}$ J is the positive square root $T^{\bullet\sharp}$ and $T^{*}T_{\mathrm{r}}$ and Theorem 12.33 shows tha $D\!\!\!\!/$ $$ {\cal U}^{*}{\cal U}={\cal P}^{-1}T^{*}T{\cal P}^{-1}={\cal P}^{-1}{\cal P}^{2}{\cal P}^{-1}=I, $$ so that $U$ is unitary. Since $U.$ is invertible, it is obvious that $\scriptstyle T P^{-1}$ is the only ${\boldsymbol{P}}$ possible choice for relation (b) Put $p(\lambda)=\left|\lambda\right|,\ u(\lambda)=\lambda/\left|\lambda\right|\lambda\left|\begin{array}{\mathrm{i}}\end{array}\right|$ ). Put ${\mathfrak{i}}P=p(T),\,U=u(T)$ Since p > 0,Theorem the 12.32 shows that $P\ {\bf\ }\geq0.$ Since $u{\vec{u}}=1,$ if $\,^{*}\lambda\neq0,\ u(0)=1.$ Then $\boldsymbol{\mathit{P}}$ and u are $J/I/$ bounded Borel functions on $\sigma(I)$ $T=U P$ $U U^{*}=U^{*}U=I.$ Since $\lambda=u(\lambda)p(\lambda),$ follow from the symbolic calculus. $\left|I/x\right|$ for every $x\in H.$ lt is not truc that ever $\boldsymbol{P}$ is the positive square root of $T^{\bullet}T_{\bullet}$ then $\left|P x\right|,$ Remark $T\in{\mathcal{P}}(H)$ has a polar decomposition.(See Exercise 19.) However, if so that the formula $V P x=T x$316 BANACH ALGEBRAS AND SPECTRAL THEORY happens when dim $H<\infty,$ since ${\mathcal{R}}(P)$ ${\mathcal{R}}(P)^{\perp}$ onto ${\mathcal{R}}(I)^{\perp},$ which has a continuous extension ${\mathcal{R}}(T).$ defines a linear isometry ${\mathbf{}}V$ Vof ${\mathcal{R}}(P)$ onto ${\mathcal{G}}(T),$ uhen ${\mathbf{}}V$ can be extended to a linear isometry of the closure of ${\mathcal{R}}(P)$ onto the closure of If there is a linear isometry of to a unitary operator on $\textstyle H_{\cdot}$ I, and then ${\boldsymbol{T}}$ and ${\mathcal{R}}(T)$ has a polar decomposition. This always have then the same codimen- sion. is extended to a member of $\beta(H)$ by defining $V y=0\;{\mathrm{for}}\;a|1\,y\in{\mathcal{R}}(P)^{\perp}.$ If ${\mathbf{}}V$ then ${\mathbf{}}V$ is called a partial isometry. thus has a factorization $T=V P$ in which ${\boldsymbol{P}}$ is positive and Every $T\in{\mathcal{P}}(H)$ V is a partial isometry. In combination with Theorem 12.16, the polar decomposition leads to an interesting result concerning similarity of normal operators. 12.36 Theorem Suppose $\bar{M}$ $N,$ $T\in{\mathcal{B}}(H)$ ), M and ${\boldsymbol{N}}$ are normal, ${\boldsymbol{T}}$ is invertible, and (1) $$ M=T N T^{-1}. $$ $I f T=U P$ is the polar decomposition of T, then (2) $$ M=U N U^{-1}. $$ Two operators $\bar{M}$ and $\textstyle N$ that satisfy I) are usually calld similar. 1 $U$ is unitary and(2) holds, $\bar{M}$ and $\boldsymbol{N}$ arc said to be unitarily cquivalent. The theorem thus asserts that similar normal operators are ctually untarily equivalent PRO0F By(1), $M T=T N$ Hence $M^{*}T=T N^{\times}$ ,by Thcorem 12.16. Conse- quently, $$ T^{*}M=(M^{*}T)^{*}=(T N^{*})^{*}=N T^{**}, $$ so that $$ {\cal N}P^{2}=N I^{*{\mathfrak{x}}}I=T^{*}M T=T^{*{\mathfrak{x}}}T N=P^{2}N, $$ since ${\boldsymbol{P}}^{2}={\boldsymbol{T}}^{*}{\boldsymbol{T}}.$ Hcncc ${\cal N}$ commutes with $f(P^{2}).$ for every $f\in C(\sigma(P^{2})).$ (See Section 12.24.)Since $\scriptstyle P\geq C$ 0, $\sigma(P^{2})\subset10,$ co)f $f(\lambda)=\lambda^{1/2}\geq0$ on $\sigma(P^{2})$ ),it follows that $N P=P N$ . Hence (I) yields $$ M=(U P)N(U P)^{-1}=U P N P^{-1}U^{-1}=U N U^{-1}.\qquad\qquad\qquad{\mathrm{~\bigwedge//f\}}} $$ The Group of Invertible Operators some featues o he groupof al invtietsin anch aleb $A={\mathcal{B}}(H).$ $\textstyle A$ Were descrbeda the end or Chapter 10. he followng two theorems contain further information about this group, in the special caserouNDED OPERATORs ON A HILBERT SPACE 317 and every $r\epsilon,G$ 12.37 Theorem The group C ${\boldsymbol{G}}$ of all iwertible operators Te M(H)is conneted is the product of two exponenials $S\in{\mathcal{O}}(H).$ Here an exoenial s o ourse, an opertor o the form exp S) win $\sigma(U)$ there is a self-adjoint $S\in{\mathcal{B}}(H)$ be the polar decomposition of some $P=\exp\,({\cal S}).$ Since $U_{\mathit{l}}$ is unitary $\boldsymbol{\f}$ fon ${\boldsymbol{U}}$ PROOF Let continuous real function on $\sigma(p)$ is positiveand inverible. Snc $T\in G.$ Recall that log is a $T=U P$ is unitary and that ${\boldsymbol{P}}$ $\sigma(P)\subset(0,\infty),$ It follows from the symbolic calculus that such that lies on the unit circle, so that te is a real bounded Borel function $\sigma(U)$ that satisfies $$ \exp{\{j f(\lambda)\}}=\lambda\qquad[\lambda\in\sigma(U)]. $$ $f(U).$ Then (Note that there may not exist any continuous $\boldsymbol{\f}$ withthis property') Put ${\boldsymbol{Q}}=$ $Q\in{\mathcal{B}}(H)$ is self-adjoint, and $U=\exp\,(i Q).$ Thus $$ T=U P=\exp\,(i Q)\exp\,(S). $$ by From ths follow easily that ${\boldsymbol{G}}$ is connected,for i $T_{r}$ is defined, fo $0\leq r\leq1,$ $$ T_{r}=\exp\,\left(i r{\cal Q}\right)\,\exp\,\left(r{\cal S}\right) $$ then $r\to T_{r}$ is a continuous mapping of theunit inteva $\scriptstyle[0,\,1]$ into $G,T_{0}=I,$ and $T_{1}=T$ This completes the proof // lt is now natural to ask whether every $\scriptstyle T\in G$ is an exponential, rather than merely the product of two exponentials In other words, is every product of two in fact, it is exponentials an exponential ? The answer is affirmative if dim $H<\infty\,;$ affirmatve in ever finitedimensonal Banach algebra,as a consequence of Theorem 10.30. Butin general the answer is negative, as we shall now see 12.38 Theorem Let L $D\!\!\!\!/$ be a bounded open set im ${\boldsymbol{C}}$ such that the set (1) $$ \Omega=\{\alpha\in C;\,\alpha^{2}\in D\} $$ is connected and such that ${\boldsymbol{D}}$ that saisy $D.$ Let $\textstyle H$ be the space of all holo- morphic fuctions f in ${\boldsymbol{0}}$ is not in the closure of (2) $$ \int_{D}\vert f\vert^{2}\,d m_{2}<\infty $$ (where m is Lebesue measure in the plane),with imer rodue (3) $$ (f,g)=\int_{D}f\overline{{{g}}}\;d m_{2}\,. $$318 BANACH ALGEBRAS AND SPECTRAL THEORY Then $H$ is a Hilbert space. Define the multiplication operator M E $\varepsilon{\mathcal{B}}(H)$ by (4) $$ (M f)(z)=z f(z)\qquad(f\in H,\,z\in D). $$ Then $\bar{M}$ is imvertible, but M has no square root in ${\mathcal{R}}(H).$ Since every exponential has roots of all orders, it follows that $\mathcal{M}$ is not an exponential. PROOF It is clear that (3)defines an inner product that makes $\textstyle H$ a unitary space.We show now that ${\boldsymbol{H}}$ is complete. Let $K$ be a compact subset of $D.$ whose distance from the complement of $D\!\!\!\!/$ is ${\bar{\partial}}.$ If $z\in K,$ if L is the open circular disc with radius $\delta$ and center ${\overline{{A}}}_{\mathrm{J}}$ 。 and if $f(\zeta)=\sum a_{n}(\zeta-z)^{n}$ for $\zeta\in\Delta,$ a simple computation shows that (5) $$ \sum_{n=0}^{\infty}(n+1)^{-1}|\,a_{n}|^{2}\,\delta^{2n+2}=\frac{1}{\pi}\int_{\Delta}|f|^{2}\,\,d m_{2}\,. $$ Since $f(z)=a_{0}\,,$ it follows that (6) $$ |f(z)|\le\pi^{-1/2}\left.\delta^{-1}\|f\|\right|\qquad(z\in K,f\in{\cal H}),\quad. $$ where $\|f\|=(f,f)^{1/2}.$ Every Cauchy sequence in ${\boldsymbol{H}}$ converges therefore uniformly on compact subsets of $D.$ From this it follows easily that ${\boldsymbol{H}}$ is com- plete. Hence ${\boldsymbol{H}}$ is a Hilbert space Since $D\!\!\!\!/$ is bounded $M\in{\mathcal{B}}(H)$ Since $1/z$ is bounded in $D_{\cdot}$ ${\cal M}^{-1}\in\mathcal{B}({\cal H}).$ $\operatorname{Fix}x\in\Omega$ Put $\lambda=\alpha^{2}.$ Assume now, to reach a contradiction, that $M=Q^{2}$ for some $Q\in{\mathcal{B}}(H).$ Then $\lambda\in D$ .Define (7) $$ M_{\lambda}=M-\lambda I,\qquad S=Q-\alpha I,\qquad T=Q+\alpha I, $$ so that (8) $$ S T=M_{\lambda}=T S. $$ Since we are dealing with holomorphic functions,the formula (9) $$ (M_{\lambda}g)(z)=(z-\lambda)g(z)\qquad(z\in D,g\in H) $$ shows that $M_{\lambda}$ is one-to-one and that its range ${\mathcal{P}}(M_{\lambda})$ consists of exactly those f∈ H that satisfy $f(\lambda)=0$ . Hence (6) shows that ${\mathcal{R}}(M_{\lambda})$ is a closed subspace of H, of codimension ${\textbf{l}},$ is one-to-one. Sincc ${\mathcal R}(M_{\lambda})\neq H,\,.$ M,is not invertible the second shows that $\boldsymbol{\mathsf{S}}$ is One-to-one, the first equation (8) shows that ${\cal T}\,$ is one-to-one; Since $M_{\lambda}$ in ${\mathcal{B}}(H).$ Hence at least one of $\boldsymbol{\mathsf{S}}$ $\mathcal{R}(M_{\lambda})\subset\mathcal{B}(S),$ so that is not invertible. Suppose is either ${\mathcal{R}}(M_{\lambda})$ is not $H.$ and ${\boldsymbol{T}}$ $\boldsymbol{\mathsf{S}}$ invertible. Since $M_{\lambda}=S T,$ ln the atereae the ope mapping theorem would imly tha $\boldsymbol{\mathsf{S}}$ or ${\mathcal{M}}(S)$ is ivetibeBOUNDED OPERATORs ON A HLBERT SPACE 319 shows that $\boldsymbol{\mathsf{S}}$ maps S s a one-to-one mapping of onto ${\mathcal{R}}(M_{\lambda}).$ onto ${\mathcal{A}}(M_{\lambda})$ But the equation and $\longrightarrow\setminus\emptyset.$ lies in 'is Hcnce $\boldsymbol{\mathsf{S}}$ ${\mathcal{B}}(H)$ . Therefore exactly one of the numbers $M_{1}=S T$ ${\boldsymbol{T}}$ )is $\textstyle H$ compact. ${\mathcal{R}}(T)$ $\mathbb{Q}$ is the union of two disjoint congruent sets $\textstyle{\mathcal{E}}$ S and $\mathbf{Q}$ is invertible in Hence ${\mathcal{R}}(T)=H,$ and another applica- tion of the open mapping theorem shows tha $T^{-1}\in{\mathcal{B}}(H)$ $\boldsymbol{\mathsf{S}}$ We have now rod tat one nd ony one r heoperator o(Q), if $\alpha\in\Omega$ lt follows that $\sigma(Q)\ r\,\Omega$ and $-\sigma(Q)\cap\Omega.$ both of which are closed Grelative to $\Omega{}_{n}$ ) since $e(Q)$ // The assumption that $M=Q^{2}$ leads thus to the conclusion that not connected, which contradicts the hypothesis This completes the proof. The simplest example of a region ${\boldsymbol{D}}$ that satisies the hypothesis of Theorem 12.38 a ciculanuswit ctea A Characterization of $B^{\sharp}$ 3*-algebras of sone The facthat every ${\mathcal{R}}(H)$ )isa $B^{\times}.$ alebra hasben expoited throughout ths chapter $B^{\kappa}.$ positive functionals. We hal ow stabis a conveseThierem 124 ihisetai evy ${\mathcal{P}}(H).$ algeba ommtive oi somtcil-sorpic som cosuaiger The prof enso t extne fasiniyaiesuppiyo 12.39 Theorem I/ Aisa $B^{\mathbf{x}}.$ -algebra and $i^{f}\,z\in A,$ then there exists a psitive func tional ${\mathbf{}}F$ on A such that (1) $$ F(e)=1\qquad a n d\qquad F(z z^{*})=\|z\|^{2}. $$ $x+y\in P.$ hermitian elements of $A_{\mathrm{\scriptsize{k}}}$ and Iet rxor Let A,(he relpr" b te ral vcto spae acs o h $x\in A_{r}$ with $\sigma(x)\subset[0,$ oo). 11.28, ln the terminology of Definition $\boldsymbol{P}$ be the set of al if and only if $x\geq0.$ By Theorem and $11.27,\;x\in P$ ${\boldsymbol{P}}$ is a cone: if $D\!\!\!\!/$ containsall elements of the form $x x^{\mathrm{x}},\mathrm{for}\ x\in A.$ To prove the that Also. $x\in P,$ $\nu\in{\mathcal{P}},$ and c is a positive scalar, uhen cxe $\Xi\,P$ satisfies(I) and thorm,i is hefoe enough to find ea-inear incton $\boldsymbol{\mathit{f}}$ on ${\mathit{A}}_{r}$ (2) $$ J(x)\geq0\qquad{\mathrm{for~every~}}x\in J $$ ${\boldsymbol{F}}$ for we can then define $F(x)=f(u)+i f(v)\,\,\,\mathrm{if}\,\,\,x=u+i v$ and $u\in A_{r},$ ve A, is positive Since this dcfnton gives $F(i x)=i F(x),\,F$ is complex-linear, and (2) shows that by Let $\bar{M}_{0}$ be the subspace of ${\cal A}_{r}$ generated by $\scriptstyle{\mathcal{C}}$ and $\mathbb{Z}\mathbb{Z}^{\mathbb{N}}.$ and define $f_{0}$ on $\M{}_{0}$ (3) Jo(ce + βzz*)=α.+ /|zz2* (α,βe R).320 BANACH AI.GEBRAS AND SPFCTR AL THEORY Note $\operatorname{that}f_{0}$ is well defined on $\mathcal{M}_{0},$ even if $\scriptstyle{\mathcal{C}}$ and $\mathbb{Z}Z^{\times}$ are linearly dependent By $\mathbf{\Psi}(a)$ In other words. $f_{0}(x)\in\sigma(x)$ if $\|{\mathcal{Z}}^{*}\|\in\sigma({\mathcal{Z}}^{*}).$ Hence $\alpha+\beta\|z z^{s}\|$ lies in $\sigma(\operatorname{v}e+\beta z z^{*}).$ of Theorem 11.28 $x\in M_{0},$ so ${\mathrm{fhat~}}f_{\omega}(x)\geq0$ for every $x\in P\cap M_{0}\,.$ Also, f satisfies $(1).$ has been extended to a real-linear functional $f_{1}$ on a Assume that $f_{0}$ subspace ${\mathcal{M}}_{1}$ of ${\cal A}_{r}\,,$ such that $f_{1}(x)\geq0$ for all $x\in P\cap M_{1},$ and assume that $y\in A_{r},\;y\notin M_{1}$ Put (4) $$ E^{'}=M_{1}\cap(y\cap P),\qquad E^{'}=M_{1}\cap(y\ +P). $$ sum, 1f $x^{\prime}\in E^{\prime}$ and $x^{\prime}\in E^{\prime},$ then $y-x^{\prime}\in P$ and $x^{\prime\prime}-y\in P;$ hencc so is their $x^{\prime\prime}-x^{\prime},$ and therefore $f_{1}(x^{\prime})\leq f_{1}(x^{\prime\prime})$ It follows that there is a real number c that satisfies (5) $$ f_{1}(x^{\prime})\leq c\leq f_{1}(x^{\prime\prime})\qquad(x^{\prime}\subset E^{\prime},\ x^{\prime\prime}\in E^{\prime\prime}). $$ Define (6) $$ f_{2}(x+\infty y)=f_{1}(x)+\infty c\qquad(x\in M_{1},\;\alpha\in R). $$ If $x+y\in P,$ then $x\in E^{n},\;f_{1}(x)\geq c,$ on $P\cap M_{2}\,.$ $f_{2}(x+y)\geq0.$ $1{\mathrm{f}}$ $x-y\in P,$ then $-x\in E^{\prime},\;f_{1}(-x)\leq c,\;f_{1}(x)\geq-c;\;\mathrm{h}$ ence It follows from and $f_{2}(x-y)\geq c-c=0$ these two cases that $f_{2}\geq0$ The proof can now be completed by transfinite induction, just as in the Hahn-Banach thcorcm. // 12.40 Theorem If A is a B*-algebra and $i f u\in A,\,u\neq0,$ there exists a Hilbert space $I I_{u}$ and there exists a homomorphism $T_{u}$ of A into ${\mathcal{B}}(H_{u})$ that satisfies $T_{u}(e)=I,$ (2) $$ \begin{array}{c c c}{{T_{u}(x^{\ast})=T_{u}(x)^{\ast}}}&{{(x\in A),}}\\ {{||T_{u}(x)|\le||x||}}&{{(x\in A),}}\end{array} $$ (1) and $\|T_{u}(u)\|=\|u\|.$ PRO0F We regard uas fived and omit the subscripts u. Fix a positive functional ${\mathbf{}}F$ on $\scriptstyle A$ that satisfies (3) $$ F(c)=1\qquad\mathrm{and}\qquad F(u^{\mathrm{a}}u)=\|u\|^{2}. $$ Such an ${\mathbf{}}F$ exists, by Theorem 12.39. Define (4) $$ Y=\{y\in A\colon F(x y)=0{\mathrm{~for~cvery~}}x\in A\}. $$ Since ${\mathbf{}}F$ is continuous (Theorem 11.31). $\boldsymbol{\mathit{I}}$ is a closed subspace of ${\bar{A}}.$ Denote cosets of ${\cal{Y}},$ that is,elements of $A/Y,$ by x' (5) x'=X+ Y (xe A).roUNprn oPexArors oN AHLDer SrAcr 321 We claim that (6) $$ (a^{\prime},b^{\prime})=F(b^{*}a) $$ defines an iner product on $A/Y.$ it is enough to show that $F(b^{*}a)=0$ if at To se that $(a^{\prime},b^{\prime})$ is wel fined G),i.tht sindendent o t choice of representatives lies i ${\cal{Y}}.$ If $a\in Y,F(b^{\kappa}a)=0$ follows from (4).If bc Y, then least one of aor $\bar{a}$ and $b_{\cdot}$ $\boldsymbol{\partial}$ (7) $$ F(b^{\star}a)=F(a^{\star}b)=0, $$ defined i inari ${\boldsymbol{a}}^{\prime}.$ by Ca o Theorem 1.31 and anoterapin of 4) Thus and is well an onuaieai $(a^{\prime},b^{\prime})$ ${b^{\prime}}_{,}$ (8) $$ (a^{\prime},a^{\prime})=F(a^{*}a)\geq0, $$ operators $\textstyle r(x)$ on ris a positive functional by then $F(a^{*}a)=0;$ hence $F(x a)=0$ Its since ${\boldsymbol{F}}$ $\operatorname{If}\left(a^{\prime},a^{\prime}\right)=0.$ and $a^{\prime}=0.$ for every $x\in A,$ by (b) of Theorem 11 31, sothat a e ${\mathbf{}}Y$ $\|a^{\prime}\|=F(a^{*}a)^{1/2}.$ completion ${\boldsymbol{H}}$ $A/Y$ is thus an inner product space, with norm $A/Y$ is thHilbet pae tai w ar oingfo. " dcine inea :(9) $$ T(x)a^{\prime}=(x a)^{\prime}. $$ that $a\in a^{\prime},\,\operatorname{for}\,i\Gamma\,y\in Y,$ (4) implies hat Again,onecesksasly tatis einin siadendn of thechoice (Yis a leftideal in A.) $\operatorname{I}\! [$ is obvious $x\to T(x)$ is linear and that $x y\in Y.$ (10) $$ T(x_{1})T(x_{2})=T(x_{1}x_{2})\qquad(x_{1}\in A,\;x_{2}\in A); $$ that in particular, (9) shows that $T(e)$ is the dentity opcrator on $A/Y.$ We now claim (11) $$ \|T(x)\|\leq\|x\|\qquad(x\in A). $$ (12) oncc hi sown, te uior coniniuty o th oerator $H.$ Note that $\scriptstyle T(x)$ enables us to extend them to bounded lincar opeatons o $$ \|T(x)a^{\prime}\|^{2}=((x a)^{\prime},\,(x a)^{\prime})=F(a^{*}x^{*}x a). $$ that For fixed $a\in A,$ 4, defin $G(x)=F(a^{*}x a)$ 、 Then Gis a positive functional on ${\bar{A}},$ A,SO (13) $$ G(x^{\ast}x)\le G(e)\vert x\vert^{2}, $$ by d) of Theorem 1131. Thus (4 $$ \|T(x)a^{\prime}\|^{2}=G(x^{*}x)\leq F(a^{*}a)\|x\|^{2}=\|a^{\prime}\|^{2}\|x\|^{2}, $$ which proves (11)322 BANACH ALGEBRAS AND SPECTRAL THEORY Next, the computation $$ \begin{array}{c}{{\displaystyle\cdot(T(x^{\star})a^{\prime},\,b^{\prime})=((x^{\star}a)^{\prime},\,b^{\prime})=F(b^{\star}x^{\star}a)=F((x b)^{\star}a)}}\\ {{}}&{{=(a^{\prime},\,(x b)^{\prime})=(a^{\prime},\,T(x)b^{\prime})=(T(x)^{\ast}a^{\prime},\,b^{\prime})}}\end{array} $$ shows that $T(x^{*})a^{\prime}=T(x)^{*}a^{\prime}.$ , for all $a^{\prime}\in A/Y$ Since $A/Y$ is dense in $\textstyle H,$ I, this proves (I). Finally,(3) and(12) show that (15) $$ \|u\|^{2}=F(u^{*}u)=\|T(u)e^{\prime}\|^{2}\leq\|T(u)\|^{2} $$ since $\|e^{\prime}\|^{2}=F(e^{*}e)=F(e)=1.$ In conjunction with (11),(15) gives $\|T(u)\|=$ // lul, and the proof is complete 12.41 Theorem If A is a $B^{\mathbf{s}}.$ algebra, there exists an isometric *-isomorphism of A onto a closed subalgebra of ${\mathcal{A}}(H),$ where ${\cal H}$ is a suitably chosen Hilbert space. PROOF Let ${\boldsymbol{H}}$ be the“direct sum”of the Hilbert spaces $\textstyle H_{u}$ constructed in Theorem 12.40. Here is a precise description of $H\colon$ Let $\scriptstyle\pi_{s}(t)$ be the $\scriptstyle H_{\alpha}$ coordinate of an element v of the cartesian product of the spaces $\textstyle H_{u}$ Then, by definition, $\scriptstyle*\in H$ if and only if (1) $$ \sum_{u}\left\|\pi_{u}(v)\right\|^{2}<\infty, $$ where llz,(0)|l denotes the $H_{u}.$ -norm of $\scriptstyle{\pi_{k}(n)}$ The convergence of(I) implies that given by at most countably many $\pi_{3}(t)$ are different from $0.$ The inner product in ${\cal I}_{\O}$ is (2) $$ (v^{\prime},\,v^{\prime})=\sum_{u}\,(\pi_{u}(v^{\prime}),\,\pi_{u}(v^{\prime}))\qquad(v^{\prime},\,v^{\prime}\in{\cal{H}}), $$ so that $\|v\|^{2}=(v,v)$ is the left side of (1). We leave it as an exercise to verify $I/\cdot$ that ll Hilbert spacc axioms are now satisfied by If $S_{u}\in{\mathcal{B}}(H_{u}),$ if $\|S_{u}\|\leq M$ for all ${\mathcal{U}},$ , and if ${\boldsymbol{S}}_{U}$ is defined to be the vector whose coordinate in $\textstyle H_{u}$ is (3) $$ \pi_{u}(S v)=S_{u}\pi_{u}(v), $$ one verifies easily that $S v\in H$ if $v\in H,$ that $\delta\in{\mathcal{B}}(H)$ , and that (4) $$ \|S\|=\operatorname*{sup}_{u}\ \|S_{u}\|. $$ We now associate with each $x\in{\mathcal{A}}$ an operator $T(x)\in{\mathcal{B}}(H),$ by requiring that $\left(5\right)$ T,(T(Kx)D) =7,(xAr,(O)BoUNDEL prxAToxs oN A HlLuBxr srAc 323 where $T_{u}$ is as in Theorem 12.40. Since (6) $$ \|T_{u}(x)\|\leq\|x\|=\|T_{x}(x)\|, $$ by Theorem 12.40, it follows from (4) that (7) $$ \|T(x)\|=\operatorname*{sup}_{u}\|T_{u}(x)\|=\|x\|. $$ That the mapping $x\to T(x)$ of $\textstyle A$ into ${\mathcal{O}}(H)$ has the other required proper- ${\it j}{\it j}{\it j}{\it j}$ ties follows from a coordinatewise appication of Theorem 12.40. Exercises Throughout these exercises, the letter ${\boldsymbol{H}}$ denotes a Hilbert space L The completion of an inner product space is a Hilbert space. Make this statement more precise, and prove it.(See the proof of Theorem 12.40 for an appication. 2 Suppose ${\cal N}$ is a positive integer,αe ${\mathcal{C}},$ $\alpha^{N}=1.$ and $\alpha^{2}\neq1$ Prove that every Hilbert space inner product satisfies the identities $$ (x,y)=\frac{1}{N}\sum_{n=1}^{N}\|x+\,\alpha^{n}y\|^{2}\alpha^{n} $$ and $$ (x,\,y)=\frac{1}{2\pi}\int_{-\pi}^{\pi}\|x-\mathrm{tr}\,e^{i\theta}y\|^{2}e^{i\theta}\,d\theta. $$ Generalize this: Which functions $\boldsymbol{\mathit{f}}$ and measures ${\boldsymbol{\mu}}$ on $\scriptstyle{\mathbf{a}}$ set $\mathbb{C}$ give rise to the identity $$ (x,y)=\int_{\Omega}\|x+f(p)y\|^{2}\,d\mu(p)^{\gamma} $$ 3 (a) Assume $x_{n}$ and $y_{n}$ are in the closed unit ball of $\textstyle H,$ , and ${\bigl(}x_{n}\,,\,y_{n}{\bigr)}\to1$ as $n\to\alpha.$ . Prove that then $\|x_{n}-y_{n}\|\to0.$ (b)Assume $x_{n}\in H,\;x_{n}\to x$ weakly, and ${\|x\|\ \|}\to\|x\|\mathbf{\|}.$ Prove that then $||x_{n}-x||\to0$ 4 Let $H^{\star}$ be the dual space of $H;$ define $\textstyle\langle:H^{*}\to H$ by $$ y^{*}(x)=(x,\psi,\!\!\!\psi^{*})\qquad(x\in H,y^{*}\in H^{*}). $$ (See Theorem 12.5.) Prove that $H^{\bullet}$ is a Hilbert space, relative to the inner product $$ [x^{\sharp},\,y^{\star}]=(\not y y^{\star},\,\not v x^{\star}). $$ $\operatorname{If}\,\phi\colon H^{*\,s}\to H^{*}$ satisfies $z^{\star\star}(y^{\star})=[y^{\star},\,\phi z^{\star\star}$ " fora $y^{\star}\in H^{\star}$ and $z^{\star\star}\in H^{\star\star},$ prove that ysp is an isomorphism of $H^{\ast\ast}$ onto $H$ whose existence implies that H is reflexive _Suppose {u} is a sequence of unit vectors in ${\boldsymbol{H}}$ (that is $\|u_{n}\|=1),$ and assume that $$ \Gamma^{2}=\sum_{i,j}|(u_{i},u_{j})|^{2}<\varnothing. $$324 BANACH ALGEBRAS AND SPECTRAL THroRv If{&a} is any sequence of scalars, prove that $$ (1-\Gamma)\sum_{i=m}^{n}|\alpha_{i}|^{2}\,\leq\Big|\Big|\sum_{i=m}^{n}\alpha_{i}\,u_{i}\,\Big|^{2}\,\leq(1\,+\,\Gamma)\sum_{i=m}^{n}|\,\alpha_{i}|\,^{2}, $$ and deduce that the following three properties of {α;} are equivalent to each other: (a) E|α| <00. (D) 白店,cnoeneom $\textstyle H.$ (c) α(u,y) converges, for every ye H This generalizes Theorem 12.6. 6 Suppose $\boldsymbol{E}$ is a rsolution of the identity, as in Section 12.17, and prove that $$ |E_{x,\,v}(\omega)|^{2}\leq E_{x,\,x}(\omega)E_{y,\,y}(\omega) $$ 7 for all ${\mathfrak{c e}}\,H,y\,\in H,$ and $\omega\in M.$ $\scriptstyle{e>0.}$ Prove that scalars $\alpha_{0}\;,\ldots\cdot,\ \alpha$ m can be chosen Suppose $U\in{\mathcal{R}}(H)$ is unitary, and so that $$ ||U^{-1}-\alpha_{0}I-\alpha_{1}U-\cdots-\alpha_{n}\,U^{n}||<\varepsilon_{1} $$ if $\sigma(U)$ is a proper subset of the unit circle, but that this norm is neve lss than lif o(U covers the whole circle Note: That $\sigma(U)$ lics on the unit circle is contained in Theorem 12.26 but can be proved in a much more elementary way. Find such a proof 8 Prove Theorem 12.35 with $P U$ in place of $U P.$ Suppose ${\boldsymbol{I}}=U P$ is the polar decomposition of an invertible Te &W(H). Prove that T is normal if and only if $U P=P U.$ 10 Prove that every normal invertible Te 9M(HD is the exponential of some normal $S\in{\mathcal{R}}(H).$ ${\cal I}{\cal I}$ Suppose $N\in{\mathcal{B}}(H)$ is normal, and $T\in{\mathcal{B}}(H)$ is invertiblc. Prove that $r N T^{\circ}$ is normal if and only if ${\mathbf{}}N$ commutes with $T^{*}T.$ 12 (a) Suppose $S\in{\mathcal{D}}(H)$ $r\in{\mathcal{B}}(H),\cdot$ S and ${\mathbf{}}T$ are normal, and $S T=T S.$ Prove that $\scriptstyle{S\,+\,T}$ and ${\boldsymbol{S}}{\boldsymbol{T}}$ are normal. (b) If, in addition, $S\geq0$ and T≥0 (see Theorem 12.32), prove that $\textstyle s+T z\geq0$ and $S T\cong0.$ (c) Show, however, that there exist $S\geq0$ and T≥0 such that ST is not even normal (of course, then ${\mathcal{L}}\gamma$ TS). In fact, such examples exist if dim $\textstyle\eta-2$ $\mathit{I}_{}^{3}$ If $T\in{\mathcal{B}}(H)$ is normal, show that ${\cal T}^{*}=U{\cal T},$ for some unitary $U.$ 1. When is $U$ unique? $I{\mathcal{A}}$ Assume Te (H) and $\scriptstyle T^{\prime}\,T$ is a compact operator.Show that $T_{\mathbf{\delta}}$ 'is then compact. ${\mathcal{L}}{\mathcal{S}}$ Find a noncompact Te M(H) such that $T^{2}=0.$ Can such an operator be normal : $I{\mathcal{O}}$ Suppose $T\in{\mathcal{B}}(H)$ is normal, and $\sigma(T)$ is a finite set. Deduce as much information about T from this as you canBOUNDED OPERATORS ON A HLBERT SFACE 325 17 Show, under the hypotheses o $(d)$ of Theorem 12.29, that the equation $T y=x\,$ has a solution $\nu\epsilon\,\mathcal{H}$ if and only i $$ \sum_{i=1}^{\infty}|\lambda_{i}|^{-2}\|x_{i}||^{2}<\infty. $$ $I{\mathcal{S}}$ (If $\lambda_{i}=0$ for one ${\hat{l}},$ , then ${\boldsymbol{x}}_{i}$ must be $\mathbf{0}_{\!,\!}$ for this $i.{\big/}$ for which $T-\lambda I$ is not one-to-one The spectrum $\sigma(t)$ of $T\in{\mathcal{B}}(H)$ can be divided into three disjoint pices The point spectrum $\sigma_{s}(T)$ consists of $\operatorname{all}\,\lambda\in{\mathcal{C}}$ The continuous spectrum $\scriptstyle v_{\circ}(T)$ consists of all $\lambda\epsilon\,c$ such that $T\-\lambda I$ is a one-to-one mapping of ${\boldsymbol{H}}$ onto a dense proper subspace of $H.$ separable. The residual spectrum $\sigma_{o}(T)$ consists of all other $\lambda\in\sigma(T),$ ${\boldsymbol{H}}$ is (a) Prove that every normal $T\in{\mathcal{P}}(H)$ has empty residual spectrum is at most countable, if (6) Prove that the point spectrum of a normal $T\in{\mathcal{B}}(H)$ (c) Let $S_{R}$ acting on the Hilbert space $\ell^{\,2}.$ be the riht and lft shts as defined in Exercise o Chapter 10) and ${\boldsymbol{S}}_{L}$ Prove that $(S_{R})^{*}-S_{L}$ and that $$ \begin{array}{l c r}{{\sigma_{r}(S_{L})=\sigma_{r}(S_{R})=\{\lambda\!:\!\left|\lambda\right|<1\},}}\\ {{\sigma_{c}(S_{L})=\sigma_{c}(S_{R})=\{\lambda\!:\!\left|\lambda\right|=1\},}}\\ {{\sigma_{r}(S_{L})=\sigma_{r}(S_{R})=\sigma_{r}(S_{R})=\mathcal{O}.}}\end{array} $$ $I{\mathcal{J}}$ with $U$ unitary and $\scriptstyle P\geq0.$ be as above. Prove that neither $S_{R}$ nor $S_{L}$ has polar decompositions ${\boldsymbol{U}}p,$ Let $S_{R}$ and ${\boldsymbol{S}}_{L}$ $2O$ Let ${\boldsymbol{\mu}}$ be a positive measure on a measure space $\Omega,$ let ${\cal H}={\cal L}^{2}(\mu),$ with the usual inner product $$ (f,g)=\int_{\Omega}f{\bar{g}}\,d\mu. $$ For $\phi\in L^{\infty}(\mu),$ define the multiplication operator $\phi$ does $M_{\phi}$ have eigenvalues ? Give an example in which Then $M_{\phi}\in\mathcal{B}(H).$ $L^{\infty}(\mu)$ Under what conditions on ${\mathcal{M}}_{\phi}$ by $M_{\phi}(f)=\phi f.$ $\sigma(M_{\phi})=\sigma_{c}(M_{\phi}).$ Show that every of $\rho(t)$ (Certain pathological mcasures ${\mu}$ have to be ex $\sigma(M_{\phi})$ and thc csscntial range of $M_{\phi}$ is normal. What is the relation betwee onto a closed subalgebra $\phi\,?$ Show that $\phi arrow M_{\phi}$ is an isomtric -isomorphism o $\scriptstyle A$ algebra of ${\mathcal{O}}(H)^{\dagger}$ Hint: If T'e MB(H)and cludinoder to make thi as stement orrect.)Is a maxima comtative sub $\scriptstyle T\in{\mathcal{A}}$ show that $\overline{{T}}$ -is a multiplication by $\scriptstyle{T(1)}$ and hence that $\begin{array}{l c r}{{T M_{\phi}=M_{\phi}\,T\mathrm{for\,all}\,\phi\in L^{\omega}(\mu),\mathrm{ardif\,}\mu(\Omega)<\infty,}}\end{array}$ $2{\cal I}$ $T^{\mathfrak{s}}.$ Suppose $T\in{\mathcal{R}}(H)$ is normal, $\scriptstyle A$ l s the closed subalgebra ot ${\mathcal{B}}(H)$ gcnerated by I, T, and and T can be approximated, in the norm topology of ${\boldsymbol{t}}{\boldsymbol{O}}$ A、 by finite linear combina- tions of projcctions that belong ${\mathcal{R}}(H),$ the same $\boldsymbol{\mathit{I}}$ Under what (necssary and suffcint) conditions on $T^{\prime}{\mathfrak{I}}$ Can it happecn that two square roots o 22 Does every normal Te ZGHD havea squae root n ${\boldsymbol{\sigma}}({\boldsymbol{T}})$ does this happen cardinality of the set of all square roots of ${\mathcal{B}}(H)^{\cdot}$ ? What can you say about th T do not commute? Can this happen when $T=I\gamma$326 BANACHI ALGEBRAS AND SPECTRAL THEORY $23$ Show that the Fourier transform $f\omega f$ is a unitary operator on $L^{2}(R^{n})$ What is its spectrum?Suggestion When $n=1,$ compute the Fourier transforms of $$ \exp\left(\frac{1}{2}\,x^{2}\right)\left(\frac{d}{d x}\right)^{m}\exp\left(-\,x^{2}\right)\qquad(m=0,\,1,\,2,\,.\,.\,. ). $$ $2{\mathcal{A}}$ Show that any two infinite-dimensional separable Hilbert spaces are isometrically iso- morphic (via countable orthonormal bases;see [23). Show that the space ${\cal H}$ in Theorem 12.38 s seprble. Show that the answer to the question that precedes Theorem 12.38 is therefore negative for every infinite-dimensional $\textstyle H_{\mathrm{i}}$ separable or not $25$ Suppose $T\in{\mathcal{B}}(H)$ is normal, $\boldsymbol{\mathit{f}}$ ris a bounded Borel function on $\sigma(T),$ and $S=f(T)$ .If $E_{T}$ and $E_{S}$ are the spectral decompositions of ${\boldsymbol{T}}$ T" and ${\boldsymbol{S}},$ , respectively, prove that $$ E_{S}(\omega)=E_{T}(f^{-1}(\omega)) $$ for every Borel set $\omega\subset G(S)$ S). 26 If SE $:{\mathcal{B}}(H)$ and $T\in{\mathcal{R}}(H),$ the notation $\mathbb{S}\geq T$ means that $\scriptstyle S-T\geq0.$ that is, that $$ (S x,\,x)\geq(T x,\,x) $$ for all $\textstyle{\epsilon\in H}$ Prove the equivalence of the following four properties of a pair of self- adjoint projections ${\boldsymbol{P}}$ and ${\boldsymbol{Q}}$ (a) $P\geq Q.$ (b) ${\mathcal{R}}(P)\supset{\mathcal{R}}(Q).$ (c) $P Q-Q.$ (d) $Q P=Q.$ is a resolution of the identity, it follows that $E(\omega^{\prime})\geq E(\omega^{\prime\prime})$ if and only if r ${\boldsymbol{E}}$ $\omega^{\prime}\ni\omega^{\prime}$ 27 Suppose is an involution in a complex algebra ${\dot{A}},$ is an invertible element of $\textstyle{\mathcal{A}}$ A such that $q^{*}=q$ and $\displaystyle{X^{\ \#}}$ is defined by $$ x^{\mu}=q^{-1}x^{*}q $$ for every $x\in A$ Show that $\scriptstyle{\vec{n}}$ is an involution in $A.$ let $\textstyle W^{*}$ be the 28 Let A be the algebra of all complex 4-by-4 matrices. If $M=(m_{i j})\in A,$ conjugate transpose of $M\colon m_{i j}^{*}=m_{j i}\,.$ Put $$ Q={\left(\begin{array}{l l l l l}{0}&{0}&{0}&{1}\\ {0}&{0}&{1}&{0}\\ {0}&{1}&{0}&{0}\\ {1}&{0}&{0}&{0}\end{array}\right)}\qquad S={\left(\begin{array}{l l l l l}{0}&{0}&{0}&{0}\\ {0}&{0}&{0}\\ {0}&{0}&{0}&{0}\\ {0}&{0}&{0}&{0}\end{array}\right)}, $$ As in Exercise $27,$ define $$ M^{*}=Q^{-1}M^{*}Q\;\;\;\;\;\;\;(M\in{\cal A}). $$ $\mathbf{\Psi}(a)$ Show that $\boldsymbol{\mathsf{S}}$ and T are normal, with respect to the involution *, that ${\mathcal{G T}}=$ TS, but that $S T^{s t}\ \overline{{{\tau}}}$ f T*SBOUNDED OPERATORs ON A HILBERT SPACE 327 (b)Show that $\textstyle s+l$ is not *-normal. (c) Compare |ISS*| with IS||]2 (d) Compute the spectral radius $\rho(S+S^{\#}){\mathrm{;}}$ show that it is different from $\|S+s^{\rho}\|.$ (e) Define $V=(v_{i j})\in A$ so that $v_{12}=v_{24}=i,\ \ v_{31}=v_{43}=-i,\ \ v_{i j}=0\ \mathrm{Othe}$ >rwise. Compute 11.28 fail for the involution ${}^{\#}.$ it does not lie in [0, co) $\sigma(V V^{\#});$ Part Go shows that Theorem 1.6 fais for sone involutons. Part b) does the same for part (a) of Exercise 12;(e),(Gd), and (Ge) show that various parts of Theorem 29 Let $\textstyle X$ be the.vecor spaceof al tigometric polynomials on the real line: thesear functions of the form $$ f(t)=c_{1}e^{i s_{1}t}+\ \cdot\cdot\cdot+c_{n}e^{i s_{n}t}, $$ where $s_{k}\in R$ and $\alpha\in C.$ for $1\leq k\leq n.$ Show that $$ \mathbf{\nabla}(f,g)=\operatorname*{lim}_{A arrow\infty}{\frac{1}{2A}}\int_{-A}^{A}f(t){\overline{{g(t)}}}\,d t $$ is an inner product on $X_{\cdot}$ ,that $$ \|f\|^{2}=(f,f)=\left|c_{i}\right|^{2}+\cdot\cdot\cdot+\left|c_{n}\right|^{2}, $$ and that the completion of X $\textstyle X$ is a nonseparable Hilbert space $\textstyle H.$ Show that ${\cal H}$ contains all uniform limits of trigonometric polynomials; these are the so-called“almost-periodic” functions on ${\boldsymbol{R}}.$ 30 Let $H_{\mathrm{w}}$ be an infinite-dimensional Hilbert space, with its weak topology. Prove that the inner product is a separately continuous function on $H_{\mathrm{w}}\times H_{\mathrm{w}}$ which is not jointly continuous ${\mathbf{\nabla}}S I$ Assume $T_{n}\in{\mathcal{B}}(H)$ for $n=1,2,3,\ldots,\mathrm{and}$ $$ \operatorname*{lim}_{n\to\infty}\left\|T_{n}\,x\right\|=0 $$ for every $x\in H.$ Does it follow that $$ \operatorname*{lim}_{n arrow\infty}||T_{n}^{*}\:x||=0 $$ for every $x\in H^{\prime}$ 32 Let $\textstyle X$ be a uniformly comvex Banach space. This means, by definition, that the assump- tions $$ \|x_{n}||\leq1,\qquad\|y_{n}\|\leq1,\qquad\|x_{n}+y_{n}\|\to2 $$ imply that $\|x_{n}-y_{n}\|\to0,$ For example, every Hilbert space is uniformiy convex (a) Prove that Theorem 12.3 holds in X. (b)Assume $||x_{n}||=1.$ $\Lambda\in X^{*},\;\;||\Lambda||=1,$ and $\Lambda x_{n} arrow\Upsilon.$ Prove that {x)} is a Cauchy sequence Gin the norm-topology of X). Hin: Consider AXn+ Xm).328 BANACH ALCGEBRASs AND SPECTRAL THEORY 33. in $L^{1},$ ,or in ${\cal{C}},$ $||x_{n}||=1.$ Consider $\Lambda(x_{n}+x),$ atains is maximum on the close nt ball $\left|\left|{\boldsymbol{x}}_{n}-{\boldsymbol{x}}\right|\right|\Rightarrow(0,$ Hint: Reduce to the case Prove the assertion about the case dim $H<\cdots$ $X.$ (c) Prove that every $\Lambda\in X^{\star}$ $\left\|V\right\|\to\left| |\gamma\right|$ Prove that (d)Assume that $\vdash x_{n}\to\ A\lor\quad$ : weakly and for a suitable A (e So tecignifrpretesi ta nah spaesfo nstanc made in the remark that follows . These are therefore not uniformly convex. Theorcm 12.35.