$$ \hat{\boldsymbol{g}} $$ APPLICATIONS TO DIFFERENTIAL EQUATIONS Fundamental Solutions 8.1 Introduction We shall be concerned with linear partial differential equations with constant coeficients. These are equations of the form (1) $$ P(D)u=v $$ whcrc ${\boldsymbol{P}}$ is a nonconstant polynomial in $\;n$ variables (with complex coefcients), $\scriptstyle{P(D)}$ is the corresponding difetial operator Gee Sectin 7.1) ' sa given function or A distribution $E\in{\mathcal{D}}^{\prime}(R^{n})$ distribution, and the function (or distribution) uis a solution of (1D) is said to be a fundamental solution of the operator $\scriptstyle{P(D)}$ if it satisfies(I) with $v=\delta.$ , the Dirac measure: (2) $$ \ P(D)E=\delta. $$ The basic result (Theorem 8.5, due to Malgrange and Ehrenpreis) that will be proved here is that such fumdamenlal soluions always exist. Suppose we have an ${\boldsymbol{E}}$ that satisfies (2), suppose $\boldsymbol{\mathit{v}}$ has compact support, and put (3) u = E * U.APPucxros ro DrrxINTIAL Eouxrnos 193 Then ${\boldsymbol{u}}$ is asolution f () becaus (4) $$ P(D)(E*v)=(P(D)E)*v-\delta*v=v $$ then $u\in C^{\infty}(R^{n})$ exists for certain v whose ${\boldsymbol{E}}$ $1/P$ by Theorems 6.35 and 6.37. $P(D)u=0.$ Moreovr . vesmmaoan would be an entir Exercises 5 to 9 However, the equation $P{\tilde{E}}=1$ Teosof-lsunsssense ec $E\star v$ so that its behavior a mation about or thogenoseuian oirnsons………s sssesn $\boldsymbol{E}$ eowsoea Ss ${\boldsymbol{U}}.$ For instance,, $\Gamma\subset{\mathcal{D}}(R^{n}),$ $\hat{E}$ namely whe l max ou apenhnatc ovouno $P{\tilde{E}}=1$ But therout o ane icon ana $E,$ furnishes a oun…sa sesiSsm cansomts e usd n $1/P$ wnos es o…oisesese functon.an ) wouda imy winomsgpopon swe eserei onodo. polyniant he nmne so c onou ieomoos…ua eioieinso Someso…iwisinsrfroi e behavior of cicauy tat is iot enh tostudy $\boldsymbol{P}$ on Aiewesososesesesonluns a noma 'in problems o hisortbut tha th euePgrsqeg $\boldsymbol{\mathit{v}}$ “……iarn-sso $D\!\!\!\!/$ in te ome sa $\textstyle R^{n}$ $Q^{n}$ is iy snica 8.2 Notations $T^{\mathit{n}}$ is htoust cos o l poin that is Lebesgue meas for at least (1) where $\theta_{1},\cdot\cdot\cdot,\theta_{n}$ are el oisthe Har masue o $T^{n},$ ${}^{c}c(x)\neq0$ (2) one z with whe ranes over mulics an $$ w=(e^{i\theta_{1}},\cdot\cdot\cdot,\,e^{i\theta_{n}}) $$ 1f C2)holds an in ${\mathcal{C}}^{n},$ A polynomial in $C^{n}{}_{i}$ ,of degree $N,$ is a functio ure divided by (2n) $$ P(z)=\sum_{|\alpha|\le X}c(\alpha)z^{x}\qquad(z\in C^{n}), $$ $|x|=N,P$ $c(x)\in{\mathcal{C}}.$ $N.$ is sai t havé exact degre 8.3 Lemma P is a polynomiol ${\mathcal{C}}^{n}.$ of cxact degre $N_{\circ}N_{\circ}$ then there s comstan $A<\infty$ , depending only on ${\boldsymbol{P}},$ such tha (1) $$ |f(z)|\leq A r^{-N}{\int}_{T^{n}}|(f P)(z+r w)|\;d\sigma_{n}(w) $$ ad for every $r>0.$ Jor ey eie eein C",Jo evr $z\in C^{n},$194 DISTRIBUTIONS AND FOURIER TRAN NSFORMS PROOF Assume first that ${\mathbf{}}F$ ris an entire function of one complex variable an that (2) $$ Q(\lambda)=c\prod_{i=1}^{N}\;(\lambda+a_{i})\;\;\;\;\;\;\;\;(\lambda\in C). $$ Put $Q_{0}(\lambda)=c\Pi|(1+\vec{a}_{i}\lambda).$ Then $c{\cal F}(0)=({\cal F}Q_{0})(0)$ Since $|Q_{\alpha}|=|Q|$ on the unit circle, t follows that (3) $$ \vert c F(0)\vert\leq{\frac{1}{2\pi}}\int_{-\pi}^{\pi}\vert(F Q)(e^{i\theta})\vert\ d\theta. $$ where ea The given polynomial Pcan be writtcninthe form $P=P_{0}+P_{1}+\cdot\cdot\cdot+P_{N}.$ $\mathbb{N}\mathbf{P}_{j}$ is a homogeneous polynomial of degree j. Define $\textstyle A$ by (4) $$ \frac{1}{A}=\int_{t^{n}}\left|P_{N}\right|\,d\sigma_{n}\,. $$ This integral is positive, since ${\boldsymbol{P}}$ has exact degree $N.$ [See part(O) of Exercise l」 1 $|{\mathrm{f}}\,z\in{\mathcal{C}}^{n}$ and $\operatorname{w}\in T^{n}.$ define (5) $$ F(\lambda)=f(z+r\lambda w),\qquad Q(\lambda)=P(z+r\lambda w)\qquad(\lambda\in\bar{C}). $$ The leading coefficient of ${\cal Q}\,$ is $r^{N}P_{N}(w).$ Hence (3) implies (6) $$ r^{N}|P_{N}(w)|\;|f(z)|\;\leq\frac{1}{2\pi}\int_{-\pi}^{\pi}|(f P)(z+r e^{i\theta}w)|\;d\theta. $$ If we integrate (G) with respect to $\sigma_{n}\,,$ we get (7) $$ |f(z)|\le A r^{-N}\cdot\frac{1}{2\pi}\int_{-\pi}^{\pi}d\theta\int_{T^{n}}|(f P)(z+r e^{i\theta}w)|~d\sigma_{n}(w). $$ The measure $\sigma_{n}$ is invariant under the change of variables $w arrow e^{i\theta}w,$ The inner integral in(T) is therefore independent of ${\boldsymbol{\theta}}.$ This gives (1 / 8.4 Theorem Suppose $D\!\!\!\!/$ is a polynomial in n variables $v\in{\mathcal{D}}^{\prime}(R^{n}),$ and vhas compact support. Then the equation (1) $$ P(D)u=v $$ has a solution with compact support $i f$ and onlyif there is an enire function $\scriptstyle{\mathcal{G}}$ in ${\boldsymbol{C}}^{n}$ such that (2) $$ \scriptstyle P g=\otimes, $$ When this condition satisfied () has a unique souton u with compact support the suppor of ths ulis ini he comvex hull of the support of vAPPLICATIONs ro DFFERENTIAL EOUATIONs 195 .xog .A as sun $\boldsymbol{\mathit{d}}$ wih coma spot 0of Theoem .2 shows that (Q) holds with $\scriptstyle g={\underline{{a}}}$ has itsuport i Conversely suppose (2) holdsfor some entire $|x|\leq r\rangle$ .By Lemma 8.3,(2) implics $\scriptstyle r\gg0$ so that v $r B=\{x\in R^{n};$ $\mathcal{G}$ 9. Choose (3) $$ |g(z)|\le A\int_{r^{n}}^{\infty}|\hat{p}(z+w)|\;d\sigma_{n}(w)\qquad(z\in C^{n}). $$ By Ga of Theorem $\scriptstyle{7.23}.$ there exist ${\boldsymbol{N}}$ and $\gamma$ such that (4) $$ |\hat{v}(z+w)|\leq\gamma(1+\mid z+w|)^{N}\mathrm{exp}\,\{r|\mathrm{Im}\,(z+w)|\}. $$ There are constants $c_{\mathrm{I}}$ and ${\boldsymbol{C}}_{2}$ that satisfy (5) $$ .\cdot\quad1+\mid z+w\mid\leq c_{1}(1+\mid z\mid) $$ and (6) $$ |\operatorname{Im}\left(z+w\right)|\leq c_{2}+|\operatorname{Im}z| $$ for all z ${\mathcal{C}}^{n}$ and all $\nu\in T^{n}.$ It follows from these inequalitis that (D $$ g(z)|\leq B(1+\vert z\vert)^{N}\exp\left\{r\vert\mathrm{Im}\,z\vert\right\}\qquad(z\in C^{n}), $$ that satisfies $P{\hat{u}}={\hat{0}}.$ tis another constant (depending on $\gamma_{\mathrm{i}}$ $A_{\mathrm{f}}$ N, $c_{1;}$ $c_{\mathrm{2}}\;;$ , and r). By $r B.$ Hence where $\boldsymbol{B}$ $\gamma_{23}$ $g={\bar{u}}$ for some distribution ${\boldsymbol{u}}$ with support in $(7)$ and (b) of Thcorem (2) becomes $P{\hat{u}}={\hat{v}},$ which is equivalent to (1). The unes obousne e a mot o enie nction implies The preceding argument showed that the support $S_{u}$ of u lies in every closedball entered at the orgin tha contansth sppot $\mathrm{S}_{v}$ of $\boldsymbol{v}$ Since (D (8) $$ P(D)(\tau_{x}u)=\tau_{x}v\qquad(x\in R^{n}), $$ the same statement is true of $x+S_{u}$ and $x+S_{v}\,.$ Consequently $S_{u}$ lies in the ${\it j}{\it j}{\it j}$ intersecto lsedbals Cented anywhere in R" hatconta the proof is complete. $S_{v}\,.$ Since this intersction is the convexhul or $S_{v}:$ 8.5 Theorem J P is a polynomial in ${\mathcal{C}}^{n},$ of exact degree $N,$ then the diferential operalor $\scriptstyle{\cal{F}}(D)$ has a fundamental solution $\boldsymbol{E}$ that satisfies (1) $$ \left|E(\vartheta)\right|\leq A r^{-N}\int_{T^{n}}\!\!d\sigma_{n}(w)\int_{R^{n}}\!\left|\hat{\psi}(t+r w)\right|\,d m_{n}(t) $$ for every V e9(R") and for every r> 0.196 DISTuBrioNs AND FoURIER TRANSroRMS Here $\textstyle A$ is the constant that appears in Lemma 8.3. The main point of the theorem is the existence of a fundamental solution, rather than the estimate (l) which arises from the proof. pRO0r $\operatorname{fri}x_{r}>0$ and define (2) $$ \left\|\psi\right\|=\int_{T^{n}}\!d\sigma_{n}(w)\int_{R^{n}}\left|\hat{\psi}(t+r w)\right|\,d m_{n}(t). $$ In preparation for the main part of the proof, let us frst show that (3) $$ \operatorname*{lim}_{j arrow\infty}\Vert\psi_{j}\Vert=0\qquad\mathrm{if}\;\psi_{j} arrow0\;\mathrm{in}\;\mathcal{D}(R^{n}). $$ Note that $\hat{\psi}(t+w)=(e_{-\nu}\psi){\bf\cdot}(t)\ \mathrm{i}\mathrm{j}\big|$ $\scriptstyle t\in K^{n}$ and $w\in C^{n}.$ Hence (4) $$ ||\psi||=\int_{T^{n}}\!d\sigma_{n}(w)\int_{\boldsymbol{R^{n}}}\!|(e_{-,w}{\rlap/\psi})^{\times}\!|\ d m_{n}\,. $$ If $\psi_{j}\to0$ in ${\mathcal{D}}(R^{n}),$ all w ${\mathcal{Y}}_{j}$ , have their supports in some compact set $K.$ It follows from the Leibniz $K.$ The func- tions $e_{r\cup r}$ (w ∈ $T^{n}$ are uniformly bounded on formula that (5) $$ \left\|D^{\alpha}(e_{-\,r w}\not\psi_{j})\right\|_{\infty}\leq C(K,\,\emptyset)\mathop{\mathrm{max}}_{\beta\leq x}\left\|D^{\circ}\not\langle j_{j}\right\|_{\infty}. $$ The right side of(5) tends to $0_{\mathrm{{J}}}$ for cvcry ${\mathcal{Q}},$ Hcnce, given $\scriptstyle a>0$ , there exists ${\dot{J}}_{0}$ such that (6) $$ \|(I-\Delta)^{n}(e_{-r w}\psi_{j})\|_{2}<\varepsilon\qquad(j>j_{0},\,w\in T^{n}), $$ where $\Lambda=D_{1}^{2}+\cdots+D_{n}^{2}$ is the Laplacian. By the Plancherel theorem,(6) is the same as (7) $$ \int_{R^{n}} |(1+\vert t\vert^{2})^{n}{\hat{\psi}}_{j}(t+r w)\vert^{2}\,d m_{n}(t)<\varepsilon^{2}, $$ $j>j_{0}$ from which it follows, by the Schwarz inequality and (2), that $\|\psi_{j}\|<C s$ for all ,where (8) $$ C^{2}=\int_{R^{n}}\left(1\,+\,\left|\,t\,\right|^{2}\right)^{-2n}d m_{n}(t)<\infty. $$ This proves (3) Suppose now that $\phi\in{\mathcal{D}}(R^{n})$ and that (9) $$ \psi=P(D)\phi. $$ Thcn ${\hat{\psi}}=P{\hat{\phi}},$ $\hat{\phi}$ and $\hat{\mathcal{\psi}}$ are entire, hence $\vartheta\;$ determines $\phi.$ In particular, dp(O) is a linear functional of ${\boldsymbol{\psi}},$ defined on the range of $P(D).$ The crux of the proofAPPLICATIONs To DIFFERENTIAL EOUATIONs 197 bution $u\in{\mathcal{D}}^{\prime}(R^{n})$ that satisie consis in showing that ths functonalis continuous, i., tht there s dst (10) $$ u(P(D)\phi)=\phi(0)\qquad(\phi\in{\mathcal D}(R^{n})), $$ because then the distribution $E={\vec{u}}$ satisfies $$ \begin{array}{c}{{(P(D)E)(\phi)=E(P(-D)\phi)=u((P(-D)\phi)^{\infty})}}\\ {{=u(P(D)\phi)=\tilde{\phi}(0)=\hat{\phi}(0)=\delta(\phi),}}\end{array} $$ so that $P(D)E=\delta$ ,as desired. yields Lemma 8.3, applied to $P{\dot{\phi}}={\dot{\psi}},$ $$ \quad\quad\quad\quad\quad\quad\quad\quad\quad\quad\Big|\hat{\phi}(t)|\leq A r^{-N}\int_{r n}|\hat{\psi}(t+r w)|\;d\sigma_{n}(w)\qquad(t\in R^{n}). $$ By the invcrsion theorem, 4(0) = r,p lm Thus(11), 2), and O) give (12) $$ \vert\phi(0)\vert\leq A r^{-N}\vert P(D)\phi\vert\vert\qquad(\dot{\phi}\in{\mathcal D}(R^{n})). $$ $\phi\in{\mathcal{D}}(R^{n})$ tional that is defined on ${\cal{Y}}$ be the subspace of 9(K") thal consists of the functions $P(D)\phi,$ Let $\boldsymbol{\mathit{I}}$ .By (12), the Hahn-Banach theorem 3.3 shows that the linear fn Y by $P(D)\phi arrow\phi(0)$ extends to a linear functional u on ${\mathcal{D}}(R^{n})$ that satisfes (10) as well as (13) $$ \left|u(\psi)\right|\leq A r^{-N}\|\psi\|\qquad(\psi\in{\mathcal{D}}(R^{n})). $$ By (3) $u\in{\mathcal{D}}^{\prime}(R^{n})$ This completes the proof // Elliptic Equations set 8.6 Introduction If uis a twice continuously differentiable function in some open $\Omega\subset R^{2}$ that satisfies the Laplace equation (1) $$ {\frac{\partial^{2}u}{\partial x^{2}}}+{\frac{\dot{\sigma}^{2}u}{\dot{\sigma}y^{2}}}=0, $$ function in $\Omega$ then itis very well known that uis actually n $C^{\alpha}(\Omega),$ simply because every real harmonic interest to se, frst ofal, tht the equation 2 is ocally he real part of a holomorphic function. Any theorem of this type-onc which asserthat very solution of crain diffential equation has stronger smoothness propertics than is a prorievident-is called egularity theorem we shall give a profof a rather general regularity theorem for elic part diTerential cquations. The term"ellitc”il bedefned presenty. It may be of (2) a-u ${\overline{{\partial x\cdot\!\partial y}}}=0$198 DISTRIBUrIoNS AND FoURIER TRANSFORMS behaves quite differently from (1), since it is satisfied by every function $\boldsymbol{\mathit{l}}$ of the form $u(x,\,y)=f(y),$ where $\boldsymbol{\mathit{f}}$ is any differentiable function. In fact, if $\operatorname{\left(2\right)}$ is interpreted to mean (3) $$ \frac{\partial}{\partial y}\left(\frac{\partial u}{\partial x}\right)=0, $$ then f can be a perfectly arbitrary function. 8.7 Definitions Suppose $\Omega$ is open in $\textstyle{\mathcal{R}}^{n}$ $\textstyle N$ is a positive integer $f_{\alpha}\in C^{\infty}(\Omega)$ for every multi-index a with $|\alpha|\leq N,$ and at least one f。 with $\left|\alpha\right|=N\ ;$ is not identically O These data determine a linear differential operator (1) $$ L=\sum_{|\alpha|\leq N}f_{\alpha}\,D_{\alpha} $$ which acts on distributions $u\in{\mathcal{D}}^{\prime}(\Omega)$ by (2) $$ L u=\sum_{|\alpha|\leq N}f_{\alpha}\,D_{\alpha}u. $$ The order of $\boldsymbol{\mathit{L}}$ L is $N.$ The operator (3) $$ \sum_{|x|=N}f_{\alpha}\,D_{\alpha} $$ is the principal part of $\boldsymbol{\mathit{L}}$ . The characteristic polynomial of T $\underline{{L}}$ (4) $$ p(x,y)=\sum_{|x|=N}f_{x}(x)y^{x}\qquad(x\in\Omega,\,y\in R^{n}). $$ coefficients in C $C^{\circ}$ This is a homogeneous polynomial of degree ${\cal N}$ in the variables $y=(y_{1},\ldots,y_{n}),$ with *(Q2). The operator ${\boldsymbol{L}}$ is said to be elliptic if $p(x,y)\neq0$ for every $x\in\mathbb{Q}$ and for every $y\in R^{n}$ ,except, of course,、 for $y=0.$ Note that ellipticity is defined in terms of the principal part of $L{\mathord{}}:$ the lower-order terms that appear in (1) play no role. For exyample, the characteristic polynomial of the Laplacian (5) $$ \Delta={\frac{{\hat{\sigma}}^{2}}{{\hat{\sigma}}x_{1}^{2}}}+\cdot\cdot\cdot+{\frac{{\hat{\sigma}}^{2}}{{\hat{\sigma}}x_{n}^{2}}} $$ is p(x $y)=-(y_{1}^{2}+\cdot\cdot\cdot+y_{s}$ ,) so that $\hat{\Delta}$ is elliptic On the other hand,if $L=\emptyset^{2}/\bar{\sigma}x_{1}\;\bar{\sigma}x_{2}\,.$ ,then $p(x,y)=-y_{1}y_{2}\,,$ and $\boldsymbol{\mathit{L}}$ is not elliptic The main result that we are aiming at CTheorem 8.12) involves some specia spaces of tempered distributions, which we now describc 8.8 Sobolev spaces Associate to each real number sa positive measure $\mu_{s}$ On $R^{n}$ by setting $\operatorname{\mathcal{(1)}}$ $d\mu_{s}(y)=(1+\mid y\mid^{2})^{s}\,d m_{n}(y).$Arucxos orort onos 19 Za ${\mathrm{If}}f_{\in}L^{2}(\mu_{x}),$ thatis, if $|j f|^{2}$ dus ${}\leq\,\infty,$ uhe e dsistion Example eo o of al ${\mathcal{U}}$ so otanewiw enoei $H^{s};$ srsus……esis snw se equipped with the norm (2) $$ \|u\|_{s}=\left(\int_{R^{n}}\|{\hat{u}}\,\|^{2}\,d\mu_{s}\right)^{1/2} $$ $H^{s}$ to each $H^{s}$ These spaces is lery iotily sophic $L^{2}(\mu_{s}).$ $\textstyle H^{-t}\!:$ note that need no b ante is therefore a lt is obvious that is continuousmappingo $H^{s}$ into tn y … emisnw nt onou of all spaces $H^{s}$ $H^{s}$ $H^{0}=L^{2}$ ou n e twmamn ifasis By the Planchet htocem vctor spac. ineir operato $H^{*}\subset H^{t}:f\colon\ell<s.$ Thc union $\textstyle X$ $\Lambda\,:$ $X\to X$ is si. hae orer tiecinorA Htre e essoiveses siv e 8.9 Theorem (b) $J^{\zeta}-\infty<t<\infty,$ fzezytn ninmcmonc on i $\dot{\boldsymbol{\imath}}$ som $H^{s}$ (a) he mapping $u\to v$ gien by (d) For every mline” $\scriptstyle{\mathcal{A}}$ $D_{\alpha}$ $$ \hat{v}(y)=({\bf i}\ +\vert y\vert^{2})^{t/2}\hat{u}(y)\qquad(y\in{\cal R}^{n}) $$ thefore an opeato o oder ${\mathbf{}}I$ whose (e) $U f\in{\mathcal{S}}_{n},$ is a inear iometry y $-\,l$ onto $H^{\circ}:a n d:$ ${\boldsymbol{\mathrm{O}}}.$ an operator o order ${\mathfrak{O}}$ inverse has order $H^{s}$ (c) 1/be $L^{\infty}(R^{n}),$ y , iun y 一~0 $\bar{\cal L}{\cal S}$ then $u arrow f u$ is aoero orer T is an operaio ore (1) PRoor If $u\in{\mathcal{D}}^{\prime}(R^{n})$ has oma uso. o The .3 sows a $$ |\hat{a}(y)|\leq C(1+\vert y\vert)^{N}\qquad(y\in R^{n}), $$ for some constans $\textstyle{\bar{C}}$ and $N.$ Hence ${\mathrm{,e}}\in H^{p}$ if s $<-N-n/2$ This proves part (0) b and e ar ovos Teati implies $$ |(D_{\alpha}u)^{\times}(y)|=_{,}y^{\alpha}|\;|\hat{a}(y)|\leq(1\,+\,|y|^{2})^{|x|/2}|\hat{a}(y)| $$ (2) $$ \|D_{\alpha}u\|_{s-|\alpha|}\leq\|u\|_{s}, $$ so that $(d)$ holds. The prof e depend on te euli (3) $$ (1+|x+y|^{2})^{s}\leq2^{|s|}(1+|x|^{2})^{s}(1+|y|^{2})^{|s|}, $$200 Disraros ANp rourueR TRASsronws valid for $x\in R^{n},y\in R^{n},-\infty<s<\infty.$ The case by $x-y$ and then $y\ {\mathrm{by}}\ -y$ The it the case $s=-1$ is obtained by replacing $\scriptstyle{\mathcal{X}}$ $\scriptstyle{s=1}$ of $\mathbf{(3)}$ is obvious. From general case $\operatorname{of}\left(3\right)$ is obtained from these two by raising everything to the power ls|. It follows from (3) that (4) $$ \int_{R^{n}}\mid h(x-y)\mid^{2}d\mu_{s}(x)\leq2^{\mid s\mid}(1+\mid y\mid^{2})^{\mid s\mid}\int_{R^{n}}\mid h\mid^{2}d\mu_{s} $$ for every measurable function $\dot{\boldsymbol{h}}$ on $\textstyle R^{n}.$ Now suppose $u\in H^{s},f\in{\mathcal{G}}_{n}.$ 、t > $|s|\,+\,n/2.$ Since ${\hat{f}}\in{\mathcal{G}}_{n}$ ll ll < OO. Pu $t~\gamma=\mu_{|s|-t}(R^{n})$ . Then $\gamma\ll\infty.$ Define $F=|{\hat{a}}|*|f|.\ \mathbf{By}$ Theorem 7.19, (5) $$ |(f u)^{\times}|=|{\hat{a}}*{\hat{f}}|\leq|{\hat{a}}|*|{\hat{f}}|=F. $$ By the Schwarz inequality, (6) $$ |F(x)|^{2}\leq\int_{\boldsymbol{R^{n}}}|{\hat{f}}(y)|^{2}\ d\mu_{t}(y)\int_{\boldsymbol{R^{n}}}|{\hat{u}}(x-y)|^{2}\ d\mu_{-t}(y) $$ for every $x\in R^{n}$ lIntegrate $\mathbf{\tau}(6)$ over $\textstyle R^{n}.$ , with respect to $\mu_{\mathrm{s}}\,.$ By (4), the result is (7) $$ \int_{R^{n}}\vert{\cal F}\vert^{2}\,d\mu_{s}\leq2^{|s|}\gamma\vert\vert\ f\vert^{2}\vert\vert u\vert\vert_{s}^{2}. $$ It follows from (5) and $(7)$ that (8) $$ \|f u\|_{s}\leq(2^{|s|}\gamma)^{|j|\,|f\,\|_{t}\|\,u\|_{s}\,. $$ This proves (e) // ${\cal{H}}^{s}$ 8.10 Definition Let S2 be open in $\textstyle K^{n}\!.$ A distribution $u\in{\mathcal{D}}^{\prime}(\Omega)$ is said to be locally in if thcrc corrcsponds to cach point $x\in\Omega$ a distribution $v\in I I^{s}$ such that $u\subseteq v$ some neighborhood o of $X.$ (See Section 6.19.) 8.11 Thcorem 1f $u\in{\mathcal{D}}^{\prime}(\Omega)$ and $-\cos\ <S\times G,$ the following two statements are equivalent: (b) (a)uis locll $I I^{s}.$ $\textstyle\psi\in\mathcal{D}(\Omega).$ $\scriptstyle{\sqrt{n}}\;\in\;H^{5}$ for every Moreover, if s is a nonnegative integer, a) and (b) are equivalent to (c) $D_{s}u$ is locally ${\cal L}^{2}$ for every α with $|\alpha|\leq s$ Statement (b) may need some clarification, since $u\quad$ acts only on test functions whose supports lie in $\Omega$ However, yu is the functional thtassigns to each $\phi\in{\mathcal{D}}(R^{n})$ the number $$ (\psi u)(\phi)=u(\psi\phi). $$ Note that ype 9(Q2), so that $u(\not\to\phi)$ is defined.ArrLcAnioNs ro DrRNTAL rouorous 201 $\textstyle K$ and in which $\boldsymbol{u}$ Assume u is locally on $K.$ If $\phi\in{\mathcal{D}}(R^{n})$ it flows tha $\psi_{i}\in{\mathcal{D}}(\omega_{i})$ $K,$ PR00F $\textstyle\sum\psi_{i}=1$ coincides with somm $H^{s}.$ Let $\textstyle K$ be the support of somc $\textstyle\psi\in{\mathcal{D}}(\Omega).$ Since such that is compacter nit man oen se $v_{i}\in H^{*}$ There exist functions $\omega_{i}\subset\Omega,$ whose union covers $u=\textstyle{\sqrt{u}\,\mathrm{in}\,\,\omega}$ Assume again that $\mathbf{\nabla}(b)$ holds. Ir $$ u(\not\psi\phi)=\sum u(\not\psi_{i}\psi\phi)=\sum v_{i}(\not\psi_{i}\psi\phi), $$ By Ge) of Theorem 8.9 $(a).$ hence $D_{x}(\psi u)$ ∈ since $\textstyle\bigvee_{i}\surd\psi\backslash\in\mathcal{D}_{i}(\omega_{i})$ Thus $\textstyle\sqrt{u}=\sum\psi_{i}\psi v_{i}.$ $\textstyle\psi_{i}\backslash\upsilon_{i}\in H^{s}$ for each i. This $\psi u\in H^{s},$ and (a mplies (0) is lin a neighborhood o f x, then If $\mathbf{\nabla}(D)$ holds, ${\mathrm{if~}}x\in\Omega,$ and if $\textstyle\psi\in{\mathcal{D}}(\Omega)$ , and $\psi u\in H^{s}$ hy asumption. Thus (b mplie $H^{n-|x|},$ $\psi\in{\mathcal{D}}(\Omega),$ then $\textstyle\psi u\in H^{s},$ by d) of Theorem . Tvi s, the $$ .\qquad H^{s-\left|s\right|}\subset H^{0}=L^{2}(R^{n}). $$ Thus $A_{\alpha}(\sqrt{u})\in L^{2}(R^{n}).$ Taking in $\Omega$ Thus O) implies e if $|x|\leq s.$ Hence Fix $\psi\in{\mathcal{D}}(\Omega).$ shows tha $D_{x}u$ is locall $\psi=1$ in some neighborhood of a poin $\left|\alpha\right|\leq s.$ $x\in\Omega$ Finally, asum ${\mathcal{L}}^{2}$ for every a with $\scriptstyle D_{s}u$ is lcall ${\cal L}^{2}$ The Leibni forula shows tha $D_{\alpha}(\psi u)\in L^{2}(R^{n})$ (r) $$ \int_{R^{n}}|y^{\alpha}|^{2}|(\langle\psi u\rangle^{\times}(y)|^{2}\,d m_{n}(y)<\,\infty\qquad(|\alpha|\leq s). $$ t i ngngtientet, ols uh hemoma $y_{1}^{s},\cdot\cdot\cdot,y_{n}^{s}$ in place of y” I folows as the ro Theorem .25, ta ((2) $$ \int_{R^{n}}(1\,+\,|y|^{2})^{s}|(\psi u)^{\times}(y)|^{2}\;d m_{n}(y)<\infty. $$ Thus $\psi u\in H^{\circ},$ (e impies $(\d a),$ and the proof is complete // 8.12 Theorem sum $\Omega$ is an open set im $R^{n}.$ and (a ${\cal L}=\sum f_{\alpha}\,D_{\alpha}$ is alinelifeial operato $\mathbb{Q}_{,}$ of order $N\geq1$ ,with (b) coeficents $\in{C^{\infty}(\Omega,\lnot)}$ ), is a constant, (c) for each α with $|\alpha|=N,f_{\alpha}$ that satisf u and vare distributions in $\Omega$ (1) $H^{s}.$ $$ \mathbf{\omega}_{\circ}\cdot\qquad L u=v, $$ and vis locall Then uis ocall $I I^{s+N}.$ Corollary J L satisies (o) and (b) and ${\mathcal{I}}\ v\in C^{\infty}(\Omega),$ then every solution $\mathrm{2}M=0$ of() belongs to $C^{\infty}(\Omega).$ M paicuan ey luio he omoesuesuni is i $C^{\infty}(\Omega).$202 DISTRIBUTIONS AND FOURIER TRANSFORMS For if v∈ $C^{\infty}(\Omega),$ then $\psi v\in{\mathcal{D}}(R^{n})$ for every $\psi\in{\mathcal{D}}(\Omega);$ hence vis locall $H^{s}$ for every s, and the theorem implies that uis locally $H^{s}$ for every $S^{\circ}$ it follows from Theorems 8.11 and 7.25 that $u\in C^{\infty}(\Omega).$ Assumption((b) can be dropped from the theorem, but its presence makes the proof considcrably casicr. PROOF Fix a point $x\in\Omega,$ let $B_{0}\subset\Omega$ be a closed ball with center at $X_{},$ and let $\phi_{0}\in{\mathcal{D}}(\Omega)$ be l on some open set containing $B_{0}$ $\mathfrak{h y}\left(a\right)$ of Theorem 8.9. $\phi_{0}\,u\in I I^{n}$ for some ${\bar{t}}.$ Since $H^{t}$ becomes larger as ${\mathbf{}}I$ decreases, we may assume that ${\mathit{i}}=$ $s+N-k,$ where $\boldsymbol{K}$ is a positive integer. Choose closed balls $$ B_{0}\supset B_{1}\Rightarrow\cdots\cdots\Rightarrow B_{k}, $$ each centered at $X,$ and each properly contained in the preceding one.Choose $\phi_{1},\,.\,.\,.\,,\,\phi_{k}\in\mathcal{D}(\Omega)$ so that $\phi_{i}=1$ on some open set containing $B_{i},$ and $\phi_{i}=0$ off $\scriptstyle B_{t-1}$ Since $\phi_{0}u\in H^{\prime},$ the following “bootstrap” proposition implies that $$ \phi_{1}u\in H^{t+1},\ldots,\phi_{k}u\in H^{t+k}. $$ It therefore leads to the conclusion that $\boldsymbol{u}$ is locally $H^{s+N},$ because $t+i k=s+N$ and $\phi_{k}=1$ on $\ B_{k}\,.$ Proposition(, in addition to the hypotheses of Theorem 8.12, yu e H" for somte $t\leq s+N-1$ and for some $y\in{\mathcal{D}}(\Omega)$ which is l on an open set containing the support of a function $\phi\in{\mathcal{D}}(\Omega).$ ther $\phi u\in H^{t+1}$ pRoOF We begin by showing that (2) $$ L(\phi u)\in H^{t-N+1}. $$ Consider the distribution (3) $$ \Lambda=L(\phi u)-\phi L u=L(\phi u)-\phi v. $$ Since its support lies in the support of d $\varnothing,$ $\boldsymbol{\mathit{l}}$ can be replaced by yu in (3), without changing $\Lambda$ (4) $$ \Lambda=L(\phi\psi u)-\phi L(\psi u)=\sum_{\mid x\mid\leq N}f_{x}\cdot \lbrack D_{\alpha}(\phi\psi u)-\phi D_{\alpha}(\psi v) \rbrack. $$ order $\textstyle N$ If the Leibniz formula is applied to $D_{\alpha}(\phi\cdot\psi u),$ one sees that the derivatives of of yu cancel in (4). Therefore $\Lambda$ is a linear combinationi Iwith coeff cients in ${\mathcal{D}}(R^{n})|$ of derivatives of $\textstyle\psi u,$ of orders at most $N-1.$ Since $\psi u\in H^{t},$ parts (d) and $\mathbf{\Psi}(e)$ of Theorems 8.9 imply that $\Lambda\in H^{t-N+1}$ By Theorem 8.11, $\phi v\in H^{s},$ and since $t-N+1\leq s,$ we have $\beta v\in H^{t-N+1}.$ Now(2)follows from (3).ArPLucAros utxrL ouxros 203 Sinc $\underline{{L}}$ is lit s hareic ponomi (5) 厂 $\textstyle R^{n}\!\!$ except at $y=0.$ Definefunction $$ p(y)=\sum_{|x|=N}\;f_{\alpha}y^{\alpha}\qquad(y\in R^{n}) $$ has nd zero i (6) $$ q(y)=\mid y\mid^{-N}p(y),\qquad r(y)=(1\ +\mid y\mid^{N})q(y), $$ for $y\in R^{n},y\neq0;$ and define operator ${\mathcal{O}},{\mathcal{R}},{\mathcal{S}}$ on the union of the Sobolev spaces by (7) $$ ({\cal Q}{\mit w})^{\star}=q{\hat{w}},\qquad({\cal R}{\mit w})^{\star}=r{\hat{w}} $$ and (8) $$ \mathrm{\boldmath~\Psi~}\cdot\mathrm{\boldmath~\Gamma~}\circ\mathrm{\boldmath~\cal~S=}\sum_{|\alpha|<N}\psi f_{\alpha}\,D_{\alpha}. $$ -N Go or Thorm 8., ha $\textstyle R$ p sa homogencous polynomial f dgre $N,$ $q(\lambda y)=q(y)$ if ${\boldsymbol{\lambda}}>0,$ of tions on Since p $D\!\!\!\!/$ and $1/q$ are bounded functions. t fllows fFrm $0.$ $\textstyle R^{n}$ Theorem 8.9 that both ${\mathcal Q}\,$ and ${\mathcal{O}}^{-1}$ nsier ins y t onmn momsies eaenepe $\left(c\right)$ implies that both $\boldsymbol{\mathit{q}}$ Since both $(1_{\cdot\cdot}+\mid y\mid^{2})^{-N/2}(1\;+\;\mid y\mid^{N})$ are operators of order wioeme $R^{-1}$ has order and $\textstyle{\mathcal{R}}^{n},$ is moparo oe $\textstyle N$ and is reciprocal are bounded func $\mathbf{\nabla}(b)$ i fow om eignpararphs mbiew have Since $\textstyle\sqrt{f}_{\alpha}\in{\mathcal{D}}(R^{n})$ it folow from d) and $(e)$ of Theorem 8.9 that $\boldsymbol{\mathsf{S}}$ is an operator of order $N-1$ and sin $D\!\!\!\!/$ is sucd ohae ostctofien $f_{\alpha}\,,$ fo,We Since $p=r-q,$ (9) $$ {\binom{}{}_{|\alpha|=N}}g_{\alpha}D_{\alpha}w\Big)\sim p{\hat{w}}=(r-q){\hat{w}}=(R w-Q w)^{\star} $$ if w les in some Sobolev space Hence (10) $$ (R-Q+S)(\phi u)=L(\phi u). $$ Henc By G2) $L(\phi u)\in H^{t-N+1},$ $\phi\psi=\phi,$ (e) of Theorem 8.9 implies tha $\phi u=\phi\backslash\mu\in H^{t}.$ since $\psi u\in H^{t}$ and (11) $$ (Q-S)(\phi u)\in H^{t-N+1}, $$ because ${\cal Q}\,$ has order ${\boldsymbol{0}}$ and. $\boldsymbol{\mathsf{S}}$ ' has order $N-1\geq0.$ lt now follows from(IO) that $\scriptstyle(12)$ $R(\phi u)\in H^{\prime-N+1},$ and since $R^{-1}$ has order -M, we fialy coclde that ue $\scriptstyle{H^{*,1}}$ ${\it j}/j{\it j}$204 DISTRIBUTIONS AND FoURIER TRANSFORMs 8.13 Example Suppose $\boldsymbol{\mathit{L}}$ is an elliptic differential operator in ${\boldsymbol{R}}^{n}$ , with constant coefficients, and ${\boldsymbol{E}}$ is a fundamental solution of ${\underline{{L}}}.$ In the complement of the origin the equation $L E=\delta$ reduces to $L E=0.$ Theorem 8.12 implies therefore that, except at the origin, ${\boldsymbol{E}}$ is an infinitely differentiable function. The nature of the singularity of ${\boldsymbol{E}}$ at the origin depends, of course, on $\underline{{L}}$ If Q is open in 8.14 Example The origin in $R^{2}$ is the only zero of the polynomial $p(y)=y_{1}+i y_{2}$ $R^{2}.$ P,and if $u\in{\mathcal{D}}^{\prime}(\Omega)$ is a distribution solution of the Cauchy-Riemann equation $$ \left(\frac{\partial}{\partial x_{1}}+i\frac{\dot{\partial}}{\partial x_{2}}\right)u=0, $$ $z=x_{1}+i x_{2}$ Theorem 8.12 implies that $u\in C^{\infty}(\Omega)$ It follows that ${\boldsymbol{u}}$ is a holomorphic function of in $\Omega.$ In other words, every holomorphic distribution is a holomorphic function. Exercises The following simple properties of holomorphic functions of several variables were tacitly used in this chapter. Prove them. (a) If fis entire in ${\mathcal{C}}^{n},$ if $\operatorname{w}\in C^{n},$ and $\mathrm{i}\mathbf{f}$ p(A) $=f(\lambda w),$ then $\phi~~~~~~~~~~~~~~~~~~~~~~~~~~~~~~~~~~~~~~~~~~~~~~~~~~~~~~~~~~~~~~~~~~~~~~~~~~~~~~~~~~~~~~~~~~~~~~~~~~~~~~~~~~~~~~~~~~~~~~~~~~~~~~~~~~~~~~~~~~~~~~~~~~~~~~~~~~~~~~~~~~~~~~~~~~~~~~~~~~~~~~~~~~~~~~~~~~~~~~~~~~~~~~~~~~{~~~~~~~~~~~~~~~~~~~~~~~~~~~~~~~~~~~~~~~~~~~~~~~~~~~~~~~~~~~~~~~~~~$ is an entire function of one complex variable. (b) If $D\!\!\!\!/$ is a polynomial in $C^{n}$ and if $$ \int_{r^{n}}\left|P\right|\,d\sigma_{n}=0 $$ then ${\boldsymbol{P}}$ (c) If Pis a polynomial (not identically $\mathbf{0} )$ and $\scriptstyle{\mathcal{G}}$ $\textstyle{\int_{r^{n}}\left|P\right|^{2}d\sigma_{n}.}$ $\textstyle{C^{n}}_{!}$ then there is P is identically O. Hint: Compute g is an cntirc function in at most one entire function fthat satisfies $P f=g$ Find generalizations of these three properties Prove t tement abou convex ull mad inth asenc of theproof o Theo- rem 8.4. 3 Find a fundamental solution for the operator $\vartheta^{2}/\partial x_{1}\;{\mathcal{C}}x_{2}$ in $R^{2}.$ (There is one that is thc characteristic function of a certain subset of $\scriptstyle R^{2}\,_{3}$ 4 Show that the equation $$ {\frac{\partial^{2}u}{\partial x_{1}^{2}}}-{\frac{\dot{\sigma}^{2}u}{\partial x_{2}^{2}}}=0 $$ is satisfed Gnthe distribution sensy by every locall inegrable function $\boldsymbol{u}$ of the for $$ u(x_{1},\,x_{2})=f(x_{1}+x_{2})\qquad\mathrm{or}\qquad u(x_{1},\,x_{2})=f(x_{1}-x_{2}) $$ and that even classical solutions $0.{\overset{\frown}{\doublebarwedge}}.$ twice continuously differentiable functions) need not be in $\scriptstyle{C^{\bullet}.}$ Note the contrast between this and the Laplace equation.ornucxros o rxrnaL onos 205 For ${\mathfrak{c r}}\in R^{3}.$ define $f(x)=(1+\vert x\vert^{2})^{-1}.$ in $R^{3},$ $\operatorname{rind}f.$ by directcoputationanaso by th soluio ne oea Show that $f\in L^{2}(R^{3})$ and that fis fundamenta followng reasong $I-\Delta$ 6 (a) Since $\boldsymbol{\mathit{f}}$ is a radial function $(\mathbf{i},\mathbf{e},\gamma$ on ht ens ony on e isance rom the (c)If 6O Away from the origin oisesesue exce itemsn and $f\in C^{\circ}.$ ${\overline{{7}}}.$ For O $<\lambda<n$ and $x\in R^{n}.$ define $(I-\Delta){\dot{f}}=0,$ suis n gsia ifgrnaeqution $F(\left\vert y\right\vert)=f(y),\ \left(b\right)$ implies that ${\mathbf{}}F$ $f(y)=(\pi/2)^{1/2}|y|^{-1}\exp{(-\|y|)}.$ Do te same wit (0,o that an asly esoiveieiciy“ams you wil me Besel fnction $\textstyle{R^{n}}$ in place of $R^{3}\,;$ Show that $$ K_{\star}(x)=|x|^{-\lambda}. $$ (a) $$ \hat{K}_{\lambda}(y):=c(n,\lambda)K_{n-\lambda}(y)\qquad(y\in R^{n}), $$ where $\begin{array}{c c c c}{{}}&{{\star}}&{{\textstyle}}&{{\textstyle}}&{{\textstyle}}&{{}}\\ {{}}&{{}}&{{\star}}\end{array}$ · For these Sugestion ${\mathrm{If~}}n<2\lambda<2n,$ $K_{\lambda}$ $$ c(n,\,\lambda)=2^{n/2-\lambda}\Gamma\left(\frac{n-\lambda}{2}\right)\int\Gamma\left(\frac{\lambda}{2}\right). $$ .'-function and an $L^{2}$ -function. ${\boldsymbol{\lambda}},$ is the sum of an $L^{1}$ Fouaion edec m e noioeienoaino $$ K_{\lambda}(t x):=\varepsilon^{-\lambda}K_{\lambda}(x)\qquad(x\in R^{n},\,t_{\begin{array}{l}{{{\bf{\scriptstyle{\sim}}}}}\\ {{{\sim}}}\end{array}}0). $$ 7 from The case $0<2\lambda<n$ fl…… seso nom n es tunuion $c(n,\,\lambda)$ can be computed , show $\mathbf{A}$ Take $\coprod f{\hat{\phi}}=\coprod f{\hat{\phi}},$ passage t th Iimi givs te case $2\lambda=n.$ The constants is a fundamental that solution 。 with $\cdot{\phi(x)}=\exp{(-\mid x\mid^{2}/2)}.$ $-c(n,\,2)K_{n-2}$ $\textstyle R^{3},$ $\scriptstyle n\geq3$ solution of the Laplacian $\Lambda$ in ${\boldsymbol{R}}^{n}$ in Exrcise o,anduce h and $\lambda=2$ $\Delta\mu=v\ \mathbf{is}$ given by For example as omatsopori 8 detii $\textstyle R^{2}$ and ${\boldsymbol{C}}$ (so that $z\equiv x_{1}\div i x_{2});$ $$ u(x)\doteq-{\frac{1}{4\pi}}\int_{R^{3}}\|x-y\|^{-1}v(y)\,d y. $$ put Show that heFourer transomor $$ \dot{\bar{\varrho}}\,=\frac{\bar{\theta}}{\bar{\langle x_{1}\rangle}}-i\,\frac{\partial}{\bar{\varrho}x_{2}}\ ,\qquad\bar{\theta}\,-\frac{\dot{\varrho}}{\partial x_{1}}\,+\,i\,\frac{\dot{\varrho}}{\partial x_{2}}\ . $$ $-i/z.$ $1/z$ (regared as a tepeditrtuion s Show a ses siuaamen ceacinon $$ \phi(z)\,=\,-\int_{R^{2}}(\overline{{{\partial}}}\phi)(w)\,\frac{d m_{2}(w)}{w-z}\qquad[\phi\in\mathcal{D}(R^{2})]. $$ Since elog $\Big|\;w\Big|\;\longrightarrow\;\sum1/w^{\prime}~\mathrm{a}\mathrm{I\!I}$ d $\Delta={\hat{\sigma}}{\hat{\partial}},$ deduce tha $$ \phi(z)=\int_{R^{2}}(\Delta\phi)(w)\;\mathrm{log~}|w-z|\mathrm{~}d m_{2}(w)\qquad[\phi\in\mathcal{D}(R^{2})]. $$ Thus o isa funausouo Lapascan $\textstyle R^{2}.$206 DISTRuBurioNs AND FOURIER TRANSFoRMs 9 Use Exercise 6 to compute that $$ \operatorname*{lim}_{x\to0}\,\left[e^{-1}\quad b-{\bar{K}}_{2-c}(y)\right]=\log\,\left|y\right|\qquad(y\in R^{2}), $$ whcre bis ctan constant. Show tathis leads to another proof o the las tatement $I O$ in Exercise 8 $\therefore P(D)=D^{2}+a D+b I.$ (We are now in the case $n=1.)$ Let fand g be solutions Suppos of $P(D)_{H}=0$ which satisfy $$ f(0)=g(0)\qquad\mathrm{and}\qquad f^{\prime}(0)-g^{\prime}(0)=1. $$ Define $$ G(x)= \{f(x)\qquad{\mathrm{if~}}x\leq0, $$ and put $$ \Lambda\phi=-\int_{-\infty}^{\infty}\phi(x)G(x)\,d x\qquad[\phi\in{\mathcal D}(R)]. $$ Prove that $\Lambda$ is a fundamental solution of $\scriptstyle{P(D)}$ 11 Suppose ${\boldsymbol{u}}$ is a distribution in ${\boldsymbol{R}}^{n}$ whose first derivatives $D_{1}u,\,\cdot\cdot\cdot,\,D_{n}u$ are locally $L^{2}$ Prove that $\textstyle u$ is then locally $L^{2}.$ .Hint: If $\textstyle\psi\in{\mathcal{D}}(R^{n})$ is lin aneighborhood of the origin and if $\Delta{}E=\delta,$ then $\Delta(\psi E)-\delta\in{\mathcal{D}}(R^{n}).$ Hence $$ u-\sum_{i=1}^{n}\,(D_{i}\,u)*\,D_{i}(\psi E) $$ 12 is in C"(R"). Each $D_{i}(\psi E)$ is an $L^{\mathrm{i}}$ -function with compact support. is a continuous function. Prove Suppose $u\quad$ is a distribution in ${\boldsymbol{R}}^{n}$ whose Laplacian $\Delta u$ that uis then a continuous function. Hint: As in Exercise 11, $$ u-(\psi E)*(\Delta u)\in C^{\infty}(R^{n}). $$ 1 3 Prove analogues of Exercises 1l and $12,$ with ${\boldsymbol{R}}^{n}$ replaced by an arbitrary open set $\mathbb{Q}.$ 14 Show, under the hypotheses of Exercise 12, that (a) a-u/oxf is locally $L^{2}.$ , but (b)21u/xf need not be a continuous function Outline of (b) for periodic distributions in $R^{2}$ (Exercise 22, Chapter $7)\colon\operatorname{If}g\in C(T^{2})$ has Fourier coefficients ${\tilde{g}}(m,n)$ and if fis defined by $$ \tilde{f}(m,n)=(1+m^{2}+n^{2})^{-1}\tilde{g}(m,n), $$ then fe $C(T^{n})$ and $\Delta f=f-g\in C(T^{2}),$ since E $|f(m,n)|<\infty$ The Fourier coefficients then of a3faxf are $-m^{2}f(m,n)$ I $\scriptstyle{\hat{\sigma}}^{2}f/{\hat{x}}x_{1}^{2}$ were continuous for every ${\mathcal{G}}.$ Hence there would be a (o2f1oxA)0,0 would be continus inear functional of $g\in C(T^{2}),$ complex Borel measure $\boldsymbol{\mathit{I}}$ L on $T^{2},$ with Fourier coefficicnts $$ \hat{\mu}(m,n)=\frac{m^{2}}{1+m^{2}+n^{2}}\,. $$ The next exercise shows that no such measure existsArrucros o rexrrL oarows。 207 ${\boldsymbol{\mathit{1}}}{\boldsymbol{5}}$ If is a complex Borel measure o $T^{2}{}_{!}$ , and if $$ \gamma(A,B)=\frac{1}{(2A+1)(2B+1)}\sum_{n=-A}^{A}{}_{m}\sum_{m=-B}^{B}\hat{\mu}(m,n), $$ prove that $$ .\cdot\operatorname*{lim}_{s arrow\infty}\left[\operatorname*{lim}_{a\rightarrow\infty}\gamma(A,B) ]=\operatorname*{lim}_{B arrow\infty}\left[\operatorname*{lim}_{A\rightarrow\infty}\gamma(A,B) ]. $$ Suggestion: If $D_{A}(t)=(2A+1)^{-1}\sum_{x}^{n}a\,e^{i a t},\ \operatorname{th}$ en $D_{A}(x)=1$ if $\scriptstyle x\;=\;0,$ D.Ax)→0 oterwise and $$ \gamma(A,B)=\int_{\tau2}D_{A}(x)D_{b}(y)\;d\mu(x,y). $$ 16 (b) If If $\underline{{\mu}}$ l were as in Exercise $14,$ Coeoyuoesosooestat o a asuo no.0 ${\boldsymbol{0}}.$ order of (a) Prove that then $n=1$ or $n=2.$ one o ictadis wud Tie oC $\Omega\subset R^{n},$ and suppose that the Suppose ${\boldsymbol{L}}$ is a linrorar s iomeoense $\boldsymbol{\mathit{L}}$ is odd. $n=2,$ rovctathcecefeists t hartic omial $\boldsymbol{\mathit{L}}$ cannot all be real. ln view of $(a),$ he Cach-iemn oeraor o veryicl amle ora elicoperator