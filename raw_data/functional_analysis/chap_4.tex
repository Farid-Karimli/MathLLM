\documentclass[10pt]{article}
\usepackage[utf8]{inputenc}
\usepackage[T1]{fontenc}
\usepackage{amsmath}
\usepackage{amsfonts}
\usepackage{amssymb}
\usepackage[version=4]{mhchem}
\usepackage{stmaryrd}
\usepackage{mathrsfs}

\begin{document}
\section{DUALITY IN BANACH SPACES}
\section{The Normed Dual of a Normed Space}
Introduction If $X$ and $Y$ are topological vector spaces, $\mathscr{B}(X, Y)$ will denote the collection of all bounded linear mappings (or operators) of $X$ into $Y$. For simplicity, $\mathscr{B}(X, X)$ will be abbreviated to $\mathscr{B}(X)$. Each $\mathscr{B}(X, Y)$ is itself a vector space, with respect to the usual definitions of addition and scalar multiplication of functions. (This depends only on the vector space structure of $Y$, not on that of $X$.) In general, there are many ways in which $\mathscr{B}(X, Y)$ can be made into a topological vector space.

In the present chapter, we shall deal only with normed spaces $X$ and $Y$. In that case, $\mathscr{B}(X, Y)$ can itself be normed in a very natural way. When $Y$ is specialized to be the scalar field, so that $\mathscr{B}(X, Y)$ is the dual space $X^{*}$ of $X$, the above-mentioned norm on $\mathscr{B}(X, Y)$ defines a topology on $X^{*}$ which turns out to be stronger than its weak*-topology. The relations between a Banach space $X$ and its normed dual $X^{*}$ form the main topic of this chapter.

4.1 Theorem Suppose $X$ and $Y$ are normed spaces. Associate to each $\Lambda \in \mathscr{B}(X, Y)$ the number

$$
\|\Lambda\|=\sup \{\|\Lambda x\|: x \in X,\|x\| \leq 1\}
$$

This definition of $\|\Lambda\|$ makes $\mathscr{B}(X, Y)$ into a normed space. If $Y$ is a Banach space, so is $\mathscr{B}(X, Y)$.

PROOF Since subsets of normed spaces are bounded if and only if they lie in some multiple of the unit ball, $\|\Lambda\|<\infty$ for every $\Lambda \in \mathscr{B}(X, Y)$. If $\alpha$ is a scalar, then $(\alpha \Lambda)(x)=\alpha \cdot \Lambda x$, so that

$$
\|\alpha \Lambda\|=|\alpha|\|\Lambda\|
$$

The triangle inequality in $Y$ shows that

$$
\begin{aligned}
\left\|\left(\Lambda_{1}+\Lambda_{2}\right) x\right\| & =\left\|\Lambda_{1} x+\Lambda_{2} x\right\| \leq\left\|\Lambda_{1} x\right\|+\left\|\Lambda_{2} x\right\| \\
& \leq\left(\left\|\Lambda_{1}\right\|+\left\|\Lambda_{2}\right\|\right)\|x\| \leq\left\|\Lambda_{1}\right\|+\left\|\Lambda_{2}\right\|
\end{aligned}
$$

for every $x \in X$ with $\|x\| \leq 1$. Hence

$$
\left\|\Lambda_{1}+\Lambda_{2}\right\| \leq\left\|\Lambda_{1}\right\|+\left\|\Lambda_{2}\right\| .
$$

If $\Lambda \neq 0$, then $\Lambda x \neq 0$ for some $x \in X$; hence $\|\Lambda\|>0$. Thus $\mathscr{B}(X, Y)$ is a normed space.

Assume now that $Y$ is complete and that $\left\{\Lambda_{n}\right\}$ is a Cauchy sequence in $\overline{\mathscr{B}}(\bar{X}, \bar{Y})$. Since

$$
\left\|\Lambda_{n} x-\Lambda_{m} x\right\| \leq\left\|\Lambda_{n}-\Lambda_{m}\right\|\|x\|
$$

and since it is assumed that $\left\|\Lambda_{n}-\Lambda_{m}\right\| \rightarrow 0$ as $n$ and $m$ tend to $\infty,\left\{\Lambda_{n} x\right\}$ is a Cauchy sequence in $Y$ for every $x \in X$. Hence

$$
\Lambda x=\lim _{n \rightarrow \infty} \Lambda_{n} x
$$

exists. It is clear that $\Lambda: X \rightarrow Y$ is linear. If $\varepsilon>0$, the right side of (4) does not exceed $\varepsilon\|x\|$, provided that $m$ and $n$ are sufficiently large. It follows that

$$
\left\|\Lambda x-\Lambda_{m} x\right\| \leq \varepsilon\|x\|
$$

for ali large $m$. Hence $\|\Lambda x\| \leq\left(\left\|\Lambda_{m}\right\|+\varepsilon\right)\|x\|$, so that $\Lambda \subseteq \mathscr{B}(X, Y)$, and $\left\|\Lambda-\Lambda_{m}\right\| \leq \varepsilon$. Thus $\Lambda_{m} \rightarrow \Lambda$ in the norm of $\mathscr{B}(X, Y)$. This establishes the completeness of $\mathscr{B}(X, Y)$.

4.2 Duality It will be convenient to designate elements of the dual space $X^{*}$ of $X$ by $x^{*}$ and to write

$$
\left\langle x, x^{*}\right\rangle
$$

in place of $x^{*}(x)$. This notation is well adapted to the symmetry (or duality) that exists between the action of $X^{*}$ on $X$ on the one hand and the action of $X$ on $X^{*}$ on the other. The following theorem states some basic properties of this duality.

4.3 Theorem Suppose $B$ is the closed unit ball of a normed space $X$. Define

for every $x^{*} \in X^{*}$.

$$
\left\|x^{*}\right\|=\sup \left\{\left|\left\langle x, x^{*}\right\rangle\right|: x \in B\right\}
$$

(a) This norm makes $X^{*}$ into a Banach space.

(b) Let $B^{*}$ be the closed unit ball of $X^{*}$. For every $x \in X$,

$$
\|x\|=\sup \left\{\mid\left\langle x, x^{*}\right\rangle: x^{*} \in B^{*}\right\} .
$$

Consequently, $x^{*} \rightarrow\left\langle x, x^{*}\right\rangle$ is a bounded linear functional on $X^{*}$, of norm $\|x\|$.
(c) $B^{*}$ is weak*-compact.

PROOF Since $\mathscr{B}(X, Y)=X^{*}$, when $Y$ is the scalar field, $(a)$ is a corollary of Theorem 4.1. such that

Fix $x \in X$. The corollary to Theorem 3.3 shows that there exists $y^{*} \in B^{*}$

$$
\left\langle x, y^{*}\right\rangle=\|x\| .
$$

On the other hand,

$$
\left|\left\langle x, x^{*}\right\rangle\right| \leq\|x\|\left\|x^{*}\right\| \leq\|x\|
$$

for every $x^{*} \in B^{*}$. Part (b) follows from (1) and (2).

Since the open unit ball $U$ of $X$ is dense in $B$, the definition of $\left\|x^{*}\right\|$ shows that $x^{*} \in B^{*}$ if and only if $\left|\left\langle x, x^{*}\right\rangle\right| \leq 1$ for every $x \in U$. Part (c) now follows directly from Theorem 3.15.

Remark The weak*-topology of $X^{*}$ is, by definition, the weakest one that makes all functionals

$$
x^{*} \rightarrow\left\langle x, x^{*}\right\rangle
$$

continuous. Part $(b)$ shows therefore that the norm topology of $X^{*}$ is stronger than its weak*-topology; in fact, it is strictly stronger, unless $\operatorname{dim} X<\infty$, since the proposition stated at the end of Section 3.11 holds for the weak*-topology as well.

Unless the contrary is explicitly stated, $X^{*}$ will from now on denote the normed dual of $X$ (whenever $X$ is normed), and all topological concepts relating to $X^{*}$ will refer to its norm topology. This implies in no way that the weak* topology will not play an important role.

We now give an alternative description of the operator norm defined in Theorem 4.1.

4.4 Theorem If $X$ and $Y$ are normed spaces and if $\Lambda \in \mathscr{B}(X, Y)$, then

$$
\|\Lambda\|=\sup \left\{\left|\left\langle\Lambda x, y^{*}\right\rangle\right|:\|x\| \leq 1,\left\|y^{*}\right\| \leq 1\right\} .
$$

Proof Apply $(b)$ of Theorem 4.3 with $Y$ in place of $X$. This gives

$$
\|\Lambda x\|=\sup \left\{\left|\left\langle\Lambda x, y^{*}\right\rangle\right|:\left\|y^{*}\right\| \leq 1\right\}
$$

for every $x \in X$. To complete the proof, recall that

$$
\|\Lambda\|=\sup \{\|\Lambda x\|:\|x\| \leq 1\} .
$$

4.5 The second dual of a Banach space The normed dual $X^{*}$ of a Banach space $X$ is itself a Banach space and hence has a normed dual of its own, denoted by $X^{* *}$. Statement $(b)$ of Theorem 4.3 shows that every $x \in X$ defines a unique $\phi x \in X^{* *}$, by the equation

$$
\left\langle x, x^{*}\right\rangle=\left\langle x^{*}, \phi x\right\rangle \quad\left(x^{*} \in X^{*}\right)
$$

and that

$$
\|\phi x\|=\|x\| \quad(x \in X) .
$$

It follows from (1) that $\phi: X \rightarrow X^{* *}$ is linear; by (2), $\phi$ is an isometry. Since $X$ is now assumed to be complete, $\phi(X)$ is closed in $X^{* *}$.

Thus $\phi$ is an isometric isomorphism of $X$ onto a closed subspace of $X^{* *}$.

Frequently, $X$ is identified with $\phi(X)$; then $X$ is regarded as a subspace of $X^{* * *}$.

The members of $\phi(X)$ are exactly those linear functionals on $X^{*}$ that are continuous relative to its weak*-topology. (See Section 3.14.) Since the norm topology of $X^{*}$ is stronger, it may happen that $\phi(X)$ is a proper subspace of $X^{* *}$. But there are many important spaces $X$ (for example, all $L^{p}$-spaces with $1<p<\infty$ ) for which $\phi(X)=X^{* *}$; these are called reflexive. Some of their properties are given in Exercise 1.

It should be stressed that, in order for $X$ to be reflexive, the existence of some isometric isomorphism $\phi$ of $X$ onto $X^{* *}$ is not enough; it is crucial that the identity (1) be satisfied by $\phi$.

4.6 Annihilators Suppose $X$ is a Banach space, $M$ is a subspace of $X$, and $N$ is a subspace of $X^{*}$; neither $M$ nor $N$ is assumed to be closed. Their annihilators $M^{\perp}$ and ${ }^{\perp} N$ are defined as follows:

$$
\begin{aligned}
& M^{\perp}=\left\{x^{*} \in X^{*}:\left\langle x, x^{*}\right\rangle=0 \text { for all } x \in M\right\}, \\
& { }^{\perp} N=\left\{x \in X:\left\langle x, x^{*}\right\rangle=0 \text { for all } x^{*} \in N\right\}
\end{aligned}
$$

Thus $M^{\perp}$ consists of all bounded linear functionals on $X$ that vanish on $M$, and ${ }^{\perp} N$ is the subset of $X$ on which every member of $N$ vanishes. It is clear that $M^{\perp}$ and ${ }^{\perp} N$
are vector spaces. Since $M^{\perp}$ is the intersection of the null spaces of the functionals $\phi x$, where $x$ ranges over $M$ (see Section 4.5), $M^{\perp}$ is a weak*-closed subspace of $X^{*}$. The proof that ${ }^{\perp} N$ is a norm-closed subspace of $X$ is even more direct. The following theorem describes the duality between these two types of annihilators.

\subsection{Theorem Under the preceding hypotheses,}
(a) ${ }^{\perp}\left(M^{\perp}\right)$ is the norm-closure of $M$ in $X$, and

(b) $\left({ }^{\perp} N\right)^{\perp}$ is the weak*-closure of $N$ in $X^{*}$.

As regards $(a)$, recall that the norm-closure of $M$ equals its weak closure, by Theorem 3.12.

PROOF If $x \in M$, then $\left\langle x, x^{*}\right\rangle=0$ for every $x^{*} \in M^{\perp}$, so that $x \in{ }^{\perp}\left(M^{\perp}\right)$. Since ${ }^{\perp}\left(M^{\perp}\right)$ is norm-closed, it contains the norm-closure $\bar{M}$ of $M$. On the other hand, if $x \notin \bar{M}$ the Hahn-Banach theorem yields an $x^{*} \in M^{\perp}$ such that $\left\langle x, x^{*}\right\rangle \neq 0$. Thus $x \notin^{\perp}\left(M^{\perp}\right)$, and $(a)$ is proved.

Similarly, if $x^{*} \in N$, then $\left\langle x, x^{*}\right\rangle=0$ for every $x \in{ }^{\perp} N$, so that $x^{*} \in\left({ }^{\perp} N\right)^{\perp}$.

This weak*-closed subspace of $X^{*}$ contains the weak*-closure $\tilde{N}$ of $N$. If $x^{*} \notin \tilde{N}$, the Hahn-Banach theorem (applied to the locally convex space $X^{*}$ with its weak*-topology) implies the existence of an $x \in{ }^{\perp} N$ such that $\left\langle x, x^{*}\right\rangle \neq 0$; thus $x^{*} \notin\left({ }^{\perp} N\right)^{\perp}$, which proves $(b)$.

Observe, as a corollary, that every norm-closed subspace of $X$ is the annihilator of its annihilator and that the same is true of every weak*-closed subspace of $X^{*}$.

4.8 Duals of subspaces and of quotient spaces If $M$ is a closed subspace of a Banach space $X$, then $X / M$ is also a Banach space, with respect to the quotient norm. This was defined in the proof of $(d)$ of Theorem 1.41. The duals of $M$ and of $X / M$ can be described with the aid of the annihilator $M^{\perp}$ of $M$. Somewhat imprecisely, the result is that

$$
M^{*}=X^{*} / M^{\perp} \quad \text { and } \quad(X / M)^{*}=M^{\perp}
$$

This is imprecise because the equalities should be replaced by isometric isomorphisms. The following theorem describes these explicitly.

4.9 Theorem Let $M$ be a closed subspace of a Banach space $X$.

(a) The Hahn-Bavach theorem extends each $m^{*} \in M^{*}$ to a functional $x^{*} \in X^{*}$. Define $\sigma m^{*}=x^{*}+M^{\perp}$.

Then $\sigma$ is an $i$. metric isomorphism of $M^{*}$ onto $X^{*} / M^{\perp}$.
(b) Let $\pi: X \rightarrow X / M$ be the quotient map. Put $Y=X / M$. For each $y^{*} \in Y^{*}$, define

$$
\tau y^{*}=y^{*} \pi .
$$

Then $\tau$ is an isometric isomorphism of $Y^{*}$ onto $M^{\perp}$.

PROOF (a) If $x^{*}$ and $x_{1}^{*}$ are extensions of $m^{*}$, then $x^{*}-x_{1}^{*}$ is in $M^{\perp}$; hence $x^{*}+M^{\perp}=x_{1}^{*}+M^{\perp}$. Thus $\sigma$ is well defined. A trivial verification shows that $\sigma$ is linear. Since the restriction of every $x^{*} \in X^{*}$ to $M$ is a member of $M^{*}$, the range of $\sigma$ is all of $X^{*} / M^{\perp}$.

Fix $m^{*} \in M^{*}$. If $x^{*} \in X^{*}$ extends $m^{*}$, it is obvious that $\left\|m^{*}\right\| \leq\left\|x^{*}\right\|$. The greatest lower bound of the numbers $\left\|x^{*}\right\|$ so obtained is $\left\|x^{*}+M^{\perp}\right\|$, by the definition of the quotient norm. Hence

$$
\left\|m^{*}\right\| \leq\left\|\sigma m^{*}\right\| \leq\left\|x^{*}\right\|
$$

But Theorem 3.3 furnishes an extension $x^{*}$ of $m^{*}$ with $\left\|x^{*}\right\|=\left\|m^{*}\right\|$. It follows that $\left\|\sigma m^{*}\right\|=\left\|m^{*}\right\|$. This completes $(a)$.

(b) If $x \in X$ and $y^{*} \in Y^{*}$, then $\pi x \in Y$; hence $x \rightarrow y^{*} \pi x$ is a continuous linear functional on $X$ which vanishes for $x \in M$. Thus $\tau y^{*} \in M^{\perp}$. The linearity of $\tau$ is obvious. Fix $x^{*} \in M^{\perp}$. Let $N$ be the null space of $x^{*}$. Since $M \subset N$, there is a linear functional $\Lambda$ on $Y$ such that $\Lambda \pi=x^{*}$. The null space of $\Lambda$ is $\pi(N)$, a closed subspace of $Y$, by the definition of the quotient topology in $Y=X / M$. By Theorem 1.18, $\Lambda$ is continuous, that is, $\Lambda \in Y^{*}$. Hence $\tau \Lambda=\Lambda \pi=x^{*}$. The range of $\tau$ is therefore all of $M^{\perp}$.

Fix $y^{*} \in Y^{*}$. If $y \in Y,\|y\|=1$, and $r>1$, the definition of the quotient norm in $X / M$ shows that there is an $x_{0} \in X$, with $\left\|x_{0}\right\|<r$, such that $\pi x_{0}=y$. Hence

$$
\left|\left\langle y, y^{*}\right\rangle\right|=\left|y^{*} \pi x_{0}\right|=\left|\tau y^{*} x_{0}\right| \leq\left\|\tau y^{*}\right\|\left\|x_{0}\right\| \leq r\left\|\tau y^{*}\right\|
$$

which implies that

$$
\left\|y^{*}\right\| \leq\left\|\tau y^{*}\right\|
$$

On the other hand, $\|\pi x\| \leq\|x\|$ for every $x \in X$. Hence

$$
\left|\tau y^{*} x\right|=\left|y^{*} \pi x\right| \leq\left\|y^{*}\right\|\|\pi x\| \leq\left\|y^{*}\right\|\|x\|
$$

which implies that

This completes the proof.

$$
\left\|\tau y^{*}\right\| \leq\left\|y^{*}\right\|
$$

\section{Adjoints}
We shall now associate with each $T \in \mathscr{B}(X, Y)$ its adjoint, an operator $T^{*} \in \mathscr{B}\left(Y^{*}, X^{*}\right)$, and will see how certain properties of $T$ are reflected in the behavior of $T^{*}$. If $X$ and $Y$ are finite-dimensional, every $T \in \mathscr{B}(X, Y)$ can be represented by a matrix $[T]$; in that
case, $\left[T^{*}\right]$ is the transpose of $[T]$, provided that the various vector space bases are properly chosen. No particular attention will be paid to the finite-dimensional case in what follows, but historically linear algebra did provide the background and much of the motivation that went into the construction of what is now known as operator theory.

Many of the nontrivial properties of adjoints depend on the completeness of $X$ and $Y$ (the open mapping theorem will play an important role). For this reason, it will be assumed throughout that $X$ and $Y$ are Banach spaces, except in Theorem 4.10, which furnishes the definition of $T^{*}$.

4.10 Theorem Suppose $X$ and $Y$ are normed spaces. To each $T \in \mathscr{B}(X, Y)$ corresponds a unique $T^{*} \in \mathscr{B}\left(Y^{*}, X^{*}\right)$ that satisfies

$$
\left\langle T x, y^{*}\right\rangle=\left\langle x, T^{*} y_{.}^{*}\right\rangle
$$

for all $x \in X$ and all $y^{*} \in Y^{*}$. Moreover, $T^{*}$ satisfies

(2)

$$
\left\|T^{*}\right\|=\|T\| .
$$

PROOF If $y^{*} \in Y^{*}$ and $T \in \mathscr{B}(X, Y)$, define

$$
T^{*} y^{*}=y^{*} \circ T \text {. }
$$

Being the composition of two continuous linear mappings, $T^{*} y^{*} \in X^{*}$. Also,

$$
\left\langle x, T^{*} y^{*}\right\rangle=\left(T^{*} y^{*}\right)(x)=y^{*}(T x)=\left\langle T x, y^{*}\right\rangle
$$

which is (1). The fact that (1) holds for every $x \in X$ obviously determines $T^{*} y^{*}$ uniquely.

$$
\begin{aligned}
& \text { If } y_{1}^{*} \in Y^{*} \text { and } y_{2}^{*} \in Y^{*} \text {, then } \\
& \qquad \begin{aligned}
\left\langle x, T^{*}\left(y_{1}^{*}+y_{2}^{*}\right)\right\rangle & =\left\langle T x, y_{1}^{*}+y_{2}^{*}\right\rangle \\
& =\left\langle T x, y_{1}^{*}\right\rangle+\left\langle T x, y_{2}^{*}\right\rangle \\
& =\left\langle x, T^{*} y_{1}^{*}\right\rangle+\left\langle x, T^{*} y_{2}^{*}\right\rangle \\
& =\left\langle x, T^{*} y_{1}^{*}+T^{*} y_{2}^{*}\right\rangle
\end{aligned}
\end{aligned}
$$

for every $x \in X$, so that

$$
T^{*}\left(y_{1}^{*}+y_{2}^{*}\right)=T^{*} y_{1}^{*}+T^{*} y_{2}^{*} .
$$

Similarly, $T^{*}\left(\alpha y^{*}\right)=\alpha T^{*} y^{*}$. Thus $T^{*}: Y^{*} \rightarrow X^{*}$ is linear. Finally, $(b)$ of Theorem 4.3 leads to

$$
\begin{aligned}
\|T\| & =\sup \left\{\left|\left\langle T x, y^{*}\right\rangle\right|:\|x\| \leq 1,\left\|y^{*}\right\| \leq 1\right\} \\
& =\sup \left\{\left|\left\langle x, T^{*} y^{*}\right\rangle\right|:\|x\| \leq 1,\left\|y^{*}\right\| \leq 1\right\} \\
& =\sup \left\{\left\|T^{*} y^{*}\right\|:\left\|y^{*}\right\| \leq 1\right\}=\left\|T^{*}\right\| .
\end{aligned}
$$

4.11 Notation If $T$ maps $X$ into $Y$, the null space and the range of $T$ will be denoted by $\mathcal{N}(T)$ and $\mathscr{R}(T)$, respectively:

$$
\begin{aligned}
\mathscr{N}(T) & =\{x \in X: T x=0\} \\
\mathscr{R}(T) & =\{y \in Y: T x=y \text { for some } x \in X\}
\end{aligned}
$$

The next theorem concerns annihilators; see Section 4.6 for the notation.

4.12 Theorem Suppose $X$ and $Y$ are Banach spaces, and $T \in \mathscr{B}(X, Y)$. Then

$$
\mathscr{N}\left(T^{*}\right)=\mathscr{R}(T)^{\perp} \quad \text { and } \quad \mathscr{N}(T)={ }^{\perp} \mathscr{R}\left(T^{*}\right)
$$

PROOF In each of the following two columns, each statement is obviously equivalent to the one that immediately follows and/or precedes it.

$$
\begin{array}{ll}
y^{*} \in \mathscr{N}\left(T^{*}\right) . & x \in \mathscr{N}(T) . \\
T^{*} y^{*}=0 . & T x=0 . \\
\left\langle x, T^{*} y^{*}\right\rangle=0 \text { for all } x . & \left\langle T x, y^{*}\right\rangle=0 \text { for all } y^{*} . \\
\left\langle T x, y^{*}\right\rangle=0 \text { for all } x . & \left\langle x, T^{*} y^{*}\right\rangle=0 \text { for all } y^{*} . \\
y^{*} \in \mathscr{R}(T)^{\perp} . & x \in{ }^{\perp} \mathscr{R}\left(T^{*}\right) .
\end{array}
$$

\section{Corollaries}
(a) $\mathcal{N}\left(T^{*}\right)$ is weak ${ }^{*}$-closed in $Y^{*}$.

(b) $\mathscr{R}(T)$ is dense in $Y$ if and only if $T^{*}$ is one-to-one.

(c) $T$ is one-to-one if and only if $\mathscr{R}\left(T^{*}\right)$ is weak*-dense in $X^{*}$.

Recall that $M^{\perp}$ is weak*-closed in $Y^{*}$ for every subspace $M$ of $Y$. In particular, this is true of $\mathscr{R}(T)^{\perp}$. Thus $(a)$ follows from the theorem.

As to $(b), \mathscr{R}(T)$ is dense in $Y$ if and only if $\mathscr{R}(T)^{\perp}=\{0\}$; in that case, $\mathscr{N}\left(T^{*}\right)=\{0\}$.

Likewise, ${ }^{\perp} \mathscr{R}\left(T^{*}\right)=\{0\}$ if and only if $\mathscr{R}\left(T^{*}\right)$ is annihilated by no $x \in X$ other than $x=0$; this says that $\mathscr{R}\left(T^{*}\right)$ is weak*-dense in $X^{*}$.

Note that the Hahn-Banach theorem 3.5 was tacitly used in the proofs of $(b)$ and $(c)$.

There is a very useful analogue of $(b)$ that allows us to decide, in terms of $T^{*}$, whether $\mathscr{R}(T)=Y$, that is, whether $T$ maps $X$ onto $Y$. This is given in Theorem 4.15 . and will be obtained by first looking for conditions on $T^{*}$ which imply that $T$ has closed range (Theorem 4.14).

4.13 Lemma Suppose $U$ and $V$ are the open unit balls in the Banach spaces $X$ and $Y$, respectively. Suppose $T \in \mathscr{B}(X, Y)$, and $c>0$.
(a) If the closure of $T(U)$ contains $c V$, then $T(U) \supset c V$.

(b) If $c\left\|y^{*}\right\| \leq\left\|T^{*} y^{*}\right\|$ for every $y^{*} \in Y^{*}$, then $T(U) \supset c V$.

PROOF (a) Take $c=1$, without loss of generality. Then $\overline{T(U)} \supset \bar{V}$. To every $y \in Y$ and every $\varepsilon>0$ corresponds therefore an $x \in X$ with $\|x\| \leq\|y\|$ and $\|y-T x\|<\varepsilon$.

Pick $y_{1} \in V$. Pick $\varepsilon_{n}>0$ so that

$$
\sum_{n=1}^{\infty} \varepsilon_{n}<1-\left\|y_{1}\right\|
$$

Assume $n \geq 1$ and $y_{n}$ is picked. There exists $x_{n}$ such that $\left\|x_{n}\right\| \leq\left\|y_{n}\right\|$ and
$\left\|y_{n}-T x_{n}\right\|<\varepsilon_{n}$. Put

$$
y_{n+1}=y_{n}-T x_{n}
$$

By induction, this process defines two sequences $\left\{x_{n}\right\}$ and $\left\{y_{n}\right\}$. Note that

Hence

$$
\left\|x_{n+1}\right\| \leq\left\|y_{n+1}\right\|=\left\|y_{n}-T x_{n}\right\|<\varepsilon_{n} .
$$

$$
\sum_{n=1}^{\infty}\left\|x_{n}\right\| \leq\left\|x_{1}\right\|+\sum_{n=1}^{\infty} \varepsilon_{n} \leq\left\|y_{1}\right\|+\sum_{n=1}^{\infty} \varepsilon_{n}<1 .
$$

It follows that $x=\sum x_{n}$ is in $U$ (see Exercise 23) and that

$$
T x=\lim _{N \rightarrow \infty} \sum_{n=1}^{N} T x_{n}=\lim _{N \rightarrow \infty} \sum_{n=1}^{N}\left(y_{n}-y_{n+1}\right)=y_{1}
$$

since $y_{N+1} \rightarrow 0$ as $N \rightarrow \infty$. Thus $y_{1}=T x \in T(U)$, which proves $(a)$.

Note that the preceding argument is just a specialized version of part of the proof of the open mapping theorem 2.11.

(b) Let $E$ be the closure of $T(U)$, pick $y_{0} \in Y, y_{0} \notin E$. Since $E$ is closed, convex, and balanced, Theorem 3.7 yields a $y^{*} \in Y^{*}$ such that

$$
\left|\left\langle y, y^{*}\right\rangle\right| \leq 1<\left|\left\langle y_{0}, y^{*}\right\rangle\right|
$$

for all $y \in E$. If $x \in U$ then $T x \in E$, so that

It follows that

$$
\left|\left\langle x, T^{*} y^{*}\right\rangle\right|=\left|\left\langle T x, y^{*}\right\rangle\right| \leq 1
$$

$$
c\left\|y^{*}\right\| \leq\left\|T^{*} y^{*}\right\| \leq 1
$$

and therefore

$$
1<\left|\left\langle y_{0}, y^{*}\right\rangle\right| \leq\left\|y_{0}\right\|\left\|y^{*}\right\| \leq c^{-1}\left\|y_{0}\right\|,
$$

or $\left\|y_{0}\right\|>c$. Thus $c V \subset E$, and $(b)$ follows from $(a)$.

4.14 Theorem If $X$ and $Y$ are Banach spaces and if $T \in \mathscr{B}(X, Y)$, then each of the following three conditions implies the other two:

(a) $\mathscr{R}(T)$ is closed in $Y$.

(b) $\mathscr{R}\left(T^{*}\right)$ is weak ${ }^{*}$-closed in $X^{*}$.

(c) $\mathscr{R}\left(T^{*}\right)$ is norm-closed in $X^{*}$.

Remark Theorem 3.12 implies that $(a)$ holds if and only if $\mathscr{R}(T)$ is weakly closed. However, norm-closed subspaces of $X^{*}$ are not always weak*-closed (Exercise 7, Chapter 3).

Proof It is obvious that $(b)$ implies $(c)$. We will prove that $(a)$ implies $(b)$ and that $(c)$ implies $(a)$.

Suppose (a) holds. By Theorem 4.12 and $(b)$ of Theorem 4.7, $\mathscr{N}(T)^{\perp}$ is the weak*-closure of $\mathscr{R}\left(T^{*}\right)$. To prove $(b)$ it is therefore enough to show that $\mathscr{N}(T)^{\perp} \subset \mathscr{R}\left(T^{*}\right)$.

Pick $x^{*} \in \mathscr{N}(T)^{\perp}$. Define a linear functional $\Lambda$ on $\mathscr{R}(T)$ by

$$
\Lambda T x=\left\langle x, x^{*}\right\rangle \quad(x \in X)
$$

Note that $\Lambda$ is well defined, for if $T x=T x^{\prime}$, then $x-x^{\prime} \in \mathscr{N}(T)$; hence

$$
\left\langle x-x^{\prime}, x^{*}\right\rangle=0 .
$$

The open mapping theorem applies to

$$
T: X \rightarrow \mathscr{R}(T)
$$

since $\mathscr{R}(T)$ is assumed to be a closed subspace of the complete space $Y$ and is therefore complete. It follows that there exists $K<\infty$ such that to each $y \in \mathscr{R}(T)$ corresponds an $x \in X$ with $T x=y,\|x\| \leq K\|y\|$, and

$$
|\hat{\Lambda} y|=|\Lambda T x|=\left|\left\langle x, x^{*}\right\rangle\right| \leq \tilde{K}\|y\|\left\|x^{*}\right\| .
$$

Thus $\Lambda$ is continuous. By the Hahn-Banach theorem, some $y^{*} \in Y^{*}$ extends $\Lambda$. Hence.

$$
\left\langle T x, y^{*}\right\rangle=\Lambda T x=\left\langle x, x^{*}\right\rangle \quad(x \in X)
$$

This implies $x^{*}=T^{*} y^{*}$. Since $x^{*}$ was an arbitrary element of $\mathscr{N}(T)^{\perp}$, we have shown that $\mathscr{N}(T)^{\perp} \subset \mathscr{R}\left(T^{*}\right)$. Thus (b) follows from $(a)$.

Suppose next that $(c)$ holds. Let $Z$ be the closure of $\mathscr{R}(T)$ in $Y$. Define $S \in \mathscr{B}(X, Z)$ by setting $S x=T x$. Since $\mathscr{R}(S)$ is dense in $Z$, Corollary $(b)$ to Theorem 4.12 implies that

$$
S^{*}: Z^{*} \rightarrow X^{*}
$$

is one-to-one.
every $x \in X$,

If $z^{*} \in Z^{*}$, the Hahn-Banach theorem furnishes an extension $y^{*}$ of $z^{*}$; for

$$
\left\langle x, T^{*} y^{*}\right\rangle=\left\langle T x, y^{*}\right\rangle=\left\langle S x, z^{*}\right\rangle=\left\langle x, S^{*} z^{*}\right\rangle .
$$

Hence $S^{*} z^{*}=T^{*} y^{*}$. It follows that $S^{*}$ and $T^{*}$ have identical ranges. Since $(c)$ is assumed to hold, $\mathscr{R}\left(S^{*}\right)$ is closed, hence complete.

Apply the open mapping theorem to

$$
S^{*}: Z^{*} \rightarrow \mathscr{R}\left(S^{*}\right)
$$

Since $S^{*}$ is one-to-one, the conclusion is that there is a constant $c>0$ which satisfies

$$
c\left\|z^{*}\right\| \leq\left\|S^{*} z^{*}\right\|
$$

for every $z^{*} \in Z^{*}$. Hence $S: X \rightarrow Z$ is an open mapping, by (b) of Lemma 4.13. In particular, $S(X)=Z$. But $\mathscr{R}(T)=\mathscr{R}(S)$, by the definition of $S$. Thus $\mathscr{R}(T)=Z$, a closed subspace of $Y$.

This completes the proof that $(c)$ implies $(a)$.

The following consequence is useful in applications.

4.15 Theorem Suppose $X$ and $Y$ are Banach spaces, and $T \in \mathscr{B}(X, Y)$. Then

(a) $\mathscr{R}(T)=Y$

if and only if

(b) $\left\|T^{*} y^{*}\right\| \geq c\left\|y^{*}\right\|$ for some constant $c>0$ and for every $y^{*} \in Y^{*}$.

PROOF Statement $(a)$ holds if and only if $\mathscr{R}(T)$ is dense and closed. By Theorem 4.14 and Corollary $(b)$ of Theorem 4.12, $(a)$ is therefore equivalent to

(c) $\quad T^{*}$ is one-to-one and $\mathscr{R}\left(T^{*}\right)$ is norm-closed in $X^{*}$.

If $(c)$ holds, the open mapping theorem, applied to $T^{*}: Y^{*} \rightarrow \mathscr{R}\left(T^{*}\right)$, gives (b). Conversely, $(b)$ obviously implies that $T^{*}$ is one-to-one; also, the inverse image (under $T^{*}$ ) of any Cauchy sequence is a Cauchy sequence, so that $\mathscr{R}\left(T^{*}\right)$ is complete and hence closed.

\section{Compact Operators}
4.16 Definition Suppose $X$ and $Y$ are Banach spaces, and $U$ is the open unit ball in $X$. An operator $T \in \mathscr{B}(X, Y)$ is said to be compact if the closure of $T(U)$ is compact in $Y$.

Since $Y$ is a complete metric space, the subsets of $Y$ whose closure is compact are precisely the totally bounded ones. Thus $T \in \mathscr{B}(X, Y)$ is compact if and only if $T(U)$
is totally bounded. Also, $T$ is compact if and only if every bounded sequence $\left\{x_{n}\right\}$ in $X$ contains a subsequence $\left\{x_{n_{i}}\right\}$ such that $\left\{T x_{n_{i}}\right\}$ converges to a point of $Y$.

Many of the operators that arise in the study of integral equations are compact. This accounts for their importance from the standpoint of applications. They are in some respects as similar to linear operators on finite-dimensional spaces as one has any right to expect from operators on infinite-dimensional spaces. As we shall see, these similarities show up particularly strongly in their spectral properties.

4.17 Definitions (a) Suppose $X$ is a Banach space. Then $\mathscr{B}(X)$ [which is an abbreviation for $\mathscr{B}(X, X)$ ] is not merely a Banach space (see Theorem 4.1) but also an algebra: If $S \in \mathscr{B}(X)$ and $T \in \mathscr{B}(X)$, one defines $S T \in \mathscr{B}(X)$ by

The inequality

$$
(S T)(x)=S(T(x)) \quad(x \in X)
$$

$$
\|S T\| \leq\|S\|\|T\|
$$

is trivial to verify.

In particular, powers of $T \in \mathscr{B}(X)$ can be defined: $T^{0}=I$, the identity mapping on $X$, given by $I x=x$, and $T^{n}=T T^{n-1}$, for $n=1,2,3, \ldots$

(b) An operator $T \in \mathscr{B}(X)$ is said to be invertible if there exists $S \in \mathscr{B}(X)$ such that

$$
S T=I=T S
$$

In this case, we write $S=T^{-1}$. By the open mapping theorem, this happens if and only if $\mathscr{N}(T)=\{0\}$ and $\mathscr{R}(T)=X$.

(c) The spectrum $\sigma(T)$ of an operator $T \in \mathscr{B}(X)$ is the set of all scalars $\lambda$ such that $T-\lambda I$ is not invertible. Thus $\lambda \in \sigma(T)$ if and only if at least one of the following two statements is true:

(i) The range of $T-\lambda I$ is not all of $X$.

(ii) $T-\lambda I$ is not one-to-one.

If (ii) holds, $\lambda$ is said to be an eigenvalue of $T$; the corresponding eigenspace is $\mathscr{N}(T-\lambda I)$; each $x \in \mathscr{N}(T-\lambda I)$ (except $x=0$ ) is an eigenvector of $T$; it satisfies the equation

$$
T x=\lambda x
$$

Here are some very easy facts which will illustrate these concepts.

4.18 Theorem Let $X$ and $Y$ be Banach spaces.

(a) If $T \in \mathscr{B}(X, Y)$ and $\operatorname{dim} \mathscr{R}(T)<\infty$, then $T$ is compact.

(b) If $T \in \mathscr{B}(X, Y), T$ is compact, and $\mathscr{R}(T)$ is closed, then $\operatorname{dim} \mathscr{R}(T)<\infty$.

(c) The compact operators form a closed subspace of $\mathscr{B}(X, Y)$, in its norm-topology.
(d) If $T \in \mathscr{B}(X), T$ is compact, and $\lambda \neq 0$, then $\operatorname{dim} \mathscr{N}(T-\lambda I)<\infty$.

(e) If $\operatorname{dim} X=\infty, T \in \mathscr{B}(X)$, and $T$ is compact, then $0 \in \sigma(T)$.

(f) If $S \in \mathscr{B}(X), T \in \mathscr{B}(X)$, and $T$ is compact, so are $S T$ and $T S$.

PROOF Statement $(a)$ is obvious. If $\mathscr{R}(T)$ is closed, then $\mathscr{K}(T)$ is complete (since $Y$ is complete), so that $T$ is an open mapping of $X$ onto $\mathscr{R}(T)$; if $T$ is compact, it follows that $\mathscr{R}(T)$ is locally compact; thus $(b)$ is a consequence of Theorem 1.22.

Put $Y=\mathscr{N}(T-\lambda I)$ in $(d)$. The restriction of $T$ to $Y$ is a compact operator whose range is $Y$. Thus $(d)$ follows from $(b)$, and so does $(e)$, for if 0 is not in $\sigma(T)$, then $\mathscr{R}(T)=X$. The proof of $(f)$ is trivial.

If $S$ and $T$ are compact operators from $X$ into $Y$, so is $S+T$, because the sum of any two compact subsets of $Y$ is compact. It follows that the compact operators form a subspace $\Sigma$ of $\mathscr{B}(X, Y)$. To complete the proof of $(c)$, we now show that $\Sigma$ is closed. Let $T \in \mathscr{B}(X, Y)$ be in the closure of $\Sigma$, choose $r>0$, and let $U$ be the open unit ball in $X$. There exists $S \in \Sigma$ with $\|S-T\|<r$. Since $S(U)$ is totally bounded, there are points $x_{1}, \ldots, x_{n}$ in $U$ such that $S(U)$ is covered by the balls of radius $r$ with centers at the points $S x_{i}$. Since $\|S x-T x\|<r$ for every $x \in U$, it follows that $T(U)$ is covered by the balls of radius $3 r$ with centers at the points $T x_{i}$. Thus $T(U)$ is totally bounded, which proves that $T \in \Sigma$.

The main objective of the rest of this chapter is to analyze the spectrum of a compact $T \in \mathscr{B}(X)$. Theorem 4.25 contains the principal results. Adjoints will play an important role in this investigation.

4.19 Theorem Suppose $X$ and $Y$ are Banach spaces and $T \in \mathscr{B}(X, Y)$. Then $T$ is compact if and only if $T^{*}$ is compaci. PRoOF Suppose $T$ is compact. Let $\left\{y_{n}^{*}\right\}$ be a sequence in the unit ball of $Y^{*}$.
Define

$$
f_{n}(y)=\left\langle y, y_{n}^{*}\right\rangle \quad(y \in Y)
$$

Since $\left|f_{n}(y)-f_{n}\left(y^{\prime}\right)\right| \leq\left\|y-y^{\prime}\right\|,\left\{f_{n}\right\}$ is equicontinuous. Since $T(U)$ has compact closure in $Y$ (as before, $U$ is the unit ball of $X$ ), Ascoli's theorem implies that $\left\{f_{n}\right\}$ has a subsequence $\left\{f_{n_{i}}\right\}$ that converges uniformly on $T(U)$. Since

$$
\begin{aligned}
\left\|T^{*} y_{n_{i}}^{*}-T^{*} y_{n_{j}}^{*}\right\| & =\sup \left|\left\langle T x, y_{n_{i}}^{*}-y_{n_{j}}^{*}\right\rangle\right| \\
& =\sup \left|f_{n_{i}}(T x)-f_{n_{j}}(T x)\right|
\end{aligned}
$$

the supremum being taken over $x \in U$, the completeness of $X^{*}$ implies that $\left\{T^{*} y_{n_{i}}^{*}\right\}$ converges. Hence $T^{*}$ is compact.

The second half can be proved by the same method, but it may be more instructive to deduce it from the first half.

Let $\phi: X \rightarrow X^{* *}$ and $\psi: Y \rightarrow Y^{* *}$ be the isometric embeddings given by the formulas

$$
\left\langle x, x^{*}\right\rangle=\left\langle x^{*}, \phi x\right\rangle \quad \text { and } \quad\left\langle y, y^{*}\right\rangle=\left\langle y^{*}, \psi y\right\rangle \text {, }
$$

as in Section 4.5. Then

$$
\left\langle y^{*}, \psi T x\right\rangle=\left\langle T x, y^{*}\right\rangle=\left\langle x, T^{*} y^{*}\right\rangle=\left\langle T^{*} y^{*}, \phi x\right\rangle=\left\langle y^{*}, T^{* *} \phi x\right\rangle
$$

for all $x \in X$ and $y^{*} \in Y^{*}$, so that

$$
\psi T=T^{* *} \phi
$$

If $x \in U$, then $\phi x$ lies in the unit ball $U^{* *}$ of $X^{* *}$. Thus

$$
\psi T(U) \subset T^{* *}\left(U^{* *}\right) .
$$

Now assume that $T^{*}$ is compact. The first half of the theorem shows that $T^{* *}: X^{* *} \rightarrow Y^{* *}$ is compact. Hence $T^{* *}\left(U^{* *}\right)$ is totally bounded, and so is its subset $\psi T(U)$. Since $\psi$ is an isometry, $T(U)$ is also totally bounded. Hence $T$ is compact.

4.20 Definition Suppose $M$ is a closed subspace of a topological vector space $X$. If there exists a closed subspace $N$ of $X$ such that

$$
X=M+N \quad \text { and } \quad M \cap N=\{0\}
$$

then $M$ is said to be complemented in $X$. In this case, $X$ is said to be the direct sum of $M$ and $N$, and the notation

$$
X=M \oplus N
$$

is sometimes used.

We shall see examples of uncomplemented subspaces in Chapter 5. At present we need only the following simple facts.

4.21 Lemma Let $M$ be a closed subspace of a topological vector space $X$.

(a) If $X$ is locally convex and $\operatorname{dim} M<\infty$, then $M$ is complemented in $X$.

(b) If $\operatorname{dim}(X / M)<\infty$, then $M$ is complemented in $X$.

The dimension of $X / M$ is also called the codimension of $M$ in $X$.

PROOF (a) Let $\left\{e_{1}, \ldots, e_{n}\right\}$ be a basis for $M$. Every $x \in M$ has then a unique representation

$$
x=\alpha_{1}(x) e_{1}+\cdots+\alpha_{n}(x) e_{n} .
$$

Each $\alpha_{i}$ is a continuous linear functional on $M$ (Theorem 1.21) which extends to a member of $X^{*}$, by the Hahn-Banach theorem. Let $N$ be the intersection of the null spaces of these extensions. Then $X=M \oplus N$.

(b) Let $\pi: X \rightarrow X / M$ be the quotient map, let $\left\{e_{1}, \ldots, e_{n}\right\}$ be a basis for $\bar{X} / M$, pick $x_{i} \in X$ so that $\pi x_{i}=e_{i}(1 \leq i \leq n)$, and let $N$ be the vector space spanned by $\left\{x_{1}, \ldots, x_{n}\right\}$. Then $X=M \oplus N$.

4.22 Lemma If $M$ is a subspace of a normed space $X$, if $M$ is not dense in $X$, and if $r>1$, then there exists $x \in X$ such that

$$
\|x\|<r \quad \text { but } \quad\|x-y\| \geq 1 \quad \text { for all } y \in M \text {. }
$$

PROOF There exists $x_{1} \in X$ whose distance from $M$ is 1 , that is,

$$
\inf \left\{\left\|x_{1}-y\right\|: y \in M\right\}=1
$$

Choose $y_{1} \in M$ such that $\left\|x_{1}-y_{1}\right\|<r$, and put $x=x_{1}-y_{1}$.

4.23 Theorem If $X$ is a Banach space, $T \in \mathscr{R}(X), T$ is compact, and $\lambda \neq 0$, then $T-\lambda I$ has closed range.

PROOF By $(d)$ of Theorem 4.18, $\operatorname{dim} \mathscr{N}(T-\lambda I)<\infty$. By $(a)$ of Lemma 4.21, $X$ is the direct sum of $\mathscr{N}(T-\lambda I)$ and a closed subspace $M$. Define an operator $S \in \mathscr{B}(M, X)$ by

$$
S x=T x-\lambda x .
$$

Then $S$ is one-to-one on $M$. Also, $\mathscr{R}(S)=\mathscr{R}(T-\lambda I)$. To show that $\mathscr{R}(S)$ is closed, it suffices to show the existence of an $r>0$ such that

$$
r\|x\| \leq\|S x\| \quad \text { for all } x \in M \text {. }
$$

For if (2) holds, and if $\left\{S x_{n}\right\}$ is a Cauchy sequence, so is $\left\{x_{n}\right\}$; the completeness of, $\mathscr{R}(S)$ is a consequence.

If (2) fails for every $r>0$, there exists $\left\{x_{n}\right\}$ in $M$ such that $\left\|x_{n}\right\|=1, S x_{n} \rightarrow 0$, and (after passage to a subsequence) $T x_{n} \rightarrow x_{0}$ for some $x_{0} \in X$. (This is wherc compactness of $T$ is used.) It follows that $\lambda x_{n} \rightarrow x_{0}$. Thus $x_{0} \in M$, and

$$
S x_{0}=\lim \left(\lambda S x_{n}\right)=0
$$

Since $S$ is one-to-one, $x_{0}=0$. But $\left\|x_{n}\right\|=1$ for all $n$, and $x_{0}=\lim \lambda x_{n}$, and so $\left\|x_{0}\right\|=|\lambda|>0$. This contradiction proves (2) for some $r>0$.

4.24 Theorem Suppose $X$ is a Banach space, $T \in \mathscr{B}(X), T$ is compact, $r>0$, and $E$ is a set of eigenvalues $\lambda$ of $T$ such that $|\lambda|>r$. Then
(a) for each $\lambda \in E, \mathscr{R}(T-\lambda I) \neq X$, and

(b) $E$ is a finite set.

PROOF We shall first show that if either $(a)$ or $(b)$ is false then there cxist closed subspaces $M_{n}$ of $X$ and scalars $\lambda_{n} \in E$ such that

$$
\begin{gathered}
M_{1} \subset M_{2} \subset M_{3} \subset \cdots, \quad M_{n} \neq M_{n+1}, \\
T\left(M_{n}\right) \subset M_{n} \quad \text { for } n \geq 1,
\end{gathered}
$$

and

$$
\left(T-\lambda_{n} I\right)\left(M_{n}\right) \subset M_{n-1} \quad \text { for } n \geq 2
$$

The proof will be completed by showing that this contradicts the compactness of $T$.

Suppose $(a)$ is false. Then $\mathscr{R}\left(T-\lambda_{0} I\right)=X$ for some $\lambda_{0} \in E$. Put $S=$ $T-\lambda_{0} I$, and define $M_{n}$ to be the null space of $S^{n}$. (See Section 4.17.) Since $\lambda_{0}$ is an eigenvalue of $T$, there exists $x_{1} \in M_{1}, x_{1} \neq 0$. Since $\mathscr{R}(S)=X$, there is a sequence $\left\{x_{n}\right\}$ in $X$ such that $S x_{n+1}=x_{n}, n=1,2,3, \ldots$ Then

$$
S^{n} x_{n+1}=x_{1} \neq 0 \quad \text { but } \quad S^{n+1} x_{n+1}=S x_{1}=0
$$

Hence $M_{n}$ is a proper closed subspace of $M_{n+1}$. It follows that (1) to (3) hold, with $\lambda_{n}=\lambda_{0}$. [Note that (2) holds because $S T=T S$.]

Suppose $(b)$ is false. Then $E$ contains a sequence $\left\{\lambda_{n}\right\}$ of distinct eigenvalues of $T$. Choose corresponding eigenvectors $e_{n}$, and let $M_{n}$ be the (finite-dimensional, hence closed) subspace of $X$ spanned by $\left\{e_{1}, \ldots, e_{n}\right\}$. Since the $\lambda_{n}$ are distinct, $\left\{e_{1}, \ldots, e_{n}\right\}$ is a linearly independent set, so that $M_{n-1}$ is a proper subspace of $M_{n}$. This gives (1). If $x \in M_{n}$, then

$$
x=\alpha_{1} e_{1}+\cdots+\alpha_{n} e_{n}
$$

which shows that $T x \in \bar{M}_{n}$ and

$$
\left(T-\lambda_{n} I\right) x=\alpha_{1}\left(\lambda_{1}-\lambda_{n}\right) e_{1}+\cdots+\alpha_{n-1}\left(\lambda_{i n-1}-\lambda_{n}\right) e_{n-1} \in M_{n-1}
$$

Thus (2) and (3) hold.

Once we have closed subspaces $M_{n}$ satisfying (1) to (3), Lemma 4.22 gives us vectors $y_{n} \in M_{n}$, for $n=2,3,4, \ldots$, such that

$$
\left\|y_{n}\right\| \leq 2 \quad \text { and } \quad\left\|y_{n}-x\right\| \geq 1 \quad \text { if } \quad x \in M_{n-1}
$$

If $2 \leq m<n$, define

$$
z=T y_{m}-\left(T-\lambda_{n} I\right) y_{n}
$$

By (2) and (3), $z \in M_{n-1}$. Hence (5) shows that

$$
\left\|T y_{\dot{n}}-T y_{m}\right\|=\left\|\lambda_{n} y_{n}-z\right\|=\left|\lambda_{n}\right|\left\|y_{n}-\lambda_{n}^{-1} z\right\| \geq\left|\lambda_{n}\right|>r .
$$

The sequence $\left\{T y_{n}\right\}$ has therefore no convergent subsequences, although $\left\{y_{n}\right\}$ is bounded. This is impossible if $T$ is compact.

4.25 Theorem Suppose $X$ is a Banach space, $T \in \mathscr{B}(X)$, and $T$ is compact.

(a) If $\lambda \neq 0$, then the four numbers

$$
\begin{aligned}
\alpha & =\operatorname{dim} \mathscr{N}(T-\lambda I) \\
\beta & =\operatorname{dim} X / \mathscr{R}(T-\lambda I) \\
\alpha^{*} & =\operatorname{dim} \mathscr{N}\left(T^{*}-\lambda I\right) \\
\beta^{*} & =\operatorname{dim} X^{*} / \mathscr{R}\left(T^{*}-\lambda I\right)
\end{aligned}
$$

are equal and finite.

(b) If $\lambda \neq 0$ and $\lambda \in \sigma(T)$ then $\lambda$ is an eigenvalue of $T$ and of $T^{*}$.

(c) $\sigma(T)$ is compact, at most countable, and has at most one limit point, namely, 0.

Note: The dimension of a vector space is here understood to be either a nonnegative integer or the symbol $\infty$. The letter $I$ is used for the identity operators on both $X$ and $X^{*}$; thus

$$
(T-\lambda I)^{*}=T^{*}-\lambda I^{*}=T^{*}-\lambda I,
$$

since the adjoint of the identity on $X$ is the identity on $X^{*}$.

The spectrum $\sigma(T)$ of $T$ was defined in Section 4.17. Theorem 4.24 contains a special case of $(a): \beta=0$ implies $\alpha=0$. This will be used in the proof of the inequality (4) below.

It should be noted that $\sigma(T)$ is compact even if $T$ is not (Theorem 10.13). The compactness of $T$ is needed for the other assertions in $(c)$.

PROOF Put $S=T-\lambda I$, to simplify the writing.

We begin with an elementary observation about quotient spaces. Suppose $M_{0}$ is a closed subspace of a locally convex space $Y$, and $k$ is a positive integer such that $k \leq \operatorname{dim} Y / M_{0}$. Then there are vectors $y_{1}, \ldots, y_{k}$ in $Y$ such that the vector space $M_{i}$ generated by $M_{0}$ and $y_{1}, \ldots, y_{i}$ contains $M_{i-1}$ as a proper subspace. By Theorem 1.42, each $M_{i}$ is closed. By Theorem 3.5, there are continuous linear functionals $\Lambda_{1}, \ldots, \Lambda_{k}$ on $Y$ such that $\Lambda_{i} y_{i}=1$ but $\Lambda_{i} y=0$ for all $y \in M_{i-1}$. These functionals are linearly independent. The following conclusion is therefore reached: - if $\Sigma$ denotes the space of all continuous linear functionals on $Y$ that annihilate $M_{0}$, then

$$
\operatorname{dim} Y / M_{0} \leq \operatorname{dim} \Sigma
$$

Apply this with $Y=X, M_{0}=\mathscr{R}(S)$. By Theorem $4.23, \mathscr{R}(S)$ is closed. Also, $\Sigma=\mathscr{R}(S)^{\perp}=\mathscr{N}\left(S^{*}\right)$, by Theorem 4.12, so that (1) becomes

$$
\beta \leq \alpha^{*} \text {. }
$$

Next, take $Y=X^{*}$ with its weak*-topology; take $M_{0}=\mathscr{R}\left(S^{*}\right)$. By Theorem 4.14, $\mathscr{R}\left(S^{*}\right)$ is weak*-closed. Since $\Sigma$ now consists of all weak*continuous linear functionals on $X^{*}$ that annihilate $\mathscr{R}\left(S^{*}\right), \Sigma$ is isomorphic to ${ }^{\perp} \mathscr{R}\left(S^{*}\right)=\mathscr{N}(S)$ (Theorem 4.12), and (1) becomes

$$
\beta^{*} \leq \alpha
$$

Our next objective is to prove that

$$
\alpha \leq \beta .
$$

Once we have (4), the inequality

$$
\alpha^{*} \leq \beta^{*}
$$

is also true, since $T^{*}$ is a compact operator (Theorem 4.19). Since $\alpha<\infty$ by $(d)$ of Theorem 4.18, $(a)$ is an obvious consequence of the inequalities (2) to (5). Assume that (4) is false. Then $\alpha>\beta$. Since $\alpha<\infty$, Lemma 4.21 shows that $X$ contains closed subspaces $E$ and $F$ such that $\operatorname{dim} F=\beta$ and

$$
X=\mathscr{N}(S) \oplus E=\mathscr{R}(S) \oplus F
$$

Every $x \in X$ has a unique representation $x=x_{1}+x_{2}$, with $x_{1} \in \mathscr{N}(S), x_{2} \in E$. Define $\pi: X \rightarrow \mathscr{N}(S)$ by setting $\pi x=x_{1}$. It is easy to see (by the closed graph theorem, for instance) that $\pi$ is continuous.

Since we assume that $\operatorname{dim} \mathscr{N}(S)>\operatorname{dim} F$, there is a linear mapping $\phi$ of $\mathscr{N}(S)$ onto $F$ such that $\phi x_{0}=0$ for some $x_{0} \neq 0$. Define

$$
\Phi x=T x+\phi \pi x \quad(x \in X) .
$$

Then $\Phi \in \mathscr{B}(X)$. Since $\operatorname{dim} \mathscr{R}(\phi)<\infty, \phi \pi$ is a compact operator; hence so is $\Phi$ (Theorem 4.18).

Observe that

$$
\Phi-\lambda I=S+\phi \pi .
$$

Since $x_{0} \in \mathscr{N}(S), \pi x_{0}=x_{0}$, and so $\phi \pi x_{0}=0$. It follows that $\lambda$ is an eigenvalue of $\Phi$ (with eigenvector $x_{0}$ ). Hence

$$
\mathscr{R}(\Phi-\lambda I) \neq X
$$

by Theorem 4.24 .

Since $\pi x=0$ for every $x \in E$, (8) shows that

$$
(\bar{\Phi}-\lambda I)(E)=S(E)=S(X)=\mathscr{R}(S)
$$

If $\dot{x} \in \mathscr{N}(S)$, then $\pi x=x$, and (8) gives

$$
(\Phi-\lambda I)(\mathscr{N}(S))=\phi(\mathscr{N}(S))=F
$$

It follows from (10) and (11) that

$$
\mathscr{R}(\Phi-\lambda I) \supset \mathscr{R}(S)+F=X .
$$

The contradiction between (9) and (12) shows that (4) is true. This completes the proof of $(a)$.

Part (b) follows from $(a)$, for if $\lambda$ is not an eigenvalue of $T$, then $\alpha(T)=0$, and $(a)$ implies that $\beta(T)=0$, that is, that $\mathscr{R}(T-\lambda I)=X$. Thus $T-\lambda I$ is invertible, so that $\lambda \notin \sigma(T)$.

It now follows from $(b)$ of Theorem 4.24 that 0 is the only possible limit point of $\sigma(T)$, that $\sigma(T)$ is at most countable, and that $\sigma(T) \cup\{0\}$ is compact. If $\operatorname{dim} X<\infty$, then $\sigma(T)$ is finite; if $\operatorname{dim} X=\infty$, then $0 \in \sigma(T)$, by $(e)$ of Theorem 4.18. Thus $\sigma(T)$ is compact. This gives $(c)$ and completes the proof of the theorem.


\end{document}