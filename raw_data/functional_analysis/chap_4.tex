$$ \begin{array}{r c l c r c l}{{\quad\quad\quad\quad}}&{{}}&{{}}&{{}}\\ {{}}&{{}}&{{}}&{{}}&{{}}\\ {{}}&{{}}&{{}}&{{}}&{{}}\\ {{}}&{{}}&{{}}&{{}}&{{}}\\ {{}}&{{}}&{{}}&{{}}&{{}}&{{}}\\ {{}}&{{}}&{{}}&{{}}&{{}}\\ {{}}&{{}}&{{}}&{{}}&{{}}&{{}}\end{array}\quad. $$ DUALITY IN BANACH SPACES 4 The Normed Dualofa Normed Space case, Introduction If $\textstyle{\bar{X}}$ and $\boldsymbol{\mathit{I}}$ are tological vector spaces ${\mathcal{B}}(X,Y)$ ${\cal{Y}},$ , not on that of X $\boldsymbol{\mathit{I}}$ is scile o In that $X^{\star}$ ${\mathcal{O}}(X,Y$ can se onome avratrar wywhe $\textstyle X$ an ${\cal{Y}}.$ wil denote the ${\mathcal{B}}(X,X)$ colecto l od inar mapng oeatoso ${\mathcal{R}}(X,Y)$ ’ For sinplicity (This dends only on the vectorspae structurc o $X$ into ${\boldsymbol{Y}}.$ '.) In general, will be abbreviated to ${\mathcal{A}}(X).$ Each is itself a vector space, with rpete ualfions o adio amuititin or ntoto ln teprestchatr w slal i oy itnmesae $X{\overset{}{\underset{}{\sim}}{\sim}}.$ ther gxwyas ih x. ca mea itosoyosaevi d ses the scalar feld so that MX,Y is th dua space x' o $X^{\star}$ which turns out o be stronger than its form the main topic of this chapter $\textstyle X$ $X,$ the above-mentioned norm on e(X, Yy dfncs a topology on and is normed dua weak*-tology. Therelatons betwcn Banach spc 4.1 Theorem Suppose $\textstyle X$ and $\boldsymbol{\mathit{I}}$ Y are normed spaces.Associate $\bar{\boldsymbol{\imath}}O$ o each $\Lambda\in{\mathcal{B}}(X,$ Y) the number $(1)$ $\|\Lambda\|={\operatorname*{sup}}\{\|\Lambda x\|:x\in X,\,\|x\|\leq1\}.$88 GENERAL THEORY $$ \begin{array}{c c c c c c c c c c c c c c}{{}}&{{}}&{{}}&{{}}&{{}}&{{}}&{{}}&{{}}&{{}}&{{}}&{{}}&{{}}&{{}}\\ {{}}&{{}}&{{}}&{{}}&{{}}&{{}}&{{}}&{{}}&{{}}&{{}}&{{}}&{{}}&{{}}&{{}}&{{}}&{{}}&{{}}&{{}}&{{}}&{{}}&{{}}&{{}}&{{}}&{{}}&{{}}&{{}}&{{}}&{{}}&{{}}&{{}}&{{}}&{{}}&{{}}&{{}}&{{}}&{{}}&{{}}&{{}}&{{}}&{{}}&{{}}&{{}}&{{}}&{{}}&{{}}&{{}}&{{}}&{{}}&{{}}&{{}}&{{}}&{{}}&{{}}&{{{}}}&{{{{}}}}&{{{{}}}}&{{}}&{{{{{{{}}}&{{{{}}}}}&{{{{{}}}}&{{{{{{{{{{}}}&{}}&{{{}}}&{{{{{{}}}}&{{{{{}&{{{{{{{{{{}}.}}}}}&{{{}}}&{{{{{}&{{{{{{}}}}}}&{{{{{{{{{{{ $$ This eio o IMl makes 8(X, YDinto a norme space.If Yisa anach space,so i 8(X, Y). some multiple of the unit bal, MAl rnoor Snce subsets of normed spaces are bounded if and only if they lic i for every $\Lambda$ e M6(X, Y).If α is a scalar, $<\emptyset$ then $(a\Lambda)(x)=\alpha\cdot\Lambda x,$ so that (2) $$ \|a\Lambda\|=|\alpha|\ \ \|\Lambda\|.\ \ \ . $$ The triangle inequality in ${\cal{Y}}$ shows that $$ \|(\Lambda_{1}+\Lambda_{2})x\|=\|\Lambda_{1}x+\Lambda_{2}x\|\leq\|\Lambda_{1}x\|+\|\Lambda_{2}\,x\|\nonumber $$ $$ \leq(\|\Lambda_{1}\|+\|\Lambda_{2}\|)\|x\|\leq\|\Lambda_{1}\|+\|\Lambda_{2}\| $$ for every $x\in X$ with $\|x\|\leq1$ Hence (3) $$ \|\Lambda_{1}+\Lambda_{2}\|\leq\|\Lambda_{1}\|+\|\Lambda_{2}\|. $$ If $\Lambda\neq0,$ then $\mathrm{A}x\neq0$ for some $x\in X;$ hence $\|\lambda\|>0.$ Thus !G(X, Y)is a normed space is complete and tha $\left\{\Lambda_{n}\right\}$ is a Cauchy sequence in Assume now that ${\mathbf{}}Y$ 3(X, Y). Since (4) $$ \|\Lambda_{n}x-\Lambda_{m}x\|\leq\lVert|\Lambda_{n}-\Lambda_{m}\|\,\|x\| $$ and since itis asumed that $\|\Lambda_{n}-\Lambda_{m}\|\to0$ as $\textstyle n$ and m tend to o $\{\Lambda_{n}.x\}$ is a Cauchy sequence in ${\cal{Y}}$ for every $x\in N.$ Hence (5) $$ \Lambda x=\operatorname*{lim}_{n\to\infty}\Lambda_{n}x $$ exceed $E\|setminus X\|,$ provided that ${\mathbf{\nabla}}m$ and $\;n$ is linear. If $\scriptstyle{s>0}$ the right side of (4) does not exists. Itis clear that $\Lambda\,;$ $X\to Y$ are suicently lrge. tflows tha (6) $$ \|\Lambda x-\Lambda_{m}x\|\leq\varepsilon\|x\| $$ for all large m. Hence $\|\Lambda x\|\le(\|\Lambda_{m}\|+s)\|x\|,$ so that $\Lambda\in{\mathcal{B}}(X,Y),$ and $\|\Lambda-\Lambda_{m}\|\leq c$ Thus $\Lambda_{m}\to\Lambda$ in the norm of 4(X, Y). This estabishes the com- $\iiint_{}^{}j\,d\rangle$ pleteness o ${\mathcal{B}}(X,Y)$ 4.2 Duality It wil beconvenient to designate lements of te dual spac $X^{\star}$ of $\textstyle{\bar{X}}$ by $\chi^{\cong}$ and to write (1) $$ \langle x,x^{\pi}\rangle $$ in place of between the action of $X^{\mathbb{N}}$ This notion s wl ated o the synmetry (or duality tht exis on $X^{\ast\ast}$ on the $x^{\mathrm{s}}(x)$ on X on the one hand and the action of $\textstyle{X}$ other.Thefllowing theorem states some basic proprtes o tisdulitDuArr N BANACn srAces 89 4.3 Theorem Suppose $\boldsymbol{B}$ is heclseduiball aormed spac $X.$ Define for every $x^{*}\in X^{*}$ $$ \|x^{\ast}\|=\operatorname s u p}\left\{|\zeta x,\,x^{\ast}\right\}|:x\in B \} $$ (a) This norm makes $\textstyle{\frac{q}{-}}\ =\ \cdot$ into a ach spae $X^{\star}.$ For ever $x\in X,$ $X^{\rtimes}$ (b) Let $B^{\star}$ be the losed umi ball $$ \|x\|=\operatorname*{sup}\{|\zeta x,\,x^{*}\}\colon x^{*}\in B^{*}\}. $$ (C) $B^{\star}$ Consequemtly $x^{*}\to{\sqrt{x}},x^{*}/$ is a bounded linear fumctional on $\scriptstyle{X^{\bullet}}$ *, of orm lIxrl is weak*-compact PROOF Since ${\mathcal{R}}(X,Y)=X^{*}.$ when ${\cal{Y}}$ is the scalarfield,(a is a corollaryo Theorem 4.1 Fix $x\in V.$ The colar Thorm 3 sow atres $y^{*}\in B^{*}$ such that (1) $$ \zeta x,y^{*}\rangle=\|x\|. $$ On the other hand. (2) $$ |\,\langle x,\,x^{*}\rangle\,|\,\leq\,\|x\|\,\|x^{*}\|\,\leq\,\|x\| $$ that for every $x^{*}\in B^{*}.$ Part( ofows fo G) and 2D the efiniton o $\left\|X^{\mathbb{X}}\right\|$ shows $x^{*}\in B^{*}$ Since the open uni bal $U$ of $\textstyle X$ Yis dense n ${\boldsymbol{B}}.$ $x\in U,$ Part $\left(c\right)$ now follow $J/{\big/}$ if and only if $|\langle x,x^{*}\rangle||\leq1$ for every directly from Theorem 3.15. Remark The weak-topolgy o $X^{\times}$ is. by definition,the eakest one tha makes allfnctionals $$ x^{*}\to{\sqrt{x}},x^{*}\rangle $$ well coniu.Pr oster ta henom tology $X^{\star}$ is stronger since hanisvaxk.togy;infat st ioneruies $X<\infty.$ 。l" woeprosusase s sesmMws ies 0 $X^{\star}$ normed dual of Unlessth contrary is expicily state, $\textstyle X$ is normed), nd al togical conctsrecai $\textstyle{\bar{X}}$ topology wil not ply an important roe $X^{\star}$ will fom now on denote the Y(whenever vw…osos n o Timsowswsasies Theorem 4.1 W now gic a atenoesritino e opratr nonm detinea90 GENERAL THEORY 4.4 Theorem If X and $\boldsymbol{\mathit{I}}$ Y are normed spaces and $i f$ A∈ 8(X, Y), then $$ \|\Lambda\|=\operatorname*{sup}\left\{|\langle\Lambda x,\,y^{*}\rangle\,|:\,\|x\|\leq1,\,\|\nu^{*}\|\leq1\right\} $$ }. PROOF Apply (b) of Theorem $4.{\underline{{3}}}$ with $\boldsymbol{\mathit{I}}$ in place of $X.$ This gives $$ \left\|\Lambda x\right\|=\operatorname*{sup}\left\{\left|\zeta\Lambda x,y^{*} \rangle\right|:\left\|y^{*}\right\|\leq1\right\} $$ for every $x\in X.$ To complete the proof, recall that $$ \|\Lambda\|=\operatorname*{sup}\left\{\|\Lambda x\|\colon\|x\|\leq1\right\}. $$ / $X^{\land\times}$ Statement (b) of Theorem $4.3$ The second dual of a Banach space The normed dual $X^{\rtimes}$ of a Banach $X^{\mathfrak{s}\star\mathfrak{n}}.$ 4.5 is iself a Banach space and hence has a normed dual of its own, denoted by defnes a unique dxe space $X$ shows thatevery xe $\textstyle X$ by the equation (1) $$ \zeta x,\,x^{*}\rangle=\zeta x^{*},\,\phi x\rangle\qquad(x^{*}\in X^{*}), $$ and that (2) $$ \|\phi x\|=\left\|x\right\|^{*}\qquad(x\in X). $$ It follows from (I) that $\varnothing\;$ $:X\to X^{**}$ is linear; by (2), 办 is an isometry. Since $\textstyle X$ is now $\varnothing\;$ of Thus assumed to be complete, $X$ is identified with $\phi(X).$ ; then $\textstyle X$ is regarded as a subspace of $X^{\ddagger\times\varepsilon}.$ But there are $X^{\lambda\times\kappa}.$ $X^{\lambda}$ isometric isomorphism $\phi$ $\phi(X)$ is closed in $\scriptstyle X^{n,i}$ P-spaces with $1<p<\infty)$ for which $\phi$ The members of $\phi(X)$ is a isometrcisorphism of X onto a closed subspace of $X^{\sharp\pm\varepsilon}$ Frequently, are exactly those inear functionals on $X^{\star}$ that are con is stronger,it may happen that tinuous relative to its wcak*-topology.(See Section 3.14.) Since the norm topology is a proper subspace of $\phi(X)$ many important spaces $\textstyle X$ (for example, all $L^{\,p}.$ $\phi(X)=X^{**};$ lt should be stressed that, in order for $\textstyle X$ ;theseare calld refexive. Some oftheir properties are given nExercise to be reflexive, the existence of some of $\textstyle X$ onto $\scriptstyle X^{**}$ is not enough; itiscrucial ththe identity (1) be satisfied by ${\boldsymbol{\phi}}.$ 4.6 subspace of $X^{*}$ Annihilators Suppose $\textstyle X$ is a Banach space, $\bar{M}$ is a subspace of $X,$ and $\boldsymbol{N}$ is a ; neither $\bar{M}$ nor $\textstyle N$ Vis assumed to be closed. Their anihilators ${\boldsymbol{M}}^{\prime}$ and lN are defined as follows: $$ \begin{array}{l}{{M^{\perp}=\{x^{\ast}\in X^{\ast}:\zeta x,\,x^{\ast}\}=0\mathrm{~for~all~}x\in M\},}}\\ {{\downarrow}}\\ {{\downarrow}}\\ {{\downarrow}}\end{array} $$ lN is the subset of $\textstyle X$ consists of all bounded linear functionals on $X$ that vanish on $\bar{M},$ and Thus ${\boldsymbol{M}}^{\prime}$ oin which every member of ${\cal N}$ vanishes. ltis clear that ${\mathrm{M}}^{1}$ and ${}^{\perp}N$DUAurr nN RANACH sPACrs 91 where _ are vector spaces. Since $\bar{M}$ Gse Setion .5 ${\mathcal{M}}^{\prime}$ s a weak*-closed subspace of $X^{\mathbb{A}}$ *. Thc $\scriptstyle{\mathcal{X}}$ ranges ovr ${\mathcal{M}}^{2}$ is ososuo $X$ is tegestio t upsas o t tionas ol proof that $\^{\bot}N$ isievesmiseisin homesiens utveen o s mns 4.7 Theorem Lmler eeiyohee .4M4/) s he nomcloure o M im $X,$ Y, and (の)(IN) is he wak-clsweo/ Win $X^{\mathbb{N}_{s}}.$ Theorem 3.12. As rgas a el the norm-closue $\bar{M}$ equl is wak closure, by if $^{1}(M^{\perp})$ pRoor Ifxe M, then $\zeta x,\,x^{*}\rangle=0$ for every $x^{*}\in M^{\perp}$ , so that $x\in{^{\perp}}(M^{\perp})$ Since If Thus Similarly $\mathbb{F}\ x^{*}\in N,$ eis nrmcos .i tnin tht noiciosu6 $\zeta x,\,x^{*}\rangle=0$ for every xe N, so ta such that $\zeta x,\,x^{*}\rangle\neq0.$ with $x\notin{\overline{{M}}}$ uhc Hahn-Banach theorem yeisa $\mathcal{M}$ of M. On the other hand, $x\not\equiv^{\perp}(M^{\perp}).$ . and (a) is provcd. $x^{*}\in M^{\perp}$ thus $x^{\ast}\not rightarrow(^{\perp}N)^{\perp},$ then contains the weak*-'osue ${\widetilde{N}}$ of $N.$ ${\dot{x}}^{\star}\not\in\tilde{N},$ which proves (b) $\textstyle x\in{\stackrel{\bot}{-}}N$ $x^{*}\in(^{bot}N)^{\perp},$ This weak-cosed subspace O $X^{\times}$ such that x $x^{*}\gamma\neq0;$ the Hoanai toemipi etiScsievxsa $X^{\star}$ is wexk*:0y) is tesxeo a orosoo a s o-o- s sae $\textstyle X$ n namnan $X^{\star}$ oris mnaoaisms sevwaso Dioe Banach space $X_{\cdot}$ ’, then $X/M$ 4.8.Duasosoe an qutitnt spacesr ${\mathcal{N}}^{\prime}$ of $\bar{M}.$ 1. Somewhat imprcisely th $X/M$ can result is that $\bar{M}$ is a closed subspace of a be described with the aid o teanhiat ea poanashs tssest tewemisno Tnisssi e imhA nasasosinsn $$ M^{\star}-X^{\star}/M^{\perp}\qquad\mathrm{and}\qquad(X/M)^{\ast}\:=M^{\perp}. $$ TRA sosososusergesuaoae oy ometiopu The foing thorm aesiesesesxeic 4. Thcoem Le Mo oe anc o an a $m^{\star}\in M^{\star}$ to a functional $X.$ $x^{*}\in X^{*}.$ Define (の) Tne laz- ahorem xids earc om* = x*+ M Then o isami metreiomorphism $M^{k}$ onto X*/M192 GINFRAL Turoxv $$ \left.{\begin{array}{l l}{-{\frac{\cdot--- \lfloor-\vphantom{\scriptstyle-- \rfloor}-\left\dots- \dots- \dots- \dots- \dots- \lfloor-\vphantom{{\displaystyle-~-~-~\mathop{-~ ~-~-~-~-~-~-~}}~}}}}\\ {\right.\ \ \ ,}\end{array}}\right. $$ (b)Let r: $X\to X/M$ be the quotient map. Pu $Y=X/M$ .For each y*e Y*, define $$ \tau y^{*}=y^{*}\pi. $$ Then r is am isometric isomorphism of $V^{\wedge i}$ onto $M^{1}.$ roor(a) If $\chi^{\wedge}$ and $x_{1}^{\ast}$ are extensions of $p n^{\ast}$ ,then $x^{*}-x_{1}^{*}$ is in ${\mathrm{M}}^{\mathrm{L}}$ ; hence is linear. Since the restriction of every $x^{*}\in X^{*}$ Thus s wiefned. A trivil verictonshows that is a member of $M^{\star},$ the $x^{\star}+M^{\perp}=x_{1}^{\star}+M^{\perp}.$ to $\bar{M}$ range of $\textstyle{\boldsymbol{\sigma}}$ is all of $X^{\star}/M^{\perp}$ it is obvious that $\|m^{*}\|\leq\|x^{*}\|,$ The Fix $m^{*}\in M^{*}$ If $x^{*}\in X^{*}$ extends $m^{*}{\mathrm{;~}}$ so obtained is $\|x^{\star}+M^{\perp}\|_{2}$ l, by the greates lower bound of the numbers $\|X^{\geq}\|$ definition of the quotient norm. Hence $$ \|m^{*}\|\leq\|\sigma m^{*}\|\leq\|x^{*}\|. $$ that closed subspace of ${\cal{Y}},$ $\tilde{\top}\mathrm{i}\times\chi^{\star}\in M^{\perp}$ . Let $\textstyle{N}$ V be the ull space of $\chi^{\Re},$ Since M $\mathbb{C}\cap N.$ t follows there (b)1 $\textsf{f}x\in X$ But Theorem 3.3 furnishes an cxtension $X^{\Re}$ of $m^{\ddagger}$ with $\|x^{\ast}\|=\|m^{\ast}\|$ is a continuous linear functional on $\textstyle X$ . This completes Ga) then $\pi x\in Y;$ hence $x\to y^{\Re}\pi x$ The linearity $\|\sigma m^{*}\|\,=\,\|p n^{*}\|$ and $y^{\star}\in Y^{\star}.$ $x\in M.$ Thus $\tau y^{*}\in M^{\perp}.$ which vanishes for of ris obvious. H $\Lambda$ is continuous, that is, $\Lambda\in Y^{*}$ Hence $\tau\Lambda=\Lambda\pi=x^{*},$ is r(N), a By is a linear functional $\Lambda$ on ${\mathbf{}}Y$ such that $\Lambda\pi=\chi^{\kappa},$ The null spacc of $\Lambda$ The ,by the einition o the quoticnt topology i $Y=X/M.$ Theorem 1.18, norm in range of ristherefore ll of $M^{\perp}$ and $r>1.$ the definition of the quotient $\pi x_{0}=y.$ $X/M$ Fix y*e Y*. Ifye Y $\|y\|=1,$ , with $\|x_{0}\|<r,$ such that shows that thcrc is an $x_{0}\in X,$ Hence $$ |\langle y,y^{*}\rangle|\,=\,|y^{*}\pi x_{0}\,|\,=\,|\,\tau y^{*}x_{0}\,|\,\leq\,\|\tau y^{*}x_{0}\,\|\,\leq\,r\|\tau y^{*}\|, $$ which implies that $$ \|\ y^{\star}\|\leq\|\tau y^{\star}\|. $$ On the other hand, $\|\pi x\|\ \leq\ \|x\|$ for every $x\in{\mathcal{X}}$ .Hence $$ \mid\tau y^{*}x\mid\,=\,\mid y^{*}\pi x\mid\,\leq\,\Vert y^{*}\Vert\,\Vert\,\Vert\pi x\Vert\,\leq\,\Vert\,J^{*}\Vert\,\Vert\,\Vert\, $$ xll, which implies that $$ \lVert\tau y^{\star}\rVert\ \leq\ \lVert y^{\star}\rVert\cdot $$ / This completes the proof Adjoints we shll w aoctwih each Te 8(x, Y)itsaioinan operat $J^{**}\in{\mathcal{B}}(Y^{*},\,X^{*}),$ and $\boldsymbol{\mathit{I}}$ ansi isesieisirpes Trtetei havio can be represented by a matrix $|T|;$ in that $\widehat{\overline{{{\cal Z}}}}^{\mathcal{Y}\mathcal{Y}\mathcal{Y}}_{\bullet}$ If $\textstyle X$ are finite-dimensional, every Te M ${\mathcal{X}},\ {\mathcal{Y}})$DUAurrv nN BANACti SPAcEs 93 case $[T^{*}]$ istns .[I. rod ath aiousveorspaebsesa theory Ruoncrososn nutueitowiwipis eitcetmesoasase whoru hsioislinaiearve eceianaimaec。 hemwtvaiona e oi c ostito o was wnwa ea and $\boldsymbol{\mathit{I}}$ which furnishesthe dfinition“o $T^{\times{\mathfrak{X}}}$ and $\boldsymbol{\mathit{I}}$ Manxot ni perte oadiontsend o te comnes $X$ wviyoseistiogouth $\textstyle X$ theon n ne i pa mram ioS Fr isison are asasxces nietem*1 4.10 Theorem Suppose $\textstyle X$ and ${\cal{Y}}$ are normed spaces. To each $T\in{\mathcal{B}}(X,\,Y)$ corre- spomds umique T* e M(r, Xx" haise (1) $$ \displaystyle{\langle T x,y^{*}\rangle=\left\{x,T^{*}y^{*}\right\}} $$ for all x e $\textstyle{\mathcal{X}}$ and all $y^{*}\in Y^{*}.$ Moreove $T^{\mathrm{st}}$ satisfies (2) $$ \|T^{\ast}\|=\|T\|. $$ PROOF If $y^{*}\in Y^{*}$ and $T\in{\mathcal{D}}(X,Y),$ define (3) $$ T^{*}y^{*}=y^{*}\circ T. $$ Being thcoposio woconinus inar mapins $T^{*}{\mathfrak{p}}^{*}\in X^{\aleph_{*}}.$ AIso, $$ \zeta x,\;T^{*}y^{*}\rangle=(T^{*}y^{*})(x)=y^{*}(T x)=\zeta T x,\,y^{*}\rangle, $$ $\scriptstyle{I^{\bullet}}\nu^{\bullet}$ uniquely which is (D Thefact that () holds for every $x\in X$ obviously determines If yfe $Y^{\mathbb{X}}$ and $y_{2}^{*}\in Y^{*}$ ,then for every $x\in X,$ $\lambda^{*}$ $$ \begin{array}{r l}{{\zeta x,\,T^{*}(y_{1}^{*}+y_{2}^{*})\gamma=\zeta T x,\,y_{1}^{*}+y_{2}^{*}\rangle}}\\ {{\,}}&{{=\zeta T x,\,y_{1}^{*}\rangle+\zeta T x,\,{T^{*}}_{2}^{*}\rangle}}\\ {{\,}}&{{=\zeta x,\,{T^{*}y_{1}^{*}}\rangle+\zeta x,\,{T^{*}}y_{2}^{*}\rangle}}\\ {{\,}}&{{=\zeta x,\,{T^{*}}_{1}^{*}\chi_{2}^{*}\rangle+\zeta x,\,{r^{*}}\gamma}}\\ {{\,}}&{{=\zeta x,\,{T^{*}}_{y^{*}}^{*}+{T^{*}}y_{2}^{*}\rangle}}\end{array} $$ (4) T*(y* + yf)= 7*y* + T*y Similarly $T^{*}(\sigma y^{*})=\alpha T^{*}y^{*}$ . Thus $T^{*}\colon Y^{\bullet}\to X^{\bullet}$ is linear. Finally,(b) of Theo- rem 4.3 leads to $$ \begin{array}{c}{{T\Vert={\sf S}{\sf I}\Vert\left\{\left\{{\cal J}X,{\scriptstyle J^{\star}}\right\}}\iota:\Vert x\Vert\leq1,\ \|\jmath^{\star}\right\}\leq\mid\S}}\\ {{={\sf s u p}\left\{\left.\left\{\left.\bigcap\left\{{\sf f}{\sf z},\ {\sf-}^{\star}\right\}^{\star}\right\}\ :\left\|{\sf k}\right\|\leq\mid\top,\left\|{\sf l}\right\}\right)\leq\mid\mid}}\\ {{-{\sf-}\cdots-{\sf L}\operatornameddot{\sf m}\boldmath-\ *\mid\neg{\sf-}.\ldots}}\end{array}\right.\quad.\quad\quad\quad\quad\quad\quad\quad\quad\quad\quad\quad\quad\quad\quad\quad\quad\quad\quad\quad\quad\quad\quad\quad\quad\quad\quad\quad\quad\quad\quad\quad\quad\quad\quad\leq1\}}\\ {{\quad\quad\quad\quad\quad\quad\quad\quad\quad\quad\quad\quad\quad\quad\quad\quad\quad\quad\quad\quad\quad\quad\quad\quad\quad\quad\quad\quad\quad\quad\quad\quad\quad\quad\quad\quad\quad\quad\quad\quad\quad\quad\quad\quad\quad\quad\quad\quad\quad\quad\quad\quad\quad\quad\quad\quad\quad\quad\quad\quad\quad\quad\quad\quad\quad\end{array}\quad\right\}1}\quad\quad \}\quad \}\quad\quad \}0\end{array} \}\quad\quad\quad\quad\end{array}\quad\quad\quad\quad\quad $$ $J/{\big/}/$ sup {T*|: /* ≤1}=|7*194 oENERAL THronv 4.11 Notation If T maps $\textstyle{X}$ into ${\boldsymbol{V}},$ , the null space and the range of ${\boldsymbol{T}}$ will bedeno ted by ${\mathcal{N}}(T)$ and ${\mathcal{R}}(T),$ rcspectively:: $$ \begin{array}{l}{{{\mathcal{N}}(T)=\{x\in X\colon T x=0\},}}\\ {{{\mathcal{M}}(T)=\{y\in Y\colon T x=y{\mathrm{~for~some~}}x\in X\}.}}\end{array} $$ The next theorem concerns annihilators; see Section 4.6 for the notation. 4.12 Theorem Suppose $X$ Y and $\mathbf{\Sigma}Y$ are Banach spaces, and $T\in{\mathcal{B}}(X,\,Y).$ .Then $$ \mathcal{N}(T^{\star})=\mathcal{B}(T)^{\perp}\qquad a n d\qquad\mathcal{N}(T)=\stackrel{\textstyle1}{_{\displaystyle1}}\mathcal{B}(T^{\star}). $$ PRoor In each ofthe follwng two columns,each statementis obviously equiv alent to the one that immediatey follows and/or precedes it $$ \begin{array}{l l}{{y^{\ast}\in{\mathcal A}(T^{\ast}).\qquad x\in{\mathcal A}(T).\qquad x\in{\mathcal A}(T).\qquad x\in{\mathcal A}(T).\qquad r\in{\mathcal A}(\mathcal A}(\mathcal{T}).\qquad{\mathrm{~for~all~}}y^{\ast}.\qquad{\mathrm{~for~all~}}\cdot\qquad{\mathrm{~for~all~}}y^{\ast}.}}\\ {{\langle T x,\qquad x\in{\mathcal A}(T)^{\perp}.\qquad{\mathrm{~for~}}\qquad x\in{\mathfrak{D}}\mathcal{R}(T^{\ast}).\qquad x\in{\mathrm{~x~}}}\end{array} $$ // Corollaries (a) ${\mathcal{N}}(T^{\kappa})$ is weak*-closed in $Y^{\rtimes}$ (b) B2(T) is dense in $\mathbf{\Omega}Y$ if and only i $T^{**}$ is one-to-one. (C) T is one-to-one if and only it ${\mathcal{R}}(T^{*})$ is weak*-dense in $X^{\ast}$ Recall that $M^{\ !}$ is weak*-closed in $Y^{\star}$ for cvery subspacc $\ M$ of ${\boldsymbol{Y}}.$ In particular, this is true of ${\mathcal{R}}(T)^{\perp}$ . Thus (a) follows from the theorem. As to (6b),8*(T) is dense in ${\mathbf{}}Y$ if and only if ${\mathcal{R}}(T)^{1}=\{0\};$ in that case ${\mathcal{N}}(T^{*})=\{0\}$ if and only if ${\mathcal{R}}(T^{*})$ is annihilated by no $x\in X$ Likewise, $^{1}{\mathcal{M}}(T^{*})-\{0\}$ other than $x=0{\mathrm{;}}$ this says that ${\mathcal{P}}(T^{*})$ is weak*-dense in $X^{\star}$ Note thal the Hahin-Banach theoem 3. was tcty used in thc proofs o (b) and (c) There is a very useful analogue of (b) that allows us to dccide, in terms o $T^{\ast},$ whether ${\mathcal{R}}(T)=Y,$ and will be obtained by first looking for conditions on $T^{\mathfrak{X}}$ This is given in Theorem ${\hat{A}}_{*}\mid S\sim.$ that is, whether ${\mathbf{}}T$ maps $\textstyle X$ onto ${\boldsymbol{V}}.$ which imply that ${\mathbf{}}T$ has closed range (Thcorcm 4.14) 4.13 Lemma Suppose ${\boldsymbol{U}}$ and ${\mathbf{}}V$ arethe open ni blls in he Banch spaces $\textstyle X$ and $V,$ respectively. Suppose $T\in{\mathcal{P}}(X,\,Y)\,$ , and $\epsilon>0.$DUALrrv IN BANAcuH SrACEs 95 (a) IJ the closure of $T(U)$ contains cV, then $T(U)\Rightarrow c V.$ の) $c\|y^{\ast}\|\le\|T^{\ast}y^{\ast}\|_{..}$ for every $y^{*}\in Y^{*},$ , he $T(U)\Rightarrow c V.$ PROOF (a) Take $c=1,$ without los o geraliy. Then $T({\mathcal{U}})\supset{\mathcal{V}}.$ To every and y ∈ ${\cal{Y}}$ and every $\scriptstyle{k>0}$ corresponds therefore an $x\in X$ with $\|x\|\leq\|y\|$ $\|y-T x\|<\varepsilon$ ${\mathrm{Pick~}}y_{1}\in V.$ Pick $\varepsilon_{n}>0$ so tha $$ .^{\cdot\;\;\;\;\;\;\;\;\;\;\;\;\;\;\;\;\;\;\;\;\;\;\;\sum_{n=1}^{\infty}\varepsilon_{n}<1\,-\,\|y_{1}\|._{} $$ Assume $r_{\mathit{l}}$ ≥1 and ${\mathbf{}}y_{n}$ is picked. There exist ${\mathcal{X}}_{n}$ such that $\|x_{n}\|\leq\|y_{n}\|$ and $\|\jmath_{n}-T x_{n}\|<\varepsilon_{n}.$ Put $$ \mathbf{\epsilon}\cdot\mathbf{\qquad}\cdot\mathbf{\qquad}\quad\mathbf{\nabla}\mathbf{\epsilon}\quad\quad y_{n+1}=y_{n}-T x_{n}. $$ By inuctn ti soses w seue $\langle x_{n}\rangle$ and $\langle y_{a}\rangle$ Note that Hlence $$ \|x_{n+1}\|\leq\|y_{n+1}\|=\|y_{n}-T x_{n}\|<6_{n}\,. $$ $$ \sum_{n=1}^{\infty}||x_{n}||\leq\|x_{1}\|\ +\sum_{n=1}^{\infty}\varepsilon_{n}\leq\|y_{1}\|\ +\sum_{n=1}^{\infty}\varepsilon_{n}<1. $$ t follows that $x=\sum x_{n}$ is in $U$ (See Exercise 23) and that $$ T x=\operatorname*{lim}_{N\to\infty}\sum_{n=1}^{N}T x_{n}=\operatorname*{lim}_{N\to\infty}\sum_{n=1}^{N}(y_{n}-y_{n+1})=y_{1} $$ since $y_{N+1}\to0$ as $N arrow\infty.$ Thus $y_{1}=T x\in T(U),$ which proves (a. Since ${\boldsymbol{E}}$ is closed, (b)Let $\boldsymbol{E}$ be the closure of $T(U),$ ohsesorentsascamseo rn $\in\,Y,\,y_{0}\notin E$ thc roo f e pe mppg hem2 pick $y_{0}$ tr sat Tn convx an ance Teorm 7 yedsS for al $y\in E.$ If $x\in U$ then $T x\in E,$ $$ \big\lbrack\,\zeta y,y^{\ast}\big\rangle\,\big\rbrack\leq1< \vert\,\zeta y_{0}\,,y^{\ast} \rangle\big\rbrack $$ so that t follows tha $$ \textstyle|\zeta x,\,T^{*}y^{*}\rangle|\ =|\zeta T x,\,y^{*}\rangle|\leq1. $$ $$ c\|y^{*}\|\leq\|T^{*}y^{*}\|\leq1, $$ and thereforc $$ \mid<\mid\zeta y_{0}\,,\,y^{\star}\rangle\,\mid\leq\mid\mid y_{0}\Vert\,\Vert y^{\star}\mid\leq c^{-1}\mid\mid y_{0}\mid\mid, $$ or $\|y_{0}\|>x$ c. Thus $c V\subset E,$ and G) folow from (a //96 GENERAL THEORY $$ \begin{array}{c c c c c c c c c c c c c c c c c c c c}{{\bar{\imath}}}&{{\quad}}&{{\quad}}&{{\quad}}&{{\quad}}&{{\quad}}&{{\quad}}&{{\quad}}&{{\quad}}&{{\quad}}&{{\quad}}&{{\quad}}\end{array} $$ 4.14 Theorem If X and $\boldsymbol{\mathit{I}}$ Y are Banach spaces and if Te 98(X, Y),then each of the following three conditions implies the other two: (a)9(T) is closed in ${\boldsymbol{V}}.$ (b) 2(T*) is weak*-closed in $X^{\sharp},$ (c) 98(T*) is norm-closed in $X^{\times}$ RemarkTheorem 3.12 implies that (a) holds $\mathrm{i}\mathbf{f}$ f and only if ${\mathcal{R}}(T)$ is weakly closed. However, norm-closed subspaces of $X^{\star}$ are not always weak*-closed (Exercise T, Chapter 3) PRoor It is obvious that (b) implies (c). We will prove that (a) implies (b) and that (c) implies (a) is the Suppose Ga holds. By Theorem 4.12 and 6b)of Theorem 4.7, ${\mathcal{N}}(T)^{\perp}$ weak*-closure of ${\mathcal{P}}(T^{*})$ To prove (b) it is therefore enough to show that ${\mathcal{N}}(T)^{\perp}\subset{\mathcal{P}}(T^{*}).$ $\mathrm{pick~}x^{*}\in{\mathcal{N}}(T)^{\perp}$ .Define a linear functional $\Lambda$ on ${\mathcal{R}}(T)$ by $$ \Lambda T x=\zeta x,x^{*}\rangle\qquad(x\in X). $$ Note that $\Lambda$ is well defined, for if $T x=T x^{\prime},$ then $x-x^{\prime}\in{\mathcal{N}}(T).$ ; hence $$ \forall x-x^{\prime},\,x^{*}\rangle=0. $$ The open mapping theorem applies to $$ T:X\to{\mathcal{R}}(T) $$ since ${\mathcal{R}}(T)$ is assumed to be a closed subspace of the complete space such that to each $y\in{\mathcal{B}}(T)$ ${\mathbf{}}Y$ and is therefore complete. It follows that there exists $K<\infty$ corresponds an $x\in X$ with $T x=y$ $\|x\|\leq K\|y\|.$ and $$ |\Lambda y|=|\Lambda T x|=|\zeta x,x^{*}\rangle|\leq K\|y\|\,\|x^{*}\|. $$ Thus A is continuous. By the Hahn-Banach theorem, some $y^{\star}\in Y^{\star}$ extends $\Lambda$ Hence. $$ \zeta T x,y^{*}\rangle=\Lambda T x=\zeta x,x^{*}\rangle\qquad(x\in X). $$ This implies $x^{*}=T^{*}y^{*}.$ Since $\star^{\gg}{\stackrel{\delta|\mathbb{R}}{\hbar}}$ was an arbitrary element of ${\mathcal{N}}(T)^{\perp}$ , we have shown that ${\mathcal{N}}(T)^{\perp}\subset{\mathcal{B}}(T^{*}).$ Thus (b) follows from (a) is dense in $\mathbb{Z},$ Corollary (b) to Define Suppose next that $\left(c\right)$ holds. Let Z be the closure of ${\mathcal{R}}(T)$ in ${\cal{Y}}.$ $S\in{\mathcal{B}}(X,\mathbb{Z})$ by setting $S x=T x.$ Since ${\mathcal{R}}(S)$ Theorem 4.12 implies that $$ S^{*}\colon Z^{\ast}\to X^{*} $$ is one-to-oneDUAurr N BANACH srACEs 97 every If $z^{*}\in Z^{*},$ the Hahn-Bacheorm fuins exenso $y^{\star}$ of $\mathbb{Z}^{\frac{3K}{4}}{\frac{\,\,\,\,\,4}{3}}$ for $x\in X$ Hencc is assmed to hold, ${\mathcal{R}}(S^{\star})$ $$ \zeta x,\,T^{*}y^{*}\rangle=\zeta T x,\,y^{*}\rangle=\zeta S x,\,z^{*}\rangle=\zeta x,\,S^{*}z^{*}\rangle. $$ and $T^{\bullet}$ haveidetical ranges.Since (e $S^{\bullet}z^{\bullet}=T^{\bullet}y^{\star}$ l follows that $S^{\star}$ is closed, hence complete Apply the open mapin theremt $$ S^{*}\colon Z^{\star}\to{\mathcal{P}}(S^{\star\tt4}). $$ Since $S^{\ast\ast}$ is oton t cusn s ate s consa $c>0$ which satisfics ${\mathcal{R}}(T)=Z,$ a closed subspace of $\boldsymbol{\mathit{I}}$ $$ {\cal C}||Z^{\star}||\ \simeq\ ||\Im^{\star\ast}Z^{\star\ast}|| $$ by the defintin of ${\boldsymbol{S}}.$ Thus forever $z^{*}\in Z^{*}$ Hence $\mathbf{S}\!:$ $X{\overset{\underset{\mathrm{F}}{}}{\to}}Z.$ is anoe upg . の ma 4.1 In particular $\displaystyle S(X)=Z.$ But $\mathcal{R}(T)=\mathcal{R}(S),$ Thiscomplets te roo tat ) implie $(a).$ // The fowin consuenc slu piato (a) ${\mathcal{R}}(T)=Y$ A1" nocm ywowo a nun x.Xo.r if amd only ir (b) $\|\,T^{\ast}\mathcal{Y}^{\ast}\|\,\geq\,c\|y^{\ast}\|\,.$ Jor some constam $\scriptstyle c\;>\;0$ and for every $y^{\star}\in Y^{\star}.$ rxoor Stement (a hls i and only ${\mathcal{R}}(T^{*})$ isnorm closed i $X^{\lambda}$ (c)T is one-fo-one and $\operatorname{f}{\mathcal{R}}(T)$ pis dcnse and closed. By Theore 41 an oy rncrm.2* A ececiasaen complete and hence close ods ts openmpg term apied ${\boldsymbol{r}}^{\bullet}$ $Y^{\star}\hookrightarrow{\mathcal{R}}(T^{\star})$ gives is (の) Conversely、(b)obvousy mpics h $T^{**}$ is one-to-one; also the inverse ${\mathcal{R}}(T^{*})$ “…eso ayxc ysece s acystecsishi $J/\rangle$ Compact Operators in $X.$ 4.16 DefinitoSuppose $T\in{\mathcal{D}}(X),$ Y is sai be compaci tecosure o $U$ is teopen uni ba in An operator $\textstyle X$ and ${\cal{Y}}$ are Banach spaces n is compact ${\cal{Y}}$ $T(U)$ Si somte psgesos e y sesu smoaua neulel tsausoe a osn nus Wx. frmsisisoisyrerp98 GENERAL THEORY is totally bounded. Also, ${\mathbf{}}T$ is compact if and only if every bounded sequence $\scriptstyle \langle x_{a} \rangle$ in $\textstyle X$ contains a subsequence $\langle x_{n}\rangle$ such that $\{T x_{n_{i}}\}$ converges to a point of ${\boldsymbol{Y}}.$ Many of the operators that arise in the study of integral equations are compact. This accounts for their importance from the standpoint of applications. They are in some respects as similar to linear operators on finite-dimensional spaces as one has any right to expect from operators on infinite-dimensional spaces. As we shall see these sinilarities show up particularly strongly in their spectral properties. 4.17 Definitions(a) Suppose $\textstyle X$ is a Banach space. Then ${\mathcal{B}}(X)$ [which is an algebra:If $S\in{\mathcal{B}}(X)$ and $T\in{\mathcal{B}}(X)$ X)] is not merely a Banach space (see Theorem 4.1) but also an by abbreviation for ${\tilde{g l}}(X)$ ,one defines $S T\in{\mathcal{B}}(X)$ $$ (S T)(x)=S(T(x))\qquad(x\in X). $$ The inequality $$ \|S T\|\leq\|S\|\ \|T\| $$ is trivial to verify. In particular, powers of TE ${\mathcal{B}}(X)$ can be defined $T^{0}=I,\,$ the identity mapping on X, given by $I x-x,$ and $T^{n}=T T^{n-1}.$ for $n=1,\,2,\,3,\,\ldots$ (b)An operator $T\in{\mathcal{P}}(X)$ is said to be invertible if there exists $\mathbf{S}\in{\mathcal{B}}(X)$ such that $$ S T=I=T S. $$ In this case, we write $S=T^{-1}.$ By the open mapping theorem, this happens if and only if ${\mathcal{N}}(T)=\{0\}$ and ${\mathcal{R}}(T)=X.$ $\left(c\right)$ The spectrum $\sigma(T)$ of an operator ${\boldsymbol{T}}$ ${\mathcal{B}}(X)$ is the set of all scalars $\lambda$ such that $T-\lambda I$ is not invertible. Thus $\lambda\in\sigma(T)$ if and only if at least one of the following two stalements is true: (i) The range of $T-\lambda I$ is not all of $X.$ (i) $T-\lambda I$ is not one-to-one. If (ii) holds, $\lambda$ is said to be an eigenvalue of $T\colon$ the corresponding eigenspace is ${\mathcal{N}}(T-\lambda I);$ each $x\in{\mathcal{N}}(T-\lambda I)$ (except $x=0)$ is an eigenvector of $T\colon$ it satisfies the equation $$ T x=\lambda x. $$ Here atre some very easy facts which wilillustrate these concepts 4,18 Theorem Let $\textstyle X$ and $\boldsymbol{\mathit{I}}$ be Banach spaces (a)I/ T∈ 8(X, Y) and dim ${\mathcal{R}}(T)<\infty,$ then ${\boldsymbol{T}}$ 'is compact. (の)If Te M(X, Y), T is compact, and 4(T) is closed,then dim .W(T)< . (c) The compact operators form a closed subspace of (X, Y), in its norm-topologyDuAurrY IN BANAcu sPACrs 99 (e) ()I Te 3M(X, If $\boldsymbol{\mathsf{S}}$ and ${\boldsymbol{T}}$ are compa opror ro $\left(f\right)$ is trva int ${\cal Y},$ so is $s+T.$ because the $S(U)$ $\sigma(T),$ then ${\mathcal{R}}(T)=X$ compact, and ${\lambda\neq0},$ then dim ${\mathcal{N}}(T-\lambda I)<\infty.$ is complete G) I/ dim $T i s$ $\in{\mathcal{B}}(X),$ and ${\mathbf{}}T$ T is compact, then $0\in\sigma(T).$ ${\mathcal{R}}(T)$ is com- $X=\infty.$ ,T (since $S\in{\mathcal{B}}(X),$ Te ${\mathcal{B}}(X),$ and ${\boldsymbol{T}}$ is compct, so are ${\boldsymbol{S}}{\boldsymbol{T}}$ and TS. ${\mathcal{P}}(T);\operatorname{if}T$ is not in ${\mathbf{}}Y$ Pnoor Statemcn ${\cal Y}.$ Ths A fos trm )an s oes OS oi ${\boldsymbol{T}}$ I'to $\boldsymbol{\mathit{I}}$ is a compact operator $\mathbf{0}$ whose range is $\mathbf{\Psi}(a)$ is obvious.1 ${\mathcal{R}}(T)$ is closed, then onto Put Yis complete) s that ${\mathcal{R}}(T)$ is locly ompact; thus の s aconsuece" pact, it follows that ${\boldsymbol{T}}$ is an open mapping or $\textstyle X$ Thcorem 1.22. $Y=,{\mathcal{N}}(T-{\mathcal{A}}I)$ in (d). The restriction of The roof o compact $T\in{\mathcal{B}}(X).$ sum of any two compac subsets o $\textstyle X$ $T(U)$ is covered by the balls o $S x_{i}$ Sincc Since $\|{\cal S}x-T x\|<r$ raius ar with cetes atepoin $\scriptstyle T x_{i}$ Thus $T(U)$ is compact. It fllowsthat the compact $\Sigma,$ choose $r>0,$ and let $S(U)$ is covrcd by the balls o adiu ${\mathbf{}}T$ ${\cal{Y}}$ To complte th proof of c), we now $\|S-T\|<r.$ show that operators form a subspace of $U,$ it follows that be in the closure of with such that $U$ $\Sigma$ is closed. Let ${\mathcal{B}}(X,Y).$ $X$ There exists ${\mathcal{S}}\in\Sigma$ in $U$ for every xe $T\in{\mathcal{B}}(X,\,Y)$ ${\mathcal{X}}_{1},\ \cdot\ \cdot\cdot\ ,\ {\mathcal{X}}_{n}$ be the open unit ball n is tally bonded,theare poins with ceters at the points proves that $T\in\Sigma.$ is totally bounded, which $$ \begin{array}{r l}{\iiint\!J/J/\,}&{{}\geqslant}&{{}\slash/\slash}\end{array} $$ important role nhisinvstiaton Te mgeioero tg s oihaue ayle etm。o Tesm. nsnis i pesiASjoswu pAay 4.19 Theorem Suppose $\textstyle X$ Y and $\boldsymbol{\mathit{I}}$ are Bach spaces and $T\in{\mathcal{B}}(X)$ Y).Tem ${\mathbf{}}T$ is compact if amd only i $T^{\ast\ast}$ is compact. Define rRoor Suppose ${\boldsymbol{\mathit{I}}}^{\prime}$ is comat t *; s sece n e ui ba $V^{\ast\ast}.$ $$ f_{n}(y)=\zeta y,y_{n}^{\star}\rangle\qquad(y\in Y). $$ Since $|\,f_{n}(y)-f_{n}(y^{\prime})|\,\leq\,\|y-y^{\prime}\|,\{f_{n}\}$ is cquicontinus Since $T(U)$ has compact $(J_{\alpha})$ closure in ${\cal{Y}}$ (as before, $U_{\mathit{l}}$ is the unit ball o $X\}$ ). Ascol's thorem implies tha Since has a subsequenc $\langle f_{n}\rangle$ that converes uniformly on $T(U).$ $$ |\vert T^{\ast\ast}y_{n_{i}}^{\ast}- .T^{\ast}y_{n_{j}}^{\star}\vert\vert\,=\,\mathrm{sup}\ \vert\,\zeta T x,\,y_{n_{i}}^{\ast\ast}-\,y_{n_{j}}^{\star}\rangle\,\vert $$ the supremum being taken over $$ =\operatorname{sup}\,|f_{n_{i}}(T x)-f_{n_{j}}(T x)|\,, $$ implies that {T*>:; convres Henc $x\in U,$ the completeness of $X^{\star}$ $T^{\star}$ is compac.100GENERAL THEoxY The second half can be proved by the same method, but it may be mor Let dp instructive to deduce it from the fist half. and $\textstyle\bigvee\cdot$ $Y\to Y^{**}$ be the isometric embeddings given by the $X\to X^{**}$ formulas $$ \begin{array}{l l}{{\displaystyle{\left\{x,\,x^{*}\right\}}=\displaystyle{\left\{{x^{*},\,\phi x}\right. .}}}&{{\mathrm{~and~}\qquad\displaystyle{\left\{y,\,y^{*}\right\}}=\displaystyle{\left\{y^{*},\,\psi y\right\}},}}\end{array} $$ as in Section 4.5. Then $$ \zeta y^{*},\,\psi T x\rangle=\zeta T x,\,y^{*}\rangle=\langle x,\,T^{*}y^{*}\rangle=\zeta T^{*}y^{*},\,\phi x\rangle=\zeta^{*},\,T^{**}\phi x, $$ for all $x\in X$ and $y^{*}\in Y^{*},$ so that $$ y T=T^{\ ]{\star}}\phi. $$ If $x\in U,$ then dpx lies in the uni bal $U^{\lambda\times8}$ of $X^{\exists{\overset{times}{\land}}}$ Thus $$ \sqrt{T}(U)\subset T^{\bullet\star}(U^{\bullet\star}). $$ compact. Now assume that $T^{\bullet}$ is an isometry $T(U)$ is compact. The first half of the theorem shows tha is subset $\psi T(U).$ Since $\psi$ is compact. Hence $T^{\star\star}(U^{\star\star})$ is totally bounded, and so is its ${\mathbf{}}T$ $J/{\big/}$ $T^{**}\colon X^{**}\to Y^{**}$ is also totally bounded. Hence 4.20 Definition Suppose $\bar{M}$ is a closed subspace of a topological vector space $X,$ $\mathrm{If}$ there exists a closed subspace ${\cal N}$ of $\textstyle X$ such that $$ X=M\ +N\qquad{\mathrm{and}}\qquad M\ ?N\backslash N=\{0\}, $$ then $\bar{M}$ is said to be complemented in $\textstyle X$ In this case,, $\textstyle X$ ris said to be hediret sum of $\bar{M}$ and $N_{\cdot}$ V, and the notation $$ X=M\oplus N $$ is sometimes uscd. We shall sexamples of uncomplemented subspaces in Chapter S.At prcsct we need only the following simplc facts 4.21 Lemma Let M be a closed subspace of atolgicl vector space $X.$ (のf dim $(X/M)<\,o.$ then $\bar{M}$ is complemented in $X.$ then M is complemented in X (a)J Xis locally omex and dim $M<\infty,$ The dimension of $X/M$ is also called thc codimension of $\ M$ in $\textstyle X$ PROOF a) Let {e,.. e,} be a basis fo $M.$ Every $x\in M$ has then a unique representation $x=\alpha_{1}(x)e_{1}+\cdot\cdot\cdot+\alpha_{n}(x)e_{n}$DuAury rs nAACh srcs 101 $\scriptstyle{r>1}$ Foc ${\mathcal{Q}}_{i}$ io ysusu sen tiona $\bar{M}$ GThoe . ictns o be a basis for // $X/M,$ member of pick $x_{i}\in X$ so that he the quoient map, c $\{e_{1},\ldots,e_{n}\}$ be te vctor spa $X^{\bullet}$ *, y he Hahn-BanachtheoremM ie ${\cal X}=M\oplus N.$ be the intersection of the (b) Let r null paces of the xtnsions. Then $\textstyle N$ $\tau\colon X\to X/M$ spanned by $\{x_{1},\ldots$ ,x,}. Ther $\pi x_{i}=e_{i}\left(1\leq i\leq n\right)$ , and let $\boldsymbol{N}$ $\mid X=M\bigoplus N.$ 4.22 Lemma $*\cdot{\begin{array}{l}{v}\\ {v}\end{array}}=\cdot$ i/ M ${\bar{l}}^{\prime}$ snot dense im X, and i U M is subspceo anormed spae $X,$ , hen the exists xe X sich than $$ \left\|x\right\|<r\;\quad\;\;\;b u t\;\quad\;\;\;\;\left\|x-y\right\|\geq1\;\quad\;\;\;f o r\;a l l\;y\in M. $$ PRO0F There exist $x_{1}\in X$ whose distance from $\bar{M}$ is $\dot{\mathbf{I}}$ that is, Choos $y_{1}\in M$ $$ :~~~~\mathrm{inf}~\{|x_{1}-y||:y\in M\}=1. $$ and put $x=x_{1}-y_{1}.$ // such that $\|x_{1}-y_{1}\|<r,$ 4.23 Theorem I广 $X$ X isp Bach space, Te M(X).T s compac, a ${\lambda\neq0}.$ ,.then T - .! as clse ange PROOF By (d) of Theorem 4.18. dim $\mathcal{N}(T-\mathcal{N})<\mathcal{N}.$ By (a) of Lemma 4.21 $\textstyle X$ is the direct sum of ${\mathcal{N}}(T-\lambda I)$ and a closed subspace $M.$ Define an operato Se M(M, X) by (1) $$ S x=T x-\lambda x. $$ Then $\mathbf{S}$ is one-to-one on $M.$ Also, $\mathcal{R}(S)=\mathcal{R}(T-\lambda I).$ To show that M(S s closcd i us o sow he estece a $r\gg\Phi$ such that (2) $$ r\left\|x\right\|\leq\left\|S x\right\|\quad\quad{\mathrm{~for~all~}}x\in M. $$ For if C) holds, and ${\boldsymbol{T}}$ ris used.) t follows that A $T x_{n}\to x_{0}$ for some $x_{0}\in X.$ CThis is where compactness of ItQD ais oreveryr > 0,the $\operatorname{svasts}\left\{x_{n}\right\}$ in A $\bar{M}$ is a Cauchy sequece s sx);thecompetns o $\|x_{n}\|=1,\,S x_{n}\to0,$ M(S) sa consequence $\operatorname{ff}\left\{S x_{n}\right\}$ and after passageto a subsequence) such that $$ \stackrel{\circ}{\sim}\stackrel{\circ}{\sim}\stackrel{\circ}{\sim}\stackrel{\circ}{\sim}\stackrel{\circ}{\sim}\stackrel{\circ}{\sim}\stackrel{\circ}{\sim}\stackrel{\circ}{\sim}\stackrel{\circ}{\sim}\stackrel{\circ}{\sim}\stackrel{\circ}{\sim}\stackrel{\circ}{\sim}\stackrel{\circ}{\sim}\stackrel{\circ}{\sim}\stackrel{\circ}{\sim}\stackrel{\circ}{\sim}\stackrel{\circ}{\times}\stackrel{\circ}{\sim}\stackrel{\circ}{\sim}\stackrel{\circ}{\times}\stackrel{\sim}\stackrel{\sim}{\sim}\stackrel{\sim}\sim}{\sim}^{\circ}\stackrel{\sim}{\sim}{\sim}\stackrel{\sim}{\sim}\stackrel{\sim}{\sim}{\times}^{\sim}\stackrel{\sim}{\Large}{\leq}}^{\sim} $$ Thus $x_{0}\in M,$ and Since $\boldsymbol{\mathsf{S}}$ is one-to-one, $x_{0}=0.$ But $\|x_{n}\|=1$ for all ${\boldsymbol{n}},$ and $x_{0}=\mathrm{i}\mathrm{i}\mathrm{/}\mathrm{/}x_{n}$ ,and so ${}/{\slash{J}}{\boldsymbol{J}}{\boldsymbol{J}}\qquad\qquad.$ $\left\|x_{0}\right\|=\,\left|\,\lambda\,\right|\,>0.$ Ths cotaitonroves 2S or som $\scriptstyle r\gg0$ .24 heormiunpox anan uce. T (X). s cmwe $r>0,$ and $\boldsymbol{E}$ is a s of eienalues ucha 3Then102 GENERAL THEORY (b) (a)for each $\lambda\in E,{\mathcal{R}}(T-\lambda I)\neq X.$ , and ${\boldsymbol{F}}$ is a inite se subspaces $M_{n}$ of We shall first show that if either $\mathbf{\Psi}(a)$ or $\mathbf{\nabla}(b)$ is fase then there exist closed PROOF such that $\textstyle X$ and scalars $\lambda_{n}\in E$ (1) $$ M_{1}\subset M_{2}\subset M_{3}\subset\ ^{..}\cdot\cdot\cdot\cdot,\qquad M_{n}\neq M_{n+1} $$ (2) $$ T(M_{n})\subset M_{n}\qquad{\mathrm{for~}}n\geq1, $$ and (3) $$ (T-\lambda_{n}I)(M_{n})\subset M_{n-1}\qquad{\mathrm{for~}}n\geq2. $$ The proof wil becompleted by showing that this contradictsthc com pactnesso $\textstyle T.$ $\mathbf{\Psi}(a)$ is false.Then ${\mathcal{R}}(T-\lambda_{0}I)=X$ $\mathbf{S}^{n}.$ ·.(See Section 4.17.) Since $\lambda_{0}\in E.$ Put $\scriptstyle S=$ quence $\{x_{n}\}$ in $\textstyle X$ such that for some Then 。 is $T-\lambda_{0}I,$ Suppose ${\boldsymbol{T}}.$ there exists x e M,×1≠0. Since ${\mathcal{R}}(S)=X,$ there is a se $\lambda_{*}{\dot{\mathbf{0}}}$ and define $M_{n}$ ,to be the null space of an eigenvalue of $S x_{n+1}=x_{n}\,,\,n=1,\,2,\,3,\,\ldots\,{.}$ (4) $$ S^{n}x_{n+1}=x_{1}\not\equiv0\qquad\mathrm{but}\qquad S^{n+1}x_{n+1}=S x_{1}=0. $$ of space o $M_{n}$ This gives () 1f is a linearly independent set, so that $M_{n-1}$ lt follows that (I)) o(3) hold. are Hence $M_{n}$ is a proper closed subspace o $M_{n+1}.$ of distinct eigenvalues $\lambda_{n}$ with $\lambda_{n}=\lambda_{0}$ INote that (2) holds becausc $S I=I S.]$ $\{\lambda_{n}\}$ be the (finite-dimen- Suppose (b) s false. Then $\boldsymbol{E}$ contains a sequcncc distinct $\{e_{1},\ldots,e_{n}\}$ Choose corresponding eigenvectors $\textstyle X$ spanned by $\{e_{1},\,\ldots,\,e_{n}\}.$ Since the ${\cal T}.$ ${\mathcal{C}}_{n}$ ,and let $M_{n}$ sional, hence closed) subspace of is a proper sub $x\in M_{n}\,;$ then $$ x=\mathcal{A}_{1}e_{1}+\cdot\cdot\cdot+\mathcal{\alpha}_{n}e_{n}\,, $$ which shows that $T\chi\in M_{n}$ and $$ (T-\lambda_{n}I)x=\alpha_{1}(\lambda_{1}-\lambda_{n})e_{1}+\cdot\cdot\cdot+\alpha_{n-1}(\lambda_{n-1}-\lambda_{n})e_{n-1}\in M_{n-1}. $$ Thus(2) and $(3)$ hold gives Oncc we have closed subspaces $M_{n}$ satisfying $\operatorname{\left(1\right)}$ to (3), Lemma $4.2^{2}$ us vectors $y_{n}\in M_{n}$ ,for $n=2,$ ${\mathfrak{I}},$ 4, $\iota_{1}\iota_{1}\iota_{1}$ such that (5) $$ \|y_{n}\|\leq2\qquad{\mathrm{and}}\qquad\|y_{n}-x\|\geq1\qquad{\mathrm{if}}\qquad x\in M_{n-1}~. $$ lf $2\leq m<n,$ define (6) $$ z=T y_{m}-(T-\lambda_{n}I)y_{n}\,. $$ By G2) and $(3),\,z\in M_{n-1}.$ Hence (5) shows tha $$ \|T y_{i}-T y_{m}\|=\|\lambda_{n}y_{n}-z\|=|\lambda_{n}|\ \|y_{n}-\lambda_{n}^{-1}z\|\geq|\lambda_{n}|\geq r $$DUAull IN BANACH SPACEs 103 The sequence bounded. This is impsible ${\boldsymbol{T}}$ is coipat has thefore no convergent subsequencs alihough $\scriptstyle(y_{n})$ is $(\mathcal{D}_{\nu_{a}})$ // 4.25 Theore Supo $X$ is a uchspace MX).ama ${\boldsymbol{T}}$ is compact (a) $f\lambda\neq0$ , then the four numbers α = dim .W(T"- 入1) β = dim X/W(T - 2D $$ x^{*}=\dim{\mathcal{M}}(T^{*}-{\lambda}I) $$ $$ \beta^{*}=\dim X^{*}/{\mathcal{R}}(T^{*}-\lambda I) $$ are equal and finite (6) $\;U\lambda\neq0$ and ${\mathcal{C}}\in{\mathcal{O}}(T)$ then $\boldsymbol{\lambda}$ is an eigenalue o/ ${\boldsymbol{T}}$ and of $T^{\mathrm{st}}.$ (c) $\sigma(T)$ is comac a mos comal an asa mo one imipoi.namey,,。 both $\textstyle X$ negative integer or the symbol Noue The imnsin vetosae s here uderso o b iher a non is used for the dnty operators o and $X^{\mathrm{A}}:$ ${\cal O}{\cal O}.$ The let $\boldsymbol{\mathit{I}}$ thus $$ (T-\lambda I)^{\star}=T^{\star}-\lambda I^{\star}=T^{\star}-\lambda I, $$ iy (4) below. since the adjoint of theidntity o of $\boldsymbol{\mathit{I}}$ was defiedin Section 4.17. Theorem 4.24 containsa The spectrum $\sigma(T)$ $X$ is theidentity o $X^{\ast}$ special case of $(a)\cdot\beta=0$ implies $x-0.$ Ths wil b uscd in theproof of ue inequa compactness of lt should be noted that $\sigma(T)$ is compact ven if $\boldsymbol{\mathit{I}}$ is mot (Theorem 10.13). The ${\boldsymbol{T}}$ ris needed for the othr asetons in $(c).$ on PROOF Put $S=T-\lambda I,$ to simplify the writing. $y_{1},\cdot\cdot\cdot,y_{i}$ contains $M_{i-1}$ as a proper sub- for all $\therefore\mathop{M}_{0}$ $\boldsymbol{\mathit{I}}$ tinuous linear functionals $\Lambda_{1},\cdot\cdot\cdot,\Lambda_{k}$ on 】 $\boldsymbol{\mathit{I}}$ we bein ih an entay oservtonabouquten spaes Sppos and $\boldsymbol{K}$ is a positive integer such that $k\ll$ dim is a closed subspace of alocally convex space $M_{0}$ and is closed. By Theorem $3.5.$ there are con vector space space. By Theorem 1.42, each $M_{i}$ Then there are vctors ${\boldsymbol{Y}},$ such that the $Y/M_{0}$ $y_{1},$ ,yk in ${\mathbf{}}Y$ $M_{i}$ generated by $y\in M_{i-1}$ that anihilae $M_{0}\,,$ then Y such that $\Lambda_{i}y_{i}=1$ but $\Lambda_{i}y-0$ Thesefuncinas ae inary indendent. h foinociuso is threore eachcd i cnots pao a cmim ymcibona (1) dim YIMo≤ dim $\textstyle\sum_{*}$104 GENERAL THEORY AIso, Apply this with $Y=X.$ $M_{0}={\mathcal{R}}(S).$ By Theorem 4.23,8(S) is closed. $\Sigma={\mathcal R}(\mathbf{S})^{\bot}={\mathcal W}(S^{\star}),$ by Theorem 4.12, so that (1) becomes (2) $$ \beta\leq\alpha^{*}. $$ Next, take $Y=X^{*}$ with its weak*-topology; take $\Sigma$ now consists of all weak*- By Theorem 4.14,92(S*)is weak-closed. Since $M_{0}={\mathcal{R}}(S^{\ast})$ continuous inear functionals on $X^{\star}$ that annihilatc ${\mathcal{R}}(S^{*}),$ $\Sigma$ is isomorphic to 14(S*) = .M(S)(Theorem 4.12), and $\operatorname{\mathcal{(1)}}$ becomes (3) $$ \beta^{*}\leq\alpha. $$ Our next objective is to prove tha (4) $$ x\leq\beta. $$ Once we have (4), the inequalit (5) $$ x^{*}\leq\beta^{*} $$ ${\boldsymbol{X}}$ is also true, since $T^{\rtimes}$ is a compact operator (Theorem 4.19).Since α< co by p, Lemma 4.21 shows that Assume that $(\lambda)$ is false. Then (d of Theorem 4.18,(a is an obvious consequence of the inequalities (2) to (5) Since $x\leqslant\alpha$ and contains closed subspaces $\boldsymbol{E}$ E and ${\mathbf{}}F$ $\alpha>\beta$ $F=\beta$ ' such that dim (6) $$ {\cal X}={\mathcal{W}}(\mathrm{S})\oplus E={\mathcal{M}}(\mathrm{S})\oplus{\cal F}. $$ Every $x\in N$ has a unique representation $\pi x\equiv x_{1}.$ lt is easy to see by the closed graph $x_{2}\in E.$ of M(S) onto ${\mathbf{}}F$ F such that $\phi x_{0}=0$ for some $x-x_{1}+*x_{2}\,,$ with $x_{1}\in{\mathcal{N}}(S),$ $\phi~~~~~~~~~~~~~~~~~~~~~~~~~~~~~~~~~~~~~~~~~~~~~~~~~~~~~~~~~~~~~~~~~~~~~~~~~~~~~~~~~~~~~~~~~~~~~~~~~~~~~~~~~~~~~~~~~~~~~~~~~~~~~~~~~~~~~~~~~~~~~~~~~~~~~~~~~~~~~~~~~~~~~~~~~~~~~~~~~~~~~~~~~~~~~~~~~~~~~~~~~~~~~~~~$ Define $\pi\colon X\to{\mathcal{W}}(S)$ by setting there is a lincar mapping theorem, for instance) that r is continuous. $\mathcal{M}(S)>\mathrm{dim}\;F,$ . Define Since we assume that dim $x_{0}\neq0$ (7) $$ \Phi x=T x+\phi\pi x\qquad(x\in X). $$ Thcn $\Phi\in{\mathcal{B}}(X),$ Since dim ${\mathcal R}(\phi)<\infty,$ pr is a compact operator; hence so is P (Theorem 4.18) Observe tha (8) $$ \Phi-\lambda I=S+\phi\pi. $$ Ssince $x_{0}\in{\mathcal{N}}(S),$ $\pi x_{0}=x_{0}$ $\textstyle{\mathcal{X}}_{0})$ Hence $\phi\pi x_{0}=0$ It follows that $\bar{\lambda}$ is an eigenvalue , and so of $\bar{\Phi}$ p (with cigcnvector (9) $$ {\mathcal{R}}(\Phi-\lambda I)\neq X, $$ by Theorem 4.24. for every $x\in E.$ (8) shows that Since $\pi x=0$ 10) $$ (\Phi-\lambda I)(E)=S(E)=S(X)={\mathcal{R}}(S). $$Durv NBANACH SrAcs 105 lfse.MS), th $\pi x=x,$ and (8) gives (11) $$ (\Phi-\lambda I)(W(S))=\phi(W(S))=F. $$ 1 flows fom (I) and (ID tha (12) $$ {\mathcal{R}}(\Phi-\lambda I)\supset{\mathcal{R}}(S)+F=X. $$ proof of Ga) Thopopoiowe n oswha str nsomae If dim $X<\infty,$ Part G flows fro $\lambda\notin\sigma(T)$ for i $\boldsymbol{\lambda}$ is not an eigenvalue or ${\mathfrak{O}}$ is the ony posible lit is compact. is point of $\sigma(T)_{*}$ implies that $(\alpha),$ that is, that ${\mathcal{R}}(T-\lambda I)=X.$ Thus $T-\lambda I$ / and the theorem. , then $\sigma(T)$ is finte; if dim $X=\infty,$ then $T,$ , then $x(T)=0,$ $\mathbf{\Psi}(\alpha)$ that $\sigma(T)$ is at most countable. and that $\sigma(T)\cup\{0\}$ invetile, So that $\beta(T)=0,$ lt now folow from 0) of Theorem .4 tha Theorem 4.18. Thus $\sigma(T)$ is compact. This gives $0\in\sigma(T),$ by (e) of $\left(c\right)$ and complctes the proof of Exercise explcty stated Throughoutis set of exercis $\textstyle X{\mathrm{~}}\quad$ and ${\mathbf{}}Y$ denote Bah space、ulsthe contary s $\left(f\right)$ 2433 Prove tha $\textstyle X$ 。 be the embedding of ${\bf{}}Y$ is a cosed subspace ot described in Section 4.5. Let $\textstyle X$ is reflexive if and Let $\phi$ ${\mathcal{B}}$ is the losed unt bSaiot into $X^{\models{\2}8}$ $(X,\,\tau)$ onto dens suspace o $(X^{\ast\ast},\,\sigma)$ p(X) of $X^{\mathfrak{s u s}}$ is refexiveani $\textstyle X{\big.}$ prove that $\scriptstyle\phi(B)$ is -dense in the closcd unit ba be the weak (e if topology of $X,$ and le be the wak-oiy $X,$ $\overline{{\tau}}$ $X^{\star}.$ (bf CO Prove th $\phi$ is exve if an only i $X,$ $X^{**}{\longrightarrow}\o{-}\left[h\right]\!\mathrm{e}$ one induced b Y s eflexive is a omcomorpismn of ${\boldsymbol{X}}^{*}+.$ CUe the Hahn-Bah satim teoem prove that $X/Y\,{\mathrm{is}}$ reflexive. (c) Use $(a),$ ,0.an te Banch-loguteorem topove tha $\textstyle X$ only if $\boldsymbol{B}$ is weakly compact $\textstyle X$ MA eco tesyr rno-coseo a aexvespe $X^{\star}$ is cflexivc. m nom“o oesem n p o e o 4 $Y^{\star}$ Which of the spaces $c_{\mathrm{O}:\iota}$ ${\mathcal{C}}^{1},$ $\ell_{{\begin{array}{c}{{\cdot}}\\ {{\cdot}}\end{array}}\ell^{\infty}$ are eiveProveha ey iteaimnsiona is an isomcty o 3 $\Lambda\in X^{*}.$ onto X*. normosoioso… .i an ${\boldsymbol{E}}$ of ${\mathcal{P}}(X,\,Y)$ ${\boldsymbol{C}}$ is th ca hed Hence A e M(C.X")for ever $T^{\bullet\bullet}$ Recall that $X^{*}={\mathcal{R}}(X,$ C'), if Y) is an isometry of $\textstyle X$ Ssimsmiseseisacosp such that lIA $\mathbb{||}\leq M\operatorname{fc}$ br every $\Delta\in E.$ ${\boldsymbol{C}}$ Prove that a subset contiuton.n tniva soev is cqucotinus i andoly i hexis M<。o Prove that Te ldentiy te range or x* onto Yif and only if ${\mathcal{R}}(X,\,1$106 GENERAL THEORY 6 Let c and r be the weak*-topologies of $X^{\mathfrak{s}}$ and $Y^{\aleph_{\bullet}}$ respectively, and prove that $\boldsymbol{\mathsf{S}}$ ' is a continuous linear mapping of $(Y^{*},\tau)$ into $(X^{*},\sigma)$ if and only if $S=T^{*}$ for some $T\in{\mathcal{B}}(X)$ Y). 7 Let $L^{1}$ be the usual space of integrable functions on the closed unit interval ${\boldsymbol{J}},$ relative to Lebesgue measure. Suppose $T\in{\mathcal{B}}(L^{1},$ $Y),$ so that $T^{*}\in{\mathcal{B}}(Y^{*},L^{\infty}).$ Supposc ${\mathcal{R}}(T^{*})$ contains every continuous function on ${\boldsymbol{J}}.$ What can you deduce about $T^{\gamma}$ 8 Prove that $(S T)^{*}=T^{*}S^{*}$ Supply the hypotheses under which this makes sense. 9 Suppose $S\in{\mathcal{D}}(X),$ Te M4(X) (a) Show, by an example, that $S T=I$ does not imply $T S=I.$ (b)However, assume ${\boldsymbol{T}}\,.$ is compact, show that $$ S(I-T)=I\models\mathrm{and~only~if~}(I-T)S=I, $$ 10 Assume $T\in{\mathcal{B}}\left(X\right)$ and show that either of these equalities implies tha $I-(I-T)^{-1}$ is compact is compact, and assume either that dim $\scriptstyle X=\alpha,$ or that the scalar field is ${\boldsymbol{C}}.$ Prove that $\sigma(T)$ is not empty. However, $\sigma(t)$ may be empty if dim $X<\infty$ and the scalar field is ${\boldsymbol{R}}$ ${\boldsymbol{\mathit{1}}}$ Suppose dim $X\leqslant\alpha$ and show that the equality $\alpha=\beta$ of Theorem 4.25 reduces to the statement that the row rank of a square matrix is equal to its column rank. ${\mathit{1}}_{}^{}$ Suppose $T\in{\mathcal{B}}(X)$ Y) and ${\mathcal{B}}(T)$ is closed in ${\cal{Y}}.$ Prove that $$ \mathrm{dim}\ {\mathcal W}(T)=\mathrm{dim}\ X^{\star}/{\mathcal R}(T^{\ast}), $$ $$ \dim{\mathcal W}(T^{*})=\dim\;Y/\mathcal R(T)\,. $$ This generalizes the assertions $x-\beta^{\star}$ and $\alpha^{\star}=\beta$ of Theorem 4.25. ${\boldsymbol{J}}{\boldsymbol{S}}$ (a) Suppose $T\in{\mathcal{B}}(X)$ Y), $T_{n}\in{\mathcal{B}}(X,\,Y)$ for $\scriptstyle{m=1{}}$ ,2, ${\mathfrak{J}}_{\mathfrak{p}}\ \ldots{\mathfrak{L}}$ each $\textstyle T_{n}$ has finite-dimen- $T\in{\mathcal{B}}(X)$ sional rangc, and lim $\|T-T_{n}\|=0$ Prove that ${\mathbf{}}T$ is compact (b)Assume ${\cal{Y}}$ is a Hilbert space, and prove the converse of (a: Every compact Y) can be approximated in the operator norm by opcrators with finite- dimensional ranges. Hint: In a Hilbert space there are linear projections of norm 1 onto any closed subspace.(See Theorems 5.16, 12.4.) $I d$ Define a shift operator $\boldsymbol{\mathsf{S}}$ and a multiplication operator $\mathcal{M}$ on ${\mathcal{E}}^{2}$ by $$ (S x)(n)={\binom{0}{x(n-1)}}\quad{\mathrm{~if~}}n=0, $$ $$ (M x)(n)=(n+1)^{-1}x(n)\qquad{\mathrm{if~}}n\geq0 $$ Put $T=M S$ . Show that $\boldsymbol{\mathit{I}}$ is a compact operator which has no eigenvalue and whose spectrum consists of exactly one point. Compute $||T^{n}||_{0}$ for $n=1,\,2_{;}\,3,\,*,$ and compute ${\mathcal{L}}{\mathcal{S}}$ $\textstyle|{\mathrm{i}}\mathbf{p}_{n\to\infty}\ ||I^{n}||^{1/n}.$ s a finite (or c-finite) positive measure on a measure space Q, $K\in L^{2}(\mu\times\mu)$ Define is the Suppose ${\boldsymbol{\mu}}$ , and $\textstyle\prime^{4}\times\mu$ corresponding product measure on $\Omega\times\Omega.$ (Tf)(s) K(s,t)/(t)du(t) $[f\in L^{2}(\mu)].$DuAurry N ANACH sepAcs 107 (a) Prove that $T\in{\mathcal{B}}(L^{2}(\mu))$ and that $$ \|T\|^{2}\leq\iint_{\mathrm{\scriptsize{fii}}}\mid K(s,t)\mid^{2}d\mu(s)\,d\mu(t). $$ (b) Suppose $a_{i}\,,\,b_{i}$ are members o ${\boldsymbol{T}}$ 'was defined in terms of $1\leq i\leq n.$ , put $K_{1}(s,t)=\sum a_{i}(s)b_{i}(t),$ and (c) Deduce uhat ${\mathbf{}}T$ in terms of $K_{1}$ as $L^{2}(\mu),$ for $L^{2}(\mu).$ Hin: Use Exercise 13 ${\mathcal{R}}(T_{1})\leq n.$ define $T_{1}$ is a compact operator on $K.$ Prove that dim (d) Suppose $\lambda\in{\mathbb{C}},\lambda\neq0$ Prove: Either the equation 16 Define has a niqe soluti $$ T f-\lambda f=g $$ or there are infniely many (e) Decribe he ajoint ${\boldsymbol{T}},$ $f\in L^{2}(\mu)$ for every $g\in L^{2}(\mu)$ pounsosonponcKsnonowsesnismsism and define $T\in{\mathcal{P}}(L^{2}(0,1)),$ ) by $$ K(\stackrel{.}{s},t)= \{\stackrel{(1-s)t}{(1-t)s}\ \ \ \ {\mathrm{if~}}s\leq t\leq s\leq1. $$ $$ (T f)(s)=\int_{o}^{1}\!K(s,t)f(t)\,d t\qquad(0\leq s\leq1). $$ 17 f $\left(a\right)$ Show that the cigenvalues or $\scriptstyle{T={\frac{1}{2}}}$ 入impis t inil ifetai t that the corresponding ${\mathrm{If~}}$ (d) Show that ${\boldsymbol{T}}\,\cdot$ ${\mathbf{}}T$ are $(n\pi)^{-2},\;t$ 1 = 1,2, ${\mathfrak{g}},\ldots$ $\lambda f^{\prime\prime}+f=0,$ $\lambda\neq0,$ the equation sienfoioss szxsntu aeaecisacasis einisosnr nn and that $f(0)=f(1)=0$ The casc $\scriptstyle\lambda-0$ can be treated separately $L^{2}(0,1)$ (c) Supposc ${\mathfrak{g}}(t)=\sum$ A2 souetne ae encintoisramntoiasof $I f-\lambda f=a.$ $L^{2}=L^{2}(0,$ e,sn. Dis tequatio the space of all continuous functions is aso compat oeaor o ${\cal{C}},$ :" 9: 1 uiuoneaiei Ss ecmuo co eatv to Lemesue ani $$ (T f)(s)-\frac{1}{s}\int_{o}^{s}\!f(t)\,d t\qquad(0<s<\infty), $$ $I{\mathcal{S}}$ $\mathbf{\Psi}(c)$ (O I is eflexvean $T\in{\mathcal{B}}(c_{0}),$ and that Tis not ompact.(CThc act ha then $\{||x_{n}|\}\}$ is boundcd is a special case prove that $T\in{\mathcal{P}}(L^{2})$ if $x_{n}\to X$ weaky, and f ris opact th $X,$ weakly. $\|T\|\leq2$ whenever e of Exercis $\scriptstyle{\mathfrak{f}}$ of arsys calySe "2 iSa) weakly, then $T x_{n}\to T x$ $\|T x_{n}-T x\|\to0.$ of Exercise (c) If $\textstyle X$ Prove te flowin tatemenh Y) and $x_{n}\to x$ ${\mathbf{3}}.$ $\|T x_{n}-T x\|\to0$ $\mathbf{\Psi}(c)$ (b) If $T\in{\mathcal{B}}(X)$ 2以i waky cnvetseunci ${\mathcal{X}}_{n} arrow{\mathcal{X}}$ $T\in{\mathcal{B}}(X,\;Y),$ is refexv, it Te.M8(X, Y), and i (d) Conversely, i $\textstyle X$ 28 in Chaptcr 3 Yot omoa i oes AadBon of Chapter i esego X.C .n sompat ec (7P+ : m: o Y), thn Tiscompact108 GENERAL THFORY ${\mathcal{L}}{\mathcal{G}}$ Suppose ${\mathbf{}}Y$ is a closed subspace of $X,$ and $x_{5}^{k}\in X^{*}$ . Put $$ \begin{array}{r}{\mu=\operatorname*{sup}\left\{|\zeta x,x{\stackrel{\ *}{\6}}\right\}|:x\in Y,\;||x||\leq1 \},}\\ {\delta=\operatorname*{inf}\left\{|x^{*}-x{\frac{\\|}{\sqrt{}}}|\!:x^{*}\in Y^{1}\right\}.}\end{array} $$ x to the annihilator of ${\boldsymbol{Y}}.$ is the norm of the restriction of xtt0 $\textstyle Y,$ and $\bar{\delta}$ is the distance from for at In other words, ${\boldsymbol{\mu}}$ Prove that $\mu=\delta,$ Prove also that $\delta=\|x^{*}-x_{0}^{*}\|$ $2{\cal I}$ Let least one $x^{*}\in Y^{1}.$ be the closed unit balls in ${\boldsymbol{E}}$ ;is weak*-closed.(Corollary: $\mathbf{A}$ subspacc of $X^{\ast}$ 20 $\boldsymbol{B}$ and $B^{*}$ $\textstyle X$ and $X^{\star},$ is a comvex set in $X^{\bullet}$ (The word “isometric” must of $E\cap(r B^{*})$ Extend Sections 4.6 t0 4. ocally convex spaces、 $\scriptstyle y.E$ such that course b deleted from the statement of Thcorem 4.9.) respectively. The following is a converse of the Banach-Alaoglu theorem: is weak*-compact for every $\scriptstyle\gamma\geqslant0.$ then is weak*-compact.) is wvak-loedifand ony i titescion wit $B^{\lambda}$ Complete the following outline of the proo (i) ${\boldsymbol{E}}$ zis norm-close its polar Gi) Associated toeach $\scriptstyle{F\in X}$ $$ P(F)=\{x^{*}\colon\left|\,\langle x,\,x^{*}\rangle\right|\leq1\,{\mathrm{for~all~}}x\in F\}. $$ The intersection of all st $P(I),$ as ${\mathbf{}}F$ Franges ove th olecto lfite subset of $\scriptstyle{r^{*}R_{s}}$ is exactly $r B^{\mathrm{s}}.$ then there exists $x\in{\mathcal{X}}$ such that Re ${\boldsymbol{J}},$ in addition to the stated hypotheses, in) The uheoem is a cosequence of th following propositio $\ <x,x^{*}\rangle\geq1\,f o r$ $E\cap B^{*}=\mathcal{C},$ every $x^{*}\in E$ $F_{0}=\{0\}$ Assume finite sets $F_{0}\,,\,\cdot\cdot\cdot,\,F_{k-1}$ have been (iD) Proof of the proposition: Put and so that chosen so that $i F_{t}\subseteq B$ (1) $$ P(F_{0})\cap\cdots\cap P(F_{k-1})\cap E\cap k B^{*}=\emptyset. $$ Note that (I) is true for $\scriptstyle k=1$ . Put $$ Q=P(F_{0})\cap\cdots\cap P(F_{k-1})\cap E\cap(k\setminus1)B^{k}. $$ If $P(F)\cap Q\neq{\mathcal{D}}$ for every finite set $F\subset k^{-1}B,$ the weak*-compactness of ${\cal Q},$ is a finite set togcther with Gi), implies tha $(k B^{\star})\cap Q\neq\emptyset,$ which contradicts $\mathbf{(1)}$ Hence there $F_{k}\subset k^{-1}B$ such that (l holds with $k+1$ in place of k. The constructin can thus proceed. It yiclds (2) $$ E\cap\bigcap_{k=1}^{\infty}P(F_{k})=\emptyset. $$ Arrange the membes or U Fin sequence x. Tn Ix.一→0. Definc $T\colon X^{\bullet}\to c_{\bullet}$ by $$ T x^{*}=\{(x_{n},x^{*})\}. $$ Then TGE) is a convex subset of co.By C2) $$ \|T x^{*}\|=\operatorname*{sup}\;|\langle x_{n},\,x^{*}\rangle|\geq1 $$Durr i BANacn srocs 109 for every $x^{*}\in E.$ Hecete sasec ,. i $\dot{\mathbf{Z}}$ $\left|\,\alpha_{n}\right|\,<\,\infty_{},$ ,such that $$ \mathrm{Re}\sum_{n=1}^{\infty}\alpha_{n}\zeta x_{n}\,,\,x^{\star}\rangle\leq $$ 22 uppos for every $x^{*}\in E.$ To complete theroo. pu and $S=T-\lambda I.$ $T\in{\mathcal{B}}(X),$ Tis compact $x=\sum\alpha_{n}x_{n}$ $\lambda\neq0_{!}$ (a) If ${\mathcal{N}}(S^{n})=:{\mathcal{N}}(S^{n+1})$ for some nonneative inte ${\boldsymbol{\eta}},$ , prove that $\mathcal{A}^{\prime}(S^{n})=\mathcal{A}(S^{n+k})$ ${\mathcal{W}}(S^{n})$ for $k=1,2,3,\dots.$ is fnte, that 8 xgpigsousoanosm/ oaepeoro 2 6 smsmmsmemissrereosns $$ X=\mathcal{N}(S^{n})\bigoplus\mathcal{M}(S^{n}), $$ $2{\dot{3}}$ and that th restiction or $\boldsymbol{\mathsf{S}}$ to ${\mathcal{R}}(S^{n})$ is a nctioe mappin ${\mathcal{R}}(S^{n})$ onto ${\mathcal{R}}(S^{n}).$ Supps x. s suce nach spac $X_{\mathrm{\mathrm{\mathrm{:}}}}$ , and $$ \sum_{n=1}^{\infty}\|x_{n}\|=M<\infty. $$ Prove that the series $\overline{{\land}}$ x.coneres to some $x\in X$ Explicitly rove that $$ \operatorname*{lim}_{n\to\infty}||x-(x_{1}+\ast\cdot\dots+x_{n})||=0. $$ Prove also that $||x||\leq M.$ GThes as e s ro of emma 4.13 24 Lct c e e aco l olesouc $$ x-\{x_{1},\,x_{2},\,x_{3},\,\dots; $$ for which $x_{\infty}=\operatorname*{lim}x_{n}\exp x_{15}$ sts Gn C). Put $||x||=\operatorname*{sup}\,|x_{n}|$ . Let co be the subspace of c that consists of all $\textstyle|\ X|$ x with $x_{\alpha}=0.$ $\mathbf{\Psi}(c)$ and $\boldsymbol{v}$ maps ${\mathcal{C}}_{0}^{\mathrm{ss}}$ onto ${\mathcal{L}}^{1}.$ o Dscreitowomeiri soris an Describete oerato $v S^{*}u^{-1}$ that maps $\ell^{1}$ to ${\mathcal{F}}^{1}$ onto ${\mathcal{I}}$ $v_{\mathrm{{J}}}$ such that u maps $C^{\mathbb{S}}$ (b)Define $S\colon c_{0} arrow c$ by $S f=f.$ Define $T\colon C\to c_{0}$ by setting $$ y_{1}=x_{\infty}\,,\quad y_{n+1}=x_{n}-x_{\infty}\qquad{\mathrm{if}}\,n\geq1. $$ Prove that ${\mathbf{}}T$ is onc-to-one and hat ${\mathcal{I}}^{1}$ to ${\mathcal{E}}^{1}.$ . Find $|T||$ and $|T^{-1}|$ Describe the operator $u f^{*}v^{-1}$ that maps $I c=c_{0}$