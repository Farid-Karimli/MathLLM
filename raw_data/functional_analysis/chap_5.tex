5 SOME APPLICATIONS A Continuity Theorem one f thevey ealy tems n functionalanlysis Hlngerand Toeplitz, 1910 which ismmetri te ses statat T s aiearopearo ailber spc $\textstyle H$ that $$ (T x,\,y)=(x,\,T y) $$ $\textstyle H$ for al $x\in H$ and $y\in H,$ then ${\mathbf{}}T$ 'is continuous. Here $(x,\,y)$ denotes the usual Hilbert $T x_{n}\to0$ space inner product.(See Section 12.1.) such that $\|x_{n}\|\to0,$ the symmetry of ${\boldsymbol{T}}$ implics that $\mathbf{I}{\hat{\mathbf{I}}}$ {x,} is a sequence in ${\mathbf{}}H$ wcakty. CThis dend n knowing hatal coniuous inar fntionals o are given by inner products.)The Helinger-Toelitz theorem istherefore a con- sequence of the following one. 5.1 Theorem Suppose $\textstyle X$ and $\boldsymbol{\mathit{I}}$ are F-spaces, $Y^{\rtimes}$ sepates oinson Y ,T:X→Y is linear, and ATX。→0 for every Ae Y* whenever $x_{n}\to0.$ Then T is continuous.SOME APPLICATIONs 111 PROOF Suppose $x_{n}\to x$ and $T x_{n} arrow y.$ If $\Lambda\in Y^{*}.$ , then so that $$ \Lambda T(x_{n}-x) arrow0 $$ $$ \Lambda y=\operatorname*{lim}\,\Lambda T x_{n}=\Lambda T x. $$ Consequently $y=T x,$ and the closed graph theorem can be applied. $////\ /$ T: irplies that $\|T x_{n}\|\to0.$ is linear,if ILx,-→0 implies tha In the context of Banach spaces, Theorem 5.l can be stated as follows: weakly, then $\|x_{n}\|\to0$ actually $X\to Y$ $T x_{n}\to0$ To see that complctcns is important here, le $f(0)=f(1)=0,$ put be the vector space of al $\textstyle{X}$ complex polynomials f such that $$ \cdot(f,\,g)=\int_{0}^{1}\!f\bar{g},\qquad\|f\|=(f,f)^{1/2}, $$ uous and define T: $X\to X$ by $(T f)(x)=i f^{\prime}(x).$ Then (T $f,g)=(f,T g),$ but ${\cal T}\,$ is not contin- Closed Subspaces of ${\bar{D}}^{\prime}.$ P-spaces theorem. Theproof te foowing theorem of Grotendiek as invovs thclosed graph 5.2 Theorem Suppose $0<p<\infty,$ and (6) (a)uis a probabiliy measure on u measure space $\Omega.$ s is a closed subspace o/ L”(uD (C) $S\subset L^{\infty}(\mu).$ Then S is finite-dimensional. PRoor Let j be the identity ap that takes $\mathbf{S}$ into ${\cal L}_{\l}^{\l\alpha}:$ where $\boldsymbol{\mathsf{S}}$ is given the $L^{p}.$ -topology, so that $\boldsymbol{\mathsf{S}}$ is complete. If $\{f_{\mathrm{{a}}}\}$ } is a sequence in $\boldsymbol{S}$ S such $\operatorname{that}f_{n}\to f\operatorname{in}\,S$ such that ${\mathrm{and}}\,f_{n}\to g$ in I $I^{\alpha},$ it is obvious $\operatorname{that}f=g$ a.c. Hence jsatisfies the hypotheses of $K<\infty$ uhe closed graphtheorem and we conclde ththe is constan (1) $$ \|f\|_{\ _{\sigma}}\,\leq\,K\|f\|_{p} $$ for al $f\in S.$ As usual, $\|f\|_{p}$ "then $\|f\|_{p}\leq\|f\|_{2}$ 1 $2<p<\infty,$ integration of supremum of $|{\mathcal{I}}|$ If ${\mathfrak{p}}\leq2$ means $(\int\mid f\mid^{p}d\mu)^{1/p},$ and $\|f\|_{\infty}$ is the essentia the inequality $|f|^{p}\leq\|f\|_{\infty}^{p-2}|f|^{2}$112 GENERAL THFoRy leads to $\|f\|_{\alpha}\leq K^{p/2}\|f\|_{2}$ . In either case, we have a constant $M<\infty$ such that (2) $$ \|f\|_{\infty}\leq M\|f\|_{2}\qquad(f\in S). $$ In the rest of the proof we shall deal with individual functions, not with equivalcncc classes modulo null sets. Let ${\mathsf{L e t}}\left\{\phi_{1},\ \ldots,\ \phi_{n}\right\}$ be an orthonormal set in ${\boldsymbol{S}},$ regarded as a subspace of of ${\mathcal{C}}^{n},$ ”.f $\scriptstyle{c=}$ $L^{2}.$ $Q_{1}={\frac{1}{\sqrt{1}}}\,{\frac{\sqrt{1}}{\sqrt{1}}}\,{\frac{\sqrt{1}}{\sqrt{1}}}\,$ be a countable dense subset of the euclidean unit ball $\boldsymbol{B}$ ${\mathcal Q}\,$ every $\left(c_{1},\ \cdot\cdot\cdot\ ,\ c_{n}\right)\in B,$ define $f_{c}=\sum c_{i}\phi_{i}.$ $\Omega^{\prime}.$ Integration of this inequality gives $\|f_{c}\|_{\infty}\leq M.$ Sincc function on ${\boldsymbol{B}}.$ and for every $x\in\Omega^{\prime}$ If Then $\|f_{c}\|_{2}\leq1,$ and so for is countable, there is a se $\Omega^{\prime}\subset\Omega,$ with $\mu(\Omega^{\prime})=1,$ such that $|f_{c}(x)|\leq M$ $c\in Q$ Hence $|f_{c}(x)|\leq M$ whenever $c\in{\mathcal{B}}$ and $c\to{\big|}f_{c}(x){\big|}$ is a continuous $\textstyle{\mathcal{X}}$ is fixed, $x\in\Omega^{\prime}.$ It follows that $\sum\left|\,\phi_{i}(x)\!\right|^{2}\leq M^{2}$ for every xe $n\leq M^{2}.$ We conclude that dim $S\leq M^{2}.$ This proves the theorem. $J/f$ It is crucial in this theorem that $L^{\infty}$ occurs in the hypothesis $\left(c\right)$ .To illustrate this we will now construct an infinite-dimensional closed subspace of ${\boldsymbol{L}}^{1}$ l which lies in $L^{4}$ For our probability measure we take Lebesgue measure on the unit circle, divided by 2元. 5.3 Theorem Let ${\boldsymbol{E}}$ be an infinite set of integers such that no integer has more than one representation as a sum of two members of E. Let F $\textstyle P_{E}$ zbe the vector space of all fnite sums f of the form (1) $$ f(c^{i\theta})=\sum_{n=-\infty}^{\infty}c(n)c^{i n\theta} $$ in which $c(n)=0$ whenever $r_{\mathit{l}}$ is not in ${\boldsymbol{E}}.$ Let $\textstyle S_{E}$ be the ${\mathcal{L}}^{1}.$ -closure of $\scriptstyle{P_{E}}$ .Then ${\boldsymbol{S}}_{E}$ is a closed subspace $v f^{2}$ An example of such a set is furnished by 2', $K=1,$ 2,3,..…. Much slower growth can also be achicved PROOF If fis as in(I), then $$ f^{2}(e^{i\theta})=\sum_{n}c(n)^{2}e^{2i n\theta}+\sum_{n\neq m}c(n)c(m)e^{i(n+m)\theta}. $$ Our combinatorial hypothesis about ${\boldsymbol{E}}$ implies that $$ \int|\,f\,|^{4}=\int|\,f^{2}\,|^{2}\not=\sum_{n}\,|\,c(n)|^{4}\,+\,4\sum_{m<n}|\,c(m)|^{2}\,|\,c(n)|^{2} $$ so that $(2)$ $$ \begin{array}{r l}{\int|f|^{4}\!\leq2(\sum\,|c(n)|^{2})^{2}=2\Biggl(\int|f|^{2}\Biggr)^{2}.}\end{array} $$sOME APT CATIONS 113 Holder's inequality, with ${\mathbf{j}}\,$ and $\frac{3}{2}$ as conjugate exponents, gives (3) $$ \int|\,{\cal f}\,|^{2}\leq\left(\int|f|^{4}\right)^{1/3}\left(\int|\,\}f|\,\right)^{2/3} $$ lt folows from (2) and (3) that (4) $$ .^{*}\:\mathbf{\delta_{\cdot}}_{\cdot\cdot\cdot,\cdot,\cdot\cdot\cdot\cdot\cdot\cdot}\mathbb{I}||_{4}\leq2^{1/4}||f||_{2}\quad\quad\mathrm{and\qquad}\|f\|_{2}\leq2^{1/2}\|f\|_{1} $$ for evry $f\in P_{E}\,,$ Every $\angle^{1}.$ -Cauchy sequence in $\textstyle P_{E}$ is therefore also a Cauchy $l/{\big/}$ that sequence in L*. Hence $\mathbf{\hat{S}}_{E}\subset L^{4}$ The obvious inequality $\|f\|_{1}\leq$ l fl then shows $\mathbf{S}_{E}$ is closed in $L^{4}$ restriction of $\hat{g}$ G to $\textstyle E.$ An interetingresult an b otane apling duality aruent to the se of every $g\in L^{\infty}$ satisfy $\textstyle\sum\left|{\hat{g}}(n)\right|^{2}<\infty.$ ond inequality (4)、Recall that the Fourier ceficins $\scriptstyle{\hat{g}}(n)$ The next theorem shows tha nothing more can be said about th 5.4 Theorem 1 $\boldsymbol{E}$ is as in Theorem 5.3 and i $$ \sum_{i=o}^{\infty}\;|a(n)|^{2}=A^{2}<\infty $$ then there exists $g\in L^{\infty}$ such that $\hat{g}(n)=a(n)f o r\;e v e r y\;n\in E.$ PRoOr $\operatorname{If}f\in P_{E},$ the preceding proof shows that $$ \left|\sum\hat{f}(n)a(n)\right|\leq A\left\{\sum|\hat{f}(n)|^{2}\right\}^{1/2}=A\|f\|_{2}\leq2^{1/2}A\|f\|_{1}. $$ the linear extension to ${\cal L}^{1}.$ Hence f→艺fn)an is a linear functional on $P_{\scriptscriptstyle{E}}$ which is continuous relative to such that $L^{1}$ Hence there cxists '-norm. By the Hahn-Banach theorem, ths funciona has a continuou (with $\|g\|_{\infty}\leq2^{1/2}A)$ $g\in L^{\infty}$ $$ \sum_{-\infty}^{\infty}\hat{J}(n)a(n)=\frac{1}{2\pi}\int_{-\pi}^{\pi}f(e^{-i\theta})g(e^{i\theta})\,d\theta\qquad(f\in P_{E}). $$ $\mathrm{With}\,f(e^{\imath\theta})=e^{\imath n\theta}\,(n\in{\cal E}).$ this shows that ${\hat{g}}(n)=a(n).$ // A: The Rang of a Vector-valued Measure Banach-Alaoglu. We now give a rathr string appication of he theorems of Krein-Milman and Let 9DR be a o-algebra. A real-valued measure $\lambda$ on Ot is said to be nonatomic if every sct $E\in{\mathfrak{M}}$ with $|\lambda|(E)>0$ contains a set A ∈ JDt with $\textstyle0<\,|\lambda|(A)<\,|\lambda|(E).$ Here 1| denotes the total variation measure of $\lambda_{\,;}$ the terminology is as in [23]114 GENERAI. THEORY 5.5 Theorem Suppose $\mu_{1},\,\cdot\cdot\,.\,$ Hp,are real-valued nonatomic measures on a o-algebra Dn. Define $$ \mu(E)=(\mu_{1}(E),\ldots,\mu_{n}(E))\quad.\quad(E\in\mathfrak{M}). $$ Then $\boldsymbol{\mu}$ is a function with domain DR whose range i $\dot{\mathit{i}}S$ s a compact convex subset of ${\boldsymbol{R}}^{n}$ PRor Associate to each bounded mcasurable real function $\mathbf{\Omega}^{g}$ the vector $$ \Lambda g=\left(\int g\,d\mu_{1}\,,\,.\,.\,.\,,\int g\,d\mu_{n}\right) $$ in $R^{n}.$ Put $(1\leq i\leq n)$ 。 $\sigma=|\mu_{1}|\,+\cdots+|\mu_{n}|.$ If $g_{1}=g_{2}\,\mathrm{a.e.}$ Io], then $\Lambda g_{1}=\Lambda\,g_{2}\,.$ .Hence $h_{i}$ do Each $\textstyle{\mathit{\nabla}}\mu_{i}$ A may be regarded as a linear mapping of $L^{\infty}(\sigma)$ into $\textstyle R^{n}.$ The Radon-Nikodym $d\mu_{i}=$ is absolutely continuous with respect to ${\boldsymbol{\sigma}}\,.$ Hence theorem [23] shows therefore that there are functions $h_{i}\in L^{1}(\sigma)$ such that into $\Lambda$ is a weak*-continuous linear mapping of $L^{\infty}(\sigma)$ $R^{\mathrm{o}};$ recall that $L^{\infty}(\sigma)=L^{1}(\sigma)^{\bullet}$ Put $$ K=\{g\in L^{\infty}(\sigma):0\leq g\leq1\}. $$ It is obvious that $K$ is convex. Since $g\in K$ if and only if $$ 0\leq\int\!f g~d\sigma\leq\int\!f\,d\sigma $$ unit ball of for every nonnegative f∈ $:L^{1}(\sigma)$ ). Kis weak*-closed. And since Klies in the closed < is weak*-compact $L^{\infty}(\sigma).$ ), he Banach-Alaoglu theorem shows that $\textstyle K$ K Hence $\Delta(K)$ is a compact convex set in $\textstyle R^{n}\!,$ $\Lambda g$ and define We shall prove that $\mu({\mathfrak{M}})=\Lambda(K).$ $E\in{\mathfrak{M}},$ then $\scriptstyle\chi_{t}\in K$ and $\mu(E)=$ If $\chi_{E}$ is the characteristic function of a set $p\in\Lambda(K)$ Thus $\mu({\mathfrak{M}})\subset{\Lambda}(K).$ To obtain the opposite inclusion, pick a poin $$ K_{p}=\{g\in K\colon\Lambda g=p\}. $$ exists We have to show that $K_{p}$ is convex; since $\mathbb{A}$ is continuous, $K_{p}$ is wcak*-compact. By Then $\textstyle E.$ $K_{\!_{P}}$ contains some $\chi_{E}\,.$ 、 for then $p=\mu(E).$ Put Note that and ${\mathcal{I}}_{0}$ is not a characteristic function in and $r\leq g_{0}\leq1-r$ $L^{\infty}(\sigma)$ on the Krein-Milman theorem, $K_{p}$ has an extrcmc point. Suppose $g_{\mathrm{o}}\in K_{p}$ there is a set $E\in{\mathfrak{M}}$ and an $\scriptstyle{r\gg0}$ such that $\sigma(E)>0$ $\ Y>n$ Hence there $Y=\chi_{E}\cdot L^{\alpha}(\sigma)$ Since $\sigma(E)>0$ and $\textstyle{\boldsymbol{\sigma}}$ is nonatomic, dim and such that $g\in Y$ , not the zero element of $L^{\omega}(\sigma),$ such that $\Lambda g=0,$ extremepoint o $K_{p}$ It follows that $g_{0}+g$ and $g_{0}-g$ are in $K_{p}$ Thus ${\mathcal{G}}_{0}$ is not an $J/{\big/}$ $-r<g<r.$ Every extreme point of $K_{\!_{p}}$ is therefore $\scriptstyle{\dot{\alpha}}$ characteristic function. This completes the proof.sOME APPuICATIoNs 115 A Generalized Stone-weierstras Teorem appied to an approximation problem Tho homso K cniMlmanlana n nachlogu wul now b metric which consists of the restrictions $f|_{x}$ .Dfnitons Let CS t tmiar up-ome ac spe o coni A set $E<S$ is said to be $\textstyle A$ -antisym- is an algebra if fg e A whencver $f_{\in{\mathcal{A}}}$ uouspope ucnton ne compat asriseSsbsbpac $g\in A$ and i $\textstyle A$ cons o l/e CSy uat ar or $C(S)$ ${\cal A}_{E}$ $\textstyle{A}$ metrie very e A which eal o and ${\boldsymbol{C}}$ in other words, the algebra holomorphc n the interior of $\mathbf{S},$ ${\boldsymbol{E}}$ s consant on $E;$ $\textstyle{\mathcal{A}}$ to $\boldsymbol{E}$ contains no non- constant rcal function .of the functons Fe For xmple,i $\boldsymbol{\mathsf{S}}$ is opat st then evey coponent ot einio SsSa-imisy Suppose $A\subset C(S),\,p\in S,\,q\in S,$ and write ${\boldsymbol{q}}.$ . Itis easiyverified tat this efines a $A.$ -anti- symmetric set ${\boldsymbol{E}}$ which contans both L $p\sim q$ provided that there is an $D\!\!\!\!/$ p and squivaceuti an tachetiviecess s iSsecqu alenc clasare the maximal Aanisymetie set set $E.$ 5.7Bihot theoem Let $\dot{A}$ he leselaler CS wii oaisn constat fuctions.Snppoe $g\in C(S)$ and $g\mid_{E}\in A_{E}/o r$ every maximal A-antisymmetric Then g ∈ A. set ${\boldsymbol{E}}$ Statiesey opotsis $\scriptstyle{\mathcal{G}}$ is ha vry maxima on $F;$ iecisis istu cornsqu nctin whic cocieswi ${\bar{A}}\cdotp$ anusmme one exs h doesthisfor vey ${\underline{{E}}},$ namne $\mathcal{G}$ ${\mathfrak{f}}={\mathfrak{g}}$ A special cae o Bishop's therem s the stone eistas theorem points on S ${\boldsymbol{S}},$ lf Aisa closed subalgebra of ${\mathsf{C}}(S)$ whicl conains the costants which separate then $A-C(S).$ al whch sef-doitatis whenene F ${\mathcal{A}}),$ ${\mathcal{A}}.$ For i i saeae mers $\textstyle{A}$ sepat pints o $\mathbf{S}.$ Since no canismti set oninsteforemre than oneoint.eve $g\in C(S)$ satisfes the hypotes or Bshop's theiore $\boldsymbol{\mu}$ on rxoor Theannhilat ${A}^{\frac{n}{2}}\circ\mathbb{F}\ A$ consis fal regular complex Borel measures $\boldsymbol{\mathsf{S}}$ such $\mathrm{that}\ J f d\mu=0$ for every /e A. Define where $\left\vert~~~~~~\right\vert~~~~~~~~~~~~~~~~~~~~~~~~~~~~~~~~~~~~~~~~~~~~~~~~~~~~~~~~~~~~~~~~~~~~~~~~~~~~~~~~~~~~~~~~~~~~~~~~~~~~~~~~~~~~~~~~~~~~~~~~~~~~~~~~~~~~~~~~~~~~~~~~~~~~~$ $$ K=\{\mu\in A^{\perp}\colon\|\mu\|\leq1\}, $$ hence $A=C(S),$ and there is nothing to prove $\|\mu\|=|\mu|(S).$ Then $K$ Kis convex, balanced, and weak*-comact, by Ge of Theorem 4.3.if $K=\{0\},$ then $A^{\perp}=\{0\};$116 GENERAL TEORY $$ \begin{array}{c c c c c c c c c c c c c c c c c}{{}}&{{}}&{{}}&{{}}&{{}}&{{}}&{{}}&{{}}&{{}}&{{}}&{{}}&{{}}&{{}}&{{}}&{{}}&{{}}&{{}}&{{}}&{{}}&{{}}&{{}}&{{}}&{{}}&{{}}&{{}}&{{}}&{{}}&{{}}&{{}}&{{}}&{{}}&{{}}&{{}}&{{}}&{{}}&{{}}&{{}}&{{}}&{{}}&{{}}&{{}}&{{}}&{{}}&{{}}&{{}}&{{}}&{{}}&{{}}&{{}}&{{}}&{{}}&{{}}&{{}}&{{}}&{{}}&{{}}&{{}}&{{}}&{{}}&{{}}&{{{}}X{{}}&{{}}&{{}}&{{}}&{}&{{}}X X X X}&{X}&{{{0}}&{{0}}&{}&{{{0}}&{X}&{{0}&{X}&{X}&{{0}&{0 $$ $f(x)<1$ E bethe support f ;this means that ${\boldsymbol{E}}\,$ be an cxtreme point of $K.$ Clearly, $\|{\boldsymbol{\mu}}\|=1$ . Let ${\boldsymbol{E}}$ Assume $K\neq\{0\},$ and $\mathrm{let}\ \mu$ is compact, that $|\mu|(E)=||\mu||$ ,and that $\mathfrak{o}:$ for every $x\in E,$ and defne fstemswi i swpeeose a sch $$ d\sigma=f d\mu,\qquad d\tau=(1-f)\,d\mu. $$ Since $\scriptstyle A$ is an algebra, $\sigma\in A^{\prime}$ and $\tau\in A^{\perp}.$ Since O $<f<1$ on $E,$ l $\sigma\|>0$ and $\|{\boldsymbol{\tau}}\|>0$ AIso, $$ \|\sigma\|\,+\,\|\tau\|=\int_{E}f\,d|\,\mu|\,+\,\int_{E}\left(1-f\right)d|\,\mu|\,=\,|\,\mu|(E)=1. $$ words, This shows that $\boldsymbol{\mu}$ Both of these arc in $K.$ Since $\boldsymbol{\mu}$ for every $x\in E$ Since $\textstyle A$ $\sigma_{1}=\sigma/\lVert\sigma\rVert$ and $\tau_{1}=\tau/\|\tau\|.$ constants, it follows that $\operatorname{every}f\in A$ is a convex combination of the mcasure is extreme in $K,\,\mu=\sigma_{1}.$ In other ${\boldsymbol{\omega}}{\boldsymbol{n}}$ $f\,d\mu=\left\|\sigma\right\|\,d\mu,$ which is real on $\boldsymbol{\mathit{L}}$ is constant on $\textstyle E.$ contains the so $\mathrm{\that}\,f(x)=\left\|\sigma\right\|$ So far we have proved that the support of u is $A^{\cdot}$ cantisymmetric if p is every $\boldsymbol{\mu}$ extreme point of Satisesthe hypotes o te theorem i oows h in the convex ul ofthe extreme $\{\,g\,d\mu=0$ for $K.$ $\scriptstyle{\mathcal{G}}$ Hence $g\in A,$ hence for eve $\boldsymbol{\mu}$ $K_{\!_{J}}$ the Krein- $l//\rangle$ annihilates ${\mathcal{G}}.$ that is extreme in $K,$ is a weak*-continuous function on hence for every $\mu\in A^{\perp}.$ points. Since $\mu\to\textstyle\int g~d\mu$ $\a{array}{l}{g~d\mu=0~\mathrm{for~every~}\mu\in{\cal K},}\end{array}$ that annihilates A also Milman theorem implies that $\ C(s)$ Thus every continuous linear functional on by the Hahn-Banach separation theorem Here is anexample hat ilustates Bisho's theorem 5.8 Theorem Suppose (a) Kis a comac subset of $R^{n}\times{\mathcal{C}}$ and (b)if $r_{t}=(t_{1},\dots,t_{n})\in R^{n},$ the set $$ K_{t}=\{z\in{\mathcal{C}}\colon(t,z)\subset K\} $$ does not separate ${\bar{C}}.$ If $g\in C(K),$ define ${\mathfrak{g}}_{t}$ , on K $\zeta_{t}\;b y\;g_{t}(z)=g(t,z)$ Assume that $g\in C(K).$ that each ${\mathcal{G}}_{t}$ is holomorphic im the interior of $K_{t}$ and tha e > 0. Then there is a polynomial ${\boldsymbol{P}}$ in the variables $t_{1},\ldots\cdot r_{n},\cdot$ z such that $$ \left|P(t,z)-g(t,z)\right|<s $$ for every $(t,z)\in K.$ PROOF Let $\textstyle A$ be the closure $\operatorname{in}^{*}C(K)$ of the se fall poyomials P(,2.Since the real polynomials on $R^{n}$ Sspaeions evry Aantismei st liesisoME APPLICATIONs 117 Some $\scriptstyle{K_{t}\cdot}$ By Theorem 5. t s tefore enough to show that to ery $\scriptstyle t\in R^{*}$ that corresponds an fe A such $\mathrm{that}f_{t}=g_{t}$ such ${\mathrm{Fix}}\ t\in R^{n}.$ By Merelyan's theorem [23] thereare polnomia $\scriptstyle{p_{i}c\rangle}$ $$ g_{t}(z)=\sum_{i=1}^{\infty}\,P_{i}(z)\qquad(z\in K_{t}) $$ and $|P_{i}|<2^{-i}$ if i> 1. There is a polynomial if s ${\boldsymbol{\tau}}\in I$ and $K_{s}\neq\varnothing$ on ${\boldsymbol{R}}^{n}$ that peaks at ${\bar{I}},$ in the The functions $\phi_{m}\,$ defined on $\textstyle K$ ${\cal Q}\,$ Consider a fixed ${}^{t}>1$ sense that $Q(t)=1$ but $|Q(s)|<1$ by $$ \phi_{m}(s,z)=\mid Q^{m}(s)P_{i}(z)\mid $$ is $<2^{-i}\,\varepsilon$ it every point of $K.$ Since forma moicaly es scuce of contnu nctios wose lin $K,$ The series integer m, such that $\phi_{m_{i}}(s$ $z)<2^{-i}$ $\textstyle K$ Kis compact it flows that there s a positiv at every point of $$ f(s,z)=\sum_{i=1}^{\infty}Q^{m}(s)P_{i}(z) $$ .converges uniformly on $K.$ Hence fe ${\mathfrak{z}}\,A_{\mathfrak{z}}$ and obviously $f_{t}=g_{t}$ $J/\hbar$ Two Interpolation Theorems rem 5.7. The pro tet tse thomsinove e aoit noperato.T hescon ${\boldsymbol{C}}(s)$ Our notation s as in Theo_ furnis aotrapictio he K intitheiorce The frst one du to Bishop again concen and $|\mu|(K)=0$ for every Thcorem Supose Y isa lose subspace o 1f $g\in C(K)$ and $|g|<1.$ ,it follows that there exists ${\boldsymbol{S}},$ 5.9 $C(S)_{i}$ $K$ is a compact subset of fe $\boldsymbol{\mathit{I}}$ suchthan J|x =g ad $\mu\in Y^{1}.$ on $\mathbf{S}.$ $|f|<1$ $\boldsymbol{\mathit{I}}$ of th unit dis $U$ in $\textstyle{\mathcal{C}}$ Thus every continuous function on maps $\boldsymbol{\mathit{I}}$ Y onto $C(K).$ $U.$ Take $S=T,$ the unit circle. Let measure the restriction map $f\to f|_{K}$ is a closed subspace of CT)、 f $K\subset T$ rextends to a member of Y. In other words, satisfies the $\textstyle K$ $t h a t f=g$ on $K.$ This theorem generalizesth followig specal case of the members of ${\bar{A}}.$ By the maximum modulus consist of the restrictions to T ${\boldsymbol{T}}$ Lct A be the dis lez, . te ofal conius ucins o the closur which are holomorphic n theorcm, $\boldsymbol{\mathit{I}}$ is compact and has Lebesgue $\textstyle K$ $\mathbf{0},$ the thcorem of F. and M. Ricsz 【23 tates pecisey tha hypotsis of Theorm 5. Cosequenly o evey gecKx evesisamFe A suc118 GENERAL THEoRv $\|\mu\|=|\mu|(K).$ same norm. In other words, there exists be the restriction map defined by $\rho f=f{\big|}_{K}$ We have $Y{\dot{\boldsymbol{I}}}$ PROOF Let $\rho\colon Y\to C(K)$ $\rho^{*}\colon M(K)\to Y^{*},$ where $M(K)=C(K)^{\ast}$ is the Banach of the to prove that $~\rho$ maps the open unit ball of ${\cal{Y}}$ onto the open unit ball o ${\boldsymbol{C}}({\boldsymbol{R}})$ Consider the adjoint with the total variation norm $C(S),$ space of all regular complex Borel measures on $K,$ with $\|\sigma\|=\|\rho^{*}\mu\|_{*}$ such that For each $\mu\in M(K),\ \rho^{*}$ u is a bounded lincar functional on by the Hahn-Banach theorem, $p^{*}p$ extends to a linear functional on $\sigma\in M(S),$ $$ \left.\int_{S}f\,d\sigma=\left\{f,\,\rho^{*}\mu\right\}=\zeta\rho f,\,\mu\right\}=\int_{K}f\,d\mu $$ for every $\scriptstyle f\in Y$ Regard $\boldsymbol{\mu}$ as a member of $M(S),$ with support in $K.$ Then Borel set $\scriptstyle{E\subset K}$ Hence and our hypothesis about $K$ implies that $\sigma(E)=\mu(E)$ for every :7/ $\sigma-\mu\in Y^{\perp},$ $\|\mu\|\leq\|\sigma\|$ We conclude that $\|\mu\|\leq\|\rho^{*}\mu\|.$ By (b) of Lemma 4.13, hs inequality proves the theorem Note: Since $\|\rho^{*}\|=\|\rho\|\leq1,$ we also have $\|\sigma\|\leq\|\mu\|$ in the preceding proot. It follows that $\sigma=\mu.$ Hence $\rho^{*}\mu$ has a unique norm-preserving extension to CS) Oursconteropoathoremcocens nte laschke pus .,functio $\boldsymbol{B}$ of the form $$ B(z)=c\prod_{k=1}^{N}\,{\frac{z-\alpha_{k}}{1-\bar{\alpha}_{k}z}}, $$ such that $|f(z)|<1$ for all $|\alpha_{k}|<1{\mathrm{~for~}}1\leq k\leq N$ lt s easy to se that the finite Blaschke $U,$ where $|c|=1$ and all ofabsolutevle less than 1, with every point of the unit circle prodctsar resthosemes o te isc algebra whose absolut vluc isl a The data of the Pick-evalinainerolaion problem are two finite sets o complex numbers. $\{z_{0}\,,\,\ast\,,\,z_{n}\}$ and $\{w_{0\cdot{\cdot}}\cdot\cdot{\cdot}\cdot{\cdot}w_{n}\}_{\cdot}$ $z_{i}\neq z_{j}{\mathrm{~if~}}i\neq j$ $z\in U,$ The probem s fn a holomorphic function fnthe oen unit dis and such that $$ f(z_{i})=w_{i}\qquad(0\leq i\leq n). $$ $\{w_{0},\,w_{1}\}=\{0,\textstyle{\frac{3}{2}}\},$ The data may verywell amit no solution. For example, i $\{z_{0},\,z_{1}\}=\{0,\frac{1}{2}\}$ and the Schwarz lemma shows ths. Bu if th problem has solution ienaongthem iemus be om vry nicones. The extheorem hows thi 5.10 Theorem Let $\{z_{0}\,,\,\cdot\cdot\cdot,\,z_{n}\}_{:}$ {w。.…. w,}be Pick-Nealima data. Le $\boldsymbol{E}$ be the set ofall holomorphic fumctions in U such that $|f|<1\;a n d f(z_{i})=w_{i}f o r\;0\leq i\leq n.$ If E is not empty, then E containsa fie Blaschke produc.soME APPLICATIONs 119 poor Wihoutos o geraity asu ${\boldsymbol{F}}$ in $U$ which satisfies $z_{0}=w_{0}=0.$ Wewill show that there is a holomorphic function (1) $$ {\mathrm{Re}}\,F(z)>0\qquad{\mathrm{for}}\,z\in U,F(0)=1, $$ $\left(2\right)$ $$ F(z_{i})=\beta_{i}=\frac{1+w_{i}}{1-w_{i}}\mathrm{~~~~for~}1\le i\le n, $$ and which has the form (3) $$ F(z)=\sum_{k=1}^{N}c_{k}{\frac{a_{k}+z}{a_{k}-z}}, $$ where $c_{k}>0,$ $\textstyle\sum c_{k}=1$ and $|a_{k}|=1$ Once such an ${\mathbf{}}F$ is found, put $\scriptstyle B\,=$ $(F-1)/(F+1)$ Tis is fnie Baschke prodictatatise $B(z_{i})=w_{i}$ for ${\mathfrak{o}}\leq i\leq n.$ Let $\textstyle K$ K be the set ofal holomorphic functions ${\mathbf{}}F$ in $U$ that satisfy (1) Associate to each $\mu\in M(T)=C(T)^{*}$ the function (4) $$ F_{\mu}(z)=\int_{-\pi}^{\pi}\frac{e^{i\theta}+z}{e^{i\theta}-z}\,d\mu(e^{i\theta})\qquad(z\in U). $$ $\operatorname{tr}P$ is th t tral Bore robiymeasures o $K.$ (Theorems 1.12 and i1.19 o 23]) Define is a one-to-one A: correspondence between ${\boldsymbol{P}}$ and ${\boldsymbol{T}},$ then $\mu rightarrow F_{\mu}$ $M(T)\to C^{n}$ by (5) $$ \Lambda\mu=(F_{\mu}(z_{1}),\,.\,.\,.\,,\,F_{\mu}(z_{n})){\mathrm{.}} $$ Since ${\boldsymbol{E}}$ is assumed to be nonempty thexs $\mu_{0}\in P$ such that (6) $$ \Lambda\mu_{0}=\beta=(\beta_{1},\,\ast\,,\,\beta_{n}). $$ ous, $\Lambda({\boldsymbol{P}})$ and positive numbers ${\boldsymbol{c}}_{k}$ is cvwxsexe-.-mce sasn $C^{n}=R^{2n}$ Since $\beta\in\Lambda(P),$ $\beta$ is a convex of ${\boldsymbol{P}}$ $\scriptstyle\mathbf{F}\ ,$ Since F ${\boldsymbol{P}}$ combination of $N\leq2n+1$ extreme points of $\Lambda(P).$ (Exercise i in s a eoni Chapter 3.) is a convex compact set n $\mathbb{A}$ is an extreme point of $\Lambda(P),$ then $19_{\mathrm{:}}$ $K,$ and every is an extreme point o $D.$ with $\textstyle\sum c_{k}=1.$ $\Lambda^{-1}(\gamma)$ is an cxtrene set of $\mu_{1},\,\cdot\,\cdot\,\cdot\,\cdot\,,\,\mu_{N}$ extreme point of $\Lambda^{-1}(\gamma)$ theirxise ofowsfrom th Krein-Mimantcocm 1t fllow tat the are extreme poins such that (7) $$ \Delta(c_{1}\mu_{1}+\cdot\cdot\cdot+c_{N}\mu_{N})=\beta. $$ $a_{k}\in T$ Being an extreme point ot ${\boldsymbol{P}},$ each $\textstyle\mu_{k}$ that occusin(T has a single point for itsupport; hence (8) $$ \cdot F_{\mu_{k}}(z)={\frac{a_{k}+z}{a_{k}-z}}. $$ If ${\boldsymbol{F}}$ is nowene , fosro an 8 ta ${\mathbf{}}F$ sais an e $l/{\big/}$120 GENERAL THEoRv A Fixed Point Theorem Fixed point theorems play an important role in many parts of analysis and topology. The one that we shall now prove is due to Kakutani; it will be used to prove the exis- tence of a Haar measure on any compact group. The proof of Kakutani's thcorcm involves only the most basic properties of locally convex spaces. 5.11 Theorem Suppose (a)Kis a nonempty compact comvex set in a locally conex space $X,$ (b) G is an equicontinuous group of linear mappings of $\textstyle X$ onto $X,$ and (c) A(K) c K for every $\Lambda\in G$ for every Then G has a common fixed point in 1 $K;$ that $I S_{*}$ there exists ${\mathfrak{p}}\in K$ such that $\mathrm{A}p=p$ $\Lambda\in{\bar{G}}.$ Part (b) of the hypothesis should perhaps be made more explicit. Equicontinuity is defined in Section 2.3. To say that G ${\boldsymbol{G}}$ F is a group means that every $\mathrm{A}\in G$ is a one-to- one mapping of $\textstyle X$ Y onto $\textstyle X$ whose inverse $\Lambda^{-1}$ also belongs to ${\boldsymbol{\mathsf{O}}}$ and that $\Lambda_{1}\Lambda_{2}\in G$ whenever $\Lambda_{i}\in G\ (i=1,$ 2). Here $(\Lambda_{1}\Lambda_{2})x=\Lambda_{1}(\Lambda_{2}\,x),$ of course. Hypothesis (b) is satisfied, for instance, when ${\boldsymbol{G}}$ is a group of linear isometries on a normed space X that $\Lambda(H)\subset H$ for every $\Lambda\in{\bar{G}}$ 2 be the collection of all nonempty compact convex sets $H\subset K_{\mathrm{SUCB}}$ PROOF Let $\mathbb{Q}$ . Partially order $\underline{{\mathbf{Q}}}$ by set inclusion. Note tihat $\Omega\neq{\emptyset},$ since totally ordered subcollection $\Omega_{0}$ By Hausdorff's maximality theorem, Q contains a maximal o of all members of $\Omega_{0}$ is a $K\in\Omega$ . The intersection $\textstyle H_{0}$ minimal member of Q. The theorem will be proved by showing that $\textstyle H_{0}$ contains only onc point. To do this, wc shall considcr a sct $H\in\Omega$ which contains at lcast two points, and we shall prove that some $H_{1}\in\Omega$ is a proper subset of $H.$ Before doing this, we prove that $\textstyle X$ has a local base consisting of balanced convex sets $U$ that satisfy $\Lambda(U)\subset U$ for every $\Lambda\in G$ Since ${\boldsymbol{G}}$ is equicontinuous, Let ${\mathbf{}}V$ be a convex neighborhood of O $\mathbf{0}$ in $X\colon$ there is a balanced neighborhood $V_{1}$ of O such that $\Lambda(V_{\mid})<V$ for every $\Lambda\in G.$ Let $U_{\mathbf{\delta}}U$ be the convex hull of the union of all sets since ${\mathbf{}}V$ is convex. Every $\mathrm{\A}$ ranges over ${\cal G}.$ Then $U$ is convex and balanced, and $\Lambda({\mathcal{V}}_{1}),$ as has the $U\subset V,$ $u\in U$ form $$ u=c_{1}\Lambda_{1}v_{1}\ +\cdot\cdot\cdot+c_{n}\Lambda_{n}v_{n}, $$ where $c_{i}\geq0,\,\Sigma\,c_{i}=1,\,\Lambda_{i}\in G,$ D;∈ $V_{1}.$ If $\Lambda\propto G,$ then $\Lambda u=c_{1}\Lambda\Lambda_{1}v_{1}+\cdots+c_{n}\Lambda\Lambda_{n}v_{n}$ lies also in $U,$ because AA;e G. Hencc A(U) c UsoME APPLICATIONs 121 s. Then Now suppose $H\in\Omega,$ and $\textstyle H$ fcontains at least two points. Then Since $H-H$ is compact, $H-H\subset s U$ for some $\mathbf{s}>0$ as above fail to cover $H-H\neq$ {0)}, and some set $U$ $H-H$ $\scriptstyle{t\geq1}$ Put $W=t U.$ Let ${\mathbf{}}I$ be the greatest lower bound of these numhers Then $\mathcal{W}$ is a convex balanced open set such that (1) $$ \begin{array}{r l}{\Lambda(W)\subset W}&{{}{\mathrm{for~every~A\inG,}}\\ {H-H\subset(1+r)W}&{{\mathrm{if~}}r>0,}\\ {\left\{1-r\right\}W\ d\mathrm{oes~not~cover}~H-H}&{{\mathrm{if~}}0<r<1}\end{array} $$ (2) (3) Properties (l and (2) are obvious. Since ${\mathcal{W}}$ is convex $$ (1-r)\overline{{{W}}}\subset(1-r)\mathcal{W}+\underline{{{1}}}r\mathcal{W}=\left(1-\frac{r}{2}\right)\mathcal{W}; $$ Since this last set does not cover $H{\boldsymbol{-}}\ H{\boldsymbol{\cdot}}$ hence (3) holds such that $\textstyle{H}$ lis compact, ${\boldsymbol{H}}$ contains points $X_{1},\,\ast\cdot,\,x_{n}$ (4) $$ H\subset_{i=1}^{n}\;(x_{i}+{\frac{1}{2}}W).\qquad\qquad\qquad. $$ ·Put $r=1/(4n),$ and defin ((5) $$ H_{1}=H\cap\bigcap_{y\in H}(y+(1-r)\overrightarrow{W}). $$ It is clear that $\textstyle H_{1}$ is compact and convex. By ( Suppose $x\in H,$ and $\operatorname{v\inH}$ Since $\Lambda^{-1}(H)\subset H,$ $y=\wedge y_{i}$ for some $y_{i}\in H.$ $5),\;x\in y_{1}+(1\div r)\overrightarrow{W}$ V Hence (l implies that $$ \Lambda x\in\Lambda y_{1}+(1-r)\Lambda(W)\subset y+(1-r)\overline{{{W}}}. $$ lt follows that $\Lambda(H_{1})\subset H_{1}$ for every $\mathrm{A\inG}.$ such that $x-y$ docs not lic in showing that $\textstyle H_{1}$ By 3)、 there are points $x\in H,\;y\in H,$ $H_{1}\neq H.$ $H_{1}\neq\varnothing.$ we do this by $(1\,-\,r)\,{\overline{{W}}}.$ Any such $\textstyle{\mathcal{X}}$ is not in $\textstyle H_{1}.$ Thus To complete the proof, we have to show that contains the point (6) $$ \cdot{\cal X}_{0}=\frac{1}{n}\,(x_{1}+\cdot\cdot\cdot+x_{n}). $$ Since ${\cal H}$ is convex $x_{0}\in H.$ $\operatorname{Fray}\in H.$ By (4), thcrc exis such that $\scriptstyle{\frac{\lambda^{*}\cdot\cdot\cdot\cdot^{*}}{x^{*}}}{\overset{\cdot\cdot\cdot\cdot}{\underset{\mu^{*}}{\overset{\cdot\cdot}{\underset{\mu^{*}}}{}}{\boldsymbol{\varepsilon}}^{*}}}.$ (7) $$ \nu\in x_{j}+{\frac{1}{2}}W. $$ lf i≠j,1<i≤n, property (2) implies tha (8) $y\in x_{i}+(1\ +r)W.$122 GENERAL THEORY Add the relations (T) and (8), divide by ${\boldsymbol{n}}_{\mathrm{{J}}}$ and use the convexity of ${\mathcal{W}}$ to obtain $$ y-x_{0}\in\frac{1}{n}\biggl[\frac{1}{2}+(n-1)(1+r)\biggr]W\subset(1-r)W, $$ since $r=1/(4n)$ Thus $x_{0}\in y+(1-r)W,$ for every $y\in H.$ Hence $x_{0}\in H_{1},$ and the proof is complete. $I I J$ Haar Measure on Compact Groups 5.12 Definitions A topological group is a group ${\boldsymbol{G}}$ in which a topology is defined that makes the group operations continuous. The most concise way to express this requirement is to postulate the continuity of the mapping $\phi~~~~~~~~~~~~~~~~~~~~~~~~~~~~~~~~~~~~~~~~~~~~~~~~~~~~~~~~~~~~~~~~~~~~~~~~~~~~~~~~~~~~~~~~~~~~~~~~~~~~~~~~~~~~~~~~~~~~~~~~~~~~~~~~~~~~~~~~~~~~~~~~~~~~~~~~~~~~~~~~~~~~~~~~~~~~~~~~~~~~~~~~~~~~~~~~~~~~~~~~~~~~~~~~~~~~~~~~~~~~~~~~~~~~~~~~~~~~~~~~~~~~~~~~~~~~~~~~~~~~~~~~~~~~~~~~~~~~~~~~$ $G\times G\to G$ defined by $$ \phi(x,\,y)=x y^{-1}. $$ onto ${\hat{\boldsymbol{S}}}\colon$ For each $a\in G,$ the mappings $x\to a x$ ${\boldsymbol{\mathsf{J}}}$ is therefore completely determined by any ${\boldsymbol{G}}$ and $x\to x a$ are homeomorphisms of ; so is $X\to X^{-1}.$ The topology of local base at the identity element e. If we require (as we shall from now on) that every poimt of G is a closed set, then the analogues of Theorems 1.10 to 1.12 hold (with exactly the same proofs, except for changes in notation); in particular, the Hausdorff separation axiom holds. its left translates $\ I_{x}J$ rand its righ translates If f is any function with domain $G_{\mathrm{{J}}}$ R,f are defined, for every $s\in G.$ by ( $$ L_{s}f)(x)=f(s x),\qquad(R_{s}f)(x)=f(x s)\qquad(x\in G). $$ corresponds a neighborhood ${\mathbf{}}V$ of $\textstyle{\mathcal{C}}$ in $\dot{\boldsymbol{\jmath}}$ such that is said to be uniformly coninuous if to every $\scriptstyle{\pi\gg0}$ A complex function $\boldsymbol{\mathit{f}}$ on ${\boldsymbol{G}}$ $$ \left|f(t)-f(s)\right|<\varepsilon $$ whenever $s\in G,\;t\in G,$ and $\,\,e^{-1}t\in V$ A topological group $\ {\bar{G}}$ whose topology is compact is called a compact group;in this case, ${\mathit{C}}(G)$ is,as usual, the Banach spac of all complex continuous functions on ${\cal G},$ with the supremum norm 5.13 3Theorem Let ${\boldsymbol{G}}$ be a compact group,suppose fe $C({\mathcal{O}}),$ and define HLf) to be the convex hull of the set ofall left translates of f. Then (の) is uniformly contimuous, and $(b)\ \ H_{L}(f)$ is a totally bounded subset of $C({\mathcal{O}}).$ In other words, the closure of H,CD in C(G) is compact. (Appendix A4.soMEAPPLICArIONs 123 PR00F Fix ${\mathfrak{s}}>0.$ Since is coninuos the corresponds to eac $V_{a}$ of $\scriptstyle{\mathcal{C}}$ that satisfy $V_{a}V_{a}^{-1}\subset W_{a}.$ borhood ${\mathcal{Y}}_{a}$ is compact, the is a finite s $A\subset G$ such that $\alpha\in G$ a neigh- Since ${\boldsymbol{G}}$ of e such that $|f(t)-f(a)|<s\mathrm{~for~all~}t\ln\,a W_{a}.$ The continuity of the group operations gives neighborhoods $$ \begin{array}{c c c c c c c c c c c c c}{{\cdot}}&{{\quad}}&{{\quad}}&{{}}&{{G=\bigcup_{a\in A}a V_{a}.}}&{{}}&{{}}&{{}}&{{}}&{{G=\bigcup_{a\in A}a V_{a}.}}\\ {{\quad}}&{{}}&{{\quad}}&{{\ddots}}&{{}}&{{}}&{{}}&{{}}&{{}}&{{}}&{{}}&{{}}\\ {{}}&{{}}&{{}}&{{}}&{{}}&{{}}&{{{}}}&{{{\quad}}&{{}}&{{}}&{{{\quad}}\\ {{}}}&{{}}&{{}}&{{}}&{{}}&{{{\quad}}&{{{\quad}}&{{}}\\ {{}}}&{{}}&{{}}&{{}}&{{}}&{{}}&{{{}}&{{{}}}&{{}}&{{{}}\quad}}&{{{}}&{{}}={{{{}_{a\epsilon_{{}}}}}}}\end{array} $$ Also, Assume $\displaystyle x^{-1}y\in V$ . Choose $a\in A$ so that $y\in a V_{a}$ Then |fGy) -f(a| <。 $|f(x)-f(a)|<c,$ because $$ .\qquad x\in y V^{-1}\subset a V_{a}V^{-1}\subset a W_{a}. $$ Hence $|f(x)-f(y)|$ < 20. This proves (a) it follows that Since $(s x)^{-1}(s y)=x^{-1}y$ for every $s\epsilon\,G.$ $$ \left|(L_{s}f)(x)-(L_{s}f)(y)\right|=\left|f(s x)-f(s y)\right|<2e $$ whenever $x^{-1}\nu\in V.$ Every $g\in H_{i}(f)$ is a finite sum of the form $\sum c_{s}L_{s}f,$ with $c_{s}\geq0,\sum^{c_{s}-1.}$ Hence $$ \left|\,g(x)-\,g(y)\,\right|\,<2\epsilon $$ if $x^{-1}y\ c\ y\ a n d\ g\in I I_{L}(f)$ This proves ta $H_{L}(f)$ is an equicontinuous subset $///\hbar$ of CO. Now b fows fom scoi tormM.(AppenixXAS 5.14 Theorem On every compact group ${\boldsymbol{G}}$ exists a unique regular Borel probabilit measure m which is lef-imarion, m the sese tha (1) $$ \textstyle\int_{G}f\,d m=\int_{G}(L_{s}f)\,d m\qquad[s\in G,f\in C(G)]. $$ This m is also right-invariant: (2) $$ \int_{G}f\,d m=\int_{G}(R_{s}f)\,d m\qquad[s\in G,f\in C(G)] $$ and $i{\dot{t}}$ satisfes he relation $\scriptstyle{\lambda^{\dagger}\begin{array}{l}{{\lambda^{\cdot}}}\\ {{\dagger}\end{array}}_{-}$ (3) $$ \int_{G}f(x)\,d m(x)=\int_{G}f(x^{-1})\,d m(x)\qquad[f\in C(G)]. $$ This ${\mathfrak{m}}$ is cald the Har measure o ${\widetilde{O}}.$ PROor The operators $:L_{s}\operatorname{satisfy}L_{s}L_{t}=L_{t s},$ because $(L_{s}L_{t}f)(x)=(L_{t}f)(s x)=f(t s x)=(L_{t s}f)(x).$124 GENERAL Tmroxy the definition of $\textstyle K_{I},$ is an isomctry of $\ C(G)$ ont itself $\{L_{s}\colon s\in G\}$ is an equicontinuous $H_{L}(f)$ Since each $L_{s}$ $K_{f}$ -is compact. Itis obvious that $\textstyle|f\in C(G),$ let $K_{f}$ be the closure of $s\in G$ $L_{s}\phi=\phi$ group of linear operators on ${\cal G}.$ In particular, $\phi(s)=\phi(e),$ so that $\phi$ is constant. By By Theorem 5.13, $c(\sigma)$ $L_{s}(K_{f})=K_{f}$ for every such tha The fixed point theorem 5.l1 now implis that $K_{\!_{J}}$ contains a function $\phi$ for every s ∈ ,.thiscsnsn can be uniforyappoxmatd b nctin in $H_{L}(f),$ $\alpha_{i}>0,$ $\beta_{j}>0,$ with $\textstyle\sum\alpha_{i}=1=\sum\beta_{j}\,,$ corresponds at least one con in ${\cal G}_{;}$ F, and thcre exist numbers So far we have proved that to each fe $s>0,$ There exist finite sets ${\mathit{C U G}}$ by convex combinations of To prove this, pick stant c which can be uniformy approximated on which bears the same relation ${\boldsymbol{G}}$ Teft tasates ot . Likewise there s onstant ${\boldsymbol{C}}^{\prime}$ and $(b_{j})$ to th right transltes of . We claim that $c^{\prime}=c$ $\textstyle|a_{i}\rangle$ , such that (4) $$ |c-\sum_{i}\alpha_{i}f(a_{i}\,x)|<\varepsilon\qquad(x\in G) $$ and (5) $$ \left|c^{\prime}-\sum_{j}\beta_{j}f(x b_{j})\right|<\varepsilon\qquad(x\in G). $$ Putx= b in (4),multiply 4A) by $\beta_{j},$ and add with respect toi. The result i (6) $$ |c-\sum_{i,j}\alpha_{i}\beta_{j}f(a_{i}b_{j})|<\kappa. $$ Put $x=a_{i}$ in (S), multiply (5) by ${\mathcal{Q}}_{i}\,,$ and add with respect to i,to obtain (7) $$ |c^{\prime}-\sum_{i,j}a_{i}\beta_{j}f(a_{i}b_{j})|<\varepsilon. $$ shall write Now (G) and (T) imply that $c=c^{\prime}.$ corresponds a unigue number, which w ${\boldsymbol{f}}.$ The It follows that to Wwhchca be nioryapoimated by convx combinain $\operatorname{cach}f\in C(G)$ ${\mathcal{M}};$ ${\mathcal{f}};$ the same ${\mathcal{N}}$ Ts as the unique numbr that can be uni of left translates of are obvious: Formiy apoxmated by convex cominations ofrih traslates o following proprties of $\mathcal{M}$ (9) $$ \begin{array}{r l}{{M f\geq0\qquad}}&{{\mathrm{if~}}f\geq0.}\\ {{M1=1.}}&{{\qquad}}\\ {{M(\alpha f)=\alpha M f\qquad}{\mathrm{~if~}}\alpha{\mathrm{~is~a~scalar.}}}\\ {{M(L_{s}f)=M f=M(R_{s}f)\qquad}}&{{\mathrm{for~every~}s\in G.}}\end{array} $$ (8) (10) (11) we now prove that ${\mathrm{(L)}}$ $$ M(f+g)=M f+M g. $$soME APPLICATIONs 125 Pick 8> 0. Then (13) $$ \left|\ M f-\sum_{i}\alpha_{i}f(a_{i}x)\right|<c\qquad(x\in G) $$ for some finite s $\{a_{i}\}\subset G$ and for some numbers $x_{i}>0$ with $\sum x_{i}=1$ Define (14) $^{r}\,-\,{}^{\circ}$ $$ h(x)=\sum_{i}\alpha_{i}g(a_{i}x). $$ Then $h\in K_{g}.$ hence $K_{h}\subseteq K_{g},$ and since each of these sets contains a uigue con , and there are numbers $\beta_{j}>0$ with $\sum\beta_{j}=1,$ such that Hence there isa finite sc $\{b_{j}\}\subset G_{i}$ stant function, we have $M h=M g.$ (15) $$ \left|M g-\sum_{j}\beta_{j}h(b_{j}x)\right|<\varepsilon\qquad(x\in G); $$ by (14), this gives $\mathbf{\vec{e}}$ (16) $$ \Big\lfloor M g-\sum_{i,j}\alpha_{i}\beta_{j}g(a_{i}b_{j}x)\Big\rfloor<\varepsilon^{\circ}\quad(x\in G). $$ Replace $x\ {\mathrm{by}}\ b_{j}x\operatorname{in}\left(13\right),$ multiply (13) by ${\boldsymbol{\beta}}_{j},$ and add with respect to j, to obtain ·(17) $$ \Big\vert M f-\sum_{i,\bar{j}}\alpha_{i}\beta_{j}f(a_{i}b_{j}x)\Big\vert<s\qquad(x\in G). $$ Thus (18) $$ \Big\lfloor M f+M g-\sum_{i,j}\alpha_{i}\beta_{j}(f+g)(a_{i}b_{j}x)\,\Big\rfloor<2\varepsilon\qquad(x\in G). $$ Since $\sum\alpha_{i}\beta_{j}=1,$ (18) implies (12). The Riesz representaton theorem, combincd with (B),(9),(10) and(12) yields a unique regular Borel probility measure m that satisfe (19) $$ M f=\int_{G}f\,d m\qquad(f\in C(G)); $$ properties(l) and (2) now follow from (11). To prove (3), denote the right side f(3) by $M f,$ and observe that $\varphi^{\prime}$ also satisfies properties (B) to (12), hence that $M^{\prime}=M$ // *: Uncomplemented Subspaces Complemented subspaces of a topological vector space were defined in Section 4.20: Lemma 4.21 furnishcd somc examples. It s also very easy to see that cvery closed subspaceof a Hilbert space s complemented CTheorem 12.4). We will now show that some very famiar closed subspaces of certain other Banach spaces are, in fact, not complemented. These examples il b derivd fom a rather general thorem about126 cENERAL TuroRx compect groups o opeatorstathav n ivarn suspc is po s cct valued integration with respect to Haar measure. we bein S ingia someiosthat existbewen complemented su spaces on the one hand and proiections on the other 5.15 Projections Let $\textstyle X$ < be a vector space. A linear mapping $\boldsymbol{P}$ $X\to X$ is called a projection in $\textstyle X$ if $$ p^{2}=P, $$ i.e. i $P(P x)=P x$ for every $x\in X.$ with null space ${\mathcal{N}}(P)$ and range ${\mathcal{R}}(P).$ The fol- Suppose ${\boldsymbol{P}}$ is a projection in $X,$ lowing facts are almost obvious. (a) ${\mathcal{R}}(P)={\mathcal{M}}(I-P)=\{x\in X\colon P x=x\}.$ (6) $\mathcal{N}(P)=\mathcal{R}(I-P).$ and $X={\mathcal{B}}(P)+{\mathcal{M}}(P).$ and $X=A+B$ , then there (c) $\mathcal{R}(P)\cap\mathcal{M}(P)=\{0\}$ are subspaces of $D\!\!\!\!/$ in $\textstyle X$ with $A={\mathcal{R}}(P)$ and $B=\mathcal{A}(P)$ (d)f $\textstyle A$ 4 and $\boldsymbol{B}$ $\textstyle X$ such that $A\cap B=\{0\}$ is a unique projection (e)、 If $\textstyle A$ $x^{\prime\prime}\in B.$ Define $(I-P)P=0,{\mathcal{R}}(P)\subset{\mathcal{M}}(I-P).$ 1 $\cdot x\in{\mathcal{N}}(I-P),$ then $x-P x-0,$ and so $x^{\prime}\in A$ Since and $\boldsymbol{B}$ satisfy $\scriptstyle(d).$ every $x\in{\mathcal{X}}$ has a unique decomposition . Ifx $\in{\mathcal{R}}(P)\cap{\mathcal{A}}(P),$ , with then $x=P x=0;$ if $x\in X,$ then This gives (a),0) follow by applying (a) to $I-P.$ This proves $x=P x\in{\mathcal{R}}(P).$ $x=P x+(x-P x),$ and $x-P x\in{\mathcal{N}}(P).$ $x=x^{\prime}+x^{\prime\prime}.$ $P x=x^{\prime},$ Trivial verifications then prove (d) 5.16 Theorem (a) I Pis a coimous pojcion i tological eco spac $X,$ then $$ X={\mathcal{R}}(P)\oplus{\mathcal{A}}(P). $$ (b) Comversely,i $\textstyle{\cal{X}}$ is an F-space and i $X=A\left(\hat{\mathbb{D}}B\right)$ , then the projection ${\boldsymbol{P}}$ P with range A and null space $\boldsymbol{B}$ is coninuous Recall that we use the notation $X=A\oplus B$ only when $\dot{A}$ and $\boldsymbol{B}$ are losed sub- spaces of $X$ such that A n $B=\{0\}$ and $A+B=X.$ that To prove that closed graph theorem: Suppose $x_{n}\to x$ Statemet o s ontand nGO o Scion 5.5,excetfor th sertio and that $\scriptstyle{T-P}$ is PRoOF is closed. To see the late, note that and $P x_{n}\to y.$ and null space ${\boldsymbol{B}},$ as in (b). $\textstyle A$ ${\mathcal{R}}(P)$ Next, suppose is continuous we verify that $\boldsymbol{P}$ ${\mathcal{R}}(P)={\mathcal{N}}(I-P)$ and is continuous. $\boldsymbol{P}$ $\textstyle A$ $D\!\!\!\!/$ is the projection with range satisfes h hypotheses of the Since $P x_{n}\in A$sOME APPLUCATiONs 127 have closed,we. have $y\in A,$ hence $y=P y.$ Since $x_{n}-P x_{n}\in B$ and $\boldsymbol{B}$ s closed, we $x-y\in B,$ hencc $P y=P x$ lt follow that $y=P x.$ Hence $D\!\!\!\!/$ is continuous // Corollary closed suspace of an F-space $\textstyle{X}$ is complemented in Xif and only if tis the range of some continvous proiection im $X.$ a topological group 5.17 Groups oflinar operators Supose that toloical vector spac $\scriptstyle X{ arrow}X$ such that and $\textstyle X$ ${\boldsymbol{G}}$ are related in the following manner:To every $s\in G$ corre- sponds a continuous linear operator $T_{s}{\mathrm{:}}$ $$ {\cal T}_{e}=I,\qquad{\cal T}_{s t}={\cal T}_{s}T_{t}\qquad(s\in G,\,t\in G); $$ on $X.$ also, the mapping $(s,\,x)\to T_{s}x$ of $G\times X$ into $X$ is continuous Under these conditions. ${\boldsymbol{G}}$ is said to act as a group of continuous linear operators 5.18 Theorem Suppose (a) X is a Frechet space, (b) r is a complemented subspace of X, , and (d) ${\cal T}_{s}(Y)\subset Y f o r\;e v e r y\;s\in G.$ Gisacopae o ih s a aop omuie oproros $X,$ (c) Then there is acontinuous projection C ${\mathcal{O}}$ 2of X onto $\boldsymbol{\mathit{I}}$ which commutes withevery T. $s^{-1}O s=O$ for all s $\epsilon\,\sigma$ For simplicity, write sx n place ot of $\textstyle X$ onto ${\boldsymbol{Y}}.$ The desired projection ${\mathcal{O}}:$ define is to satisfy PRO0F $\boldsymbol{P}$ $I_{\circ}x.$ By (b) and Theorem 5.16, there ${\mathcal{Q}}$ is a continuous projcction operators The idea of the proof is to obtain ${\mathcal Q}\,$ by averaging the $s^{-1}{\cal P}_{S}$ with respect to the Haar measure m of (1) $$ Q x=\int_{G}s^{-1}P s x\;d m(s)\qquad(x\in X). $$ To show that ths intgral exists, in ccordance with Defntion 3.26, put (2) $$ f_{x}(s)=s^{-1}P s x\qquad(s\in G). $$ $U$ By Theorem 3.27,it suffies to show that , Pu $y=P s_{0}x,$ so that $\mathsf{F K S}_{0}\in G_{1}$ ; let be a ncighborhood of fAso) in $G\to X$ is continuous. $X,$ (3) $$ s_{0}^{-1}y=f_{x}(s_{0}). $$ Since $(s,z)\to s z$ is assumed to be continuous, $S_{0}$ has a neighborhood $V_{\mathrm{1}}$ and y has a neighborhood ${\mathcal{W}}$ such that (4) $$ s^{-1}(W)\subset U\qquad\mathrm{if}\,s\in V_{1}. $$128 GINERAL Tumorv Also, $S_{0}$ has a neighborhood $V_{2}$ such that (5) $$ P s x\in W\qquad{\mathrm{if~}}s\in V_{2}\,. $$ The continuity of ${\boldsymbol{P}}$ D was used here ${\mathrm{If~}}s\in V_{1}\cap V_{2}\nonumber,$ it follows from (2),(4), and (5) th $\operatorname{tat}f_{x}(s)\in U.$ $\operatorname{Thus}f_{x}$ $X.$ To every convex neighborhood $\mathbf{\partial}Y$ by $(d),$ for every of O in $X$ corre- ${\cal{Y}}$ Since ( ${\boldsymbol{G}}$ theorem 2.6 implies therefore tha is continuous. $X.$ The Banach-Steinhaus Fis compact, each , has compact range in is an equicontinuous collection $\ t\left\{s^{-1}P s\colon S\in G\right\}$ 1f $x\in X,$ then $P s x\in Y,$ hence $s^{-1}P s x\in$ $U_{\mathrm{I}}$ $s\in G$ Since Hence ${\mathcal{O}}\,$ of linear operators on of t ${\boldsymbol{0}}$ such that $s^{-1}P s(U_{2})\subset U_{1}$ It now sponds therefore a neighborhood $U_{2}$ that $Q(U_{2})\subset U_{1}$ (See Theorem $3.27.)$ follows from (1) and the convexity of $U_{\mathrm{1}}$ is obvious. is continuous. The linearity of ${\mathcal Q}\,$ is closed $Q x\in Y$ for every se G. Hence lfxe Y,ten sxe Y, Psx = sx, and so $s^{-1}P s x=x,$ $Q x=x$ These two statements prove that ${\cal Q}\,$ is a projection of $\textstyle X$ onto $\mathbf{Y},$ To com- plete the proof, we have to show tha (6) $$ Q s_{0}=s_{0}\,\mathcal{Q}\qquad\mathrm{for~every~}s_{0}\in G. $$ Note that $s^{-1}{\cal P}_{S S_{0}}=s_{0}(s s_{0})^{-1}{\cal P}(s s_{0}),$ It now follows from $(1)$ and $\left(2\right)$ that $$ \begin{array}{r l}{{Q s_{0}.x=\int_{G}s^{-1}P s s_{0}\,x\,d m(s)}}\\ {{}}&{{=\int_{G}s_{0}f_{x}(s s_{0})\,d m(s)}}\\ {{}}&{{=s_{0}\int_{G}f_{x}(s)\,d m(s)}}\end{array} $$ (moving $S_{0}$ The third cqality isdue to the translation-invariance of $m_{\mathrm{\scriptscriptstyle4}}$ ; for the fourth $J/I$ , cros th integral sign), see Exercise $2{\dot{\boldsymbol{4}}}$ of Chapter ${\mathbf{3}}.$ satisfy ${\dot{f}}(n)=0$ Examples In our frst examplc, we take $X=L^{1},~Y=H^{1}$ Here $L^{1}$ is the 5.19 for all $\scriptstyle n\,<\,0$ Reca ita o entesthe th Fourircicnt of /- ${\mathcal{L}}^{1}$ that spac falierablefunctoson the unt cicle an $H^{1}$ consists of those J∈ I (1) $$ \hat{f}(n)=\frac{1}{2\pi}\int_{-\pi}^{\pi}f(\theta)e^{-i n\theta}\;d\theta\qquad(n=0,\pm1,\;\pm2,\,\cdot\cdot). $$ Note that we write RO in place ofe") or siplictSOME APPLICATIONs 129 For ${\boldsymbol{G}}$ we taketeuiticl . thutipleaive goup orallcoplx number thc translation operators ${\boldsymbol{\tau}}_{s}$ defined by of abslute value I, and we asociat to eac $e^{i s}\in G$ (2) $$ (\tau_{s}f)(\theta)=f(s+\theta). $$ that 1t is a simple mattcr to verify that ${\boldsymbol{\mathit{J}}}$ then acts on ${\mathcal{L}}^{1}$ as described in Section 5.17 and (3) $$ (\tau_{s}f)^{\times}(n)=e^{i n s}{\hat{f}}(n). $$ Hence $\tau_{*}(H^{1})=H^{1}$ for every real s.(See Exercise 12.) If ${\boldsymbol{J}}^{1}$ were complemented in ${\mathcal{L}}^{1}$ onto $\textstyle H^{1}$ such tha tinuous projection ${\mathcal{Q}}$ of $L^{1}\!\!$ , Theorem 5.18 would imply that there is a con- (4) $$ \tau_{s}\,Q=Q\tau_{s}\qquad\mathrm{for~all~}\mathrm{s}. $$ Put Let us see what such a ${\mathcal Q}\,$ would have to be and $e_{n}(\theta)=e^{i n\theta}$ Then $\tau_{s}e_{n}=e^{i n s}e_{n}.$ (5) $$ Q\tau_{s}\epsilon_{n}=e^{i n z}Q e_{n}, $$ since $\underline{{{Q}}}$ is linar. t fllows from (4 and SJ that (6) $$ ({\cal Q}e_{n})(s+\theta)=e^{i n\varsigma}({\cal Q}e_{n})(\theta). $$ Put $c_{n}=(Q e_{n})(0).$ With $\theta=0,$ (6) becomes (7) $$ Q e_{n}=c_{n}e_{n}\qquad(n=0,\ \pm1,\ \pm2,\ \cdot\cdot.). $$ Fouricr serie So far we have just used $\left(4\right)$ Since $Q e_{n}\in H^{1}$ for all, $\scriptstyle{c_{a}=0}$ when $\scriptstyle n\,<0,$ Since $Q f=f\operatorname{for}\operatorname{every}f\in H^{1}.$ ${\mathcal{L}}^{1}$ onto ${\boldsymbol{H}}^{\star}$ , the one that replaces $\scriptstyle{\hat{f}}(n)$ by $\mathbf{0}$ when $\scriptstyle n\,<\,0.$ In terms of projection of $c_{n}=1$ when $\scriptstyle n\geq0$ .Thus ${\mathcal Q}\,$ Gif it cxists at al) is the“natural” (8) $$ Q{\biggl(}\sum_{-\infty}^{\infty}a_{n}\,e^{i n\theta}{\biggr)}=\sum_{0}^{\infty}a_{n}\,e^{i n\theta}. $$ To get our contradictin consider the functions (9) $$ f_{r}(\theta)=\sum_{-\infty}^{\infty}r^{|n|}e^{i n\theta}~~~~~~~(0<r<1). $$ $\lfloor\mathrm{hat}f_{r}\geq0$ Hence Thes t e-kwoPosorels Epitusmaio sis O) show (10) for all , Bu $$ ||f_{r}||_{1}=\frac{1}{2\pi}\int_{-\pi}^{\pi}|f_{r}(\theta)|\;d\theta=\frac{1}{2\pi}\int_{-\pi}^{\pi}f_{r}(\theta)\;d\theta=1 $$ $\mathrm{(11)}$ $$ (Q f_{r})(\theta)=\sum_{0}^{\infty}r_{e}e^{i n\theta}=\frac{1}{1-r e^{i\theta}}, $$130 CENERAL THEoRY $\begin{array}{c c c c c c c c c c c c c}{{}}&{{}}&{{}}&{{}}&{{}}&{{}}&{{}}&{{}}&{{}}&{{}}&{{}}&{{}}\end{array}$ $\scriptstyle n\,<0$ Hence ${\mathcal{H}}^{1}$ The same analysis can be appied to $\textstyle A$ the operator ${\cal Q}\,$ since $\mathbf{\nabla}^{\cdot}\mid1-e^{i\theta}\mid^{-1}d\theta=\infty$ By and Fatou's lemma implies that ${\boldsymbol{C}}$ onto A. Application of as $r\to1,$ where ${\boldsymbol{C}}$ is the spac of all continu for all $\|Q f_{r}\|_{1}\to\infty$ ${\mathcal Q}\,$ $L^{1}.$ (10), his contradicts the continuity of is not complemented in If A were complemented in ${\cal{C}},$ 4 and ${\cal{C}},$ described by $({\mathcal{S}})$ would be a ous functions on the unit circlc, and $\textstyle A$ consts of those fe C that have ${\hat{f}}(n)=0$ continuous projection from ${\cal Q}\,$ to real-valued fe ${\boldsymbol{C}}$ ' shows that there is a constant $M<\infty$ that satisfies (12) $$ \operatorname*{sup}_{\theta}\left|f(\theta)\right|\leq M^{\cdot\;\mathrm{sup}\;\left|\;\mathrm{Re}\,f(\theta)\right|} $$ for every f∈ A. To see that no such $\mathcal{M}$ Tcan exist, consider conformal mappings of th closed unit disc onto tall thin ellipses. Hence Hence $\textstyle{A}$ is not complemented in ${\cal{C}}.$ $L^{p}.$ ”.This is a theorem of M. Riesz if $1<p<\infty,$ However, the projection (8) is continuous as an operator in $L^{p},$ $H^{p}$ is then a complemented subspace of (Th. 17.26 of [231). We conclude with an analogue of (b)of Theorem 5.16; t wil sed inth proof of Theorem 11.31. 5.20 Theorem Suppose X is a Banach space, ${\boldsymbol{A}}$ and $\boldsymbol{B}$ are closed subspaces of $x\in{\mathcal{X}}$ has a $X,$ and $X=A+B.$ Then there exists a constant $r\prec\omega$ such that every representation $x=a+b,$ where $a\in A,$ $b\in B$ and $\|a\|+\|b\|\leq\gamma\|x\|,$ This difers from (b)of Theorem 5.16 inasmuch as it is not assmed that $A\cap B=\{0\}$ PROOF Let ${\cal{Y}}$ be the vector space of all ordered pairs $(a_{i},b)_{i}$ with $u\in{\mathcal{A}}$ $b\in B,$ and componentwise adition and scalar multiplication, normed by $$ \|(a,b)\|=\|a\|+\|b\|. $$ Since $\textstyle A$ and $\boldsymbol{B}$ are complete, ${\cal{Y}}$ is a Banach space. The mapping $\operatorname{A}\colon Y\to X$ defined by $$ \Lambda(a,b)=a+b $$ is continuous, since $|a+b||\leq\|(a,$ b)|, and maps Yonto $X.$ By the open mapping wth theorem, there exists $\gamma\ll c o$ such that each $x\in X{\mathrm{~is~}}\Lambda(a,\ b)$ for some (a, ${\mathfrak{b}}\}$ H(a, $b)\|\leq\gamma\|x\|.$ $J/J$ Exercises Define measures ${\boldsymbol{\mu}}_{1},{\boldsymbol{\mu}}$ Lz on the unit circle by $d\mu_{1}=\cos\theta\,d\theta,\qquad d\mu_{2}=\sin\theta\,d\theta$ and find the range of the measure $\mu=(\mu_{1},\mu_{2}).$soME ArrLCATIONs 131 Construct two functions f and $\scriptstyle{\mathcal{G}}$ on D0, 1 wihth olowng propety $\mathrm{If}$ $$ d\mu_{1}=f(x)\,d x,\qquad d\mu_{2}=g(x)\,d x,\qquad\mu=(\mu_{1},\,\mu_{2}), $$ 6 Suppose ${\boldsymbol{G}}$ $x^{-1}H x-H$ for every $\scriptstyle x\in G.$ then he range o is the suarewith eties a (,0.0.1.(-1,0,.0, -10 $\scriptstyle A$ l and $\boldsymbol{B}$ 3 are connected subsets of on S. Hint: 、 so 1 are not assume that ${\boldsymbol{G}}$ Ssups tht the hypothes o Theorem aresatste, ta ${\boldsymbol{H}}$ is a ormasberoupo ${\cal G},$ hat sSaSuiroupu th ${\cal G},$ 3 $|g|<\phi|_{\kappa}$ Prove that there exists fe Ysuch that ${\mathcal{H}}$ lis the largest connected subset of $\phi\in C(S),\phi>0,\,g\in C(K),$ and $f|_{\mathbf{x}}=g$ and $|f|<\phi$ that contains satisfies Apply Theorem $5.9$ to the space ofallfuncions /, with e Y has its support at a single Supply thedtais the ro ht very exteponttoe Hint: If ${\boldsymbol{\mathit{P}}}$ ${\boldsymbol{G}}$ point.(CTis rers the end of the roof of Theore $\operatorname{suo}$ is commutative.) Prove the alois or Thorems 10 1.1 tareaiude t n Section .12. Do is atopological group and the ity ente Prove t ${\mathit{A}}B$ and $A^{-1}$ 7 obviously false.) roveha vyopen suorou or oca ou s loed(The converse 9 ${\cal G}.$ Suppose ${\mathfrak{m}}$ nis the Haar measure of a compact group for $n=0,1,2,\ldots,$ B, and ${\mathbf{}}V$ is a nonempty open set in 8 Prove that $m(V)>0.$ re te Har measure ofteuit cicle. Let bethe smales be the smalest closed ${\cal G},$ ${\mathrm{Put}}_{s}e_{n}(\theta)=e^{t n\theta}.$ Let $L^{2}$ that contains e ${\mathcal{C}}_{n}$ for $n=1,2,3,\ldots.$ Prove he follwing closed subspace of $L^{2}$ $e_{-n}+n e_{n}$ let B $\boldsymbol{B}$ subspace of $L^{2}$ that contains $(a)\ A\cap B=\{0\}.$ 10 (b) If $X=A+B$ then $\textstyle X$ is dense in $L^{2},$ but $X\neq L^{2}.$ and ${\boldsymbol{P}}$ and ${\boldsymbol{Q}}$ are projections. is not (c) Although $\scriptstyle{X=A}$ 9B the rojction in $X{\dot{\textbf{X i}}}$ is, of course, thc onc that $\textstyle X$ inherits from $L^{2}.$ $\boldsymbol{B}$ Com- pare with Theorem 5.16.) $\textstyle X$ with range $\textstyle{\mathcal{A}}$ and null space continuous. (The topology of Suppose $X$ is a Banach space $\;,{\cal P}\subset{\mathcal R}(X),\;{\cal Q}\in{\mathcal R}(X),$ $X^{\star}.$ (a) Show that the adjoint $P^{\star}$ of I $\boldsymbol{P}$ is a projection in $I I$ (b) Show that $~{\boldsymbol{P}}$ and ${\boldsymbol{Q}}$ are projections in a vector space $X.$ Suppose $|P-Q||\cong1{\mathrm{~if~}}P Q=Q P{\mathrm{~and~}}P\neq Q.$ (a) Prove that $\scriptstyle P+Q\,$ is a projction if and only ir $P Q=Q P=0$ In that case, $$ \begin{array}{c l c r}{{{\mathcal A}(P+Q)={\mathcal A}(P)\cap{\mathcal A}(Q),}}\\ {{{\mathcal R}(P+Q)={\mathcal R}(P)+{\mathcal R}(Q),}}\\ {{{\mathcal R}(P)\cap{\mathcal R}(Q)=\{0\},}}\end{array} $$ (b)If $P Q=Q P.$ prove that $P Q.$ is aprojection and tha $$ \begin{array}{l c r}{{{\mathcal W}(P Q)=\mathcal W}(P)+\mathcal W(Q),}}\\ {{{\mathcal R}(P Q)-\mathcal R(P)\,r\mathcal R(Q).}}\end{array} $$ (c) What do the matrices show about part $$ \left(\stackrel{1}{0}\stackrel{0}{-0}\right)\mathrm{~~\bar{~}~^{2n d}~~}\stackrel{\prime}{\longrightarrow}\stackrel{-1}{\bigcup} $$ $(b)t^{\prime}$132 CENERAL TuFORY ${\mathit{1}}_{}^{}$ Prove that the translation operators ${T}_{s}$ used in Example 5.19 saisty the continuit property described in Section 5.17.Explicity, prove tha $$ \|\tau_{r}g-\tau_{s}f\|_{1} arrow0 $$ 13 if $\scriptstyle\gamma\to s$ and $g\to f\operatorname{in}\,I_{*}^{1}$ cannot be omitted from Use the following example to show that the compactness of ${\boldsymbol{G}}$ 12), measure; fe $\boldsymbol{\mathit{I}}$ the hypotheses of Theorem 5.18. Take $i f\int_{R}f=0;$ $\scriptstyle G=R$ with the usual topology;C relative to Lebesgue $L^{1}$ by $\scriptstyle X=L^{1}$ on the real line ${\boldsymbol{R}},$ $\tau_{*}\,Y=Y$ for every ${\boldsymbol{S}},$ and ${\bf Y_{\nu}}$ is complemented in $X.$ ${\boldsymbol{G}}$ acts on translation: if and only The joint continuity property is satified (see Exercise $(\tau_{s}f)(x)-f(s+x)$ Yet there is no projection of ${\cal J}\phi$ $\textstyle X$ Suppose $\boldsymbol{\mathsf{S}}$ and (continuous or not) that commutes with every $\tau_{s}$ onto ${\cal{Y}}$ are continuous linear operators in a topological vector space, and ${\mathbf{}}T$ $$ T=T S T. $$ Prove that ${\mathbf{}}T$ has closed range.(See Theorem 5.16 for the case $S=I.$ 15 Suppose $\scriptstyle{A}$ is a closed subspace of $c(5).$ where $\boldsymbol{\mathsf{S}}$ and suppose f∈ ${\mathsf{C}}(S)$ 'is a compact Hausdorff space; suppose p is an extreme point of the unit ball o $A^{\perp};$ is arel function such that $$ \textstyle [s g/d\mu=0 $$ for every $g\in A$ Prove that ris then constant on te support o ${\boldsymbol{\mu}}$ (Compare with Theo rem 5.73°Show, by an example,that h concusio s as th wora rel omite from the hypotheses.$$ \begin{array}{c c c c c c c c c c c c c c}{{}}&{{}}&{{}}&{{}}&{{}}&{{}}&{{}}&{{}}&{{}}&{{}}&{{}}&{{}}&{{}}&{{}}&{{}}&{{}}&{{}}&{{}}&{{}}&{{}}&{{}}\end{array} $$ $$ \qquad\qquad\qquad\qquad\qquad\qquad\qquad\qquad\qquad\qquad\qquad\qquad\qquad\qquad\qquad\qquad\qquad\qquad\qquad\qquad\qquad\qquad\qquad\qquad\qquad\qquad\qquad\qquad\qquad\qquad\qquad\qquad\qquad\qquad\qquad\qquad\qquad\qquad\qquad\qquad\qquad\qquad\qquad\qquad\qquad\qquad\qquad\qquad\qquad\qquad\qquad\qquad\qquad\qquad\qquad\qquad\qquad\qquad\qquad\qquad\qquad\qquad\qquad\qquad\qquad\qquad Oqquad Oqquad O O O O O O O O O O O O O O O O O O O O O O O O O O O O O O O O O O O O O O O O O O O O O O O O O O O O O O O O O O O O O O O O O O O O O O O O O O O O O O O O O O O O O O O O O O O O O O O O O O O O O O O O O O O O O O O O O O O O O O O O O O O O O O O O O O O O O O O O O O O O O O O O O O O O O O O O O O O O O O O O O O O O O O O O O O O O O O O O O O O O O O O O O O O O O O O O O O O O O O O O O O O O O O O O O O O O O O O O O O O O O O O O O O O O O O O O O O O O O O O O O O O O O O O O O O O O O O O O O O O O O O O O O O O O O O O O O O O O O O O O O O O O O O O O O O O O O O O O O O O O O O O O O O O O O O O O O O O O O O O O O O O O O O O O O O O O O O O O O O O O O O O O O O O O O O O O O O O O O O O O O O O O O O O O O O O O O O O O $$ PART TWO Distributions and Fourier Transforms