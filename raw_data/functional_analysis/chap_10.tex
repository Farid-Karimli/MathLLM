10 BANACH ALGEBRAS Introduction 10.1 Dfinition Acomple alebra s vetor spac $\textstyle{\mathcal{A}}$ over the complex fel $\textstyle{\mathcal{C}}$ in which a multiplication is defined that satisfies (1) $$ x(y z)=(x y)z, $$ (2) ( $$ x+y)z=x z+y z,\qquad x(y+z)=x y+x z, $$ and (3) $$ \mathcal{X}(x y)=(\mathcal{x}\chi)y=x(\alpha y) $$ for all ${\mathcal{X}},\,{\mathcal{Y}},$ and $\mathbb{Z}$ in $\dot{A}$ and for all scalars $\textstyle{\mathcal{A}}$ $\mathrm{If},$ in addition, $\textstyle{A}$ is a Banach space with rcspect to a norm that satisfies the multiplicative inequality (4) $$ \|x y\|\leq\|x\|\ \|y\|\qquad(x\in A,y\in A) $$ and if $\textstyle A$ contains a unit element e such tha (5) xe = ex =x (x∈ A)228 BANACH ALGBRAS AND SPFCTR AL THEORY and (6) $$ \|e\|=1, $$ all $\textstyle{\mathcal{X}}$ and $\mathbf{\nabla}y$ in $A_{\cdot}$ is called a Banach algebra_ be commutative, i.e., that ${\mathcal{E}}^{\prime}$ also satisfies for then $\textstyle{\mathcal{A}}$ Note that we have not required that $\textstyle A$ $x y=y x$ and we shall no do s excpt when explicitly state lt is clear that there is at most one ee A that satisfies (5), or if (5), then $e^{\prime}=e^{\prime}e=e$ can be defined in a more natural way than is otherwise possibl of an element of $\textstyle A$ Te presence of unit is ey often omited fom he efition ofa Banach alebra However, when thereisa ntitmakes sens to talk about inverses, so tht the spectum This leads to a more intuitive development of th basic theory. Moreover, the resulting loss of generliy is sal, because many naturally curring Banach algebras have a unit, and because the others can be suppied with one in the following canonical fashion. Suppose $\scriptstyle{\dot{A}}$ satisfes conditions (I) to(4), but $\textstyle A$ has no unit element. Let ${\mathcal{A}}_{1}$ consist of all ordered pairs $(x,\,\alpha)_{:}$ componentwise, define multiplication in ${\mathcal{A}}_{1}$ , by Define the vector space where $x\in A$ and $\alpha\in{\mathcal{C}}$ operations in $A_{1}$ (7) $$ (x,\alpha)(y,\beta)=(x y+\alpha y+\beta x,\alpha\beta), $$ and define (8) $$ \|(x,\alpha)\|=\|x\|\ +\left|\alpha\right|,\qquad e=(0,1). $$ that if $X_{n} arrow X$ and $y_{n}\to y$ then satisfes properties(1) to(6), and the mapping $x\to(x,0)$ is an isometric $\scriptstyle A_{1}\quad{\mathcal{L}}$ Then $A_{1}^{*}$ onto a subspace of ${\boldsymbol{A}}_{1}$ in fact, onto a closed two-sided ideal of is simply $\scriptstyle A$ . This means isomorphism of $\mathcal{A}$ $\textstyle{\mathcal{X}}$ is identified with $(x,0),$ then ${\mathcal{A}}_{1}$ $\scriptstyle{A}$ plus the whose codimension is 1. If and 11.13(e) one-dimensional vctor space generated by e. See Examples $10.3(d)$ The inequality (4) makes multiplicaton a continuous operation in , which follows from the identity $x_{n}y_{n}\to x y_{*}$ (9) $$ x_{n}y_{n}-x y=(x_{n}-x)y_{n}+x(y_{n}-y). $$ In particular muliplication is lf-coninuous and rgh-coninuous (10) $$ \begin{array}{l l}{{x_{n},y arrow x y}}&{{\ \ \ \ \mathrm{and}\qquad\ x y_{n} arrow x y}}\end{array} $$ if $x_{n}\to X$ and $y_{n}\to y.$ 1t sniestig that (4 can bereplaced by the aparenty weaker equiremen (10) and that(G) can be dropped without enlarging the lass of algebras under con- sideration. 10.2 Theorem Assume thul A isa Bamach space as well as a complex algebr with i elemente 0, m wich muipici s ef-conimous ad rightconimosBANACH ALGEBRAS 229 makes $\textstyle{\mathcal{A}}$ Then there isa norm on $\scriptstyle{\dot{A}}$ which induces $t h e$ same topology as the given one and which ino a Bach lgcbra (Te assumption $\scriptstyle e\,r\,=\,0$ rules out the untestig cas $A=\{0\}_{\sim}\}$ PROOF Assign to cach $x\in A$ the lef-muliplication operator $M_{x}$ defined by (1) $$ M_{x}(z)=x z\qquad(z\in A). $$ $M_{x}\,M_{y}$ Let A be the set ofal $\scriptstyle M_{x}$ Sinceriht mutipicationsmed to be ontinus $\boldsymbol{A}$ $M_{x y}=$ ${\vec{4}}\subset{\mathcal{P}}(A).$ lt is clear that te Bahspe o ionieieoeaos Ifxe A,then $x\to M_{x}$ is liear The ssociaive lwimplies th (2) $$ \|{\boldsymbol{x}}\|\,=\,\|{\boldsymbol{x}}e\|\,=\,\|M_{x}\,e\|\,\leq\,\|M_{x}\|\,\|\,e\|. $$ onto the alebra ${\tilde{A}},$ Thes atca umarize by sying th $\textstyle x\! arrow\!M_{x}$ is an isomorphism of $\textstyle A$ whoseinvese scnous. Sine (3) $$ \|M_{x}M_{y}\|\le\|M_{x}\|\Vert M_{y}\|\qquad\mathrm{and}\qquad\|M_{e}\|=\|I\|=1, $$ $\widetilde{A}$ Theorem is left mutiplication by $x_{i}\in A.$ is naheae.-zniei smte . od i ed su are cquivalent norms on ${\mathcal{A}}.$ $x arrow M_{x}$ $\textstyle T_{i}$ space of ${\mathcal{R}}(A),$ relaive to he topoy given by the opeatr norm Se ${\mathcal{R}}(A).$ If is also continuous. Hence $\|\!\|,x\!\|,$ and $\|M_{x}\|$ onceti i o t oe mapprntermie t in the topology of $\lambda_{\cdot}1_{\cdot})$ Suppose $T\in{\mathcal{B}}(A).$ $T_{i}\in\vec{A}$ ,and $T_{i}\to T$ , then (4) $$ T_{i}(y)=x_{i}y=(x_{i}e)y=T_{i}(e)y. $$ As (4) tends to $T(\epsilon)y.$ the frstemin 4) ends t Then it follow that the last term of Since mult- $i arrow\infty,$ . Pu $x=\ T(e)$ $T(y),$ $A_{\cdot}$ and $T_{i}(e)\to T(e).$ plicato isasumed o be e-oniusin (5) $$ T(y)=T(e)y=x y=M_{x}(y)\qquad(y\in{\cal A}). $$ so that $T=M_{x}\in{\tilde{A}},$ and $\widetilde{A}$ is closed. / 10.3 If $\textstyle K$ Examples(a) Let $C(\underline{{K}}_{\underline{{\times}}})$ be he Banacth spaceof all omlex continu This makes $C(K)$ into ln particular, when functions on a nonemptycompact Haso spc $K,$ with the supremum norm. $C^{n},$ with Define multiplication i the usual way: $(f g)(p)=f(p)g(p)$ is simply coordinatewise multiplication. a comuativeBachaler h cosa YncionYs i eme ${\mathsf{C}}({\mathsf{R}})$ is a fite set onsting o say, pons, tCh $n=1,$ we obtain the siplest Bach algebra namel ${\underline{{C}}},$ with the abslte vle as norm230 BANACH ALGEBRAS AND SPECTRAL THEORY (b)Let $X$ Y be a Banach space. Then ${\mathcal{B}}(X),$ , the algebra of all bounded linear operators on $X.$ is a Banach algebra, with respect to the usual operator norm. The identity operator Iis its unit element. If dim $X=n<\infty,$ then ${\mathcal{B}}(X).$ is (isomorphic to) the algebra of all complex n-by-n matrices. If dim $X>1.$ ,then ${\mathcal{B}}(X)$ is not com- mutative.(The trivial space $X=\{0\}$ must be excluded.) is also a Banach algebra. The Every closed subalgebra of ${\mathcal{B}}(X)$ that contains ${\mathbf I}\,$ proof of Theorem 10.2 shows, in fact, that every Banach algebra is isomorphic to one of these of $C(K)$ (c)f K is a nonempty compact subset of $f\in C(K)$ that are holomorphic in the interior of is the subalgebra $K,$ ${\underline{{C}}},$ or of C ${\mathcal{C}}^{n},$ ~", and if $\scriptstyle A$ that consists of those then $\scriptstyle{A}$ is complete (relative to the supremum norm) and is therefore a Banach algebra. When $K\mathbf{\partial}K$ Kis the closed unit disc in ${\mathcal{C}},$ then $\textstyle A$ is called the disc algebra. (d) $L^{1}(R^{n});$ ,with convolution as multiplication, satisfies all requirements of Definition $|0,1,$ except that it lacks a unit. One can adjoin one by the abstract procedure outlined in Section 10.1 or one can do it more concretely by enlarging $L^{1}(R^{n})$ to the algebra of all complex Borel measures $\boldsymbol{\mu}$ on $\textstyle R^{n}$ of the form $$ d\mu=f\,d m_{n}+\lambda\,d\delta $$ where fe $\in L^{1}(R^{n}),\;{\dot{\alpha}}$ S is the Dirac measure on $\textstyle R^{n}\!\!,$ ", and $\lambda$ is a scalar. (e) Let $M(R^{n})$ be the algebra of all complex Borel measures on $\textstyle R^{n},$ with con- volution as multiplication, normed by the total variation. This is a commutative Banach algebra, with unit ${\bar{\partial}},$ which contains $(d)$ as a closed subalgebra. 10.4 Remarks There are several reasons for restricting our attention to Banach algebras over the complex field, although real Banach algebras (whose definition should be obvious) have also been studied. One reason is that certain elementary facts about holomorphic functions play an important rolein th foundations of the subject. This may be observed in Theorems 10.9 and 10.13 and becomes even more obvious in the symbolic calculus. Another reason--one whose implications are not quite so obvious--is that ${\boldsymbol{C}}$ has a natural nontrivial involution(see Definition 11.14), namely, conjugation, and that many of the deeper properties of certain types of Banach algebras depend on the presence of an involution.(For the same reason, the theory of complex Hilber spaces is richer than that of real ones.) At one point (Theorem 10.44) a topological difference between ${\boldsymbol{C}}$ Z and ${\boldsymbol{R}}$ will even play a role. Among the important mappings from one Banach algebra into another are thc homomorphisms. These are linear mappings $\ \ \! (\begin{array}{l l}{h\!}&{$ that are also multiplicative: h(x) = h(x)hGy)BANACH ALGEBRAs 231 nomel ${\boldsymbol{C}}$ Qf prclcristes e es n hran siso l nhalebra ${\bar{C}}.$ itselMay ft t infn ietes emuivey sen crucilly n sfi upply of homomorphsm on Complex Homomorphisms 10.5 Definition Suppose $\textstyle{\mathcal{A}}$ tis a complex alebra and $\phi$ is alinear functional on $\mathcal{A}$ which is not identcall $\mathbf{0}$ 1 (1) $$ .\qquad\phi(x y)=\phi(x)\phi(y) $$ for all $x\in{\mathcal{A}}$ and $y\in A.$ then $\phi$ is caled a complex homomorphism $\textstyle{\mathcal{A}}$ (The exclusion of $\phi\equiv0$ is, of course,just mater o cnvenience. that is, if there An elemen $x\in A$ is said o be imerile if i has an inerse i ${\mathcal{A}}.$ exists an element $x^{-1}\in A$ such that (2) $$ x^{-1}x=x x^{-1}=e, $$ where $\scriptstyle{\mathcal{C}}$ pis the unit element of ${\bar{A}}.$ then Note that no $x\in A$ has more than one inverse, for i $y x=e=x z$ $$ y=y e=y(x z)=(y x)z=e z=z. $$ 10.6 Proposition ${\boldsymbol{J}}$ p is $\textstyle{\mathcal{A}}$ complex homomorphism on $\bar{a}$ complex algebra $\scriptstyle A$ l with umi e, hen $\phi({\boldsymbol{e}})=1$ ,and $\phi(x)\neq0\,J o r$ every invertible $x\in A.$ PROOF For some $y\in A,$ $\phi(y)\neq0$ Since $$ \phi(y)=\phi(y e)=\phi(y)\phi(e), $$ it follows that $\phi(e)=1.$ lf xis invertible, then $$ \phi(x)\phi(x^{-1})=\phi(x x^{-1})=\phi(e)=1, $$ so that $\phi(x)\neq0.$ // Parts o and(Q of te flwong heorem arepechapsthe most widel s facts u thery o Bach aleras aita 0Cimis itai icomne homomorphisms of Banach alebrarec ninuou 10.7 Theorem Suppos $\scriptstyle{\dot{A}}$ is a Bach alebra, xe A $\|x\|\ <\ |.$ Then (a)e-xis invertible $\begin{array}{c l c r}{(b)}&{\|(e-x)^{-1}-e-x\|\leq{\frac{\|x\|^{2}}{1-\|x\|}}}\end{array}$ (e IdG)l <l for ery comple omorhismp on A232 BANACH ALGEBRAS AND SPECTRAL THEoRY PROOF Since $\|x^{n}\|\ \leq\ \|x\|^{n}\ \mathrm{and}\ \ \|x\|<1,$ the elements (1) $$ s_{n}=e+x+x^{2}+\cdots*+x^{n} $$ form a Cauchy sequence in ${\mathcal{A}}.$ Since $\scriptstyle{\mathcal{A}}$ is complete, there exists $S\in{\mathcal{A}}$ such that S。→s.Since $x^{n}\to0$ and (2) $$ s_{n}\cdot(e-x)=e-x^{n+1}=(e-x)\cdot s_{n}, $$ (l) shows tha the continuity of multiplication implies that $\boldsymbol{\mathsf{S}}$ is the inverse of $e arrow X.$ Nex, ls $$ \begin{array}{l}{{-\,e-x\parallel=|x^{2}+x^{3}+\cdots\parallel\leq\sum_{n=1}^{\infty}\|x\|^{n}=\frac{\|x\|^{2}}{1-\downarrow x\|^{2}}.}}\\ {{\,\,\,\,\,\,\,\,\,\,\,\,\,\,\,\,\,\,\,\,\,\,\,\,\,\,\,\,\,\,\,\,1-\lambda^{-1}\phi(x)=\phi(e-\lambda^{-1}x)\not=0.\,\,\,\,\,\,\,\,}}\end{array} $$ tible. By Finally, su Proposition 10.6 Hence $\phi(x)\neq\lambda$ . This completes the proof. // We now interrupt the main line of development and insert a thcorem which shows, for Banach algebras, that Proposition 10.6 actually characterizes the complex homomorphisms among the linear functionals. This striking result has apparently found no interesting applications as yet. 10.8 Lemma Suppose f is an entire function of one complex variable $f(0)=1,$ $f^{\prime}(0)=0,a n d$ (1) $$ 0<|f(\lambda)|\leq e^{|\lambda|}\qquad(\lambda\in C). $$ $T h e n f(\lambda)=1\,f o r\,a l l\,\lambda\in C.$ PROOF Since_ $\boldsymbol{\mathit{f}}$ has no zero, there is an entire function $\scriptstyle{\mathcal{G}}$ such that $f=\exp{\{g\}},$ $g(0)=g^{\prime}(0)=0,$ and $\operatorname{Re}\left[g(\lambda)\right]\leq\,\left|\,\lambda\,\right|.$ This inequality implics (2) $$ |g(\lambda)|\leq|2r-g(\lambda)|\qquad(|\lambda|\leq r). $$ The function (3) $$ h_{r}(\lambda)=\frac{r^{2}g(\lambda)}{\lambda^{2}[2r-g(\lambda)]} $$ is holomorphic in $\{\lambda\colon|\lambda|<2r\},$ and $|b_{i}(\lambda)|\leq1$ if |2|= r. By the maximum modulus theorem, (4) $$ \left|h_{r}(\lambda)\right|\leq1\qquad(\left|\lambda\right|\leq r). $$ Fix A and let r→00.、 Then (3) and (4) imply that $g(\lambda)=0.$ $J/f$BANACH ALGEBRAS 233 10.9 ,Teore Cleason, Kahane, Zelazko) and $\phi(x)\neq0$ for every invertible x∈ A,then Banach algebra A, such tha $\phi~~~~~~~~~~~~~~~~~~~~~~~~~~~~~~~~~~~~~~~~~~~~~~~~~~~~~~~~~~~~~~~~~~~~~~~~~~~~~~~~~~~~~~~~~~~~~~~~~~~~~~~~~~~~~~~~~~~~~~~~~~~~~~~~~~~~~~~~~~~~~~~~~~~~~~~~~~~~~~~~~~~~~~~~~~~~~~~~~~~~~~~~~~~~~~~~~~~~~~~~~~~~~~~~~~~~~~~~~~$ is inur mciol on $\phi(e)=1$ (1) $$ \phi(x y)=\phi(x)\phi(y)\qquad(x\in A,y\in A). $$ Note that the continuity of $\varnothing\;$ b is not part of the hypothesis PROOF Let $\textstyle N$ be the null space o $\phi.$ $\operatorname{If}\,x\in A$ and $y\in A,$ the assumption $\phi(e)=1$ shows that (2) $$ x=a+\phi(x)e,\ \ \ \ \ \ y=b+\phi(y)e, $$ where $a\in N,$ $b\in N.$ If $\phi$ is applied to the product of the equations $(2),$ one obtains (3) $$ \phi(x y)=\phi(a b)+\phi(x)\phi(y). $$ The dsied conclusio s thefreuvlent t aertion tha (4) $$ a b\in N\qquad{\mathrm{if~}}a\in N $$ and $b\in N.$ Suppose we had proved a special case of $(\mathrm{d})_{s}$ namely, .(5) $$ \begin{array}{c c c}{{a^{2}\in N}}&{{}}&{{\mathrm{~if~}\ a\in N.}}\end{array} $$ Then (3), with $x=y,$ implies (6) $$ \phi(x^{2})=[\phi(x)]^{2}~~~~~~~(x\in A). $$ Replacement of xb $x+y$ in(G) results in (7) $$ \phi(x y+y x)=2\phi(x)\phi(y)\qquad(x\subset A,\,y\in A). $$ Hence (8) $$ x y\,+\,y x\in N\qquad{\mathrm{if~}}x^{<}\in N,\,y\in A. $$ Consider the identity (9) $$ (x y-y x)^{2}+(x y+y x)^{2}=2[x(y x y)+(y x y)x]. $$ $\operatorname{If}\,x\in N,$ thc right side of (9) is in $N,$ and another application of (G yields $(x y+y x)^{2},\,\mathrm{by}\,(8)$ and (6) Hence $(x y-y x)^{t}$ is in V , by(8), and so is ${\N},$ (10) $$ x y.\equiv y x\in N\qquad{\mathrm{if~}}x\in N,y\in A. $$ analytic methods. Addition or G8) and (IO) gives (4), hence(OD Ths mis .。or ry iertar easos The roo S u for cvcry By hypothesis by $\mathbf{\Psi}(a)$ of Thcorem 10.7. Hence Thus $\|e-x\|$ ≥1 $\textstyle N$ contains no invertible element of ${\cal A}.$ $x\in N,$ (1) lle -xI≥|4=14(ke- x1 (xe N,AeC)234 BANACH ALGEBRAS AND SPECTRAL THEoRY To prove (5), fix $a\in N,$ is a continuous linear functional on $|a||=1$ without loss of generality, and We conclude that $\phi$ $A_{\mathrm{{z}}}$ ,of norm 1 assume define (12) $$ f(\lambda)=\sum_{n=0}^{\infty}\frac{\phi(a^{n})}{n!}\,\lambda^{n}\qquad(\lambda\in C). $$ Sincc $|\phi(a^{n})|\leq\|a^{n}\|\leq\|a\|^{n}=1,f\,;$ is entire and satisfies $|f(\lambda)|\leq\exp|\lambda|$ for all $\lambda\in C.$ Also, $f(0)=\phi(e)=1,$ $,\,\operatorname{and}f^{\prime}(0)=\phi(a)=0.$ $\lambda\in C,$ Lemma 10.8 will imply that If we can prove that $f(\lambda)\neq0$ for every $f^{\prime\prime}(0)=0;$ hence $\phi(a^{2})=0.$ which proves (5) The series (13) $$ E(\lambda)=\sum_{n=0}^{\infty}{\frac{\lambda^{n}}{n!}}\,a^{n} $$ converges in the norm of $A_{\mathrm{{,}}}$ A,for every $\lambda\in C.$ The continuity of $\phi$ shows that (14) $$ f(\lambda)=\phi(E(\lambda))\qquad(\lambda\in{\mathcal{C}}). $$ The functional equation $E(\lambda+\mu)=E(\lambda)E(\mu)$ follows from (13) exactly as in the scalar case. In particular, (15) $$ E(\lambda)E(-\lambda)=E(0)=e\qquad(\lambda arrow C). $$ Hence $\scriptstyle{E(\lambda)}$ is an invertible element of ${\mathcal{A}}_{s}$ for every $\lambda\in{\mathcal{C}}$ This implies, by hy- // pothesis, that $\phi(E(\lambda))\neq0,$ and therefore $f(\lambda)\neq0,$ by (14). This completes the proof Basic Properties of Spectra 10.10 Definitions Let $\bar{A}$ be a Banach algebra; let $G=G(A)$ be the set of all If invertiblc clements of $\textstyle A$ 4. If $x\in G$ and $\textstyle{\mathcal{X}}$ is the set of all complex numbers $\boldsymbol{\lambda}$ L such that thus $y\in G,$ then $y^{-1}x$ is the inverse of $x^{-1}y;$ $x^{-1}y\in G,$ and ${\mathfrak{C}}$ Fis a group $x\in A.$ the spectrum $\sigma(x)$ of x $\lambda e-x$ is not invertible. The complement of o(x)is the resolvent set of x;itconsists of all $\lambda\in{\mathcal{C}}$ for which $\scriptstyle\left(\lambda e-x\right)^{-1}$ exists. The spectral radius of $\textstyle{\mathcal{X}}$ is the number (1) $$ \rho(x)=\operatorname*{sup}\;\{|\lambda|:\lambda\in\sigma(x)\}. $$ $^{p}$ It is the radius of the smallest closed circular disc in 《 ${\mathcal{C}},$ with center at O, which contains shall see o(x). Of course,(I) makes no sense if ox) is empty. But this never happens, as weBANACH ALOEBRAs 235 10.11 Theorem Suppose $\textstyle{\mathcal{A}}$ is a Banach algebra $x\in G(A),\,\,h\in A,\,\,\|h\|<\frac{1}{2}\|x^{-1}\|^{-1}$ Then $x+h\in G(A),$ and (1) $$ \|(x+h)^{-1}-x^{-1}+x^{-1}h x^{-1}||\leq2||x^{-1}||^{3}||h||^{2}. $$ that PROOF Since $x+h=x(e+x^{-1}h)$ and $\|x^{-1}h\|<{\frac{1}{2}}$ Theorem 10.7 implies $x+h\in G(A)$ and hate normo te rinht me S tent $$ (x,+h)^{-1}-x^{-1}+x^{-1}h x^{-1}=[(e+x^{-1}h)^{-1}-e+x^{-1}h]x^{-1} $$ is at most 2"41x-1I // 10.12 mapping x→ ${\boldsymbol{x}}^{-1}$ Theorem If Ais a Bamach alebra, the $G(A)$ onto $G(A).$ is amopen subset of A, and h isa homeomorphism $\sigma(A)$ PRoor That $G(A)$ is open and that $G(A)$ onto $\scriptstyle{G(A)}$ and since it s itsown inverse,itis // homeomorphism. $x\to x^{-1}$ is continuos follows from Theorem 1011. Since $x\to x^{-i}$ maps 10.13 Theorem I Aisa Bach lera ad xe ${\cal A},$ then (a) thesetru $\scriptstyle{\sigma(x)}$ 0/ xis compact and nonempty, and (6) he speca radi $\rho(x)$ of x satisfies (1) $$ \rho(x)=\operatorname*{lim}_{n arrow\infty}\|x^{n}\|^{1/n}=\operatorname*{inf}_{n\ge1}\|x^{n}\|^{1/n}. $$ inequality Coi ese o i u o ousn nut (2) $$ \rho(x)\leq\lVert x\rVert $$ is contained i the specral radius ormula (D) PRoOr If $|\,\lambda\,|\,>\,\|x\|_{}^{||}$ then,e - 2-'× lies in $G(A)$ by Theorem 10.7, and so does Then $\mathbf{\nabla}\sim{\mathbf{}}$ is ke -x. Thus $\lambda\not\in\sigma(x)$ This proves C2) I patcuar by $g(\lambda)=\lambda e-x.$ Now define f: $\Omega arrow G(A)$ is closed aefine $\sigma(x)$ is a bounded set To prove that $\scriptstyle{\sigma(x)}$ $:C\to A$ which is open, by continuous, and uhe complemen $\mathbb{Q}$ of o(x))is $g^{-1}(G(A))$ Theorem 10.12. Thus $\sigma(x)$ is compact. by (3) $$ f(\lambda)=(\lambda e-x)^{-1}\qquad(\lambda\in\Omega). $$236 BANACH ALGEBRAS AND SPECTRAL THEORY Replace $\mathbf{\Omega}\cdot\mathbf{\Omega}$ by $\lambda e-x$ and $\dot{h}$ by $(\mu-\lambda)e$ in Theorem 10.11.1f $\lambda\in\Omega$ and $\boldsymbol{\mu}$ u is sufficiently close to $\lambda_{\mathrm{:}}$ the result of this substitution is (4) $$ \|f(\mu)-f(\lambda)+(\mu-\lambda)f^{2}(\lambda)\|\leq2\|f(\lambda)\|^{3}\|\mu-\lambda\|^{2}, $$ so that (5) $$ \operatorname*{lim}_{\mu\to\lambda}\frac{f(\mu)-f(\lambda)}{\mu-\lambda}=-f^{2}(\lambda)\qquad(\lambda\in\Omega). $$ Thus f is a strongly holomorphic $A\!\cdot\!$ valued function in $\Omega.$ If $|\lambda|>\|x\|,$ the argument used in Theorem 10.7 shows that (6) $$ f(\lambda)=\sum_{n=0}^{\infty}\lambda^{-n-1}x^{n}=\lambda^{-1}e+\lambda^{-2}x+\cdots. $$ This series converges uniformly on every circle $\textstyle\Gamma_{r}$ with center at O and radius $r\gg\lVert X\rVert.$ By Theorem 3.29,term-by-term integration is therefore legitimate. Hence (7) $$ x^{n}=\frac{1}{2\pi i}\int_{{\Gamma}_{r}}{\lambda}^{n}f(\lambda)\,d\lambda\qquad(r>\lVert x\rVert,\,n=0,\,1,\,2,\,\dots).\quad. $$ If $\sigma(x)$ were empty, $\Omega$ would be C, and the Cauchy theorem 3.31 would imply that all integrals in (T) are O. But when $n=0,$ the left-hand side of $(7)$ is $\scriptstyle e\,\wedge\,0.$ This contradiction shows that $\sigma(x)$ is not empty. Since $\underline{{\mathbf{Q}}}$ contains all $\lambda$ with $|\lambda|>\rho(x),$ an application of (3) of the Cauchy 1r theorem 3.31 shows that the condition r $\mathbf{\Phi}_{,}\mathbf{\Phi}\setminus\hat{\Vert}_{,}\lambda\hat{\Vert}_{,}\lambda\hat{\Vert}$ can be replaced in(7) by r >p(x) (8) $$ M(r)=\mathrm{ma}_{\mathrm{~}\mathrm{~}\parallel f(r e^{i\theta})\parallel}\quad\quad(r>\rho(x)), $$ the continuity offimplies that $M(r)<\infty$ Since (T) now givcs (9) $$ \|x^{n}\|\leq r^{n+1}M(r), $$ we oblain (10) $$ \operatorname*{lim}_{n arrow\infty}\left|\vert x^{n}\vert\vert^{1/n}\leq r\right.\qquad\left(r>\rho(x)\right) $$ so that (11 $$ \operatorname*{lim}_{n\to\infty}\operatorname*{sup}_{\alpha}\|x^{n}\|^{1/n}\leq\rho(x). $$ On the other hand, ${\mathsf{f f}}\lambda\in\sigma(x),$ the factorization (12) 2 $$ {}^{n}e-x^{n}=(\lambda e_{--}x)(\lambda_{\ --}^{n-1}e+\cdot\cdot\cdot+x^{n-1}) $$BANACH ALGEBRAs 237 shows that $\lambda^{n}e-\chi^{n}$ is not invertible. Thus ${\lambda}^{n}\in{\mathcal{O}}({\chi}^{n})$ By (2) $|\lambda^{n}|\leq\Vert x^{n}\Vert$ for $n=1,\,2,\,3,\,\ldots.$ Hence (13) $$ \rho(x)\leq{\mathrm{inf}}\ \|x^{n}\|^{1/n}, $$ and (I) is an immediate consequence or $\scriptstyle{(11)}$ and(13) // The nonemptiness o $\scriptstyle v(x)$ leadsto an easy chatrzation of those Banach algebras that ar dvision algebras 10.14 $\lambda(x)e-x$ therefore an isomorphism of $\scriptstyle{\mathcal{A}}$ onto ${\mathcal{C}},$ .Theore GCeand-Mlazu)I A 。 ach aler m which wer $x=\lambda(x)e.$ The mapping $x\to\lambda(x)$ is PROOF If monzo em is pnile,hn and $\lambda_{1}\neq\lambda_{2}.$ ,then at most one of the elements $\lambda_{1}e-x$ and $\textstyle{\mathcal{L}}\epsilon_{s,A}$ $\mathcal{A}$ is isomeiealfisomoipnc o ne compe e $\lambda_{2}\,e-x\,\mathrm{is}\,0;$ is not invertible, it is ${\boldsymbol{0}}.$ hence at least one of them is invertible. Since $\scriptstyle\sigma(x)$ is not empty, it Since folwtat o cos xaty oept a2.)soec Hence $x\in A$ $\|\lambda(x)e\|=\|x\|$ for every xe A. , which is also an isometry since $\left|\lambda(x)\right|=$ $J/\slash$ thc content or Chapters $\mathbf{i}\,{\boldsymbol{1}}$ Theoes . n .0. amon e e sus ihspter. Much。 to 13 isinenent o theremainer o Chapter I0 ways metric propcrties of ${\bar{A}}.$ 10.15 Remarks a) Whether an ement o $\scriptstyle A$ is or s notinvetible ${\mathcal A}$ isa puely alebraic property. Te spectum and the spctar aius o an $x\in{\mathcal{A}}$ are thus defined in erms of the algebrai structrce o ${\mathcal{A}},$ regardless of any metric (or topologica considerations. On the other hand, Tim $\|\,X^{n}\|^{1/n}$ depends obviously on This is one of the remarkble featurs o te spctal radiu formulal stst equaiy o crtanquanitiswic as im cly ife detail (b) Ouralgebr $\scriptstyle{\mathcal{X}}$ depens therefore on the aigebra"The inciusi to ${\mathcal{B}},$ , since the spectral radius ${\boldsymbol{B}},$ and jt ${\boldsymbol{B}}.$ The spectrum o dependent of anything that happens outside ${\bar{A}}.$ may be a subalgebra of a larger Banach algebra $\textstyle{\mathcal{A}}$ buisimeirie in greater $\textstyle{\mathcal{A}}$ maythen vy wel hapen tat somecxc $\scriptstyle{A}$ is otieveite $\textstyle{\mathcal{A}}$ formuesti erms o metrcpete opowe $\textstyle\star_{\nu}^{\!\bullet}{\underline{{\bullet}}}$ $\sigma_{A}(x)\Rightarrow\sigma_{B}(x)$ holds tenti sepatory) te peran ien TiesepectY athseaeni radius is, howver,unaffected by the pasage fro Theorem 10.18 il describe the relation ctwee $\sigma_{A}(x)$ and $\sigma_{n}(x)$ Ve W,and ${\mathcal{W}}$ 10.6 Lemma Svppose V and W are open ses im some toial spae ${\mathcal{X}},$ V comais owndary point o . Then Pisauniono componemis ${\mathcal{W}}.$238 ANACH ALCEBRAS AND sptCTRAL THuEoRv Recall that a component of ${\mathcal{W}}$ is, by definition, a maximal connected subset of ${\mathcal{W}}.$ PRoO0r Let $\mathbb{Q}$ be a component of F $W$ 'that intersects ${\mathit{V}}.$ Let $U_{\mathbf{\delta}}U$ be the complement of ${\overline{{V}}}.$ Since ${\mathcal{W}}$ contains no boundary point of $\nu,\mathbf{a}$ is the union of the two disjoint open sets Q $\mathbf{\varepsilon}\cap V$ and $\Omega\cap U$ Since $\mathbb{Q}$ is connected. $\mathrm{Q}\cap U$ is empty. Thus $\Omega\subset V.$ ${\it1}/j$ 10.17 Lemma Suppose $\scriptstyle A$ is a Banach algebra,x,∈ G(A) for $n=1_{\mathrm{enskipJ}}$ 2,3,..., xis a boundary point of $G({\mathcal{A}}),$ and $x_{n}\to x$ as $n arrow\infty.$ Then $\|x_{n}^{-1}\|\to$ oO as $n arrow G0.$ PROOF for infinitely many $r_{\mathit{l}}$ If the conclusion is false, there exists $M<\infty$ such that $\|x_{n}^{-1}\|<M$ For one of these, $\|x_{n}-x\|<1/M.$ For this , $$ \|e-x_{n}^{-1}x\|=\|x_{n}^{-1}(x_{n}-x)\|<1, $$ so that $x_{n}^{-1}x\in G(A).$ Since This contradicts the hypothesis, since $G({\mathcal{A}})$ is open is a group, it follows that $x=x_{n}(x_{n}^{-1}x)$ and $\scriptstyle{G(A)}$ $x\in G(A).$ 10.18 Theorem (a) I/ $\scriptstyle A$ is a closed subalgebra of a Banach algebra B,and if A contains the unit element of B,then $\scriptstyle{G(A)}$ is a union of components of $A\cap G(B).$ (b) Under these conditions $i^{f}\,x\in A,$ then $\sigma_{a}(x)$ is the union o $\sigma_{p}(x)$ and $\bar{\boldsymbol{a}}$ (possibly emply) collection o/ bounlel componenls o the complemenl o $\zeta\sigma_{B}(x).$ In parlicular, the boundary of $\sigma_{a}(x)$ lies in $\sigma_{n}(x)$ in $\boldsymbol{B}$ Thus PROOr(a) Every member of $\scriptstyle{\dot{A}}$ that has an inverse in $\textstyle{\mathcal{A}}$ has the same inverse By $G(A)\subset G(B)$ Both $G(A)$ and $A\cap G(B)$ are open subsets of ${\bar{A}}.$ Lemma 10.16, it is sufficient to prove that $\sigma(B)$ contains no boundary point $\mathbf{\nabla}y$ p of $\ G(A).$ Any such $\mathbf{y}$ is the limit of a scquence {x,}in $G(A).$ By Lemma 10.17, $\|x_{n}^{-1}\|\to\infty$ If $\mathbf{\vec{y}}$ were in $\sigma(B)$ , the continuity of inversion in $\sigma(B)$ (Theorem 10.12) would force ${\underline{{X}}}_{n}^{-1}$ to converge to $y^{-1}$ In particular {lx, Il} would be $\ G(\lambda)$ bounded. Hence ${\bar{\lambda}}_{0}$ bea boundary point of ${\Omega}_{A}$ Then $\lambda_{0}e-x$ is a boundary point of relative to ${\mathcal{C}}.$ (b) Let $y\not\in G(B),$ and (a) is proved. $\lambda_{0}\notin\Omega_{B}$ Lemma 10.16 implies now that $\sigma_{i}(x)$ $\lambda e-x\in$ By (a), $\lambda_{0}\,e-x\notin G(B)$ Hence be the complements of $\sigma_{A}(x)$ and of $\Omega_{A}$ and $\Omega_{B}$ G(aA). Let The inclusion $\Omega_{A}\subset\Omega_{B}$ is obvious, since $\lambda\in\Omega_{\lambda}$ if and only if $\Omega_{A}$ is the union of certain components of $\scriptstyle\Omega_{n}.$ The other components of $\Omega_{B}$ are therefore subsets>of $\sigma_{A}(x).$ This proves $\{b\}.$ $J/\slash$BANACH ALCBRAs 239 then CorollaryUfoAxy does searae C.ha iscmole $\Omega_{B}$ is connected, $\sigma_{A}(x)=\sigma_{B}(x).$ For then $\Omega_{B}$ has no bounded components contains only real numbers. Thc mos ipotnu apiatonof uis olary ocurs when op(v are not nearly so important As anoter ppicaton of Lemma 10.17 we now prove a theorem whose co clui th sam s that o teGefand-aur theorematiutiht scsosece 10.19 Theorem I Ais a Bach alebra amd i the exis $M<\infty$ such that (1) $$ \|{\vec{x}}\|\,\|y\|\leq M\|x y\|\qquad(x\in A,y\in A), $$ then A is isomerically isomorphic to)C PROOF Let y be a boundary point of $\ G(A)$ Then $\gamma=\operatorname*{lim}y_{n}$ for some sequenc $\scriptstyle(y_{n}|$ in $G(A)$ By Lemma 10.17, $\|y_{n}^{-1}\|\to\infty$ By hypothesis (2D $$ \|y_{n}\|\lVert y_{n}^{-1}\rVert\leqslant M\rVert\qquad(n=1,\,2,\,3,\dots) $$ Hence $\|y_{n}\|\to0$ and therefore $y=0.$ of $\scriptstyle{\sigma(x)}$ gives rise to a boundary point $J/f$ If $x\in A,$ each boundary point $\lambda$ $\chi e-x$ of $\scriptstyle{G(A)}$ Thus $x=\lambda e.$ In other words, $A=\{\lambda e:\lambda\in C\}.$ lt is natual to ask whethcr the specta or iwo elements $\mathbf{y}$ are close to cach other. The next $\mathbf{\nabla}y$ of $\textstyle{\mathcal{A}}$ are close together, in some suitably defined sense, if and $\textstyle{\mathcal{X}}$ and theorem gives a very simple answer. $\sigma(x)<\Omega$ 10.20 Theorem Suppose $\textstyle A$ is aDanach lyebra $x\in A.$ $\Omega$ is an open set in ${\mathcal{Q}},$ and Then the exis $\delta>\Phi$ such than o(x + )) for every ye d with $\|y\|<\delta$ of PROOF Since $\|(\lambda e-x)^{-1}\|$ is a continuous function of $\lambda arrow\infty.$ therc is a number $M<\infty$ $\sigma(x),$ and since this norm tends to $\mathbf{0}$ as $\boldsymbol{\lambda}$ in the complement such that $\stackrel{\vec{\mathbf{a}}}{=}$ $$ \|(\lambda e-x)^{-1}\|<M $$ for all $\bar{\lambda}$ outside $\mathbb{C}$ .If $y\in A,$ $\|\nu\|<1/M,$ and $\lambda\notin\Omega,$ it follows that $$ \lambda e-(x+y)=(\lambda e-x)[e-(\lambda e-x)^{-1}y] $$ is invertible in ${\cal A},$ since $|\vert(\lambda e-x)^{-1}y\vert<1\,;\;\mathrm{hence}\;\lambda\not=\sigma(x+y).$ This gives the $l/{\slash{\prime}}/$ desired conclusion, with =1/M240 BANACH ALGEBRAS AND SPECTRAL THEORY Symbolic Calculus 10.21 Introduction If xis an element of a Banach algcbra $\textstyle A$ and $i\mathbf{f}f(\lambda)=$ $\mathcal{Q}_{0}\,+\,\cdot\,\cdot\,\cdot\,+\,\mathcal{Q}_{n}\,\lambda^{n}$ is a polynomial with complex coefficients ${\mathcal{Q}}_{i}\,,$ there can be no doub defined by about the meaning of the symbol f(x);it obviously denotes the element of $\scriptstyle A$ $$ f(x)=\alpha_{0}\,e+\alpha_{1}x+\cdots+\alpha_{n}x^{n}. $$ The question arises whether f(x) can be defined in a meaningful way for other functions ${\boldsymbol{\mathsf{f}}}.$ We have already encountered some examples of this. For instance, during the proof of Theorem 10.9 we came very close to defining the exponential function in ${\cal A}.$ In fact, if ${\mathcal{I}}(\lambda)=\sum^{\alpha_{k}\lambda^{k}}$ is any entire function in ${\mathcal{C}},$ it is natural to define f(x) ∈ A by $f(x)=\sum_{k}x^{k};$ this series always converges. Another example is given by the mero- morphic functions $$ f(\lambda)={\frac{1}{\alpha-\lambda}}. $$ In this case, the natural definition $\operatorname{of}f(x)$ is $$ f(x)=(\alpha e-x)^{-1} $$ which makes sense for all $\scriptstyle{\mathcal{X}}$ whose spectrum does not contain α One is thus led to the conjecture that $\textstyle f(x)$ should be definable, within ${\bar{A}}.$ , when- ever fis holomorphic in an open set that contains o(x). This turns out to be correct and can be accomplished by a version of the Cauchy formula that converts complex functions defined in open subsets of ${\boldsymbol{C}}$ to $A.$ -valued ones defined in certain open sub- sets of A. Just as in classical analysis, the Cauchy formula is a much more adaptable tool than the power series representation.) Moreover, the entities $f(x)$ )so defined (see Definition 10.26) turn out to have interesting properties. The most important of these are summarized in Theorems 10.27 to 10.29. $\operatorname{\k}_{\downarrow_{N}}^{}$ In certain algebras one can go further. For instance, ir $\scriptstyle{\mathcal{X}}$ is a bounded normal operator on a Hilbert space $\textstyle H,$ the symbol $\scriptstyle{f(x)}$ can be interpreted as a bounded normal operator on ${\cal H}$ l when fis any continuous complex function on $\sigma(x).$ and even when fis any complex bounded Borel function on $\sigma(x).$ In Chapter 12 we shall see how this Ieads to an efflcient proof of a very general form of the spcctral theorem. 10.22 Integration of $\mathbf{A}.$ -valued functions If $\int f\,d\mu$ exists and has all the properties that ${\boldsymbol{Q}}$ on which a $\boldsymbol{f}$ is a $\textstyle{\mathcal{A}}$ is a Banach algebra and continuous A-valued function on some compact Hausdorff space complex Borel measure $\boldsymbol{\mu}$ is defined, then is a Banach space. However, an addi- were discussed in Chapter 3, simply becausc $\textstyle A$BANACH ALGEBRAS 241 then tiona pert c add e a wi usd n sequl namely $\;{\sqrt{r}}x\in A,$ (1) $$ x\int_{Q}f\,d\mu=\int_{Q}x f(p)\,d\mu(p) $$ and (2) $$ \quad\quad\quad\quad\quad\quad\quad\Bigl(\int_{Q}f\,d\mu\Bigr)x=\int_{Q}f(p)x\ d\mu(p) $$ 10.2, and let To prove (), let $\scriptstyle M_{x}$ be left mutiplication by ${\mathcal{A}}.$ Then $\Lambda M_{x}$ is a bounded linear $\Lambda$ be a bounded linear functional on $X_{\uparrow}$ as in the proof of Theorem functional. Definition 3.26 mplis theforetha for every $\mathrm{A}_{\mathrm{{,}}}$ so that $$ \Lambda M_{x}\int_{\mathcal{Q}}f\,d\mu=\int_{Q}(\Lambda M_{x}f)\,d\mu=\Lambda\,\int_{Q}(M_{x}f)\,d\mu, $$ $$ M_{x}\int_{Q}f\,d\mu=\int_{Q}(M_{x}f)\,d\mu, $$ plication by which sjs ante way o witig( ). o prove 2). ner $M_{x}$ to be right multi $\textstyle{\mathcal{X}}$ 10.23 Contours Suppose $\textstyle K$ is a compact subse of an open in $\Omega.$ , none of which inter- is a sects $K.$ collcion of finitely many oriented line intervas $\Omega\subset{\mathcal{C}}$ , and ${\boldsymbol{\Gamma}}$ In this situation, integration over $\gamma_{1},\,\cdot\,\cdot\,\cdot\,,\,\ \gamma_{n}$ $\Gamma$ 'is defined by (1) $$ \int_{\Gamma}\phi(\lambda)\,d\lambda=\sum_{j=1}^{n}\int_{\gamma_{j}}\phi(\lambda)\,d\lambda. $$ $\mathbf{f}\mathbf{f}$ is ell known that ${\boldsymbol{\Gamma}}$ can be so chosen that (2) $$ {\mathrm{Ind_{I}}}\left({\mathord{\zeta}}\right)={\frac{1}{2\pi i}}\int_{\Gamma}{\frac{d\lambda}{\lambda-\gamma}}= \lbrace{\begin{array}{l l}{{}}&{{\mathrm{if\,}}\gamma\in K}\\ {{}}&{{\mathrm{if\,}\,\forall\,}}\end{array}} $$ and that the Cauchy formula (3) $$ f(\zeta)_{..}{=}\frac{1}{2\pi i}\int_{\Gamma}(\lambda-\zeta)^{-1}f(\lambda)\;d\lambda $$ Theorem 135 of [231 then hls ore orpi uncton in an orevey K.Sce, or instan K in Q. We shall decibte stuation 2) brefly by syin tat the conou ${\boldsymbol{\Gamma}}$ surrounds be connected Notc that neither $K\mathbf{\partial}K$ nor $\Omega$ nor the union of the intervals $\gamma_{i}$ ;has bee assumed to242 BANACH ALGEBRAS AND srCTRAL TuroRv 10.24 Lemma Suppose $\scriptstyle A$ is a Banach algebra, xe A $\alpha\in C,\;\alpha\not\in\sigma(x),$ Q is the complement of c in ${\boldsymbol{C}}$ , and ${\boldsymbol{\Gamma}}$ surrounds o(x) in $\underline{{\otimes}}$ Then (1) $$ {\frac{1}{2\pi i}}\int_{\Gamma}(\alpha-\lambda)^{n}(\lambda e-x)^{-1}\,d\lambda=(\alpha e-x)^{n}\qquad(n-0,\pm1,\,\pm2,\,\cdots). $$ PROOF Denote the integral by ${\mathcal{V}}_{n}$ When $\lambda\notin\sigma(x),$ then $$ (\lambda e-x)^{-1}=(\omega e-x)^{-1}+(\alpha-\lambda)(\alpha e-x)^{-1}(\lambda e-x)^{-1}. $$ By Section 10.22, ${\mathbf{}}y_{n}$ is therefore the sum of (2) $$ (\sigma e-x)^{-1}\cdot\frac{1}{2\pi i}\int_{\Gamma}(\alpha-\lambda)^{n}\,d\lambda=0, $$ since ${\mathrm{Ind}}_{T}\left(x\right)=0,$ and (3) $$ (\alpha e-x)^{-1}\cdot{\frac{1}{2\pi i}}\int_{\Gamma}(\alpha-\lambda)^{n+1}(\lambda e-x)^{-1}~d\lambda. $$ Hence (4) $$ (\alpha e-x)y_{n}=y_{n+1}\qquad(n=0,\,\pm1,\,\pm2,\,\ldots). $$ $\operatorname{Wc}$ thus have to prove that This recursion formula shows that(I) follows from the case $n=0.$ (5) $$ \frac{1}{2\pi i}\int_{\Gamma}(\lambda e-x)^{-1}\;d\lambda=e. $$ Let $\textstyle\Gamma_{r}$ be a positively oriented circle, centered at 0, with radius $r\gg\|x\|.$ On $\mathbf{0}_{;}$ T, $\cdot,\ \left(\lambda e-x\right)^{-1}=\sum\lambda^{-n-1}x^{n}.$ $\textstyle\Gamma.$ Sincc the integrand in() is a holomorphic $A\!\cdot\!$ valued Termwise integration of this series gives(5) with $\Gamma_{r}$ in placc of function in the complement of $\sigma(x)$ (see the proof of Theorem 10.13), and since (6) $$ {\mathrm{Ind}}_{{\mathbf{T}}_{r}}(\zeta)=1={\mathrm{Ind}}_{\mathbf{T}}\left(\zeta\right) $$ for every $\gamma\in\sigma(x),$ ), the Cauchy theorem 3.31 shows that the integral(5)i unaffected if ${\boldsymbol{\Gamma}}$ is replaced by $\Gamma_{p}\,.$ This completes the proof. // 10.25 Theorem Suppose (1) $$ R(\lambda)=P(\lambda)+\sum_{m,\,k}c_{m,\,k}(\lambda-\alpha_{m})^{-k}\qquad\qquad. $$ is a rational funclion wilh poles at the points ${\mathcal{Q}}_{m}$ $\mathbf{\nabla}[P]$ is a polynomial, and the sum in(1) define has only finitely many terms.] $\;U f x\in A$ and $i^{f}\sigma(x)$ contains no pole of ${\boldsymbol{R}},$ (2) $$ R(x)=P(x)+\sum_{m,k}c_{m,k}(x-\alpha_{m}e)^{-k}. $$BANACH ALOEBRASs 243 1 . isam open se in C tha conain $\sigma(x)$ and in which ${\boldsymbol{R}}$ is holomorphic, and if T suroumds o(x)in Q, then (3) $$ R(x)={\frac{1}{2\pi i}}\int_{\Gamma}R(\lambda)(\lambda e-x)^{-1}~d\lambda. $$ PROOF Apply Lemma 10.24 // ${\mathcal{X}}\in{\mathcal{A}}.$ Note ta ) etinly the mos aral efion o aratioal fnction o This motivates thefollwing dfinition Thcnusion hows tathe Cuhy frmaciesesameu 10.26 Definition Suppose $\textstyle A$ is a Banach algcbra, $\mathbb{Q}$ 2 is an open set in ${\underline{{C}}},$ ,and $H(\mathbb{C})$ is heaer a me oi nioss yieorm 2. (1) $$ A_{\Omega}=\{x\in A\colon\sigma(x)\subset\Omega\} $$ is an open subset of $\scriptstyle A$ to be the set of al $A\!\cdot\!$ valued functions ${\tilde{f}},$ , with domain ${\mathcal{A}}_{\Omega}$ ,that arise from We defne ${\widetilde{H}}(A_{\Omega})$ by the formula $\mathrm{an}\,f\in H(\Omega)$ (2) $$ \tilde{f}(x)=\frac{1}{2\pi i}\int_{\Gamma}f(\lambda)(\lambda e-x)^{-1}~d\lambda, $$ where $\boldsymbol{\Gamma}$ Tis any contour that surrounds $\sigma(x)$ in $\Omega.$ This definito casfrsome comments of $\textstyle A$ (a)Since $\boldsymbol{\Gamma}$ stays away from $\sigma(x)$ and sinceinversion is continuous in ${\mathcal{A}}_{s}$ the inteadi cnus 2) a tienaeixsaenS Sxsasne (c)If $x=x e$ (0)The internd aul a olorphic $\mathbb{Q}.$ valued ucion n te compe provided only that ${\boldsymbol{\Gamma}}$ ${\mathcal{A}}.$ surrounds o(x) in ment o..:(This asoserei teprof oTerem 03Se xecee3 Th Ccyhe.3 ilis ster t GJSmieiem iehecte y and $x\in\Omega,\;(2$ P) becomes (3) $$ \ ^{*}\ {\widehat{f}}(\alpha e)=f(\alpha)e. $$ Note that oe $\scriptstyle\epsilon\:A_{0}$ if and only $\mathbb{F}\alpha\in\Omega$ If we identify $\lambda\in{\mathcal{C}}$ with Ae e A, every e $H(\Omega)$ may bc regarded as mapping certan seto ${\mathcal{A}}_{\Omega}$ (namely, the intersection of ${\mathcal{A}}_{\Omega}$ with the one-dimensiona subspace of $\textstyle{\mathcal{A}}$ l gcnerated by e) into ${\bar{A}},$ and then (3) shows that f may be regarded as an exensiomof广 ln most teatnsotistoi,.foxis wten lace of ourf(x)The notato fis se h eacs ivodcais iaimitcasesisinesean244 BANACH ALOGEBRAS ANp srCTRAL HroR on $\boldsymbol{S}$ (d)f $\boldsymbol{\mathsf{S}}$ pointwise. For instanc,if u and $\boldsymbol{v}$ map $\boldsymbol{\mathsf{S}}$ lis any algebra, thecollction of al ,then -valued functions is any set and $\textstyle A$ in to ${\bar{A}}.$ $\textstyle A$ is analgebra f ca multipication,addion, and multipication are defind $$ (w)(s)=u(s)v(s)\qquad(s\in S). $$ This will b applied to $A\!\cdot\!$ valued functions defined i $\scriptstyle A_{\alpha}$ 10.27 Theorem Suppose ${\bar{A}}.$ ,1(2),and ${\widetilde{H}}(A_{\underline{{{\alpha}}}})$ are as in Definition 1026. Then $H(\Omega)$ ${\tilde{H}}(A_{\Omega})$ is a complex algebra.The mapping $f\to f$ is am algebra isomorphism of 2, then onto ${\tilde{H}}(A_{\Omega})$ which is continuous in the following sense: and f,→ uiformly on compact subses o $\mathbb{Q}_{*}$ $I f_{n}\in H(\Omega)\;(n=1,\,2,\,3,\,\ldots)$ (1) $$ \hat{f}(x)=\operatorname*{lim}_{n arrow\infty}\tilde{f}_{n}(x)\qquad(x\in A_{\Omega}). $$ If u(4) = 入 and $v(\lambda)=1$ in ( $\mathbf{Q}.$ then ${\tilde{u}}(x)=x$ and ü(x) = e for every xe Ao PROOF The last sentenefollows from Theorem 10.25 The interal represc is inear. If ${\tilde{f}}=0$ , then tation (2) in Section 10.26 makes it obvious $\mathrm{[hat~}f arrow{\tilde{f}}$ (2) $$ f(\alpha)e=\hat{f}(\alpha e)=0~~~~~~(\alpha\in\Omega), $$ so that $f=\mathbb{C}$ 0. Thus $f{\boldsymbol{\tau}}{\boldsymbol{\hat{f}}}$ is one-to-one in Section 1 0.26, since The asserted continuity follows directly from the integral $\mathbf{(2)}$ $\|(\lambda e-x)^{-1}\|$ is bounded on T. (Use the same $\boldsymbol{\Gamma}$ for all , and apply Theorem 3.29.) and $h(\lambda)=f(\lambda)g(\lambda)$ for all $\scriptstyle\chi\in\mathbb{N}.$ it has to be shown Lhat ltremains to bc proved that $f{\boldsymbol{\tau}}{\boldsymbol{\tau}}$ Fis multiplicative. Explicity, if f e H(Q) $g\in H(\Omega),$ (3) $$ \tilde{h}(x)=\tilde{f}(x)\tilde{g}(x)\qquad(x\in A_{\Omega}). $$ $\boldsymbol{f}$ and $\boldsymbol{\mathit{g}}$ by rational functions $\ f_{n}$ are rational functions without poles in $\mathbf{Q}.$ , and if $h=f g$ , then $\mathbb{Q}.$ 1f $\boldsymbol{f}$ and $\mathbf{\Omega}^{g}$ and sincc Thcorem 10.25 asscrts that $R(x)=\tilde{R}(x).$ (3) holds // $h(x)=f(x)g(x),$ $\scriptstyle f\to f.$ is a commutative algebra, since $H(\mathbb{C})$ is obviously com- ,and Thnth enea case,Runge's theorem Th 13.9 f 2J) aosus o proximate uniformly on compact subscts of ${\mathcal{G}}_{n}\,,$ Then . onves th amemne and foos fo conui of the mapping Note that ${\widetilde{H}}(A_{\alpha})$ mutative. 10.28 Theorem Suppose $x\in A_{\Omega}\;a n d f\in H(\Omega).$ (a)了(x) is invertible in Aif and oly if/(2) = O for enery $\lambda\in\sigma(x).$ (b)α(了(x))= f(o(x)) Part (b) is called the spectral mapping theoremBANACH ALGEBRAS 245 $\Omega_{1}$ $f(x)=0$ rRoor(a)If has no zero on $x\in\sigma(x)$ then there exists then $g=1/f!$ is holomorphic in an open set $\Omega_{1}$ in such that $\sigma(x)\subset\Omega_{1}\subset\Omega.$ $\sigma(x),$ $f g=1$ in $\Omega_{1},$ Theorem 10.27(with place of Q) shows that Since is invertible. Conversely i for some $f(x){\widetilde{g}}(x)=e,$ and th $\operatorname{us}f(x)$ such that $h\in H(\Omega)$ (1) $$ (\lambda-\alpha)h(\lambda)=f(\lambda)\qquad(\lambda\in\Omega), $$ which implie (2) $$ (x-\alpha e)\tilde{h}(x)=\tilde{f}(x)=\tilde{h}(x)(x-\alpha e), $$ ${\mathcal{I}}-{\boldsymbol{\beta}}$ by Theorem 10.27. Since $\beta\in C.$ By definition $\beta\in\sigma({\bar{f}}(x))$ if and only if $\scriptstyle{\hat{f}}(x)-\beta e$ is not $\mathbf{\phi}(b)$ has a zero in $\sigma(x),$ $x-\alpha e$ is not invertible in $\scriptstyle A$ , neither i ${\tilde{f}}(x),$ by (2) Fix 、By (a), applied $\mathrm{to}f-\beta$ is place of , this hapes if and only if $///f$ invertie n $\overline{{A}}$ that is,if and only if $\beta\in f(\sigma(x))$ To sotaingnormas osl ielaeoin functions among the opeations the symboicacu $g\in H(\Omega_{1}),$ and Theorem Suppose and ${\tilde{h}}(x)={\tilde{g}}({\tilde{f}}(x)).$ the set of all) es $\Omega_{1}$ is an open set containing $f({\boldsymbol{\sigma}}({\boldsymbol{x}})),$ 10.29 $x\in A_{\Omega},f\in H(\Omega)$ $\Omega\ w i t h f(\lambda)\in\Omega_{1}$ $h(\lambda)=g(f(\lambda))$ in $\Omega_{\mathrm{o}},$ $T h e n f(x)\in A_{\Omega_{1}}$ Briely ${\tilde{h}}={\tilde{g}}$ 。fif $h=g\circ f.$ PROOr By(b) of Theorem 10.28 $-\sigma({\hat{f}}(x))\subset\Omega_{1},$ and therefore ${\tilde{g}}({\tilde{f}}(x))$ is defined. ${\mathcal{N}}.$ with Fix a contour ${\boldsymbol{\Gamma}}_{1}$ that surrounds $f({\boldsymbol{\sigma}}({\boldsymbol{x}}))$ in $\complement_{1}$ There is an open sct $\sigma(x)\subset W\subset\Omega_{0}\,,$ , so small that (I) $$ [\mathrm{nd}_{\Gamma_{t}}\left(f(\lambda)\right)=1\qquad(\lambda\in W). $$ Fix a contour $\Gamma_{0}$ that surounds ${\mathcal{W}}$ in placc of in $\textstyle W.$ If $\zeta\in\Gamma_{1},$ then $1/(\zeta-f)\in H(W).$ Hence Theorem 10.27, with $\scriptstyle v(x)$ $\mathbb{Q}_{*}$ ,shows that (2) $$ [\zeta e-\tilde{f}(x)]^{-1}=\frac{1}{2\pi i}\int_{\Gamma_{0}}[\zeta-f(\lambda)]^{-1}(\lambda e-x)^{-1}\,d\lambda\qquad(\zeta\in\Gamma_{1}). $$ Since ${\boldsymbol{\Gamma}}_{1}$ surrounds $\sigma({\hat{f}}(x))$ n $\Omega_{1},\;(1){\mathrm{~and~}}(2)\;{\mathrm{i}}{\mathrm{~j}}$ mply G(f(x)) = _9(5)[te -fx) ur 2ri J $$ =\frac{1}{2\pi i}\int_{\Gamma_{0}}\frac{1}{2\pi i}\int_{\Gamma_{1}}g(\zeta)[\zeta-f(\lambda)]^{-1}\;d\zeta(\lambda e-x)^{-1}\;d\lambda $$ $$ =\frac{1}{2\pi i}\int_{\Gamma_{0}}g(f(\lambda))(\lambda e-x)^{-1}\;d\lambda=\frac{1}{2\pi i}\int_{\Gamma_{0}}h(\lambda)(\lambda e-x)^{-1}\;d\lambda=\tilde{h}(x). $$ 7246 RANACH ALGEBRAS AND SPECTRAL THEORY nth root in $\textstyle A$ means that $x=y^{n}$ we sal now give someaplictons o hisymoicus The frst o If $x=\exp{(y)}$ for some $y\in A,$ , then deals with the existence of roots and logarithms. To say that an element $x\in A$ has an for some $y\in A.$ y is a logarithm of x $\exp{(y)}=\sum^{\infty}y^{n}/n!$ but that the exponential function can also be Note that defined by contor ntegation 'as in Defion 1.6. The coninity asetion of Theorem 10.27 sows that these dfnitions coincide (as they do for every entire function) 10.30 Theorem Suppose $\scriptstyle A$ is a Banach algebra, $x\in A_{\cdot}$ , and the spectrum o(x) of x does not separate O from $\infty$ Then (a)x has roots o all rders im ${\bar{A}},$ (c)if ${^{\circ}\kappa>0},$ x has a logarithm in ${\bar{A}},$ and $\|x^{-1}-P(x)\|<\varepsilon.$ (6) there is a polynomial $D\!\!\!\!/$ such that Moreover, if (x) lies the posiereal axis,th roots in ay can be chosen so a to satisfy the same condition of PR0OF By hypothess,oO ies in the unboundcd component of the complement $\sigma(x)$ .Hence there is a function ${\mathcal{f}},$ holomorphic in a simly connected open set $\Omega\supset\sigma(x).$ which satisfies $$ \exp{(f(\lambda))}=\lambda. $$ lt follows from Theorem 10.29 that $$ \exp\left({\hat{f}}(x)\right)=x, $$ so that $y={\hat{f}}(x)$ is a logarithm of $\sigma(x),$ so that oy lies in the real axis by te specta $\lambda<\sigma(x),f$ can be ${\mathcal{X}}.$ If $0<\lambda~.$ < co for every chosen so as to be real on and another application of the This mapping theorem. If z = exp O/n), then $z^{n}=x,$ $\sigma(y)\subset(-\infty,\,\infty).$ spectral mapping theorem shows that $\sigma(z)\subset(0,\,\infty)$ if provs a and b), of oue a coud havebee proedicty wtho passing through (b). can be approximated by polynomials, uniforml $J/I$ To prove (c), note that $1/\lambda$ assertion of Theorem 10.27 on someopen et ong oco)(Rungesteorem) and se thecontinuit Thesc susar noutialevn whe $\textstyle{\mathcal{A}}$ is afienoalaleba is For examitsciaise Dta complex -y-n matx $M_{\mathrm{{s}}}$ that is, if and only if $\ M$ (or the algcbra of all bounded linear operators on C") $\bar{M}$ is thexoni of some matrix if and only if O is not an eigenvalue of inveibte.To deduc this from b),.e bethe algebra f al oplex mby-n matricBANACH ALGEBRAS 247 10.31 Theorem (c) $P(x)=0$ Then Suppose A is a banach algebra, xe A, $D\!\!\!\!/$ is a polynomiol $i{\ddot{n}}$ one variable,and (a) (b) $\scriptstyle{\sigma(x)}$ lies in he se o/ zeros of ${\boldsymbol{P}}.$ then $\sigma(x)\subset\{0,\,1.$ }. contains a non trial idempotent lIn paricular, f xisidempotenf, .e, ) $f x^{2}=x,$ $\textstyle{\mathcal{A}}$ 1If the spectrum of some element of $\textstyle A$ is nol connected, then The trivial idempotents are $\mathbf{0}$ and ${\mathcal{C}},$ of course PROOF By the spectral mapping theorem, $$ P(\sigma(x))=\sigma(P(x))=\sigma(0)=\{0\}. $$ $f(\lambda)=0$ This gives (a) and (b).If $\scriptstyle{\sigma(x)}$ is not conncted, there are disjoint open set $\mathbb{Q}$ covers $\sigma(x).$ Put and $\Omega_{1},$ in $\Omega_{0},f(\lambda)=1$ in $\Omega_{1}$ Then $f\in H(\Omega).$ Put $y={\hat{f}}(x)$ Since ${\mathfrak{G}}_{0}$ both of which intersec $\sigma(x)$ and whose union Theorem 10.27 implies that $y^{2}=y,$ and so $\mathbf{\nabla}y$ $f^{2}=f,$ mapping theorem, is idempotent. By the spectral $$ \sigma(y)=f(\sigma(x))=\{0,1\}. $$ Hence $\mathbf{\nabla}y$ is nontrivial, since ${\boldsymbol{0}}$ and e have onc-point spcctra_ // of 10.32 Definition Let $X.$ The point spectrum $\sigma_{\mu}(T)$ of an operator $T\in{\mathcal{B}}(X)$ is the on the Banach space ${\mathcal{B}}(X)$ be the Banach algebra ofall bounded linear operator way. set of all eigenvalues of $A={\mathcal{R}}(X).$ , th spectral mapping theorem can bc rcfnd inthe follwing ${\mathcal{A}}(T-\lambda I)$ When ${\bar{I}}.$ Thus $\lambda\in\sigma_{p}(T)$ if and only if the null space $T-\lambda I$ has positive dimension. 10.33 Theorem Suppose $T\in{\mathcal{B}}(X)$ $\underline{{\mathbf{Q}}}$ is open in C, $\sigma(T)\ll\Omega$ ,and fc H(S2) (6)/(o。 (a)1f xe X,αe SQ, and $T x=\alpha x_{i}$ , hen ${\tilde{f}}(T)x=f(\alpha)x.$ (c)If α $(T))\subseteq\sigma_{p}({\bar{f}}(T))$ and - c does mot taish ideicly ary componem o/ S,then $\varepsilon\,\sigma_{p}({\widehat{f}}(T))\,,$ (d) X $\in f(\sigma_{p}(T)).$ $f(\sigma_{p}(T))=\sigma_{p}({\hat{f}}(T)).$ 1// is not constan i any componen o/ Q, then Part (a) states that every eigenvector of ${\boldsymbol{T}},$ with eigenvalue ${\mathcal{Q}},$ is also an eigen vector of f(T), with ienvaluc /(a) PRoor(a) $\operatorname{If}x=0$ there is nothing to be proved. Assume such that $x\neq0$ and $T x=\alpha x.$ Then $\alpha\in\sigma(T),$ and there exists $g\in H(\Omega)$ (1) $f(\lambda)-f(x)-g(\lambda)(\lambda-\alpha).$248 BANACH ALGEBRAS AND SPECTRAL THEORY By Theorem 10.27,(1) implies (2) $$ {\tilde{f}}(T)-f(x)I={\tilde{g}}(T)(T-\alpha I). $$ Since $(T-\alpha I)x=0,(2)$ proves $(a).$ ${\tilde{f}}(T)$ whenever $\scriptstyle{\mathcal{X}}$ is an eigenvalue of ${\boldsymbol{T}}.$ It Thus $J(x)$ is an eigenvalue of follows that (a) implies (b). Under the hypotheses of(c) (3) $$ \alpha\in\sigma_{p}(\widetilde{f}(T))\subset\sigma(\widetilde{f}(T))=f(\sigma(T)), $$ so that (4) $$ f^{-1}(\alpha)\cap\sigma(T)\neq\emptyset. $$ of $f-\alpha$ in $\sigma(T),$ does not vanish identically in any component of $\Omega$ is a compact subset of $\Omega$ and $f-\alpha$ Moreover, the set (4) is finite, because $\sigma(T)$ . Let $\zeta_{1},\ \star\cdot,\ \zeta_{n}$ be the zeros counted according to their multiplicities. Then (5) $$ f(\lambda)-\alpha=g(\lambda)(\lambda-\zeta_{1})\cdots(\lambda-\zeta_{n}), $$ where $g\in H(\Omega)$ and ${\mathcal{G}}$ has no zero on $\sigma(T),$ so that (6) $$ {\tilde{f}}(T)-\alpha I={\tilde{g}}(T)(T-\zeta_{1}I)\cdot\cdot\cdot(T-\zeta_{n}I). $$ operators $T-\zeta_{i}I$ of Theorem 10.28,g(T) is invertible in ${\mathcal{P}}(X)$ . Since $\scriptstyle{\mathcal{A}}$ is an eigenvalue of $\sigma_{\mu}(U),$ $\operatorname{sy}(a)$ is not one-to-one on $\textstyle X$ .Hence $\mathbf{\tau}(6)$ implies that at least one of the is in $\tilde{f}(T),\tilde{f}(T)-o d$ must fail to be one-to-one. The corresponding $\zeta_{i}$ and $\mathrm{since}f(\zeta_{i})=\alpha$ the proof of (c) is complete. and $(c).$ $J/I$ Finally, (d) is an immediate consequcence of $\mathbf{\Psi}(b)$ Differcntiation We shall now investigate the extent to which the members of $H(A_{0})$ (see Definition 10.26)) behave like holomorphic functions,as far as differentiability, power series representation,and the open mapping property are concerned. As might be expected. is commutative some of the results are more similar to the classical ones when $\scriptstyle{\dot{\boldsymbol{A}}}$ than when it is not. ${\mathbf{}}F$ maps $\underline{{\otimes}}$ 10.34 Definition Suppose $\textstyle X$ and ${\mathbf{}}Y$ are Banach spaces, $\mathbb{Q}$ is an open subset o $X,$ into ${\boldsymbol{Y}},$ and $a\in\Omega$ 、If therc exists $\Lambda\in{\mathcal{R}}(X)$ Y)(the Banach space of al bounded linear mappings of X $\textstyle X$ into Y) such that $$ \operatorname*{lim}_{x\to0}{\frac{\|F(a+x)-F(a)-\Lambda x\|}{\|x\|}}=0, $$ then A is called the Fréchet derivative of ${\mathbf{}}F$ at a.(The uniqueness of A is trivial.)BANACH ALGEBRAS 249 If The notation $(D F)_{a}$ will usd for the Frichet dervative ${\mathbf{}}F$ at a $(D F)_{a}$ exists for every $a\in\Omega,$ and ${\mathfrak{i}}{\mathfrak{f}}$ $$ a arrow(D F)_{a} $$ tiable in $\Omega$ is a continuous mapping of $\underline{{\mathbf{Q}}}$ into ${\mathcal{G}}(X,$ Y), then ${\mathbf{}}F$ is said to be continuowsly difen 10.35 Difference Quotients If $\scriptstyle A$ is a Banach algebra $x\in A,$ and $x+h\in A,$ and if both sides of the identit $$ (\lambda e-x)-(\lambda e-x-h)=h $$ obtains are multiplied by $(\lambda e-x-h)^{-1}$ on the left and by $(\lambda e-x)^{-1}$ on the right, one (1) $$ (\lambda e-x-h)^{-1}-(\lambda e-x)^{-1}=(\lambda e-x-h)^{-1}h(\lambda e-x)^{-1} $$ so that i surrounds provided, of course, that these inverses exist $C,\,x\in A_{\Omega},\,x+h\in A_{\Omega},$ ${\mathfrak{a n d}}f\in I I(\Omega)$ Choose ${\boldsymbol{\Gamma}}$ Suppose now that Q is open in $\ O(X)\ \cup\ \ O(X\ +\ \mu)$ in $\Omega.$ Then (I) eads to (2) $$ \tilde{f}(x+h)-\tilde{f}(x)=\frac{1}{2\pi i}\int_{\Gamma}f(\lambda)(\lambda e-x-h)^{-1}h(\lambda e-x)^{-1}~d\lambda. $$ 1 $\dot{\boldsymbol{h}}$ and xcomute,. z kx,then can be movd ousd negral(2 as we saw in Section 10.22. Tis motvates the defnion the ieme ioie (3) $$ (Q{\tilde{f}})(x;h)={\frac{1}{2\pi i}}\int_{\Gamma}f(\lambda)(\lambda e-x-h)^{-1}(\lambda e-x)^{-1}\,d\lambda, $$ which satisfies (4) $$ \tilde{f}(x+h)-\tilde{f}(x)=h(Q\tilde{f})(x;h), $$ provided tha $x h=h x,$ an assumption which applies to the remainder of this section $\ \lambda$ on ${\boldsymbol{\Gamma}}_{\cdot}$ then the series If $\|{\boldsymbol{h}}\|<1/M,$ where $M>\|(\lambda e-x)^{-1}\|$ for every (5) $$ \left(\dot{x}e-x_{..}^{\phantom{a b}}.h\right)^{-1}=\sum_{n=1}^{\infty}\left(\lambda e-x\right)^{-n}h^{n-1} $$ converges, in the norm topology of ${\cal A},$ uniformly for $\lambda$ on $\textstyle\Gamma\,.$ Hence 3)) becomes (6) $$ (Q\widetilde{f})(x;\,h)=\sum_{n=1}^{\infty}\frac{1}{2\pi i}\int_{\Gamma}f(\lambda)(\lambda e-x)^{-n-1}\,d\lambda\,h^{n-1} $$ $$ =\sum_{n=1}^{\infty}\frac{1}{n!}\frac{1}{2\pi i}\int_{\Gamma}f^{(n)}(\lambda)(\lambda e-x)^{-1}\,d\lambda\,h^{n-1}=\sum_{n=1}^{\infty}\frac{\tilde{f}^{(n)}(x)}{n!}\,h^{n-1}, $$250 DANACII ALCEBRAS AND SPECTRAL THEOR where on f and T) times $M^{n}$ is the nth derivative of /, and ${\tilde{f}}^{(n)}$ is an abbreviation for $\left[f^{(n)}\right]^{.}$ The norm ${\mathcal{f}}^{(n)}$ in the last power series is dominated by aconstant(depending of the coefficicnt of $h^{n-1}$ The eris cnves threfore.in norm. By 4) and (6), uh power seres representation (7) $$ \tilde{f}(x+h)=\sum_{n=0}^{\infty}\frac{1}{n!}\tilde{f}^{(n)}(x)h^{n} $$ holds if $x h=h x$ and llisuficintly small.(See Exercise 9) The following facts have now been proved: 10.36 Theorem Suppose A is a commutative Banach algebra, $\Omega\subset{\mathcal{C}}$ is open, xe $4_{\Omega},$ $a n d f\in H(\Omega)$ Then there exists $\delta>0$ such that (D) $$ \tilde{j}(x+h)=\sum_{n=0}^{\infty}\frac{1}{n!}\tilde{j}^{(n)}(x)h^{n} $$ for al ${\mathfrak{h}}\in{\mathfrak{A}}$ with $\|h\|<\delta$ Consequently, (2) $$ ({\cal D}{\tilde{f}})_{x}(h)={\tilde{f}}^{\prime}(x)h\qquad(h\in{\cal A}). $$ ln other words, the operator(Df)。∈ LW(A) is multiplication by f'(x) This is, of course, exactly as in th clasical casc $A=C$ We now consider the noncommutative situation. algebra $\scriptstyle A$ will now be denoted by Commutators Left and right multiplication by an element $\textstyle K_{x}$ _, respectively. Since the associative law of a Banach 10.37 $\chi_{\downarrow_{-}}$ plicr $\textstyle\exp_{z}.$ In particular, $L_{x}$ and $\textstyle R_{x}$ $L_{\mathrm{x}}$ ,and commutes with every right multi $y(x z)=(y x)z$ holds in ${\bar{A}},$ every left multiplie $L_{\mathrm{y}}$ commute with each other and with the operator (1) $$ C_{x}=R_{x}-L_{x}.\, $$ Note that $C_{x}(y)=y x-x y.$ the so-called commutator of ${\mathcal{B}}(A).$ lt is easilysen that $\mathbf{\nabla}y$ y and x Of course, $L_{x}.$ $R_{x}.$ , and $C_{x}$ are members of (2) $$ \sigma(L_{x})=\sigma(x)=\sigma(R_{x}) $$ and that $\|C_{x}\|\leq2\|x\|.$ Some further information about $\sigma(c_{x})$ will be obtained in the corollary to Theorem 11.23. 10.38 Theorem Suppose $\scriptstyle A$ is a Banach algebra, $\mathrm{Q^{2}}$ $\dot{\boldsymbol{\imath}}S$ open in。 $\zeta_{s},x\in A_{\alpha}$ and f e $H(\Omega)$ Then fis a continuously differentiable mapping o/ ${\mathcal{A}}_{\Omega}$ into $A_{\mathrm{{J}}}$ and $$ \quad(1)\qquad\qquad\qquad(D\bar{f})_{x}(y)=\frac{1}{2\pi i}\int_{{\Gamma}}f(\lambda)(\lambda e-x)^{-1}y(\lambda e-x)^{-1}\,d\lambda\qquad(y\in{\cal A}) $$ $i f\cap i s$ any contour that surrounds o(x) in QLBANACH ALGEBRAs 251 The operator $(D{\tilde{f}})_{x}$ can alsobe represented by the 8(A-aledinera (2) $$ (D{\tilde{J}})_{x}={\frac{1}{2\pi i}}\int_{\Gamma}f(\lambda)(\lambda I-R_{x})^{-1}(\lambda I-L_{x})^{-1}\,d\lambda $$ and by thediffrence quotien (3) $$ ({\cal D}\tilde{f})_{x}=(Q\tilde{f})({\cal L}_{x};\,C_{x}). $$ I Q conains ${\mathfrak{a l}}\wedge$ with $|\lambda|\leq3\|x\|,$ then (4) $$ (D\tilde{f})_{x}=\sum_{m=1}^{\infty}\frac{1}{m!}\,\tilde{f}^{(m)}(x)C_{x}^{m-1}. $$ into ${\mathcal{B}}(A)\,;$ The notation used in $(3)$ is perhaps not explicit enough. On the lef sideof(3, ${\mathcal{B}}(A)_{\alpha}$ f is a function from ${\mathcal{A}}_{\Omega}$ into A;onth right sid $\tilde{f}$ stands for fucton fro both sides of (3) represent members Of ${\mathcal{R}}(A).$ of PRor f $M>\|(\lambda e-x)^{-1}\|$ for al $\lambda$ on ${\mit\Gamma}$ and if $2M\|y\|<1,$ Theorem 10.1 $\left|{\mathcal{F}}\right|$ can be defined by shows that the norm of the dfence btwcc ${\hat{f}}(x+y)-{\tilde{f}}(x)$ and tsmximni on every Banach algebra, and hence in ${\mathcal{B}}(A),\,g$ is continuous on $\mathbf{T},$ and the right $\textstyle\Gamma.$ side r( i amost 2M*1-utied y hengn o ${\boldsymbol{\Gamma}}$ $\lceil\operatorname{et}\ g(\lambda)\in{\mathcal{B}}(A)$ This proves the formula (1). be th intcgrand in 2). sineinvson is coninous n and $T\in{\mathcal{P}}(A)$ (5) $$ T=\int_{\Gamma}\theta(\lambda)\ d\lambda. $$ But (S) implie (6) $$ T y=\int_{\Gamma}g(\lambda)y\,d\lambda\qquad(y\in A), $$ for ${\mathcal{S}}\in{\mathcal{B}}(A),$ then $F_{1}\in{\mathcal{B}}(A)^{\aleph},$ , and is exactly the integrand in (1),(6) shows tha if $y\in A,$ , and if $F_{1}S=F(S y)$ and since $\theta(\lambda)y$ (the dual space of ${\mathcal{A}}{\big\}},$ $T=2\pi i(D{\bar{J}})_{x}$ Tis proves 2) $F\in{\mathcal{A}}^{*}$ Definition 3.26: 1f It may be wotwhie indicte n detai how(O follows from S) an $$ F(T y)=F_{1}T=\int_{\Gamma}F_{1}g(\lambda)\,d\lambda=\int_{\Gamma}F[g(\lambda)y]\,d\lambda=F\int_{\Gamma}g(\lambda)y\,d\lambda. $$ $\sigma(x_{a})$ in $\Omega,$ Let us return to (2)、f ${\mathcal{X}}_{n}$ in the integrand. Since Discard these. Then $\boldsymbol{\Gamma}$ used in $\left(\mathbf{2}\right)$ will surround (2), if xis replaced by $x_{n}\to x,$ the contour $(D{\hat{f}})_{x_{n}}{\mathrm{i}}$ is given by for all bt finitely many ${\boldsymbol{n}}.$ (T) $$ (\lambda e-x_{n})^{-1} arrow(\lambda e-x)^{-1}\qquad\mathrm{as}\quad n arrow\infty, $$252 BANACH ALGEBRAS AND SPECTRAL THEORY uniformly on $\textstyle\Gamma_{:}$ the integrands in(2) converge uniformly. We conclude that $x\to(D/)_{*}$ is continuous. Thus $\tilde{f}$ is continuously differentiable If Since $R_{x}=L_{x}+C_{x},\,($ 3) is just another way of writing (2). and ${\boldsymbol{\Gamma}}$ can be chosen in (2) so as to be a circle with radius $r\gg3\|x\|$ center at $0_{\mathrm{,}}$ then $$ \|(\lambda I-L_{x})^{-1}\|=\left\|\sum_{n=0}^{\infty}\lambda^{-n-1}L_{x}^{n} \|\leq\sum_{n=0}^{\infty}r^{-n-1}\|x\|^{n}={\frac{1}{r- ||x|\right|}} $$ for every $\boldsymbol{\lambda}$ on $\boldsymbol{\Gamma}$ ,so that (8) $$ \|(\lambda I-L_{x})^{-1}\|\,\|C_{x}\|\leq\frac{2\|x\|}{r-\|x\|}<1. $$ By (8), the computation (G) of Section 10.35 can now be applied t $(Q{\hat{J}})(L_{x};C_{x}).$ t yicls (9) $$ (Q\tilde{f})({\cal L}_{x};\,C_{x})=\sum_{m=1}^{\infty}\,\frac{1}{m!}\tilde{f}^{(m)}({\cal L}_{x})C_{x}^{m-1}. $$ Finally,(( follows from (3)) and (9), because (10) $$ \begin{array}{c}{{\tilde{g}(L_{x})y=\frac{1}{2\pi i}\int_{\Gamma}g(\lambda)(\lambda I-L_{x})^{-1}y\,d\lambda}}\\ {{}}&{{=\frac{1}{2\pi i}\int_{\Gamma}g(\lambda)(\lambda e-x)^{-1}y\,d\lambda=\tilde{g}(x)y}}\end{array} $$ for every $y\in{\mathcal{A}}$ and every $g\in H(\Omega)$ , hence in particular for every /"” // This completes the proof. The serie (4) may actulyfai to converge i $\boldsymbol{f}$ has a sinuaitat distanc 3x|l fom the origin. An example o this s described in Exercise 22. The constant that occus n th last part of Thcorem 10.38 is thereforc th correct one. If A is commutative, then 《 $C_{x}=0.$ The term with $\scriptstyle{m=1}$ is then the only one that remains in the series 4). This agrees with Theorem 10.36. The following theorem willallw us to xtract information about local mapping properties of the functions from Theorem 10.36. 10.39 The inverse function theorem Suppose (a)W is an open subset of a Banach space X b) F:W-X is continuously differentiable (C for every xe W,(DFP),is an inertible member of .W(X)BANACH ALOEBRAs 253 Erery point $\varepsilon\,m$ has then a neighborhoo $U$ such that (1D) ${\mathbf{}}F$ is one-to-one in $U,$ $X.$ Gi) $F(U)=V$ is am open subset o (ii) F-1: $V\to U\ i s$ contimos dfeniable PRoor If ae W, if Tne cou n as esiey tysnun , and if ns fldion ${\mathbf{}}F$ $T=(D F)_{a}$ (1) $$ f(x)=T^{-1}[F(a+x)-F(a)]\qquad(x\in M-a), $$ $F\mathrm{\by}f.$ then sats th hyothess o theorem, i $F.$ We may therforereplac $\boldsymbol{f}$ satifis th cusion. te same vi Stueo $W-a$ in place of W.I1f l oterwrs sumewitit s neniy ush (2) $$ 0\in W,\ \ \ \ \ F(0)=0,\ \ \ \ \ (D F)_{0}=I $$ $\operatorname{Fix}\,\alpha,\,0<\alpha<1$ and w havetoproc tht as a nihbrhood $\ U$ that stisfies i),(i), and(ii Define (3) $$ \phi(x)=x-F(x)\qquad(x\in W). $$ ·Then $(D\phi)_{0}=0$ and since $\phi$ is continuously difentable i ${\mathcal{W}},$ there is an open ball $B\subset W,$ centered at $0_{\mathrm{,}}$ such that (4) $$ \|(D\phi)_{x}\|<\alpha\qquad{\mathrm{if}}\quad x\in B. $$ Suppose x'e B,x"∈ B $$ x_{t}=(1-t)x^{\prime}+t x^{\prime\prime}, $$ and (5) $$ \begin{array}{c c c}{{\psi(t)-\phi(x_{t})}}&{{}}&{{(0\leq t\leq1).}}\end{array} $$ Then ${\mathcal{Y}}\colon$ [O, $1\rfloor\to X$ is coninuousy ifentab (6) $$ \psi^{\prime}(t)=(D\phi)_{x}(x^{\prime\prime}-x^{\prime})\qquad(x=x_{t}) $$ by the chinrule, an e 4 impi (7) $$ \|\vartheta^{\prime}(t)\|\leq\vartheta\|x^{\prime\prime}-x^{\prime}\|\,; $$ note that $x_{t}\in B,$ by the convexity o ${\boldsymbol{B}}.$ (See Exercise 10.) Since (8) $$ \phi(x^{\prime\prime})-\cdot\phi(x^{\prime})=\psi(1)-\psi(0)=\int_{0}^{1}\psi^{\prime}(t)\,d t, $$ wc conclude from (T) that $\phi~~~~~~~~~~~~~~~~~~~~~~~~~~~~~~~~~~~~~~~~~~~~~~~~~~~~~~~~~~~~~~~~~~~~~~~~~~~~~~~~~~~~~~~~~~~~~~~~~~~~~~~~~~~~~~~~~~~~~~~~~~~~~~~~~~~~~~~~~~~~~~~~~~~~~~~~~~~~~~~~~~~~~~~~~~~~~~~~~~~~~~~~~~~~~~~~~~~~~~~~~~~~~~~~~~~~~~~~~~~~~~~~~~~~~~~~~~~~~~~~~~~~~~~~~~~~~~~~$ satisfies the Lipschitz conditio (9) $$ \begin{array}{c c c}{{\|\phi(x^{\prime\prime})\ -\ \phi(x^{\prime})\|\,}}&{{\qquad(x^{\prime}\in B,\,x^{\prime\prime}\in{}_{l}}}\end{array} $$ B). t now follows from (3) tha (10) IFKx")-F(x)1 2(1- a)|”-x (x'e B,x"∈ B)254 BANACH ALGEBRAS AND SPEC CTRAL THEORY This implics that ${\mathbf{}}F$ is onc-to-onc in ${\boldsymbol{B}}.$ Also, if $G\colon F(B) arrow B$ is defined by $G(F(x))=x,$ then(10) shows that ${\boldsymbol{G}}$ is continuous exist so that Our next aim is to show that Put $x_{0}=0,\ x_{1}{}^{\prime}=y.$ Supposc $\scriptstyle n\geq1.$ and $X_{0},\ \cdot\cdot\ ,\ \ {\mathcal{X}}_{n}$ Fix $y\in(1-\alpha)B$ $F(B)_{\parallel}\lnot(1-\alpha)B.$ (11) $$ x_{i}=y+\phi(x_{i-1})\qquad(1\leq i\leq n) $$ and (12) $$ \|x_{i}-x_{i-1}\|\leq\alpha^{i-1}\|y\|\qquad(1\leq i\leq n). $$ (These conditions hold when $n=1.3$ By (12), (13) $$ \|x_{n}\|\leq\sum_{i=1}^{n}\|x_{i}-x_{i-1}\|\leq\sum_{i=1}^{n}\alpha^{i-1}\|y\|\leq(1\ -\alpha)^{-1}\|y\|, $$ so that $x_{n}\in B,$ $\phi(x_{n})$ exists, and one can define (14) $$ x_{n+1}=y+\phi(x_{n}). $$ It follows from(14),(11), and (9) that (15) $$ \|x_{n+1}-x_{n}\|=\|\phi(x_{n})-\phi(x_{n-1})\|\leq\alpha\|x_{n}-x_{n-1}\|. $$ If $V=(1-\alpha)B$ and Our induction hypotheses hold now with $\textstyle n+1$ in place ot ${\mathfrak{n}},$ and the and for all n, Since $\scriptstyle x\,<\,1.$ construction can proceed, to yield a sequence $\langle x_{n}\rangle$ that satisfies (I1) and(12) to some $x\in B,$ by (13). Now (12) shows that $\scriptstyle\{x_{n}\}$ is a Cauchy sequence, which converges $(\mathbb{1}\rfloor)$ and $(3)$ imply that $F(x)-y.$ $U=G(V)=B\cap F^{-1}(V),$ then conclusions $\left(i\right)$ Gi) hold. To complete the proof, we now show that C ${\boldsymbol{0}}$ is continuously differen- tiablc in ${\boldsymbol{V}}.$ Suppose yeV,y+keV, k ≠ 0,x= GGy),×+万= G(y + k); put $S=(D F)_{x}$ Then $$ \begin{array}{c}{{G(y+k)-G(y)-S^{-1}k-h-S^{-1}k}}\\ {{}}&{{=S^{-1}(S h-k)=-S^{-1}[F(x+h)-F(x)-S h].}}\end{array} $$ B ${\mathfrak{y}}\left({\mathfrak{l}}0\right),\left({\mathfrak{l}}-\emptyset\right)\mathbb{|}h |\leq\left\|k\right\|.$ Hence $$ {\frac{\|G(y+k)-G(y)-S^{-1}k\|}{\|k\|}}\leq\|S^{-1}\|{\frac{\|F(x+h)-F(x)-S h\|}{(1\cdot\alpha)\|h\|}}\,. $$ As $k\hookrightarrow0,$ (10) implies that $h arrow0,$ and since $S=(D F)_{x}$ the last inequality shows that $S^{-1}=_{\zeta}(D G)_{\gamma}.$ In other words (16) $$ ({\cal D}G)_{y}=[(D F)_{G(y)}]^{-1}\qquad(y\in V). $$BANACH ALGEBRAs 255 ${\mathbf{}}V$ Since ${\boldsymbol{G}}$ maps ${\mathbf{}}V$ continuously into $\beta(X)$ 、and since inversion is continuous in // ${\mathcal{P}}(X)$ (Theorem 10.12),(16) shows that $y arrow(D G)_{r}$ is a continuous mapping of into ${\mathcal{B}}(X).$ This completes the proof. $x\in A_{\alpha}$ 10.40 ’ Theorem Suppose $\scriptstyle{\mathcal{A}}$ is a commuaive Banach algebra、S is open in C, neighborhood and the derivative $f^{\prime}$ of some fe H(Q2) has no zero on o(x). Then xhas a is a diffeomorphism whose $U\subsetneq A_{\alpha}$ such that the restriction o/ $\bar{f}$ 10o $U$ range is an open subset of A PROOF By Theorem 10.28, ${\widetilde{f}}^{\prime}(x)$ is invertible in A. By Theorem 10.36, ${\mathcal{A}}_{\Omega}$ continuously into $\scriptstyle(D f)_{1}.$ ${\mathcal{B}}(A).$ is therefore invertible in ${\mathcal{B}}(A).$ Since $y arrow(D{\tilde{f}})_{y}$ maps // neighborhood in which and since the invertible members of ${\mathcal{R}}(A)$ form an open set, x has a ,from Theorem 10.39 $(\,{\cal{D}}{\tilde{f}}\,)_{y}$ is invertible. The conclusion follows therefore true, namely, that fis an open mapping of Note thathe thcorem just proved does not assert what one might expect to be $\scriptstyle{\mathcal{X}}$ at which $\textstyle{\mathcal{A}}$ is dropped from the assump- is not constani does notimply that fis open at $\textstyle{\mathcal{X}}$ ${\mathcal{A}}_{\Omega}$ , into $\scriptstyle A$ whenever $f\in H(\Omega)$ in some open set that contains $\sigma(x)$ is open near points This is actually false;se Exercise 13 for an example. The $(D{\tilde{J}})_{x}$ is invertible in any component of $\Omega.$ theorem does prove tha $\overline{{{\int}}}$ has no zero on $\sigma(x)$ means $t\mathrm{hat}~f$ is locally one-to-one The hypothesis $\operatorname{that}f^{\prime}$ if commtativity o . Theorem 10.42 will sow that this local condition tions. An analogous global theorem does, however, turn out to be true: 10.41 Theorem Suppose $\scriptstyle A$ is a Banach algebra, $\mathbb{Q}$ is open in $C,f\in H(\Omega),$ and f is one-1o-one in S. Then $\widetilde{f}$ is a diffeorphism o $A_{\Omega}$ onto $A_{r i\alpha_{1}}$ PROOF Let $g\colon f(\Omega) arrow\Omega$ be the inverse of f. Since $g\circ f\operatorname{and}f\circ g$ z are the identity mappings in $\mathbb{C}\mathbf{2}$ p and in $f(\mathbb{Q}).$ respectively, Theorem I0.29 shows that ${\tilde{g}}\circ{\tilde{f}}$ is the identity in $\scriptstyle A_{\alpha}$ ,so that $\tilde{f}$ is one-to-one, and $\scriptstyle f_{\infty}\,g$ is the identity in $4_{r a s},$ so that $I/I/$ $A_{f(\Omega)}$ is the range c ${\mathfrak{o f}}j$ Since both fand its inverse $\tilde{g}$ are continuously differen- tiable (Theorem 10.38), the proof is complete 10.42 Theorem Suppose $A\bumpeq{\mathcal{B}}(X),$ where $\textstyle X$ is a complex Banach space and dim $X>1$ 1 Q is open in C, $i{\mathcal{J}}{\mathcal{J}}\in H(\Omega).$ ), and ${\mathcal{I}}{\mathcal{I}}$ is no1 one-to-one in Q2,then some $T_{0}$ S ${\mathcal{A}}_{\Omega}$ has a neighborhood ${\boldsymbol{U}}$ IJ such $t h a t\tilde{f}(U)$ contains no neighborhood $o f{\hat{f}}(T_{0})$ Thus $\tilde{f}$ is not open at $T_{0}$ say. Let ${\cal{Y}}$ By assumption, $\mathbb{C}$ contains two points $X,$ of codimension l; choose $x_{1},x_{2}\in X,$ PRO0F be a closed subspace or $\scriptstyle x\cdot\rho^{\beta}$ at which $f(x)=f(\beta)=c,$ $x_{i}\neq0,$ such that x is in Y but x, is not; and define T。 e A by (1) Tx, =αx, To = βy if yeY.256 BANACH ALGEBRAS AND SPECTRAL THEORY $$ \left. |\begin{array}{l l l l l}{{}}&{{}}&{{}}&{{}}&{{}}&{{}}\\ {{}}&{{}}&{{}}&{{}}&{{}}\\ {{}}&{{}}&{{}}&{{}}&{{}}\\ {{}}&{{}}&{{}}&{{}}&{{}}\end{array}\right| $$ 10.33 If 九 $\overline{{\tau}}$ $(\lambda I-T_{0})^{-1}.$ Thus $\sigma(T_{0})=\{\circ,\beta\}.$ and $T_{0}\in A_{\Omega}.$ By and of y by $(\lambda-\beta)^{-1}$ defines α and $\lambda\neq\beta,$ multiplication of ${\mathcal{X}}_{1}$ by $(\lambda-\alpha)^{-1}$ $\mathbf{\Psi}(a)$ of Theorem (2) $$ {\hat{f}}(T_{o})-c L. $$ by $X_{3}$ ,and let $\delta$ $\alpha\neq^{*}\beta.$ Let be the distance from $T_{0}x_{3}$ , let M be the one-dimensional subspace of $\textstyle X$ generated Put $x_{3}=x_{1}+x_{2}$ to $M.$ Then $\delta>0,$ since $T_{0}\,x_{3}=$ $x x_{1}+\beta x_{2}$ and $\Omega_{0}$ be the union of the components of $\Omega$ that contain α and ${\boldsymbol{\beta}}.$ (There are either one or two of these components.) Let $U$ be the set of all $T\in{\mathcal{A}}$ such that (3) $$ \|T-T_{0}\|<{\frac{\delta}{\|x_{3}\|}}\qquad{\mathrm{and}}\qquad\sigma(T)<\Omega_{0}\,. $$ contain is a neighborhood of $T_{0}$ We shall prove that ${\tilde{f}}(U)$ does not Then $U$ if $\eta\neq0$ and i $S\in{\mathcal{A}}$ is defined by ${\tilde{f}}(T_{0})+\eta S$ (4) $$ {\cal S}x_{1}=x_{3}\,,\qquad{\cal S}y=0\qquad\mathrm{for}\quad y\in Y. $$ We argue by contradiction. Suppose $\sigma(T)\subset\Omega_{0}\,,\,\eta\neq0,$ and (5) $$ {\tilde{f}}(T)={\bar{f}}(T_{0})+\eta S=c I+\eta S. $$ Then (6) $$ \tilde{f}(T)x_{3}=(c+\eta)x_{3}\,,\qquad\tilde{f}(T)y=c y^{\prime}\qquad\mathrm{for}\quad y\in Y. $$ $\scriptstyle{\mathcal{Y}}$ of hence equals $M.$ is an eigenvalue of ${\tilde{f}}(T),$ with eigenspace $M.$ Since $f-(c+\eta)$ $\bar{M},$ Thus $c arrow\eta$ $\left(c\right)$ of Thcorcm $\scriptstyle0.33$ implics that $c\circ\iota-\eta=f(\gamma)$ for some eigenvalue vanishes neither at α nor at ${\boldsymbol{\beta}},$ it does not vanish identicy any componcnt o $\Omega_{0},$ and By $(\alpha)$ of Theorem 10.3, the corresponding eigenspace lies in Our choice of S implies ${\boldsymbol{T}}.$ , since dim $M=1.$ Thus $T x_{3}\in M.$ now that (7) $$ \delta<\|T x_{3}-T_{\mathrm{o}}\,x_{3}\|\leq\|T-T_{\mathrm{o}}\|\,\|x_{3}\|. $$ Hence ${\mathbf{}}T$ is not in ${\boldsymbol{U}}.$ // 10.43 The exponential function To illustrate the preceding results, let us see what they tell about the exponential function, defined by the power series $$ \exp\left(x\right)=\sum_{n=0}^{\infty}{\frac{1}{n!}}\,x^{n} $$ in every Banach algebra $A.$ (See also Theorem $10.30.30.3{}_{!}$BANACH ALGEBRAs 257 (b) The Fréchet derivative of exp at $\textstyle{\mathcal{X}}$ (o)I ox) contains no two points whose difece ninegral muliple of xinto ${\bar{A}},$ by Theorem 10.41. 2mi, the compactness ot oxy shows ta exp s n-one nsme opens is ${\mathcal{A}}_{\Omega}$ $\Omega\Rightarrow\sigma(x);$ hencc cxpis a diffeomorphism o he neighborhood $$ (D\exp)_{x}=\exp\left(x\right)\tilde{\Phi}(C_{x}), $$ where $\Phi$ bis the entire function defined by $$ .\qquad\Phi(\lambda)={\frac{\exp\left(\lambda\right)-1}{\lambda}}\,. $$ then The zeros of $\Phi$ This follows from the last part of Theorem 10.38,sinc $k=\pm1,\;\pm2,\;\ldots.$ If none of these lies in for $m\geq1$ when $\Phi(C_{x})$ are at 2kri $f^{(m)}=f$ $\sigma(c_{x}),$ $f(\lambda)=\exp\,(\lambda).$ cxp is again a diffeomorphism near x. is invetibe by the spectral mapping theorem, and soi $(D\exp)_{x},$ and (o)We shall selater (Theorem 11.23) tha $$ \sigma(C_{x})\subset\sigma(x)-\sigma(x). $$ (d) If $\textstyle A$ Thsproids in betwe the preding paragraphs (ob and $(b).$ $x\in{\mathcal{A}}.$ since it is $A={\mathcal{R}}(X),$ is commutative, then $(D\exp)_{x}$ is invertible, for cvcry $A.$ (Theorem 10.36.) $A.$ simply multiplication by exp (x)、an invertible member of $A=C.$ However,if Hence cxp is a local difeorphism, as in the familiar cas into as in Theorem 10.42,then exp is not an open mapping of $\scriptstyle A$ The Group of Invertible Elements ${\cal G}_{1}$ Sometimes ${\widehat{G}}_{1}$ We shall now take a closer lok at the structure of ${\dot{A}}.$ $G=G(A),$ the multiplicative group ${\widetilde{D}}.$ $G_{\mathrm{i}}$ of all invetible elements of a Banach algebra that contains ${\cal G}.$ By the defnition of componcnt, wil denote the componcnt of ${\boldsymbol{G}}$ $\bar{G}$ that contain e. the identity element of is called the principal component of ${\mathcal{C}},$ is the union of all connected subsets of The group ${\boldsymbol{G}}$ F contains the sct $$ \exp\,(A)=\left\{\exp\,(x)\!:\,x\in A\right\}, $$ functional equation the range of the exponential function in ${\bar{A}},$ simply because exp $(-x)$ is the inverse of exp Ox)、 In fact, the power series definition of cxp (x) <se Section 10.43) yields the $$ \exp\,(x+y)=\exp\,(x)\exp\,(y), $$ provided that xy = yx; also, exp (0) =258 BANACH ALGEBRAS AND SPECTRAL THEORy Note also that ${\boldsymbol{G}}$ is atpologica group se Section 5.2 ince uliplication an inversion are continuous in ${\cal G}.$ 10.44 Theorem (a) $G_{1}$ is an open normal subgroup oJ G. ${\mathfrak{b}}\gamma$ exp(4). contains no element of fnite order (6) $G_{1}$ is the group generated $G_{i}=\exp{(A)}.$ $G/G_{1}$ (c) IJ A is commutative, then (d) IJ Aiscommunative,th quoient growp (except for the ideniy) ball $U\subset G.$ Since $U_{\mathit{l}}$ Theorem 10.11 shows that every ${\tilde{G}}_{1}$ and $U$ is connected, $U\subset G_{1}.$ Therefore $G_{1}$ PROOF (a) intersects is the center of an open $x\in G_{1}$ is open. Also, If $x\in G_{1}$ then $x^{-1}G_{1}$ is a connected subset of ${\boldsymbol{G}}$ which contains $G_{1}$ is a normal ${\cal G}.$ contains $y^{-1}G_{1}y$ is homeomorphic to ${\cal G}_{1},$ hence connected, for every $\nu\in G$ $x^{-1}x=e.$ Hence $x^{-1}G_{1}\subset G_{1},$ for every $x\in0_{1}$ This proves that $G_{1}$ is a subgroup of $\scriptstyle{\mathcal{C}}$ Thus $y^{-1}G_{1}y<G_{1}$ By definition, this says that ,and subgroup of $\ {\bar{\boldsymbol{P}}}$ ${\mathcal{C}},$ and so homeomorphism of ${\boldsymbol{G}}$ Hence ${\mit\Gamma}$ bc the group generated by exp $E_{n}\,$ 。Since the product of any two connected eexp (A) whenever contains $\mathbf{\nabla}(\partial)$ Let $\mathbf{I}$ $(A)$ For $n=1,\,2,\,3,\,\ldots,\,\vert\mathrm{et~}E_{n}$ be the se of all prducts of n members of $\exp{(A)}.$ Since $y^{-1}$ y e exp (A), T is the union of the sets $E_{n}\subset G_{1}$ scts i conncted, induction shows that each ${\boldsymbol{\Gamma}}$ is a group and since mutplication by any (see Theorem 10.30): is a hence so has T. Since Next. exp A) has nonempty interor, relaive to ${\boldsymbol{G}}$ is connected. Each $E_{n}$ $x\in G$ is a subgroup of $E_{n}$ $G_{1}.$ onto G, r is open (c)If $\textstyle{A}$ Each coset of T in G ${\cal G}_{1}$ is therefore open,and so is any union of these cosets is closed, relative to $G_{1}.$ Thus Sic F s the coumplement of a union f ts csets $G_{1}.$ Since ${\bar{G}}_{1}$ is connected $\Gamma=G_{1}.$ ${\boldsymbol{\Gamma}}$ is an opcn and closd subset or $\dot{\mathbf{I}}$ that exp (A) is a group. Hence $\mathbf{\nabla}(D)$ implies (c) is commutative, the functional eqution satisfed by cxp shows (d)We have to prove th fllowing propositio then $x\in G_{1},$ If A is commutative,i $x\in G,$ and i $x^{n}\in G_{1}$ for some positive integer n, $(n^{-1}a)$ and Under these conditions Since $y\in G_{1},$ it suffices to prove that $a\in A,$ , by Gc). Put $y=\exp$ The commutativity of $\textstyle A$ $x^{n}=\exp\,(a)$ for some $z\in G_{1},$ $\scriptstyle{Z}=x y^{-1}$ shows that $$ z^{n}=x^{n}y^{-n}=\exp\left(a\right)\exp\left(-\,a\right)=e. $$BANACH ALGEBRAS 259 Pu $[{\bf\nabla}f(\lambda)=\lambda z-(\lambda-1)e,$ If $\lambda\notin E,$ it follows that $(\lambda-1)^{n}=\lambda^{n}.$ This equation then and let $E=\{\lambda\in C;f(\lambda)\in G\}.\quad{\mathrm{If}}\ \ \alpha\in\sigma(z),$ $\alpha^{n}\in\sigma(z^{n})=\sigma(e)=\{1\}.$ has only $\scriptstyle n\,-\,1$ solutions in ${\mathcal{C}}.$ Hence $\boldsymbol{E}$ is connected. Consequently, /(E) is a connected subset of ${\boldsymbol{G}}$ which contains $f(0)=e.$ ${\mathrm{Thus}}\,f(E)\subset G_{1}$ In particular, $z=f(1)\in G_{1}$ // This completes the proof. Theorem 12.3 willshow that exp (A)is not always a group. Exercises Throughout this set of exercis, A denotes a Banach algebra. 厂 Suppose $x\in A,\,y\in A.$ (a) If $\scriptstyle{\mathcal{X}}$ and $x y$ are invertible in ${\cal A},$ prove that y is invertible left shifts $S_{R}$ and (b) If xy and yx are invertible in A, prove that $\scriptstyle{\mathcal{X}}$ and $\mathbf{\vec{y}}$ are invertible in $A.$ (The com- (e) Show that it is possible to have mutative case of this was tacily used in the proofs of Theorems 10.13 and 10.28.) For example, consider the right and on the non- ${\boldsymbol{S}}_{L}$ $x y=e\not arrow y x$ $\boldsymbol{\mathit{f}}$ .,defined on some Banach space of functions negative integers by $$ (S_{R}f)(n)={\biggl\{}_{f(n-1)}\quad{\begin{array}{l l}{{\mathrm{if~}}n-0}\\ {{\mathrm{if~}}n\geq1,}\\ {{(S_{2}f)(n)=f(n+1)}}&{{\mathrm{for~}n\geq0.}}\end{array}}\, $$ (d) If $x y\colon-e$ ≠ Jx, show that yx isa nontrivial idempotent (e) ${\mathfrak{I}}$ dim $A<\infty,$ show that $y x=e$ whenever $x y=e.$ 2 Supposc $x\subset A,\,y\subset A.$ (a) Prove that $e-y x$ is invertible if $e-x y$ is invertible. Hint::If $\overline{{\mathbb{Z}}}$ inverts $e-x y$ conside $\mathrm{r}\ e+y z x$ (b) If $\lambda\in C,\;\lambda\neq0,$ and $\lambda\in\sigma(x y),$ prove that 入∈ o(yx). Show,however, that o(xy) nay contain O although odyx) does not $\mathbf{\Psi}(c)$ 1f $\scriptstyle{\mathcal{X}}$ is invertible, show that $\sigma(x y)=\sigma(y x)$ 5 Call Suppose SQ is open in ${\mathcal{L}};f\colon\Omega arrow A$ and d $\ :\Omega\to C$ are holomorphic. Prove that $\phi f\colon\Omega arrow A$ 中 theorem 3.32 and the fact that is holomorphic. TThis was used inthe proof of Thcorem 10.13, wiuh $\phi(\lambda)=\lambda^{n}.\}$ Another proof of the theorem that $\scriptstyle{G(x)}$ is never empty can be based on Liouville' $(\lambda e-x)^{-1} arrow0$ as $\lambda\to\infty$ . Complete the dctails $x\in A$ a topological divisor of zero if there is a sequence {yn} in $A.$ ,with $|\vert y_{n}\vert|=1,$ such that $$ \operatorname*{lim}_{n arrow\infty}x y_{n}=0\Longrightarrow\operatorname*{lim}_{n arrow\infty}y_{n}\,x. $$ (a) Prove that every boundary point $\scriptstyle{\mathcal{X}}$ of the set of all invertible elements of $\textstyle{\mathcal{A}}$ lis a $\mathbf{\nabla}(b)$ topological divisor of zero. Hint: Take $y_{n}=x_{n}^{-1}/\|x_{n}^{-1}\|_{}.$ where $X_{n}-\succ X.$ Which Banach algebras have no topological divisors of zero other than O?260 RANACH ALGEBRAS AND sPECTRAL THEoRy 6 Supposc $K=\{\lambda\in C\colon1\leq|\lambda|\leq2\};$ put $f(\lambda)-\lambda.$ Let $\scriptstyle{A}$ be the smallest closcd sub- that contains fand 1/. Describe the spectra $\sigma_{A}(J)$ and $\sigma_{s}/f)$ be the smallest closed subalgebra of C(K) algebra of C(K)that contains l and $f.$ Let $\boldsymbol{B}$ Do the same when $\textstyle K$ is a circle 7 Strengthen the continuity assertion in $\operatorname{\left(1\right)}$ of Theorem 10.27 in the following way: If $\textstyle K$ is any compact subset of $\Omega,$ and if $$ A_{K}=\{x\in A\colon\sigma(x)\subset K\}, $$ then ${\tilde{f}}_{n}(x)\to{\tilde{f}}(x)$ uniformly on $\scriptstyle A_{\mathrm{s}}$ 8 (a) Fubini's theorem was applied to vector-valued integrals in the proof of Theorem 10.29. Justify this. (b) Construct another proof of Theorem 10.29, that uses no contour integrals, as follows: Prove the theorem first for polynomials $\scriptstyle{\mathcal{G}}$ and then for rational functions $g\in H(\Omega_{1}),$ and obtain the general case from Runge's theorem 9 In the computation (6) of Section 10.35, integration by parts was applied to a vector- valued integral. Justify this $I O$ Prove the version of the chain rule that was used in the proof of Theorem 10.39, and prove the fundamental theorem of calculus for vcctor-valucd integrals, as used in (8) of Theorem 10.39. ${\mathcal{I}}$ Prove in detail that the convergence in (T) of Theorem 10.38 is indced uniform on T ${\mathit{1}}_{}^{}{\bigg(}{\mathit{X}}$ Suppose $\boldsymbol{k}$ k is a positive integer $\omega=\exp\left(2\pi i/k\right),$ $\operatorname{and}f\colon A\to A$ is defined by $f(x)=x^{k}.$ condition (a) Prove that fis a diffeomorphism in some neighborhood of $x_{n}\in A$ if xo satisfies the $$ \sigma(x_{0})\cap\omega^{n}\sigma(x_{0})=\mathcal{T}\qquad\mathrm{for}\ n=1,\ldots,k-1. $$ (b) Prove that the same conclusion holds if $\scriptstyle A$ is commutative and $\scriptstyle x_{0}$ ois invertible in $\scriptstyle{\mathcal{A}}$ ${\mathcal{L}}{\mathcal{S}}$ Let $\scriptstyle{A}$ d be the algebra of all matrices of the form $$ \left( (\begin{array}{l l}{\alpha}&{\beta}\\ {0}&{\alpha}\right) $$ with αe C, $\beta\epsilon\,\epsilon.$ Show that $\vert\,\alpha\,\vert\,+\,\vert\,\beta\vert$ is a Banach algcbra norm on ${\bar{A}}.$ Dcfinc $f(x)=x^{2}$ for xe A. Find $\textstyle{f_{U O}}$ Is f(A) open in $A\ ?$ Is fan open mapping?[Compare with part (b) of Exercise 12.] $I{\boldsymbol{\lambda}}$ Show that every two-dimensional complex algebra $\scriptstyle A$ with unit $\scriptstyle{\mathcal{C}}$ is isomorphic either to ${\boldsymbol{C}}^{2}$ with coordinatewise addition and multiplication or to the algebra described in with $x^{2}-e;$ in the other, there Exercise 13. Hint: In one case there exists $x\neq x\pm e$ exists $\scriptstyle x\neq0$ with $x^{2}=0$ Prove that one of these must occur Show that there exists a thee-dimensional noncommutative Banach algebra ${\mathcal{L}}{\mathcal{L}}$ Prove the relation $$ \exp\left(C_{x}\right)=\exp\left(R_{x}\right)\exp\left(-L_{x}\right)\!, $$ and use it to derive the formula $$ \exp\,(-x)y\exp\,(x)=[\exp\,(C_{x})]y, $$ valid for xand yin any Banach algebra A. CThe notation is as in Scction 10.37.)BANACH ALGEBRAS 261 ${\mathcal{U}}{\mathcal{T}}$ 16 Suppose $A=C(I),$ the aler f l coninuous complex functoson th uitci ${\mathbf{}}T$ into te set oanonco $G/G_{1}$ is uncountable: 1 $\lnot\alpha\in R,$ let & $\delta_{\alpha}$ $\mu_{x}\in M(R);$ hence, for 。 be te uni mass cocentrated at z.Assum $\delta_{*}\in G_{1}$ ${\cal T},$ coset of $G_{1}$ with the suprcmum norm. Show that woivetemberso ${\mathsf{c}}(I)$ are in the same line; see $\mathbf{\Psi}({\boldsymbol{\varphi}})$ i if nd oly i thy are hotopi mainso is isomorphic o te aditvegoup of Suppose $A=M(R),$ complex numbers. Deduce from this that $\scriptstyle{G/G_{1}}$ the inters. (Thenotion s as inTheorem 1.4. the convolution algebra o al complex Borel measures on the rea of Exapl 1. upy h diais ntheooimeproi ha $\delta_{x}=\exp\left(\mu_{x}\right)$ for some $-\infty<I<\infty,$ Then $$ -i\alpha t={\hat{\mu}}_{\alpha}(t)+2k\pi i, $$ $I{\mathcal{B}}$ integer, and where kis an integer. Since in ${\boldsymbol{G}}$ B contains therefore more than one $\delta_{\alpha}$ Thus ${\mathfrak{P}}_{0}$ is the only $\delta_{\alpha}$ in $G_{1}$ No coset of ${\hat{\mu}}_{\alpha}$ is a bounded function $\scriptstyle\alpha=0$ $G_{1}$ Suppose $\mathbb{Q}$ X-valued function in $\Omega$ (wherc $\textstyle X$ cis an isolated boundary point of $\Omega,f\colon\Omega\to X$ is a holomorphic is open in ${\underline{{C}}},$ is some complex Banach space), $r_{\mathit{l}}$ is a nonnegative $$ |\lambda-\alpha|^{n}|f(\lambda)| $$ ${\mathbf{}}P$ is bounded as $\lambda{ arrow}x$ Then fis said to have a pole (of order $\leq n\rangle$ at α .(a) Suppose $x\!\in\!A$ and $(\lambda e-x)^{-1}$ has a poleat every point of o(x). INote that this such that $P(x)=0.$ can happnonly when ox)sa fniteset] Prove thathere is nontiva olynoma has a pole of order n at O. Prove that ()Asaspecia case f a), assume c $(x)=\{0\}\operatorname{and}\,(\lambda e\quad x)^{-1}$ $x^{n}=0.$ 19 Let $S_{R}$ be the right shit, acting on but $c_{n} arrow0$ as $n\to\varnothing$ .Define $M\in{\mathcal{R}}(Y^{2})$ by numbers such that $c_{n}\neq0$ ${\mathcal{E}}^{2},$ , as in Exercise 1. Let fco} be a sequence of complex $$ (M f)(n)=c_{n}f(n)\qquad(n\geq0), $$ and define $T\in{\mathcal{D}}(\ell^{2})$ by $T=M S_{K}$ (a) Compute $\|T^{m}\|_{},$ for $m=1,\,2,\,3,\,\ldots\,{.}$ $\mathbf{\Psi}(b)$ Show that $\sigma(T)=\{0\}$ (c) Show that $\boldsymbol{\mathit{I}}$ rhas no eigenvaluc. Cts point spectrum isthefore empty, alhough it spectrum consists of a singlc point!) (d) Show that $(\lambda I-T)^{-1}$ does not have a pole at O (e) Show that ${\mathbf{}}T$ tis a compact operator 20 Suppose xe A, $x_{n}\in A,$ and lim $x_{n}\equiv x.$ Suppose $\Omega$ is an open set in $\Omega_{0}$ is an open set disjoint strengthens Theorem 10.20.)Hint: 1f $\sigma(x)\subset\Omega\cup\Omega_{0}\,,$ ${\boldsymbol{C}}$ that contains a component of o(x). rove ihat otx,) intesects for al sfiety large .(This where $2{\cal{I}}$ from S $\Omega_{\mathrm{{,}}}$ 2, consider the function f that is 1 in Q,0 in $\mathbb{C}_{0}$ Let $C_{R}$ be the alebra of al real continuous functions on I0,1, with the supremum now real norm. Thisats al equirements of a Banach alebra exet that the scarsar a If AD =f8 tD d, then 40 =1, and $\phi(f)\neq0$ if is invertible in $C_{R}\,,$ but $\phi$ is not multiplicative262 oac" Auoenxs s srcxL meov (b)If ${\boldsymbol{G}}$ F and ${\boldsymbol{G}}_{\mathrm{i}}$ i are defined in $C_{R}$ as in Theorem 1.44, show that $G/G_{1}$ is a group of order 2. 22 Let normed b $||(\alpha,\beta)||=|\alpha|+|\beta|.$ The aloes o heoem 1. and o Theorem 10. usase or rea $\scriptstyle A$ A aS ${\mathcal{R}}(C^{2})_{!}$ where ${\boldsymbol{C}}^{2}$ is $\scriptstyle{A}$ scalu Excty where od te ror o Xo Theoim 1.4 te iown Dtfine xe A by L e hloer l opie 2- mtecs er CThis determines the norm on ${\dot{A}}.{\dot{\gamma}}$ $$ x={\binom{1}{0}}\quad-1 \}. $$ (b)If (ao) Find Il, 0x), and $e^{i(G_{n})_{n}}$ so that $\ ^{t}\subset C,\;t^{2}\neq\;1,$ and $f(\lambda)=1/(t-\lambda),$ $$ \tilde{f}(y)=(t e-y)^{-1}\qquad(y\in A,\,t\notin\sigma(y)), $$ compute $(D f)_{\sim}$ and show that $\textstyle\sum\left(n!\right)^{-1}{\tilde{f}}^{(s)}(x)C_{x}^{n-1}$ converges if and only $\mathbf{i}\mathbf{f}^{'}$ $|t-1|>2$ and $|t+1|>2.$ (The number ${\mathbf{3}}\,$ can therefore not be replaced by a Partial answer to smaller nein the ls part of Theorem 10.3.5 $\mathbf{\nabla}(b)$ $$ (D\tilde{f})_{x}{\binom{a}{d}}=\left(a/(t-1)^{2}\quad b/(t^{2}-1)\right).\qquad\qquad{\mathrm{~.~}} $$ $26$ Suppose $x\in{\mathcal{A}}$ and Use Exercisc $24$ 23 Whthapsn epoes adong antaecie Scion1 pie if $\textstyle w$ y is invertible in $A.$ Replace w by ${\mathcal{A}}_{1}$ $25$ (b) Prove that $\scriptstyle A$ $x^{n}=e$ to deduce that A hih alely has ni Cleary esuanot anangeb $(x y)^{n}=x(y x)^{n-1}y.$ $\|\leq M\|y x\|$ to an algebra $\scriptstyle A$ $2{\mathcal{A}}$ for all with two unis. Explain is commutative if ther is aconstant $\|x^{2}\|=\|x\|^{2}$ for every $x\in A.$ Hint:Show that exp (Ax), where $x\in{\mathcal{A}}$ Show that y nd x always haethe same specta radius. Hin such that lxy (a) Prove that $\scriptstyle{\mathcal{A}}$ in ${\cal A}.$ Hint: $\|w^{-1}y w||\leq M\|_{J}||\ .$ $M<\infty$ $\scriptstyle{\mathcal{X}}$ r and $\mathbf{y}$ is commutative if and Ac C. Continue as in Therem 12.16. $\|x\|=\rho(x).$ $||w^{-1}\nu w||=||\nu||.\stackrel{\cdot}{\ }($ Continueas in (a. (Thc nota for some positive inter m. Prove that $x\subset G_{1}.$ tion s as in Theorem 10.44 eplae henyoihes $x^{n}=e\,{\mathfrak{b}}$ y more general ones.