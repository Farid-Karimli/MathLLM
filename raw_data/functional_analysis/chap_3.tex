\documentclass[10pt]{article}
\usepackage[utf8]{inputenc}
\usepackage[T1]{fontenc}
\usepackage{amsmath}
\usepackage{amsfonts}
\usepackage{amssymb}
\usepackage[version=4]{mhchem}
\usepackage{stmaryrd}
\usepackage{mathrsfs}
\usepackage{bbold}

%New command to display footnote whose markers will always be hidden
\let\svthefootnote\thefootnote
\newcommand\blfootnotetext[1]{%
  \let\thefootnote\relax\footnote{#1}%
  \addtocounter{footnote}{-1}%
  \let\thefootnote\svthefootnote%
}

%Overriding the \footnotetext command to hide the marker if its value is `0`
\let\svfootnotetext\footnotetext
\renewcommand\footnotetext[2][?]{%
  \if\relax#1\relax%
    \ifnum\value{footnote}=0\blfootnotetext{#2}\else\svfootnotetext{#2}\fi%
  \else%
    \if?#1\ifnum\value{footnote}=0\blfootnotetext{#2}\else\svfootnotetext{#2}\fi%
    \else\svfootnotetext[#1]{#2}\fi%
  \fi
}

\begin{document}
\section{CONVEXITY}
This chapter deals primarily (though not exclusively) with the most important class of topological vector spaces, namely, the locally convex ones. The highlights, from the theoretical as well as the applied standpoints, are $(a)$ the Hahn-Banach theorems (assuring a supply of continuous linear functionals that is adequate for a highly developed duality theory), $(b)$ the Banach-Alaoglu compactness theorem in dual spaces, and $(c)$ the Krein-Milman theorem about extreme points. Applications to various problems in analysis are postponed to Chapter 5.

\section{The Hahn-Banach Theorems}
The plural is used here because the term "Hahn-Banach theorem" is customarily applied to several closely related results. Among these are the dominated exiension theorems 3.2 and 3.3 (in which no topology is involved), the separation theorem 3.4, and the continuous extension theorem 3.6. Another separation theorem (which implies 3.4) is stated as Exercise 3.

3.1 Definitions The dual space of a topological vector space $X$ is the vector space $X^{*}$ whose elements are the continuous linear functionals on $X$.

Note that addition and scalar multiplication are defined in $X^{*}$ by

$$
\left(\Lambda_{1}+\Lambda_{2}\right) x=\Lambda_{1} x+\Lambda_{2} x, \quad(\alpha \Lambda) x=\alpha \cdot \Lambda x
$$

It is clear that these operations do indeed make $X^{*}$ into a vector space.

It will be necessary to use the obvious fact that every complex vector space is also a real vector space, and it will be convenient to use the following (temporary) terminology: An additive functional $\Lambda$ on a complex vector space $X$ is called reallinear (complex-linear) if $\Lambda(\alpha x)=\alpha \Lambda x$ for every $x \in X$ and for every real (complex) scalar $\alpha$. Our standing rule that any statement about vector spaces in which no scalar field is mentioned applies to both cases is unaffected by this temporary terminology and is still in force.

If $u$ is the real part of a complex-linear functional $f$ on $X$, then $u$ is real-linear and

$$
f(x)=u(x)-i u(i x) \quad(x \in X)
$$

because $z=\operatorname{Re} z-i \operatorname{Re}(i z)$ for every $z \in \mathscr{C}$.

Conversely, if $u: X \rightarrow R$ is real-linear on a complex vector space $X$ and if $f$ is defined by (1), a straightforward computation shows that $f$ is complex-linear.

Suppose now that $X$ is a complex topological vector space. The above facts imply that a complex-linear functional on $X$ is in $X^{*}$ if and only if its real part is continuous, and that every continuous real-linear $u: X \rightarrow R$ is the real part of anique $f \in X^{*}$.

\subsection{Theorem Suppose}
(a) $M$ is a subspace of a real vector space $X$,

(b) $p: X \rightarrow R$ satisfies

$$
p(x+y) \leq p(x)+p(y) \quad \text { and } \quad p(t x)=t p(x)
$$

if $x \in X, y \in X, t \geq 0$

(c) $f: M \rightarrow R$ is linear and $f(x) \leq p(x)$ on $M$.

Then there exisis a linear $\Lambda: X \rightarrow R$ such that

and

$$
\Lambda x=f(x) \quad(x \in M)
$$

$$
-p(-x) \leq \Lambda x \leq p(x) \quad(x \in X)
$$

Proor If $M \neq X$, choose $x_{1} \in X, x_{1} \notin M$, and define

$$
M_{1}=\left\{x+t x_{1}: x \in M, t \in R\right\}
$$

It is clear that $M_{1}$ is a vector space. Since

we have

$$
f(x)+f(y)=f(x+y) \leq p(x+y) \leq p\left(x-x_{1}\right)+p\left(x_{1}+y\right)
$$

$$
f(x)-p\left(x-x_{1}\right) \leq p\left(y+x_{1}\right)-f(y) \quad(x, y \in M) .
$$

Let $\alpha$ be the least upper bound of the left side of (1), as $x$ ranges over $M$. Then

$$
f(x)-\alpha \leq p\left(x-x_{1}\right) \quad(x \in M)
$$

and

$$
f(y)+\alpha \leq p\left(y+x_{1}\right) \quad(y \in M) .
$$

Define $f_{1}$ on $M_{1}$ by

$$
f_{1}\left(x+t x_{1}\right)=f(x)+t \alpha \quad(x \in M, t \in R)
$$

Then $f_{1}=f$ on $M$, and $f_{1}$ is linear on $M_{1}$.

Take $t>0$, replace $x$ by $t^{-1} x$ in (2), replace $y$ by $t^{-1} y$ in (3), and multiply the resulting inequalities by $t$. In combination with (4), this proves that $f_{1} \leq p$ on $M_{1}$.

The second part of the proof can be done by whatever one's favorite method of transfinite induction is; one can use well-ordering, or Zorn's lemma, or Hausdorff's maximality theorem:

Let $\mathscr{P}$ be the collection of all ordered pairs $\left(M^{\prime}, f^{\prime}\right)$, where $M^{\prime}$ is a subspace of $X$ that contains $M$ and $f^{\prime}$ is a linear functional on $M^{\prime}$ that extends $f$ and satisfies $f^{\prime} \leq p$ on $M^{\prime}$. Partially order $\mathscr{P}$ by declaring $\left(M^{\prime}, f^{\prime}\right) \leq\left(M^{\prime \prime}, f^{\prime \prime}\right)$ to mean that $M^{\prime} \subset M^{\prime \prime}$ and $f^{\prime \prime}=f^{\prime}$ on $M^{\prime}$. By Hausdorff's maximality theorem there exists a maximal totally ordered subcollection $\Omega$ of $\mathscr{P}$.

Let $\Phi$ be the collection of all $M^{\prime}$ such that $\left(M^{\prime}, f^{\prime}\right) \in \Omega$. Then $\Phi$ is totally ordered by set inclusion, and the union $\tilde{M}$ of all members of $\Phi$ is therefore a subspace of $X$. If $x \in \tilde{M}$ then $x \in M^{\prime}$ for some $M^{\prime} \in \Phi$; define $\Lambda x=f^{\prime}(x)$, where $f^{\prime}$ is the function which occurs in the pair $\left(M^{\prime}, f^{\prime}\right) \in \Omega$.

It is now easy to check that $\Lambda$ is well defined on $\tilde{M}$, that $\Lambda$ is linear, and that $\Lambda \leq p$. If $\tilde{M}$ were a proper subspace of $X$, the first part of the proof would give a further extension of $\Lambda$, and this would contradict the maximality of $\Omega$. Thus $\tilde{M}=X$.

Finaliy, the inequality $\Lambda \leq p$ implies that

$$
-p(-x) \leq-\Lambda(-x)=\Lambda x
$$

for all $x \in X$. This compietes the proof.

3.3 Theorem Suppose $M$ is a subspace of a vector space $X, p$ is a seminorm on $X$, and $f$ is a linear functional on $M$ such that

$$
|f(x)| \leq p(x) \quad(x \in M)
$$

Then $f$ extends to a linear functional $\Lambda$ on $X$ that satisfies

$$
|\Lambda x| \leq p(x) \quad(x \in X)
$$

PROOF If the scalar field is $R$, this is contained in Theorem 3.2, since $p$ now satisfies $p(-x)=p(x)$.

Assume that the scalar field is $\ell$. Put $u=\operatorname{Re} f$. By Theorem 3.2 there is a real-linear $U$ on $X$ such that $U=u$ on $\bar{M}$ and $U \leq p$ on $X$. Let $\Lambda$ be the complex-linear functional on $X$ whose real part is $U$. The discussion in Section 3.1 implies that $\Lambda=f$ on $M$.

Finally, to every $x \in X$ corresponds an $\alpha \in \mathscr{C},|\alpha|=1$, such that $\alpha \Lambda x=$ $|\Lambda x|$. Hence

$$
|\Lambda x|=\Lambda(\alpha x)=U(\alpha x) \leq p(a x)=p(x)
$$

III

Corollary If $X$ is a normed space and $x_{0} \in X$, there exists $\Lambda \in X^{*}$ such that

$$
\Lambda x_{0}=\left\|x_{0}\right\| \quad \text { and } \quad|\Lambda x| \leq\|x\| \quad \text { for all } x \in X \text {. }
$$

PROOF If $x_{0}=0$, take $\Lambda=0$. If $x_{0} \neq 0$, apply Theorem 3.3, with $p(x)=\|x\|$, $M$ the one-dimensional space generated by $x_{0}$, and $f\left(\alpha x_{0}\right)=\alpha\left\|x_{0}\right\|$ on $M$. $\quad / / /$

3.4 Theorem Suppose $A$ and $B$ are disjoint, nonempty, convex sets in a topological vector space $X$.

(a) If $A$ is open there exist $\Lambda \in X^{*}$ and $\gamma \in R$ such that

$$
\operatorname{Re} \Lambda x<\gamma \leq \operatorname{Re} \Lambda y
$$

for every $x \in A$ and for every $y \in B$.

(b) If $A$ is compact, $B$ is closed, and $X$ is locally convex, then there exist $\Lambda \in X^{*}$, $\gamma_{1} \in R, \gamma_{2} \in R$, such that

$$
\operatorname{Re} \Lambda x<\gamma_{1}<\gamma_{2}<\operatorname{Re} \Lambda y
$$

for every $x \in A$ and for every $y \in B$.

Note that this is stated without specifying the scalar field; if it is $R$, then $\operatorname{Re} \Lambda=$ $\Lambda$, of course.

PROOF It is enough to prove this for real scalars. For if the scalar field is $\varnothing$ and the real case has been proved, then there is a continuous real-linear $\Lambda_{1}$ on $X$ that gives the required separation; if $\Lambda$ is the unique complex-linear functional on $X$ whose real part is $\Lambda_{1}$, then $\Lambda \in X^{*}$. (See Section 3.1.) Assume real scalars.

(a) Fix $a_{0} \in A, b_{0} \in B$. Put $x_{0}=b_{0}-a_{0}$; put $C=A-B+x_{0}$. Then $C$ is a convex neighborhood of 0 in $X$. Let $p$ be the Minkowski functional of $C$. By Theorem 1.35, $p$ satisfies hypothesis $(b)$ of Theorem 3.2. Since $A \cap B=\varnothing, x_{0} \notin C$, and so $p\left(x_{0}\right) \geq 1$.

Define $f\left(t x_{0}\right)=t$ on the subspace $M$ of $X$ generated by $x_{0}$. If $t \geq 0$ then

$$
f\left(t x_{0}\right)=t \leq t p\left(x_{0}\right)=p\left(t x_{0}\right)
$$

if $t<0$ then $f\left(t x_{0}\right)<0 \leq p\left(t x_{0}\right)$. Thus $f \leq p$ on $M$. By Theorem 3.2, $f$ extends to a linear functional $\Lambda$ on $X$ that also satisfies $\Lambda \leq p$. In particular, $\Lambda \leq 1$ on $C$, hence $\Lambda \geq-1$ on $-C$, so that $|\Lambda| \leq 1$ on the neighborhood $C \cap(-C)$ of 0 .
By Theorem $1.18, \Lambda \in X^{*}$.

If now $a \in A$ and $b \in B$, we have

$$
\therefore a-\Lambda b+1=\Lambda\left(a-b+x_{0}\right) \leq p\left(a-b+x_{0}\right)<1
$$

since $\Lambda x_{0}=1, a-b+x_{0} \in C$, and $C$ is open. Thus $\Lambda a<\Lambda b$.

It follows that $\Lambda(A)$ and $\Lambda(B)$ are disjoint convex subsets of $R$, with $\Lambda(A)$ to the left of $\Lambda(B)$. Also, $\Lambda(A)$ is an open set since $A$ is open and since every nonconstant linear functional on $X$ is an open mapping. Let $\gamma$ be the right end point of $\Lambda(A)$ to get the conclusion of part $(a)$.

(b) By Theorem 1.10 there is a convex neighborhood $V$ of 0 in $X$ such that $(A+V) \cap B=\varnothing$. Part (a), with $A+V$ in place of $A$, shows that there exists $\Lambda \in X^{*}$ such that $\Lambda(A+V)$ and $\Lambda(B)$ are disjoint convex subsets of $R$, with $\Lambda(A+V)$ open and to the left of $\Lambda(B)$. Since $\Lambda(A)$ is a compact subset of $\Lambda(A+V)$, we obtain the conciusion of $(b)$.

Corollary If $X$ is a locally convex space then $X^{*}$ separates points on $X$.

PROOF If $x_{1} \in X, x_{2} \in X$, and $x_{1} \neq x_{2}$, apply $(b)$ of Theorem 3.4 with $A=$ $\left\{x_{1}\right\}, B=\left\{x_{2}\right\} . \quad$ II/I

3.5 Theorem Suppose $M$ is a subspace of a locally convex space $X$, and $x_{0} \in X$. If $x_{\odot}$ is not in the closure of $M$, then there exists $\Lambda \in X^{*}$ such that $\Lambda x_{0}=1$ but $\Lambda x=0$ for every $x \in M$.

PRoOF By $(b)$ of Theorem 3.4, with $A=\left\{x_{0}\right\}$ and $B=\bar{M}$, there exists $\Lambda \in X^{*}$ such that $\Lambda x_{0}$ and $\Lambda(M)$ are disjoint. Thus $\Lambda(M)$ is a proper subspace of the scalar field. This forces $\Lambda(M)=\{0\}$ and $\Lambda x_{0} \neq 0$. The desired functional is obtained by dividing $\Lambda$ by $\Lambda x_{0}$.

Remark This theorem is the basis of a standard method of treating certain approximation problems: In order to prove that an $x_{0} \in X$ lies in the closure of some subspace $\bar{M}$ of $X$ it suffices (if $X$ is locally convex) to show that $\Lambda x_{0}=0$ for every continuous linear functional $\Lambda$ on $X$ that vanishes on $M$. 3.6 Theorem If $f$ is a continuous linear functional on a subspace $M$ of a locally
convex space $X$, then there exists $\Lambda \in X^{*}$ such that $\Lambda=f$ on $M$.

Remark For normed spaces this is an immediate corollary of Theorem 3.3. The general case could also be obtained from 3.3, by relating the continuity
of linear functionals to seminorms (scc Exercise 8, Chapter 1). The proof given below shows that Theorem 3.6 depends only on the separation property of Theorem 3.5.

PROOF Assume, without loss of generality, that $f$ is not identically 0 on $M$. Put

$$
M_{0}=\{x \in M: f(x)=0\}
$$

and pick $x_{0} \in M$ such that $f\left(x_{0}\right)=1$. Since $f$ is continuous, $x_{0}$ is not in the $M$-closure of $M_{0}$, and since $M$ inherits its topology from $X$, it follows that $x_{0}$ is not in the $X$-closure of $M_{0}$.

Theorem 3.5 therefore assures the existence of a $\Lambda \in X^{*}$ such that $\Lambda x_{0}=1$ and $\Lambda=0$ on $M_{0}$.

If $x \in M$, then $x-f(x) x_{0} \in M_{0}$, since $f\left(x_{0}\right)=1$. Hence

$$
\Lambda x-f(x)=\Lambda x-f(x) \Lambda x_{0}=\Lambda\left(x-f(x) x_{0}\right)=0
$$

Thus $\Lambda=f$ on $M$.

We conclude this discussion with another useful corollary of the separation theorem.

3.7 Theorem Suppose $B$ is a convex, balanced, closed set in a locally convex space $X, x_{0} \in X$, but $x_{0} \notin B$. Then there exists $\Lambda \in X^{*}$ such that $|\Lambda x| \leq 1$ for all $x \in B$, but $\Lambda x_{0}>1$.

PROOF Apply $(b)$ of Theorem 3.4, with $A=\left\{x_{0}\right\}$, note that $\Lambda(B)$ is convex and balanced, and multiply the corresponding $\Lambda$ by an appropriate scalar.

\section{Weak Topologies}
3.8 Topological preliminaries The purpose of this section is to explain and illustrate some of the phenomena that occur when a set is topologized in several ways.

Let $\tau_{1}$ and $\tau_{2}$ be two topologies on a set $X$, and assume $\tau_{1} \subset \tau_{2}$; that is, every $\tau_{1}$-open set is also $\tau_{2}$-open. Then we say that $\tau_{1}$ is weaker than $\tau_{2}$, or that $\tau_{2}$ is stronger than $\tau_{1}$. [Note that (in accordance with the meaning of the inclusion symbol $\subset$ ) the terms "weaker" and "stronger" do not exclude equality.] In this situation, the identity mapping on $X$ is continuous from $\left(X, \tau_{2}\right)$ to $\left(X, \tau_{1}\right)$ and is an open mapping from $\left(X, \tau_{1}\right)$ to $\left(X, \tau_{2}\right)$.

As a first illustration, let us prove that the topology of a compact Hausdorff space has a certain rigidity, in the sense that it cannot be weakened without losing the Hausdorff separation axiom and cannot be strengthened without losing compactness:
(a) If $\tau_{1} \subset \sim \tau_{2}$ are topologies on a set $X$, if $\tau_{1}$ is a Hausdorff topology, and if $\tau_{2}$ is

To see this, let $F \subset \cdot X$ be $\tau_{2}$-closed. Since $X$ is $\tau_{2}$-compact, so is $F$. Since $\tau_{1} \subset \tau_{2}$, it follows that $F$ is $\tau_{1}$-compact. (Every $\tau_{1}$-open cover of $F$ is also a $\tau_{2}$-open cover.) Since $\tau_{1}$ is a Hausdorff topology, it follows that $F$ is $\tau_{1}$-closed.

As another illustration, consider the quotient topology $\tau_{N}$ of $X / N$, as defined in Section 1.40, and the quotient map $\pi: X \rightarrow X / N$. By its very definition, $\tau_{N}$ is the strongest topology on $X / N$ that makes $\pi$ continuous, and it is the weakest one that makes $\pi$ an open mapping. Explicitly, if $\tau^{\prime}$ and $\tau^{\prime \prime}$ are topologies on $X / N$, and if $\pi$ is continuous relative to $\tau^{\prime}$ and open relative to $\tau^{\prime \prime}$, then $\tau^{\prime} \subset \tau_{N} \subset \tau^{\prime \prime}$.

Suppose next that $X$ is a set and $\mathscr{F}$ is a nonempty family of mappings $f: X \rightarrow Y_{f}$, where each $Y_{f}$ is a topological space. (In many important cases, $Y_{f}$ is the same for all $f \in \mathscr{F}$.) Let $\tau$ be the collection of all unions of finite intersections of sets $f^{-1}(V)$, with $f \in \mathscr{F}$ and $V$ open in $Y_{\boldsymbol{f}}$. Then $\tau$ is a topology on $X$, and it is in fact the weakest topology on $X$ that makes every $f \in \mathscr{F}$ continuous: If $\tau^{\prime}$ is any other topology with that property, then $\tau \subset \tau^{\prime}$. This $\tau$ is called the weak topology on $X$ induced by $\mathscr{F}$, or, more succinctly, the $\mathscr{F}$-topology of $X$.

The best-known example of this situation is undoubtedly the usual way in which one topologizes the cartesian product $X$ of a collection of topological spaces $X_{\alpha}$. If $\pi_{\alpha}(x)$ denotes the $\alpha$ th coordinate of a point $x \in X$, then $\pi_{\alpha}$ maps $X$ onto $X_{\alpha}$, and the product topology $\tau$ of $X$ is, by definition, its $\left\{\pi_{\alpha}\right\}$-topology, the weakest one that makes every $\pi_{\alpha}$ continuous. Assume now that every $X_{\alpha}$ is a compact Hausdorff space. Then $\tau$ is a compact topology on $\bar{X}$ (by Tychonoff's theorem), and proposition (a) implies that $\tau$ cannot be strengthened without spoiling Tychonoff's theorem.

In the last sentence a special case of the following proposition was tacitly used:

(b) If $\mathscr{F}$ is a family of mappings $f: X \rightarrow Y_{f}$, where $X$ is a set and each $Y_{f}$ is a Hausdorff space, and if $\mathscr{F}$ separates points on $X$, then the $\mathscr{F}$-topology of $X$ is a Hausdorff topology.

For if $p \neq q$ are points of $X$, then $f(p) \neq f(q)$ for some $f \in \mathscr{F}$; the points $f(p)$ and $f(q)$ have disjoint neighborhoods in $Y_{f}$ whose inverse images under $f$ are open (by definition) and disjoint.

Here is an application of these ideas to a metrization theorem.

(c) If $\bar{X}$ is a compact topological space and if some sequence $\left\{f_{n}\right\}$ of continuous realvalued functions separates points on $X$, then $X$ is metrizable.

Let $\tau$ be the given topology of $X$. Suppose, without loss of generality, that $\left|f_{n}\right| \leq 1$ for all $n$, and let $\tau_{d}$ be the topology induced on $X$ by the metric

$$
d(p, q)=\sum_{n=1}^{\infty} 2^{-n}\left|f_{n}(p)-f_{n}(q)\right|
$$

This is indeed a metric, since $\left\{f_{n}\right\}$ separates points. Since each $f_{n}$ is $\tau$-continuous and the series converges uniformly on $X \times X, d$ is a $\tau$-continuous function on $X \times X$. The balls

$$
B_{r}(p)=\{q \in X: d(p, q)<r\}
$$

are therefore $\tau$-open. Thus $\tau_{d} \subset \tau$. Since $\tau_{d}$ is induced by a metric, $\tau_{d}$ is a Hausdorff topology, and now (a) implies that $\tau=\tau_{d}$.

The following lemma has applications in the study of vector topologies. In fact, the case $n=1$ was needed (and proved) at the end of Theorem 3.6.

3.9 Lemma Suppose $\Lambda_{1}, \ldots, \Lambda_{n}$ and $\Lambda$ are linear functionals on a vector space $X$. Let

$$
N=\left\{x: \Lambda_{1} x=\cdots=\Lambda_{n} x=0\right\} .
$$

The following three properties are then equivalent:

(a) There are scalars $\alpha_{1}, \ldots, a_{n}$ such that

$$
\Lambda=\alpha_{1} \Lambda_{i}+\cdots+\alpha_{n} \Lambda_{n} .
$$

(b) There exists $\gamma<\infty$ such that

$$
|\Lambda x| \leq \gamma \quad \max _{1 \leq i \leq n}\left|\Lambda_{i} x\right| \quad(x \in X)
$$

(c) $\Lambda x=0$ for every $x \in N$.

PROOF It is clear that $(a)$ implies $(b)$ and that $(b)$ implies $(c)$. Assume $(c)$ holds. Let $\Phi$ be the scalar field. Define $\pi: X \rightarrow \Phi^{n}$ by

$$
\pi(x)=\left(\Lambda_{1} x, \ldots, \Lambda_{n} x\right)
$$

If $\pi(x)=\pi\left(x^{\prime}\right)$ then (c) implies $\Lambda x=\Lambda x^{\prime}$. Hence $\Lambda=F \circ \pi$, for some function $F$ on $\Phi^{n}$. This $F$ is a linear functional on $\Phi^{n}$. Hence there exist $\alpha_{i} \in \Phi$ such that

$$
F\left(u_{1}, \ldots, u_{n}\right)=\alpha_{1} u_{1}+\cdots+\alpha_{n} u_{n} .
$$

Thus

$$
\Lambda x=F(\pi(x))=F\left(\Lambda_{i} x, \ldots, \Lambda_{n} x\right)=\sum_{i=1}^{n} \alpha_{i} \Lambda_{i} x
$$

which is $(a)$.

3.10 Theorem Suppose $X$ is a vector space and $X^{\prime}$ is a separating vector space of linear functionals on $X$. Then the $X^{\prime}$-topology $\tau^{\prime}$ makes $X$ into a locally convex space whose dual space is $X^{\prime}$.

The assumptions on $X^{\prime}$ are, more explicitly, that $X^{\prime}$ is closed under addition and scalar multiplication and that $\Lambda x_{1} \neq \Lambda x_{2}$ for some $\Lambda \in X^{\prime}$ whenever $x_{1}$ and $x_{2}$ are distinct points of $X$.

PROOF Since $R$ and $\bar{C}$ are Hausdorff spaces, ( $b$ ) of Section 3.8 shows that $\tau^{\prime}$ is a Hausdorff topology. The linearity of the members of $X^{\prime}$ shows that $\tau^{\prime}$ is translation-invariant. If $\Lambda_{1}, \ldots, \Lambda_{n} \in X^{\prime}$, if $r_{i}>0$, and if

$$
V=\left\{x:\left|\Lambda_{i} x\right|<r_{i} \quad \text { for } \quad 1 \leq i \leq n\right\}
$$

then $V$ is convex, balanced, and $V \in \tau^{\prime}$. In fact, the collection of all $V$ of the form (1) is a local base for $\tau^{\prime}$. Thus $\tau^{\prime}$ is a locally convex topology on $X$.

If (1) holds, then $\frac{1}{2} V+\frac{1}{2} V=V$. This proves that addition is continuous. Suppose $x \in X$ and $\alpha$ is a scalar. Then $x \in s V$ for some $s>0$. If $|\beta-\alpha|<r$ and $y-x \in r V$ then

$$
\beta y-\alpha x=(\beta-\alpha) y+\alpha(y-x)
$$

lies in $V$, provided that $r$ is so small that

$$
r(s+r)+|\alpha| r<1 .
$$

Hence scalar multiplication is continuous.

We have now proved that $\tau^{\prime}$ is a locally convex vector topology. Every $\Lambda \in X^{\prime}$ is $\tau^{\prime}$-continuous. Conversely, suppose $\Lambda$ is a $\tau^{\prime}$-continuous linear functional on $X$. Then $|\Lambda x|<1$ for all $x$ in some set $V$ of the form (1). Condition (b) of Lemma 3.9 therefore holds; hence so does $(a): \Lambda=\sum \alpha_{i} \Lambda_{i}$. Since $\Lambda_{i} \in X^{\prime}$ and $X^{\prime}$ is a vector space, $\Lambda \in X^{\prime}$. This completes the proof.

Note: The first part of this proof could have been based on Theorem 1.37 and the separating family of seminorms $p_{\Lambda}\left(\Lambda \in X^{\prime}\right)$ given by $p_{\Lambda}(x)=|\Lambda x|$.

\subsection{The weak topology of a topological vector space Suppose $X$ is a}
 topological vector space (with topology $\tau$ ) whose dual $X^{*}$ separates points on $X$. (We know that this happens in every locally convex $X$. It also happens in some others; see Exercise 5.) The $X^{*}$-topology of $X$ is called the weak topology of $X$.We shall let $X_{w}$ denote $X$ topologized by this weak topology $\tau_{w}$. Theorem 3.10 implies that $X_{w}$ is a locally convex space whose dual is also $X^{*}$.

Since every $\Lambda \in X^{*}$ is $\tau$-continuous and since $\tau_{w}$ is the weakest topology on $X$ with that property, we have $\tau_{w} \subset \tau$. In this contexi, the given topology $\tau$ will often be called the original topology of $X$.

Self-explanatory expressions such as original neighborhood, weak neighborhood, original closure, weak closure, originally bounded, weakly bounded, etc., will be used to make it clear with respect to which topology these terms are to be understood. ${ }^{1}$

For instance, let $\left\{x_{n}\right\}$ be a sequence in $X$. To say that $x_{n} \rightarrow 0$ originally means that every original neighborhood of 0 contains all $x_{n}$ with sufficiently large $n$. To say that $x_{n} \rightarrow 0$ weakly means that every weak neighborhood of 0 contains all $x_{n}$ with sufficiently large $n$. Since every weak neighborhood of 0 contains a neighborhood of the form

$$
V=\left\{x:\left|\Lambda_{i} x\right|<r_{i} \text { for } 1 \leq i \leq n\right\}
$$

where $\Lambda_{i} \in X^{*}$ and $r_{i}>0$, it is easy to see that $x_{n} \rightarrow 0$ weakly if and only if $\Lambda x_{n} \rightarrow 0$ for every $\Lambda \in X^{*}$.

Hence every originally convergent sequence converges weakly. (The converse is usually false; see Exercises 5 and 6.)

Similarly, a set $E \subset X$ is weakly bounded (that is, $E$ is a bounded subset of $X_{w}$ ) if and only if every $V$ as in (1) contains $t E$ for some $t=t(V)>0$. This happens if and only if there corresponds to each $\Lambda \in X^{*}$ a number $\gamma(\Lambda)<\infty$ such that $|\Lambda x| \leq \gamma(\Lambda)$ for every $x \in E$. In other words, a set $E \subset X$ is weakly bounded if and only if every $\Lambda \in X^{*}$ is a bounded function on $E$.

Let $V$ again be as in (1), and put

$$
N=\left\{x: \Lambda_{1} x=\cdots=\Lambda_{n} x=0\right\}
$$

Since $x \rightarrow\left(\Lambda_{1} x, \ldots, \Lambda_{n} x\right)$ maps $X$ into $\mathscr{C}^{n}$, with nullspace $N$, we see that $\operatorname{dim} X \leq$ $n+\operatorname{dim} N$. Since $N \subset V$, this leads to the following conclusion.

If $X$ is infinite-dimensional then every weak neighborhood of 0 contains an infinitedimensional subspace; hence $X_{w}$ is not locally bounded.

This implies in many cases that the weak topology is strictly weaker than the original one. Of course, the two may coincide: Theorem 3.10 implies that $\left(X_{w}\right)_{w}=X_{w}$. We now come to a more interesting result.

3.12 Theorem Suppose $E$ is a convex subset of a locally convex space $X$. Then the weak closure $\bar{E}_{w}$ of $E$ is equal to its original closure $\bar{E}$.
\footnotetext{
${ }^{1}$ When $X$ is a Fréchet space (hence, in particular, when $X$ is a Banach space) the original topology of $X$ is usually called its strong topology. In that context, the terms "strong" and "strongly" will be used in place of "original" and "originally." For locally convex spaces in general, the term "strong topology" has been given a specific technical meaning. See [15], pp. 256-268; also [14], p. 104. It seems therefore advisable to use "original" in the present general discussion.
}

PROOF $\quad \bar{E}_{\dot{w}}$ is weakly closed, hence originally closed, so that $\bar{E} \subset \bar{E}_{w}$. To obtain the opposite inclusion, choose $x_{0} \in X, x_{0} \notin \bar{E}$. Part (b) of the separation theorem 3.4 shows that there exist $\Lambda \in X^{*}$ and $\gamma \in R$ such that, for every $x \in \bar{E}$,

$$
\operatorname{Re} \Lambda x_{0}<\gamma<\operatorname{Re} \Lambda x
$$

The set $\{x: \operatorname{Re} \Lambda x<\gamma\}$ is therefore a weak neighborhood of $x_{0}$ that does not intersect $E$. Thus $x_{0}$ is not in $\bar{E}_{w}$. This proves $\bar{E}_{w} \subset \bar{E}$.

Corollaries 'Let $X$ be a locally convex space.

(a) A subspace of $X$ is originally closed if and only if it is weakly closed.

(b) A convex subset of $X$ is originally dense if and only if it weakly dense. 3.12:

The proofs are obvious. Here is another noteworthy consequence of Theorem

3.13 Theorem Suppose $X$ is a metrizable locally convex space. If $\left\{x_{n}\right\}$ is a sequence in $X$ that converges weakly to some $x \in X$, then there is a sequence $\left\{y_{i}\right\}$ in $X$ such that

(a) each $y_{i}$ is a convex combination of finitely many $x_{n}$, and

(b) $y_{i} \rightarrow x$ originally.

Conclusion ( $a$ ) says, more explicitly, that there exist numbers $\alpha_{i n} \geq 0$, such that

$$
\sum_{n=1}^{\infty} \alpha_{i n}=1, \quad y_{i}=\sum_{n=1}^{\infty} \alpha_{i n} x_{n}
$$

and, for each $i$, only finitely many $\alpha_{i n}$ are $\neq 0$.

PROOF Let $H$ be the convex hull of the set of all $x_{n}$; let $K$ be the weak closure of H. Then $x \in K$. By Theorem 3.12, $x$ is also in the original closure of $H$. Since the original topology of $X$ is assumed to be metrizable, it follows that there is a sequence $\left\{y_{i}\right\}$ in $H$ that converges originally to $x$.

To get a feeling for what is involved here, consider the following example.

Let $K$ be a compact Hausdorff space (the unit interval on the real line is a sufficiently interesting one), and assume that $f$ and $f_{n}(n=1,2,3, \ldots)$ are continuous complex functions on $K$ such that $f_{n}(x) \rightarrow f(x)$ for every $x \in K$, as $n \rightarrow \infty$, and such that $\left|f_{n}(x)\right| \leq 1$ for all $n$ and all $x \in K$. Theorem 3.13 asserts that there are convex combinations of the $f_{n}$ that converge uniformly to $f$.

To see this, let $C(K)$ be the Banach space of all complex continuous functions on $K$, normed by the supremum. Then strong convergence is the same as uniform convergence on $K$. If $\mu$ is any complex Borel measure on $K$, Lebesgue's dominated
convergence theorem implies that $\int f_{n} d \mu \rightarrow \int f d \mu$. Hence $f_{n} \rightarrow f$ weakly, by the Riesz representation theorem which identifies the dual of $C(K)$ with the space of all regular complex Borel measures on $K$. Now Theorem 3.13 can be applied.

After this short detour we now return to our main line of development.

3.14 The weak*-topology of a dual space Let $X$ again be a topological vector space whose dual is $X^{*}$. For the definitions that follow, it is irrelevant whether $X^{*}$ separates points on $X$ or not. The important observation to make is that every $x \in X$ induces a linear functional $f_{x}$ on $X^{*}$, defined by

$$
f_{x} \Lambda=\Lambda x
$$

and that $\left\{f_{x}: x \in X\right\}$ separates points on $X^{*}$.

The linearity of each $f_{x}$ is obvious; if $f_{x} \Lambda=f_{x} \Lambda^{\prime}$ for all $x \in X$, then $\Lambda x=\Lambda^{\prime} x$ for all $x$, and so $\Lambda=\Lambda^{\prime}$ by the very definition of what it means for two functions to be equal.

We are now in the situation described by Theorem 3.10, with $X^{*}$ in place of $X$ and with $X$ in place of $X^{\prime}$.

The $X$-topology of $X^{*}$ is called the weak-topology of $X^{*}$ (pronunciation: weak star topology).

Theorem 3.10 implies that this is a locally convex vector topology on $X^{*}$ and that every linear functional on $X^{*}$ that is weak $^{*}$-continuous has the form $\Lambda \rightarrow \Lambda x$ for some $x \in X$.

The weak*-topologies have a very important compactness property to which we now turn our attention. Various pathological features of the weak- and weak*topologies are described in Exercises 9 and 10.

\section{Compact Convex Sets}
3.15 The Banach-Alaoglu theorem If $V$ is a neighborhood of 0 in a topological vector space $X$ and if

$$
K=\left\{\Lambda \in X^{*}:|\Lambda x| \leq 1 \quad \text { for every } \quad x \in V\right\}
$$

then $K$ is weak*-compact.

Note: $K$ is sometimes called the polar of $V$. It is clear that $K$ is convex and balanced, because this is true of the unit disc in $\varnothing$ (and of the interval $[-1,1]$ in $R$ ). There is some redundancy in the definition of $K$, since every linear functional on $X$ that is bounded on $V$ is continuous, hence is in $X^{*}$.

PROOF Since neighborhoods of 0 are absorbing, there corresponds to each $x \in X$ a number $\gamma(x)<\infty$ such that $x \in \gamma(x) V$. Hence

$$
|\Lambda x| \leq \gamma(x) \quad(x \in X, \Lambda \in K)
$$

Let $D_{x}$ be the set of all scalars $\alpha$ such that $|\alpha| \leq \gamma(x)$. Let $\tau$ be the product topology on $P$, the cartesian product of all $D_{x}$, one for each $x \in X$. Since each $D_{x}$ is compact, so is $P$, by Tychonoff's theorem. The elements of $P$ are the functions $f$ on $X$ (linear or not) that satisfy

$$
|f(x)| \leq \gamma(x) \quad(x \in X) .
$$

Thus $K \subset X^{*} \cap P$. It follows that $K$ inherits two topologies: one from $X^{*}$ (its weak*-topology, to which the conclusion of the theorem refers) and the other, $\tau$, from $P$. We will see that

(a) these two topologies coincide on $K$, and

(b) $K$ is a closed subset of $P$.

Since $P$ is compact, $(b)$ implies that $K$ is $\tau$-compact, and then $(a)$ implies that $K$ is weak-compact.

Fix some $\Lambda_{0} \in K$. Choose $x_{i} \in X$, for $1 \leq i \leq n$; choose $\delta>0$. Put

and

$$
W_{1}=\left\{\Lambda \in X^{*}:\left|\Lambda x_{i}-\Lambda_{0} x_{i}\right|<\delta \text { for } 1 \leq i \leq n\right\}
$$

$$
W_{2}=\left\{f \in P:\left|f\left(x_{i}\right)-\Lambda_{0} x_{i}\right|<\delta \text { for } 1 \leq i \leq n\right\} .
$$

Let $n, x_{i}$, and $\delta$ range over all admissible values. The resulting sets $W_{1}$ then form a local base for the weak*-topology of $X^{*}$ at $\Lambda_{0}$ and the sets $W_{2}$ form a local base for the product topology $\tau$ of $P$ at $\Lambda_{0}$. Since $K \subset P \cap X^{*}$, we have

This proves $(a)$.

$$
W_{1} \cap K=W_{2} \cap K
$$

Next, suppose $f_{0}$ is in the $\tau$-closure of $K$. Choose $x \in X, y \in X$, scalars $\alpha$ and $\beta$, and $\varepsilon>0$. The set of all $f \in P$ such that $\left|f-f_{0}\right|<\varepsilon$ at $x$, at $y$, and at $\alpha x+\beta y$ is a $\tau$-neighborhood of $f_{0}$. Therefore $K$ contains such an $f$. Since this $f$ is linear, we have

$f_{0}(\alpha x+\beta y)-\alpha f_{0}(x)-\beta f_{0}(y)=\left(f_{0}-f\right)(\alpha x+\beta y)+\alpha\left(f-f_{0}\right)(x)+\beta\left(f-f_{0}\right)(y)$, so that

$$
\left|f_{0}(\alpha x+\beta y)-\alpha f_{0}(x)-\beta f_{0}(y)\right|<(1+|\alpha|+|\beta|) \varepsilon
$$

Since $\varepsilon$ was arbitrary, we see that $f_{0}$ is linear. Finally, if $x \in V$ and $\varepsilon>0$, the same argument shows that there is an $f \in K$ such that $\left|f(x)-f_{0}(x)\right|<\varepsilon$.

Since $|f(x)| \leq 1$, by the definition of $K$, it follows that $\left|f_{0}(x)\right| \leq 1$. We conclude that $f_{0} \in K$. This proves $(b)$ and hence the theorem.

When $X$ is separable (i.e., when there is a countable dense set in $X$ ) then the conclusion of the Banach-Alaoglu theorem can be strengthened by combining it with the following fact:

3.16 Theorem If $X$ is a separable topological vector space, if $K \subset X^{*}$ and if $K$ is weak*-compact, then $K$ is metrizable, in the weak*-topology.

Warning: It does not follow that $X^{*}$ itself is metrizable in its weak*-topology. In fact, this is false whenever $X$ is an infinite-dimensional Banach space. See Exercise 15.

Proof Let $\left\{x_{n}\right\}$ be a countable dense set in $X$. Put $f_{n}(\Lambda)=\Lambda x_{n}$, for $\Lambda \in X^{*}$. Each $f_{n}$ is weak*-continuous, by the definition of the weak-topology. If $f_{n}(\Lambda)=f_{n}\left(\Lambda^{\prime}\right)$ for all $n$, then $\Lambda x_{n}=\Lambda^{\prime} x_{n}$ for all $n$, which implies that $\Lambda=\Lambda^{\prime}$, since both are continuous on $X$ and coincide on a dense set.

Thus $\left\{f_{n}\right\}$ is a countable family of continuous functions that separates points on $X^{*}$. The metrizability of $K$ now follows from (c) of Section 3.8. I//I

3.17 Theorem If $V$ is a neighborhood of 0 in a separable topological vector space $X$, and if $\left\{\Lambda_{n}\right\}$ is a sequence in $X^{*}$ such that

$$
\left|\Lambda_{n} x\right| \leq 1 \quad(x \in V, n=1,2,3, \ldots)
$$

then there is a subsequence $\left\{\Lambda_{n_{i}}\right\}$ and there is a $\Lambda \in X^{*}$ such that

$$
\Lambda x=\lim _{i \rightarrow \infty} \Lambda_{n_{i}} x \quad(x \in X)
$$

In other words, the polar of $V$ is sequentially compact in the weak*-topology.

PROOF Combine Theorems 3.15 and 3.16.

The next application of the Banach-Alaoglu theorem involves the HahnBanach theorem and a category argument.

3.18 Theorem In a locally convex space $X$, every weakly bounded set is originally bounded, and vice versa.

Part $(d)$ of Exercise 5 shows that the local convexity of $X$ cannot be omitted from the hypotheses.

PROOF Since every weak neighborhood of 0 in $X$ is an original neighborhood of 0 , it is obvious from the definition of " bounded" that every originally bounded subset of $X$ is weakly bounded. The converse is the nontrivial part of the theorem. 0 in $X$ Suppose $E \subset X$ is weakly bounded and $U$ is an original neighborhood of

Since $X$ is locally convex, there is a convex, balanced, original neighborhood $V$ of 0 in $X$ such that $\bar{V} \subset U$. Let $K \subset X^{*}$ be the polar of $V$ :

$$
\dot{K}=\left\{\Lambda \in X^{*}:|\Lambda x| \leq 1 \text { for all } x \in V\right\}
$$

We claim that

$$
\bar{V}=\{x \in X:|\Lambda x| \leq 1 \quad \text { for all } \Lambda \in K\}
$$

It is clear that $V$ is a subset of the right side of (2) and hence so is $\bar{V}$, since the right side of (2) is closed: Suppose $x_{0} \in X$ but $x_{0} \notin \bar{V}$. Theorem 3.7 (with $\bar{V}$ in place of $B$ ) then shows that $\Lambda x_{0}>1$ for some $\Lambda \in K$. This proves (2). $\gamma(\Lambda)<\infty$ such that

$$
|\Lambda x| \leq \gamma(\Lambda) \quad(x \in E)
$$

Since $K$ is convex and weak-compact (Theorem 3.15) and since the functions $\Lambda \rightarrow \Lambda x$ are weak*-continuous, we can apply Theorem 2.9 (with $X^{*}$ in place of $X$ and the scalar field in place of $Y$ ) to conclude from (3) that there is a constant $\gamma<\infty$ such that

$$
|\Lambda x| \leq \gamma \quad(x \in E, \Lambda \in K) .
$$

Now (2) and (4) show that $\gamma^{-1} x \in \bar{V} \subset U$ for all $x \in E$. Since $V$ is balanced,

$$
E \subset t \bar{V} \subset t U \quad(t>\gamma)
$$

Thus $E$ is originally bounded.

Corollary If $X$ is a normed space, if $E \subset X$, and if

$$
\sup _{x \in E}|\Lambda x|<\infty \quad\left(\Lambda \in X^{*}\right)
$$

then there exists $\gamma<\infty$ such that

$$
\|x\| \leq \gamma \quad(x \in E) .
$$

PROOF Normed spaces are locally convex; (6) says that $E$ is weakly bounded, and (7) says that $E$ is originally bounded.

The following analogue of part $(b)$ of the separation theorem 3.4 will be used in the proof of the Krein-Milman theorem.

3.19 Theorem Suppose $X$ is a topological vector space on which $X^{*}$ separates points. Suppose $A$ and $B$ are disjoint, nonempty, compact, convex sets in $X$. Then there exists $\Lambda \in X^{*}$ such that

$$
\sup _{x \in A} \operatorname{Re} \Delta x<\inf _{y \in B} \operatorname{Re} \Lambda y .
$$

Note that part of the hypothesis is weaker than in $(b)$ of Theorem 3.4 (since local convexity of $X$ implies that $X^{*}$ separates points on $X$ ); to make up for this, it is now assumed that both $A$ and $B$ are compact.

PROOF Let $X_{w}$ be $X$ with its weak topology. The sets $A$ and $B$ are evidently compact in $X_{w}$. They are also closed in $X_{w}$ (because $X_{w}$ is a Hausdorff space). Since $X_{w}$ is locally convex, (b) of Theorem 3.4 can be applied to $X_{w}$ in place of $X$; it gives us a $\Lambda \in\left(X_{w}\right)^{*}$ that satisfies (1). But we saw in Section 3.11 (as a consequence of Theorem 3.10) that $\left(X_{w}\right)^{*}=X^{*}$.

3.20 Extreme points Let $K$ be a subset of a vector space $X$. A nonempty set $S \subset K$ is called an extreme set of $K$ if no point of $S$ is an internal point of a line interval whose end points are in $K$ but not in $S$. Analytically, the condition can be expressed as follows: If $x \in K, y \in K, 0<t<1$, and

$$
t x+(1-t) y \in S
$$

then $x \in S$ and $y \in S$.

The extreme points of $K$ are the extreme sets that consist of just one point. Recall that the convex hull of a set $E \subset X$ is the smallest convex set in $X$ that contains $E$, and that the closed convex hull of $E$ is the closure of its convex hull.

At present it is not known whether it is true in every topological vector space $X$ that every compact convex set has an extreme point. In a large class of spaces, the supply of extreme points is actually very abundant. This is shown by Theorems 3.21 and 3.22.

3.21 The Krein-Milman theorem Suppose $X$ is a topological vector space on which $X^{*}$ separates points. If $K$ is a compact convex set in $X$, then $K$ is the closed convex hull of the set of its extreme points.

PROOF Let $\mathscr{P}$ be the collection of all compact extreme sets of $K$. Since $K \in \mathscr{P}$, $\mathscr{P} \neq \varnothing$. We shall use the following two properties of $\mathscr{P}:$

(a) The intersection $S$ of any nonempty subcollection of $\mathscr{P}$ is a member of $\mathscr{P}$, unless $S=\varnothing$.

(b) If $S \in \mathscr{P}, \Lambda \in X^{*}, \mu$ is the maximum of $\operatorname{Re} \Lambda$ on $S$, and

then $S_{\wedge} \in \mathscr{P}$.

$$
S_{\Lambda}=\{x \in S: \operatorname{Re} \Lambda x=\mu\}
$$

The proof of $(a)$ is immediate. To prove $(b)$, suppose $t x+(1-t) y=$ $z \in S_{\Lambda}, x \in K, y \in K, 0<t<1$. Since $z \in S$ and $S \in \mathscr{P}$, we have $x \in S$ and $y \in S$. Hence $\operatorname{Re} \Lambda x \leq \mu, \operatorname{Re} \Lambda y \leq \mu$. Since $\operatorname{Re} \Lambda z=\mu$ and $\Lambda$ is linear, we conclude: $\operatorname{Re} \Lambda x=\mu=\operatorname{Re} \Lambda y$. Hence $x \in S_{\Lambda}$ and $y \in S_{\Lambda}$.

Choose some $S \in \mathscr{P}$. Let $\mathscr{P}^{\prime}$ be the collection of all members of $\mathscr{P}$ that are subsets of $S$. Since $S \in \mathscr{P}^{\prime}, \mathscr{P}^{\prime}$ is not empty. Partially order $\mathscr{P}^{\prime}$ by set inclusion, let $\Omega$ be a maximal totally ordered subcollection of $\mathscr{P}^{\prime}$, and let $M$ be the intersection of all members of $\Omega$. Since $\Omega$ is a collection of compact sets with the finite intersection property, $M \neq \varnothing$. By $(a), M \in \mathscr{P}^{\prime}$. The maximality of $\Omega$ implies that no proper subset of $M$ belongs to $\mathscr{P}$. It now follows from $(b)$ that every $\Lambda \in X^{*}$ is constant on $M$. Since $X^{*}$ separates points on $X, M$ has only one point. Therefore $M$ is an extreme point of $K$.

We have now proved that every compact extreme set of $K$ contains an extreme point of $K$. (So far, the convexity of $K$ has not been used.)

If $H$ is the convex hull of the set of extreme points of $K$, it follows, for every $S \in \mathscr{P}$, that $H \cap S$ is not empty.

Since $K$ is compact and convex, we have $\bar{H} \subset K$. Hence $\bar{H}$ is compact. Assume (to get a contradiction) that some $x_{0} \in K$ is not in $\bar{H}$. By Theorem 3.19 there is a $\Lambda \in X^{*}$ such that $\operatorname{Re} \Lambda x<\operatorname{Re} \Lambda x_{0}$ for every $x \in \bar{H}$. If $K_{\Lambda}$ is defined as in $(b)$, then $K_{\Lambda} \in \mathscr{P}$. Since $\bar{H}$ does not intersect $K_{\Lambda}$ we have our contradiction.

Remark The convexity of $K$ was used only to show that $\bar{H}$ is compact. If $X$ were assumed to be locally convex, the compactness of $\bar{H}$ would not be needed, since one could use $(b)$ of Theorem 3.4 in place of Theorem 3.19. The above argument then proves that $K \subset \bar{H}$. The following version of the Krein-Milman theorem is thus obtained:

3.22 Theorem If $X$ is a locally convex space and if $E$ is the set of extreme points: of a compact set $K$ in $X$, then $K$ lies in the closed convex hull of $E$.

Equivalently, $E$ and $K$ have the same closed convex hull.

The preceding remark raises a question: What can one say about the convex hull $H$ of a compact set $K$ ? Even in a Hilbert space, $H$ need not be closed, and there are situations in which $\bar{H}$ is not compact (Exercises 20, 22). In Fréchet spaces, the latter pathology does not occur (Theorem 3.25). The proof of this will depend on the fact that a subset of a complete metric space is compact if and only if it is closed and totally bounded. (Appendix A4.)

Let us recall that a subset $E$ of a metric space $X$ is said to be totally bounded if $E$ is contained in the union of finitely many open balls of radius $\varepsilon$, for every $\varepsilon>0$. The same concept can be defined in any topological vector space $X$, metrizable or not.

3.23 Definition A set $E$ in a topological vector space $X$ is said to be totally bounded if to every neighborhood $V$ of 0 in $X$ corresponds a finite set $F \subset X$ such that $E \subset F+V$.

If $X$ happens to be a metrizable topological vector space, then these two notions of total boundedness coincide, provided that we restrict ourselves to invariant metrics that are compatible with the topology of $X$. (The proof of this is as in Theorem 1.26.)

3.24 Theorem If $X$ is a locally convex space and $H$ is the convex hull of a totally bounded set $E \subset X$, then $H$ is totally bounded.

PROOF Let $U$ be a neighborhood of 0 in $X$. There is a convex neighborhood $V$ of 0 in $X$ such that $V+V \subset U$, and there is a finite set $E_{1} \subset X$ such that $E \subset E_{1}+V$. Let $H_{1}$ be the convex hull of $E_{1}$.

If $e_{1}, \ldots, e_{m}$ are the points of $E_{1}$ and if $S$ is the simplex in $R^{m}$ consisting of all $t=\left(t_{1}, \ldots, t_{m}\right)$ that satisfy $t_{i} \geq 0$ and $\sum t_{i}=1$, then

$$
\left(t_{1}, \ldots, t_{m}\right) \rightarrow \sum t_{i} e_{i}
$$

maps the compact set $S$ continuously onto $H_{1}$. Hence $H_{1}$ is compact.

If $x \in H$, then $x=\alpha_{1} x_{1}+\cdots+\alpha_{n} x_{n}$, where $x_{i} \in E, \alpha_{i} \geq 0, \sum \alpha_{i}=1$. There are points $y_{i} \in E_{1}$ such that $x_{i}-y_{i} \in V$; this follows from the choice of $E_{1}$. Decompose $x$ into the sum

$$
x=x^{\prime}+x^{\prime \prime}
$$

where $x^{\prime}=\sum \alpha_{i} y_{i}$ and $x^{\prime \prime}=\sum \alpha_{i}\left(x_{i}-y_{i}\right)$. The convexity of $V$ implies that $x^{\prime \prime} \in V$. It is clear that $x^{\prime} \in H_{1}$. Hence

$$
H \subset H_{1}+V .
$$

Since $H_{1}$ is compact, there is a finite set $F$ such that $H_{1} \subset F+V$. Thus

$$
H \subset F+V+V \subset F+U
$$

Since $U$ was arbitrary, it follows that $H$ is totally bounded.

3.25 Theorem Suppose $H$ is the convex hull of a compact set $K$ in a topological vector space $X$.

(a) If $X$ is a Fréchet space, then $\bar{H}$ is compact.

(b) If $X=R^{n}$, then $H$ is compact.

Proof (a) By Theorem 3.24, $H$ is totally bounded. Since Fréchet spaces are complete metric spaces, the closure $\bar{H}$ of $H$ is compact.
(b) Let $S$ be the simplex in $R^{n+1}$ consisting of all $t=\left(t_{1}, \ldots, t_{n+1}\right)$ with $t_{i} \geq 0$ and $\sum t_{i}=1$. By the lemma that follows, $x \in H$ if and only if

$$
x=\sum_{i=1}^{n+1} t_{i} x_{i}
$$

for some $t \in S$ and $x_{i} \in K(1 \leq i \leq n+1)$. In other words, $H$ is the image of

$$
S \times K \times \cdots \times K
$$

( $K$ occurs $n+1$ times) under the continuous mapping

$$
\left(t, x_{1}, \ldots, x_{n+1}\right) \rightarrow \sum_{i=1}^{n+1} t_{i} x_{i}
$$

Hence $H$ is compact.

Lemma If $x$ lies in the convex hull of a set $E \subset R^{n}$, then $x$ lies in the convex hull of some subset of $E$ that contains at most $n+1$ points.

PROOF It is enough to show that if $r>n$ and $x=\sum t_{i} x_{i}$ is a convex combination - of some $r+1$ vectors $x_{i} \in E$, then $x$ is actually a convex combination of some $r$ of these vectors.

Assume, without loss of generality, that $t_{i}>0$ for $1 \leq i \leq r+1$. The $r$ vectors $x_{i}-x_{r+1}(1 \leq i \leq r)$ are linearly dependent, since $r>n$. It follows that there are real numbers $a_{i}$, not all 0 , such that

$$
\sum_{i=1}^{r+1} a_{i} x_{i}=0 \quad \text { and } \quad \sum_{i=1}^{r+1} a_{i}=0
$$

Choose $m$ so that $\left|a_{i} / t_{i}\right| \leq\left|a_{m} / t_{m}\right|$, for $1 \leq i \leq r+1$, and define

$$
c_{i}=t_{i}-\frac{a_{i} t_{m}}{a_{m}} \quad(1 \leq i \leq r+1)
$$

Then $c_{i} \geq 0, \sum c_{i}=\sum t_{i}=1, x=\sum c_{i} x_{i}$, and $c_{m}=0$.

\section{Vector-valued Integration}
Sometimes it is desirable to be able to integrate functions $f$ that are defined on some measure space $Q$ (with a real or complex measure $\mu$ ) and whose values lie in some topological vector space $X$. The first problem is to associate with these data a vector in $X$ that deserves to be called

$$
\int_{Q} f d \mu
$$

i.e., which has at least some of the properties that integrals usually have. For instance, the equation

$$
\Lambda\left(\int_{Q} f d \mu\right)=\int_{Q}(\Lambda f) d \mu
$$

ought to hold for every $\Lambda \in X^{*}$, because it does hold for sums, and because integrals are (or ought to be) limits of sums in some sense or other. In fact, our definition will be based on this single requirement.

Many other approaches to vector-valued integration have been studied in great detail; in some of these, the integrals are defined more directly as limits of sums (see Exercise 23).

3.26 Definition Suppose $\mu$ is a measure on a measure space $Q, X$ is a topological vector space on which $X^{*}$ separates points, and $f$ is a function from $Q$ into $X$ such that the scalar functions $\Lambda f$ are integrable with respect to $\mu$, for every $\Lambda \in X^{*}$; note that $\Lambda f$ is defined by

$$
(\Lambda f)(q)=\Lambda(f(q)) \quad(q \in Q)
$$

If there exists a vector $y \in \widetilde{X}$ such that

$$
\Lambda y=\int_{\underline{Q}}(\Lambda f) d \mu
$$

for every $\Lambda \in X^{*}$, then we define

$$
\int_{Q} \int d \mu=y
$$

Remarks It is clear that there is at most one such $y$, because $X^{*}$ separates points on $X$. Thus there is no uniqueness problem.

Existence will be proved only in the rather special case (sufficient for many applications) in which $Q$ is compact and $f$ is continuous. In that case, $f(Q)$ is compact, and the only other requirement that will be imposed is that the closed convex hull of $f(Q)$ should be compact. By Theorem 3.25, this additional requirement is automatically satisfied when $X$ is a Fréchet space.

Recall that a Borel measure on a compact (or locally compact) Hausdorff space $Q$ is a measure defined on the $\sigma$-algebra of all Borel sets in $Q$; this is the smallest $\sigma$-algebra that contains all open subsets of $Q$. A probability measure is a positive measure of total mass 1 .

\subsection{Theorem Suppose}
(a) $\bar{X}$ is a topological vector space on which $X^{*}$ separates poinis, and

(b) $\mu$ is a Borel probability measure on a compact Hausdorff space $Q$.

If f: $Q \rightarrow X$ is continuous, and if the convex hull $H$ of $f(Q)$ has compact closure $\bar{H}$ in $X$, then the integral

$$
\ddot{y}=\int_{Q} f d \mu
$$

exists, in the sense of Definition 3.26.

Moreover, $y \in \bar{H}$.

Remark If $v$ is any positive Borel measure on $Q$, then some scalar multiple of $v$ is a probability measure. The theorem therefore holds (except for its last sentence) with $\nu$ in place of $\mu$. It can then be extended to real-valued Borel $X$ measures (by the Jordan decomposition theorem) and (if the scalar field of
$\mathbb{C}$ ) complex ones.

Exercise 24 gives another generalization. PROOF Regard $X$ as a real vector space. We have to prove that there exists
$y \in \bar{H}$ such that

for every $\Lambda \in X^{*}$.

$$
\Lambda y=\int_{Q}(\Lambda f) d \mu
$$

Let $L=\left\{\Lambda_{1}, \ldots, \Lambda_{n}\right\}$ be a finite subset of $X^{*}$. Let $E_{L}$ be the set of all $y \in \bar{H}$ that satisfy (2) for cvery $\Lambda \in L$. Each $E_{L}$ is closed (by the continuity of $\Lambda$ ) and is therefore compact, since $\bar{H}$ is compact. If no $E_{L}$ is empty, the collection of all $E_{L}$ has the finite intersection property. The intersection of all $E_{L}$ is therefore not empty, and any $y$ in it satisfies (2) for every $\Lambda \in X^{*}$. It is therefore enough to prove $E_{L} \neq \varnothing$.

Regard $L=\left(\Lambda_{1}, \ldots, \Lambda_{n}\right)$ as a mapping from $X$ into $R^{n}$, and put $K=$ $L(f(Q))$. Define

$$
m_{i}=\int_{Q}\left(\Lambda_{i} f\right) d \mu \quad(1 \leq i \leq n)
$$

We claim that the point $m=\left(m_{1}, \ldots, m_{n}\right)$ lies in the convex hull of $K$.

If $t=\left(t_{1}, \ldots, t_{n}\right) \in R^{n}$ is not in this hull, then [by Theorem 3.25 and

(b) of Theorem 3.4 and the known form of the linear functionals on $R^{n}$ ] there are real numbers $c_{1}, \ldots, c_{n}$ such that

if $u=\left(u_{1}, \ldots, u_{n}\right) \in K$. Hence

$$
\sum_{i=1}^{n} c_{i} u_{i}<\sum_{i=1}^{n} c_{i} t_{i}
$$

$$
\sum_{i=1}^{n} c_{i} \Lambda_{i} f(q)<\sum_{i=1}^{n} c_{i} t_{i} \quad(q \in Q)
$$

Since $\mu$ is a probability measure, integration of the left side of (5) gives $\sum c_{i} m_{i}<\sum c_{i} t_{i}$. Thus $t \neq m$.

This shows that $m$ lies in the convex hull of $K$. Since $K=L(f(Q))$ and $L$ is linear, it follows that $m=L y$ for some $y$ in the convex hull $H$ of $f(Q)$. For this $y$ we have

$$
\Lambda_{i} y=m_{i}=\int_{Q}\left(\Lambda_{i} f\right) d \mu \quad(1 \leq i \leq n)
$$

Hence $y \in E_{L}$. This completes the proof.

\subsection{Theorem Suppose}
(a) $X$ is a topological vector space on which $X^{*}$ separates points,

(b) $Q$ is a compact subset of $X$, and

(c) the closed convex hull $\bar{H}$ of $Q$ is compact.

Then $y \in \bar{H}$ if and only if there is a regular Borel probability measure $\mu$ on $Q$ such that

$$
y=\int_{Q} x d \mu(x)
$$

Remarks The integral is to be understood as in Definition 3.26, with $f(x)=x$. Recall that a positive Borel measure on $Q$ is said to be regular if

$$
\mu(E)=\sup \{\mu(K): K \subset E\}=\inf \{\mu(G): E \subset G\}
$$

for every Borel set $E \subset Q$, where $K$ ranges over the compact subsets of $E$ and $G$ ranges over the open supersets of $E$.

The integral (1) represents every $y \in \bar{H}$ as a "weighted average" of $Q$, or as the "center of mass" of a certain unit mass distributed over $Q$.

We stress once more that $(c)$ follows from $(b)$ if $X$ is a Fréchet space.

PROOF Regard $X$ again as a real vector space. Let $C(Q)$ be the Banach space of all real continuous functions on $Q$, with the supremum norm. The Riesz representation theorem identifies the dual space $C(Q)^{*}$ with the space of all real Borel measures on $Q$ that are differences of regular positive ones. With this identification in mind, we define a mapping

$$
\phi: C(Q)^{*} \rightarrow X
$$

by

$$
\phi(\mu)=\int_{Q} x d \mu(x)
$$

Let $P$ be the set of all regular Borel probability measures on $Q$. The theorem asserts that $\phi(P)=\bar{H}$.

For each $x \in Q$, the unit mass $\delta_{x}$ concentrated at $x$ belongs to $P$. Since $\phi\left(\delta_{x}\right)=x$, we see that $Q \subset \phi(P)$. Since $\phi$ is linear and $P$ is convex, it follows that $H \subset \phi(P)$, where $H$ is the convex hull of $Q$. By Theorem 3.27, $\phi(P) \subset \bar{H}$. Therefore all that remains to be done is to show that $\phi(P)$ is closed in $X$.

This is a consequence of the following two facts:

(i) $P$ is weak*-compact in $C(Q)^{*}$.

(ii) The mapping $\phi$ defined by (4) is continuous if $C(Q)^{*}$ is given its weak*topology and if $X$ is given its weak topology.

Once we have $(i)$ and $(i i)$, it follows that $\phi(P)$ is weakly compact, hence weakly closed, and since weakly closed sets are strongly closed, we have the desired conclusion.

To prove $(i)$, note that

$$
P \subset\left\{\mu:\left|\int_{Q} h d \mu\right| \leq 1 \text { if }\|h\|<1\right\}
$$

and that this larger set is weak*-compact, by the Banach-Alaoglu theorem. It is therefore enough to show that $P$ is weak*-closed.

If $h \in C(Q)$ and $h \geq 0$, put

$$
E_{h}=\left\{\mu: \int_{Q} h d \mu \geq 0\right\}
$$

Since $\mu \rightarrow \int h d \mu$ is continuous, by the definition of the weak*-topology, each $E_{h}$ is weak*-closed. So is the set

$$
E=\left\{\mu: \int_{Q} 1 d \mu=1\right\}
$$

Since $P$ is the intersection of $E$ and the sets $E_{h}, P$ is weak*-closed.

To prove (ii) it is enough to prove that $\phi$ is continuous at the origin, since $\phi$ is linear. Every weak neighborhood of 0 in $X$ contains a set of the form

$$
W=\left\{y \in X:\left|\Lambda_{i} y\right|<r_{i} \quad \text { for } \quad 1 \leq i \leq n\right\}
$$

where $\Lambda_{i} \in X^{*}$ and $r_{i}>0$. The restrictions of the $\Lambda_{i}$ to $Q$ lie in $C(Q)$. Hence

$$
V=\left\{\mu \in C(Q)^{*}:\left|\int_{Q} \Lambda_{i} d \mu\right|<r_{i} \text { for } 1 \leq i \leq n\right\}
$$

is a weak*-neighborhood of 0 in $C(Q)^{*}$. But

$$
\int_{Q} \Lambda_{i} d \mu=\Lambda_{i}\left(\int_{Q} x d \mu(x)\right)=\Lambda_{i} \phi(\mu)
$$

by Definition 3.26. It follows from (8), (9), and (10) that $\phi(V) \subset W$. Hence $\phi$ is continuous.

The following simple inequality sharpens the last assertion in the statement of Theorem 3.27.

3.29 Theorem Suppose $Q$ is a compact Hausdorff space, $X$ is a Banach space, $f: Q \rightarrow X$ is continuous, and $\mu$ is a positive Borel measure on $Q$. Then

$$
\left\|\int_{Q} f d \mu\right\| \leq \int_{Q}\|f\| d \mu
$$

PRoOF. Put $y=\int f d \mu$. By the coroliary to Theorem 3.3, there is a $\Lambda \in X^{*}$ such that $\Lambda y=\|y\|$ and $|\Lambda x| \leq\|x\|$ for all $x \in X$. In particular,

$$
|\Lambda f(s)| \leq\|f(s)\|
$$

for all $s \in Q$. By Theorem 3.27, it follows that

$$
\|y\|=\Lambda y=\int_{Q}(\Lambda f) d \mu \leq \int_{Q}\|f\| d \mu .
$$

\section{Holomorphic Functions}
In the study of Banach algebras, as well as in some other contexts, it is useful to enlarge the concept of holomorphic function from complex-valued ones to vectorvalued ones. (Of course, one can also generalize the domains, by going from $\varnothing$ to $\varnothing^{n}$ and even beyond. But this is another story.) There are at least two very natural definitions of "holomorphic" available in this general setting, a "weak" one and a "strong" one. They turn out to define the same class of functions if the values are assumed to lie in a Fréchet space.

3.30 Definition Let $\Omega$ be an open set in $Q$ and let $X$ be a complex topological vector space.

(a) A function $f: \Omega \rightarrow X$ is said to be weakly holomorphic in $\Omega$ if $\Lambda f$ is holomorphic in the ordinary sense for every $\Lambda \in X^{*}$.

(b) A function $f: \Omega \rightarrow X$ is said to be strongly holomorphic in $\Omega$ if

$$
\lim _{w \rightarrow z} \frac{f(w)-f(z)}{w-z}
$$

exists (in the topology of $X$ ) for every $z \in \Omega$.

Note that the above quotient is the product of the scalar $(w-z)^{-1}$ and the vector $f(w)-f(z)$ in $X$.

The continuity of the functionals $\Lambda$ that occur in $(a)$ makes it obvious that every strongly holomorphic function is weakly holomorphic. The converse is true when $X$ is a Fréchet space, but it is far from obvious. (Recall that weakly convergent sequences may very well fail to converge originally.) The Cauchy theorem will play an important role in this proof, as will Theorem 3.18 .

The index of a point $z \in \mathbb{C}$ with respect to a closed path $\Gamma$ that does not pass through $z$ will be denoted by $\operatorname{Ind}_{\Gamma}(z)$. We recall that

$$
\operatorname{Ind}_{\Gamma}(z)=\frac{1}{2 \pi i} \int_{\Gamma} \frac{d \zeta}{\zeta-z}
$$

3.31 Theorem Let $\Omega$ be open in $\ell$, let $X$ be a complex Fréchet space, and assume
that

$$
f: \Omega \rightarrow X
$$

is weakly holomorphic. The following conclusions hold:

(a) $f$ is strongly continuous in $\Omega$.

(b) The Cauchy theorem and the Cauchy formula hold: If $\Gamma$ is a closed path in $\Omega$ : such that $\operatorname{Ind}_{\Gamma}(w)=0$ for every $w \notin \Omega$, then

and

$$
\int_{\Gamma} f(\zeta) d \zeta=0
$$

$$
f(z)=\frac{1}{2 \pi i} \int_{\Gamma}(\zeta-z)^{-1} f(\zeta) d \zeta
$$

if $z \in \Omega$ and $\operatorname{Ind}_{\Gamma}(z)=1$. If $\Gamma_{1}$ and $\Gamma_{2}$ are closed paths in $\Omega$ such that

for every $w \notin \Omega$, then

$$
\operatorname{Ind}_{\Gamma_{1}}(w)=\operatorname{Ind}_{\Gamma_{2}}(w)
$$

$$
\int_{\Gamma_{1}} f(\zeta) d \zeta=\int_{\Gamma_{2}} f(\zeta) d \zeta
$$

(c) $f$ is strongly holomorphic in $\Omega$.

The integrals in $(b)$ are to be understood in the sense of Theorem 3.27. Either one can regard $d \zeta$ as a complex measure on the range of $\Gamma$ (a compact subset of $\mathscr{C}$ ), or one can parametrize $\Gamma$ and integrate with respect to Lebesgue measure on a compact interval in $R$.

Proof (a) Assume $0 \in \Omega$. We shall prove that $f$ is strongly continuous at 0 .
Define

$$
\Delta_{r}=\{z \in \varnothing:|z| \leq r\}
$$

Then $\Delta_{2 r} \subset \Omega$ for some $r>0$. Let $\Gamma$ be the positively oriented boundary of $\Delta_{2 r}$.

Fix $\Lambda \in X^{*}$. Since $\Lambda f$ is holomorphic,

$$
\frac{(\Lambda f)(z)-(\Lambda f)(0)}{z}=\frac{1}{2 \pi i} \int_{\Gamma} \frac{(\Lambda f)(\zeta)}{(\zeta-z) \zeta} d \zeta
$$

if $0<|z|<2 r$. Let $M(\Lambda)$ be the maximum of $|\Lambda f|$ on $\Delta_{2 r}$. If $0<|z| \leq r$, it follows that

$$
\left|z^{-1} \Lambda[f(z)-f(0)]\right| \leq r^{-1} M(\Lambda) .
$$

The set of all quotients

$$
\left\{\frac{f(z)-f(0)}{z}: 0<|z| \leq r\right\}
$$

is therefore weakly bounded in $X$. By Theorem 3.18, this set is also strongly bounded. Thus if $V$ is any (strong) neighborhood of 0 in $X$, there exists $t<\infty$ such that

$$
f(z)-f(0) \in z t \bar{V} \quad(0<|z| \leq r) .
$$

Consequently, $f(z) \rightarrow f(0)$ strongly, as $z \rightarrow 0$.

This was the crux of the matter. The rest is now almost automatic.

(b). By (a) and Theorem 3.27, the integrals in (1) to (3) exist. These three formulas are correct (by the theory of ordinary holomorphic functions) if $f$ is replaced in them by $\Lambda f$, where $\Lambda$ is any member of $X^{*}$. The formulas are therefore correct as stated, by Definition 3.26.

(c) Assume, as in the proof of (a), that $\Delta_{2 r} \subset \Omega$, and choose $\Gamma$ as in (a). Define

$$
y=\frac{1}{2 \pi i} \int_{\Gamma} \zeta^{-2} f(\zeta) d \zeta
$$

The Cauchy formula (2) shows, aft er a small computation, that

$$
\frac{f(z)-f(0)}{z}=y+z g(z)
$$

if $0<|z|<2 r$, where

$$
g(z)=\frac{1}{2 \pi} \int_{-\pi}^{\pi}\left[2 r e^{i \theta}\left(2 r e^{i \theta}-z\right)\right]^{-1} f\left(2 r e^{i \theta}\right) d \theta
$$

Let $V$ be a convex balanced neighborhood of 0 in $X$. Put $K=$ $\{f(\zeta):|\zeta|=2 r\}$. Then $K$ is compact, so that $K \subset t V$ for some $t<\infty$. If $s=$
$g(z) \in s \bar{V}$ if $|z| \leq r$. The left side of (10) therefore converges strongly to $y$, as $z \rightarrow 0$.

The following extension of Liouville's theorem concerning bounded entire functions does not even depend on Theorem 3.31. It can be used in the study of spectra in Banach algebras. (See Exercise 4, Chapter 10.)

3.32 Theorem Suppose $X$ is a complex topological vector space on which $X^{*}$ separates points. Suppose $f: \dot{C} \rightarrow X$ is weakly holomorphic and $f(\varnothing)$ is a weakly bounded subset of $X$. Then $f$ is constant.

PROOF For every $\Lambda \in X^{*}, \Lambda f$ is a bounded (complex-valued) entire function. If $z \in \mathscr{C}$, it follows from Liouville's theorem that

$$
\Lambda f(z)=\Lambda f(0)
$$

Since $X^{*}$ separates points on $\bar{X}$, this implies $f(z)=f(0)$, for every $z \in \mathscr{C}$.

Part $(d)$ of Exercise 5 describes a weakly bounded set which is not originally bounded, in an $F$-space $X$ on which $X^{*}$ separates points. Compare with Theorem 3.18.


\end{document}