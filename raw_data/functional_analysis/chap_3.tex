3 CONVEXITY $$ \qquad\qquad\qquad\times\times\qquad\qquad\times\times\times\qquad\qquad\qquad\times\times\quad\qquad\qquad\qquad\qquad\times\quad. $$ Tios ng fgo onesneson nemsnusoranana toeos rosi nesgysioeos rmeninsr on eor oeo p oms… sos e isisna eo gasos…ny ooios…isemisni n ae ts i ne eosousoioeunaiois omisesemnn snooo Km… eman siemesmteseno variousproems nasarspsoice cCacper The Hahn-Banach Theorems implis 3.4 s stated as Exercise3 Tnopuioshogosete semluanantueorm uaomn negeosog oso ste st"mseserusemseasenmsns “nqm 2 n- im osmssesoiwnoemm mngy…sm”*3*mmesasamemowA space $X^{\mathrm{ax}}$ 31Dotion dl ze oa oiavetse $\textstyle X$ is the vector whosens arete coimisinenctian $X.$56 GENERAL THEORY Note that addition and scalar multiplication are defined in $X^{\ast}$ by $$ (\Lambda_{1}+\Lambda_{2})x=\Lambda_{1}x+\Lambda_{2}\,x,\qquad(x\Lambda)x=\alpha\cdot\Lambda x. $$ It is clear that these operations do indeed make $X^{\2\mathbb{K}}$ into a vector space t wilbenecessary to use the obvious fact that every complex vector space i ’ is called real also a real vector space, and it will be convenient to use the following (temporary) terminology: An additive functional A on a complex vector space $\textstyle X$ linear (complex-linear) f $\Lambda(a x)=a\Lambda x$ for every $x\in X$ and for every real (complex) scalar . our standing rule that any statement about vector saces in which no scala field is mentioned applies to both cases is unaffectcd by this temporary terminology and is still in force. then uis real-linear and Ifuisthereal part of a complex-linear functional fon $X,$ (1) $$ f(x)=u(x)-i u(i x)\qquad(x\in X) $$ because $z=\mathrm{Re}\,z-i$ Re (iz) for every $z\in C,$ is in $\textstyle X^{\geq}$ if and only if its real part is and if f is imply that a complex-linear functional on $\textstyle X$ is real-linear on a complex vector space $X$ Conversely, if u: $X\to R$ is a complex topological vector spac. Thc above facts Suppose now that defined y (), astaihtforward computation shows tha s comle-linea $\textstyle X$ fe X*. continuous,and that every continuous real-inear u $X\to R$ is the real part of a unique 3.2 Theorem Suppose (の八 $:X\to R$ M is a subspace of a real vector space $X,$ (a) satisfes $$ p(x+y)\leq p(x)+p(y)\qquad a n d\qquad p(t x)=t p(x) $$ (e)f: $f\ x\in X,$ y e $X_{\mathrm{\mathrm{\mathrm{:}}}}$ $t\geq0,$ on $M.$ $M\to R$ is linear $a n d f(x)\leq p(x)$ Then there exists a linear $\operatorname{A}\!:X{ arrow}\,R$ such uhat $$ \Lambda x=f(x)\qquad(x\in M) $$ and $$ -p(-x)\leq\Lambda x\leq p(x)\qquad(x\in X). $$ PRoor If M ${\boldsymbol{\neq}}\ X,$ choose x $\in{\mathcal{X}},\,x_{1}\not\in M.$ and define $$ M_{1}=\{x+d x_{1}:x\in M,\,t\in R\}. $$ It is clear that $M_{1}$ is a vector space. Since f(x) +f0y) = /(×+)≤p(x+)) ≤ P(x-x)+P(x+少) we have $(1)$ J心)-p(x-X)≤NO+X)-./) (x, y e McoNVEXITrY $57$ Let bete esuprbound of en sice ${\boldsymbol{(}}1),$ as ${\mathcal{A}}_{1}$ ranges over $\bar{M}.$ Then (2) $$ f(x)-\alpha\leq p(x-x_{i})\qquad(x\in M) $$ and (3) $$ f(y)+\alpha\leq p(y+x_{1})\qquad(y\in M). $$ $\mathrm{Define}\,f_{1}$ on $\ M_{1}$ by (4) $$ f_{1_{c}}(x+t x_{1})=f(x)+t\alpha\qquad(x\in M,\,t\in R). $$ that $\Lambda\subseteq p.$ give a further extension of $\Lambda,$ is linear on $\varphi_{1}.$ ${\mathcal{P}}$ by declaring by $t^{-1}y$ in (), and multiply and of Then $f_{1}=f$ ${\mathcal{P}}$ 'be te collectofall orded pairs in 2), replace $X,$ the frst part of the proof would $\Lambda$ is linear, and $\boldsymbol{f}$ to where 九 $f^{\prime}$ If on $\therefore\varphi_{1}$ and $1\,f_{1}$ by $t^{-1}x$ $\mathbf{\nabla}y$ $\Omega$ of ${\mathcal{O}}$ is a subspace satisfies Take $\scriptstyle t\;>0.$ replace $\textstyle{\mathcal{X}}$ In combinaion with (4), this provesth $\varphi_{1}^{2}$ p is totally the resulig incqualities by . $f_{1}\leq p$ on $M_{1}.\qquad.$ on $\ M^{\prime}$ Partially order on ${\boldsymbol{M}}^{\prime}$ By Hausdorf's maximaliy theorem Then $\Phi$ $X$ subspace of $X.$ $\bar{\mathcal{M}}$ lt is now easy to check that $\Lambda$ Th scon pan f h pro cn be done by whater ones avort $(M^{\prime},f^{\prime}),$ where that extends $\Lambda x=f^{\prime}(x),$ Let or Hausdort's maximality theorem meiho ogtsiniecuciois o n se -dengs r onsienmm Let $\Phi$ that contains $\bar{\cal M}$ and $f^{\prime}$ is a linear functional on ${\mathcal{M}}^{\prime}$ is therefore a mean that $f^{\prime}\leq p$ If $x\in{\bar{M}}$ then $x\in M^{\prime}$ for some $M^{\prime}\in\Phi;$ define ordered by set inclusion, and the union $\widetilde{\cal M}$ of all members or $\bar{\Phi}$ $(M^{\prime},f^{\prime})\leq(M^{\prime},f^{\prime\prime})$ $M^{\prime}\subset M^{\prime\prime}$ and $f^{\prime\prime}=f^{\prime}$ $\ M^{\prime}$ such that $(M^{\prime},f^{\prime})\in\Omega$ there exis maximal totaly orded subcoecto be the collection of all is the function which occus n the pai $({\cal M}^{\prime},f^{\prime})\in\mathrm{G}^{2}.$ ${\widetilde{\cal M}},$ that Y were a proper subspace of is wll dcfined on Thus ${\tilde{M}}=\,X$ and this would contradict the maximality o $\Omega.$ Finally the inequality $\Lambda\leq p$ implies that $$ -p(-x)\leq-\Lambda(-x)=\Lambda x $$ for al $x\in{\mathcal{X}}.$ This completes the proof. // 3.3 Theorem Suppose $\mathcal{M}$ is subspace of a vector space ${\cal{N}},{\qquad}$ p is $\bar{\boldsymbol{Q}}$ seminorm on $X,$ and f is a limear junctional on $\mathcal{M}$ such that $$ \vert f(x)\vert\leq p(x)\qquad(x\in M). $$ Then fexendstoalinear fucioal A on $X$ that satisfe lAxl≤ p(e) (xe X)58 GENERAL TuoRY PRoor If the scalar field is ${\boldsymbol{R}},$ this is contained in Theorem $\mathbb{3}z,$ since $D\!\!\!\!/$ now satisfies $p(-x)=p(x).$ ${\underline{{C}}},$ Put $u=\mathrm{{Re}}\,j.$ By Theorem 3.2 there is Assume that the scalar field is a real-linear ${\boldsymbol{U}}$ on $X$ such that $U=u$ on $\mathcal{M}$ and $U\leq p$ on $X.$ Let $\Lambda$ be the complex-linear functional on $\textstyle X$ whose real part is $U,$ The discussion in Section 3.1 implies that $\Lambda=f\operatorname{on}\,M.$ Finally, to cvery $x\in X$ corresponds an $x\in{\mathcal{C}},\;|x|=1,$ such that αAx= |Ax| Hence $$ \vert\Lambda x\vert=\Lambda(x x)=U(x x)\le p(a x)=p(x). $$ // Corollary If $\textstyle X$ is a normed space and $x_{0}\in X.$ there exists $\Lambda\in X^{*}$ such that $$ {\bf\lambda}{x}_{0}=\|x_{0}\|\qquad a n d\qquad|{\bf\Lambda}{x}|\leq\|x\|\qquad f o r\;a l l\;{x}\in X. $$ PRoor If $x_{0}=0,$ take $\Lambda=0$ .1f $x_{0}\neq0,$ apply Theorem 3.3, with $p(x)=\|x\|,$ M the one-dimensional space generated by $x_{0}\,,\,\mathrm{and}\,f(x x_{0})=x\|x_{0}\|\,\mathrm{on}\,M.$ $J/{\mathrm{/~}}$ 3.4 Theorem Suppose A and B are disioin, nonempty, convex sets in a topological vector space $X.$ (a)If A is open there exist $\Lambda\in X^{*}$ and $\gamma\in R$ such that $$ \operatorname{Re}\land x<\gamma\leq\operatorname{Re}\land\gamma $$ (6) ${\boldsymbol{J}}$ A is compact, $\boldsymbol{B}$ andl Jor every $y\in B.$ is locallyconvex, then here exist $\Lambda\in X^{\rtimes},$ for every $x\in{\mathcal{A}}$ is closed,and $\textstyle X$ $\gamma_{1}\in R,$ $\gamma_{2}\in R,$ such that $$ \mathrm{Re}\,\Lambda x<\gamma_{1}<\gamma_{2}<\mathrm{Re}\wedge y $$ for every x∈ A and for every $y\in B$ Note that this is stated without specifying the scalar field; fit is $\textstyle K,$ then Re A = A,of course pRoor It is enough to prove this for real scalars. For if the scalar field is ${\boldsymbol{C}}$ $\textstyle X$ and the real case has been proved, then there is a continuous real-inear $\mathbb{A}_{1}$ on that gives the required separation; if $\mathrm{\A}$ is the uniquc complex-linear functional on $\textstyle{X}$ (a)Fix $a_{0}\in A$ $b_{0}\in B.$ Put $x_{0}=b_{0}-a_{0}\mathrm{;}$ put (See Section 3.1.))Assume real scalars Then whose real part is $\Lambda_{1},$ then $\Lambda\in X^{\bullet}$ $C=A-B+x_{0}.$ ${\boldsymbol{C}}$ is a convex neighborhood of ${\boldsymbol{0}}$ in $X.$ Let $D\!\!\!\!/$ be the Minkowski functional of ${\boldsymbol{C}}.$ By Theorcm 1.35 $D\!\!\!\!/$ satisfies hypothesis(b) of Theorem 3.2.Since $A\;Y\;B=\mathcal{D}$ ${\mathfrak{c}}_{0}\not\in C,$ and so $p(x_{0})\geq1$ of $\textstyle X$ generated by ${\mathcal{X}}_{0}\ .$ If $\scriptstyle t\,\geq\,0$ then ${\mathrm{Define}}\,f(t x_{0})=t$ on the subspace $\mathcal{M}$ $f(t x_{0})=t\leq t p(x_{0})=p(t x_{0});$covExiy59 if $t<0\ {\mathrm{then}}\,f(t x_{0})<0\leq p(t x_{\alpha}),$ ). Thus /≤p on $\mathcal{M}.$ By Theorem $3.2,J$ extends ${\cal{C}},$ hence to a inear ftnctionaiX on on $-C.$ so that $|\lambda|\leq1$ on the neighborhood ln patic $[\operatorname{ar},\Lambda\leq1$ on ${\boldsymbol{0}},$ If now $a\in A$ $\textstyle X$ that asatsfe $\Lambda\leq p.$ of $\mathrm{A}\geq-1$ $8,\Lambda\in X^{*}.$ $C\cap(-C)$ By Theorem 1.1 and $b\in B,$ we have 红x) PRor If $x_{1}\in X,$ x,∈ $X,$ and ${\boldsymbol{x}}_{1}\ \barwedge^{-}\ x_{2}\ ,$ $$ ^{*}\Lambda a-\wedge b+1=\Lambda(a-b+x_{0})\leq p(a-b+x_{0})<1 $$ $\Lambda a<\Lambda b.$ ${\mathit{V}}$ of $\mathbf{0}$ in $\textstyle{X}$ such since $\Lambda x_{0}=1.$ $a-b+x_{0}\in C,$ and ${\boldsymbol{C}}$ is open. Thus in place of ${\cal A},$ shows that there ,with $\Lambda(A)$ It follows uhat $\operatorname{\mathcal{N}}(A)$ and $\Lambda(B)$ are disjint convx subsets o $X.$ $\textstyle R_{\mathrm{{,}}}$ ${\boldsymbol{R}},$ that $\Lambda(d+V),$ to the left of $\Lambda(B).$ Also, $\Lambda(A)$ is a non stsc s opend n be the with $\Lambda(d+V)$ exrxaonsanineran $\Lambda(A)$ to xet cnso part o is anopen mapping Cet $\scriptstyle{\mathcal{Y}}$ // exists $\Lambda\in X^{\bullet}$ such that open and to the left of I $\textstyle{\mathcal{X}}$ are disint convex subsets o (6) right cnd pointo B Tnoe eswevwenoAo $d+V$ $(A+V)\cap B=\mathcal{Q}.$ Part Go), wih $\Lambda(A+V)$ and $\Lambda(B)$ $\Lambda(B)$ Since $\Lambda(d)$ is a compact subset o we otantheconcsionon'O) eomouy rnohlel-sonse $X^{\ast\ast}$ separates points o $B=\{x_{2}\}.$ apply (b) of Theorem 3.4 with ${\boldsymbol{A}}=$ for every $x\in M.$ 少/ e es .nmeog $\Lambda\in X^{\bullet}$ such that $\mathrm{A}x_{0}-1$ $J/{\big/}/$ 3.5 Thore,A xaro olone un ${\cal X},$ and $x_{0}\in X.$ $\scriptstyle f o x_{\mathrm{e}}$ bu $\mathbb{A}x=0$ some subspace $\bar{M}$ By (の) of Thcoren 3.4, wit $\Lambda(M)=\{0\}$ and $\mathrm{A}{x_{0}}~\neq\;0.$ is a proper subspace of the $\Lambda\in X^{*}$ PRO0OF $A=\{x_{0}\}$ $\textstyle X$ Wasnaen $\bar{\cal M},$ // such that $\Delta x_{0}$ and ACM are dijoin.”Ths and $B={\bar{M}},$ there exis scalarfeld.This foc $\Lambda(\mathcal{M})$ The desiedfitinai obuain diA y"x for voy ounun nmicnto $\Lambda$ Som…………分……… esones o $x_{0}\in X\rfloor\log$ in th losurc o 叫 $\textstyle X$ aprowionpoes re oeta S Celycy w $\Lambda x_{0}=0$ n si $X$ 心 Ao $X,$ then there exis $\Lambda\in X^{*}$ such thai ^ =于onM comvex space M从… og no u n Aomrog esososin s usesoeoun noemn Tne emne esa ososomsgsersm60 ctNERAL rno of inear functionals to seminorms (see Exercise 8, Chapter i). The proof given below shows that Theorem .6 depends only on the scparation property of Theorem 3.5. PROOF Assume withou loss of geraliy that $\boldsymbol{\mathit{f}}$ is not idetical ${\boldsymbol{0}}$ on $M.$ Put $$ M_{0}=\{x\in M\colon f(x)=0\} $$ $X_{0}$ and pick $x_{0}\in M$ such th $\mathrm{\lat}\;f(x_{0})=1.$ Since $\boldsymbol{f}$ is continuous, $X_{0}$ is not in the $M{\mathrm{:}}$ -closure of $\scriptstyle M_{\alpha}$ , and sinc $\mathcal{M}$ inherits its topology from $X,$ it follows that is not in the ${\dot{X}}\cdotp$ -closure of $\scriptstyle M_{\alpha}$ $\Lambda\in X^{*}$ such that Theorem 3.5 therefore assures thc existence of a $\wedge x_{0}=1$ and $\Lambda=0$ on $M_{0}$ , since $|(x_{0})=1$ Hence If $x\in M,$ then $x-f(x)x_{0}\in{\cal M}_{0}$ $$ \Lambda x-f(x)=\Lambda x-f(x)\Lambda x_{0}=\Lambda(x-f(x)x_{0})=0 $$ Thus $\Lambda=f$ on $M.$ // We conclude this discussion with another useful collary $\mathbf{Of}$ the separation theorem. 3.7 space X $x_{0}\in X,$ but $x_{0}\notin B.$ Then there exists $\Lambda\in X^{*}$ is a convex, balanced, closed set im a locally convex such that $|\Lambda x|\leq1$ for al Theorem Suppose $\boldsymbol{B}$ $x\in B;$ but $\Lambda x_{0}>1$ PRoor Apply(O) of Theorem 3.4, with $A=\{x_{0}\},$ notc that $\Lambda(B)$ is convex and // balancd, and multipy te corresponding $\mathrm{\A}$ by an appropriate scala. Weak Topologies than $\tau_{1}$ identity mapping on $\textstyle{X}$ Topological preliminaries The purpose of hiseton is to explain an and assume $\tau_{1}\in\tau_{2};$ that ${\mathfrak{I S}},$ every 3.8 ilturatsome of th phenomena that occtr wen set istologiedin sveral ways Let $\tau_{1}$ T, and $\tau_{2}$ be two topologies on $\underline{{\mathbf{a}}}$ set $X,$ to $(X,\tau_{1})$ and is an open mapping is stronger t,-open set is also $\tau_{2}.$ -opcn. Then we say that r $\tau_{1}$ r, is weaker than $\tau_{2\,:}$ ,or that ${\boldsymbol{\tau}}_{2}$ E)the .TNot that inacordance with the meaning o the incuion symbol $(X,\tau_{2})$ $\mathbb{C},$ terms iweaxervand stonger” donot excludc quality」 n hisiution, th is continuous from from $(X,\tau_{1})$ to(X, T2) As'ai fsi iston e us prove that tetogy of a compact Hausdo spac hasa etaniy t ses hat cano veked ithot osing t Hhasrf spaionaxom an cano e stengthened wihout osin copactnes$\mathbf{\hat{\Pi}}$ CONVExrr 61 .(a) $J/\tau_{1}\subset\tau_{2}$ are tologies o $\boldsymbol{\alpha}\,$ set X,; $\tau_{1}$ is a Hlusdorf olgy, and ${\dot{J}}\ \tau_{2}$ is compact, then $\tau_{1}=\tau_{2}$ every $\pi_{\alpha}$ To see this、le $\tau_{1}$ isasiostsn closed. Since $\textstyle{X_{\alpha}}$ s is compac Hiusor space.”Ten qin $X_{\alpha}$ α,and the Since all topology on 星 ,it follows that $F\subset{\mathcal{X}}$ be $\tau_{2}$ ${\mathcal{P}}$ is a nonempty family of m ${\tau}^{\prime} arrow{\tau}_{N}\subset{\tau}^{\prime}.$ 叫 ${\boldsymbol{F}}.$ is as az-eoe is the ${\mathrm{with}}\,f\in{\mathcal{F}}$ tia ooa 。 $\textstyle X$ that roq s estinc wnispsn yvoiseonoP -compac, so s 哈 $F.$ e h $\tau_{1}\subset\tau_{2}\,.$ that property, then $\tau\subset\ \tau^{\prime}.$ Th $\overline{{\tau}}$ .io, cmoctCrxr $\textstyle X$ is $\tau_{2}$ A $\tau_{1}$ y of $X/N,$ as defined in ,Or, $f\in{\mathcal{F}}.\quad$ lt ${\mathbf{}}V$ 1f .x dentst e tcoonaeita pon $x\in\,X_{i}$ $\tau_{1}.$ Sa open sove $\tau_{N}$ $\mathrm{if}$ is in fact the weakest $\tau_{N}$ ${\mathcal{P}}.$ $X_{\alpha}$ where each and open in ${\mathbf{}}F$ ${\mathit{Y}}_{f}$ Then ris atology on ama then ${\boldsymbol{F}}$ 炒 $X/N_{*}$ , and if r is covern $\textstyle Y_{f}$ .isa polesae mnymoiamas ${\boldsymbol{\tau}}^{\prime}$ $\displaystyle{\Bigg T}^{\not{p}}$ aie toroies $\mathrm{ppings}f\!:X\to Y_{f}.$ Suppsenex tha $\textstyle X$ Ao aoruiocsosei oieme o $\pi\!\cdot\!\,A\!\!\!/\,A\!\!\!/\,\to\,A\!\!\!/\,J\!\!\!/\,\Lambda.$ $\displaystyle{\overline{{\chi}}}^{\beta\beta}$ By its very definition Section 1. and he quonen ma uhatms e ous a evak strongest topology on $X/N$ is a etan moatomo onmpipLEiuity ${\boldsymbol{\tau}}^{\prime}$ and peieavet coninus reatvet $\textstyle X$ $\overline{{\tau}}$ one topologizes thecartesian produc $X$ be tce ecio imso neie $\textstyle Y_{f}$ W is the same for continus. Assme ow tatve , then $\pi_{\alpha}$ mv $\textstyle{X}$ ${\mathfrak{s e t s}}\,f^{-1}(V),$ Y that makes every $J\in{\mathcal{P}}$ .continus: i1 $X,$ K, and mae imno ereos XS…s:toio ${\boldsymbol{\tau}}^{\prime}$ is an oer toiy w more sucnctly, he -oloy o X is cal ek oy”" $\textstyle{\bar{X}}$ onto ionhohososmispoisusnouousul u w yn wie maps of a colectionof oica spac uy CSononAesomomo6ohr (の)If S sa Jamily o mapings : l h aente secaseomwposo…mws setnu n $X_{\mathrm{{s}}}$ hente -oy a。Aa $\textstyle Y_{J}$ is a Haus dorf space, and if $X\to Y_{f},$ where $\textstyle X$ is a set and each dorff topology. ${\mathcal{F}}$ searaosoims $|f_{n}|\leq1$ For if $P\not\subset q$ are points of $X,$ ,then $f(p)\neq f(q)$ for some Je E;thepoints fOo for a ${\boldsymbol{n}}_{\mathrm{{,}}}$ agsAaoisosnsitsos ${\boldsymbol{\tau}}_{d}$ be the oy ncet o wonowasemseatne eo Let r be the given topology of $X.$ $\mathbf{}Y_{f}$ , uhen $\textstyle X$ is metrzable by the metric definition and disjoint. $X,$ lier piot s ses o mtioneore valed uctions spae oitso 0Misomn ne s m omesecue ., omus a and le $X$ Suppse, witout oss of ealit, ta $$ d(p,q)=\sum_{n=1}^{\infty}2^{-n}|f_{n}(p)-f_{n}(q)|\,. $$62 GENERAL THEORY This is indeed a metric, since{fA} separates points. Since $\operatorname{cach}f_{n}$ is r-continuous and the series converges uniformly on $X\times X,$ , d is a r-continuous function on $X\times X.$ The balls $$ B_{r}(p)=\{q\in X\colon d(p,q)<r\} $$ are therefore r-open. Thus $\tau_{d}\subset\tau.$ Since ra ${\boldsymbol{\tau}}_{d}$ a is induced by a metric, ${\boldsymbol{\tau}}_{d}$ 二 , is a Hausdorff topology, and now $(a)$ implies that $\mathbf{t}=\tau_{d}\,.$ The following lemma has applications in the study of vector topologies. In fact the case $\scriptstyle n\;=\;1$ was needed (and proved) at the end of Theorem 3.6. 3.9 Lemma Suppose $\mathrm{A}_{1},\ldots,$ $\mathrm{\A}_{n}$ and $\mathrm{\A}$ A are linear functionals on a vector space $X.$ Let $$ N=\{x;\mathrm{A}_{1}x=\cdot\cdot\cdot\cdot=\Lambda_{n}x=0\}. $$ The following three properties are then equivalent: (a)There are scalars $x_{1},\ldots,a_{n}$ such that $$ \Lambda=\sigma_{1}\Lambda_{1}+\cdot\cdot\cdot+\sigma_{n}\Lambda_{n}. $$ (6) There exists $\gamma<\alpha.$ such that $$ \big|\Lambda x\big|\leq\gamma\ \ \mathrm{max}_{\begin{array}{l}{{1}\leq i\leq n}\end{array}}\ |\Lambda_{i}x| $$ (c) $\Lambda x=0\,f o r\;e v e r y\,\,x\in N.$ PRO0F It is clear that $\mathbf{\Psi}(a)$ implies $\mathbf{\Psi}(D)$ and that (b)implies (c).Assume(c) holds. Let $\Phi$ be the scalar field. Define $\pi\colon X\to\Phi^{n}$ by $$ \pi(x)=(\Lambda_{1}x,\ldots,\Lambda_{n}x). $$ ${\mathbf{}}F$ on ( $\Omega^{n}.$ $\pi(x)=\pi(x^{\prime})$ then $\left(c\right)$ implies $\Lambda x=\Lambda x^{*}.$ Hencc $\mathrm{A}=F\circ\pi,$ for some function If This ${\mathbf{}}F$ F s alinear functional on $\bar{\Phi}^{n}.$ ”. Hence there exist $x_{i}\in\Phi$ such that $$ F(u_{1},\ldots,u_{n})=\alpha_{1}u_{1}+\cdot\cdot\cdot+\alpha_{n}u_{n}. $$ Thus which is (a) $$ \begin{array}{c c}{{\Lambda x=F(\pi(x))=F(\Lambda_{1}x,\,\cdot\cdot\cdot\cdot\cdot\cdot\Lambda_{n}x)=\sum_{i=1}^{n}\alpha_{i}\Lambda_{i}x,}}&{{}}&{{}}\\ {{}}&{{}}&{{}}\\ {{,}}&{{}}&{{}}\end{array}\qquad\qquad\qquad\begin{array}{c}{{}}\\ {{\Lambda x=\rho_{i}x,}}\\ {{\Lambda y/\lambda_{i}x,}}\\ {{\eta}}&{{}}&{{}}\end{array} $$ 3.10 Theorem Suppose $\textstyle X$ is a vector space and $X^{\prime}$ is a separating vector space of linear functionals on $X.$ Then the $X^{\prime}.$ -topology ${\boldsymbol{\tau}}^{\prime}$ makes $X$ into a locally convex space whose dual space is XYcoxvExrr 63 Te assumptions on $X,$ 'are, more expicity, that for some $\Lambda\in X^{\prime}$ whenever $X_{\mathrm{1}}$ and $X_{2}$ are distinct points or $X^{\prime}$ $\Lambda x_{1}\neq\Lambda x_{2}$ is closed under addition and scalar multiplication and that $X^{\prime}$ PROr Since ${\boldsymbol{R}}$ and ${\boldsymbol{C}}$ Z are Hlausorf paces $\mathbf{\phi}(\partial)$ of Ssection 3. shows that $\neq{\mathit{\frac{^{\prime}}}{\ell}}$ is trans- is a lation-invariant $\operatorname*{lf}\Lambda_{1},\,\ldots,\,\Lambda_{n}\in X^{\prime},\,\mathrm{if}\,r_{i}>0,$ , and if $X^{\prime}$ shows that ${\boldsymbol{\tau}}^{\prime}$ Hsrtolox heiniy emen (1) $$ V=\{x;\ |\Lambda_{i}x|<r_{i}\quad\mathrm{for}\quad1\leq i\leq n\}, $$ then ${\mathbf{}}V$ is convex, balanced an ${\boldsymbol{\tau}}^{\prime}.$ Thus ${\boldsymbol{\tau}}^{\prime}$ is a caly convex topology on $\scriptstyle s\;>0$ .Ir $|\beta-\alpha|<r$ and If form O)isa loa base fr $V\in\tau^{'}.$ ln fact thcoclctio $\mathbf{o}\mathbb{I}$ all ${\mathbf{}}V$ of the Suppose $x\in{\mathcal{X}}$ and $\scriptstyle{\mathcal{X}}$ is a sclar. Then $x\in s{\mathcal{V}}$ $\textstyle X$ $\mathbf{(1)}$ holds, the $\textstyle{\frac{1}{2}}V+{\frac{1}{2}}V=V.$ This pryes thadion s ontinu $y-x\in r V$ then for some $$ \beta y-\,o x=(\beta-\alpha)y+\alpha(y-\,x) $$ lies in ${\mathit{V}},$ provided th ${\mathbf{}}T$ is sosmall ta $$ r(s+r)+|\alpha|r<1. $$ $\Lambda_{i}\in X^{\prime}$ and $X^{\prime}$ Hencesca mutipicton i continuou is a oal covxvectortopolgy: Evcr $\Lambda=\sum^{\alpha_{i}}\Lambda_{i}.$ Since $\Lambda\in X^{\prime}$ is We have now proved tha $|\Lambda x|<1:$ ior all ${\mathfrak{A}}_{\mathfrak{Y}}^{\mathfrak{Y}}$ in some s $\mathbb{A}$ is a T ${\boldsymbol{\tau}}^{\prime}.$ continuous linear func // tional on $X.$ Then 'is vecor space, ${\boldsymbol{\tau}}^{\prime}$ of the orm T.Condito ${\boldsymbol{\tau}}^{\prime}$ -coninous Coveses sipos $\Lambda\in X^{\prime}.$ ${\mathbf{}}V$ 62 C Lm tecr oios ics oe o This completes the proo Aorofsus ou ave sea o horem 15 $p_{\Lambda}(x)=|\Lambda x|.$ amd hes nts stismmxx a son implies that calle te orima tology ot $X.$ ,wvaktox.。 oeavetorspae uppo $\tau{\mathrm{geqslantJ}}$ whose dual $X.$ lt also happens insome others; $X$ is a Sinc every topoloical vcorspace 《wih opoiogy -topogy o topoloize tswx o ${\overline{{\overline{{\cal C}}}}}_{\mathrm{{\scriptsize~\mathcal{U}}}}$ is the weakest topology on $X$ $X.$ see Exercise 5.)The CWe know utispapns er ialycov $X^{\ast}$ $\overline{{{\mathbf{f}}}}_{\nu\nu}$ separates points on $X$ We shall le $X^{3\star}.$ $\textstyle X$ is cald he weak oyo $X^{\geq}$ Theorem 3.10 $X_{\w}$ $X_{\mathrm{w}}$ denot $\textstyle{\mathcal{X}}$ hn thi contxte given tology wioten b with thatproperty, we have $\tau_{w}\subset\tau.$ is oaly convex space whose u is s' $\Lambda\in X^{*}$ is -continuous and since64 CENERAL THFORY that $x_{n}\to0$ For instance, let Self-explanatory expressions such as orinal neighborhood, weak neighborhood. originally means To say origna ciosureweak closur, oinalybonded. weakly bounded. t. wl be us to make it cea with respet to which topology these terms are to be undrstood that every original neighborhood of O contains all $\textstyle x_{n}$ To say that $x_{n}\to0$ ${\boldsymbol{n}}.$ with $\scriptstyle\{x_{a}\}$ be a sequencc in $X,$ with sufficiently large weakly means that every weak neighborhood $\mathbf{0}$ contains a neighborhood of $\textstyle{X_{n}}$ ${\mathfrak{o}}\colon0$ contains al sufficienuly large n. Since every weak neighborhood of the form (1) $$ V=\{x\colon|\Lambda_{i}x|<r_{i}\quad{\mathrm{for}}\quad1\leq i\leq n\}, $$ where $\Lambda_{i}\in X^{\bullet}$ and $r_{i}>0,$ it is easy to see that $x_{n}\to0$ weakly if and only if $\wedge x_{n}\to0$ for every $\Lambda\in X^{\ast}$ Hencevery orinaly convergent sequence converges weakly(The convers for every is usually false; sce Exercises $E\subset X$ is weakly bounded (that is, ${\boldsymbol{E}}$ is a bounded subset of $\scriptstyle{X_{\circ}}$ if $\bar{\bf S}$ and 6.) Similarly, a set as in $\operatorname{\mathcal{}}(1)$ contains ${\boldsymbol{t}}{\boldsymbol{E}}$ for some $t=t(V)>0$ This happens if and every and only if every $x\in E$ In other words $\bar{\boldsymbol{a}}$ set $E\subset X$ is weakly bounded if and only $|\land x|\leq\gamma(\Lambda)$ ${\mathbf{}}V$ only if there corresponds to each $\Lambda\in X^{s}$ a number $\eta(\Lambda)<\alpha_{c}$ such that $i f$ $\Lambda\in X^{*}$ is a bounded function on $k.$ Let ${\mathcal{V}}$ ragain be as in (1), and put $$ N=\{x\colon\Lambda_{1}x\equiv{}\cdot{}\cdot{}\cdot{}=\Lambda_{n}x=0\}. $$ $n+\dim N.$ Since $x arrow(\Lambda_{1}x,\ .\cdot,\Lambda_{n}x)$ maps $\textstyle X$ into $C^{n},$ , with nullspace $N_{\cdot}$ V, we see that dim $\textstyle X$ Since $N<V$ , this leads to the following conclusion. dimensional subspace; hence I/ X siniedimnsiomal hen every weak neighorhodofO contains an infini $\textstyle X_{\w}$ is not locally bounded. This impies in many cases ha the weak topology is ticty weaker han the $(X_{\nu})_{\nu}=X_{\nu}.$ original onc. Of course,the two may coincide:Theorem 3.10 implis tha We now come to a more interestng result 3.12 Theorem Suppose ${\boldsymbol{E}}$ is a convex subset of。locally convex space $X.$ Then the weak closure ${\tilde{E}}_{w}$ of工 $\boldsymbol{\mathit{F}}$ yis equal to its original closure ${\overline{{E}}}.$ i When Xisa Fechetspac hece, in pricuar, when Xis a Banach spacy theorieginal iopbos rsycysesionovoy… Atcxtee0… ana'sirony”Wiie use in piae or originai“and orinallyFr loca eonosesaneesee…gsoogsan . tesuanaSisani s se us pP 56-28 a1.0 semnieroio avisabe usesorina te resenteaisussicoNVExrrv65 PROOF ${\bar{E}}_{\ ;\nu}$ is weakly closed, hence originally loed,so that $\gamma\in{\mathcal{R}}$ such that, for every $x\in E,$ To obtain $3.4$ the opposite inclusion, choose $x_{0}\in X,\,x_{0}\notin F.$ Part ${\overline{{E}}}\subset{\overline{{E}}}_{w}$ shows that there exist $\Lambda\in X^{**}$ and $\mathbf{\nabla}(b)$ of the separation theorem $$ \mathrm{Re}\,\Lambda x_{0}<\gamma<\mathrm{Re}\,\Lambda x. $$ The sct {x: $\textstyle E.$ Thus ${\mathcal{X}}_{0}$ is not in ${\tilde{E}}_{w}$ This proves $E_{w}\subset E.$ that does not // intersect $\operatorname{Re}\land x<\gamma\}$ is thererore a weak neighborhood of $X_{\mathrm{0}}$ Corollaries Let X be aloally comex space (b) $\scriptstyle A$ (o) A subspace o $\textstyle X$ is orialy cosed f md oly ifisweakl closed convex subset of $\textstyle X$ Y is origially denseif and only ifit is weakly dense 3.12: The roofs are obious. Here is aoter noteworthy consequence of Theore 3.13 Theorem Suppose $X$ is a metrzble locally comex space. $J f\left\{x_{n}\right\}$ is a sequence im $X$ that converges weakly $I O$ some ${\mathcal{X}}\in X,$ then there is a sequence $\langle y_{i}\rangle$ in $\textstyle X$ such that (a) each $y_{i}$ is a comwex combination of fnitely many $X_{n}$ , and (b) $y_{i}\to x$ originally. Conclusio $\mathbf{\Psi}(a)$ says, more explicitly, that there exist numbers $x_{i n}\geq0,$ such that $$ \sum_{n=1}^{\infty}\alpha_{i n}=1,\qquad y_{i}=\sum_{n=1}^{\infty}\alpha_{i n}x_{n}, $$ and, for cach i, only finitely many ${\mathcal{Q}}_{i n}$ ${\mathrm{arc}}\neq0$ $H.$ sequence the origina topology of $\textstyle X$ be the conve hull te set of al $\sim\sum_{y\mid_{J\mid2}}^{v}\omega_{J}$ let $\textstyle K$ he the weak closure of $H.$ Since PROOF Let ${\boldsymbol{H}}$ Then $x\in K.$ By Theorem 3.12,xis also in the original closure o $J/\slash$ $\langle y_{\bar{\imath}}\rangle$ in ${\boldsymbol{H}}$ I that converges orgnally to x is asmd to b metrizable i olowsthatere sa that Let $|f_{n}(x)|\leqslant1$ for all $\scriptstyle n$ n and all $x\in K$ To get a feeling for what is involved here, consider the following example. and f $(n=1,\,2,\,3,\,\cdot\cdot)$ as $n arrow\infty,$ and such complex functions on ${\cal K}$ such that $f_{n}(x)\to f(x)$ be a compact Hausdrf spc the unit intrval on the real ine is $x\in K,$ are continuous $\textstyle K$ suficienty interesting one), and assume that $\boldsymbol{f}$ combinations of $\mathrm{the}\,f_{n}$ for every that converge uniformly tof Theorem 3.3 asserts that there are convex on $K,$ To see this, Het ${\boldsymbol{C}}({\boldsymbol{R}})$ be the Banach space of all complex continuous functions convergence on $K.$ normed by the suprcmum. . Then strong convergence is the same as uniform Tf $\boldsymbol{\mathit{1}}$ is any complex Borel measure on K, Lebesgue's dominated66 CENERAL THEORY $$ \hat{\boldsymbol{l}} $$ convergence theorem implics tha $\textstyle{\int}_{n}\,d\mu\to{\int}f\,d\mu.$ Hence $f_{n}\to f$ weakly, by the Riesz representation theorem which idenies the dualof ${\mathcal{C}}(K)$ with the space of all regular complex Borel measures on K. Now Theorem 3.13 can be appi After hshort detour we now return to our main line of delopment separates points on $X$ 3.14 The weak*-topology of a dual space Let $\textstyle X$ again be a topological vector $X^{\2{^{\geq}}}$ space whose dual is $X^{\geq}$ For the definitions that follow it s irrelevant whether $x\in X$ or not. The important observation to make isthatery induces a linear functional $f_{x}$ on $X^{\mathfrak{r}}$ *, defined by $$ f_{x}\Lambda=\Lambda x, $$ and that $\{f_{x}:x\in X\}$ separates points on $X^{\star}$ for all $x\in X,$ then $\Lambda x=\Lambda^{\prime}x$ for all x, and so The Iinearity of each $f_{x}$ is obvious; if $f_{x}\wedge=f_{x}\wedge^{\prime}$ $\Lambda=\Lambda^{\prime}$ by the vry efniono whatit mes fo tw functions to b equal We are now in the situation described by Theorem 3.10, with $X^{\rtimes\varepsilon}$ in place of $X$ The X and with X in place of $\textstyle X^{\prime}$ (pronunciation: weak $X.$ topology of $X^{\rtimes}$ is called the weak*-topology of $X^{\rtimes}$ star topology) that every linear fumctional on $X^{\rtimes}$ Theorem 3.10 implies that this is a locally convex vector topology on $X^{\bullet}$ and that is weak*-continuous has the form $\mathrm{A}{ arrow}\,\mathrm{A}x$ for some $x\in X.$ Various pathological features of the wcak- ard weak*- The weak*-topologies have avery important compactness prperty to which w now turn our attention. topologies are described in Exercises 9 and $10.$ Compact Convex Sets 3.15 The Banach-Alaoglu theorem 1 ${\mathbf{}}V$ is aneighbohood o O in atoloyicl vector space $\textstyle X$ Y and i/ $$ K=\{\Lambda\in X^{*}\colon|\Lambda x|\leq1\quad{\mathrm{for~every}}\quad x\in V\} $$ then ${\cal K}$ is weak*-compact. Note: $\textstyle K$ is sometimes called the polar of ${\mathbf{}}V$ It is clear that $K$ is convex and balanced, because this is true of the unit disc in C (and of the interval $[-1,1]$ in $R_{1}=C(--x).$ $\textstyle X$ There is some redundancy in the definition of K, since cvery lincar flunctional on that is bounded on Vis continuous, hence is in X*CONVEXrTY $67$ $x\in{\mathcal{X}}$ pxoor Sincc neighborhoods or $\gamma(x)<\infty$ such that are absorbing, the corrsponds to each $\mathbf{0}$ a number $x\in\gamma(x)V$ Hence (1) $$ |\operatorname{A}\!x|\leq\gamma(x)\qquad(x\in X,\land\in K). $$ $D_{x}$ Let is compact so is ${\boldsymbol{P}},$ be the et o al sars such that $|\alpha|\leq\gamma(x)$ Letr be the produc are the $D_{x}$ the cartesianprodnct o a ${\mathfrak{p}}_{x},$ one for cach ${\mathcal{X}}\in X.$ Since each topology on ${\boldsymbol{P}}_{\mathrm{\bar{z}}}$ functions f on $\textstyle X$ C(inear ornot thatsatis by Tychof's theorm The ementso $D\!\!\!\!/$ (2) $$ .\qquad|f(x)|\leq\gamma(x)\qquad(x\in X). $$ $X^{\star}$ Thus $K\subset X^{*}\cap P.$ lt follows that $\textstyle K$ inheris two tologies one fro other, , from $\boldsymbol{P}$ We wil se tha i so oyi. ih tecusmo teoemeen sn (a) hese two togis cincide o ${\boldsymbol{P}}.$ , and (b) $\textstyle K$ is a closed subset o $K.$ $\textstyle K$ Since ${\boldsymbol{P}}$ P s compact,(b) implies ta Choose $x_{i}\in X.$ , for $1\leq i\leq n;$ choose $\delta>\mathbb{C}$ Put is weak*-compact $\textstyle K$ is r-compact, and then $\mathbf{\Psi}(a)$ implies that Fix soe $\Lambda_{0}\in K.$ (3) $$ \mathcal{W}_{1}=\left\{\Lambda\in{\cal X}^{\star}\!:\, |\Lambda x_{i}-\Lambda_{0}\,x_{i}\right|\,<\delta\quad\mathrm{for}\quad1\leq i\leq n\rangle $$ and (4) $$ W_{2}=\{f\in P\colon|f(x_{i})-\Lambda_{0}\,x_{i}|\,<\delta\quad\mathrm{for}\quad1\leq i\leq n\}. $$ Let ${\boldsymbol{n}}_{*}$ $X_{i}\,,$ and 8 $\bar{\partial}$ range ovr all admisilevalues of $D\!\!\!\!/$ at $\Lambda_{0}$ Since $K\subset P\cap X^{*}.$ we have form ol s otit eaeoy The resuling sets $W_{1}$ then local base for the product topoloy ${\boldsymbol{X}}^{\lambda}~{\boldsymbol{\alpha}}\mathbf{t}~{\boldsymbol{\Lambda}}_{0}$ and te sets ${\hat{\mathcal{N}}}_{2}$ form a $\tau$ α and ${\boldsymbol{\beta}},$ and Next, sppose $f_{\mathrm{0}}$ 0. Thc sct of all /e ${\mathbf{}}{\mathbf{}}\quad$ $$ W_{1}\cap K=W_{2}\cap K. $$ $K.$ Choose $x\in X,\;y\in X,$ at ${\boldsymbol{y}},$ and at This proves a) $\alpha x+\beta y$ fis linear, we have is in the r-closure of P such that $|f-f_{0}|<\varepsilon{\mathrm{~at~}}$ $X_{},$ scalars ${\cal S}>\ ^{4}$ is a r-neighborhood of $\Gamma f_{0}$ Therefore $\textstyle K$ contains such an fSince this so that Ja /)- 60- 60=(6。- 0x 0+ (/-/0+ (0/- /0 same argument shows that thie $\mathbf{i}\mathbf{s}$ $$ |f_{0}(\alpha x+\beta y)-{\alpha}f_{0}(x)-\beta f_{0}(y)|<(1+|\alpha|+|\beta|)\varepsilon. $$ $f\in K$ such that fO)-/6(01<。 $\varepsilon>0_{s}$ the sincexasatay se tat inearTFnly $c\in V$ and an68 GENERAL THEo Since $|/(x)|\leq1,$ by the definition of $K,$ K, it follows that $|f_{0}(x)|\leq1.$ We conclude $l/{\big/}/$ that $f_{0}\in K$ This proves $\mathbf{\phi}(D)$ and hence the theorem. When $\textstyle X$ is separable (i.e,, when there is a countable dense set in X) then th conclusion of the Banach-Alaoglu theorem can be strengthened by combining it with the following fact: 3.16 Theorem If $\textstyle X$ is a separable topological vector space, if $K\subset X^{*}$ and if $K$ is weak*-compact, then $\textstyle K$ is metrizable,in the weak*-topology. Warning: It does not follow that $X^{2k}$ itself is metrizable in its weak*-topology. In fact, this s flse whenever $X$ is an ifinite-dimensional Banach space. See Exercise 15 Each pRoor Let $\scriptstyle\{x_{n}\}$ be a countable dense set in X $\textstyle X$ Y. Put $f_{n}(\Lambda)=\Lambda x_{n}$ for $\Lambda\in X^{*}.$ If $\ f_{n}$ is weak*-continuous, by the definition of the weak*-topology. $f_{n}(\Lambda)=f_{n}(\Lambda^{\prime})$ for all ${\boldsymbol{n}},$ then $\Lambda x_{n}=\Lambda^{\prime}x_{n}$ for all ${\boldsymbol{n}},$ which implies that $\Lambda=\Lambda^{\prime},$ since both are continuous on $\textstyle X$ and coincide on a dense set. Thus $\scriptstyle[f_{3}]$ is a countable family of continuous functions that separates of Section 3.8. $J/I$ points on $X^{\rtimes}$ The metrizability of K now follows from $\left(c\right)$ 3.17 Theorem If ${\mathbf{}}V$ is a neighborhood of O in a separable topological vetor space $X,$ and $i f\left\{\Lambda_{n}\right\}$ is a sequence in $X^{\geq}$ such that $$ \vert\Lambda_{n}x\vert\le1\qquad(x\in V,\,n=1,\,2,\,3,\,\ldots), $$ then there is a subsequence $\{\Lambda_{n_{i}}\}$ and there is $a\land\in X^{*}$ such that $$ \Lambda x=\operatorname*{lim}_{i arrow\infty}\Lambda_{n_{i}}x\qquad(x\in X). $$ In other words, the polar of ${\mathbf{}}V$ is sequentially compact in the weak*-topology PROOF Combine Theorems 3.15 and 3.16. // The next application of the Banach-Alaoglu theorem involves the Hahn- Banach theorem and a category argument. 3.18 Theorem lm u locally convex space X, every weakly bounded set is originally bounded, and vice versa. Part (d) of Exercise S shows that the local convexity of $\textstyle X$ cannot be omitted from the hypotheses.coNVExrr 69 O in $\textstyle X$ Suppose $E\subset{\mathcal{X}}$ gronsoyrsoigono o $U$ is mnminieotio sriosoeoso……n $\textstyle X$ A an neono iswalybondedan Aa s…sgsoan sunsosegowsesesuonsaoionehen (1) Since $X:$ scologypes po ox、 anc iniepho $V;$ hood V ofo in $\textstyle X$ such tha $V\subset U,$ Let $K\subset X^{*}$ be thc polar o Wwe claim that $$ K=\{\Lambda\in X^{*}:|\Lambda x|\leq1\quad{\mathrm{for~all}}\quad x\in{\mathcal{V}}\}. $$ (2) $$ {\bar{V}}=\left\{x\in X\colon\left|\Lambda x\right|\leq1\quad{\mathrm{for~all}}\quad\Lambda\in K\right\}. $$ and hence so is ${\overline{{V}}},$ since the $\overline{{V}}$ lt is cla that ${\mathbf{}}V$ isasubst o t riht sie $\mathbf{(2)}$ Theorem 3.7 (with a number in place of rih side o doseSppos $x_{0}\in X$ but $x_{0}\notin V,$ This proves (2 Since ${\boldsymbol{B}}\}$ then shows tha $\Lambda x_{0}>1$ for some $\Lambda\in K.$ $\Lambda\in X^{*}$ $\gamma(\Lambda)<\infty$ ${\boldsymbol{E}}$ AivsonsCn es Sel such that :(3) $$ |\Lambda x|\leq\gamma(\Lambda)\qquad(x\in E). $$ $X^{\sharp}$ in place of Since $K$ is osxos e rom . snsne nctio $\mathrm{A}{ arrow}\,\mathrm{A}x$ are x.onus enaipy Teoerm zs ew $\gamma\ll\infty$ X and the sarfl paceo ${\cal{Y}})$ to cocue fo ) hate s a consta such that (4) $$ |\operatorname{Ax}|\leq\gamma\qquad(x\in E,\operatorname{A}\in K). $$ (5) Now E2 and(G) show tha $\gamma^{-1}x\in\overline{{{V}}}\subseteq\,U$ for all $x\in E.$ Since ${\mathbf{}}V$ is balance // Thus Eisorignly bounded $$ E\subset t\,\overline{{{V}}}\subset t\,U\qquad(t>\,{\gamma}). $$ Corollay Xisa wome sace $E\subset X,$ and if (6) $$ \operatorname*{sup}_{x arrow E}|\Lambda x|<\infty\qquad(\Lambda\in X^{*}) $$ then there exists $\gamma<\infty$ such that (T) $(7)$ says that ${\boldsymbol{E}}$ is orignalybounded. $\boldsymbol{E}$ is weakly bounded. ${\it j}/j$ and $$ \|x\|\leq\gamma\qquad(x\in E). $$ rmoxomespes aeoly onvx:(O asha the roof o th Kren-Miman theoce oe ogoesoo no srnaotoeoe a i ieueu70 GENERAL THEORY 3.19 points.Suppose $\scriptstyle A$ and Theorem Suppose Xis a tological vector space on which $X^{\wedge}$ separates $\boldsymbol{B}$ are disoint, nonempty, compact, comvex sets in $X.$ Then there exists $\Lambda\in X^{*}$ such that (1) $$ \operatorname*{sup}_{x\in A}\operatorname{Re}\Lambda x<\operatorname*{inf}_{y\in B}\operatorname{Re}\Lambda y. $$ convexity of $\textstyle X$ Note that part of the hypothesis is weaker than in $\mathbf{\nabla}(b)$ of Theorem 3.4 (Gsince local implies that $X^{\mathfrak{X}}$ separates points on X);to make up for this, t isnow assumed that both $\textstyle A$ and $\boldsymbol{B}$ are compact. $X{\dot{\boldsymbol{r}}}:$ PR00F Let $X_{\w}$ he $\textstyle X$ with its weak topology. The sets (because $X_{\ w}$ and A $\boldsymbol{B}$ B are evidently $\textstyle A$ Since $X_{\it w}$ it gives us a $\Lambda\in(X_{w})^{\star}$ They are also closed in $X_{\mathrm{w}}$ Yis a Hausdorff spacc). in place of compact in $\textstyle X_{\ w}$ of Theorem 3.4 can be applied to $X_{\mathrm{w}}$ is locally convex, $\mathbf{\nabla}(b)$ that satisfies (1). But we saw in Section 3.11 as a consequence of Theorem 3.10) that $(X_{w})^{*}=X^{*}.$ $I/f$ $S\subset{\mathcal{K}}$ whose end points are in $\textstyle K$ but not in $\mathbf{S}.$ be a subset of a vector space $X.$ A nonempty set 3.20 Extreme points Let $\textstyle K$ if no point of Sis an internal point of a line interval is called an extreme set of $K$ Analytcally the condition can be expresse as follows: If $x\in K,\;y\in K,\;0<t<1,$ , and then $x\in S$ and The extreme poins of $\textstyle K$ $$ t x+(1-t)y\in S, $$ $\nu\in S$ are the extreme sets that consist of just one point contains Recall that the convex hull of a set $E\subset X$ is the smallest convex set in $\textstyle X$ Y that $\textstyle{X}$ ${\boldsymbol{E}}$ and that the closed convex hull o $\boldsymbol{E}$ is the closureof its convex hul At presen itis not known wheter it strue in every topological vctor space that every compact convex set has an extreme point. In a largeclass of spaces, the supply of extreme points is actually very abundant. Ths is shown by Theorems 3.21 and 3.22 which $X^{\star}$ The Krein-Milman theorem Suppose $\textstyle{\bar{X}}$ Y is a lopological vector space on 3.21 separates points. I ${\cal K}\,$ is acompact conex set in x, hen K is the cosed conpex hull of the set of its extreme points PRoor Let JP be the collecton of all compact extreme sets of $K.$ Sincc ${\boldsymbol{K}}$ Ke JP ${\mathcal{P}}\neq{\mathcal{D}}$ We shall use the following two properties of ${\mathcal{P}};$ (a) The intersection S of any nonempty subcollection of ${\mathcal{P}}$ is a member o ${\mathcal{O}}_{}\ {\mathcal{O}}_{}.$ unless $S={\mathcal{D}}.$ is the maximum of ReA on $\mathbf{S},$ (b)If S∈ 9P,AeX*, $\boldsymbol{\mu}$ S, and $S_{\mathrm{A}}=\{x\in S\colon\mathbf{Re}\wedge x=\mu\},$ then S,∈gcoNVEXIrY 71 ${\boldsymbol{X}}$ contradiction The proof of Since ${\cal S}\in{\mathcal{O}}^{\prime},{\mathcal{O}}^{\prime}$ is not empty. Parilly order and ${\mathbf{}}S\in{\mathcal{P}}.$ we have $x\in{\mathcal{S}}$ and every 3.9 there s a $x\in K,\ y\in K,\ 0<t<1.$ Since is immdiate. Toprove b), supos $\Lambda z=\mu$ and $\Lambda$ is linear, we $\Omega$ every If ${\boldsymbol{H}}$ defned as in(b). then $\mathbf{\Psi}(a)$ $\textstyle|\exp\leq\mu}$ $\Omega$ is collectinof compact setswith the $\mathcal{M}$ be the inter- 、is let $\Omega$ $S\in{\mathcal{P}},$ implies that no proper subset of $\mathcal{M}$ Hence $x\in S_{\mathfrak{A}}$ and be the colcto ail mbers o ${\mathcal{P}}$ $\iota x+(1-t)y=$ $K_{\lambda}$ $z\in S_{\mathrm{A}}\,,$ Hence Re Ax≤山R $\mathbf{S}\in{\mathcal{P}}$ Let ${\mathcal{P}}^{\prime}$ $z\in S$ $y\in S_{\mathrm{A}}$ that are $y\in S.$ $ \{\mathbf{e}\,\Lambda x=\mu=\operatorname{Re}\,\Lambda y$ $M.$ Since $X^{\mathfrak{X}}$ separates points on $X,$ M has only onc conclude: R Since Re subsets of Choose some belongs to $K.$ $\displaystyle K$ has not been used. by set inclusion, $\mathbf{S},$ is constant on is not empty Since $\overline{{H}}$ does not intcrscc for every $x\in{\bar{H}}.$ If $\Lambda\in X^{\kappa}$ be a maximaltal oredsbocitino ${\mathcal{P}}^{\prime}$ fitintcsctnpoperty . By o ${\mathcal{O}}.$ ${\mathcal{O}}^{\prime},$ and let sectio almbers of .Sinc $M\in{\mathcal{O}}^{\prime}.$ The maximality of lt now follow from b) that that point Therefore M san extemepoint such that Re A $x<\mathrm{Re}\land x_{0}$ ${\bar{H}}\subset K.$ Hence $\overline{{H}}$ is compac. $\Lambda\in X^{*}$ MWe have now provd hat evey comac cxeme e o $x_{0}\in K$ is not in ${\overrightarrow{H}}.$ By Theorem $H\cap S$ is th cox ul o o xtemoints o $K_{\mathrm{\scriptscriptstyleI}}$ $K$ contains an extreme poim (K(Sofar h convxty on K,it follows, for Since K is compact and convex we have Assme to et cotaracom tat som $K_{\mathrm{A}}\in{\mathcal{P}}$ $K_{\Lambda}$ we have our If X Remark The convexity or $\textstyle K$ xvereasmed to lcll covx the compacns $\overline{{{\cal H}}}$ / was used only to show that $\overline{{H}}$ is compact. Krein-Mimantherem s thusotain $K\subset{\overline{{H}}}.$ would not be need, sinc onc could use $\mathbf{\nabla}(b)$ of Theorem 3.4 n place of Theorem 3.19 The above argument then proves ta The followng version of th 3.22 Theorem ${\cal J}_{\mathrm{\footnotesize{0}}}$ Xisa loalycorex space and ; ${\boldsymbol{E}}$ As hese o extre poins o/ compac e $\textstyle K$ ” ${\mathcal{X}},$ ihen $\textstyle K$ lies e ci ome / hull ${\boldsymbol{H}}$ Equivalenty, ${\boldsymbol{E}}$ and $\textstyle{K}$ have thc same losed convex hu need not beclosed, and there $\boldsymbol{E}$ lesscatstasubc ${\boldsymbol{E}}$ The peos rmeraesstwiasa e s abo e onv $\textstyle H$ $*>0.$ of a compact set $K\,{\mathcal{K}}\,{\mathcal{Y}}$ Even in a Hilbert space, are situations in which $\overline{{H}}$ is not oat xecis 202. I Fecesisisth and totally bounded.(Appendix A45 lougphso so osc nms25 easfit sw iean tho gousstoempeseseioms r nai m s eso ora mticspe sai aly unae i onoin h un fiey manyoSnas aisxze v Te smecea en noaevosic ereras${\mathcal{T}}{\mathcal{Z}}$ GENERAL THEORY 3.23 bounded if to every neighborhood ${\mathbf{}}V$ of ${\boldsymbol{0}}$ in in a topological vector space $\textstyle X$ is said to be totall such that Definition A set ${\boldsymbol{E}}$ corresponds a finite se ${\boldsymbol{F}}\subset X$ $X$ $E\subset F+V.$ that are compatible with thc topology of $X.$ lf X happens to be mtrizable tologcal vecor spac,then these two notin of total boundeness concide provided that werestritouselves to inariant metric GThe prof f hs s n Thcorem 1.26.) 3.24 Theorem If X is a locally convex space and ${\boldsymbol{H}}$ is the convex hull of a totall bounded set $F\subset X.$ then ${\boldsymbol{H}}$ is totally bounded. of of all 1f $e_{1},\ \cdot\cdot\cdot,\ e_{m}$ are the points of $E_{1}$ ≥0 and $\textstyle\sum t_{i}=1,$ ,then There is convex neighborhood ${\mathbf{}}V$ ${\boldsymbol{0}}$ $E\subset E_{1}+V.$ Let $\textstyle H_{1}$ be a neighborhood of $\mathbf{0}$ in $X,$ $E_{1}$ $\boldsymbol{S}$ is the simplex in $R^{m}$ consisting PR00F Let $U$ $V+\,V\subset U,$ and there is a finite set $E_{1}\subset X$ such that in $\textstyle X$ such that be the convex hull of and il $t\equiv\left(t_{1},\,\cdot\cdot\cdot,\,t_{m}\right)$ that satisfy ${\mathit{t}}_{i}$ $$ (t_{1},\ldots,t_{m})\to\sum t_{i}e_{i} $$ If maps the compact set $\boldsymbol{\mathsf{S}}$ continuously onto $T_{1}.$ Hence $\textstyle H_{1}$ is compact. $x\in H,$ then $x=\alpha_{1}x_{1}+\cdots\cdot+\alpha_{n}x_{n}$ where $x_{i}\in E,\;\alpha_{i}\geq0,\;\;\Sigma x_{i}=1.$ There are points $y_{i}\in E_{1}$ such tha $x_{i}-y_{i}\in V;$ this follows from thc choice of $E_{1}.$ Decompose $\scriptstyle{\mathcal{X}}$ into the sum $x=x^{\prime}+x^{\prime},$ where $x^{\prime}=\sum^{\alpha_{i}}y_{i}$ and $x^{\prime\prime}=\sum\!\mathcal{Q}_{i}(X_{i}-y_{i}).$ The convexity of ${\mathbf{}}V$ implies that $x^{\prime\prime}\in V$ It is clear that $x^{\prime}\in H_{1}$ Hence $$ I I\k<H_{1}+V. $$ Since $\textstyle H_{1}$ is compact, there is a finite set ${\mathbf{}}F$ such that $H_{1}\subset F+V.$ Thus $$ H\subset F+V+V\subset F+U. $$ Since $U$ was arbitrary, it follows that $\bar{H}$ is tally bounded. $J/i J$ 3.25 Theorem Suppose H is the convex hull of a compact set $\textstyle K$ inc $\bar{a}$ topological vector space $X.$ (a)If Xis a Frechet space, then $\overline{{H}}$ is compact (b)1f $X=R^{n}.$ , then $\textstyle H$ is compact. PROOF (a) By Theorem 3.24, $\textstyle H$ is tally bounded. Since Fréchet spaces are complete metric spaces, the closure $\bar{H}$ of H is compact.coNVExIrY 73 (b) Let $\boldsymbol{S}$ be the simplex in $R^{n+1}$ consisting of all $t=(t_{1},\ \cdot\cdot\cdot,\,t_{n+1})$ $\scriptstyle{v i c h\;}t_{i}\geq0$ and $\textstyle\sum t_{i}=1.$ By the lem thatfo $x\in H$ if and only if $$ x=\sum_{i=1}^{n+1}t_{i}x_{i} $$ for some $\scriptstyle t\in{\mathcal{S}}$ and $x_{i}\in K\ (1\leq i\leq n+1)$ In other words, $\textstyle H$ is the image of S ×K ×…×K $(K$ occurs $\;n$ + times undr th couius mappin Hence ${\mathcal{H}}$ is compact $$ (t,\,x_{1},\,\ldots,\,x_{n+1})\to\sum_{i=1}^{n+1}t_{i}x_{i}\,. $$ // Lemma I xlies he come hullof ve $E\subset{\boldsymbol{R}}^{n},$ then $\textstyle{\mathcal{X}}$ lies im the comvex hul ome ube o h conasatmo $n+1$ poinis. PROor lt isenough to show that if r>nand ; he $\scriptstyle{\mathcal{X}}$ is acaya onvex omimaiaomonsm · of some $\scriptstyle r+1$ vectors $x_{i}\in E.$ $x=\sum t_{i}x_{i}$ is a convex combinatio r of these vectors r vectors Assue, wtout loss o erality ta $a_{i}\,,$ not all $0_{!}$ such that $\scriptstyle t_{\mathrm{i}}\geq0$ for $1\leq i\leq r+1.$ The $x_{i}-x_{r+1}\ (1\leq i\leq r)$ are linarly dependent, sinc $r\geq n.$ It follow that there are real numbers Choos ${\mathfrak{p}}{\mathfrak{n}}$ so that $\left|\left.a_{i}/t_{i}\right|\ \leq | .a_{m}/t_{m}\right|,$ $$ \begin{array}{c c}{{\sum_{i\equiv1}^{r+1}a_{i}\,x_{i}=0~~~~~~\mathrm{and}~~~~~~r{\underline{{{\Lambda}}}}^{~+1}a_{i}=0.}}\end{array} $$ for l≤i≤r+1, and define $$ c_{i}\longrightarrow t_{i}-\frac{a_{i}\,t_{m}}{a_{m}}\qquad(1\leq i\leq r+1). $$ : Then $c_{i}\geq0,$ $\mathbf{\sum}\cdot\mathbf{\hat{z}}_{i}=\sum t_{i}=1,\,\mathbf{\varepsilon}=\sum c_{i}\,x_{i}\,,\,\mathbf{\dot{z}}$ and $c_{m}=0.$ // Vector-valued Integration 梦 in $\textstyle{X}$ topological vector space $X.$ somisisel h e inunea ens a a enonom $\mu\rangle$ and whose values ie in some measure space ${\cal Q}\,$ (with a real or complex measure that deserves to becalle Th fist roblem scatw tea veco 0 fdn74 GENERAL THrOR i.e., which has at least some of the properties that integrals usually have. For instance the equation $$ \Lambda{\Big(}\int_{Q}f\,d\mu{\Big)}=\int_{Q}(\Lambda f)\,d\mu $$ ought to hold for every $\Lambda\in X^{*}$ because it does hold for sums, and because integrals are(or ought to be limits of sums in some sense or oher. In fact,our definion wi be based on this single requiremcnt. Many other approaches to vector-valued integration have been studied in great detail; in some of these, the integrals are defined more directly as limits of sums (se Exercise 23). $\mathrm{\A}/i(1+x+1+1+1+1+1+1+1+\frac{1}{3}+\frac{1}{2}+\frac{1}{2}+\frac{1}{2}+\frac{1}{2}+\frac{1}{2}+\frac{1}{2}+\frac{1}{2}/3}+\frac{1}{2}/3-1\cal{\Lambda}$ 3.26 Definition Suppose $X^{\geq{\mathfrak{k}}}$ separates points, and fis a function from for every $\Lambda\in X^{\star};$ note that $\boldsymbol{\mu}$ is a measure on a measure space $\textstyle{\mathcal{Q}}$ $\textstyle X$ is a topological vector space on which ${\mathcal Q}\,$ D into $\textstyle X$ such that the scalar functions $\Lambda f$ are integrable with respect to ${\boldsymbol{\mu}},$ ris defined by (1) $$ (\operatorname{A}\!f)(q)=\Lambda(f(q))\qquad(q\in Q). $$ If there exists a vector $y\in X$ such that (2) for every $\mathrm{A}\in X^{\bullet},$ then we define $$ \Lambda y=\int_{Q}(\Lambda f)\,d\mu $$ (3) $$ \dagger_{_Q}f d\mu=y. $$ Remarks It is clear that there is at most one such ${\boldsymbol{y}},$ because $\textstyle X^{\geq\varepsilon}$ separates points on $X.$ .Thus there is no uniqueness problem. Existence will be proved only in the rather special case (sufficient for many space ${\cal Q}\,$ applications) in whichC ${\mathcal Q}\,$ is compact and fis continuous. In that case ${\mathcal{I}}(\mathcal{O})$ is compact $f({\mathcal{O}})$ mcasure of total mass 1 and the only other requirement that willbe imposed is that the closed convex hull o .A probability measure is a positive should be compact. By Theorem 3.25, his diinal requirement is automatiely satisfied when $\textstyle X$ is a Fréchet space Recall that a Borel measure on a compact (or locally compact) Hausdoff is a measure defined on te -algebra o all Boe sts in Q;this is th snales o-algebra that contains all open subsets of 《 ${\cal Q},$ 3.27 Theorem Suppose $(a)\ \ X$ is a topological vector space on which $X^{\mathfrak{s t}}$ separates points ad ${\mathcal{Q}}.$ (份 isa Borel probability measure on acompact Hawsdorf spaccONVEXIr 75 F in If f: $Q\to X$ is cmuwu,an he comer hmn $\textstyle H$ o0/ /(Q) has compacr closure $X,$ then the integra () Moreover eA $$ y=\int_{Q}f d\mu $$ exists. te sense f Defiio 3.26 $X$ RemakLI nynosiveBore masure ${\mathcal{Q}},$ then some scalar multip orenloryuyss .easil e is C) to complex ones mesowi pa nmemeisosesna sn aecosonacoo oistsmsia er ea Exercie givsante eaiaio $y\in H$ PRoor Regard $\textstyle{X}$ us aravto sac W hve opro that exi such that (2) $e v e r y\land\in X^{*},$ $$ \Lambda y=\int_{Q}(\Lambda f)\,d\mu $$ for of $\Lambda\rfloor$ Let $L=\{\Lambda_{1},\ldots,\Lambda_{n}\}$ be a finite subset o Each $E_{L}$ is closed by the continut $E_{L}$ $y\in{\bar{H}}$ lection of all and is thereforecoma sic $\widehat{\cal H}$ $X^{\star}\!\cdot$ Let $E_{L}$ be the set of al $\scriptstyle{K^{*}}$ thatsatisy 2)"oTr evey ${\mathrm{A}}\in L$ $L(f(Q))$ is ther oemty an an $\mathbf{}y$ is compact. I no $\textstyle{\bar{\cal X}}$ into $\textstyle R^{n}\!\!$ and pu fore enough to prove $E_{L}\neq{\mathcal{D}}$ hatsoniesotooiy"Tn $E_{L}$ is empty, the col $E_{L}$ in iststis 2 for ever Ascoin $\Lambda\in X^{\kappa}$ ltis there- Regard ${\cal L}=\left(\Lambda_{1},\ \cdot\cdot,\ \Lambda_{n}\right)$ as a mapping from Define (3) $$ m_{i}=\int_{\cal Q}(\Lambda_{i}f)\,d\mu\qquad(1\leq i\leq n). $$ If We claim thathe poin $m=(m_{1},\,\ldots,\,m_{n})$ lies in the convex hu ${\mathfrak{r}}\,K$ thcre are real numbers $\ t=(t_{1},\ \dots,\ t_{n})\in R^{n}$ is o i ul tm D Tnem izs an $\textstyle R^{n}$ $c_{1},\ \cdot\cdot\ ,\ C_{n}$ Aogoro…a atwmnis sanson such that if $u=(u_{1},\ \ldots,u_{n})\in K.$ Hence $$ \sum_{i=1}^{n}c_{i}u_{i}<\sum_{i=1}^{n}c_{i}t_{i} $$ (4) (5) $$ \sum_{i=1}^{n}c_{i}\wedge_{i}f(q)<\sum_{i=1}^{n}c_{i}t_{i}\qquad(q\in Q). $$76 cENERAL THEORY Since $\boldsymbol{\mu}$ is a probabilitymeasure, integraton of the left side of() give $\textstyle\sum c_{i}m_{i}<\sum c_{i}t_{i}$ Thus $t\uparrow\uparrow\uparrow\uparrow m.$ $K.$ Since $K=L\left(f(Q)\right)$ and $\underline{{L}}$ This shows that ${\mathfrak{p}}$ lies in the convex hull of in the convex hull ${\boldsymbol{H}}$ of f(O). For is lincar, it follows that $m=L\y$ for some J $\mathbf{\nabla}y$ this y we have (6) $$ \Lambda_{i}y=m_{i}=\int_{Q}(\Lambda_{i}f)\;d\mu\qquad(1\leq i\leq n). $$ Hence $y\in E_{L}$ This completes the proof. // 3.28 Theorem Suppose (a) Xis a topological vector space on which $X^{\geq}$ separates points (6) Q is a compact subset of $X,$ and (c) the closed convex hull $H$ l of ${\cal Q}\,$ Dis compact. Then $y\in{\tilde{H}}$ if and only $i f$ there is a regular Borel probabiliy measure p on ${\cal Q}\,$ such that (1) $$ y=\int_{A}x\,d\mu(x). $$ Remarks The intega s to be nderstoo ${\boldsymbol{\alpha}}\mathbf{S}$ in Definition 326, with f(x)-× Recall that a positive Borel measure on ${\cal Q}\,$ 厂 is said to be regular i (2) $$ u(E)=\operatorname*{sup}\left\{\mu(K);K\subset E\right\}=\operatorname*{inf}\left\{\mu(G);\,E\subset G\right\} $$ for every Borel set $E\subset{\mathcal{Q}}.$ , where $\textstyle K$ K ranges over the compact subsets of ${\boldsymbol{E}}$ and ${\mathfrak{C}}$ ranges over the open supersets of $L.$ as a“weighted average”of $Q.$ The integral (1) represents every $y\in H$ or as the “center of mass"”of a certain unif mass distibted over ${\boldsymbol{Q}}.$ We stress once more that $\left(c\right)$ follows from (b) if $\textstyle X$ is a Frechet space PRoOF real Borel measures on ${\mathcal{Q}}$ Y again as a real vector space. Let $c(\sigma)$ be the Banach space Regard $X$ with the supremum norm. The Riesz of all real continuous functions on ${\underline{{Q}}},$ $C(Q)^{\sharp}$ with the space of al representation theorem identifes the dual space that ar dfences o reguar positive ones. With thi identification in mind, we define a mapping (3) $$ \phi\colon C(Q)^{*}\to X $$ by $(\mathbf{d})$ $$ \phi(\mu)=\int_{\cal Q}x\,d\mu(x). $$coNvExrr $77$ that $\phi(\partial_{x})=x$ , we see that $Q\subset\phi(P).$ Since $\phi~~~~~~~~~~~~~~~~~~~~~~~~~~~~~~~~~~~~~~~~~~~~~~~~~~~~~~~~~~~~~~~~~~~~~~~~~~~~~~~~~~~~~~~~~~~~~~~~~~~~~~~~~~~~~~~~~~~~~~~~~~~~~~~~~~~~~~~~~~~~~~~~~~~~~~~~~~~~~~~~~~~~~~~~~~~~~~~~~~~~~~~~~~~~~~~~~~~~~~~~~~~~~~~~~~~~~~$ eoe spoglfuls n lo simue om m ${\cal Q}.$ .Ha $\phi(P)$ is closed in $X.$ Since (ii) Let $D\!\!\!\!/$ $\phi(p)=H$ $C(Q)^{*}.$ concentrated a $\textstyle{\mathcal{X}}$ belongs to $D.$ theorem asserts tha () For each $x\in Q.$ $\phi~~~~~~~~~~~~~~~~~~~~~~~~~~~~~~~~~~~~~~~~~~~~~~~~~~~~~~~~~~~~~~~~~~~~~~~~~~~~~~~~~~~~~~~~~~~~~~~~~~~~~~~~~~~~~~~~~~~~~~~~~~~~~~~~~~~~~~~~~~~~~~~~~~~~~~~~~~~~~~~~~~~~~~~~~~~~~~~~~~~~~~~~~~~~~~~~~~~~~~~~~~~~~~~~$ definel bJ(4)is comimious $\delta_{x}\,$ is inear and $\boldsymbol{\mathit{P}}$ is convx, i folow the unit mass $H\subset\phi(P)$ , where ${\mathbf{}}H$ is the convx hull o By Theorem 3.27元 $\bar{\cal L}S$ given its weak*- Once we have $\left(i\right)$ Treroi i sasos soso s $c(g)^{*}$ $\phi(P)\subset H$ P isweak-ompuet This acosec o ioisinw ia The mapping noloy amd X soni weak mm To prove $(i),$ and üi), t follos ta $\phi(P)$ mes seses desied conclusion Siesos s saess is weakly compact, hence note that (5) I1 $\operatorname{I}h\in C(O)$ and $h\geq0_{s}$ , put P s weak*-closed lt steonoug soiwh $\boldsymbol{P}$ $$ P\subset\left\{\mu\colon_{Q}\right\}_{d}\;d\mu\left|\leq1\;{\mathrm{if~}}\|h\|<1\right> $$ eoe esos -s t y o nheAo neor (6) $$ E_{h}=\left\{\mu\colon_{Q}h\,d\mu\geq0\right\}. $$ $E_{h}$ Since $h\to\textstyle\int h\,d\mu$ a soe u anono ne s on ae is weak-closed So s the s where $\Lambda_{i}\in X^{*}$ is a weak-neighborhood o o in $C(Q)^{\bullet}.$ $$ E=\left\{\mu\colon_{Q}|\ d\mu=1\right\}. $$ is wcak*-closed $C({\mathcal{O}}).$ Hence (7) (8) Since $\boldsymbol{\mathit{P}}$ is the intersection or $\boldsymbol{E}$ and the sets $E_{h},\,P$ is coninus at he origi since $\phi$ $\mathbf{T_{O}}$ prove i t is nogho pove h $\phi~~~~~~~~~~~~~~~~~~~~~~~~~~~~~~~~~~~~~~~~~~~~~~~~~~~~~~~~~~~~~~~~~~~~~~~~~~~~~~~~~~~~~~~~~~~~~~~~~~~~~~~~~~~~~~~~~~~~~~~~~~~~~~~~~~~~~~~~~~~~~~~~~~~~~~~~~~~~~~~~~~~~~~~~~~~~~~~~~~~~~~~~~~~~~~~~~~~~~~~~~~~~~~~~~~~~~~~~~~~~~~~~~~~~~~~~~~~~~~~$ is n vy waisoso Sx mastine $$ {\mathcal W}=\{y\in X_{\varsigma}\ |\Lambda_{i}y|\,<r_{i}\quad\mathrm{for}\quad1\leq i\leq n\}, $$ and $r_{i}\gg0$ The rtion or h $\mathrm{\A}_{i}$ to ${\boldsymbol{Q}}$ lie in (9) $$ V=\left\{\mu\in C(Q)^{*}:\left|\int_{Q}\Lambda_{i}\,d\mu\right|_{i}<r_{i}\mathrm{\boldmath~\for~\quad~l\leqi\leq~}n\right\} $$ But (10) $$ \int_{Q}^{}\Lambda_{i}\,d\mu=\Lambda_{i}{\biggl(}{\biggl)}_{Q}^{}\,x\,d\mu(x){\biggr)}=\Lambda_{i}\phi(\mu), $$78 GENERAL TuEORY by Definition 3.26. It follows from (8),(9), and(10) that $\phi(V)\subset W.$ Hence $\phi~~~~~~~~~~~~~~\phi~~~~~~~~~~~~~~~~~~~~~~~~~~~~~~~~~~~~~~~~~~~~~~~~~~~~~~~~~~~~~~~~~~~~~~~~~~~~~~~~~~~~~~~~~~~~~~~~~~~~~~~~~~~~~~~~~~~~~~~~~~~~~~~~~~~~~~~~~~~~~~~~~~~~~~~~~~~~~~~~~~~~~~~~~~~~~~~~~~~~~~~~~~~~~~~~~~~~~~~~~~~~~~~~~~~~~~~~~~~~~~~{{~~~~~~~~~~~~~~~~~~~~~~~~~~~~~~~~~~~~~~~~~~~~~~~~~~~~~~~~~~~~~~~~~~~~~~~~~~~~~~~~~~~~~~~~~~~~~~~~~~~~~~~~~~~~~~~~~~~~~~~~~~~~~~~~~~~~~~~~~~~~~~~~~~~~~~~~~~~~~~~~~~~~~~~~~~~~~~~~~~~~~~~$ is continuous. // The following simple inequality sharpens the last assertion in the statemcnt of Theorem 3.27 3.29 Theorem Suppose Q is a compact Hausdorff space, $\textstyle X$ is a Banach space, $f\colon Q\to X$ is continuous, and p $\boldsymbol{\mu}$ u is a positive Borel measure on ${\cal Q}.$ Then $$ \left\|\int_{Q}f\,d\mu\right\| \|\leq\int_{Q}\|f\|\,d\mu. $$ such that $\mathrm{A}y=|y||$ and $|\land x|\leq\|x\|$ By the corollary to Theorem 3.3, thcrc In particular, $\mathbf{i}\mathbf{S}$ $\mathrm{A}\in X^{\bullet}$ PROOF. Put $y=\textstyle5/d\mu.$ for al $x\in X.$ $$ |\Lambda f(s)|\leq\|f(s)\| $$ for all se Q.By Theorem ${\mathfrak{d}}{\mathfrak{s}}{\mathfrak{s}}$ it fllows that $$ \left\|y\right\|=\Lambda y=\int_{Q}(\Lambda f)\,d\mu\leq\int_{Q}\left\|J\right\|\,d\mu. $$ // Holomorphic Functions In the study of Banach algebras, as well as in some other contexts, it is useful to enlarge the concept of holomorphic function from complex-valued ones to vcctor- valued ones.(Or course, one can also generalize the domains, by going from $\textstyle{\mathcal{C}}$ !'to ${\mathcal{C}}^{n}$ and even beyond. But this is another story.)There are at least two very natural definitions of “ holomorphic”avaiable in this genral seting, a“weak” onc and a “ strong"” onc. They turn out to define the same cass of functions if the values are assumed to lie in a Fréchet space. 3.30 Definition Let C $\Omega$ be an open set in ${\mathcal{C}}$ and et $\textstyle X$ be a complex topological vector space (a) A function $f\colon\Omega arrow X$ is said to be weakly olomorphic in $\Omega$ if A is holomorphic in the ordinary sense for every $\Lambda\in X^{*}.$ (b) A function $f\colon\Omega\to X$ is said to be strongly holomorphic in $\mathbb{Q}$ h if $$ \operatorname*{lim}_{w\to z}{\frac{f(w)-f(z)}{w-z}} $$ exists Gin the topology of X) for every $z\in\Omega.$ vector $f(w)-f(z)$ i X Note that the above quotient is the product of the scalar $(w-z)^{-1}$ and thecoNvExrr 79 The index of a poin $z\in C$ TtsqouosostsnaA naou n im as on ua eo $\textstyle X$ sto gngogos "ommea seso isueswe ae rsosos mo…es miswsiesosionei Hiomsuyoo osermy SCc cseriwnfiny impotnt o i irot asviTems it scet cose pat ha dos ot ps throuh wi b ented In,(eS Weeini th $$ .\quad\operatorname{Ind}_{\Gamma}{\bigl(}_{Z}{\bigr)}={\frac{1}{2\pi i}}\int_{\Gamma}{\frac{d\zeta}{\zeta-z}}. $$ 3.31 Theorem Let 4 ${\boldsymbol{\otimes}}$ be open in ${\underline{{C}}},$ let $\textstyle X$ Y be $\textstyle{\mathcal{A}}$ comlex Freche spccad assm that $$ \mathrm{\boldmath~\Gamma~}\cdot\mathrm{\boldmath~\Gamma~}\quad\mathrm{\boldmath~\nabla~}\bar{\cal~}\bar{\cal~}\bar{\cal~}\bar{\cal~}\bar{\cal~}\bar{\cal~}\bar{\cal~}\bar{\cal~}\bar{\cal~}\bar{\cal~}\bar{\cal~}\bar{\cal~}\bar{\cal~}\bar{\cal~}\bar{\cal~}\bar{\cal~}\bar{\cal~}\bar{\cal~}\bar{\cal~}\bar{\cal~}\bar{\cal~}\bar{\cal~}\bar{\cal~}\bar{\cal~}\bar{\cal~}\bar{\cal~}\bar{\cal~}\bar{\cal~}\bar{\cal~}\bar{\cal~}\quad\bar{\cal~}\quad\bar{\cal~~\bar{\cal~}\bar{\cal~}\bar{\cal~}\bar{\cal~}\bar{\cal~}\bar{\cal~}\bar{\cal~}\bar{\cal~}\bar{\cal~}\bar{\cal{\cal{\cal{\cal~}}}\cal{\cal{\cal~}}}\bar{\cal{\cal{\cal{\cal{\cal{\cal~}}}}\cal{\cal{\cal{\cal{\cal{\cal{\cal{\cal{\cal{\cal{\cal{\cal{\cal{\cal{\cal{\cal{\cal{\cal{\cal~}}}}}}}}}}}}}}}\cal{\cal{\cal{\cal{\cal{\cal{\cal{\cal{\cal{\cal{\cal{\cal{\cal{\cal{\cal{\cal{\cal{\cal{\cal{\cal{ $$ is weky oie- T Jloiny cusios ol fis stoly cotiuous im Jfor every w≠ S, then .suchar Ind $(\nu)=0$ 6Ozyco/: ncoeluanh m (1) $$ \textstyle\int_{\Gamma}f(\zeta)\,d\zeta=0, $$ and (2) $$ f(z)={\frac{1}{2\pi i}}\int_{\Gamma}(\zeta-z)^{-1}f(\zeta)\;d\zeta $$ $f z\in\Omega$ und Ind $(z)=1.$ ${\mathfrak{g}}\cap_{1}$ and $\Gamma_{\mathit{2}}$ are closed paths in $\mathbb{Q}$ such thal for ewery $w\notin\Omega,$ then $$ {\mathrm{Ind}}_{{\Gamma}_{1}}(w)=\mathrm{Ind}_{{\Gamma}_{2}}(w) $$ (3) $$ \int_{\mathbf{F}_{1}}f(\zeta)\,d\zeta=\int_{\mathbf{F}_{2}}f(\zeta)\,d\zeta. $$ (c)f is strongly holomorphic in Q mecoamcsuc $\boldsymbol{\Gamma}$ Tteupsoe ueses nso nrm a32 Eine Gacomastueo ${\cal C}),$ 呵u one can regard ${\boldsymbol{R}}$ Aas mepex su n angeo $\bar{\boldsymbol{I}}$ euemesesa intrval in $d{\boldsymbol{\zeta}}$ and imnewi wseseis PRoor (a) Assume $0\in\Omega.$ we shal rovetat stony conuous $0.$ Define (4) $\Delta_{r}=\{z\in C\colon|z|\leq r\}.$80 crNuxAL Tmuox $$ \begin{array}{c c c c c c}{{}}&{{}}&{{}}&{{}}&{{}}&{{}}\\ {{}}&{{}}&{{}}&{{}}&{{}}\end{array} $$ Then $\Delta_{2r}\subset\Omega$ for some $\scriptstyle r\gg0$ Let $\Gamma$ be the positively oriented boundary of $\Delta_{2r}$ ${\mathrm{Fix}}\ \Lambda\in X^{\kappa}.$ Since $\Lambda f!$ is holomorphic, (5) $$ {\frac{(\Lambda/f)(z)-(\Lambda/f)(0)}{z}}={\frac{1}{2\pi i}}\int_{\Gamma}{\frac{(\Lambda f)(\zeta)}{(\zeta-z)\zeta}}\,d\zeta $$ if $0<|z|<2r$ Let $M(\Lambda)$ be the maximum of $|\Lambda f|$ on $\Delta_{2},$ .If $0<|z|\leq r,$ it follows that (6) $$ \left|z^{-1}\Lambda[f(z)-f(0)]\right|\ \leq r^{-1}M(\Lambda). $$ The set of all quotients (7) $$ \langle{\frac{f(z)-f(0)}{z}}:0<|z|\leq r \rangle $$ is therefore weakly bounded in $X.$ By Theorem 3.18, thset is also strongly there exists $\scriptstyle t\,<\,\omega$ bounded. Thus if ${\mathbf{}}V$ 'is any (strong) neighborhood of O in $X,$ such that (8) $$ f(z)-f(0)\in z t V\qquad(0<|z|\leq r). $$ Consequently, f(2)→/f0) strongly, as $\scriptstyle z\div\circ$ This was the crux of the matter. The rest is now almost automatic if f is replaced in them by Af, where $\Lambda$ (6b).By (a) and Theorem 3.27, the integrals in(1)to (3))exist. Thes .The formulas are three formulas are correct by the theory of ordinary holomorphic functions) is any member of $\textstyle X^{\ast}$ therefore correct as stated, by Definition 3.26 Define (e)Assume, as inthe proof of Ga), that $\Delta_{2r}\subset\Omega,$ and choosc $\mathbf{\hat{I}}$ "as in $(a).$ (9) $$ y={\frac{1}{2\pi i}}\int_{\Gamma}\zeta^{-2}f(\zeta)\;d\zeta. $$ The Cauchy formula C) shows,after a small computation, tha (10) $$ \frac{f(z)-f(0)}{z}=y+z g(z) $$ .if o $~\backslash<|z|<2r,$ where (11) $$ g(z)=\frac{1}{2\pi}\int_{-\pi}^{\pi}\left[2r e^{i\theta}(2r e^{i\theta}-z)\right]^{-1}f(2r e^{i\theta})\,d\theta. $$ Let V be a, convex balanced ncighborhood of O in X for some $\scriptstyle t\,<\,\infty.\ \ \operatorname{If}\,s=$ $\scriptstyle{K_{\circ}}$ Put {f(6):1(| = 27/}. Then K is compact, so that $K\subset t V$cONVEXrr 81 Z → 0. $g(z)\in s{\mathcal{V}}\mathrm{if}\ |z|\leq r$ and rlS Aitflow tat negrand G1 ies ${\mathfrak{s}}V$ V for every 0. Thus Th e sie o O tererecestsiny ojis $J/{\cal J}$ Th foni soesn Lnulesetoremocn bomdae eui foiosos qte veno om .T c teuen eseaso spctra Bact algcbrasSeEeicse Chpte io separates poins. Suppose : $C\to X$ 3.3.home_2X 。.comle olilueor pae o whc $X^{\mathfrak{s g}}$ subset o/ X. Then fis constan iswek ohic ama/(C)isawek ome TP92g o eorx A X. oune opeavauoe enct lf z C iflows rom iovistemth $$ \Lambda f(z)=\Lambda f(0). $$ Since $X^{\star}$ separates points on $X,$ this implies $f(z)=f(0).$ , for every zeC. // bounded, n an F-spa $\textstyle X$ on whict $X^{\rtimes}$ Bar opL xes sieris eay ond s ih soin * separaespoints Compare wit Theorem 3.18 Exercises (if Call a set $H\subset R^{n}$ a hyperplane texstreal numbers a. a,,c(with $a_{i}\neq0$ for each at least one i ${\boldsymbol{E}}$ rovetatere is a yperplane ${\mathcal{R}}^{a},$ wvith nonempty interior, an $E_{\boldsymbol{B}}$ n are disjoint convex sets $\sum a_{i}x_{i}=c.$ $L^{2},$ $E_{x}$ (a) Suppose $\scriptstyle{\mathcal{A}}$ $\overline{{{l}}}$ such tha $H$ cons a ll point $x=\left(x_{1},\ast\cdot,\,x_{n}\right)$ that satis is a boundary $\Lambda(\beta)$ point of Suppose I $\boldsymbol{E}$ is a convex set in I such that $E_{x}$ a and $\mathbf{\vec{y}}$ lies entirely on rem point of $\textstyle E,$ let $\mathcal{M}$ be th oinisa aoma ${\boldsymbol{H}}$ $\nu\epsilon\,H\,$ and E ${\boldsymbol{E}}$ oi smeno Hint: What is $\Lambda(E_{z})\,?$ onsoqpSsssosianse ss Thus $f(0)=\alpha$ . Show that $\alpha\neq\beta)$ 。 s convx and that ach is dense in $L^{2}$ … and aiy ite on point of 2 Suppose $L^{2}=I^{2}([-1,$ $\mathbf{y}_{z}$ $3.2.$ $E_{x}$ 1D. se c t Lesemeasue For ac sca 。 le he s i inu tons Sh Y-S Jsn h whicno spatc y ay onosinat ineon $\mathbf{\hat{A}}$ Suppose $\textstyle X$ is eveto saevitou o allapint $\in{\mathcal{A}}\subset X$ an internal $A\ {\mathrm{if~}}A\ {\stackrel{...}{\to}}\,x_{0}$ is an absorbing set (6) Show (with and $\boldsymbol{B}$ P re disjoint onvex sets i $X,$ and $\scriptstyle{A}$ has an iternal point. Pove contains $X=R^{2},$ that therc is a nonconstant linear functiona $\Lambda$ $\textstyle X$ such that $\Lambda(A)\cap\Lambda(B)$ and at momt oe on. CThe rosimaf ta hem.4 $\Delta(A)$ disjoin, undcr the hyoteso o for example tat may not pssie ot hav${\mathsf{S}}2$ GENERAL THEORY 4 Let ${\mathcal{E}}^{\infty}$ be the space of all real bounded functions $\lambda_{1}$ on the positive integers. Let r be the translation operator defined on $\ell^{\alpha}$ by the equation $$ (\tau x)(n)=x(n+1)\qquad(n=1,\,2,\,3,\,\dots). $$ Prove that there exists a linear functional $\Lambda$ on $\ell^{\alpha}$ (called a Banach limit) such that (a) Arx $=\Lambda x,$ and $\mathbf{\nabla}(b)$ lin ${\underset{\mathrm{ninf}}{\operatorname{nup}}}\,x(n)\leq\Lambda\,x\leq\operatorname*{lim}_{n\,\operatorname{evp}}\,x(n)$ for every $x\in\ell^{\infty},$ Suggestion: Define and apply Theorem 3.2 $$ \begin{array}{c}{{\Lambda_{n}x=\frac{x(1)+\cdots+x(n)}{n}}}\\ {{M-\{x\in\xi^{\infty}:\operatorname*{lim}_{n\to\infty}\Lambda_{n}x=\Lambda x\exp\Lambda_{n}x=\Lambda x\exp\Lambda_{n}}}\\ {{n=\operatorname*{lim}_{n\to\infty}\omega}}\end{array} $$ For $0<p<\varpi,$ let ${\mathcal{F}}^{p}$ be the space of all functions x $\scriptstyle{\mathcal{X}}$ (real or complex, as the case may be) on the positive integers, such that $$ \sum_{n=1}^{\infty}|x(n)|^{p}<\infty. $$ For $1\leq p<\infty$ , dcfine $||x||_{p}=\{\sum|x(n)|^{p_{j}!/p}.$ , and dcfine $:\|x\|_{\infty}={\mathfrak{s u p}}_{n}~\|x(n)\|$ correspondence $\Lambda{\longleftrightarrow}y$ between $(\ell^{p})^{*}$ Prove that ILxll and Ilxllo make $\ell^{p}$ and $\ell^{\infty}$ into Banach spaces. a) Assume 1 $<p<\infty$ prove that $(\ell^{p})^{\star}=\ell^{q},$ in the following scnse: There is a one-to-one $\Gamma\mathbf{f}\,p^{-1}+q^{-1}=1,$ and ${\mathcal{E}}^{q},$ given by $$ \mathrm{A}x-\sum x(n)y(n)\qquad(x\in\ell^{p}). $$ 6)Assumc $1<p<\infty$ and provc that $\ell^{p}$ contains scqucnccs that converge weakly but not strongly. $\mathbf{\Psi}_{C}\mathbf{\Lambda}_{C}\mathbf{\Lambda}_{C}$ On the other hand, prove that every weakly convergent sequence in ${\mathcal{F}}^{1}$ converges strongly, in spite of the fact that the weak topology of $\ell^{1}$ is different from its strong (d) If topology (which is induced by the norm) metrized by $0<p<1,$ prove that $\ell^{p},$ $$ d(x,y)=\sum_{n=1}^{\infty}\mid x(n)-y(n)\mid^{p}, $$ is a locally bounded F-space which is not locally convex but that $(\ell^{p})^{\ast}$ nevertheless separates points on $\ell^{p}.$ (Thus there are many convex open sets in $\ell^{\,p\ }$ but not enough to form a base for its topology.) Show that $(\ell^{\prime})^{*}=\ell^{\infty},$ in the same sense as in (a). Show also that thc sct of all x with $\Sigma|x(n)|$ <lis weakly boundcd but not originally bounded $\mathbf{\Psi}({\boldsymbol{e}})$ For $0<p\leq1$ let $\tau_{p}$ , be the weak*-topology induced on ${\mathcal{E}}^{\infty}$ by $\ell^{p};$ see $\mathbf{\tau}(a)$ and $\scriptstyle(d).$ If $0<p<r\leq1,$ show that a $\tau_{p}$ p and $\tau_{r}$ arc different topologics Gis one wcaker than the other?) but that they induce the same topology on each norm-bounded subset of ${\mathcal{E}}^{\infty}.$ Hint: The norm-closed unit ball of $\ell^{\alpha_{-}}$ is weak*-compact.coNVEXITY 83 7 9 6Put $f_{s}(t)=e^{i n t}$ $(-\pi\leq t\leq\pi);$ ${\boldsymbol{C}}$ for which ${\mathsf{N}}(\beta)$ is an open subset of the complex lane i and its weak*- 8 Let ${\boldsymbol{C}}$ $\mathbf{r}\,1\leq p<\infty,$ prove tha lct $L^{p}=L^{p}(-\pi,\,\pi)_{i}$ wih respect to Lebesgue measure $|0,1|_{2}$ with the Let linear functionals $\Lambda$ on ${\mathfrak{t}}\,f_{n}\to0$ weakly in $L^{\,p},$ but not strongly $|f_{0}\rangle$ $L^{\infty}([0,1])$ has its norm topology $L^{1}.$ Show that ${\cal{C}},$ . he pce of al onusraucns ni0,1 topology as the dual of $\scriptstyle\{f\}_{x}$ is thessentia supremum ot is dense in $L^{\infty}$ in on g eogsu not tie oe"Combpae wlh Show that there exist continuous supremum norm. Let $\boldsymbol{B}$ sropisi rto .2. so tsamw oos is is iae ${\boldsymbol{C}}.$ be heach pco a omex onousn ntons be the closed uni balfo particular $|\operatorname{A}|$ attains no maximum on ${\boldsymbol{B}}.$ $E\subset L^{2}(-\pi_{)}\;;$ m) be the se f al functions $$ f_{m,\,n}(t)=e^{i m t}+m e^{i n t}, $$ where $m,$ n are integers and but $\mathbf{0}$ b is not in $E_{1},$ alihough lies the weak sequntialclosur such that some $E.$ (の) Find al $\scriptstyle{\mathcal{G}}$ in the weak closure ${\overline{{E}}}_{w}$ of $E.$ L.et $E_{1}$ be the set of all $g\subset L^{2}$ (a) Find all $g\in E_{1}.$ $0\leq m<n.$ is caldthe weak sequenial coswe o sequence in $\boldsymbol{\mathit{E}}$ converges weakly to g $(E_{1}$ (e) Show tha $0\in{\overline{{E}}}_{w}$ of $E_{1}$ $I O$ Represent ${\mathcal{F}}^{1}$ Tisxestsexe sanaose eo oeasncnia on $S=\{(m,n)\colon m\geq1,\,n\geq1\},$ such that Exercise 280 clo psesro s s se eetosie etecnostseou epsrog sne wisn a imsiuay smoe esea as the spaceofal eal function $\scriptstyle{\mathcal{X}}$ $$ \|x\|_{1}=\sum|x(m,n)|<\cdots. $$ Let $C_{\mathfrak{G}}$ Let $\mathcal{M}$ be the spaceo al elfunctions ${\mathcal{I}}$ consisting of all such that yGm, $n)\to0$ as $m+n arrow\infty,$ with norm $\mathbf{\vec{y}}$ y on $\boldsymbol{S}$ tha satisy te equation $|\nu||_{\infty}={\mathrm{sup}}\ |\nu(m,n)|$ $x\in{\mathcal{Y}}^{t}$ be the subspace of $$ m x(m,\,1)=\sum_{n=2}^{\infty}x(m,\,n)\qquad(m=1,\,2,\,3,\,\cdot\cdot). $$ $\mathbf{\nabla}(b)$ (oの) Prove that $\boldsymbol{B}$ B be the norm-closed tnit ball o Ielaixe t te weak-oy xiven by o) then (c) Prove that $\mathcal{M}$ is weak*-dense in ${\mathcal{I}}^{1}$ (See also Exercise 24, Chapter 4. ${\mathcal{E}}^{1}.$ In spitc of c) provc that the weakt $\ell^{1}=(c_{o})^{*}.$ ${\mathcal{E}}^{1}$ Prove that M is a norm-close suspaceof (d) Let closure of $M\cap B$ contains no ball、 Sugestion: if $\delta>0$ and $m>2/\delta,$ $$ |x(m,1)|\leq{\frac{||x||}{m}}<\frac{\delta}{2} $$ ${\cal I}{\cal I}$ Let $\textstyle X$ ${\mathrm{if~}}x\in M\cap B,$ alithough $x(m,1)=\delta$ for some $x\in\delta B.$ Thus B is not in the weak*- closure of $M r\cap B.$ Extn tis obas wtecenter wit s wakt-toloy, is of thefist category in isel beanfieinioal Frehtsac Pove ha $X^{\bullet}{\mathrm{:}}$84 GENERAL THEORY $$ \mathbf{\Sigma}_{\frac{1}{\lambda}} $$ ${\cal I}\dot{{\cal X}}$ Show that the norm-closed unit ball o $C_{\mathrm{O}}$ is not weakly compact; rcal tha (co ${}^{\mathsf{P}}$ (Exercise 10). 1 14 O in the $L^{2}$ $\mathrm{\boldmath~J~}\mathrm{\boldmath~Put}\,f_{\scriptscriptstyle N}(t)=N^{-1}\sum_{n=1}^{N^{2}}\,e^{i n t}$ Prove that $f_{N}\to0$ weakly in $L^{2}(-\pi,\,\pi).$ converges to seminorm $p_{K}$ By neoren3.3, ome sequence or covex omintins of te $f_{N}$ will not do (a) Suppose $\Omega$ on -norm. Find such a sequence. Show that $g_{N}=N^{-1}(f_{1}+\cdots+f_{N})$ define a $c(\mathbb{R}),$ is cly opact Hausoff space. For each compat $K\subset\Omega$ the spce o all omplex continuous functions on Q, b $$ p_{K}(f)=\operatorname*{sup}\;\{|f(x)|:x\in K\}. $$ every $\Lambda\in C(\Omega)^{*}$ correspond a compact Give CO te topology induced by hs colecon of sminorms. Prove hat t and a complex Borel measure p on $\textstyle K$ $K\subset\Omega$ such that $$ \Lambda f=\int_{\kappa}\,f\ \,d\mu\qquad(f\in C(\Omega)). $$ ${\boldsymbol{J}}{\boldsymbol{S}}$ 20 (1 (cos tion of at most is a compact convex set in C(Q) which satisfy $\int f$ $d u=0$ for every $\mu\in\Gamma$ of measures with ${\boldsymbol{L}}^{p}$ (b) Suppose $\Omega$ is an open set in ${\boldsymbol{C}}.$ . Find a countable collection ${\bf I}$ compact support in $\mathbb{Q}$ such that $\scriptstyle I(0)$ Gthe space o al holorphic functions in O) $I{\mathcal{S}}$ Let Suppose $\textstyle K$ consists of exactly those $f\in$ $\textstyle K$ through xto where it leaves C, the space of all continuous ${\boldsymbol{n}}.$ Draw a ${\cal{I}}{\cal{J}}$ Le $\textstyle X$ $\textstyle K$ ${\boldsymbol{\theta}},$ $<p<\infty)$ Suppose a topological vector space $R^{n},$ Prove that every $\operatorname{reg}K$ $K.$ sepats oins Proe that the weak- Use Exercise ${\bf1}.$ and ${\mathcal{I}}{\mathcal{I}}$ points of $\displaystyle K$ be a topological vector space on which $X^{\bullet}$ has a finite or countable Hamel basis $I{\mathcal{C}}$ (See Exercise , Chapter $\mathbf{\Omega}^{2}$ is metrizable if and only if $\textstyle{\cal{X}}$ ${\boldsymbol{C}}.$ $(1,0,1),(1,0,-1),$ topology of $X^{\ast}$ for the definition.) that contains the points $\scriptstyle R^{\prime}{}_{1}$ is a convex combina- Prove that the closed unit ball of $L^{1}$ Geiaive to Lebesgue measure on the uit interva) choice of the scalar field.) fas sieieos u tey poit te stae or uiu ba is an extreme point of the ball Determine the exteme points of the clsed unitball o hncis tunievv t suemorm CThe ansver ens on th sin 8,0), for be the smallest convex set in $R^{3}$ $0\leq\theta\leq2\pi.$ Show hatKiscompact but thathe set fall xtrem is no compact. Does such an example exist i $n+1$ extreme points of $K.$ Ssuggestion: Use inducion on line from some extreme point of contains a countable set $E=\{e_{1},\,e_{2}\,,\,e_{3}\,,\,\cdot\,\cdot\,\}$ $\textstyle X$ with the following properte (a) $e_{n}\to0$ as $\scriptstyle n\, arrow\,\infty$ $X$ rcould be uhe space of all omplcx polynomial ${\boldsymbol{E}}.$ $x=\sum\gamma_{n}(x)e_{n}.$ (6) Every $x\in{\mathcal{X}}$ is afnite incar combination of members o generated by the othe $e_{i}\,,$ (c) No ${\mathcal{C}}_{n}$ is in the closed subspace of $\textstyle X$ For example, $$ f(z)=a_{0}+a_{1}z+\cdot\cdot\cdot+a_{n}z^{n}, $$ with norm $$ ||f||=\biggl\{\biggl\}_{-\pi}^{\pi}|f(e^{t\theta})|^{2}\,d\theta\biggr\}^{1/2}, $$ and with $e_{n}(z)=n^{-1}z^{n-1}~(n=1,\,2,\,3,\,\cdot\cdot\cdot).$CONVEXIrY 85 22 If O $<p<1.$ Prove that each $\gamma_{n}$ in (b) is in $\textstyle K$ is cosed but notcompact and tha thextem $\ell^{p}\,;$ see Exercise ${\mathsf{S}}.$ Prove that the convex hull ${\boldsymbol{H}}$ of $X^{\star}$ Put $K=E\smallsetminus$ {0}. Then $\boldsymbol{K}$ is compact. 21 If O points of ${\boldsymbol{H}}$ are exactly the points of $K.$ (wit $\mathbf{0}$ as its only limit poin) which has $9<p<1,$ every $f\in L^{p}$ (except $f=0)$ $\textstyle K$ in ${\boldsymbol{L}}^{p}$ distance from O is less than that is the arithmetic mean of two functions whose no extreme point. $\operatorname{e}\!S\,$ (Sce Sction 1.47. Ue thst cosrct n expic example of a countable compact se show that ${\mathcal{I}}^{p}$ contains a compact set $(\ell^{p})^{*}$ separates points on Suggestion: Define This happens in spite of the fact uha $\boldsymbol{K}$ r whose convex hullis unbounded $x_{n}\in{\mathcal{E}}_{o}$ by $$ x_{n}(n)\ --n^{p-1},\qquad x_{n}(m)=0\qquad\mathrm{if~}m\neq n. $$ Let $\textstyle K$ consist or ${}\flat_{\mathsf{J}}\,.$ x, $X_{2}\,,\;x\,_{3}\,,\;\ldots\,.\;$ lf $$ \begin{array}{r l}{{\mathrm{\boldmath~\cdot~}}}&{{\mathrm{\boldmath~\cdot~}}y_{N}=N^{-1}(x_{1}+\cdots\cdot\cdot\mathrm{\boldmath~|~\cdot~}x_{N}),}\end{array} $$ hood ${\mathbf{}}V$ of show that {yw)}is unbounded in ${\hat{f}}\colon$ $Q\to X$ is continuous. A partition of ${\boldsymbol{Q}}$ Prove that to every neighbor- $\textstyle{\left(\begin{array}{l}{x}\end{array}\right)}$ is a $\ell^{p},$ 23 Suppose $\boldsymbol{\mu}$ is a Dore proabit masur on a compat Hasdor space is, by definition, a finite Fréchet space, and $\mathbf{0}$ O in $\textstyle X$ there corresponds $\mathbf{\dot{a}}$ partitio $(E_{\circ})$ such that the diferenc collecto of disioint Borelsusets o ${\boldsymbol{Q}}$ whose union is ${\boldsymbol{\cal Q}}$ $$ z=\int_{o}f\,d\mu-\sum_{i}\mu(E_{i})f(s_{i}) $$ ${\mathcal{D}}$ lies in ${\mathbf{}}V$ for every choice of $S_{i}\in E_{i}$ CThis exhibisthe integral as a strong limit o and if $|\mathrm{A}x|\leq1$ mapping of $\textstyle X$ “Rieman sums.")Ssugesin Take ${\mathit{V}}$ convex and balanced. If A∈ $X^{\ast}$ $f(s)-f(t)\in V$ whcncvcr for every xe V, then $\vert\Lambda z\vert\leq1,$ providedtht the sets $3.27,$ assume that ${\boldsymbol{T}}$ is a continuous linear prove that into a tological vector space ${\mathbf{}}Y$ $\textstyle E_{i}$ are chosen sothat separates points, and $\scriptstyle{\mathcal{S}}$ and lie in the same $E_{i}$ ln addtion to the hypotheses of Theorem on which $Y^{\star}$ $$ T\int_{_Q}f\,d\mu=\int_{_Q}(T f)\,d\mu. $$ 25 spacc $\textstyle X$ on which $X^{\star}$ for every Ae $Y^{\rtimes}$ $\alpha=E$ such that in a topological vector Hint: $\Lambda T\in X^{*}$ Let ${\boldsymbol{E}}$ be the t a exte onts o compact onvx s $\textstyle K$ corresponds a regular Borel probability measure ${\boldsymbol{\mu}}$ separates poins. Prove that to every $\scriptstyle y\in K$ on $$ y-\int_{Q}x\,d\mu(x). $$ 26 Suppose S is a region in ${\boldsymbol{C}},$ $\textstyle X$ is a Fréchet space, and $f\colon\Omega\to X$ is holomorphic that is, concerning the formula o Statean prove a theorem cncng te powe e eaito where $c_{n}\in X.$ $f_{\uparrow}$ $f(z)=\sum(z-a)^{n}c_{n},$ 6b Ceneralize Morers thorm t xvaled orpic uncton86 GENERAL THEORY 27 Suppose $\{\alpha_{i}\}$ (e) Fr a sequence of complex holomorphic functinsin Q,uniform convergence o is an entire compact subsets of $\Omega$ implies that the it is holomorphic. Dos this generalize t X-valued holomorphic functions? is a bounded set of distinct complex numbers, $f(z)=\sum_{\alpha}^{\alpha}\,c_{n}\,z^{n}$ function with every $c_{n}\neq0,$ and $$ g_{i}(z)=f(\alpha_{i}z). $$ Prove that the vector space generated by the function ${\mathfrak{g}}_{i}$ is dense in the Frchet space H(C) defined in Section 1.45 is a measure with compact support such that $\textstyle\int g_{i}\,d\mu=0$ Suggestion:: Assume ${\boldsymbol{\mu}}$ for all i. Put $$ \phi(w)=\int f(w z)\,d\mu(z)\qquad(w\in C). $$ Prove that $\phi(w)-0$ for all w. Deduce that $\textstyle\bigcap z^{n}\,d\mu(z)=0$ for $n=1$ ${\mathrm{2,}}$ 3,... Use Exercise $14.$ Describe the closed subspace of $H({\mathcal{C}})$ generated by the functions ${\mathcal{G}}_{i}$ if some of the ${\boldsymbol{c}}_{n}$ are ${\boldsymbol{0}}.$ is a Frechet space (or, more generally a metrizablc ocally convex space) 28 SuppOse $\textstyle X$ Prove the following statements (b) If $\textstyle X$ is separable, each $E_{n}$ is the union of countably many weak*-compact sets $E_{n}.$ $\textstyle X$ (Compare with (a) $X^{*}$ is metrizable. The weak*-topology of $X^{\ast\ast}$ is therefore separable, and some countable subset o $X^{\ast\ast}$ separates points on Exercise 15.) (c) If $\textstyle K$ is a weakly compact subset of $\textstyle X$ and if $x_{\mathrm{s}}\in K$ is a weak limit point of soine countable set $E\mathop{\mathop{}E\mathop{<}K,}$ then there is a sequence txm} in $\boldsymbol{E}$ which converges weakly to (6) to ${\bf Y_{\nu}}$ Remark: The point of $\mathbf{\Psi}(c)$ be the smallest closed subspace of $\textstyle X$ that contains $\textstyle E.$ Apply $\textstyle x_{0}\,.$ .Hint: Let ${\cal Y}\,$ metrizablc. to conclude that the weak topology of $K\cap Y{\mathrm{~is}}$ is the existence of convergent subsequences rather than swbnes. Note that the exist compact Hausdor spces in which no sequencc $29$ Let $\Lambda_{p}f=f(p).$ of distinct points converges. $K,$ with the supremum norm. For $p\in K,$ define $\Lambda_{p}\in C(K)^{*}$ by ${\mathsf{C}}({\boldsymbol{R}})$ be the Banach space of all continuous complex functions on the compact Hausdorff space Show that $p\to\Lambda_{p}$ is a homeomorphism of $\textstyle K$ into $C(K)^{*}$ ,equipped with its weak*-topology. Part eo of Exercise 2 can therefore not be extended to weak*- compact sets.