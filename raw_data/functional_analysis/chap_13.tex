1 3 UNBOUNDED OPERATORS Introduction 13.1 Definitions L $\mathrm{e}$ be a Hlilbert space. By an operalor in ${\boldsymbol{H}}$ we shall now mean a linear mapping ${\boldsymbol{T}}$ whose domain ${\mathcal{D}}(T)$ is a subspace of ${\boldsymbol{H}}$ and whose range ${\mathcal{R}}(T)$ lies in $H.$ [relative to the norm topology that It is not assumed that T is bounded or continuous. Of course,if Tis continuous has a continuous ${\mathcal{D}}(T)$ inherits from H] then ${\boldsymbol{T}}$ extension to the closure of ${\mathcal{D}}(T),$ hence to $\textstyle H,$ since ${\overline{{\sigma}}}({\overline{{T}}})$ is complemented in $H.$ 1. In of ${\boldsymbol{T}}$ [that is, of the ordered pairs {x, Tx}, where $\mathbf{\mathcal{V}}_{\nu}^{*}$ $S x=T x$ of some member of ${\mathcal{D}}(T).$ Obviously, $\boldsymbol{\mathsf{S}}$ is an extension that case, ${\boldsymbol{T}}$ T is the restriction to ${\mathcal{D}}(I)$ ${\boldsymbol{T}}$ in $H$ s the subspace of ${\mathcal{B}}(H)$ $H\times H$ that consists The graph ${\mathcal{G}}(T)$ of an operator ranges over ${\mathcal{D}}(T)\subset{\mathcal{D}}(S)$ and for $x\in{\mathcal{D}}(T)]$ if and only if $\mathcal{G}(T)\subset\mathcal{G}(S)$ This inclusion will often be written in the simpler form (1) $$ T\subset\mathbf{\nabla}S. $$ the closed graph theorem, $T\in{\mathcal{B}}(H)$ if and only if is one whose graph is a closed subspace of and ${\cal T}\,$ is closed By A closed operator in $\textstyle H$ $H\times H.$ $\mathbb{Z}(T)=H$330 BANACH ALGEBRAS AND SPrCTRAL THEoRY We wish to associale a Hilbert space adjoint $T^{\star}$ LO ${\boldsymbol{T}}.$ Its domain ${\mathcal{D}}(T^{*})$ is to consist of all $y\in H$ for which the linear functional (2) $$ x\to(T x,\,y) $$ is continuous on ${\mathcal{D}}(T).$ If $y\in{\mathcal{D}}(T^{\bullet})$ ,then the Hahn-Banach theorem extends the functional (2) to a continuous linear functional on $\textstyle H,$ and therefore there exists an element $T^{*}y\in H$ that satisfies (3) $$ (T x,y)=(x,\,T^{*}y)\qquad[x\in{\mathcal{D}}(T)]. $$ Obviously $T^{\bullet}y$ will be uniquely determined by (3) if and only if 9(T) is dense in $\textstyle H,$ that is, if and only if T is densely defined. The only operators ${\boldsymbol{T}}$ T that will be given an adjoint $T^{\mathbf{\wedge}}$ are thercfore the dcnscly defined ones. Routine verifications show then that $T^{\bullet\sharp}$ is also an operator in $\textstyle H_{\cdot}$ that is, that ${\mathcal{D}}(T^{*})$ is a subspace of $\textstyle H$ I and that $T^{\infty}$ is linear. Ordinary algebraic operations with unbounded operators must be handled with care, because the domains have to be watched. Here are the natural defntions for the domains of sums and products: (4) $$ \begin{array}{c l c r}{{{\mathcal D}(S+T)={\mathcal D}(S)\cap{\mathcal D}(T),}}\\ {{{\mathcal D}(S T)=\{x\in{\mathcal D}(T)\colon T x\in{\mathcal D}(S)\}.}}\end{array} $$ (5) 13.2 Theorem Suppose S, T,and ${\boldsymbol{S}}{\boldsymbol{T}}$ tare densely defined operators in $H.$ . Then (1) $$ T^{*}S^{*}\subset(S I)^{*}. $$ IJ, n adio $\boldsymbol{\mathsf{S}}$ ${\mathcal{B}}(H)_{:}$ , then (2) $$ T^{*}S^{*}=(S T)^{*}. $$ Note that (1) asserts that $(S T)^{\mathrm{sg}}$ is an extension of T*S*. The equality (2) implies that T*S* and (ST)* actually have the same domains. PROOF Suppose $x\in{\mathcal{D}}(S T)$ and $y\subset{\mathcal{D}}(T^{*}S^{*}).$ Then (3) $$ (T x,\,S^{*}y)=(x,\,T^{*}S^{*}y), $$ because $x\in{\mathcal{D}}(T)$ and $S^{\bullet}y\in{\mathcal{D}}(T^{\bullet}),$ and (4) $$ (S T x,\,y)=(T x,\,S^{*}y), $$ because $T x\in{\mathcal{D}}(S)$ and $y\in{\mathcal{D}}(S^{s}).$ Hence (5) $$ (S T x,\,y)=(x,\,T^{*}S^{*}y). $$ This proves (1)uNmounDrn orR Arons 331 ${\mathcal{D}}(S^{*})=H$ Assume now that $S\in{\mathcal{B}}(H)$ and $y\in{\mathcal{D}}((S T)^{*}).$ Then $S^{\ast}\in{\mathcal{B}}(H),$ so that , and (6) $$ (T x,\,S^{*}y)=(S T x,\,y)=(x,\,(S T)^{*}y) $$ for every $x\in{\mathcal{D}}(S T)$ Hence $S^{\bullet}y\in{\mathcal{D}}(T^{\bullet}),$ and therefore $y\in{\mathcal{D}}(T^{*}S^{*}).$ 、Now ${j}//{j}$ (2) follows $\operatorname{from}\left(1\right).$ 13.3 Definition An operator $\boldsymbol{\mathit{I}}$ 'in $H$ l s said to be symmetric if (1) $$ (T x,y)=(x,T y) $$ whenever $x\in{\mathcal{D}}(T)$ and $y\in{\mathcal{D}}(T).$ The densely defined symmetric operators are thus exactly those that satisfy (2) $$ T\subset T^{*}. $$ If $T=T^{\ ^{\overset{.}{\land}}},$ then ${\boldsymbol{T}}$ ris said to be self-adjon do not. These tworopertisesedety onie wen Te (") I eneral the 13.4 :Example Let ${\cal H}={\cal L}^{2}={\cal L}^{2}(|0,$ 1), relative to Lebesgue measure. We define operators $T_{1}.$ $\textstyle T_{2}$ ,and $T_{3}$ in $L^{2}.$ Their domains are as follows: ${\mathcal{D}}(T_{1})$ consists f all asolutely continuous functions $\boldsymbol{\mathit{f}}$ on [O, $\mathrm{i}\models\mathrm{i}\bigstar$ with derivative $f^{\prime}\in L^{2}.$ $$ \begin{array}{l c r}{{\mathcal{Q}(T_{2})=\mathcal{Q}(T_{1})\cap\{f_{2}f(0)=f(1)\}.}}\\ {{\mathcal{Q}(T_{3})=\mathcal{Q}(T_{1})\cap\{f_{2}f(0)=f(1)=0}}\end{array} $$ 0}. These are dense in $L^{2}$ Define (1) $$ T_{k}f=i f^{\prime}\qquad{\mathrm{for}}\,f\in{\mathcal{D}}(T_{k}),k=1,2,3. $$ We claim that (2) $$ {\cal T}_{1}^{\ast}=T_{3},\qquad T_{2}^{\ast}=T_{2},\qquad T_{3}^{\ast}-T_{1}. $$ Since $T_{3}\subset T_{2}\subset T_{1},i t f o l o$ $T_{3}$ and that the extension $T_{1}$ 0f $T_{2}$ is not symmetric (but not self-adijoint) operator ws that $T_{2}$ is a self-adjoint extension of the symmetric Let us prove (2). Note frst that (3) $$ (T_{k}f,\,g)=\int_{0}^{1}(i f^{\prime})\bar{g}=\int_{0}^{1}\!f(\overline{{{i}}}g^{\prime})=(f,\,T_{m}g) $$ when $f\in{\mathcal{D}}(T_{k}),\,g\in{\mathcal{D}}(T_{m}),$ and $m+k=4,$ since $\mathrm{then}\,f(1)\bar{g}(1)=f(0)\bar{g}(0)$ .It follow that $T_{m}\subset T_{k}^{*}$ ,0r (4) Tc T3, T,c T艺, T, c Tt332 BANACH ALGEBRAS AND SPECTRAL THEORY fe 9(TA), Suppose now that $g\in{\mathcal{D}}(T_{k}^{*})$ and $\phi=T_{k}^{*}\,g$ Put $\Phi(x)=\textstyle\int_{0}^{x}\phi$ Then, for (5) $$ \int_{0}^{1}i f^{\prime}{\bar{g}}=(T_{k}f,\,g)=(f,\,\phi)=f(1){\overline{{\Phi(1)}}}-\int_{0}^{1}f^{\prime}{\overline{{\Phi}}}. $$ When $k=1$ or $2,$ then ${\mathcal{D}}(T_{k})$ contains nonzero constants, so that (5) implies $\Phi(1)=0$ When $k=3,$ ,then $f(1)=0.$ lt follows, in all cases, that (6) $$ i{\mathcal{O}}-\Phi\in{\mathcal{P}}(T_{k})^{\perp}. $$ Sincc ${\mathcal{R}}(T_{1})=L^{2};$ $i g=\Phi$ if $k\simeq1,$ and since $\Phi(1)-0$ in that case, $g\in{\mathcal{D}}(T_{3}),$ Thus $T_{1}^{*}\subset T_{3}$ If $k=2$ or 3, then ${\mathcal{R}}(T_{k})$ consists of al $u\in L^{2}$ such $\operatorname{that}\rfloor{\overset{1}{\underset{\mathrm{tot}}{}}{\boldsymbol{u}}=0}.$ Thus (7) $$ \mathcal{R}(T_{2})=\mathcal{R}(T_{3})=Y^{\perp}, $$ where ${\cal{Y}}$ is the one-dimensional subspace of L ${\cal L}^{2}$ that contains the constants. Hence (6) implies that $i g-\Phi$ is constant. Thus $\mathcal{G}$ is absolutely continuous and $g^{\prime}\in L^{2}.$ , that is, $g\in{\mathcal{B}}(T_{1}).$ Thus $T_{3}^{*}\in{\mathfrak{T}}_{\infty}\cap\nu$ If $K=2,$ then $\Phi(1)=0,$ , hence g(0) = g(1), and g e 9(T》). Thus $T_{2}^{*}\subset T_{2}\,.$ This completes the proof Before we turn to a more detailed study of the relations between symmetric operators and sef-adjoint ones, we insert another example 13.5 Example Let $H=L^{2},$ as in Example 13.4, define $D f-f^{\prime}$ for fe ${\mathcal{D}}(T_{2}),$ say (the exact domain is now not very important) and define $(M f)(t)=t f(t).$ Then $(D M-M D)f=f,$ or (1) $$ {\cal D}M-M{\cal D}=I, $$ where $\boldsymbol{\mathit{I}}$ denotes the identity operator on the domain of ${\mathcal{D}}.$ The identity operator appcars thus as a commutator of two operators, of which only one is bounded. The question whether the identity is the commutator of two bounded operators on $\textstyle{H}$ arose in quantum mechanics. The answer is negative, not just in ${\mathcal{A}}(H)$ ), but in every Banach algebra: 13.6 Theorem Uf A is a Banach aloebra with unit element e $\;j\colon x\in A$ and $y\in A.$ uhen $$ x y-y x\neq e. $$ The following proof, due to Wielandt, does not even use the completeness of ${\bar{A}}.$ PROOF Assume xy -yx =e. Make the induction hypothesis (1 刈y-yx”=nx"-1≠ 0,UNpoUNDED orRATORs 333 then $x^{n}\neq0$ which is assumed to hold for $n=1.$ If (I) holds for some positve integer ${\boldsymbol{n}}_{\mathrm{{J}}}$ and $$ \begin{array}{r l}{x^{n+1}y-y x^{n+1}=x^{n}(x y-y x)+(x^{n}y-y x^{n})x}\\ {=x^{n}e+n x^{n-1}x=(n+1)x^{n},}\end{array} $$ so that (1) holds with $\textstyle n+1$ in place of . It follows that $$ n\|x^{n-1}\|=\|x^{n}y-y x^{n}\|\leq2\|x^{n}\|||y||\leq2\|x^{n-1}\|\|x|\|y\|, $$ orn 21ll orvey posiviner . hs sovousy impossib // Graphs and Symmetric Operators 13.7 Graphs If F ${\big.}H{\big/}$ is a Hilbert space, then $H\times H$ can be made into a Hilbert to be snace by defining theinner product of two elements $\scriptstyle[a,\,b]$ and $\scriptstyle(c,\,d)$ of $H\times H$ (1) $$ (\{a,b\},\{c,d\})=(a,c)+(b,d), $$ is given by where (a, c) denotes the inner product in ${\cal H}$ We leave it as an exercise o vrify tha $H\times H$ thisasl herpetesd n Sction 2.1. paticta henom (2) Define $$ \|\{a,b\}\|^{2}=\|a\|^{2}+\|b\|^{2}. $$ (3) $$ V\{a,b\}=\{-b,a\}\ \;\;\;\;\;\;\;(a\in H,\,b\in H). $$ if Then ${\mathbf{}}V$ is a umiary operator on $H\times H$ which satisfe $V^{2}={-I}.$ Thus $V^{2}M=M$ $\bar{M}$ is any subspace of $H\times H,$ in terms of $T\colon$ This pcrtor ilds arenarkabledescription o $T^{\ast}$ 13.8 Theorem $\textstyle I\!\!\!/I\,I$ isa densely defied operaior in $\textstyle H,$ then (1) $$ {\mathcal{O}}(T^{*})=[V{\mathcal{G}}(T)]^{1}, $$ he orthogonal complement o $V{\mathcal{G}}(T)\;i n\;H\times H.$ Note that once ${\mathcal{G}}(T^{\ast})$ is known, so are ${\mathcal{D}}(T^{\ast})$ and $T^{\aleph_{\star}}.$ PRO0F .Each of thefowinfur statents s clearly euivulent o the on that follows andor precedes t (2) {y, z}e 9(T*) (3) $$ (T x,\,y)-(x,\,z)\qquad{\mathrm{for~every~}}x\in{\mathcal{D}}(T). $$ $(4)$ $$ (\{-T x,\,x\},\,\{y,\,z\})=0\qquad{\mathrm{for~every~}}x\in{\mathcal{D}}(T $$ ) $({\mathcal{5}})$ {y, z}e[V9(T)1 ${\it j}/{\it j}/{\it j}$334 ANACH ALCEBRxAs AND srcCTRAL THEoxv 13.9 Theorem 1If T is a densely defined operator in $\textstyle H.$ , then $T^{\cong}$ is a closed operator. ln particular, self-adjoint operatiors are closed PROOF $M^{\perp}$ is closed, for every $M\subset H\times H$ . Hence ${\mathcal{G}}(T^{*})$ is closed in $H\times H,$ by Theorem 13.8 / 13.10 Theorem I Tis a densely defined closed operator in $\textstyle H_{\cdot}$ 1, then (1) $$ {\cal H}\times{\cal H}=V{\mathcal G}(T)\oplus{\mathcal G}(T^{*}), $$ a drect sum of two orlhogonal subspaces PROOP If ${\mathcal{G}}(T)$ is closed, so is $V{\mathcal{G}}(T),$ since ${\mathcal{V}}$ 'is unitary, and therefore Theorem $J/I\!J$ 13.8 implies that $V{\mathcal{G}}(T)=[{\mathcal{G}}(T^{*})]^{\perp};$ see Theorem 12.4. Corollary 1 $f\,a\in H$ and $b arrow H$ the system of equations $$ \begin{array}{c}{{-T x+\;y=a}}\\ {{x+\,T^{*}y=b}}\end{array} $$ has a unique solution with $x\in{\mathcal{D}}(T)\;a n d\;y\in{\mathcal{D}}(T^{*}).$ Our next theorem states some conditions under which a symmetric operator is self-adjoint. 13.11 Theorem Supposc ${\boldsymbol{T}}$ is a desly defined operaior in $\textstyle H_{\cdot}$ H, and ${\cal T}\,$ is symmetric. (の)r (a)1f 9(T) = 1/, then Tis self-adjoinl and $T\in{\mathcal{P}}(H).$ is dense in $\textstyle H.$ ,and $T^{-1}$ is self ${\boldsymbol{r}}\,h$ self-adjoint and one-to-one, then ${\mathcal{R}}(T)$ adjoint. (c) If ${\mathcal{R}}(T)$ is dense in ${\cal H}$ H, then ${\boldsymbol{T}}$ 'is one-to-one. $T^{-1}\in{\mathcal{R}}(H).$ (d)If ${\mathcal{R}}(T)=H;$ ,1hen T is sef-adjoint, and $T=T^{*}.$ PRoor() By assumption, $T\subset I^{*}$ ir ${\mathcal{D}}(T)=H.$ it is thus obyious that Hcncc ${\mathbf{}}T$ is closed (Theorem 13.9) and therefore continuous, by the closed graph theorem.(We could also refer to Theorem 5.1.) Since ${\boldsymbol{T}}$ (b) Suppose y $\mathrm{\boldmath~\nabla~}\mathcal{M}(V).$ Then $x\to(T x,\;y)=0$ is continuous in ${\mathcal{D}}(T),$ hence $y\in{\mathcal{D}}(T^{*})={\mathcal{D}}(T),$ and $(x,T y)=(T x,\,y)=0$ for all $x\in{\mathcal{D}}(T).$ Thus $T y=0.$ is assumed to be one-to-one,it follows that $y=0.$ This proves that ${\mathcal{R}}(T)$ is dense in ${\boldsymbol{H}}$ $\mathcal{D}(T^{-1})=\mathcal{B}(T),$ and $(T^{-1})^{s}$ $T^{-1}$ is therefore densely defined, with exists. The relations (1) 9(T-1)= V9(-T) and VG(T-1)=9(-7)UNBOUNDED OPERATORs 335 applied to are easily verifed. Since and to $-T,$ yiels he orthogonal decomposition $-T.$ Hence Theorem 13.10, ${\mathbf{}}T$ is self-adjoint, so is $T^{-1}$ (2) $$ H\times H=V\mathcal{P}(T^{-1})\oplus\mathcal{P}((T^{-1})^{*}) $$ and (3) $$ H\times H-V\mathcal{G}(-T)\oplus\mathcal{G}(-T)=\mathcal{G}(T^{-1})\oplus V\mathcal{G}(T^{-1}). $$ Consequently (4) $\mathcal{G}((T^{-1})^{*})=[V\mathcal{G}(T^{-1})]^{\perp}=\mathcal{G}(T^{-1}),$ If so that Awhich shows that $(T^{-1})^{*}=T^{-1}$ Then $(x,\,T y)=(T x,\,y)=0$ for every ye 9V(T)) $w\in{\mathcal{L}}(T),$ Thuts $\left(c\right)$ Supos $T x=0.$ $x=0.$ is one-to-one, and ${\mathcal{D}}(T^{-1})=H.$ $x\in H$ $x\perp\mathcal{R}(T),$ and therefore (c) implies that ${\boldsymbol{T}}$ for some $z\in{\mathcal{D}}(T)$ and $(d)$ Since ${\mathcal{R}}(T)=H,$ and $\gamma\in H$ , then $x=T z$ and $y=T w,$ $$ (T^{-1}x,y)=(z,T w)=(T z,w)=(x,T^{-1}y). $$ Hence ${\boldsymbol{T}}^{-1}$ is symtrc.(G implies tha that $T=(T^{-1})^{-1}$ is also self-adjoint. // and now it follows from $T^{-1}$ is selfadjoint (and bounded) $\mathbf{\nabla}(b)$ 13.12 Theorem 1f T is a densely defined closed operator in $\textstyle H,$ 1, then ${\mathcal{D}}(T^{\mathbb{A}})$ is dense and $T^{\ ]}{}^{\star}{}^{*}=T.$ PROOF Since ${\cal{V}}$ is unitary, and $V^{2}=-I.$ , Theorem 13.10 gives the orthogonal decomposition (1) $$ H\times H={\mathcal{G}}(T)\oplus V{\mathcal{G}}(T^{*}). $$ Suppose $z\perp{\mathcal{D}}(T^{*}).$ Then Gz, )=0 and therefore (2) $$ (\{0,z\},\{-T^{\bullet}y,\,y\})=0 $$ for all ye 90(T*). Thus $\{0,z\}\in[V\mathcal{G}(T^{*})]^{\perp}=\mathcal{G}(T),$ I, and $T^{\bullet\bullet\bullet}$ is defined. $z=T(0)=0.$ Conscqucntly ${\mathcal{D}}(T^{*})$ is dense in $H.$ which implies that Another appication of Theorem 13.10 gives therefore (3) $$ {\cal H}\times{\cal H}=V{\mathcal G}(T^{*})\oplus{\mathcal G}(T^{*}*). $$ By () and (3) (4) $$ {\mathcal{O}}(T^{*\;*})=[{\mathcal{V}}{\mathcal{G}}(T^{*})]^{\bot}={\mathcal{G}}(T), $$ so that $T^{\cong{\frac{1}{\theta}}}\equiv T.$ $J/j f$ we shall nowse tht operators of the form $T^{\ast}\star T$ have interesting properties In particular, 9(7*T) cannot be very smal336 pANACH ALorpRAs ANp s RAL THEORY 13.13 Theorem Suppose ${\mathbf{}}T$ is a densely defined closed Operator in $\textstyle H,$ and $Q=I+T^{*}T$ (a)Under the assumptions, ${\cal Q}_{\eta}=1~~~~~~~~~~~~~~~~~~~~~~~~~~~~~~~~~~~~~~~~~~~~~~~~~~~~~~~~~~~~~~~~~~~~~~~~~~~~~~~~~~~~~~~~~~~~~~~~~~~~~~~~~~~~~~~~~~~~~~~~~~~~~~~~~~~~~~~~~~~~~~~~~~~~~~~~~~~~~~~~~~~~~~~~~~~~~~~~~~~~~~~~~~~~~~~~~~~~~~~~~~~~~~~~~~~~~~~~~~~~~~~~~~~~~~~~~~~~~~~~~~~~~~~~~~~~~~~~~~~~~~~~~~~~~~~~~~~~~~~~~~~~~~~~~~~~~~~~~~~~~~~~~~~~~~~~~~~~~~~~~~~~~~~~~~~~~~~~~~~~~~~~~~~~~~~~~~~~~~~~~~~~~~~~~~~~~~~~~~~~~~~~~~~~~~~~~~~~~~~~~~~~~~~~~~~~~~~~~~$ is a one-to-one mapping of $$ \mathcal{D}(\mathcal{O})=\mathcal{D}(T^{*}T)=\{x\in\mathcal{D}(T)\colon T x\in\mathcal{D}(T^{*})\} $$ onto $\textstyle H,$ and there are operators $B\in{\mathcal{B}}(H),$ ${\boldsymbol{C}}$ e M(H) that satisfj $\|B\|\leq1,$ $\|C\|\leq1,\ C-T B,$ ,and (1) $$ B(I+T^{*}T)\subset(I+T^{*}T)\,B=I. $$ Also, $B\geq0,$ and $T^{\ast}T$ is self-adjoint (b) If $T^{\prime}$ 'is the restriction of T to ${\mathcal{D}}(T^{\ast}T),$ then S(T') is dense in G(T"). Here,and in the sequel,theletter I denotes the identity operator with domain $\textstyle H.$ PROOF I $\textsf{f}x\in{\mathcal{D}}(Q)$ then $T x\in{\mathcal{D}}(T^{*}),$ so that (2) $$ (x,\,x)+(T x,\,T x)=(x,\,x)+(x,\,T^{*}T x)=(x,\,Q x). $$ Therefore $\|x\|^{2}\leq\|x\|\ \|Q x\|,$ which shows that C ${\mathcal{Q}}$ is one-to-one $h\in H$ a unique vector By Theorem 13.10 there corresponds to every $B h\in{\mathcal{D}}(T)$ and a unique $C h\in{\mathcal{D}}(T^{*})$ such that (3) $$ \{0,h\}=\{-T B h,\,B h\}+\{C h,\,T^{*}C h\}. $$ $\mathrm{I}\mathbf{t}$ is clear that $\boldsymbol{B}$ and ${\boldsymbol{C}}$ are linear operators in $H.$ , with domain $H.$ The two vectors on the right of $\left(3\right)$ are orthogonal to each other(Theorem 13.10). The definition of the norm in $H\times H$ implies therefore that (4) $$ \|h\|^{2}\geq\|B h\|^{2}-1~\|C h\|^{2}\qquad(h\in H), $$ so that $\|B\|\leq1$ and $\|C\|\leq1.$ and that Consideration of the components in (3) shows that $C=T B$ (5) $$ \begin{array}{l}{{\\!{\hat{\Sigma}\cdot}}}\end{array}{\hat{\Pi}=B h+T^{*}C h=B h+T^{*}T B h=\ Q B h}\end{array} $$ ${\cal H}$ for every ${\mathfrak{k}}\in H.$ Hence $Q{\boldsymbol{B}}=I.$ In particular, $\bar{\boldsymbol{B}}$ is a one-to-one mapping of onto ${\mathcal{D}}(Q).$ If $y\in{\mathcal{D}}(Q),$ then $y=B h$ for some $h\in H,$ hence ${\cal Q}y=Q B h=h,$ and B $|Q\rangle=B h$ = y: Thus $B Q\subset I,$ and $\operatorname{\mathcal{(1)}}$ is proved If $h\in H,$ , then $h\in Q x$ for some $x\in{\mathcal{P}}(O)$ , so that (6) $$ (B h,h)=(B Q x,\,Q x)=(x,\,Q x)\geq0, $$ by (2). Thus $B\geq0,$ $\boldsymbol{B}$ is self-adjoint (Theorem 12.32), and now (b) of Theorem 13.1l shows that ${\cal Q}\,$ is self-adjoint, hence so is $T^{*}T=Q-I.$ This completes the proof of part (a).UNnoUNDED OrERATOns 337 ${\mathcal{G}}(T)$ Since ${\boldsymbol{T}}$ r s a closed operator ${\mathfrak{g}}(T)$ is a closed subspace of $H\times H;$ hence for every is a Hibert space. Assume $\{z,T z\}\in{\mathcal{G}}(T)$ is orthogonal to ${\mathcal{O}}(T^{\prime}).$ Then, $x\in\mathcal{D}(T^{*}T)=\mathcal{D}(\mathcal{O}),$ 0 =(({z,T7z},{x, Tx) =(z,20 +(7z,7Tx) =(2,x)+ (2, 7*7×) = (2,Q0 But ${\mathcal{R}}(Q)=H$ Hence $z=0.$ This proves (b) // i 13.14 Definition A symmetric operato $T\operatorname{in}H$ is said to be maximally symmetric ${\boldsymbol{T}}$ has no proper synimtri exension . te assumtion (1) $$ T\subset S,\ S\operatorname{symmetric} $$ imply that ${\mathcal{S}}=T$ 13.15 Theorem Sef-adjoint operators are maximalysymmetric $S^{*}\subset T^{*}$ PR0OoF Suppose ${\cal T}\,$ 'is self-adjoint, $\boldsymbol{\mathsf{S}}$ is symmetric (that is, $S\subset S^{*}),$ and $T\subset S.$ .Hence This inclusion implies obviously (by the very definition of the adjoint) that which proves that $S\subset S^{*}\subset T^{*}=T\subset S,$ // $\ S=Y.$ 13.16 Theorem If Tisa ymmetric operator in ${\cal{I}}{\cal{I}}$ (not necessarily densely defined) the folloing statements are true (a)HTXx + ixll - lx| + ||7X1 [x∈ 9(T)] (の)Tis closed operatorif and only i (7T i)isclose (c) T + il is one-to-one. (d) ${f\!\!{\mathcal{R}}}(T+i I)=H$ , then ${\cal T}\,$ is maximally symmetric $b y\,-i.$ (e) The preceding statements are aso trueif isreplaced PRO0F Statement $\mathbf{\Psi}(a)$ follows from the identity $$ |T x+i x||^{2}=\|x\|^{2}+\|T x\|^{2}+(i x,T x)+(T x,i x), $$ combined with the symmetry of ${\cal T}.$ By a), $$ (T+i J)x rightarrow\{x,\,T x\} $$ lf subset of ${\mathcal{D}}(T_{1})_{\mathrm{{I}}}$ then $T_{1}+i I$ is an isometric onc-to-one correspondence between the range or '[that is, ${\mathcal{D}}(T)$ is a proper $(a).$ graph of ${\mathcal{R}}(T+i I)=H$ and $T_{1}$ is a proper extension of ${\boldsymbol{T}}$ $T+i I$ and the ${\cal T}.$ This proves (b).Nxt,(e is also an immediate consequence o one-to-one. By $c_{i},T_{i}$ is a proper extension of $T+i I$ which cannot be is not symmetric. This proves $\langle d\rangle.$ in place of ${\dot{t}}.--iiint\!/\!/{\slash{\prime}}$ It is clear that this prof is equaly valid with $-{\dot{l}}$338 BANACH ALOEBRAS AND SPECIRAL THEORY The Cayley Transform 13.17Definition The mappin (1) $$ t arrow{\frac{t-i}{t+i}} $$ the point $\mathbf{l}\rangle.$ sets up a one-to-one correspondcnc between the real line and the unit circle (minus The symbolic calculus studied in Chapter l2 shows therefore that every self-adjoint $T\in{\mathcal{B}}(H)$ gives rise to a unitary operator (2) $$ U=(T-i I)(T+i I)^{-1} $$ and that every unitary $U_{\mathbf{\delta}}U$ whose spectrum does not contain the point l is obtained in this way This relation $T\hookrightarrow U$ will now be extended to a one-to-one correspondence between symmetric operators, on the one hand, and isometries, on the other. Let ${\boldsymbol{T}}$ be a symmctric operator in $I{\bar{I}}.$ 。 Theorem 13.16 shows that 3) $$ \|T x+i x\|^{2}=\|x\|^{2}+\|T x\|^{2}=\|T x-i x\|^{2}\qquad(x\in{\mathcal{D}}(T)). $$ Hence there is an isometry $U_{\mathrm{,}}$ with (4) $$ \mathcal{D}(U)=\mathcal{A}(T+i I),\qquad\mathcal{A}(U)=\mathcal{A}(T-i I), $$ defined by (5) $$ U(T x+i x)=T x-i x\qquad(x\in{\mathcal{D}}(T)). $$ Since $(T+i I)^{-1}$ maps $\mathcal{B}(U)$ onto $\mathbb{Z}(T)$ $U$ can also be written in the form (6) $$ U=(T-i I)(T+i I)^{\textrm{t}} $$ This operator $U_{\mathit{\Phi}}$ is called the Cayley transform of $\textstyle T.$ Its main features are summarized in Theorem 13.19. It will lad to an eay proof of the spectral theorem for slf-adjoint (not necessarily bounded) operators for every 13.18 Lemma Suppose $U$ is an operator in $\textstyle{H}$ which is an isometry: $\|U x\|=\|x\|$ $x\in{\mathcal{D}}(U).$ (b)1/ (a)If xe90(U) and $y\in{\mathcal{D}}(U),$ then(Ux $\cdot\,U\!\cdot\!\slash\,\cdot$ is closed,so are the other two. ${\mathcal{R}}(I-U)$ is densein H, hen $I\subset U$ is one-to-one. ${\mathcal{G}}(U)$ (c)If any one of the three spaces ${\mathcal{D}}(U),$ 82(U),und PROOF Any of the identieslised in Exercise 2 of Chapter 12 proves (a) $x=U x.$ Then To prove (b), suppose xe Q(UD and (I- U)x - 0, that is, x,(I- U)y) =(6,y)-(x.Uy)=(Ux, Uy)-kx, Uy)=0UNBOUNDED OrLRAioRs 339 $\textstyle H.$ for every $y\in{\mathcal{D}}(U).$ . Thus $x\perp\mathcal{R}(I-U),$ so that $\scriptstyle x\;=\;0$ if ${\mathcal{R}}(I-U)$ is dense in / The proof of $\left(c\right)$ is letas an exercise 13.19 Theorem Sppose U is the Cayley ranyform of a symmerc operaor ${\boldsymbol{T}}$ in H.Then the following statemenus are rne (a) U is closed if andl only f Tis close (历) $\mathcal{R}(I-U)=\mathcal{D}(T),\,I-U$ is one-to-one, and ${\boldsymbol{T}}$ can ${\mathfrak{b}}e$ reconstructed from $U$ by the formula $$ T=i(I+U)(I-U)^{-1}. $$ (c) GTheCayey transom die symerc operatos are thefore dsticn. V is unitary if and only if Tis sef-adioin. Comersely, Vis a oprator i ${\boldsymbol{H}}$ whic $~i S$ an isometry, ad $H.$ is one-to- one, then Vis he Cayley transform of a symmetric operator in $i f I-V\,i$ 92(T ${+i l}\mathrm{,}$ pRoor By Theorem 13.16, ${\boldsymbol{T}}$ is closed if and only if ${\mathcal{R}}(T+i I)$ is closed. By Lemma 13.18, $U$ is closed if and only if ${\mathcal{D}}(U)$ is closed. Sinc $:{\mathcal{D}}(U)={\mathcal{B}}(T+i I),$ $\operatorname{\mathcal{D}}(U)=$ given by by the definition of the Cayley transform (a is proved. ${\mathcal{D}}(T)$ and The one-to-one corespondencex→2 betwee (1) $$ z=T x+i x,~~~~~~U z=T x-i x $$ can be rewritten in the form (2) $$ (I-U)z=2i x,\qquad(I+U)z-2T x. $$ maps This shows that $I-U$ is one-to-one,that ${\mathcal{R}}(I-U)={\mathcal{D}}(T),$ so that $(I-U)^{-1}$ ${\mathcal{D}}(T)$ onto ${\mathcal{D}}(U),$ and that (3) $$ 2T x=(I+U)z=(I+U)(I-U)^{-1}(2i x)\qquad[x\in{\mathcal{D}}(T)]. $$ This proves $(\partial).$ Assume now that ${\boldsymbol{T}}$ is self-adjoint. Then (4) $$ {\mathcal{R}}(I+T^{2})=H $$ by Theorem 13.13. Since (5) $$ (T+i I)(T-i I)=I+T^{2}=(T-i I)(T+i I) $$ [the three operators $\mathbf{\Psi}(5)$ have domain ${\mathcal{D}}(T^{2})|,{\mathrm{i}}$ t follows from (4) that (6) $$ \mathcal{D}(U)=\mathcal{P}(T+i I)=H $$ and (7) $$ \mathcal{R}(U)=\mathcal{R}(T-i I)=H. $$340 nANACH ALCEBRAs AND serCTRAL THroxv Since $U$ is an isomety G) and (D imply ha $(c),$ assume that is unitary CThcorem 12.13) $U_{\mathit{l}}$ To complete the proof of $U$ is unitary. Then (8) $$ |\mathcal{P}(I-U)|^{\perp}=\mathcal{W}(I-U)=\{0\}, $$ by (b) and the normality of $I-U$ (Theorem 12.12), so that $\mathcal{B}(T)=\mathcal{A}(I-U)$ is dense in $H.$ Thus $T^{\ast\ast}$ is defined, and $T\subset T^{*},$ there exis $y_{0}\in{\mathcal{B}}(T)$ such Fix $y\in{\mathcal{D}}(T^{*}).$ Since ${\mathcal{R}}(T+i I)={\mathcal{D}}(U)=H,$ that (9) ( $$ T^{*}+i I)y=(T+i I)y_{0}=(T^{*}+i I)y_{0}\,. $$ The last equality holds because $T\subset T^{*}.$ If y, =y- yo,then $y_{\mathrm{i}}$ E ${\mathcal{Q}}(T^{*})$ and, for every $x\in{\mathcal{D}}(T)$ (10) $$ ((T-i I)x,\,y_{1})=(x,(T^{*}+i I)y_{1})=(x,0)=0. $$ Thus $y_{1}\perp{\mathcal{R}}(T-i I)={\mathcal{R}}(U)=H,$ and so $y_{1}=0,$ and $y=y_{0}\in{\mathcal{D}}(T).$ to-one correspondence $Z\hookrightarrow X$ and (c) is provcd. and ${\mathcal{R}}(I-V),$ given by Hence $T^{*}\subset T,$ be as in the statement of the converse.。 Then there is a one Finally lt ${\mathbf{}}V$ between ${\mathcal{L}}(V)$ (1) $$ x arrow z\ -y z. $$ Define $\boldsymbol{\mathsf{S}}$ on ${\mathcal{D}}(S)={\mathcal{R}}(I-V)$ by (12) $$ S x=i(z+V z)\qquad{\mathrm{if~}}x=z-V z. $$ 1f $x\in{\mathcal{D}}(S)$ and $y\in{\mathcal{D}}(S)$ then $x=z-y z$ and $y=u-V u$ for some $z\in{\mathcal{D}}(V)$ and $u\in{\mathcal{D}}(V).$ Since ${\mathbf{}}V$ is an isomctry, $\mathbf{i}\mathbf{t}$ now follows from (a) of Lemma 13.18 that ( $$ \begin{array}{c c c}{{\vert3\rangle}}&{{,\qquad}}&{{(S x,y)=i(z+V z,u-V u)-i(z,V u)}}\\ {{}}&{{=(z-V z,i u+i V u)=(x,S y).}}&{{\qquad}}&{{,}}\end{array}\qquad. $$ Hence $\boldsymbol{\mathsf{S}}$ 金 Sis symmetric. Sinc (12) can be witenin the form (14) $$ 2i V z=S x-i x,\qquad2i z=S x+i x\qquad[z\in{\mathcal{D}}(V)], $$ we see tha (15) $$ V(S x+i x)=S x-i x\qquad[x\in{\mathcal{D}}(S)] $$ and that $\mathcal{D}(V)=\mathcal{P}(S+i I)$ 、Therefore L ${\mathbf{}}V$ is the Cayley transform of ${\boldsymbol{\mathbf{S}}}.$ $J/I$ operators $T_{1}$ and $\left.T_{2}\right.$ The defciency indices If $U_{1}$ and $U_{\mathrm{2}}$ are Cayley transforms of symmetri Problems about 13.20 it is clear that $T_{1}\subset T_{2}$ if and only i $U_{1}\subset U_{2}\,.$ problems about extensions of isometries symmetric exensions of symmetrc operators reduc therefore to (usually easienuNnouNprp OPRAToRs 341 defined. is the closure of For examplevev isomet ${\cal{I}}{\cal{I}}$ has a cloed ymmeicexenso and ${\mathcal{R}}(T-i I)$ are closed, and ${\overline{{T}}}.$ ${\boldsymbol{T}}$ in $\textstyle H_{\cdot}$ $13.18)$ and with Cayley transform ${\cal U}.$ ${\mathcal{R}}(I-U_{1})$ ${\boldsymbol{U}}$ extend iqely iocty whose doman is an ${\mathcal{D}}(U).$ Therefore i follows from $\mathbf{\Psi}(a)$ of Theorem 13.19 that every symmeric operator i Let us now consider lse adensiy eie smmetic operaor extension $U_{\mathrm{1}}$ of $U$ Then ${\mathcal{R}}(T+i I)$ 1,so that $I-U_{1}$ is one-to-one(Lemma ${\boldsymbol{U}}$ Since $U_{1}$ J has isomet rin to cecn T e isins thertoionan cm $\textstyle H,$ every isometric ${\mathcal{R}}(I-U)={\mathcal{D}}(T)$ plmet f tse spaes ar cal t defcicyices oTCThe msionoa $H.$ Hilbes sae s b deion.th aiaty an oie is itiom as is now assumed to be dense in dense in $T_{1}$ of is the Cayley tranfom of asmetricextension is closed, symmetric, and densely The foowng trte staens r esy conses or horem 1319 an ${\mathbf{}}T$ the prceding disussin; we sil ssume tha (a)Tis sefadjointi md oly if boh its deficency indies are O its w denyimisaeua ()T is maximally ymeri ad oly fat last one of is deficincy inices is $~i{\mathcal{I}}$ (e) T has sef-adoit extesio i amd ony onto $\textstyle\left[{\mathcal{R}}(T-i I)\right]^{\perp}$ and note that every unitary extension of The pofs of(a and (b) are obvious. To see(ce, use Ge)of Theoem 13.19 must be an isometry of 【L ${\mathcal{A}}(T+i I)]^{\bot}$ $U_{\mathit{l}}$ indices of T $\boldsymbol{\mathit{I}}$ 13.21 Example Let ${\mathbf{}}V$ be the rght shift o and ${\mathcal{R}}(V)$ has codimension 1, the deficiency $\textstyle I-V$ metric operator is one-to-one (Chapter 12, Exercis 8),and so ${\mathcal{E}}^{2}.$ ${\mathcal{V}}$ Yis the Cayley transform of a sym- Then ${\mathbf{}}V$ V is an isometry and $\textstyle T.$ Since ${\mathcal{D}}(V)=\ell^{2}$ are $\mathbf{\hat{\Pi}}$ and 1. closed operator ${\mathbf{}}T$ which is not self-adjoint Tis rovies s wit a xmple o densy ened. maialysmmtr Resolutions of te ldentity $E\colon\mathbb{N} arrow B(H)$ 13.22Notation il now aera s . l b liertsc a will bea resolution of theidntity, wtha the pettsiste Definion 12.17. Theorem 121 describesa symbicacus hasciaest $\mathsf{C V C I I}\subseteq L^{\infty}(E)$ an operator $\Psi(f)\in{\mathcal{B}}(H),$ by the formula (1) $$ (\Psi(f)x,y)=\int_{\Omega}f\,d E_{x,y}\qquad(x\in H,\,y\in H). $$ This wi nowbe extended to unbounded mesrablefunctions (Theorem 13.24 we shall use the same notations as in Definition 12.17342 RANACH ALGERR AS AND SPECTRAL THEORY 13.23 Lemma $L e t f;\Omega\to{\mathcal{C}}$ be measurable. Pu (D) $$ {\mathcal{D}}_{f}=\lbrace x\in H\colon\int_{\Omega}\vert f\vert^{2}\;d E_{x,x}<\infty\rbrace. $$ Then ${\mathcal{Q}}_{f}$ is a dense subspace of $H$ , If x∈ $H$ and $y\in H.$ , then (2) $$ \int_{\Omega}|f\,|\,d\,|\,E_{x,\,y}|\,\leq\,\|\,y\|\biggl\{\int_{\Omega}|f|^{2}\,\,d E_{x,\,x}\biggr\}^{1/2}. $$ I/ f is bounded and $v=\Psi(f)z,$ then (3) $$ d E_{x,v}=\bar{f}\,d E_{x,z}\qquad(x\in H,\,z\in H). $$ PROOF If $z=x+y,$ and o∈ 9D, then l E $$ {\bf\nabla}(\omega)z\|^{2}\leq(\|E(\omega)x\|+\|E(\omega)y\|)^{2}\leq2\|E(\omega)x\|^{2}+2\|E(\omega)y\|^{2} $$ or (4) $$ E_{z,z}(\omega)\leq2E_{x,x}(\omega)+2E_{y,y}(\omega). $$ lt follows that ${\mathcal{Q}}_{f}$ is closed under addition. Scalar multiplication is even easier Thus ${\mathcal{D}}_{f}$ is a subspace of $H.$ $\mathbf{}\omega_{n}$ be the subset of $\mathbb{Q}$ in which $|f|<n.$ If For $n=1,$ 2,3, ..., et xe ${\mathcal{R}}(E(\omega_{n}))$ then (5) $$ E(\omega)x=E(\omega)E(\omega_{n})x=E(\omega\cap\omega_{n})x $$ so that (6) $$ E_{x,\,x}(\omega)=E_{x,\,x}(\omega\cap\omega_{n})\qquad(\omega\subset\Re|), $$ and therefore (7) $$ \int_{\Omega}|f|^{2}\;d E_{x,\,x}=\int_{\omega_{n}}|f|^{2}\;d E_{x,\,x}\leq n^{2}\|x\|^{2}<\infty. $$ Thus implies that $y=1$ lim E(o,)y for every $y\in H$ , so that $\boldsymbol{y}$ the countable additivity of $\omega arrow E(\omega)y$ $\scriptstyle\pi_{f}$ $\mathcal{A}(E(\omega_{n}))\subset\mathcal{D}_{f}\,.$ Since S2 $= (\begin{array}{l}{{\sim}}\\ {{n=1}}\end{array}\omega_{n}\,,$ lies in the closure of Hence ${\mathcal{D}}_{f}$ is dense and fis a bounded measurable function on $\Omega,$ the Radon- If $x\in H,\,y\in H,$ Nikodym theorem [23] shows that there is a measurable function $\boldsymbol{u}$ on $\Omega_{\mathrm{,}}$ with $[u]=1$ , such that (8) $$ u f\,d E_{x,y}=|f|\,d|E_{x,y}|\,. $$ Hence $\mathbf{\nabla}(9)$ Jg $$ |f|\,d|\,E_{x,\,y}|\,=({\mathfrak{T}}^{(}\!^{\prime}\!\!_{(U\!f)x},\,y)\leq\|{\mathcal{\Psi}}(u\!f)x\|\,\|y\|. $$UNBoUNDED OPERATORs 343 By Theorem 12.21, (10) $$ \|\Psi(u f)x\|^{2}=\int_{\Omega}|u f|^{2}\,d E_{x,x}=\int_{\Omega}|f|^{2}\,d E_{x,x}\,. $$ Now $\operatorname{\left(S\right)}$ and (0O give 2) for bounded . The generalcase ollows fom this Finally 3) holds because $$ \begin{array}{r}{\int_{\Omega}g\,d E_{x,\,v}=(\Psi(g)x,\,\Psi(f)z)=(\Psi(g)x,\,\Psi(f)z)}\\ {\cdot}&{{=(\Psi({\bar{f}}g)x,\,z)=(\Psi({\bar{f}}g)x,\,z)=\int_{\Omega}g{\bar{f}}\,d E_{x,\,z}}\end{array} $$ for every bounded measurable ${\mathcal{G}},$ by Theorem 12.21. // 13.24 Theorem Lei $\boldsymbol{E}$ ibe a resolution o the identity, on a set $\Omega.$ in $\textstyle H,$ (a)To every measurable f: $\Omega\to{\mathcal{C}}$ corresponds a densey dfined closed oerator Y() with domain $\mathcal{D}(\Psi(f))=\mathcal{D}_{f},$ which is characterized by () $$ (\Psi(f)x,\,y)=\int_{\Omega}f\,d E_{x,\,y}\qquad(x\in{\mathcal{D}}_{f},\,y\in H) $$ and which satifie (2) $$ \|\Psi(f)x\|^{2}=\int_{\Omega}|f|^{2}\,d E_{x,x}\qquad(x\in{\mathcal{D}}_{f}). $$ (b) Te muinthemhols th llwg orm: amd ar mesrab then (3) $$ \Psi(f)\Psi(g)\subset\Psi(f g)\qquad\mathrm{and}\qquad\mathcal{D}(\Psi(f)\Psi(g))=\mathcal{D}_{g}\cap\mathcal{D}_{f g}\,. $$ Hence $\mathbb{V}^{\prime}(f)\mathbb{V}^{\prime}(g)=\Psi(f g)$ if anl only $i f\,{\mathcal{D}}_{f g}\subset{\mathcal{D}}_{g}$ (c) For every measurable $f\colon\Omega\to{\mathcal{C}}$ (4) $$ \Psi^{\prime}(f)^{*}=\Psi({\tilde{f}}) $$ and (5) $$ \Psi(f)\Psi(f)^{*}=\Psi(|f|^{2})=\Psi(f)^{*}\Psi(f). $$ that PROOr If $x\in{\mathcal{D}}_{f}$ then $y arrow\textstyle\int_{\Omega_{f}}f d E_{x,\,y}$ is a bounded conjugate-linear functional and on $\textstyle H,$ , whose norm is at most (G $|f|^{2}\,d E_{x,\,x}\rangle^{1/2},\;\mathrm{by}\;(2)$ of Lemma 13.23. It follows $y\in H$ that there is a unique element $\Psi(f)x\in H$ that satsfis(I) for every (6) $$ \|\Psi(f)x\|^{2}\leq\int_{\Omega}|f|^{2}\,d E_{x,x}\qquad(x\in{\mathcal{D}}_{f}). $$ The lincarity of Y()on follows from I) sinc $\scriptstyle{E_{x},\gamma}$ is linear inx344 BANACH ALGEBRAS AND SPECTRAL THEORY Associate with each f its truncations f,= 中,, where $\phi_{n}(p)=\operatorname{if}\left|f(p)\right|\leq n,$ $\phi_{n}(p)=0$ if $|f(p)|>n.$ since each $f_{n}$ is bounded, and therefore (G) shows, by Then ${\mathcal{Q}}_{f-f_{n}}={\mathcal{Q}}_{f}\,;$ the dominated convergence theorem, that (7) $$ \|\Psi(f)x-\Psi(f_{n})x\|^{2}\leq\int_{\Omega}|f-f_{n}|^{2}\;d E_{x},\dot{x}\to0\qquad48\;n\to\infty, $$ for every $x\in{\mathcal{D}}_{f}\cdot\ \operatorname{Since}f_{n}$ is bounded,(2) holds with $\ f_{n}$ in place of f(Theorem 12.21). Hence (T) implies that (2) holds as stated. This proves (a),except for the assertion that $\Psi(f)$ is closed. The latter in place follows from Thcorem 13.9 if (4) (to be proved prescntly is appied to $\boldsymbol{\f}$ of f. We turn to the proof of (b) ${\mathcal D}_{;\overline{{{g}}}}\frac{2}{\mathcal V}_{;}\mathcal{D}_{g}$ If Z $\:\in H$ and $v=\Psi(\bar{f})z,$ Assume first that fis bounded. Then Equation (3) of Lemma 13.23 and Theorem 12.21 show that $$ (\Psi(f)\Psi(g)x,\,z)=(\Psi(g)x,\,\Psi(\tilde{f})z)=(\Psi(g)x,\,v)\, $$ $$ .\qquad-\int_{\Omega}g\,d E_{x,v}=\int_{\Omega}f g E_{x,z}=(\Psi(f g)x,z). $$ Hence (8) $$ \Psi(f)\Psi(g)\,x=\Psi(f g)x\qquad(x\in{\mathcal{D}}_{g},f\in L^{\infty}). $$ If $y=\Psi(g)x,$ it follows from (8) and (2) that (9) $$ \int_{\Omega}|f|^{2}\;d E_{y,y}=\int_{\Omega}|f g|^{2}\;d E_{x,x}\;\;\;\;\;\;\;(x\in D_{g},f\in L^{\infty}). $$ $f\in L^{\infty},$ Now let $\boldsymbol{f}$ it holds for all measurable ${\mathcal{J}}.$ Since be arbitrary (possibly unbounded.、Since (9) holds for a consists of all $x\in{\mathcal{D}}_{g}$ ${\mathcal{D}}(\Psi(f)\Psi(g))$ such that $y\in{\mathcal{D}}_{f},$ and since (9) shows that $y\in{\mathcal{D}}_{f}$ if and only if $x\in\emptyset_{f g},$ we see that (10) $$ \mathcal{D}(\Psi(f)\Psi(g))=\mathcal{D}_{g}\cap\mathcal{D}_{f g}\,. $$ If $x\in\tilde{\mathcal{D}}_{g}\cap\tilde{\mathcal{D}}_{f g}\,,$ if $y=\Psi(g)x$ , and if the truncations $f_{n}$ are defined as above, $\operatorname{hen}f_{n}\to f\operatorname{in}\,L^{2}(E_{y\,,\,y}),f_{n}\,g\to f g\ {\operatorname{in}}\,L^{2}(E_{x\,,\,x}).$ , and now (8)(with $f_{n}$ in place of f) and (Q2) imply $$ \Psi^{\prime}(f)\Psi^{\prime}(g)x=\Psi(f)y=\operatorname*{lim}_{n arrow\infty}\Psi(f_{n})y=\operatorname*{lim}_{n arrow\infty}\Psi(f_{n}g)x=\Psi(j g)x. $$ This proves (3) and hence (b). and $y\in{\mathcal{D}}_{f}={\mathcal{D}}_{f}.$ lt follows from(T) and Supposc now that $x\in{\mathcal{D}}_{f}$ Theorem 12.2 that $$ (\Psi(f)x,\,y)=\operatorname*{lim}_{n arrow\infty}(\Psi(f_{n})x,\,y)=\operatorname*{lim}_{n arrow\infty}(x,\,\Psi(\tilde{f_{n}})y)=(x,\,\Psi(\tilde{f})y) $$UNBoUNDED OPERATORs 345 Thus ye9(Y(OD*), and (11) $$ \Psi(\hat{f})\subset\Psi(f)^{*}. $$ ${\mathcal{A}}_{f}$ Fix $Z{\dot{\boldsymbol{z}}}$ put To pass fom (Il) to $(\lambda)$ we have to show that $\mathrm{every}\,\,z\in{\mathcal{D}}(\Psi(f)^{*})$ lies i (12) $v=\Psi(f)^{*}z$ Since $f_{n}=f\phi_{n}\,.$ , the multiplicaton theorem gives $$ \Psi(f_{n})=\Psi(f)\Psi(\phi_{n}). $$ Since $\Psi(\phi_{n})$ is set-ajoint w concude fom Theorems 1.2 nd 12.21 ta Hence $$ \Psi(\phi_{n})\Psi(f)^{*}\subset[\Psi(f)\Psi(\phi_{n})]^{*}=\Psi(f_{n})^{*}=\Psi(\bar{f}_{n}). $$ (13) $|\phi_{n}|\leq1.$ ,(13) and $\mathbf{(2)}$ imply Since $$ \Psi(\phi_{n})v=\Psi(\tilde{f}_{n})z\qquad(n=1,2\,\,\,3,\,\ldots). $$ (14) $$ \int_{\Omega}|\sp{2}f_{n}|\sp2\ d E_{z,z}=\int_{\Omega}|\phi_{n}|\sp2\ d E_{v,v}\leq E_{v,v}(\Omega) $$ for $n=1,2,3,\ldots$ Hence $z\in{\mathcal{D}}_{f}.$ and (4) is proved. ${a\!\!\!/}/{b\!\!\!/}$ theorem, because Finally(S follows from $(4)$ by another application of the multiplication ${\mathcal{D}}_{f f}\subset{\mathcal{D}}_{f}$ Remark If $\mathcal{G}$ is bounded, then $\mathcal{D}_{f s}\subset\mathcal{D}_{g}$ (simply because ${\mathcal{D}}_{g}^{\,\,\cdot}=H)$ so that $\Psi(f)\Psi(g)=\Psi(f g).$ This was uscd in(2. rt aso shows tha (15) $$ \Psi(g)\Psi(f)\subset\Psi(f)\Psi(g), $$ because $\Psi(g)\Psi(f)\subset\Psi(g f)=\Psi(f g).$ If $\scriptstyle{\mathcal{G}}$ is the characteristic function of a mcasurable set w $c\,\Omega.$ (15) becomes (16) $$ E(\omega)\Psi(f)\subset\Psi(f)E(\omega). $$ (17) I ${\textsf{f}}x\in{\mathcal{D}}_{f}\cap{\mathcal{B}}(E(\!(n)),$ it folows that $$ E(\omega)\Psi(f)x=\Psi(f)E(\omega)x=\Psi(f)x. $$ Thus $\mathbf{W}(f)$ maps ${\mathcal{A}}_{f}\cap{\mathcal{R}}(E(o))$ into ${\mathcal{R}}(E(\omega))$ Section 12.27 Thisould oe cupared wit he dsono ivan spaces 13.25 Theorem Im he siaion J Therem 13.24 ${\mathcal{D}}_{f}=H$ if and only ifJe L°(E) PROOF Assume ${\mathcal{D}}_{f}=H.$ Since $\Psi(J)$ is a closed operator, the closed grap ${\mathfrak{o}}\mathbf{f},$ it follows theorem implies tha $\Psi(f)\in{\mathcal{B}}(H)$ $\operatorname{If}f_{n}=f\phi_{n}$ is a truncation from the muluiton heorem ombimed wih Tem 2. h sinc 1.1。= 1(/,)I = I1Y/JY(4如)l≤ IHYGO)1 $J/j$ $\P\Vert\Psi(\phi_{n})\Vert=\Vert\phi_{n}\Vert_{\infty}\leq1.$ Thus flo IYCODl. and fe 1*(E) The converse is contained in Theorem 12.21.346 BANACH ALGEBRAS AND SPECTRAL THEORY 13.26 Definition The resolvent set of a linear operator ${\mathbf{}}T$ in ${\boldsymbol{H}}$ is the set of all $\lambda\in{\mathcal{C}}$ such that $T-\lambda I$ is a one-to-one mapping of ${\mathcal{D}}(T)$ onto $H$ whose inversc belongs to ${\mathcal{B}}(H).$ should have an inverse $S\in{\mathcal{P}}(H),$ . which satisfies In other words $T-\lambda I$ 天 $$ S(T-\lambda I)\subset(T-\lambda I)S=I. $$ lies in the resolvent set of $T^{\bullet}T$ if ${\boldsymbol{T}}$ For instance, Theorem 13.13 states that $-\1$ is densely defined and closed The spectrum $\sigma({\boldsymbol{r}})$ of ${\boldsymbol{T}}$ is the complement of the resolvent set of ${\boldsymbol{T}},$ just as for bounded operators $\sigma(T),$ for unbounded $T,$ $\textstyle17$ to 20. Some properties of are described in Exercises For the next theorem, we refer to Section 12.20 for the definition of the ssential range of a function, with respect to a given resolution of the identity 13.27 Theorem Suppose $\boldsymbol{E}$ is a resolution of the identity on a set $\Omega,f\colon\Omega_{i} arrow C\ i$ is measurable, and $$ o_{\alpha}=\{p\in\Omega\colon f(p)=\alpha\}\qquad(\alpha\in{\mathcal{C}}). $$ (a) f α is in the essential range off and $E(\omega_{x})\neq0.$ , then $\Psi(f)-\alpha I$ is not one-to (b) one. $\|x_{n}\|=1$ i, such tha l, and there exist vectors $x_{n}\in H,$ 1f αis in the essential range of f but $E(\omega_{a})=0,$ then $\Psi(f)-\alpha I$ is a one-to-one mapping of ${\mathcal{Q}}_{f}$ onto a dense proper subspace of $H$ wilh $$ \operatorname*{lim}_{n arrow\infty}[\Psi(f)x_{n}-\alpha x_{n}]=0. $$ (c)c(Y(f)) sthe essenial ange of f case (b). The conclusion of $\mathbf{\nabla}(b)$ ln the terminology used earier for bounded operators, we may say that a ie $\Psi(f)$ in in the point spectrum of $\Psi(f)$ in case (a) and in the contimuous spectrum of is sometimes stated by saying thai ais an approximate eigenvalue $o f\Psi(f)$ $\mathbf{\Psi}(a)$ If characteristic function of $\scriptstyle\sigma_{0}$ We shall asume, without loss of generality, that with $\|x_{0}\|=1$ Let $\phi_{0}$ be the PROOF $\scriptstyle x\;=\;0$ $E(\omega_{0})\neq0,$ there exists $x_{0}\in{\mathcal{R}}(E(\omega_{0}))$ hence $\Psi(f)\Psi(\phi_{0})=0.$ 、by the Then $f\phi_{0}=0,$ multiplication theorem. Since $\Psi(\phi_{0})=E(\omega_{0}),$ it follows tha $$ \Psi(f)x_{0}=\Psi(f){\cal E}(\omega_{0})x_{0}=\Psi(f)\Psi(\phi_{0})x_{0}=0. $$ (b) The hypothesis is now that $E(\omega_{0})=0$ but $E(\omega_{n})\neq0$ for n = 1, 2,3 where $\omega_{n}=\left\{d\in\Omega\colon\left|f(p)\right|<{\frac{1}{n}}\right\}$UNBoUNDED OPERATORs 347 Choose $x_{n}\in{\mathcal{R}}(E(\omega_{n}));$ $\|x_{n}\|=1\cdot$ let $\phi_{n}$ be the characteristic functions of $\omega_{n}.$ Thc argument used in (a) leads to $$ \|\Psi^{\prime}(f)x_{n}\|=\|\Psi(f\phi_{n})x_{n}\|\leq\|\Psi(f\phi_{n})\|=\|f\phi_{n}\|_{\infty}\leq\frac{1}{n}. $$ 1f Thus $\Psi(f)x_{n} arrow0$ although $\|x_{n}\|=1.$ $\Psi(f)x=0$ for some $x\in{\mathcal{D}}_{f},$ then $$ .\qquad\qquad\int_{\Omega}|f|^{2}\,d E_{x,x}=0. $$ Since $|f|>0$ a.e. $[{\mathcal{L}}_{x,x}],$ we must have $E_{x,x}(\Omega)=0.$ But $E_{x,x}(\Omega)=\|x\|^{2}.$ Hence $\Psi(f)$ is one-to-one. ${\mathcal{C}}_{f},$ hence ye 92(Y()*), and $\cdot{\mid\mathcal{M}}(\Psi(f))$ , then $x\to(\Psi(f)x,\,y)$ Likewis $\Psi(f)^{*}=\Psi(f)$ is one-to-one. Ify = 0 is continous in $$ (x,\Psi(f)y)=(\Psi(f)x,\,y)=0\qquad(x\in{\mathcal{D}}_{f}). $$ Thercfore, ${\mathsf{P}}({\hat{f}})y=0,$ and $y=0.$ This proves that ${\mathcal{R}}(\Psi(f))$ is dense in $H.$ Since $\Psi({\mathcal{f}})$ is closed, so is $\Psi(f)^{-1}$ If ${\mathcal{R}}(\Psi(f))$ filled $\textstyle H,$ the closed graph the sequence theorem would imply that $\Psi(f)^{-1}\in{\mathcal{D}}(H)$ But this is impossibe, in view of $\scriptstyle\{x_{n}\}$ constructed above Hence (b) is proved graph theorem. lt follows from $\mathbf{\Psi}(a)$ and (b) that the essential range of fis a subset of which $(c)$ To ouain te ppstiuson ssme is not tessent $\Psi(f)\Psi(g)=\Psi(1)=I,$ by the closed $o(\Psi(f))$ ${\big.}{\mathcal{f}}.$ Then $g=1/f\in L^{\infty}(E),f g=1,$ hence $\Psi(f)^{-1}\in{\mathcal{B}}(H),$ range of proves that ${\mathcal{P}}(\Psi(f))=H$ and therefore that This completes the proof. // The following theorem is sometimes caled the change of measure principl 13.28 Theorem Svppose (c) (a)OP and D'’are o-algebras $i n$ selS $\mathbf{G}$ p and $\Omega^{\prime},$ $\omega^{\prime}\in{\mathfrak{M}}^{\prime}$ (b) $E\colon{\mathfrak{M}}\to{\mathcal{B}}(H)$ is aresolution of the identity, and $\phi^{-1}(\omega^{\prime})\in\mathfrak{O l}\,f o r$ every $\phi\colon\Omega\to\Omega^{\prime}$ has the property that and 1/ $E^{\prime}(\omega^{\prime})=E(\phi^{-1}(\omega^{\prime})),$ then $E^{\prime}:{\mathfrak{g l}}^{\prime}\to{\mathcal{R}}(H)$ is also a resolution of the identity (1) $\int_{\Omega^{\prime}}\!f\,d E^{\prime}{}_{x,y}=\hat{\ \}}_{\Omega}(\bar{f}\circ\phi)\,d E_{x,y}$ for every JDY-measurable f: 9′→C for which either of these integrals exis348 BANACH ALGEBRAS AND sPECTRAL THEOR PR00F For characteristic functions ${\boldsymbol{\jmath}},$ (1) is just the definition of $E^{\prime}$ .Hence (1) holds for simple functions f.The general case follows from this. Thc proof that L $E^{\prime}$ ” is a resolution of the identity is a matter of straightforward verifications and is omitted // The Spectral Theorem 13.29 Normal operators A(not necessarily bounded) linear operator ${\mathbf{}}T$ 'in ${\cal I}_{\scriptscriptstyle I J}$ is said to be normal if ${\mathbf{}}T$ 'is closed and densely defined and if $$ T^{*}T=T I^{*}. $$ Every Y(f) that arises in Theorem 13.24 is normal; this is part of the statement of the theorem. We shall now see, just as in the bounded case discussed in Chapter 12 that all normal operators can be represented in this way, by means of resolutions of the identity on their spectra (Definition 13.26).For self-adjoint operators, this can be deduced very quickly from the unitary case,via the Cayley transform (Theorem 13.30) For normal operators in general, a different proof will be given in Theorem 13.33 13.30 Theorem To every self-adjoint operator A in $\textstyle H$ l corresponds a unique resolution ${\boldsymbol{E}}$ of the identity, on the Borel subsets of the real line, such that (1) $$ (A x,\,y)=\int_{-\infty}^{\infty}t\,d E_{x,\,y}(t)\qquad(x\in\mathcal{D}(A),\,y\in H). $$ Moreover, $\boldsymbol{\mathit{L}}$ is concentrated on $o(A)\subset(-\infty,\,\infty),\,i n$ the sense that $E(\sigma(A))=I.$ As before, this $\boldsymbol{E}$ E will be called the spectral decomposition of $\textstyle A$ PROOF Let ${\boldsymbol{U}}$ / be the Cayley transform of ${\cal A}_{\cdot}$ let $\Omega$ be the unit circle with the point l removed, and let $F^{\prime}$ ’be the spectral decomposition of $U$ (see Theorems 12.23 and 12.26). Since $I-U$ is one-to-one (Theorem 13.19), $E\left(\left|1\right|\right)=0,$ by (b) of Thcorem 12.29, and therefore (2) $$ (U x,y)=\int_{\Omega}\lambda\,d E_{x,y}^{\prime}(\lambda)\qquad(x\in H,\,y\in H). $$ Detine (3) $$ f(\lambda)=\frac{i(1+\lambda)}{1-\lambda}\;\;\;\;\;\;(\lambda\in\Omega), $$ and define $\Psi(J)$ as in Theorem 13.24 with $E^{\prime}$ in place of $\textstyle Z\colon$ (4) $$ (\Psi(f)x,y)=\int_{\Omega}f d E_{x,y}^{\prime}\longrightarrow(x\in{\mathcal{D}}_{f},y\in H). $$UNBOUNDED OFERAToRs 349 Sincc $\boldsymbol{f}$ is real-valued $\Psi(f)$ is sef-djoint Theorem 13.24, and sin $f(\lambda)(1-\lambda)=i(1+\lambda),\mathrm{the~multi}$ pliction theorem give (5) $$ \Psi(f)(I-U)=i(I+U). $$ ln particular,(G) imples that $$ \mathcal{R}(I-U)\subset\mathcal{D}(\Psi(f)). $$ By Theorem 13.19 (6) $$ A(I-U)=i(I+U), $$ and G) shows now that and $\mathcal{B}(A)=\mathcal{B}(I-U)\subset\mathcal{D}(\Psi(f))$ Comparison of $({\mathcal{S}})$ ${\mathcal{A}}.$ By Theorem 13.15, $A=\Psi(f)$ Thus is sl-goi t txeiso hesiaoeriao $\Psi(f)$ (7) $$ (A x,y)=\int_{\Omega}f d E_{x,y}^{\prime}\qquad[x\in{\mathcal D}(A),y\in H]. $$ By Ge) of Theorem 13.27, $\sigma(A)$ is the essential range of ${\boldsymbol{\ f}}.$ Thus $\sigma(\lambda)$ C (-00, o) Note that fis one-to-one in $\underline{{\mathbf{Q}}}$ . If we define (8) $$ E(f(\omega))=E(\omega) $$ for every Borel s $\omega\subset\Omega$ 。 we obtain the desiredrsolutio ${\boldsymbol{E}}$ which converts (T) to (1) Just as $\operatorname{\mathcal{}}(1)$ ya iv fom ) by mens o he y tnsfom.D)ca // the resolutio This completes the proo. be qcrom osegi iece eceaytasrneiniuien $\boldsymbol{E}$ that atifes O of tohostit AGhoem 25 saeo e iueies Thewlmchinened Thom 13 a now e pied o se adjoi portor. h foin tormisaeampeo Sti 13.31 Theorem Len be sel djoin oeator ${\cal{I}}{\cal{I}}$ (b) (a)(Ax, $x)\geq0\,f o r\;e v e r y\;x\in\mathcal{D}(A)$ (briefty: $A\geq0)$ if and only $i f\,\sigma(A)\subset[0,$ oo) $\;U/A\geq0.$ here exis uiqe sef-adoi $\scriptstyle B\geq0$ such that ${\dot{B}}^{2}$ = A. PROOF The proof of $\mathbf{\phi}(a)$ is so simiar o that of Theorem 12.2 that we omit i Assume ${\bar{A}}\geq0,$ so that $\sigma(A)\subset[0,\,\alpha(A)$ D), and (1) $$ (A x,y)=\int_{0}^{\infty}t\,d E_{x,y}(t)\qquad[x\in{\mathcal{D}}(A),y\in H], $$ where ${\mathcal D}(A)=\{x\in H:\int i^{2}\,d E_{x,y}(t)<\infty\};$ the domain of integration is $B=\Psi(s);$ explicitly, Let so e the nonncgative square rooto $\scriptstyle t\,\geq\,0,$ and put $[0,\,\infty),$ $\left(2\right)$ $$ (B x,y)=\int_{0}^{\infty}s(t)\,d E_{x,y}(t)\,\qquad(x\in\mathcal{D}_{s},y\in H). $$350 BANACH ALGEBRAS AND SPECTRAL THEORY $B^{2}=A.$ Since $\boldsymbol{\mathsf{S}}$ is real, $\boldsymbol{B}$ The multiplication theorem ((b) of Theorem 13.24, with $f=g=s,$ shows that is self-adjoint [Cc) of Theorem 13.24], and since $s(t)\geq0,\;(2).$ , with $x=y\colon$ shows that $\scriptstyle B\geq0.$ and $E^{C}$ is To prove uniqueness, suppose ${\boldsymbol{C}}$ is self-adjoint, $C\geq0,\,C^{2}=A,$ its spectral decomposition (3) $$ (C x,y)=\int_{0}^{\infty}\!s\,d E_{x,y}^{C}(s)\qquad(x\in\mathcal{D}(C),y\in H). $$ Apply Theorem 13.28 wit $\textrm{h}\Omega=[0,\,\infty),\,\phi(s)=s^{2},f(t)=t,$ and (4) $$ E^{\prime}(\phi(\omega))=E^{c}(\omega)\qquad{\mathrm{for~}}\omega\subset[0,\infty), $$ to obtain (5) $$ (A x,\,y)=(C^{2}x,\,y)=\int_{0}^{\infty}s^{2}\,d E_{x,\,y}^{C}(s)=\int_{0}^{\infty}t\,d E_{x,\,y}^{\prime}(t). $$ By (1) and (5), theuniqueness statement in Theorem 13.30 shows tha $E^{\prime}=E.$ By (4), ${\boldsymbol{E}}$ determines $E^{C}$ EC, and hence ${\boldsymbol{C}}$ / spectal theorem 13 The following properties of normal operators will be used in the proof of the 13.32 Theorem If ${\cal N}$ is a normal operator in $\textstyle H,$ , then (a) $\mathcal{D}(N)=\mathcal{D}(N^{*}),$ for every xe 9(N), and (6) $\|N x\|=\|N^{\sharp}x\|$ (c) $\textstyle N$ is maximally normal. pRoor If $\nu\in\mathcal{D}(N^{\star}N)=\mathcal{D}(N N^{\star}),$ then(Ny, $N y=(y,N^{*}{\bar{N}}y)$ because Ny ∈ ${\mathcal{D}}(N^{\ast}),$ and(N*y, $N^{*}y)=(y,\,N N^{*}y)$ because $N^{\star}y\in{\mathcal D}(N)$ and $N=N^{\mathrm{s.s}}$ (Theorem 13.12). Sincc $N^{\star}N=N N^{\star},$ it follows that (1) $$ \left\|N y\right\|=\left\|M^{\prime*}y\right\|\quad\quad\mathrm{~if~}y\in\mathcal{D}(N^{\bullet\star}N). $$ Now pick $x\in{\mathcal{D}}(N).$ Let $N^{\prime}$ be the restriction of $\textstyle N$ to ${\mathcal{D}}(N^{*k}N).$ By Theorem 13.13,{x, Nx} ies in the closure of the graph of ${\cal N}^{\prime}.$ Hence there are vectors $y_{i}\in{\mathcal{D}}(N^{i\kappa}N)$ such that (2) $$ \|y_{i}-x\|\to0{\mathrm{~as~}}i\to\infty $$ and (3) $$ \|N y_{i}-N x\|\to0\ \pm s\ i\to\infty. $$ Cauchy sequence in $H.$ Hence there exists $z\in H$ so that (3) implies that $\{N^{*}y_{i}\}$ is a ${\mathrm B y}\,\left(1\right),\,\left|\right|N^{\ast}y_{i}-N^{\ast}y_{j} |=\left| |N y_{i}-N y_{j}\right|,$ such thatUNBouNDeD OPERArous 331 (4) $$ \|N^{\ast}y_{i}-z\| arrow0\mathrm{~4s~}i arrow\infty. $$ Since $N^{\star}$ is a closed operator,(2) and (4) imply tha $x\in\mathcal{D}(N^{\star}),\,\mathrm{s}$ o that $\mathcal{D}(N)\subset\mathcal{D}(N^{*}),$ and From this we conclude first tha $\{x,z\}\in{\mathcal{P}}(N^{*}).$ secondly that (5) $$ \|N^{*}x\|=\|z\|=\operatorname*{lim}\left\|N^{*}y_{i}\right\|=\operatorname*{lim}\left\|N y_{i}\right\|=\|J x\|. $$ (since $N^{\sharp\star}=N),$ , so that Ti . ana o(O or h orhl oet $N^{\ast\ast}$ is also normal (6) $$ \mathcal{D}(N^{*})\subset\mathcal{D}(N^{**})=\mathcal{D}(N). $$ Finally, suppose $\bar{M}$ is normal and $N\subset M$ Then $M^{*}\subset N^{*}.$ so that (7) $$ \mathcal{P}(M)=\mathcal{D}(M^{*})\subset\mathcal{D}(N^{**})=\mathcal{D}(N)\subset\mathcal{D}(M). $$ which gives $\mathcal{D}(M)=\mathcal{D}(N)\,;$ hence $M=N.$ // $\textstyle E,$ 13.33 、Theorem Every normal operator ${\cal N}$ in ${\cal H}$ has a umique spectral decomposition which satisfies (1) $$ (N x,y)=\int_{\sigma(N)}\lambda\,d E_{x,y}(\lambda)\qquad(x\in\mathcal{D}(N),\,y\in H). $$ Moreover $E(\omega)S=S E(\omega)$ for every Borel set o $\textstyle\subset\sigma(N)\;a\!\!\!$ md for every $S\in{\mathcal{B}}(H)$ that commutes with $N_{\cdot}$ in the sense that ${\cal S N}\subset N{\cal S}.$ lt also follows from(I) and Thcorcm 13.24 that $E(\omega)N\subset N E(\omega).$ for every be applied to the operators PRoor Our first objctive is to find selfadjoint projections $B\in{\mathcal{B}}(H)$ and $C\in{\mathcal{B}}(H)$ such that $\ B\geq0.$ orthogonal ranges, such that $P_{i}N\subset N P_{i}\in{\mathcal{B}}(H).$ $N P_{i}$ $\textstyle P_{i}$ , with pairwise $x\in H$ By Theorem 13.13, there exist is normal, an $x=\sum P_{i}x$ $\mathsf{N}P_{i}$ . The spectral theorem for bounded normal operators wi the ;, and this will lead to the desired result. $\|B\|\leq1,$ $C=N B$ , and (2) $$ B(I+N^{*}N)\subset I=(I+N^{*}N)B. $$ Since $N^{\star}N=N N^{\star},\,(2)$ implies (3) $$ {\cal B}N={\cal B}N(I+N^{*}N){\cal B}={\cal B}(I+N^{*}N)N{\cal B}\subset N{\cal B}=C. $$ Consequently, $B C=B(N B)=(B N)B\subset C B.$ Since $\boldsymbol{B}$ and ${\boldsymbol{C}}$ are bounded, it follows that $B C=C B$ and therefore that ${\boldsymbol{C}}$ commules with every bounded Borel function of ${\boldsymbol{B}}.$ (See Section 12.24. Choose t} so that 1=1。>1>1,>, lim 1,=0. Let p be the352 BANACH ALCEBRAS AND sPrCTRAL THroRv Each $f_{i}$ characteristic function of $(t_{i},t_{i-1}),$ for $i=1,2,3,\ldots,$ $\mathbf{0}$ b s not in the point spectrum $f_{i}(t)=p_{i}(t)/t$ ${\boldsymbol{B}}.$ and put is bounded on $\sigma(B)\subset[0,$ 1]. Let $E^{B}$ be the spectral decomposition o The equality (2) shows that $\boldsymbol{B}$ is one-to-one, that is, of $\boldsymbol{B}$ Hence ${\cal E}^{a}(\{0\})=0,$ and $E^{B}$ is concentrated in (0,1] Define (4) $$ P_{i}=p_{i}(B)\qquad(i=1,2,\,3,\,\cdot\cdot). $$ Since $p_{i}p_{j}=0$ if $i\neq j,$ the projections $\textstyle P_{i}$ have mutually orthogonal ranges. Since $\scriptstyle\sum p_{i}$ is the characteristic function of (0,1], we have (5) $$ \sum_{i=1}^{\infty}P_{i}\,x=E^{b}((0,\,1))x=x\qquad(x\in H). $$ Since $p_{i}(t)=i f_{i}(t),$ (6) $$ N P_{i}=N B f_{i}(B)=C J_{i}(B)\in{\mathcal{B}}(H), $$ and $P_{i}\,N=f_{i}(B)B N\subset f_{i}(B)C,$ by (3), so that (7) $$ P_{i}\,N\subset N P_{i}. $$ By $\d\L(6),\,\mathcal D(\Lambda P_{i})={\cal H},$ so that (8) $$ \mathcal{A}(P_{i})\subset\mathcal{D}(N)\qquad(i=1,2,3,\cdot\cdot). $$ Hence, if $P_{i}x=x,$ (7) implies $P_{i}\,N x=N P_{i}\,x=x$ Thus ${\cal N}$ carries ${\mathcal{R}}(P_{i})$ into ${\mathcal{R}}(P_{i}).$ ,Or: ${\mathcal{R}}(P_{i})$ is an invariant subspace of $N.$ By (T) and Theorem Next, we wish to prove that each ${\mathsf{N P}}_{i}$ is normal. 13.2, (9) $$ (N P_{i})^{*}\subset(P_{i}N)^{*}=N^{*}P_{i}\,. $$ But $N P_{i}\in{\mathcal{B}}(H),$ so that $(N P_{i})^{\star}$ has domain $H.$ Hence (10 $$ (N P_{i})^{*}=N^{*}P_{i}, $$ and now Theorem 13.32 shows, by 8) and (10) that (11) $$ \|N P_{i}x\|=\|N^{\ast}P_{i}x\|=\|(N P_{i})^{\ast}x\| $$ By Theorem 12.12,(11 implies tha $({\mathcal{I}})$ show that u fist objective has now ben rcached $N P_{i}$ is normal Hence(5),(6), and By Theorem 12.23,each $N P_{i}$ has a spcctral decomposition $E^{i},$ z', defined on the Borel subsets of ${\cal N}$ carries ${\mathcal{M}}(P_{i})$ into ${\mathcal{R}}(P_{i})_{i}$ $\textstyle P_{i}$ commutes with $N P_{i}$ Therefore $\textstyle P_{i}$ $\textstyle{\mathcal{Q}}$ Since commutes with $E^{i}(o),$ for every Borel set $\omega\subset{\mathcal{C}}.$ ,so that (12) E'o)P,x=P,E'(o)xe M(P) (xe H,i=1,2,3,..UunouNDtp oPERA Tos 353 Since theranes are pais ortonal and inc S impie (13) $$ \sum_{i=1}^{\infty}\|E^{i}(\omega)P_{i}x\|^{2}\leq\sum_{i=1}^{\infty}\|P_{i}x\|^{2}=\|x\|^{2}, $$ the series $\sum E^{i}(\omega)P_{i}x$ converges, in the norm of $\textstyle H_{\cdot}$ and it makes sense to define (14) $$ E(\omega)=\sum_{i=1}^{\infty}E^{i}(\omega)P_{i} $$ frall Borel sets $\omega\subset{\mathcal{C}}$ normal operator lt is easy to check that ${\boldsymbol{E}}$ is a resolution of the identity. Hence there is a $\therefore\varphi_{1{\cdot}}$ , defincd by (15) $$ (M x,{\underline{{y}}})=\int\!\!\!\lambda\,d E_{x,{\underline{{y}}}}(\lambda)\qquad(x\in{\mathcal{D}}(M),\,y\in H), $$ where the domain of integration is ${\mathcal{C}},$ and (16) $$ \mathcal{D}(M)=\left\{x\in H:\int|\lambda|^{2}\;d E_{x,x}(\lambda)<\infty\right\}. $$ For any ourassetin ( wil no b proved by showing tha $M=N$ $x\in H,$ (14) shows that (17) $$ E_{x,x}(\omega)=\|E(\omega)x\|^{2}=\sum_{i=1}^{\infty}\|E^{i}(\omega)P_{i}\,x\|^{2}=\sum_{i=1}^{\infty}E_{x i,x_{i}}^{i}(\omega), $$ where $x_{i}-P_{i}x$ If $x\in\mathcal{D}(N),\,\,\mathrm{then}\,\,P_{i}N x=N P_{i}x,$ so that (18) $$ \sum_{i=1}^{\infty}\ \int|\lambda|^{2}\ d E_{x_{i},x_{i}}^{i}(\lambda)=\sum_{i=1}^{\infty}\|{\cal N}P_{i}x_{i}\|^{2}=\sum_{i=1}^{\infty}\|P_{i}{\cal N}x\|^{2}=\|{\cal N}x\|^{2}. $$ Hence lt follows from $\scriptstyle({\mathcal{W}})$ and 18 hate integal n(1 s inie for ever $x\in{\mathcal{D}}(N).$ (19) $$ \mathcal{A}(N)\subset\mathcal{D}(M). $$ every If $x\in{\mathcal{P}}(P_{i})$ , then $x=P_{i}x,$ and so $E(\omega)x=\bar{E}^{i}(\omega)x;$ thus $E_{x,y}=E_{x,y}^{i}$ for $y\subset H.$ Hencc $$ (N x,y)=(N P_{i}x,y)=\hat{\int}\lambda\,d E_{x,y}(\lambda)=\hat{\int}\lambda\,d E_{x,y}(\lambda)=(M x,y) $$ Consequently (20) $$ P_{i}N x=N P_{i}x=M P_{i}x\qquad[x\in{\mathcal{D}}(N),\,i=1,2,3,\ldots] $$ $\operatorname{If}\,Q_{i}=P_{1}+\cdots+P_{i},$ it follows that $Q_{i}N x=M Q_{i}x.$ Thus (21) {Q,x, Q,Nx}∈ S(M) xe0(N),i=1,2,3,.354 BANACH ALGEBRAS AND SPECTRAL THEORY that $N x=M x$ for every $x\in{\mathcal{D}}(N)$ is closed, it follows from(5) and (21) ihat $\{x,N x\}\in{\mathcal{G}}(M)$ that is Since ${\mathcal{G}}(M)$ Thus $N\subset M,\log(19),$ and now the maximality of N (Theorem 13.32) implies ${\mathit{N}}={\mathit{M}}$ concentrated on $\sigma(N)$ This gives the representation (I), with ${\boldsymbol{C}}$ in place o $\sigma(N).$ That $\boldsymbol{E}$ is actually follows from(c) of Theorem 13.27 To prove the uniqueness of $E_{\mathrm{{s}}}$ consider thc operator (22) $$ T=N(I+\sqrt{N^{*}N})^{-1}, $$ where $\sqrt{N^{*}N}$ is the unique positive square root of $N^{\star}N.$ If (1) holds, it follows from Theorem 13.24 that (23 $$ T=\int\phi\,d E, $$ where $\phi(\lambda)=\lambda/(1+\vert\lambda\vert),$ so that $T\in{\mathcal{B}}(H),$ and since $\phi$ is one-to-one on ${\mathcal{Q}},$ Theorem 13.28 implies that the spectral decomposition $E^{T}$ of ${\boldsymbol{T}}$ r satisfie (24) $$ E(\omega)=E^{T}(\phi(\omega)) $$ for every Borel set $\omega\subset{\bar{C}}.$ The uniqueness of $\boldsymbol{E}$ follows now from that of ${\boldsymbol{E}}^{T}$ (Theorem 12.23) ${\tilde{\omega}}=\{\lambda;|\lambda|<n\},$ and $\;n$ $S\in{\mathcal{B}}(H)$ and ${\cal S H}\subset N{\cal S}.$ Put ${\cal Q}=Q_{n}=E(\tilde{\omega})$ 、 where Finally, assume is some positive integer. Then $N Q\in{\mathcal{B}}(H)$ is normal and is given by (25) $$ N Q=\textstyle\int\!f\,d E, $$ where $f(\lambda)=\lambda$ on ${\tilde{\omega}},f(\lambda)=0$ outside $\tilde{\omega}$ . Theorem 13.28 implies that the spectra decomposition ${\boldsymbol{F}}^{\prime}$ of ${\mathsf{N}}{\mathcal{Q}}$ satisfies $E^{\prime}(\omega)=E(f^{-1}(\omega)).$ or (26) $$ \begin{array}{l}{{\displaystyle{\cal F}^{\prime}(\omega)=E(\omega\cap\tilde{\omega})=Q E(\omega)\mathrm{}\quad\mathrm{if\0}\not{F}\omega,}}\\ {{\displaystyle{\left[E^{\prime}(\{0\})=E(\zeta(\vartheta\}\cup(C-\tilde{\omega}))=E(\zeta(0\})+I-\ Q.}}}\end{array}\right. $$ Hence (27) $$ {\cal E}(\omega)=Q{\cal E}(\omega)=Q{\cal E}(\omega)\qquad\mathrm{if~}\omega<\bar{\omega}. $$ By Theorem 13.24. $Q N\subset N Q=Q N Q$ so that (28) $$ (Q S Q)(N Q)=Q S N Q\subset Q N S Q\subset(N Q)(Q S Q). $$ Since $(Q S Q)(N Q)\in{\mathcal{B}}(H),$ the inclusions in 28) are atually equalities. Now Theorem 12.23 implies that $\sigma S Q$ commutes with every $E(\omega).$UNBoUNDED OPERAToRs 355 Consider a bounded ${\boldsymbol{\omega}}_{;}$ , and take ${\boldsymbol{n}}$ n so large that $\omega\subset{\tilde{\omega}}.$ By (27) $$ {\cal Q S}{\cal E}(\omega)=Q S_{\bar{Q}}G^{\prime}(\omega)={\cal E}^{\prime}(\omega)Q s_{}Q={\cal E}(\omega)S Q. $$ so that (29) $$ Q_{n}S E(\omega)=E(\omega)S Q_{n}\qquad(n=1,2,3,\dots). $$ lt now follows from Proposition 12.18 that (30) $$ S E(\omega)=E(\omega)S $$ if ${\boldsymbol{\omega}}$ is bounded [let $\textstyle n\!\to\!\infty$ in (29)], and hence also if ${\boldsymbol{\omega}}$ ) is any Borel set in ${\mathcal{C}}.$ ${j}/{j}/{j}$ Semigroups of Operators 13.34 Definitins Let $\textstyle X$ be a Banach space,an suppse hat vey 0, co is associated an opcrator $Q(t)\in{\mathcal{B}}(X),$ in such a way that (a) $Q(0)=I.$ (b) $Q(s+t)=Q(s)Q(t)$ for all $s\geq0$ and $\scriptstyle t\,\geq\,0.$ and 1一0 $\|Q(t)x-x\|=0$ for cvcry xe X (c)lim If (a) and $\mathbf{\nabla}(b)$ hold, $\{Q(t)\}$ is caled semiroup (or, mor prcisly oe-param eter semigroup). Such semigroups have cxponential reresentations provided that the mapping $t\to Q(t)$ satisfies some continity assumption. The one that is chosen here, namely Ge), is easy to work with $f(s+t)=f(s)f(t)$ has the for Motivated by the fact that every cotinuous complex function ha satisie and thtfis determined by the number $A=f^{\prime}(0),$ , we associate with $\{Q(\iota)\}$ $\textstyle{1}f(t)=\exp\left(A t\right),$ ,,by the operator $A_{\kappa}.$ (1) $$ A_{\varepsilon}\,x={\frac{1}{\varepsilon}}\,[Q(\varepsilon)x-x]\qquad(x\in X,\varepsilon>0), $$ and define (2) $$ A x=\operatorname*{lim}_{s\to0}A_{e}x $$ in for all $x\in{\mathcal{D}}(A).$ that is, for all x for which the limit $\textstyle X$ and that A is uhus a linear operator of $X.$ lt is clear that ${\mathcal{D}}(A)$ $\left(2\right)$ exis in the norm topology is a subspace of $X.$ the semigroup {Q(t)} This operator, which is essentiall ${\mathcal{Q}}(0),$ is called the infinitesimal generator of356 BANACH ALGEBRAS AND spECTRAL THEoRY 13.35 Theorem I the semiroup {Q() aiesterecedig hpohess th (a)t→ Q0x is a continuous mapping of [0, o) into $X,$ for every $x\in X,$ (b)A is a closed densely defined inear operator in $X,$ (c) for every $x\subset\emptyset(A),$ Q(t)x satisfies the differential equation $$ {\frac{d}{d t}}\,Q(t)x=A\,Q(t)x=Q(t)A x, $$ and (d)for every xe $^{\mathrm{g}}X$ $$ {\cal O}(t)x=\operatorname*{lim}_{\varepsilon arrow0}\left[\exp\,\,(t A_{\varepsilon})\right]x, $$ the comwergence being uniform on every compact subset of [O, o) The limit in $(d)_{i}$ ltsremarkible hat the conclusion d holds for every xe X,not just for $x\in{\mathcal{D}}(A).$ , as well as the one that is implicit in the derivative used in (c), is under- stood to refer to the norm topology of $X.$ PROOF If there were a sequencc $t_{n}\to0$ such that $\|Q(t_{n})\|\to\infty,$ the Banach Steinhaus theorem would imply the existence of an $x\in X$ for which { 1t,)x|} is unbounded. This contradicts our assumption tha (1) $$ \|Q(t)x-x\|\to0\qquad{\mathrm{as~}}t\to0. $$ Hence there exist $\delta>0$ and $\gamma_{0}<\infty$ such that (2) $$ \|Q(t)\|\le\gamma_{0}\qquad\mathrm{i}\Gamma\,0\le t\le\delta. $$ Put $\gamma=\operatorname*{sup}\;\{\|Q(s)\|\,;\,0\leq s\leq1\}$ . By the functional equation (3) $$ Q(s+t)=Q(s)Q(t), $$ (2) implies that $y<\infty.$ Moreover, $r\geq1,$ and (4) $$ \|Q(t)\|\leq\gamma^{1+t}\qquad(0\leq t<\infty), $$ for if r $i\leq t<n+1$ , then $Q(t)=Q(1)^{n}Q(t-n).$ The equality (4) can be applied to and yields $$ \exp\,\left(t A_{\varepsilon}\right)=e^{-t/\varepsilon}\exp\,\left\{_{\varepsilon}^{t}Q(\varepsilon)\right\}=e^{-t/\varepsilon}\sum_{n=0}^{\infty}\,\frac{t^{n}Q(n\varepsilon)}{n!\,\varepsilon^{n}} $$ $$ \left\|\exp\left(t A_{s}\right)\right\|\leq e^{-t/\varepsilon}\sum_{n=0}^{\infty}\frac{t^{n}\gamma^{1+n\varepsilon}}{n!s^{n}}=\gamma\exp\left\{t\frac{\gamma^{\varepsilon}-1}{\varepsilon}\right\}. $$ lf $\scriptstyle0\,<\,\varepsilon\leq1$ , then $\gamma^{\circ}-1\le s(\gamma-1).$ Hence (5) lexp (tA)H ≤yexp ty) (0<8≤1,0≤1<0) After thse prepaions, w turn to the main pat of the proo