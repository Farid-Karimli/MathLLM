2 COMPLETENESS appications) are also given. Th viy o mnyimorterms asdendson h comples the stes ih whichey da. his acouts rthenaequcy ortian numer sem an of he Riemantntra" o mioi sthew besitnwo cxape. n or scess cnued e r faiacmes eieainbesa the Lesgetegal Baircs tet aotu comotemi sisisienecii heceoy erm s h asc o isareai oiet eipasueiheoi playc by theconet o category om heies is cpaeo snstaine Thorm . and . ae sad n ite moreiay aiansuil iecec W hethis s oneimlervesions oreasy frmece usici ionrmo Baire Category 2,1 pfinition Let $\overline{{E}}$ has cmpty interio. The sts o te rt egoy》 ${\boldsymbol{S}}.$ is said to be nowhere are thos dense if its closure $\boldsymbol{\mathsf{S}}$ be a tologcl space. A set $F,\subset\mathbf{\Sigma}S$ $\boldsymbol{S}$ ha coaleuins o nwhestSxy subst $\boldsymbol{\mathsf{S}}$ that s nt o fis cateory is said to be of the second category i This eminogy ducto Bare) amitiy aer bland and unsuestiv42 GENERAL THEORY Meager and nonmeager have been used insead in some texs. But $\;\subseteq\mathbb{S}$ category argu ments are so entrenched in the mathematical literature and are so well known thatit seems pointess to insist on a change Here are some obvious properties of category that will be frly used in the sequcl: (a) If $A\subset B$ and $\boldsymbol{B}$ is of the first category in $S_{s}\,_{t0}$ S $\textstyle A$ (b)Any countable unio of sets of the first category is of the frst category (c) Any closed set $\scriptstyle{E\subset S}$ whose interior is empty is of the first category in $\mathbf{S}.$ (d)1f $\boldsymbol{\mathit{h}}$ is a homeomorphism of $\boldsymbol{\mathsf{S}}$ S onto $\boldsymbol{S}$ and if $E\subset S$ then $\boldsymbol{E}$ E and $\scriptstyle i=E_{\mathrm{r}}$ have the same category in $\mathbf{S}.$ 2.2 Baire's theorem If S is either (a)a complete metric space,on (b)a locally compact Hausdorff space then the inesection of every countble collecion of dense open subsets of $\boldsymbol{\mathsf{S}}$ is dense in $\mathbf{S}.$ This is often called the category theorem, for the following reason complement of ${\widetilde{E}}_{i}\,.$ E, then each is a countable collection of nowhere dense subsets of $\mathbf{S},$ , and if $V_{i}$ ; is the Tf $(E_{\mathrm{{3}}})$ is dense, and the conclusion of Baire's theorem is that $V_{i}$ $\bigcap{V_{i}}\neq\varnothing.$ Hence $S\neq\bigcup E_{i}.$ Therefore, complete metric spaces, as well as locally compact Hausdorff spaces. are of the second category in themselves. PROOF arbitrary nonempty open set in $\mathbf{S}.$ If $\scriptstyle n\geq1$ … arc densc open subscts of $\boldsymbol{\mathsf{S}}$ Let $B_{0}$ be an Suppose $V_{1}.$ $V_{\mathrm{2}}$ $V_{\mathrm{3}}$ and an open $B_{n-1}\neq\varnothing$ has been chosen, then (because $V_{n}$ is dense) there exists an open $B_{n}\neq\emptyset$ with $$ \bar{B}_{n}\subset V_{n}\subset y_{n-1}. $$ In case (a), 年 $B_{n}$ , may be taken to be a ball of radius $<1/n\colon\mathrm{in}\,\mathrm{case}\,(b)$ the choice can be made so that ${\overline{{B_{n}}}}_{n}$ is compact. Put $$ {\cal K}=\bigcap_{n=1}^{\infty}\overline{{{B_{n}}}}\,. $$ In case (a), the centers of the nested balls $\textstyle B_{n}$ form a Cauchy sequence which converges to some point of $K,$ K, and so $K\neq{\mathcal{O}}$ In case (b),K≠C by compactness Our construction shows that $K\subset B_{0}$ and $K\subset V_{n}$ for each ${\boldsymbol{\eta}}.$ Hence $B_{0}$ intersects $\left(\ \right){\cal V}_{n}\,.$ $J/J$coMPLETENESs43 Thc Banach-Steinhaus Theorem that $\Lambda(V)\subset W$ 2.3 Equicontinuity Suppose of $\mathbf{0}$ in ${\mathbf{}}Y$ there corresponds a neighborhood $\boldsymbol{\Gamma}$ is equicontinvous if to is a colletof linearmapings from $\textstyle X$ into ${\boldsymbol{Y}}.$ are topological vector spaces and of O in $\textstyle X$ such for all $\Lambda\in\Gamma.$ $\textstyle X$ and $\boldsymbol{\mathit{I}}$ We say that ${\cal V}_{\circ}$ ${\boldsymbol{\Gamma}}$ every neighborhood $\textstyle|\mathcal{N}$ If ${\boldsymbol{\Gamma}}$ contains only one A equicontinuity is, of course, the same as continuity (Theorem .7. We alrcady saw (Theorem 1.32) hat oninusinarmapimgsa bounded. Euconinus olectons havethisboundenes pety nanuifor manner (Theorem 2.4. 1 s or thsreason that the Ranach-Steihas theorem(2.5 is often rfredto as the miform boudes rincipe ${\cal{Y}}$ 2.4 Theorem Suppose ${\mathbf{}}F$ such that $\Lambda(E)\subset F/o$ r every $\Lambda\in\Gamma.$ inuous olleio linermppings rom Xino ${\cal{Y}},$ and are topological vector spaces, T is an equicon- $\textstyle X$ and $\boldsymbol{\mathit{I}}$ has a boumded subset $\boldsymbol{E}$ isa bumdsbse o X. Then of PRoor Let ${\boldsymbol{V}}.$ Since ${\boldsymbol{\Gamma}}$ is equicontinuous, there is a neighborhood for $\Lambda\in\Gamma.$ Let ${\mathcal W}$ be a neighborhood such that ${\mathbf{}}F$ be the union of the sets $\Lambda(E),$ $E\subset t V$ for all suffciently $X$ ${\boldsymbol{0}}$ in for all A F. Since ${\boldsymbol{E}}$ is bounded ${\mathbf{}}V$ of $\mathbf{0}$ O in $\Lambda(V)\subset W$ large t. For these t, $$ \Lambda(E)\subset\Lambda(t V)=t\Lambda(V)\subset t W, $$ so that $F\subset t W.$ Hence ${\mathbf{}}F$ Fis bounded // 0f 2.5 Theorem (Banach-Steinhaus) Suppose $\textstyle X$ and $\mathbf{\Sigma}Y$ are topological vector is the set $a/l\ x\in X$ whose orbis spaces, T is colction o coninus linearmppigs /om $\textstyle X$ into ${\cal{Y}},$ and $\boldsymbol{B}$ $$ \Gamma(x)=\{\Lambda x:\Lambda\in\Gamma\} $$ are bounded in ${\boldsymbol{Y}}.$ $\boldsymbol{B}$ is of the second category in $X,$ then $B=X$ and I ${\boldsymbol{\Gamma}}$ is equicontimuows Put PROOF Pick balanced neighborhoods ${\mathcal{W}}$ and $U$ fof O in Ysuch that $\bar{U}\vdash\bar{U}\subset W.$ $$ \xi=\bigcap_{\ k\in\bf{i n f}}\Lambda^{-1}(\bar{U}). $$ If $x\in B,$ then $1\left(x\right)\subset n U$ for some ${\boldsymbol{n}},$ so that $x\in n E$ Conscqucntly $$ B\subset\bigcup_{n=1}^{\infty}n E. $$ At least one $\,n E$ is of the second category in $\scriptstyle{X,E}$ is itself o the second category in ${\boldsymbol{B}}.$ Since $X.$ x→nxis a homeomorphism of X onto $X,$ since this is true of44 GENERAL THEORY But x. Then $x-E$ contains a neighborhood ${\mathbf{}}V$ is continuous. Therefore ${\boldsymbol{E}}$ has an interior point $\boldsymbol{E}$ E is closed because each $\Lambda$ V of ( $\mathbf{0}$ in $X,$ , and $$ \Lambda(V)\subset\Lambda x-\Lambda(E)\subset\overline{{{U}}}-\overline{{{U}}}\subset\mathcal{U}\subset\mathcal{W} $$ for every $\Lambda\in\Gamma.$ is equicontinuous. By Theorem $2.4,\,\mathrm{T}$ is uniformly This proves that ${\bf I}$ bounded; in particular, each $\Gamma(x)$ is bounded in ${\cal{Y}}.$ Hence $B=X.$ $J/{\big/}$ In many applications, the hypothesis that $\boldsymbol{B}$ is of the second category is a con- sequence of Baire's theorem. For example, F-spaces are of the second category This gives te following corollary of the Banach-Steinhaus theorem 2.6 Theorem J T is acllection of conimuous linear mappings from an F-space $\textstyle{X}$ into a topological vetor space ${\boldsymbol{V}},$ and $i f$ the sets $$ \Gamma(x)=\{\Lambda x\colon\Lambda\in\Gamma\} $$ are bounded in Y, for every $x\in X,$ then ${\boldsymbol{\Gamma}}$ is equicontinuous. Briefy pointwise boundedness implies uniform boundedness (Theorcm 2.4. As a special case of Theorem 2.6, let $X$ and ${\cal{Y}}$ be Banach spaces, and suppose that (1) $$ \operatorname*{sup}_{\lambda\operatorname{erp}\parallel}\|\Lambda x\|<\infty\qquad{\mathrm{for~every~}}x\in X. $$ The conclusion is that there exists M <o such that (2) $$ \|\land x\|<M $$ Hence (3) lΛx ≤ M|lxl if xe X and $\Lambda\in{\hat{\bf I}}^{*}$ Th following thcorcm cstablishs thc continuity of limits of sequences of continuous linear mappings 2.7 Theorem Suppose $X$ Y and $\boldsymbol{\mathit{I}}$ are topological vector spaces, $a n d\{\Lambda_{n}\}$ is a sequence of continuous linear mappings of $\textstyle X$ into ${\boldsymbol{V}}.$ (a)1/ ${\boldsymbol{C}}$ is the set of all $X\in$ e X for which $\{\Lambda_{n},x\}$ is a Cauchy sequence i Y,and if C $\dot{\iota}{\boldsymbol{S}}$ of the second category in $X,$ then $C=X.$ (b)If Lis the set of al $x\in{\mathcal{X}}$ at which Ax = lim $\Lambda_{n}{\mathcal{X}}$ n→00cowru rNEss45 and A: $X\to Y$ exi、L .g ecom aeoy i $X,$ and if $\bar{Y}$ 'is an F-space, hen $L=X$ is cotinuous $\bar{C}$ Since there is a neighborhod ${\mathbf{}}V$ of $\operatorname{\mathbb{O}}$ $\{\land_{n}\}$ secnas sosssus oad cso 2D nan $X.$ Hence ${\boldsymbol{C}}$ is dense. (OUher- wise, $\overline{{C}}$ ${\boldsymbol{C}}.$ Steinu hermsaet th let ${\mathcal{W}}$ be a neighborhood of oin ${\cal{Y}}.$ Since $[\Lambda_{a}]$ is equcontinuous Fix xre $X{\mathrm{:}}$ is a proper subspace o $X{\dot{\boldsymbol{r}}}$ is equicontinu gnc chesiy th ${\boldsymbol{C}}$ Ssasbace woud be the frtctaeory properspaces have empy ierory h is dese thexis J in $\textstyle X$ such that $\Lambda_{n}(V)\subset W$ for $\scriptstyle n\;=\;1.$ 2,3. $x^{\prime}\in C\cap(x+V).$ If ${\boldsymbol{\pi}}$ and m are so large that A,x'- 人mx'e W, the idenit ( $$ \Lambda_{n}-\Lambda_{m})x=\Lambda_{n}(x-x^{\prime})+(\Lambda_{n}-\Lambda_{m})x^{\prime}+\Lambda_{m}(x^{\prime}-x) $$ that shows that $\forall_{n}\,x\to\Lambda_{m}\,x\in W+\ W+W.$ Consequently $\{\Lambda_{n}x\}$ is a Cauchy // If ${\mathbf{}}V$ sequcnce in $\textstyle Y,$ and $x\in C.$ implies that $L=C.$ Hence $L=X,$ by (a). and (6) The compltcness $\boldsymbol{\mathit{I}}$ valid for al ${\boldsymbol{n}},$ implies now $\Lambda(V)\subset\bar{W}.$ Thus are as above, the inclusio $\Lambda_{n}(V)\subset W,$ ${\mathcal{W}}$ $\Lambda$ is continuous an easily remembered verson Tyseso oe n otisnsxn nre X imo tologica eo sac ${\cal Y},$ , and -mo/ymw mzonm n $~i f$ $$ \Lambda x=\operatorname*{lim}_{n arrow\infty}\Lambda_{n}x $$ exists or ever $x\in X,$ then A $\dot{\boldsymbol{k}}$ continuous ${\mathbf{}}V$ continuous Theorem 2.6 implies tha we have $\Lambda_{n}(V)\subset W$ forall n fo somenihboho $\mathcal{W}$ is a PROOF in $X.$ It follows that $\Lambda(V)\subset{\bar{W}};$ hence (being obviousy linar $\Lambda$ is of neighborhood ofOin $\{\lambda_{\lambda}\}$ is equcontinuos. Thefore $\textstyle Y,$ ${\mathfrak{O}}$ ${\slash/}{\slash/}\quad.$ enters e n esnita way Eeies lie susus onossuneoen cesnonaum enogesosos amioaosmsmsesesosl ns46 GENHERAL THEORY 2.9 Theorem Swppose X anl $\boldsymbol{\mathit{I}}$ are topological vector spaces, $\textstyle K$ is a compact convex set in X, ${\bf I}$ is collection of continuows linear mapings of $\textstyle X$ into Y, and the orbits $$ \Gamma(x)=\{\Lambda x\colon\Lambda\in\Upsilon\} $$ are bounded subsets of Y,for ever $\textstyle{\mathcal{X}}\in K.$ such tha $\Lambda(K)\subset B f o r\ e v e r y\wedge\in\Gamma.$ Then there is a bounded set $B\subset Y$ PRO0F Let $\boldsymbol{B}$ be the union of all sets $\Gamma(x),$ for $x\in K$ Pick balanced neighbor hoods $\textstyle W$ and $U$ of $\mathbf{0}$ in ${\cal{Y}}$ such that $\bar{U}\vdash\bar{U}\subset W.$ Pu (1) $$ E=\bigcap_{\Lambda\in\Upsilon}\Lambda^{-1}({\bar{U}}). $$ I $\textsf{f}x\in K,$ then $\Gamma(x)\subset n U$ for some ${\boldsymbol{n}}_{\mathrm{{,}}}$ so that $x\in n E$ Consequently (2) $$ K=\bigcup_{n=1}^{\infty}(K\cap n E). $$ Since ${\boldsymbol{E}}$ is closed, Bair's theorem shows that $K\cap n E$ has noncmpty interior (relative to $K_{\mathrm{\scriptsize{S}}}$ for at least one ${\boldsymbol{n}}.$ of $K\cap n E.$ we fix a balanced We fix such an ${\boldsymbol{n}}_{\cdot}$ , we fix an interior point $X_{\mathrm{O}}$ neighborhood ${\mathbf{}}V$ of O in $\textstyle X$ such that (3) $$ K\cap(x_{0}+V)\subset n E, $$ and we fix ${\textrm{a}}p>1$ such that (4) $$ K\subset x_{0}+p V. $$ Such a $D\!\!\!\!/$ p exists since ${\cal K}$ is compact and If now $\textstyle{\mathcal{X}}$ ;is any point of $\textstyle K$ (5) $$ z=(1-p^{-1})x_{0}+p^{-1}x, $$ then $z\in K.$ since $K$ is convex. Also, (6) $$ z-x_{\mathrm{o}}=p^{-1}(x-x_{\mathrm{o}})\in V, $$ by (4) Hence zenE, by (3). Since $\Lambda(n E)\subset n\bar{U}$ for every $\mathrm{A}\subset\Gamma$ and since $x=p z-(p-1)x_{0}\,,$ we havc $$ \Lambda x\in p n\overline{{{U}}}-(p-1)n\overline{{{U}}}\subset p n(\overline{{{U}}}+\overrightarrow{U})\subset p n W. $$ Thus $B\subset p n W;$ which proves that $\boldsymbol{B}$ is bounded. // The Open Mapping Theorem 2.10 Open mappings、Suppose f maps $\boldsymbol{\mathsf{S}}$ into $T,$ where $\boldsymbol{\mathsf{S}}$ and ${\boldsymbol{T}}$ are topological spaces. We say that ${\boldsymbol{\mathsf{f}}}$ f is open at a point ${\mathfrak{p}}\in{\mathcal{S}}$ if $f(V)$ contains $^{'}\!\!\ A$ neighborhood ofcoMPLETENEss47 ${\mathcal{I}}(p)$ whenever $U$ is open in ${\boldsymbol{S}}.$ pbonoa ${\boldsymbol{p}}.$ w sy a a o ocmm ont Tis a homeo- whenever ${\mathbf{}}V$ Je morphism pecisey wen sioe "osaec s以.nsoronus Bce $\mathbf{S}$ og ewse…ssm………sa lieuesgso……s mas issseo- hsuosossemsmsaimn9P 2.11 Te pe mpnheoe Suppose (c) A: Xis an F-spac is coniuous and limear, and (a) (6) $X\to Y$ Yis a toical eor spac (d) AX iso eoiaeioy "m" Then () $\Lambda(X)=Y.$ (i)A $\bar{i}S$ an open mapping, and (ii) Y is an F-space $\textstyle X.$ PRoor ,Notha (n imlis OD.、sinc $\mathbf{0}$ in ${\boldsymbol{Y}}.$ $\textstyle X$ that s compatie wihthetopogy o ${\cal{Y}}.$ To prove (ii), let ${\mathbf{}}V$ be a ncighborhood of O n $X.$ is the only open subspace o $\Lambda(V)$ $\boldsymbol{\mathit{I}}$ Lt contains a neighborhood orc We haveto show th Define $\mathcal{A}$ be an nvran tetric (1) $$ V_{n}=\{x;d(x,0)<2^{-n}r\}\qquad(n=0,\;1,\,2,\dots), $$ of where $\scriptstyle r\gg0$ is sml tha $V_{\mathrm{o}}\subset V,$ we wil ro taseneihoho $W_{\phi} arrow V_{\phi}^{ arrow}$ $\mathbf{0}$ in ${\cal{Y}}$ satisfies (2) $$ {\cal W}\subset\overline{{{\Lambda(V_{1})}}}\subset\Lambda(V). $$ Since $V_{1}\Rightarrow V_{2}-V,$ , statement $\mathbf{\nabla}(\partial)$ of Thcorem 1.13 implie (3) $$ \overline{{{\Lambda(V_{1})}}}\to\overline{{{\Lambda(V_{2})-\Lambda(V_{2})}}}\to\overline{{{\Lambda(V_{2})}}}-\overline{{{\Lambda(V_{2})}}}\to\overline{{{\Lambda(V_{2})}}} $$ we can show that $\overline{{\land({\cal{Y}}_{2})}}$ has Th t pa(2、 lelere e prov ${\mathfrak{i f}}$ nonempty interior. But $(\lambda)$ $\Lambda(X)=\bigcup_{k=1}^{\omega}k\Lambda({\cal V}_{2}),$$48$ GENERAL THEORY category in ${\boldsymbol{Y}}.$ Since $y arrow k y$ is a neighborhood of O. Atleast one $k\Lambda(V_{2})$ is thereforc of thc second is of the because ${\mathit{V}}_{2}$ is a homeomorphism of ${\mathbf{}}Y$ onto Y $\gamma,\,\Lambda(V_{2})$ second category in ${\boldsymbol{V}}.$ Its closure therefore has nonempty interior. To prove the second inclusion in (2), fix $y_{1}\in{\overline{{\Lambda(V_{1})}}}$ Assume $\scriptstyle n\geq1$ and ${\boldsymbol{\gamma}}_{n}$ has been chosen in ${\overline{{\Lambda(V_{n})}}}.$ What was just proved for $V_{1}$ holds equally well for $V_{n+1},$ so that $\overline{{\Lambda(V_{n+1})}}$ contains a neighborhood of ${\boldsymbol{0}},$ D. Hence (S) $$ \left(y_{n}-{\overline{{\Lambda(V_{n+1})}}}\right)\cap\Lambda(V_{n})\not=\mathcal{Q}. $$ This says that there exists $x_{n}\in V_{n}$ such that (6) $$ \Lambda x_{n}\in y_{n}-\overline{{{\Lambda(V_{n+1})}}}. $$ Put Since $d(x_{n},0)<2^{-n}r,$ for $n=1,$ $y_{n+1}\in{\overline{{\Lambda(V_{n+1})}}},$ 2, $3,\ \cdot\cdot\cdot\cdot.$ the sums $\chi_{1}\ \vdash\ \cdot\ \cdot\ \vdash\ \chi_{n}$ form a $y_{n+1}=y_{n}-\Lambda x_{n}.$ Then and the construction proceeds. Cauchy sequence which converges (by the completeness of $X_{\mathrm{\bf{\Lambda}}}$ ) to some $x\in{\mathcal{X}}.$ with $d(x,0)<r.$ Hence $x\in V$ Since (7) $$ \sum_{n=1}^{m}\Lambda x_{n}=\sum_{n=1}^{m}(y_{n}-\,y_{n+1})=y_{1}-\,y_{m+1}\,, $$ $\Lambda x\in\Lambda(V).$ Theorem 1.41 shows that $X/N$ (by the continuity of $\Lambda\}_{*}$ 、 we conclude that $\nu,\pi$ and since $y_{m+1}\to0$ as $m\to\infty$ is proved. This gives the second part of (2), and $({\dot{u}}i)$ is an ${\boldsymbol{F}}.$ -space, if ${\boldsymbol{N}}$ is the null space of $\Lambda.$ Hence(ii) will follow as soon as we exhibit an isomorphism $\boldsymbol{\f}$ of $X/N$ onto $\boldsymbol{\mathit{I}}$ which is also a homeomorphism. This can be done by defining (8) $$ f(x+N)=\Lambda x\qquad(x\in X). $$ It is trivial that this f is an isomorphism and that $\Lambda x=f(\pi(x)),$ where $\textstyle\pi$ is the quotient map described in Section 1.40. 1f ${\mathbf{}}V$ is open in Y, then (9) $$ f^{-1}(V)=\pi(\Lambda^{-1}(V)) $$ open in $X/N,$ 1 $\textstyle\pi$ is open. Hence $\boldsymbol{\mathit{f}}$ is continuous. lf $\boldsymbol{E}$ is is open, since A is continuous and then (10) $$ f(E)=\Lambda(\pi^{-1}(E)) $$ is open, sincc $\textstyle\pi$ is continuous and $\Lambda$ is open. Consequently, fis a homeormor- phism. // 2.12 Corollaries (a)If A is a continuous linear mapping of an ${\boldsymbol{F}}.$ -space $\textstyle X$ onto an F-space ${\boldsymbol{Y}},$ then A is openCOMPLETENESSs 49 (b) 人 satisfes ay amul $\boldsymbol{\mathit{I}}$ A…ss……:.2m…m…饼…eoeo $Y\to X$ is cotiuou e)/ X om $\bar{\cal{I}}S$ one-to-one, then $\Lambda^{-\,1}\,;\,$ such that onecxsismanmmo asm $D\!\!\!\!/$ (d) $J/\tau_{1}\subset\tau_{2}$ cne ceolies $\bar{a}$ $$ a\|x\|\leq\|\Lambda x\|\leq b\|x\| $$ amu i o $\iota\left(X,\tau_{1}\right)\,a n d\left(X,\tau_{2}\right)$ for every $\textstyle{\mathcal{X}}$ x∈ X are F-spaces, then wecosao $\textstyle{\mathcal{X}}$ $\tau_{1}\,=\,\tau_{2}\;.$ (D)to te identity mapping o nonosfseosom hre l sanaue ueoen.n $\boldsymbol{\mathit{I}}$ express the continuity o $\Lambda^{-\,1}$ $(X,\tau_{2})$ seneososm mseismema5Ssf sease Stemnt d s obane S apiyn // and of esguesgo… …: ses eiss $\Lambda.$ onto $(X,\tau_{1})$ The Closed Graph Theorem sets 2.13 Graphs f with $U$ h ${\mathit{V}}$ ol siail $X\times Y.$ lf $\textstyle{X}$ Y an $\mathbf{\Sigma},$ are toica spac if all points $X$ and $\boldsymbol{\gamma}$ ars sa $\boldsymbol{\f}$ mas $\textstyle X$ imto $\textstyle Y_{*}$ uhe rayho sthe set $U\times V$ linear mappings betwen ${\boldsymbol{F}}\cdot$ i h catsanprouc and ${\Upsilon},{\star}$ rspctivy.an i: ”、"'is ctiniio $X\times Y$ $(x,f(x))$ and open in $\textstyle{\mathcal{X}}$ to be closed in $X\times Y$ (Proposition 2.14): For PS oes sas one would expect the graph of $\boldsymbol{f}$ spcpisiavlesco ismsfinien中 assrnunuy T s moranaci svei ieom f 2.14 roposition $\textstyle X$ C.pl pac,) s 。las/oy aoce、a $X\to Y$ is comois n e yna $\boldsymbol{\mathit{G}}$ 6/: cbse Since PROOF Let $\mathbb{Q}$ be the complement of ${\boldsymbol{G}}$ in $X\times Y;$ fix $(x_{0}\,,\,y_{0})\in\mathbb{C}.$ Then $y_{0}\neq f(x_{0}).$ Thus $y_{0}$ and ${\mathcal{J}}(x_{0})$ have disoint neighborhoods $U$ such that $f(U)\subset W.$ in Y $\boldsymbol{f}$ is continuous, ${\mathcal{X}}_{0}$ has a neighborhood ${\mathbf{}}V$ and ${\mathcal{N}}$ The neighborhood $U\times V\mathrm{of}\,(x_{0},\,y$ oJ lie therefore in $\mathbb{Q}$ This proves tha $\mathbb{Q}$ is open // graph is the diagona NXi: ne canomth hyothsis th $X\colon$ ,and let is a Hasor sace. T e th consier a arbitry oicani eac ${\cal{Y}}$ be the identity. Its $f\colon X\to X$ The statement $^{\circ}\!D$ is clsed in $X\times X^{\gamma}$ $$ D=\{(x,x)\colon x\in X\}\subset X\times X. $$ axiom is us reworing ot h Husotsparati50 GENERAL THEORY 2.15 The closed graph theorem Suppose (a) $\textstyle X$ and ${\mathbf{}}Y$ are $\boldsymbol{\mathit{f}}$ -spaces, (b) $\operatorname{A}\!:X\to Y$ is linear, (c) $G=\{(x,\operatorname{A}x)\colon x\in X\}$ is closed in $X\times\,Y.$ Then $\Lambda$ is continuous. PRO0F $X\times\,\}$ Y is a vector space if addition and scalar multiplication are defined componentwise: $$ \wp(x_{1},y_{1})+\wp(x_{2},y_{2})=(\circ x_{1}+\wp x_{2},\circ y_{1}+\beta y_{2}). $$ There are complete invariant metrics $d_{X}$ and $d_{Y}$ on $\textstyle X$ and ${\boldsymbol{Y}},$ respectively, which induce their topologies. I1 $$ d((x_{1},y_{1}),(x_{2},y_{2}))=d_{\chi}(x_{1},\,x_{2})+d_{Y}(y_{1},\,y_{2}), $$ then $\ d$ is an invariant metric on $X\times Y$ which is compatible with its produc topology and which makes $X\times Y$ into an ${\boldsymbol{F}}.$ -space. (The easy but tedious verifications that are needed here are left as an exercise.) Since $\Lambda$ is linear, $\complement\flat$ is a subspace of $X\times Y.$ Closed subsets of complete metric spaces are complete. Therefore ${\boldsymbol{G}}$ is an ${\boldsymbol{F}}\cdot$ F-space. Define $\pi_{1}\colon G\to X$ and $\pi_{2}$ $X\times\ Y\lnot Y$ by $$ \pi_{1}(x,\Lambda x)=x,\qquad\pi_{2}(x,y)=y. $$ Now $\pi_{1}$ is a continuous linear one-to-one mapping of the F-space ${\boldsymbol{G}}$ onto the F-space $\textstyle X$ It follows from the open mapping theorem that $$ \pi_{1}^{-1}\colon X\to G $$ is continuous. But $\Lambda=\pi_{2}\circ\pi_{1}^{-1}$ and $\pi_{2}$ is continuous. Hence $\Lambda$ is continuous $l/{\big/}{\big/}$ Remark The crucial hypothesis(C),that $\dot{\boldsymbol{J}}$ is closed,is often verified in applications by showing that $\Lambda$ satisfies propcrty $\left(c^{\prime}\right)$ bclow: (c') $|f|x_{s}\rangle$ is a sequence in $\textstyle X$ such that the limits $$ x=\operatorname*{lim}_{n arrow\infty}x_{n}\qquad\mathrm{and}\qquad y=\operatorname*{lim}_{n arrow\infty}\Lambda x_{n} $$ exist, then $y=\Lambda x$ Let us prove that(c')implies (c). Pick a limit point $(x,y)$ of ${\cal G},$ Since $X\times Y\mathbb{I}$ s metrizable (x,) = lim $(x_{n},$ $\Lambda{x}_{n})$cowrLErrNiss S1 that $x_{n}\to x$ and $\Lambda x_{n}\to y.$ Hence foesose :.uo om eno ouaoe by $\scriptstyle(c)$ , and so $(x,y)\in G,$ and ${\boldsymbol{G}}$ is closed. $y=\Lambda x,$ l is u sy orovethat impli $\scriptstyle(c)$ Bilinear Mapping 2.16 Defnitions Suppose X $\cdot Y,Z$ are vector spaces and $\boldsymbol{B}$ maps $X\times Y$ into $\mathbb{Z}.$ Associate to each $x\in X$ and to cach $y\in Y$ the mappings by defin $$ B_{x}\colon Y\to Z\quad{\mathrm{and}}\quad B^{\psi}:X\to Z $$ $\boldsymbol{B}$ If product topology of $X\times Y)$ then $\boldsymbol{B}$ $$ B_{x}(y)=B(x,y)=B^{y}(x). $$ ${\boldsymbol{B}}^{\nu}$ are linear ${\mathcal{B}}_{x}$ and every $B^{\vee}$ is con- $X_{\mathrm{\bar{z}}}$ ${\cal{Y}},$ $\mathbb{Z}$ is said to bilinear ifevery ${\mathcal{B}}_{x}$ and every B is continuous Grelative to the are tologca etorspaces an iv ${\hat{\mathbf{I}}}{\hat{\mathbf{r}}}\,,$ tinu. heB sa tbsesiiyc ocomi ji .bvusyseratlyconinus i ceti suaun s eoes eowmea is esaeasesmisnseo 2.17 Theorem Suppose B: $X\times\,Y\lnot\,Z$ is bie ad sepatey contios $\textstyle X$ is an T-space, amd $\boldsymbol{\mathit{I}}$ and Z are oca eoiesTh (1) in $\textstyle X$ and $y_{n} arrow y_{\mathrm{u}}$ in Y.I $\boldsymbol{\gamma}$ ismetizble, n fllows ha ${\mathcal{B}}$ is con- whemncver $$ B(x_{n},\,y_{n}) arrow\,B(x_{0}\,,\,y_{0}) $$ in Z tinuous $x_{n} arrow x_{0}$ PRoor Let $U$ and $\mathcal{W}$ be neighborhoods of ${\boldsymbol{0}}$ in $\mathbb{Z}$ such that $U+U\subset W.$ Define b $$ \begin{array}{c l}{{J_{n}(x)=B(x,\,y_{n})}}&{{\quad(x\in X,\,n=1,\,2,\,3,\,\ldots).}}\end{array} $$ Since $\boldsymbol{B}$ is cntinus as a ucion o $$ \operatorname*{lim}_{n arrow\infty}b_{n}(x)=B(x,y_{0})\qquad(x\in X). $$ Thus $\{b_{n}(x)\}$ is a bounded subset o ${\boldsymbol{F}}.$ -space $X,$ the Banach-Steinhas theorem 2.6 $\ b_{n}$ is a con- ${\boldsymbol{X}}$ implies that tinus linear mapping of he $\mathbb{Z},$ for each $x\in X$ Since each of Oin such that $(b_{a})$ is equicontinuous. Hence there is a neighborhood ${\mathbf{}}V$ Note that $$ b_{n}(V)\subset U\qquad(n=1,2,3,\ldots). $$ $$ B(x_{n},y_{n})-B(x_{0}\,,y_{0})=b_{n}(x_{n}-x_{0})+B(x_{0},y_{n}-y_{0}). $$52 GENERAL THEORY If $\;n$ is suficienly large, then $(i)\ x_{n}\in x_{0}+V$ , so tha $:b_{n}(x_{n}-x_{0})\in U,$ and (i) $B(x_{0}\,,\,y_{n}-y_{0})\in U,$ since $\boldsymbol{B}$ is continuous in $\mathbf{\nabla}y$ y and $B(x_{0},0)=0.$ Hence $$ B(x_{n},y_{n})-B(x_{0},y_{0})\in U+U\subset W $$ for all large ${\boldsymbol{n}}.$ This gives(ID and the continuity of $\boldsymbol{B}$ then follows // If ${\cal{Y}}$ is metrizable, so is $X\times Y,$ from(1).(See Appendix A6.) Exercises 2 Put $K= \{{-1},\cdot\cdot$ Hiaiecen nimimison ${\boldsymbol{F}}{\boldsymbol{\epsilon}}.$ is anisnoiaeosisunon orouo $\textstyle X$ is o e fr teoy tslI Prove ). Suppose $\{f_{n}\}$ is a 2 If $\textstyle X$ spac a oualaml bsi has a unique representatin many fnieinona sbspacs,prove that subset of $X\mathrm{:}$ l; define ${\mathcal{Q}}_{K}$ as in Section 1.46 (with ${\boldsymbol{R}}$ in place of ${\boldsymbol{R}}^{n}$ isa maximal iearly independent CA set Bisa ame bas or ctor spac $X{\mathfrak{M}}$ $x\in{\mathcal{X}}$ A1tenaively,βisa Haml asi ever asafie inar combinaion of lets of . sa eamsmcs…… .oososn Asneaisenewehsananas esouos osouo eoewiemstes muycacao to oiog seisiosuo snsetumeiawii si eicteo wocteosousqa seuc ot Cbcseietal fuctonsch tha there is a positive integer $D\!\!\!\!/$ Show that $\mathrm{\A}$ number $$ \Lambda\phi=\operatorname*{lim}_{n arrow\infty}\int_{-1}^{1}f_{n}(t)\phi(t)\,d t\qquad. $$ such tha ${\mathcal{O}}_{K}\,.$ Show that exists for every $\phi\in{\mathcal{D}}_{K}\,,$ isa continuous inr fctionalo and ${\mathcal{Q}}$ $M<\infty$ $$ \left|\int_{-1}^{1}f_{n}(t)\phi(t)\,d t\right|\leq M\vert D^{p}\phi\vert\vert\L_{\infty} $$ for all $\;n$ 1. For example, $\mathrm{i}\Upsilon f_{n}(t)=n^{3}t\;\mathrm{on}\;[-1/n,\,1\}$ 川mland oelsewhesowthat tis cane but not with done with $p=1.$ “Constctanexample whei can be donc wih $\scriptstyle{p-2}$ 4 L $p=1.$ and $L^{2}$ be the usua Lebesuesacs o unieral. rove tha $L^{2}$ is of thc $^{\mathrm{a}}L^{\mathrm{j}}$ first category in $L^{1}$ in three ways: is cloed in $L^{1}$ but has empty interio (a) Show that $\{f\colon$ $|f|^{2}\leq n\rangle$ and show tha (b)Put $\scriptstyle{\theta_{a}=n}$ on 【O $n^{-3}{\big\rfloor},$ $$ \textstyle\bigcap{f g_{n}}\sim0 $$ for every $\scriptstyle{f\in L^{2}}$ but not for every $f\in L^{\cdot}$ is continuous but not onto. e Note thath inclusion map of and $L^{\alpha}\operatorname{if}p<q.$ $L^{1}$ $L^{2}$ into Do the sàme for ${\boldsymbol{L}}^{p}$coMrLETENEss 53 rosusagos to o er o t s op onC $1,2,\ldots\}$ whose norm $\ell^{p}{}_{:}$ , where ${\mathcal{E}}^{p}$ is the Banach space ofl ompe unction $\scriptstyle{\mathcal{X}}$ is fnite $$ ||x||_{p}=\left(\sum_{n=0}^{\infty}\,|x(n)|^{p}\right)^{1/p} $$ 6 Dfne tc Fourececncin $f(n)$ of a function 厂e $L^{2}(T)$ (Tis thc n icleDb $$ .\qquad\qquad f(n)=\frac{1}{2\pi}\int_{-\pi}^{\pi}f(e^{i\theta})e^{-i n\theta}~d\theta $$ for all $n\in\mathbb{Z}$ Gh nters) Put 7 Let Prove that $\{f\in L^{2}(T):\operatorname*{lim}_{n\to\infty}\ \Lambda_{n}f$ $$ \Lambda_{n}f=\ S_{k\subseteq-n}f(k). $$ $\operatorname{each}f\in C(T)$ a functio $\Lambda f\in C(T)$ $\{\gamma_{n}\}\ (n\in Z)$ is a ople squene htacats exiss is a dense subspace o ${\mathcal{L}}(I)$ of the first categor or tso onusoseo oco e unicir ${\boldsymbol{T}}.$ Suppose $c(I)$ whose Foure cefecnst only i hre s complex Borel masue ${\boldsymbol{\mu}}$ $$ (\Lambda f)^{\times}(n)=\gamma_{n}{\hat{f}}(n)\qquad(n\in Z). $$ has his mutipierpery if an CThe ntotois as nEeise Prove th $\{\gamma_{n}\}$ on ${\boldsymbol{T}}$ such that $$ \gamma_{n}\,=\,\int e^{-\,i n\theta}\,d\mu(\theta)\qquad(n\in Z). $$ Suoestion With the supremum norm ${\mathfrak{c}}(T)$ is a Bach spac. Aply the close graph uerem. Thencose e fnicton $$ f{ arrow}(\Lambda f)(1)=\sum_{-\infty}^{\infty}{\gamma}_{n}{\hat{f}}(n) $$ 8 Definc functionals $\mathrm{\A}_{m}$ , on ${\mathcal{L}}^{2}$ mo oeoiesononem n1.13、 eo sism nores tnyo nsnsoceconan ) by (see Exercise ${\boldsymbol{5}})$ Define $x_{n}\in\ell^{2}$ by is compact. Compute $\Lambda_{m}\,x_{n}\,.$ $$ \Lambda_{m}x=\sum_{n=1}^{m}n^{2}x(n)\qquad(m=1,\,2,\,3,\,\dots). $$ Let $K\subset C^{2}$ consist of $0,\,x_{1},\,x_{2}\,,\,x_{3}\,,\,\ldots\,.$ $x\in K$ that $\{\Lambda_{m}.x\}$ is not bounded $x_{n}(n)=1/n,\;x_{n}(i){\bar{\ l}}=0\mathrm{~id~}$ f $i\neq n.$ is boundedfr ac but Prove that $\textstyle K$ Show that $\{\Lambda_{m}.x\}$ A om ioo Theorem 2.9. in the closed cove u O $\textstyle K$ s no ovty niecemo $\left|\right. .\sum{n c_{n}}=\infty.$ Take $x=\sum c_{n}\,x_{n},$ Show that x li $\{\Lambda_{m}\,x_{m}\}$ Choose $c_{n}>0$ so that $\textstyle\sum c_{n}=1,$ ty dfintis icos te oweui an Show tat h convx ul $\textstyle K$ is no close54 GENERAL THEoRx 9 Suppose $X,$ Y, Z are Banach spaces and $$ B\colon X\times Y\to Z $$ is bilinear and continuous. Prove that there exists $M<\infty$ such that $$ |b(x,y)|\leq M\|x\||y\rangle|\qquad(x\in X,y\in Y). $$ Is completeness needed here? $I O$ Prove that a bilinear mapping is continuous if it is continuous at the origin (O,0) ${\cal I}{\cal I}$ Define $B(x_{1},x_{2};y)=(x_{1}y,\,x_{2}y)$ Show that $\boldsymbol{B}$ is a bilinear continuous mapping of $R^{2}\times R$ onto $R^{2}$ which is not open at $(\mathbf{1},\mathbf{1};\mathbf{0}).$ Find all points where this $\boldsymbol{B}$ is open. ${\mathit{1}}{\mathit{2}}$ Let $X$ be the normed space of all real polynomials in one variable, with $$ \|f\|=\int_{0}^{1}\left|f(t)\right|\,d t. $$ Put $B(f,g)=\textstyle\int_{0}^{1}f(t)g(t)\,d t,$ and show that $\boldsymbol{B}$ is a bilinear functional or $\ X\times X$ which is separately continuous but is not continuous. 13 Suppose $X$ is a topological vector space which is of the second category in itself. Let $\textstyle K$ be a closed, convex, absorbing subset of $X,$ Prove that $\textstyle K$ contains a neighborhood of $^{\circ}0$ Suggestion: Show first that $H-K\cap(-K)$ is absorbing. By a category argument, ${\boldsymbol{H}}$ has interior. Then use $$ 2H=H\mid\ H=H\ -H. $$ Show that the result is false without convexity of $K,$ even if $X=R^{2}.$ Show that the result is false if $X:{\bar{L}}^{2}$ topologized by the $L^{1}$ -norm (as in Exercise $4\mathbf{\partial}\lambda.$ 14 (a) Suppose $\textstyle X$ Y and ${\mathbf{}}Y$ are topological vector spaces, $\{\Lambda_{n}\}$ is an equicontinuous sequence of linear mappings of $X$ into ${\boldsymbol{Y}},$ and $C{\mathrm{i}}\cdot$ ${\mathcal{A}}_{\mathrm{I}}^{*}$ at which {A,(x)} is a Cauchy Pis the set of all sequence in Y. Prove that ${\boldsymbol{C}}$ D is a closed subspace of $X.$ (b)Assumc, in addition to the hypotheses of $(a),$ that $\boldsymbol{\mathit{I}}$ is an ${\boldsymbol{F}}^{\sim}$ -space and that {A,(x)} converges in some dense subset of $X,$ .Prove that then $$ \Lambda(x)=\operatorname*{lim}_{n\to\infty}\Lambda_{n}(x) $$ exists for every $x\!\in\!{\mathcal{X}}$ and that $\mathrm{\A}$ is continuous.