\documentclass[10pt]{article}
\usepackage[utf8]{inputenc}
\usepackage[T1]{fontenc}
\usepackage{amsmath}
\usepackage{amsfonts}
\usepackage{amssymb}
\usepackage[version=4]{mhchem}
\usepackage{stmaryrd}

\begin{document}
\section{COMPLETENESS}
The validity of many important theorems of analysis depends on the completeness of the systems with which they deal. This accounts for the inadequacy of the rational number system and of the Riemann integral (to mention just the two best-known examples) and for the success encountered by their replacements, the real numbers and the Lebesgue integral. Baire's theorem about complete metric spaces (often called the category theorem) is the basic tool in this area. In order to emphasize the role played by the concept of category, some theorems of this chapter (for instancè, Theorems 2.7 and 2.11) are stated in a little more generality than is usually needed. When this is done, simpler versions (more easily remembered but sufficient for most applications) are also given.

\section{Baire Category}
2.1 Definition Let $S$ be a topological space. A set $E \subset S$ is said to be nowhere dense if its closure $\bar{E}$ has empty interior. The sets of the first category in $S$ are those that are countable unions of nowhere dense sets. Any subset of $S$ that is not of the first category is said to be of the second category in $S$.

This terminology (due to Baire) is admittedly rather bland and unsuggestive.

Meager and nonmeager have been used instead in some texts. But "category arguments" are so entrenched in the mathematical literature and are so well known that it seems pointless to insist on a change.

Here are some obvious properties of category that will be freely used in the sequel:

(a) If $A \subset B$ and $B$ is of the first category in $S$, so is $A$.

(b) Any countable union of sets of the first category is of the first category.

(c) Any closed set $E \subset S$ whose interior is empty is of the first category in $S$ :

(d) If $h$ is a homeomorphism of $S$ onto $S$ and if $E \subset S$, then $E$ and $h(E)$ have the same category in $S$.

2.2 Baire's theorem If $S$ is either

(a) a complete metric space, or

(b) a locally compact Hausdorff space, ,

then the intersection of every countable collection of dense open subsets of $S$ is dense in $S$.

This is often called the category theorem, for the following reason.

If $\left\{E_{i}\right\}$ is a countable collection of nowhere dense subsets of $S$, and if $V_{i}$ is the complement of $\bar{E}_{i}$, then each $V_{i}$ is dense, and the conclusion of Baire's theorem is that $\bigcap V_{i} \neq \varnothing$. Hence $S \neq \bigcup E_{i}$.

Therefore, complete metric spaces, as well as locally compact Hausdorff spaces, are of the second category in themselves.

PRoOF Suppose $V_{1}, V_{2}, V_{3}, \ldots$ are dense open subsets of $S$. Let $B_{0}$ be an arbitrary nonempty open set in $S$. If $n \geq 1$ and an open $B_{n-1} \neq \varnothing$ has been chosen, then (because $V_{n}$ is dense) there exists an open $B_{n} \neq \varnothing$ with

$$
\bar{B}_{n} \subset V_{n} \cap B_{n-1}
$$

In case $(a), B_{n}$ may be taken to be a ball of radius $<1 / n$; in case $(b)$ the choice can be made so that $\bar{B}_{n}$ is compact. Put

$$
K=\bigcap_{n=1}^{\infty} \overline{B_{n}}
$$

In case $(a)$, the centers of the nested balls $B_{n}$ form a Cauchy sequence which converges to some point of $K$, and so $K \neq \varnothing$. In case $(b), K \neq \varnothing$ by compactness. Our construction shows that $K \subset B_{0}$ and $K \subset V_{n}$ for each $n$. Hence $B_{0}$ intersects $\bigcap V_{n}$.

\section{The Banach-Steinhaus Theorem}
2.3 Equicontinuity Suppose $X$ and $Y$ are topological vector spaces and $\Gamma$ is a collection of linear mappings from $X$ into $Y$. We say that $\Gamma$ is equicontinuous if to every neighborhood $W$ of 0 in $Y$ there corresponds a neighborhood $V$ of 0 in $X$ such that $\Lambda(V) \subset W$ for all $\Lambda \in \Gamma$.

If $\Gamma$ contains only one $\Lambda$, equicontinuity is, of course, the same as continuity (Theorem 1.17). We already saw (Theorem 1.32) that continuous linear mappings are bounded. Equicontinuous collections have this boundedness property in a uniform manner (Theorem 2.4). It is for this reason that the Banach-Stcinhaus theorem (2.5) is often referred to as the uniform boundedness principle.

2.4 Theorem Suppose $X$ and $Y$ are topological vector spaces, $\Gamma$ is an equicontinuous collection of linear mappings from $X$ into $Y$, and $E$ is a bounded subset of $X$. Then $Y$ has a bounded subset $F$ such that $\Lambda(E) \subset F$ for every $\Lambda \in \Gamma$.

PROOF Let $F$ be the union of the sets $\Lambda(E)$, for $\Lambda \in \Gamma$. Let $W$ be a neighborhood of 0 in $Y$. Since $\Gamma$ is equicontinuous, there is a neighborhood $V$ of 0 in $X$ such that $\Lambda(V) \subset W$ for all $\Lambda \in \Gamma$. Since $E$ is bounded, $E \subset t V$ for all suifficiently large $t$. For these $t$,

$$
\Lambda(E) \subset \Lambda(t V)=t \Lambda(V) \subset t W
$$

so that $F \subset t W$. Hence $F$ is bounded.

2.5 Theorem (Banach-Steinhaus) Suppose $X$ and $Y$ are topological vector spaces, $\Gamma$ is a collection of continuous linear mappings from $X$ into $Y$, and $B$ is the set of all $x \in X$ whose orbits

$$
\Gamma(x)=\{\Lambda x: \Lambda \in \Gamma\}
$$

are bounded in $Y$.

If $B$ is of the second category in $X$, then $B=X$ and $\Gamma$ is equicontinuous.

PROOF Pick balanced neighborhoods $W$ and $U$ of 0 in $Y$ such that $\bar{U}+\bar{U} \subset W$. Put

$$
F^{E}=\bigcap_{\Lambda \in \Gamma} \Lambda^{-1}(\bar{U})
$$

If $x \in B$, then $\Gamma(x) \subset n U$ for some $n$, so that $x \in n E$. Consequently,

$$
B \subset \bigcup_{n=1}^{\infty} n E
$$

At least one $n E$ is of the second category in $X$, since this is true of $B$. Since $x \rightarrow n x$ is a homeomorphism of $X$ onto $X, E$ is itself of the second category in $X$.

But $E$ is closed because each $\Lambda$ is continuous. Therefore $E$ has an interior point $x$. Then $x-E$ contains a neighborhood $V$ of 0 in $X$, and

$$
\Lambda(V) \subset \Lambda x-\Lambda(E) \subset \bar{U}-\bar{U} \subset W
$$

for every $\Lambda \in \Gamma$.

This proves that $\Gamma$ is equicontinuous. By Theorem $2.4, \Gamma$ is uniformly

bounded; in particular, each $\Gamma(x)$ is bounded in $Y$. Hence $B=X$. I///

In many applications, the hypothesis that $B$ is of the second category is a consequence of Baire's theorem. For example, $F$-spaces are of the second category. This gives the following corollary of the Banach-Steinhaus theorem:

2.6 Theorem If $\Gamma$ is a collection of continuous linear mappings from an $F$-space $X$ into a topological vector space $Y$, and if the sets

$$
\Gamma(x)=\{\Lambda x: \Lambda \in \Gamma\}
$$

are bounded in $Y$, for every $x \in X$, then $\Gamma$ is equicontinuous.

Briefly, pointwise boundedness implies uniform boundedness (Theorem 2.4.)

As a special case of Theorem 2.6, let $X$ and $Y$ be Banach spaces, and suppose that

$$
\sup _{\Lambda \in \Gamma}\|\Lambda x\|<\infty \quad \text { for every } x \in X \text {. }
$$

The conclusion is that there exists $M<\infty$ such that

$$
\|\Lambda x\| \leq M \quad \text { if }\|x\| \leq 1 \text { and } \Lambda \in \Gamma \text {. }
$$

Hence

$$
\|\Lambda x\| \leq M\|x\| \quad \text { if } x \in X \text { and } \Lambda \in \Gamma \text {. }
$$

The following theorem establishes the continuity of limits of sequences of continuous linear mappings.

2.7 Theorem Suppose $X$ and $Y$ are topological vector spaces, and $\left\{\Lambda_{n}\right\}$ is a sequence of continuous linear mappings of $X$ into $Y$.

(a) If $C$ is the set of all $x \in X$ for which $\left\{\Lambda_{n} x\right\}$ is a Cauchy sequence in $Y$, and if $C$ is of the second category in $X$, then $C=X$.

(b) If $L$ is the set of all $x \in X$ at which

$$
\Lambda x=\lim _{n \rightarrow \infty} \Lambda_{n} x
$$

exists, if $L$ is of the second category in $X$, and if $\dot{Y}$ is an F-space, then $L=X$ and $\Lambda: X \rightarrow Y$ is continuous.

PRoof (a) Since Cauchy sequences are bounded (Section 1.29) the BanachSteinhaus theorem asserts that $\left\{\Lambda_{n}\right\}$ is equicontinuous.

One checks easily that $C$ is a subspace of $X$. Hence $C$ is dense. (Otherwise, $\bar{C}$ is a proper subspace of $X$; proper subspaces have empty interior; thus $\bar{C}$ would be of the first category.)

Fix $x \in X$; let $W$ be a neighborhood of 0 in $Y$. Since $\left\{\Lambda_{n}\right\}$ is equicontinuous, there is a neighborhood $V$ of 0 in $X$ such that $\Lambda_{n}(V) \subset W$ for $n=1,2,3, \ldots$. Since $C$ is dense, there exists $x^{\prime} \in C \cap(x+V)$. If $n$ and $m$ are so large that

$$
\Lambda_{n} x^{\prime}-\Lambda_{m} x^{\prime} \in W
$$

the identity

$$
\left(\Lambda_{n}-\Lambda_{m}\right) x=\Lambda_{n}\left(x-x^{\prime}\right)+\left(\Lambda_{n}-\Lambda_{m}\right) x^{\prime}+\Lambda_{m}\left(x^{\prime}-x\right)
$$

shows that $\Lambda_{n} x-\Lambda_{m} x \in W+W+W$. Consequently, $\left\{\Lambda_{n} x\right\}$ is a Cauchy
sequence in $Y$, and $x \in C$.

(b) The completeness of $Y$ implies that $L=C$. Hence $L=X$, by $(a)$. If $V$ and $W$ are as above, the inclusion $\Lambda_{n}(V) \subset W$, valid for all $n$, implies now that $\Lambda(V) \subset \bar{W}$. Thus $\Lambda$ is continuous. IIII

The hypotheses of $(b)$ of Theorem 2.7 can be modified in various ways. Here is an easily remembered version:

2.8 Theorem If $\left\{\Lambda_{n}\right\}$ is a sequence of continuous linear mappings from an $F$-space $X$ into a topological vector space $Y$, and if

$$
\Lambda x=\lim _{n \rightarrow \infty} \Lambda_{n} x
$$

exists for every $x \in X$, then $\Lambda$ is continuous.

PROOF Theorem 2.6 implies that $\left\{\Lambda_{n}\right\}$ is equicontinuous. Therefore if $W$ is a neighborhood of 0 in $Y$, we have $\Lambda_{n}(V) \subset W$ for all $n$ and for some neighborhood $V$ of 0 in $X$. It follows that $\Lambda(V) \subset \bar{W}$; hence (being obviously linear) $\Lambda$ is
continuous.

In the following variant of the Banach-Steinhaus theorem the category argument is applied to a compact set, rather than to a complete metric one. Convexity also enters here in an essential way (Exercise 8).

2.9 Theorem Suppose $X$ and $Y$ are topological vector spaces, $\bar{K}$ is a compact convex set in $X, \Gamma$ is a collection of continuous linear mappings of $X$ into $Y$, and the orbits

$$
\Gamma(x)=\{\Lambda x: \Lambda \in \Gamma\}
$$

are bounded subsets of $Y$, for every $x \in K$.

Then there is a bounded set $B \subset Y$ such that $\Lambda(K) \subset B$ for every $\Lambda \in \Gamma$.

PROOF Let $B$ be the union of all sets $\Gamma(x)$, for $x \in K$. Pick balanced neighborhoods $W$ and $U$ of 0 in $Y$ such that $\bar{U}+\bar{U} \subset W$. Put

$$
E=\bigcap_{\Lambda \in \Gamma} \Lambda^{-1}(\bar{U})
$$

If $x \in \bar{K}$, then $\Gamma(x) \subset n U$ for some $n$, so that $x \in \tilde{n} E$. Consequently,

$$
K=\bigcup_{n=1}^{\infty}(K \cap n E)
$$

Since $E$ is closed, Baire's theorem shows that $K \cap n E$ has nonempty interior (relative to $K$ ) for at least one $n$.

We fix such an $n$, we fix an interior point $x_{0}$ of $K \cap n E$, we fix a balanced neighborhood $V$ of 0 in $X$ such that

$$
K \cap\left(x_{0}+V\right) \subset n E
$$

and we fix a $p>1$ such that

$$
K \subset x_{0}+p V
$$

Such a $p$ exists since $K$ is compact.

If now $x$ is any point of $K$ and

$$
z=\left(1-p^{-1}\right) x_{0}+p^{-1} x
$$

then $z \in K$, since $K$ is convex. Also,

$$
z-x_{0}=p^{-1}\left(x-x_{0}\right) \in V
$$

by (4). Hence $z \in n E$, by (3). Since $\Lambda(n E) \subset n \bar{U}$ for every $\Lambda \in \Gamma$ and since $x=p z-(p-1) x_{0}$, we have

$$
\Lambda x \in p n \bar{U}-(p-1) n \bar{U} \subset p n(\bar{U}+\bar{U}) \subset p n W .
$$

Thus $B \subset p n W$, which proves that $B$ is bounded.

\section{The Open Mapping Theorem}
2.10 Open mappings 、Suppose $f$ maps $S$ into $T$, where $S$ and $T$ are topological spaces. We say that $f$ is open at a point $p \in S$ if $f(V)$ contains a neighborhood of
$f(p)$ whenever $V$ is a neighborhood of $p$. We say that $f$ is open if $f(U)$ is open in $T$
whenever $U$ is open in $S$. It is clear that $f$ is open if and only if $f$ is open at every point of $S$. Because of the invariance of vector topologies, it follows that a linear mapping of one topological vector space into another is open if and only if it is open at the origin.

Let us also note that a one-to-one continuous mapping $f$ of $S$ onto $T$ is a homeomorphism precisely when $f$ is open.

\subsection{The open mapping theorem Suppose}
(a) $X$ is an $F$-space,

(b) $\bar{Y}$ is a topological vector space,

(c) $\Lambda: X \rightarrow Y$ is continuous and linear, and

(d) $\Lambda(X)$ is of the second category in $Y$.

Then

(i) $\Lambda(X)=Y$,

(ii) ' $\Lambda$ is an open mapping, and

(iii) $Y$ is an $F$-space.

PROOF Note that (ii) implies (i), since $Y$ is the only open subspace of $Y$. To prove (ii), let $V$ be a neighborhood of 0 in $X$. We have to show that $\Lambda(V)$ contains a neighborhood of 0 in $Y$.

$X$. Define

Let $d$ be an invariant metric on $X$ that is compatible with the topology of

$$
V_{n}=\left\{x: d(x, 0)<2^{-n} r\right\} \quad(n=0,1,2, \ldots)
$$

where $r>0$ is so small that $V_{0} \subset V$. We will prove that some neighborhood $W$ of 0 in $Y$ satisfics

$$
W \subset \overline{\Lambda\left(V_{1}\right)} \subset \Lambda(V)
$$

Since $V_{1} \supset V_{2}-V_{2}$, statement $(b)$ of Theorem 1.13 implies

$$
\overline{\Lambda\left(V_{1}\right)} \supset \overline{\Lambda\left(V_{2}\right)-\Lambda\left(V_{2}\right)} \supset \overline{\Lambda\left(V_{2}\right)}-\overline{\Lambda\left(V_{2}\right)}
$$

The first part of (2) will therefore be proved if we can show that $\overline{\Lambda\left(V_{2}\right)}$ has
nonempty interior. But

$$
\Lambda(X)=\bigcup_{k=1}^{\infty} k \hat{\Lambda}\left(V_{2}\right)
$$

because $V_{2}$ is a neighborhood of 0 . At least one $k \Lambda\left(V_{2}\right)$ is therefore of the second category in $Y$. Since $y \rightarrow k y$ is a homeomorphism of $Y$ onto $Y, \Lambda\left(V_{2}\right)$ is of the second category in $Y$. Its closure therefore has nonempty interior.

To prove the second inclusion in (2), fix $y_{1} \in \overline{\Lambda\left(V_{1}\right)}$. Assume $n \geq 1$ and $y_{n}$ has been chosen in $\overline{\Lambda\left(V_{n}\right)}$. What was just proved for $V_{1}$ holds equally well for $V_{n+1}$, so that $\overline{\Lambda\left(V_{n+1}\right)}$ contains a neighborhood of 0 . Hence

$$
\left(y_{n}-\overline{\Lambda\left(V_{n+1}\right)}\right) \cap \Lambda\left(V_{n}\right) \neq \varnothing .
$$

This says that there exists $x_{n} \in V_{n}$ such that

$$
\Lambda x_{n} \in y_{n}-\overline{\Lambda\left(V_{n+1}\right)}
$$

Put $y_{n+1}=y_{n}-\Lambda x_{n}$. Then $y_{n+1} \in \overline{\Lambda\left(V_{n+1}\right)}$, and the construction proceeds.

Since $d\left(x_{n}, 0\right)<2^{-n} r$, for $n=1,2,3, \ldots$, the sums $x_{1}+\cdots+x_{n}$ form a Cauchy sequence which converges (by the completeness of $X$ ) to some $x \in X$, with $d(x, 0)<r$. Hence $x \in V$. Since

$$
\sum_{n=1}^{m} \Lambda x_{n}=\sum_{n=1}^{m}\left(y_{n}-y_{n+1}\right)=y_{1}-y_{m+1}
$$

and since $y_{m+1} \rightarrow 0$ as $m \rightarrow \infty$ (by the continuity of $\Lambda$ ), we conclude that $y_{1}=$ $\Lambda x \in \Lambda(V)$. This gives the second part of (2), and (ii) is proved.

Theorem 1.41 shows that $X / N$ is an $F$-space, if $N$ is the null space of $\Lambda$. Hence (iii) will follow as soon as we exhibit an isomorphism $f$ of $X / N$ onto $Y$ which is also a homeomorphism. This can be done by defining

$$
f(x+N)=\Lambda x \quad(x \in X)
$$

It is trivial that this $f$ is an isomorphism and that $\Lambda x=f(\pi(x))$, where $\pi$ is the quotient map described in Section 1.40. If $V$ is open in $Y$, then

$$
f^{-1}(V)=\pi\left(\Lambda^{-1}(V)\right)
$$

is open, since $\Lambda$ is continuous and $\pi$ is open. Hence $f$ is continuous. If $E$ is open in $X / N$, then

$$
f(E)=\Lambda\left(\pi^{-1}(E)\right)
$$

is open, since $\pi$ is continuous and $\Lambda$ is open. Consequently, $f$ is a homeomorphism.

\subsection{Corollaries}
(a) If $\Lambda$ is a continuous linear mapping of an $F$-space $X$ onto an $F$-space $Y$, then $\Lambda$ is open.
(b) If $\Lambda$ satisfies (a) and is one-to-one, then $\Lambda^{-1}: Y \rightarrow X$ is continuous.

(c) If $X$ and $Y$ are Banach spaces, and if $\Lambda: X \rightarrow Y$ is continuous, linear, one-to-one, and onto, then there exist positive real numbers $a$ and $b$ such that

$$
a\|x\| \leq\|\Lambda x\| \leq b\|x\|
$$

for every $x \in X$.

(d) If $\tau_{1} \subset \tau_{2}$ are vector topologies on a vector space $X$ and if both $\left(X, \tau_{1}\right)$ and $\left(X, \tau_{2}\right)$
are $F$-spaces, then $\tau_{1}=\tau_{2}$.

PROOF Statement (a) follows from Theorem 2.11 and Baire's theorem, since $Y$ is now of the second category in itself. Statement $(b)$ is an immediate consequence of $(a)$, and $(c)$ follows from $(b)$. The two inequalities in $(c)$ simply express the continuity of $\Lambda^{-1}$ and of $\Lambda$. Statement $(d)$ is obtained by applying (b) to the identity mapping of $\left(X, \tau_{2}\right)$ onto $\left(X, \tau_{1}\right)$.

IIII

\section{The Closed Graph Theorem}
2.13 Graphs If $X$ and $Y$ are sets and $f$ maps $X$ into $Y$, the graph of $f$ is the set of all points $(x, f(x))$ in the cartesian product $X \times Y$. If $X$ and $Y$ are topological spaces, if $X \times \dot{Y}$ is given the usual product topology (the smallest topology that contains all sets $U \times V$ with $U$ and $V$ open in $X$ and $Y$, respectively), and if $f: X \rightarrow Y$ is continuous, one would expect the graph of $f$ to be closed in $X \times Y$ (Proposition 2.14): For linear mappings between $F$-spaces this trivial necessary condition is also sufficient to assure continuity. This important fact is proved in Theorem 2.15

2.14 Proposition If $X$ is a topological space, $Y$ is a Hausdorff space, and $f: X \rightarrow Y$ is continuous, then the graph $G$ of $f$ is closed.

PROOF Let $\Omega$ be the complement of $G$ in $X \times Y$; fix $\left(x_{0}, y_{0}\right) \in \Omega$. Then $y_{0} \neq f\left(x_{0}\right)$. Thus $y_{0}$ and $f\left(x_{0}\right)$ have disjoint neighborhoods $V$ and $W$ in $Y$. Since $f$ is continuous, $x_{0}$ has a neighborhood $U$ such that $f(U) \subset W$. The neighborhood $U \times V$ of $\left(x_{0}, y_{0}\right)$ lies therefore in $\Omega$. This proves that $\Omega$ is open.

Note: One cannot omit the hypothesis that $Y$ is a Hausdorff space. To see this, consider an arbitrary topological space $X$, and let $f: X \rightarrow X$ be the identity. Its graph is the diagonal

$$
D=\{(x, x): x \in X\} \subset X \times X
$$

The statement " $D$ is closed in $X \times X$ " is just a rewording of the Hausdorff separation
axiom.

\subsection{The closed graph theorem Suppose}
(a) $X$ and $Y$ are $F$-spaces,

(b) $\Lambda: X \rightarrow Y$ is linear,

(c) $G=\{(x, \Lambda x): x \in X\}$ is closed in $X \times Y$.

Then $\Lambda$ is continuous.

PROOF $X \times Y$ is a vector space if addition and scalar multiplication are defined componentwise:

$$
\alpha\left(x_{1}, y_{1}\right)+\beta\left(x_{2}, y_{2}\right)=\left(\alpha x_{1}+\beta x_{2}, \alpha y_{1}+\beta y_{2}\right)
$$

There are complete invariant metrics $d_{X}$ and $d_{Y}$ on $X$ and $Y$, respectively, which induce their topologies. If

$$
d\left(\left(x_{1}, y_{1}\right),\left(x_{2}, y_{2}\right)\right)=d_{X}\left(x_{1}, x_{2}\right)+d_{Y}\left(y_{1}, y_{2}\right)
$$

then $d$ is an invariant metric on $X \times Y$ which is compatible with its product topology and which makes $X \times Y$ into an $F$-space. (The easy but tedious verifications that are needed here are left as an exercise.)

Since $\Lambda$ is linear, $G$ is a subspace of $X \times Y$. Closed subsets of complete metric spaces are complete. Therefore $G$ is an $F$-space.

Define $\pi_{1}: G \rightarrow X$ and $\pi_{2}: X \times Y \rightarrow Y$ by

$$
\pi_{1}(x, \Lambda x)=x, \quad \pi_{2}(x, y)=y .
$$

Now $\pi_{1}$ is a continuous linear one-to-one mapping of the $F$-space $G$ onto the $F$-space $X$. It follows from the open mapping theorem that

$$
\pi_{1}^{-1}: X \rightarrow G
$$

is continuous. But $\Lambda=\pi_{2} \circ \pi_{1}^{-1}$ and $\pi_{2}$ is continuous. Hence $\Lambda$ is continuous.

Remark The crucial hypothesis $(c)$, that $G$ is closed, is often verified in applications by showing that $\Lambda$ satisfies property $\left(c^{\prime}\right)$ below:

(c') If $\left\{x_{n}\right\}$ is a sequence in $X$ such that the limits

$$
x=\lim _{n \rightarrow \infty} x_{n} \quad \text { and } \quad y=\lim _{n \rightarrow \infty} \Lambda x_{n}
$$

exist, then $y=\Lambda x$.

Let us prove that $\left(c^{\prime}\right)$ implies $(c)$. Pick a limit point $(x, y)$ of $G$. Since $X \times Y$ is metrizable,

$$
(x, y)=\lim _{n \rightarrow \infty}\left(x_{n}, \Lambda x_{n}\right)
$$

for some-sequence $\left\{x_{n}\right\}$. It follows from the definition of the product topology that $x_{n} \rightarrow x$ and $\Lambda x_{n} \rightarrow y$. Herice $y=\Lambda x$, by $\left(c^{\prime}\right)$, and so $(x, y) \in G$, and $G$ is
closed.

It is just as easy to prove that $(c)$ implies $\left(c^{\prime}\right)$.

\section{Bilinear Mappings}
2.16 Definitions Suppose $X, Y, Z$ are vector spaces and $B$ maps $X \times Y$ into $Z$. Associate to each $x \in X$ and to each $y \in Y$ the mappings

by defining

$$
B_{x}: Y \rightarrow Z \text { and } B^{y}: X \rightarrow Z
$$

$$
B_{x}(y)=B(x, y)=B^{y}(x) .
$$

$B$ is said to be bilinear if every $B_{x}$ and every $B^{y}$ are linear.

If $X, Y, Z$ are topological vector spaces and if every $B_{x}$ and every $B^{y}$ is continuous, then $B$ is said to be separately continuous. If $B$ is continuous (relative to the product topology of $X \times Y$ ) then $B$ is obviously separately continuous. In certain situations, the converse can be proved with the aid of the Banach-Steinhaus theorem.

2.17 Theorem Suppose $B: X \times Y \rightarrow Z$ is bilinear and separately continuous, $X$ is an $F$-space, and $Y$ and $Z$ are topological vector spaces. Then

$$
B\left(x_{n}, y_{n}\right) \rightarrow B\left(x_{0}, y_{0}\right) \text { in } Z
$$

whenever $x_{n} \rightarrow x_{0}$ in $X$ and $y_{n} \rightarrow y_{0}$ in $Y$. If $Y$ is metrizable, it follows that $B$ is continuous.

PROOF Let $U$ and $W$ be neighborhoods of 0 in $Z$ such that $U+U \subset W$. Define

$$
b_{n}(x)=B\left(x, y_{n}\right) \quad(x \in X, n=1,2,3, \ldots) .
$$

Since $B$ is continuous as a function of $y$,

$$
\lim _{n \rightarrow \infty} b_{n}(x)=B\left(x, y_{0}\right) \quad(x \in X)
$$

Thus $\left\{b_{n}(x)\right\}$ is a bounded subset of $Z$, for each $x \in X$. Since each $b_{n}$ is a continuous linear mapping of the $F$-space $X$, the Banach-Steinhaus theorem 2.6 implies that $\left\{b_{n}\right\}$ is equicontinuous. Hence there is a neighborhood $V$ of 0 in $X$ such that

Note that

$$
b_{n}(V) \subset U \quad(n=1,2,3, \ldots)
$$

$$
B\left(x_{n}, y_{n}\right)-B\left(x_{0}, y_{0}\right)=b_{n}\left(x_{n}-x_{0}\right)+B\left(x_{0}, y_{n}-y_{0}\right)
$$

If $n$ is sufficiently large, then $(i) x_{n} \in x_{0}+V$, so that $b_{n}\left(x_{n}-x_{0}\right) \in U$, and (ii) $B\left(x_{0}, y_{n}-y_{0}\right) \in U$, since $B$ is continuous in $y$ and $B\left(x_{0}, 0\right)=0$. Hence

$$
B\left(x_{n}, y_{n}\right)-B\left(x_{0}, y_{0}\right) \in U+U \subset \bar{W}
$$

for all large $n$. This gives (1).

If $Y$ is metrizable, so is $X \times Y$, and the continuity of $B$ then follows from (1). (See Appendix A6.)


\end{document}