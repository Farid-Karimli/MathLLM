6 TEST FUNCTIONS AND DISTRIBUTIONS Introduction 6.1 Thether o ditrbtons ees iferetaculus fom certanifiuies ta arise casnifertileunctonscis tTis s one veini titacia of ojet l diuios o eri fnciosywch s mcniaiththnh clas ofifetiable functios to which calcusapplesi t oinanifor $\textstyle R^{n}\!\cdot$ useful; oursetting is some open subset or Hcre are some features that any such extension ought to have inorder to be (d) QEvery continuous function should be a distributin $_{i}\frac{\chi\tilde{\mathrm{Z}}^{*}\,\tilde{\mathrm{z}}}{\vdots}$ (c) the usual limit processes. 6の)Fper distiutin should havepatativivsihae andititos For diffetiable functions, the new notion of dervative shudconcie Wi the old one.(Every distribution should thereforebe initey ifentableD The usual formal ruls of calculus should hoid Theresul spyo cove toromstat sadequat or hanli the case $\scriptstyle n\;=\;1.$ To ot teitns come e temriyrstu tion The integalt flowaretaken with espet o Lcbegemasure and thyexend over thewhole line 、 unest hecotary s nicae136 DISTRIBUIONS AND FOURIER TRANSFORMS 广天 $|f|<\infty$ A complex function $\boldsymbol{f}$ is said to be locally integrable if $\Xi\Xi$ test function”p, rather than as being something that assigns the number $f(x)$ The idea is to reinterpret $\boldsymbol{\mathit{f}}$ is measurable and for every compact $K\subset R.$ to each $x\in R$ $\boldsymbol{\mathit{f}}$ as being something that assigns the number $\textstyle\bigcap{f\phi}$ to every suitably chosen .(This point of view is particularly appropriate for functions that arise in physics since measured quantitie are almost always averages. In fact,distributions were used by physicists long before their mathematical thery was constructcd.)Of course, a well-chosn class of tes functions must be specified Then We let ${\mathcal{D}}={\mathcal{D}}(R)$ exists for every locally inegrable $\boldsymbol{f}$ and for every $\phi\in{\mathcal{D}}.$ whose support is compact is $\vert{j\phi}$ be the vector space of all $\phi\in C^{\infty}(R)$ Moreover, ${\mathcal{D}}$ sufficetly large to assure that fis determined (a.e.) by the integrals (Jf中。(To se ths. note that the uniform closure of ${\mathcal{D}}$ contains every continuous function with compact support))If f happens to be continuously differentiable,then (1) $$ \textstyle\int\!f^{\prime}\phi=-\,\int\!f\phi^{\prime}\,\,\,\,\,\,\,\,\,\,\,(\phi\in\mathcal{D}). $$ If f ∈ CW(R) then (2) $$ \begin{array}{c c c}{{\int f^{(k)}\phi=(-1)^{k}\int f\phi^{(k)}\qquad(\phi\in\mathcal{D},\,k=1,\,2,\,3,\,\ldots).}}\end{array} $$ The compactness of the support of $\phi$ was used in these integrations by parts ) make sense whether fis Observe that the integrals on the right sides of(I) andC $(2)$ differentiable or nol and that they define linear functionals on ${\mathcal{D}}.$ $f^{(n)}$ is the linear functional on We can therefore assign a“kth derivative” to everyftat is locally intgrable: to $(-1)^{k}\,\vert\,f\phi^{(k)}.$ Note that $\boldsymbol{f}$ itsel ${\mathcal{D}}$ that sends $\phi~~~~~~~~~~~~~~~~~~~~~~~~~~~~~~~~~~~~~~~~~~~~~~~~~~~~~~~~~~~~~~~~~~~~~~~~~~~~~~~~~~~~~~~~~~~~~~~~~~~~~~~~~~~~~~~~~~~~~~~~~~~~~~~~~~~~~~~~~~~~~~~~~~~~~~~~~~~~~~~~~~~~~~~~~~~~~~~~~~~~~~~~~~~~~~~~~~~~~~~~~~~~~~~~~~~~~~~~~~~~~~~~~~~~~~~~~~~~~~~~~~~~~~~~~~~~~~~~~~~~~~~~~~~~~~$ corresponds to the functional $\phi arrow\textstyle\int f\phi.$ that are continuous with The distributions will be those linear functionals on ${\mathcal{D}}$ respect to a certain topology.(See Definition 6.7.) The preceding discussion suggests that we associate to each distribution $\Lambda$ its “derivative” $\Lambda^{\prime}$ by the formula (3) $$ \Lambda^{\prime}(\phi)=-\Lambda(\phi^{\prime})\;\;\;\;\;\;\;\;(\phi\in{\mathcal{D}}). $$ ltturns out that this definition (when extended ton variables)has all th desirable properties that were listed earlier. One of the most important features of th resulting theory is that it makes it possible to apply Fourier transform techniques to many problems in partial differential equations where this cannot be done by moreclassca methods. Test Function Spaces ${\mathcal{Q}}_{K},$ 6.2 The space 9(2) Consider a nonempty open set $\Omega\subset R^{n}.$ For each compact $K\subset\Omega,$ as K ranges over all compact subsets of $\Omega,$ was described in Scction 1.46. The union of the spaces the Fréchet spacc ${\mathcal{D}}_{K}$ , is the test function space 9(Q). lt isTEST FUNCrIONS AND DISTRIBUTIONS 137 clear that 9(QD) is a vector space, with respect to the usual definitions of addition and $\phi\in C^{\infty}(\Omega)$ scalar mutiplication ofcomplex functions. Explicitly, pe 9V(&2)if and only t and thc support of $\phi$ is a compact subset of $\Omega$ Let us introduce the norms (1) $$ \left\|\phi\right\|_{N}=\operatorname*{max}\left\{\left|D^{\alpha}\phi(x)\right|:x\in\Omega,\ |\alpha\right|\leq N \}, $$ for $\phi\in{\mathcal{D}}(\Omega)$ and $N=0,\;1,\;2,\;\cdot,$ see Section 1.46 for the notations $D^{x}$ and lcl on The restrictions of these norms to any fixed ${\mathcal{D}}_{K}\subset{\mathcal{D}}(\Omega)$ induce the same topology $K$ corre- ${\mathcal{D}}_{K}$ as do the seminorms ${\mathcal{P}}_{N}$ g/ Section 1.46. To see this, note that to each if sponds an integer $\textstyle{N_{0}}$ such that $K\subset K_{N}$ for all $N\geq N_{0}\,.$ For these $N,$ $|\phi||_{N}=p_{N}(\phi)$ $\phi\in{\mathcal{D}}_{K}$ .Since (2) $$ \|\phi\|_{N}\leq\|\phi\|_{N+1}\qquad\mathrm{and}\qquad p_{N}(\phi)\leq p_{N+1}(\phi), $$ the topologies induced by either sequence of seminorms are unchanged if we let $\textstyle N$ start at $N_{\mathrm{0}}$ rather than at I. These two topologies of ${\mathcal{D}}_{\bar{K}}$ coincide therefore; a local base is formed by the sets (3) $$ V_{N}=\left\{\phi\in{\mathcal{D}}_{K^{*}}\left\vert\right\vert\phi\right\vert_{N}<{\frac{1}{N}} \}\qquad(N=1,\,2,\,3,\,\ldots). $$ The same norms (I) can be used to define a locally convex metrizable topology on ${\mathcal{D}}(\Omega);$ see Theorem ${\mathsf{2.37}}$ and (b) of Section 1.38.However, this topology has the pick $\phi\in{\mathcal{D}}(R)$ disadvantagc of not bcing complctc. For cxample, take $n=1,$ $\Omega=R,$ with support in [0,1, $\phi>0$ in (0,1), and define $$ \psi_{m}(x)=\phi(x-1)-1\stackrel{1}{2}\phi(x-2)+\cdots+\frac{1}{m}\phi(x-m). $$ Then $\langle\psi_{m}\rangle$ is a Cauchy sequence in the suggested topology of ${\mathcal{D}}(R)$ but lim ${\mathcal{Y}}_{m}$ does not have compact support, hence is not in ${\mathcal{D}}(R).$ We shall now define another locally convex topology r on $\tau_{\mathrm{\bf~\Delta}}$ is not metrizable is only a minor incon- in which Cauchy sequences do converge. The fact that this ${\mathcal{D}}(\Omega)$ venience, as we shall see 6.3 Definitions L $\operatorname{ct}\Omega$ be a nonempty open set in $\textstyle R^{n}.$ (a) For every compact $K\subset\Omega,\;{\dot{v_{k}}}$ denotes the Frechet space topology of ${\mathcal{D}}_{K};$ ,as described in Sections 1.46 and 6.2. (b) β is the collection of all convex balanced sets $W\subset{\mathcal{D}}(\Omega)$ such that ${\mathcal{D}}_{K}\cap W\in\tau_{K}$ for every compact $K\subset\Omega$ (c) ris the collection of all unions of sets of the form $\phi+W,$ with $\phi\in{\mathcal{D}}(\Omega)$ and $W\in\beta.$ Throughout this chapter, ${\cal K}\,$ will always denote a compact subset of Q138 DISTRuBurioNS AND roURIER TRANSFORMs 6.4 Theorem (a) rt is a topology im O(Q), and $\beta$ is a local base for t (b) t makes ${\mathcal{D}}(\Omega)$ into a loclly convex topological vector space. PROOF Suppose $V_{1}\in x$ r,V。 $\L_{2}\in\tau,\;\phi\in V_{1}\;\Gamma\cap\;V_{2}$ .To prove $(a)_{*}$ , t is clearly enough to show that (1) $$ \phi+\,\mathcal{W}\subset V_{1}\cap\,V_{2} $$ for some $\textstyle W\in{\beta}$ The definition of 工 shows that there exist $\phi_{i}\in{\mathcal{D}}(\Omega)$ and $\mathcal{W}_{i}\in\beta$ such that $\overline{{\tau}}$ (2) $$ \phi\in\phi_{i}+W_{i}\subset V_{i}\qquad(i=1,2) $$ Choose $\textstyle K$ so that ${\mathcal{D}}_{K}$ contains $\phi_{1},\ \phi_{2},$ and $\varnothing\;$ p.Since $\mathcal{A}_{K}\cap\mathcal{M}_{i}$ is open in ${\mathcal{D}}_{K},$ we have (3) $$ \phi-\phi_{i}\in(1-\delta_{i})W_{i} $$ for some $\delta_{i}>0,$ The convexity of $W_{i}$ implies therefore that (4) $$ \phi-\phi_{i}+\delta_{i}\,W_{i}\subset(1-\delta_{i})W_{i}+\delta_{i}\,W_{i}=W_{i}, $$ so that (5) $$ \begin{array}{c c c}{{\phi+\delta_{i}\,W_{i}\subset\phi_{i}+W_{i}\subset V_{i}\,\qquad(i=1,2).}}\end{array} $$ Hence () holds with $W=(\delta_{1}W_{1})\cap(\delta_{2}\,W_{2})\colon$ , and $\mathbf{\Psi}(a)$ is proved. Suppose next that $\phi_{1}$ and $\phi_{2}$ arc distinct elements of ${\mathcal{D}}(\Omega),$ and put (6) $$ {\mathcal W}=\{\phi\in\mathcal{D}(\Omega)\colon||\phi||_{0}<\|\phi_{1}-\phi_{2}\|_{0}\}, $$ where $\left\|\phi\right\|_{0}$ is as in(I) in Section 6.2. Then $W\in{\boldsymbol{\beta}}$ and $\phi_{1}$ is not in $\phi_{2}+W.$ It follows that the singleton $\langle\phi_{1}\rangle$ is a closed set, relative to ${\boldsymbol{\tau}}.$ implies that Addition is t-continuous, since the convexity of every $\scriptstyle V\in{\hat{p}}$ (7) $$ (\vartheta_{1}+{\textstyle\frac{1}{2}}W)+(\vartheta_{2}+{\textstyle\frac{1}{2}}W)=(\vartheta_{1}+\vartheta_{2})+M $$ for any $\psi_{1}\in{\mathcal{D}}(\Omega),$ $\psi_{2}\in{\mathcal{D}}(\Omega).$ . Then To deal with scalar multiplication, pick a scalar ${\mathcal{Q}}_{0}$ , and a $\phi_{0}\in{\mathcal{D}}(\Omega)$ (8) $$ \mathcal{A}\phi-\alpha_{0}\,\phi_{0}=\alpha(\phi-\phi_{0})+(\alpha-\alpha_{0})\,\phi_{0}\,. $$ If $W\in{\boldsymbol{\beta}},$ there exists $\bar{\mathcal{O}}$ > Osuchthat $\delta\phi_{0}\in\frac{1}{2}W$ Choosecsotha $2c(|\alpha_{0}|+\delta)=1.$ Since $\mathcal{W}$ is convex and balanced, it follows that $(\Theta)$ $$ \alpha\phi-\alpha_{0}\,\phi_{0}\in W $$ whenever $|x-x_{0}|<\delta$ and $\phi-\phi_{0}\in c W.$ 1 This completes the proof.TEST FUNCTIoNS AND DISTRIBUToNs 139 Note: From now on, the symbol ${\mathcal{D}}(\Omega)$ will denoe the tological vector spac $\mathcal{D}(\Omega)$ will refer to this topolgy t. (340). tht as us e dcrite i oicthreaiet 6.5 Theorem (a) A comvex balanced subset of any $\mathcal{D}_{K}\subset\mathcal{D}(\Omega)$ coincides with the subspce iopology tha (b) The topology ${\mathbf{}}V$ of 9(G2) is open if and only if Ve B $\tau_{K}$ 9x inhrits from 9(2) (c) If numbers $\mathcal{M}_{N}$ is a bounded subse of 9(40), then $E\subset{\mathcal{D}}_{K}$ for some $K\subset\Omega.$ and there are $\boldsymbol{\mathit{F}}$ < o such that ery pe saisfes the inequalite $$ \|\phi\|_{N}\leq M_{N}\qquad(N=0,\;1,\;2,\,...\,). $$ (e) (d)9(Q) has the Heine- Borel property. $\{\phi_{i}^{\setminus}\}\subset{\mathcal{D}}_{K}f{\boldsymbol{\omega}}^{\gamma}$ some compact $K\subset\Omega.$ , and 1{p;} is a Cauchy sequence in 9(D), then $$ \operatorname*{lim}_{i,j arrow\infty}\|\phi_{i}-\phi_{j}\|_{N}=0\qquad(N=0,1,2,\ldots, $$ () $I f\,\phi_{i} arrow0$ the support of every $\phi_{i}\,,$ and in the topology of 9(D), hn there is a compact uniformly, as $i arrow\infty,$ for every muli $K\subset\Omega$ which contains index o $D^{x}\phi_{i} arrow0$ (g) Im 90(Q2), every Cauchy sequence coneres that $\boldsymbol{E}$ is also bounded in ${\mathcal{D}}(\Omega).$ Remark In view of(O). he necsaryconditions xpesed by Ce,.(e), an $E\subset{\mathcal{D}}_{K}$ and $\|\phi\|_{N}\leq\ M_{N}<\infty$ for every $\phi\in E,$ are also sufficient. For example,if (Section 1.46), and now Gb)implies $(f)$ then $\boldsymbol{E}$ E s a boundcd subset of ${\mathcal{D}}_{K}$ for some $W\in\beta.$ Suppose first that $V\in\tau$ . Pick $\phi\in{\mathcal{D}}_{K}\cap V$ . By Theorem 6.4, $\phi+W\subset V$ PRO0F Hence $$ \phi+(\mathcal{D}_{K}\cap\mathcal{W})\subset\mathcal{D}_{K}\cap\,V. $$ Since ${\mathcal{Q}}_{K}\cap W$ is open in ${\mathcal{Q}}_{K};$ , we have proved tha (1) $$ \mathcal{D}_{K}\cap V\in_{\tau_{X}}\qquad\mathrm{if~}V\in\tau{\mathrm{~and~}}K\subset\Omega. $$ $\beta<\tau$ Statement $(a)$ is an immediate consequence of O), sne it i obvous tha have to show that $E={\mathcal{D}}_{K}\cap V$ One-half f (O s proed by (1). For the other half,sppose $V\in\tau.$ The definition of $\tau_{K}$ implies that to every $\phi\in E$ correspond $\textstyle N$ and $\delta>0$ such that $E\in\tau_{K},$ We for some $(2)$ $$ \{\psi\in{\mathcal{D}}_{K}\colon\|\psi-\phi\|_{N}<\delta\}\subset E. $$140 pisrRupurioNs AND rouRIER TRANSFORMs Pu $W_{\phi}=\{\psi\in\mathcal{D}(\Omega)\colon$ $\|\psi\|_{N}<\delta\}.$ Then $\;W_{\phi}\subset\beta,$ , and (3) $$ \mathcal{D}_{K}\cap(\phi+W_{\phi})=\phi+(\mathcal{D}_{K}\cap\mathcal{W}_{\phi})\subset E. $$ If $\mathcal{V}$ is the union of these sets $\phi+W_{\phi}$ ,one for each $\phi\in E,$ then ${\mathbf{}}V$ has the desired property For (c), consider a set $E\subset{\mathcal{D}}(\Omega)$ which lies in no ${\mathcal{D}}_{K},$ Then there are func that $\phi_{m}(x_{m})\neq0$ (m and there are distinct points $x_{m}\in\Omega,$ without limit point in Q, such that satisfy tions $\phi_{m}\subset E$ $\iota=1,2,3,\cdot*).$ Let $\textstyle W$ be the set of all $\phi\in{\mathcal{D}}(\Omega)$ (4) $$ \vert\,\phi(x_{m})\,\vert<m^{-1}\vert\phi_{m}(x_{m})\,\vert\ \ \ \ \ (m=1,\,2,\,3,\,\ldots). $$ Thus is not bounded. t follows that every bounded subset ${\boldsymbol{E}}$ E of ${\mathcal{D}}(\Omega)$ it is easy to see that $\textstyle E.$ This shows that By $(b),E$ ${\boldsymbol{E}}$ Since each $\textstyle K$ contains only finitely many $\ x_{m}\,,$ contains ${\mathcal W}_{K}\cap W\in\tau_{K}.$ $W\in{\boldsymbol{\beta}}$ Since $\phi_{m}\not\in m W$ , no mutiple of ${\mathcal{W}}$ is then a bounded subsct of ${\mathcal{D}}_{K}$ lies in some ${\mathcal{D}}_{K}$ .Consequently(Se Section 1.46) (5) $$ \mathrm{sup}\left\{\left|\phi\right|_{N}\!:\phi\in E\right\}<\infty\qquad(N=0,1,2,\ldots). $$ This completes the proof of (c) in 90(Q) lies in some ${\mathcal{D}}_{K}$ By 6b),{ep is then also a Cauchy Statement (d follows from Cc), since ${\mathcal{D}}_{K}$ has the Heine-Borcl property Since Cacy sequcsarbond Scion 12). e implie tever Cauchy sequence{ $\textstyle|\phi_{3}|$ ${\mathcal{D}}_{K}$ sequence relative to $\tau_{K}$ This proves (e). -.(Reall tha $///\d J$ Statement GC)isjust a restatement of (e) ${\mathcal{D}}_{K}$ Finally J follows from b),(e), and the complenesso is a Fréchet space.) 6.6 Theorem Supose is a linarmpin of 9(Q) itoa locally ovex space Y. Then each f the olloig or properties implies uhe ouher: (d) (a) A is continuous. O in ${\cal Y}.$ are coninuous (c) (の)Ais bounded $\textstyle{\lambda\phi_{i}-\psi}$ ${\mathcal{D}}_{K}\subset{\mathcal{D}}(\Omega)$ I/ 中h→0O in 9(S2) then The restrictions of A to every to $\phi_{i} arrow0$ in ${\mathcal{D}}(\Omega).$ Hence (c) implies that is contained in Theorem 1.32. , By 6b) of Thcorem 6.5 $\phi_{i}\to0$ in is PROOF The implication $(a)\to(b)$ in 9(2). By Theorem 6.5 asi→0. Since ${\mathcal{D}}_{K}$ Some Assume $\Lambda$ is bounded and $\phi_{i}\to0$ ${\boldsymbol{V}}.$ Thus b) implies (c) $\Lambda\colon{\mathcal{D}}_{K}\to Y$ Assume (c) holds $\{\phi_{i}\}\subset{\mathcal{D}}_{K}\,,$ and $\phi_{i}\to0$ in ${\mathcal{D}}_{K}$ is boundcd. Theorem 1.32,applied ${\mathcal{D}}_{K}\,;$ and thc rcstriction of $\Lambda$ to this ${\mathcal{D}}_{K}$ , shows that $\Lambda\phi_{i} arrow0$ in $\Lambda\phi_{i} arrow0$ in ${\cal{Y}},$ mctrizable,(d followsTEST FUNCTIoNs AND DsSTRuBUroNs 14 O in ${\cal{Y}},$ and put To prove that(d) implies a), let Then ${\mathbf{}}V$ is convex and balanced. ny of Theore for every ${a\!\!\!/}/{j\!\!\!/}/$ $U_{\mathbf{\delta}}U$ be a convex balanced neighborhood of $5.5,\ V$ is open in $V=\Lambda^{-1}(U).$ if and only if ${\mathcal{D}}_{K}:\mathbf{V}$ is open in ${\sigma_{K}},$ ${\mathcal{D}}_{K}\subset{\mathcal{D}}(\Omega)$ ${\mathcal{D}}(\Omega)$ $\langle{\bar{d}}\rangle$ . This proves the equivalence of Go and into 9(2). Corollay Dery diferenial opertor D is a comimuous muying o 9(0) ${\mathcal{D}}_{K}$ rRor Sice I Del Il4lr+1, fr =0,1,.… $D^{\mathrm{o}}$ is continuous on each $///\hbar$ the topolog The space fal disributons in $\mathbb{Q}$ 6.7 Definitio A iner funcional on (9) which is continuous withrepet ${\mathcal{A}}^{\prime}(\Omega).$ $\overline{{\Gamma}}$ describedin Definitio 6.3 calea disriomimn is denoted t Notethat Theorm 6 ppis t inar fnctoas on (0).1 leads to t follongueful characterizato o istition are equivalent: 6.8 Theorem Ais ainer ucioalom 9(0),the folowimg two comiio (a)A∈ 9(Q) (b)To every compact $K\subset\Omega$ corresponds a nonnegative integer ${\boldsymbol{N}}$ and a constant $C<\infty$ such that the inequality $|\wedge\phi|\leq C||\phi||_{N}$ holds for every $\phi\in{\mathcal{D}}_{K}$ given in Section 6.2. PRor This is precisely the cquivalence of ${}_{(a)}$ )and (d)in Theorem 6.6, combined $\|\phi\|_{N}$ with the description of the topolgy of ${\mathcal{D}}_{K}$ by means of the seminorms $it//j$ Note: If A is such that one $\textstyle{\bigwedge}$ wildo or al $\textstyle K$ (but not necesariy with the same then A C), then the smallst such $\textstyle N$ is caled the order of ${\boldsymbol{\Lambda}}.$ If no $\textstyle N$ V will do for all $K_{\mathrm{\scriptscriptstyleI}}$ is said to have infinite order. 6.9 Remark Each $x\in\Omega$ determines a linear functional $\delta_{x}\,$ on 9(), y the formula $$ \delta_{x}(\phi)=\phi(x). $$ If Theorem 6.8 shows that $\partial_{x}\,$ is a distribution,of order $0.$ is frequently called the Dirac $x\equiv0,$ the origin of $\textstyle R^{n},$ the functional $\delta=\delta_{\mathrm{o}}$ measure on $R^{n}.$ Since ${\mathcal{R}}_{K},$ for $K\subset\Omega,$ is the intersection of the null spaces of these $\delta_{x},$ as $\textstyle{\mathcal{X}}$ ranges over the complement of K,it follows that each ${\mathcal{D}}_{K}$ is a closed subspace of 9(Q). [This142 DsrRumrioNs AND roUruIER TRANSroRMs follows also from Theorem 1.27 and part (b) of Theorem 6.5, since each ${\mathcal{D}}_{K}$ is com- plete. Itis obvious that cach ${\mathcal{D}}_{K}$ has empty interior, rclative to 90(9). Since there is a is of the frs countable collection of sets $K_{i}\subset\Omega$ such that $\mathcal{D}(\Omega)=\bigcup\mathcal{D}_{K_{i}},$ ${\mathcal{D}}(\Omega)$ category in itself. Since Cauchy sequences converge in ${\mathcal{D}}(\Omega)$ (Theorem 6.3), Baire's theorem implies that 9J(D) is not metrizable. Calculus with Distributions 6.10 Notations As before,S $\mathbf{G}$ will denote a nonempty open set in $R^{n},$ If $x=(x_{1},$ ……,,) and ${\boldsymbol{\beta}}=(\beta_{1},\ldots,\ \beta_{n})$ are multi-indices (see Section 1.46) then (4) $$ \begin{array}{r l}{\left|\alpha\right|=\alpha_{1}+\cdots+\alpha_{n},}\\ {\beta\leq\alpha\;\operatorname*{means}\beta_{i}\leq\alpha_{i}\,{\mathrm{for}}\ 1\leq i\leq n,}\\ {\alpha\pm\beta=(\alpha_{1}\pm\beta_{1},\ldots,\alpha_{n}\pm\beta_{n}).}\end{array} $$ (1) (2) (3) If $x\in R^{n}$ and $y\in R^{n}.$ then (6) $$ \begin{array}{c}{{x\cdot y=x_{1}y_{1}+\cdot\cdot\cdot\cdot+x_{n}y_{n},}}\\ {{\mid x\mid=(x\cdot x)^{1/2}=(x_{1}^{2}+\cdot\cdot\cdot\cdot+x_{n}^{2})^{1/2}.}}\end{array} $$ (5) The fact that the absolute value sign has different meanings in $\operatorname{\mathcal{(1)}}$ and in (6) should cause no confusion $\operatorname{If}\,x\in R^{n}$ and $\textstyle{\mathcal{Q}}$ is a multi-index, the monomial ${\mathcal{X}}^{\mathcal{F}}$ is defined b (7) $$ x^{\alpha}=\,x_{1}^{\alpha_{1}}\cdot\ast\cdot\,x_{n}^{\alpha_{n}}. $$ 6.11 Functions and measures as distributions Suppose fis a lcally integrable $[g]f(x)[d x<\infty$ complcx function in . This means thtfis Lebesgue measurable and for every compact $K\subset\Omega$ : dx denotes Lebesgue measure. Define (1) $$ \Lambda_{f}(\phi)=\int_{\Omega}\phi(x)f(x)\,d x\qquad[\phi\in{\mathcal{D}}(\Omega)]. $$ Sincc (2) $$ |\Lambda_{f}(\phi)|\leq\left(\int_{\cal K}|\,f\,|\,\right)\cdot\|\phi\|_{0}\qquad(\phi\in{\mathcal D}_{\cal K}),\qquad $$ $\mathrm{{It}}$ Theorem 6.8 shows that $\Lambda_{f}\in{\mathcal{D}}^{\prime}(\Omega).$ with the functionf and to say that is customary to identify the distribution $\Lambda_{f}$ such distributions “are”functions.TSsr ruNcrios NpD Disrunorros 143 with $\nu(K)<\infty$ Similarly,if pis a complex Borel measure on $K\subset\Omega,$ the equation , rif pis a positive measure on Q $\Omega,$ for every compact (3) $$ \Lambda_{\mu}(\phi)=\int_{\Omega}\phi\,d\mu\qquad[\phi\in{\mathcal{D}}(\Omega)] $$ defines a distribution $\Lambda_{\mu}$ in $\Omega_{\mathrm{,}}$ which isualy dentified with formula 6.12 Diferentiation of distributions I is a muli-index and $\Lambda\in{\mathcal{D}}^{\prime}(\Omega),$ the (1) $$ (D^{\alpha}\Lambda)(\phi)\stackrel{\wedge}{=}(-1)^{|\pi|}\Lambda(D^{\alpha}\phi)\;\;\;\;\;\;\;\;\;[\phi\in\mathcal{D}(\Omega)] $$ (motivated in Section 6.) eines a linear functiona ${\mathcal{D}}^{\prime}\Lambda$ on ${\mathcal{Q}}(\Omega)$ .If (2) $$ .\qquad|\Lambda\phi|\leq C\|\phi\|_{N} $$ for all $\phi\in{\mathcal{D}}_{K},$ then (3) $$ |(D^{x}\Lambda)(\phi)|\leq C\|D^{x}\phi\|_{N}\leq C\|\phi\|_{N+|x|}\,. $$ Theorem 6.8 shows therefore that $D^{*}\Lambda\in{\mathcal{D}}^{\prime}(\Omega).$ Note that the formula (4) $$ D^{\alpha}D^{\beta}\Lambda=D^{\alpha\,+\beta}\Lambda=D^{\beta}L^{\alpha}\Lambda $$ holds for every distribution $\Lambda$ and for all multi-indices c and ${\boldsymbol{\beta}},$ simply because the operators $D^{\alpha}$ and $D^{\beta}$ commute on $C^{\infty}(\Omega);$ $$ \begin{array}{l l}{{(D^{x}D^{\beta}\Lambda)(\phi)=(-1)^{|x|}(D^{\beta}\Lambda)(D^{x}\phi)}}\\ {{{}}}&{{=(-1)^{|x|+|\beta|}\Lambda(D^{\beta}D^{x}\phi)}}\\ {{{}}}&{{=(-1)^{|x+\beta}{\Lambda(D^{x+\beta}\phi)}}}\\ {{{}}}&{{=(D^{x+\beta}{\Lambda})(\phi).}}\end{array} $$ If locall integrable unction f in 6.13 Distribution derivatives of functions The cth distribution derivative of $D^{\alpha}\Lambda_{f}\,.$ ${\mathcal{D}}{\mathcal{T}}$ $\Omega$ is, by definitin, the distribution ralso exis n the lassical snsc nd s locally ntegrable, then Dy is also distribution in the sense of Secton .11. The obvious consisency problem is whcther the equation (1) $$ {\tilde{D}}^{n}\Lambda_{f}=\Lambda_{D^{n}f} $$ always holds under these conditions. More explicitly, the question is whether (2) $$ (-1)^{|x|}{\int}_{\Omega}f(x)(D^{x}\phi)(x)\,d x={\int}_{\Omega}(D^{x}j)(x)\phi(x)\,d x $$ for every pe 9(Q)144 DISTrRrourIoNs AND roURIER TRANSFoRMs If f has continuous partial derivatives of all orders up to ${\cal N},$ integrations by part give (2) without difficulty, if $|\alpha|\leq N.$ In general, (l) may be false. The following example illustrates this, in the case $\scriptstyle n\;=\;1$ 6,14 Example Suppose $\Omega$ is a segment in ${\boldsymbol{R}},$ and $\boldsymbol{f}$ f is a left-continuous function that $D/\in L^{1}.$ of bounded variation in Q.If $D=d/d x.$ it is well known that $(D f)(x)$ exists a.e. and We claim that (1) $$ D\Lambda_{f}=\Lambda_{\mu} $$ where $\boldsymbol{\mathit{I}}$ is the measure defined in $\mathbb{Q}$ by (2) $$ \mu([a,\,b))=f(b)-f(a). $$ Thus $D\Lambda_{f}=\Lambda_{D f}$ if and only if fis absolutely continuous. To prove(1), we have to show that $$ (\Lambda_{\mu})(\phi)=(D\Lambda_{f})(\phi)=-\ \Lambda_{f}(D\phi) $$ for ever $\phi\in{\mathcal{D}}(\Omega)$ ,that is, that (3) $$ \int_{\Omega}\phi\,d\mu=\,-\,\int_{\Omega}\phi^{\prime}(x)f(x)\,d x. $$ But (3)is a simple consequence of Fubini's theorem,since each side of(3) is equal to the integral o $\phi\left(x\right)$ over the set (4) $$ \{(x,y)\colon x\in\Omega,y\in\Omega,\;x<y\} $$ with respect to the product measure of $d x$ and $d_{\mu}.$ The fact that $\phi$ has compact support in $\Omega$ is used in this computation 6.15 Multiplication by functions Suppose $\Lambda\in{\mathcal{D}}^{\prime}(\Omega)$ and ${\mathcal{F}}\in C^{\infty}(\Omega).$ The right side of the equation (1) $$ (f\Lambda)(\phi)=\Lambda(f\phi)\qquad[\phi\in{\mathcal D}(\Omega)] $$ makes sense because fA on 9(Q2).We shall see that fXA is, in fact, a distribution in $\mathbb{Q}.$ Thus (1) defines a linear functional $f{\boldsymbol{\phi}}\in{\mathcal{D}}(\Omega)$ when $\phi\in{\mathcal{D}}(\Omega).$ Observe that the notation must be handled with care: $\operatorname{If}f\in{\mathcal{D}}(\Omega),$ then $\Lambda f$ is a number, whereas fA is a distribution. The proof that $f\Lambda\in{\mathcal{D}}^{\prime}(\Omega)$ depends on the Leibniz formula (2) $$ D^{\alpha}(f g)=\sum_{\beta\leq\alpha}c_{\alpha\beta}(D^{\alpha-\beta}f)(D^{\beta}g), $$ valid for all f and $\scriptstyle{\mathcal{G}}$ in $C^{n}(\Omega)$ and all multi-indices ${\mathcal{Q}},$ which is obtained by iteration of the familiar formula (3) (uD)’= u'v+ uv’isT FuNcrioNS AND DISTRiBurioNs 145 The numbers ${\mathcal{C}}_{\alpha\beta}$ are positive integers whose exact value is easily computed but s irrelevant to our present nceds all $\phi\in{\mathcal{D}}_{K}$ To each compact $K\subset\Omega$ correspond ${\boldsymbol{C}}$ and ${\cal N}$ such that ${\boldsymbol{f}},$ $K_{\circ}$ and $\textstyle N$ ), such that for By(2), there is a constant ${\cal C}^{\prime}{}_{i}$ $|\wedge\phi|\leq C\|\phi\|_{N}$ , depending on 川 $f\phi\|_{N}\leq C^{\prime}\|\phi\|_{N}$ for $\phi\in{\mathcal{D}}_{K},$ Hence (4) $$ .\cdot\cdot\cdot\cdot\cdot\cdot\cdot\cdot\cdot\cdot\cdot\cdot\cdot\cdot\cdot\cdot\cdot\cdot\cdot\cdot\cdot\cdot $$ By Theorem $6.8,/\Lambda\in\mathcal{D}^{\prime}(\Omega).$ in place of ${\mathcal{G}},$ so that Now we want io show that the Leibniz formula(2) holds wih $\Lambda$ (5) $$ D^{\alpha}(f\Lambda)=\sum_{\beta\leq x}c_{\alpha\beta}(D^{\alpha-\beta}f)(D^{\beta}\Lambda). $$ defined by The proof is a purely formal calculation. Associate to each $v\in R^{\circ}$ the function $\textstyle\int_{u}$ $$ h_{u}(x)=\exp{(u\cdot x)}. $$ Then $D^{\alpha}h_{u}=u^{x}h_{u}$ .If (2) is applied to $h_{\mathrm{u}}$ and $\hbar_{v}$ in place of f and ${\mathcal{G}}_{\cdot}$ the identity (6) $$ (u+v)^{x}=\sum_{\beta\leq x}c_{x\beta}u^{x-\beta}v^{\beta}~~~~~~~(u\in R^{n},\,v\in R^{n}) $$ is obtained. In particular $$ \begin{array}{l}{{\left|\begin{array}{l}{{\mathrm{f}}^{i}\equiv\left[\left[\mp\frac1{\sqrt{1+}}\right]\right]\right] |^{1}}}\\ {{\left.\mathrm{~}}}\\ {{\mathrm{~}}}\\ {{\mathrm{~}}}\\ {{\mathrm{~}}}\\ {{\mathrm{~}}}\\ {{\mathrm{~}}}\\ {{\mathrm{~}}}\\ {{\mathrm{~}}}\\ {{\mathrm{~}}}\\ {{\mathrm{~}}}\\ {{\mathrm{~}}}\\ {{\mathrm{~}}}\\ {{\mathrm{~}}}\\ {{\mathrm{~}}}\\ {{\mathrm{~}}}\\ {{\mathrm{~}}}\\ {{\mathrm{~}}}\\ {{\mathrm{~}}}\\ {{\mathrm{~}}}\\ {{\mathrm{~}}}\\ {{\mathrm{~}}}\\ {{\mathrm{~}}}\\ {{\mathrm{~}}}}\\ {{\mathrm{~}}}\\ {{\mathrm{~}}}\sum_{\mathrm{~}}}\left\{\bar{\mathrm{~}}_{{ |}}_{i j}_{\mathrm{\mathrm{ij}}}\end{array}{\right.}\mathrm{\mathrm{\mathrm{r}}}_{ \{}}\end{array}{\mathrm{ij}}_{{{\right.}}}\mathrm{~}}\mathrm{{\mathrm{\mathrm{\mathrm{\mathrm{\mathrm{\mathrm{ij}}}}}}}\mathrm{\mathrm{{ij}\mathrm{\mathrm{\mathrm{\mathrm{{\mathrm{ij}}}}}}}\end{array} $$ Hence (7) $$ \sum_{r<\beta\leq\alpha}(-1)^{|\beta|}c_{\alpha\beta}c_{\beta\gamma}\equiv\left\{(-\uparrow)^{|\alpha|}\right.\qquad\mathrm{if~\cdot~-}\sigma_{\cdot} $$ Apply (2) to $D^{\beta}(\phi D^{x-\beta}f),$ and use (T), o obtain the identit (8) $$ \sum_{\beta\leq\alpha}(-1)^{|\beta|}c_{\alpha\beta}D^{\beta}(\phi D^{\alpha-\beta}f)=(-1)^{|\alpha|}f D^{\alpha}\phi. $$ The point of all this is that ${\binom{\sigma_{*}^{*}}{8}}{\bar{\mathfrak{R}}}\mathrm{ives}\ (5$ 5). For if $\phi\in{\mathcal{D}}(\Omega)$ ). then $$ \begin{array}{l l}{{D^{x}(f\Lambda)(\phi)=(-1)^{|x|}(f\Lambda)(D^{x}\phi)=(-1)^{|x|}\Lambda(f D^{x}\phi)}}\\ {{}}&{{=\sum_{\beta\leq x}(-1)^{|\beta|}c_{\alpha\beta}\ \Lambda(D^{\beta}(\phi D^{x-\beta}f))}}\\ {{}}&{{=\sum_{\beta\leq x}c_{\alpha\beta}(D^{\beta}\Lambda)(\phi D^{x-\beta}f))(\phi).}}\\ {{}}&{{=\sum_{\beta\leq x}c_{\alpha\beta}[(D^{x-\beta}f)(D^{\beta}\Lambda)](\phi).}}\end{array} $$146 DSTRuBUrioNs AND FoURIER NSFORM $$ \hat{\mathrm{\boldmath~\hat{~}~}}\qquad\qquad\qquad\qquad\qquad\hat{\mathrm{\boldmath~\hat{~}~}}\qquad\qquad\qquad\hat{\mathrm{\boldmath~\hat{~}~}}\qquad\qquad\qquad\qquad\qquad\qquad\times\qquad\qquad\qquad\times\qquad\qquad\times\qquad\qquad\qquad\times\qquad\qquad\qquad\times\qquad\qquad\qquad\qquad\times\qquad\qquad\qquad\qquad\times\qquad\qquad\qquad\qquad\qquad\qquad\qquad\qquad\qquad\times\qquad\qquad\qquad\qquad\qquad\qquad\qquad\qquad\qquad\qquad\times\qquad\qquad\qquad\qquad(\qquad\qquad(\times\qquad\qquad(\times\cdots\qquad\qquad(\ X^{\prime}\ X) $$ 6.16 Sequences of distributions Since ${\mathcal{D}}^{\prime}(\mathbf{G})$ is the space of all continuous linear functionals on 90(2),the general considerations made in Section 3.14 providc a topology for ${\mathcal{D}}^{\prime}(\Omega).$ 一-its weak*-topology induced by 9(Q)--which makes ${\mathcal{D}}^{\prime}(\Omega)$ into a locally convex space.If $\{\Lambda_{i}\}$ is a sequence of distributions in SQ, the statement (1) $$ \Lambda_{i} arrow\Lambda~{\mathrm{in}}~\mathcal{D}^{\prime}(\Omega) $$ refers to this weak*-topology and means, explicitly, that (2) $$ \operatorname*{lim}_{i arrow\infty}\Lambda_{i}\,\phi=\Lambda\phi\qquad[\phi\in\mathcal*\mathcal*}(\Omega)]. $$ ments In particular, $\mathbb{H}\langle f_{i}\rangle$ ${\mathcal{D}}^{\prime}(\Omega)^{\prime\prime}\;\mathrm{or}^{\star\prime}\{f_{i}\}$ converges to is a sequence of locally integrable functions in $\mathbb{Q}_{,}$ the state $^{*}f_{i}\to\Lambda$ in $\Lambda$ in the distribution sense”mean that (3) $$ \operatorname*{lim}_{i arrow\infty}\int_{\Omega}\phi(x)f_{i}(x)\,d x=\Lambda\phi $$ for every $\phi\in{\mathcal{D}}(\Omega).$ The simplicity of the next theorem, concerning termwise diffrentiation of a sequence,is rather striking. 6.17 Theorem Suppose $\Lambda_{i}\in{\mathcal{D}}^{\prime}(\Omega)\,J o r\ i=1,\,2,\,3,\,\ldots,\,a n a$ (1) $$ \Lambda\phi=\operatorname*{lim}_{i arrow\infty}\Lambda_{i}\phi $$ exists (as a complex number) Jor every ${\bf\nabla}\phi\in{\mathcal D}(\Omega).\ T h e n\wedge\in{\mathcal D}^{\prime}(\Omega),\,c$ and (2) $$ D^{x}\Lambda_{i} arrow D^{x}\Lambda~i n~{\mathcal{D}}^{\prime}(\Omega), $$ for every multi-index α. PROOF: Let $\textstyle K$ be an arbitrary compact subset of $\Omega.$ Since(1) holds for every $\phi\in{\mathcal{D}}_{K},$ implies that the restriction of $\Lambda$ A to ${\mathcal{D}}_{K}$ is a Fréchet space, the Banach-Steinhaus theorem 2.8 and since ${\mathcal{Q}}_{K}$ is continuous. It follows from Theorem 6.6 that $\Lambda$ is continuous on ${\mathcal{D}}(\Omega)$ ; in other words, $\Lambda\in{\mathcal{D}}^{\prime}(\Omega)$ Consequently (I implies that $$ \begin{array}{l l}{{(D^{x}\Lambda)(\phi)=(-1)^{|x|}\Lambda(D^{x}\phi)}}&{{}}\\ {{=(-1)^{|\alpha|}\operatorname*{lim}_{i arrow\alpha}\Lambda_{i}(D^{x}\phi)=\operatorname*{lim}_{i arrow\infty}(D^{x}\Lambda_{i})(\phi).}}&{{}}\end{array}\qquad\qquad.,\,\not=\not=1 $$ 1/ 6.18 Theorem If $\Lambda_{i} arrow\Lambda$ in 9'(Q) and $\scriptstyle{\theta_{i} rightarrow g}$ in $C^{\infty}(\Omega)$ ), then $g_{i}\wedge_{i} arrow g\wedge$ in 9′(Q). Note: The statement $^{*}g_{i} arrow g$ in $C^{\infty}(\Omega)^{,\;\gamma}\operatorname{referstot}\mathbf{t}\}$ he Fréchet space topology of C*(Q2) described in Section 1.46. PRoOF Fix $\phi\in{\mathcal{D}}(\Omega).$ Definc a bilincar functional $\boldsymbol{B}$ on $C^{\infty}(\Omega)\times{\mathcal{D}}^{\prime}(\Omega)$ by $$ B(g,\Lambda)=(g\Lambda)(\phi)=\Lambda(g\phi). $$TEST FUNCrIoNs AND DSrpIBrIoNs 147 Then $\boldsymbol{B}$ is separately continuous, and Theorem 2.17 implies that $$ B(g_{i},\Lambda_{i}) arrow B(g,\Lambda)\qquad\mathrm{as~}i arrow\infty. $$ Hence $$ (g_{i}\Lambda_{i})(\phi)\to(g\Lambda)(\phi). $$ / Localization 6.19 Local equality Suppose $\Lambda_{i}\in{\mathcal{D}}(\Omega)\ (i=1,2)$ and ${\boldsymbol{\omega}}$ is an open subset of $\mathbb{Q},$ The statement (1) $$ \Lambda_{1}=\Lambda_{2}\mathrm{~in~}\alpha $$ in α $\mu(E)=0$ means, by definition, that $\Lambda_{1}\phi=.\Lambda_{2}\,\phi$ for every $\phi\in{\mathcal{D}}(\omega).$ is a measure, then $\Lambda_{f}=0$ ${\boldsymbol{\omega}}$ for every Borel set For example, if fis a locally integrable function and $\boldsymbol{\mu}$ in o if and only if if and only if $f(x)=0$ for almost every xeo, and $\Lambda_{\mu}=0$ $E\subset\omega.$ This defnition makes it possible to discuss distributions locally、 On the othe hand, it is also possible to describe a distribution globally if is local behavior is which we now construct known. This istated preisely in Theorem 6.21. The proof uses partitions of unity, 6.20 Theorem If ${\boldsymbol{\Gamma}}$ is a collection of open sets in $R^{n}$ whose union is $\mathbb{Q},$ then there exists a sequence $\{\vartheta_{i}\}\subset{\mathcal{D}}(\Omega),$ with $\psi_{i}\geq0,$ such that (a)each w has itsupprt in some member of T, (b) $\sum_{i=1}^{\infty}\psi_{i}(x)=1\,f o r\,\,e v e r y\,\,x\in\Omega,$ (c)to every compact $K\subset\Omega$ correspond an integer ${\it m}{\it l}$ and an open set $W{\boldsymbol{\Sigma}}~K$ such that (1) $$ \textstyle\psi_{1}(x)+\cdot\cdot\cdot+\psi_{m}(x)=1 $$ Jor l $x\in W.$ Such a collection $\langle\nu_{3}\rangle$ is calld a ocally fnite partition of unity in Q,subordinate to the open cover $\mathbf{\hat{T}}$ of $\mathbb{Q}$ Note that it follows from $\mathbf{\Psi}(b)$ and $\left(c\right)$ that every point of $\mathbb{Q}$ reason for calling has a neighborhood which intesct te supports of onlyfiniey many $\textstyle\psi_{i}$ This is the $\scriptstyle\{\nu_{3}\}$ locally finite. radius ball with center z $\mathbf{p}_{i}$ sequence that contains every closed ball $B_{i}$ whose center $\{B_{1},\,B_{2},\,B_{3},\,\ldots\}$ be a PROOF Let $\boldsymbol{S}$ be a countable dense subset of Q. Let ${\boldsymbol{r}}_{i}$ and radius r,/2.1t is easy to see that $\boldsymbol{p_{i}}$ lies in ${\boldsymbol{S}},$ whose is rational, and which lies in some member of F. Let $V_{i}$ be the open $\Omega=\cup\;V_{i}.$148 DisSTRIorioNs AND rouaurER TRANSFoRMs $\phi_{i}\in{\mathcal{D}}(\Omega)$ such that $\phi_{i}\geq0,$ $\phi_{i}=1$ The constructon described in Sction .4 shows that there arefunctin off $\textstyle B_{i}\,.$ Define ${\mathit{\mathcal{I}}_{1}}=\phi_{1},$ and, in $V_{i},\;\phi_{i}=0$ inductively (2) $$ \begin{array}{c c c}{{\psi_{i+1}=(1-\phi_{1})\cdot\cdot\cdot\cdot(1-\phi_{i})\phi_{i+1}}}&{{}}&{{(i\geq1).}}\end{array} $$ Obviousy $\textstyle\psi_{i}=0$ outside $B_{i}$ This gives (a).The relation (3) $$ \begin{array}{c}{{\psi_{1}\ +\cdot\cdot\cdot\cdot+\psi_{i}\longrightarrow1\ -(1\ -\phi_{1})\cdot\cdot\cdot(1\ -\phi_{i})}}\end{array} $$ with $i+1$ in place of ${\hat{l}}.$ If (3) holds for some i addition of (2) and 3) yields ${\hat{l}},$ Since $\phi_{i}=1$ in $V_{i},$ it is trivial when Hence (3) holds for every $\left(3\right)$ $i=1.$ follows that (4) $$ \begin{array}{c c c}{{\psi_{1}(x)+\cdots+\psi_{m}(x)=1~~~}}&{{\mathrm{if~}x\in V_{1}\cup\cdots\cdots\cup V_{m}.}}\end{array} $$ $m,$ This gives (b). Moreover, if $\textstyle K$ is compact, thn $K\subset V_{1}\cup\cdots\sim V_{m}$ for some // and (cy follows 6.21 Theorem Suppose ${\boldsymbol{\Gamma}}$ is an open cover of an open set $\Omega\subset{\boldsymbol{R}}^{n}$ ;,and suppose thal to each oer corresponds a distibuin $\Lambda_{\omega}\in{\mathcal{D}}^{\prime}(\omega)$ such tha (1) $$ \Lambda_{\omega^{\ast}}=\Lambda_{\omega^{\ast}} $$ in o’ ${\mathbf{T}}$ , o” whenever o'"个 $\L{\omega^{\prime}}\neq Q.$ such that Then there exists a unique $\Lambda\in{\mathcal{D}}^{\prime}(\Omega)$ (2) $$ \Lambda=\Lambda_{\omega}\;i n\;\omega $$ for every oe T. If different from O, since $\phi$ be a localyfie patitonof unty subordinateto $\mathbf{T},$ as in pRoor Let $\langle\nu_{3}\rangle$ and associate to each ia set $\omega_{i}\in\Gamma$ such that ${\boldsymbol{\omega}}_{i}$ contains the Theorem ${\mathfrak{o}},\scriptstyle0.$ , then $\phi=\sum\psi_{i}\,\phi.$ Only fintely many terms in this sum are support of $\textstyle\psi_{i}$ $\phi\in{\mathcal{D}}(\Omega,$ has compact support. Define (3) $$ \Lambda\phi=\sum_{i=1}^{\infty}\Lambda_{o_{i}}(\vartheta_{i}\phi). $$ $\mathbf{f}\mathbf{f}$ $K\subset\Omega$ To show that $\Lambda$ A is a linear functional on ${\mathcal{Q}}(\mathbf{0}).$ in 90(Q). There is a compact $\mathrm{\boldmath~k~}$ is clear that $\Lambda$ is continuous, suppose $\phi_{j}\to0$ which contains the support of every $\phi_{j}$ ,. If is chosen as in part(cy of Theorem 6.20, then (4) $$ \cdot\Lambda\phi_{j}=\sum_{i=1}^{m}\Lambda_{\omega_{i}}(\vartheta_{i}\phi_{j})\mathrm{~-~(j=1,2,3,\ldots).} $$TEST FUNCTioNS AND DISTRIBUrIoNs 149 Since ${\sqrt{i}}_{i}\phi_{j}\to($ O in 9(0), asj→oo,it flows from (4) that $\Lambda\phi_{j} arrow0.$ By Theorem 6.6, $\Lambda\in{\mathcal{D}}^{\prime}(\Omega).$ To prove (2), pick $\phi\in{\mathcal{D}}(\omega).$ Then (5) $$ \psi_{i}\,\phi\in{\mathcal D}(\omega_{i}\cap\omega)\qquad(i=1,\,2,3,\dots) $$ so that $\mathbf{(1)}$ implies $\Lambda_{\omega_{i}}(\psi_{i}\phi)=\Lambda_{\omega}(\psi_{i}\phi).$ .Hence (6) $$ \ ^{\cdot\cdot}\wedge\phi=\sum^{\cdot}\Lambda_{\omega}(y_{i}\phi)=\Lambda_{\omega}(\sum\vartheta_{i}\phi)=\Lambda_{\omega}\,\phi, $$ which proves (2). in place of o) implies that This gis th eitece of . The uniqunes s iaisnce(2(with ${\it j}/j$ $\Lambda$ must satisfy (3) Supports of Distributions for every 6.22 Definition Suppose $\Lambda\in{\mathcal{D}}^{\prime}(\Omega).$ 1f ${\boldsymbol{\omega}}$ is an open subset of $\underline{{\otimes}}$ and if $\Lambda\phi=0$ $\phi\in{\mathcal{D}}(\omega),$ we say that A vanishes in o. Let $\mathcal{W}$ be the union of all open c $\mathfrak{g}\in\Omega$ in which A vanishes. The complement of ${\mathcal{W}}$ Grelative to SD)is th support / 人. 6.23 Theorem J Wisas above, the $\Lambda$ vanishes $W.$ ordinate to tion of these o's, and let $\scriptstyle\{\vartheta_{3}\}$ is the union of open sets o inwhich A vanishes. Let then $\phi=\sum\psi_{i}\phi$ 'be the collec- sub PROOF $\mathcal{W}$ ${\boldsymbol{\Gamma}}$ ${\boldsymbol{\Gamma}}$ many terms of this sum are diferent from ${\boldsymbol{0}}.$ be a locally finite partition of unity in $W_{\mathrm{{J}}}$ , as in Theorem 6.20. If $\phi\in{\mathcal{D}}(W),$ Hence Only finitely $$ \Lambda\phi=\sum\Lambda(\psi_{i}\phi)=0 $$ since each y ${\mathcal{Y}}_{i}$ . has its support in some oer ${\it j}/{\it j}/$ The mos sinifiant part of thenex theren s (c). Exercise O complements t 6.24 Theorem Suppose $\Lambda\in{\mathcal{D}}^{\prime}(\Omega)$ and $S_{\Lambda}$ is the support of Λ. (b)1f $S_{\Lambda}$ IJ the support of some pe Q(G2) does not interse $S_{\mathrm{A}},$ then $\Lambda\phi=0.$ (a) is empty, then $\Lambda=0.\quad\mathbf{\partial}\cdot\Lambda$ (c) $I/{\boldsymbol{\psi}}\in C^{\infty}(\Omega)$ and $\psi=1$ in some open set $\Lambda$ has fite order in jact the is a constam $\psi\Lambda=\Lambda.$ (d) If $S_{\Lambda}$ is compact subse o Q, then ${\mathbf{}}V$ V containing $S_{\Lambda}$ then C< αO and $\bar{\boldsymbol{a}}$ nonnegative integer ${\cal N}$ N such that $$ \left|\Lambda\phi\right|\leq C\left| |\phi\right|_{N} $$ for every $\phi\in{\mathcal{B}}(\Omega).$ Furthermore, A extends in a wnique way to a contiuows linear fumctional on $C^{\infty}(\Omega).$150 DISTRIBUrIONS AND FOURIER TRANSFORMS support of $\phi-\psi\phi$ does not intersect and (b) are obvious.If V is as in (cy and if Thus $\Lambda\phi=\Lambda(\psi\phi)=(\psi\Lambda)(\phi),$ , then the PROOF Parts $\mathbf{\Psi}(a)$ $S_{\Lambda}.$ $\phi\in{\mathcal{D}}(\Omega).$ by (a) If $S_{\Lambda}$ is compact,it follows from Theorem 6.20 that there exists $\textstyle\bigcup\in{\mathcal{D}}(\Omega)$ that satisfies Cc). Fix such a y; call its support $K.$ By Theorem 6.8, there exist $c_{\mathrm{I}}$ and ${\boldsymbol{N}}$ V such that $|\wedge\phi|\leq c_{1}||\phi||_{N}$ for all $\phi\in{\mathcal{D}}_{K}$ .The Leibniz formula shows that there is a constant ${\boldsymbol{c}}_{2}$ such that $\|\phi\|_{N}\leq c_{2}\|\phi\|_{N}$ for every $\phi\in{\mathcal{D}}(\Omega).$ Hence $$ |\wedge\!\phi|=|\wedge\!\langle\!\langle\!\slash\phi)|\leq c_{1}|\!|\cup\!\phi||_{N}\leq c_{1}c_{2}|\!|\phi||_{N} $$ for every $\phi\in{\mathcal{D}}(\Omega).$ Since $\Lambda\phi=\Lambda(\psi\phi)$ for all $\varnothing\;$ ∈ 9(2), the formula (1) $$ \Lambda_{\mathrm{3}}\!f=\Lambda(\psi\!f)\;\;\;\;\;\;\;\;\;\;\left[f\in C^{\infty}(\Omega)\right] $$ defines an extension of $\Lambda.$ This extension is continuous, for $\operatorname{iff}f_{i}\to0$ in $C^{\infty}(\Omega).$ then each derivative of $\ f_{i}$ tends to O, uniformly on compact subsets of $\Omega$ ; the follows that Leibniz formula shows therefore that $\psi f_{i} arrow0$ in ${\mathcal{D}}(\Omega).$ ; since $\Lambda\in{\mathcal{D}}^{\prime}(\Omega),$ ,it $\Lambda f_{i}{ arrow}0$ such tha ${\hat{\Pi}}{\mathcal{I}}\in C^{\infty}(\Omega)$ and if $K_{0}$ is any compact subset of $\Omega_{\mathrm{\iota}}$ ,there exists Each $\Lambda\in{\mathcal{D}}^{\prime}(\Omega)$ $\phi\in{\mathcal{D}}(\Omega)$ $\phi=f$ on $K_{0}$ It follows that Q(2) is dense in $C^{\alpha}(\Omega).$ has therefore at most one continuous extension to $C^{\infty}(\Omega)$ $J/J$ Note: In $\mathbf{\Psi}(a)$ it is assumed that $\phi$ b vanishes in some open set containing $S_{\mathrm{A}},$ not merely that $\phi$ vanishes on ${\cal S}_{\Lambda}.$ consists of a single In view of $(b),$ the next simplest case is the one in which $S_{\mathrm{A}}$ point. These distributions will now be completely described. 6.25 Theorem Suppose $\Lambda\in{\mathcal{D}}^{\prime}(\Omega),\,p\in\Omega,\,\{p\}$ is the support o $\Lambda$ ,and $\Lambda$ hs order N. Then there are constants ${\mathcal{C}}_{\alpha}$ such that (1) $$ \Lambda=\sum_{|\alpha|\leq N}c_{x}D^{\alpha}\delta_{p}, $$ where $\delta_{p}$ is the evaluation functional defned by (2) $$ \delta_{p}(\phi)=\phi(p). $$ for all Comversely, every distribution of the form (I) has $D\!\!\!\!/$ for its support (unless $\scriptstyle c_{x}=0$ PROOF It is clear that the support of ${\mathcal{D}}^{\bullet}\partial_{r}$ is {pl}, for every muti-index α. This proves the conversc To prove the nontrivial half of the theorem, assume that $p=0$ (the origin of $\scriptstyle{R^{\prime}}\;,$ and consider a $\phi\in{\mathcal{D}}(\Omega)$ that satisfies (3) (Dp)(0) = 0 for all α with $|\alpha|\leq N$ Our first objective s to prove that (3) implies A4 = 0TEST FUNCTIONS AND DISTRIBUTioNs 151 $15\,\eta>0.$ there isacompact bal $K\subset\Omega.$ with center at O, such that (4) $$ \begin{array}{c c c}{{\lbrack{\cal D}^{x}\phi\rbrack\leq\eta\stackrel{.}{\partial}\rbrack\pi\ K,}}&{{\;\;\;\mathrm{if}\ |\alpha|=N.}}\end{array}N. $$ we claim hat (5) $$ |D^{x}\phi(x)|\leq\eta n^{N-|\alpha|}|x|^{N-|\alpha|}\quad\quad(x\in K,\,|\alpha|\leq N). $$ When $|\alpha|\doteq N$ 、 hsis(4). Suppose $1\leq i\leq N.$ assume $(\,5\,)$ is proved for all a with $|x|=i,$ and suppose $|\beta|=i-1.$ Thc gradient of $D^{\beta}\phi$ is the vector (6) $$ {\bf\hat{\mathrm{erad}}}\ D^{\theta}\phi=(D_{1}D^{\beta}\phi,\ldots,D_{n}D^{\beta}\phi). $$ Our inductionr hypothesis imples tha (7) $$ |(\operatorname{grad}_{,\lfloor}D^{p}\phi)(x)|\leq n\cdot\eta n^{N-i}|x|^{N-i}\qquad(x\in K), $$ $\beta$ and since $(D^{\beta}\phi)(0)=0$ the mean value theorem now shows that (S) holds with in place of . Thus (S) is proved. which islin some neighborhood Choose an auxiliary function $\psi\in{\mathcal{D}}(R^{n}),$ B of ${\mathit{R}}^{n}$ Define or O and whose suport sin the unit bal $\boldsymbol{B}$ " $$ \psi_{r}(x)=\psi\!\left({\frac{x}{r}}\right)\qquad(r>0,\;x\in R^{n}). $$ (8) If ris small enough, the support of $\textstyle\psi_{r}$ lies in $r B\subset K$ By Leibniz' formula (9) $$ D^{\alpha}(\psi_{r}\phi)(x)=\sum_{\beta\leq x}c_{x\beta}(D^{x-\beta}\psi){\binom{x}{-}}(D^{\beta}\phi)(x)\prime^{\vert\beta\vert-\vert\alpha\vert}. $$ lt now follows from(S) that (10) $$ \|\psi_{r}\,\phi\|_{N}\leq\eta C\|\psi\|_{N} $$ as soon as ris small enough; here ${\boldsymbol{C}}$ depends on $r_{\mathit{l}}$ and $N.$ all Since $\Lambda$ has order $N_{\colon}$ there is a constant $C_{1}$ such that $|\operatorname{A}\!\psi|\leq C_{1}||\psi||_{N}$ for $\psi\in{\mathcal{D}}_{K}$ Since $\psi_{r}=1$ in a neighborood of the suport o $\Lambda_{\!_{J}}$ it now follows from(10) and $\left(c\right)$ of Thcorem 6.24 that $$ |\,\Lambda\phi\,|=|\,\Lambda(\vartheta_{r}\phi)\,|\leq C_{1}||\psi_{r}\phi||_{N}\leq\eta C C_{1}||\psi||_{N}\,. $$ Since $\textstyle\eta$ was arbitrary, we have proved that , since $\Lambda\phi=0$ whenever (3) holds functionals In other words, $\Lambda$ vanishes on the intersection of the nul saccs of he $D^{\theta}\partial_{\theta}$ $(|\ x|\leq N)$ (11) (D*50)4 =(-1)/=| 。(D~4p)=(-1)"*(D4p)(O) The representation t follows now from Lemma 3.9. /152 DisrRuoros AND rouERTRANsroxws $$ \begin{array}{c c c c c c c c c c c c c c c c c c c}{{}}&{{}}&{{}}&{{}}&{{}}&{{}}&{{}}&{{}}&{{}}&{{}}&{{}}&{{}}&{{}}&{{}}&{{}}&{{}}&{{}}&{{}}&{{}}&{{}}&{{}}&{{}}&{{}}&{{}}&{{}}&{{}}&{{}}&{{}}&{{}}&{{}}&{{}}&{{}}&{{}}&{{}}&{{}}&{{}}&{{}}&{{}}&{{}}&{{}}&{{}}&{{}}&{{}}&{{}}&{{}}&{{}}&{{}}&{{}}&{{}}&{{}}&{{}}&{{}}&{{}}&{{}}&{{}}&{{}}&{{}}&{{}}&{{}}&{{}}&{{}}&{{}}&{{}}&{{}}&{{}}&{{}}&{}&{{{}}}}&{{{{}}}&{{{}}&{{}}&{{{{}&{{}}}&{{{{{}}}}&{}&{{{{{{{}}}&{}}&{.}&{{{}}}&{{{{{. $$ Distributions as Derivatives Dyf for some continuous function $\boldsymbol{f}$ l was pointed out in the introuction to this chapter that one of the aims f the theory of dstiutions is o nlarge theconcept of function insuch a way that patal i ferentiaions can be carred ou unestrictedl.、The distributions do satisy hi requirement. Conversely-as w shal no see-cery disribution s at east lcall concept is as economical as it posibly can be and some multi-index α.If every continuous fanction is to havepartal derivativs fallodes no proper subclas o h distribu tions can therefore beadequate. I ths s th istiution extension ofthe function 6.26 Theorem Suppose $\Lambda\in{\mathcal{D}}^{\prime}(\Omega),$ and $\textstyle K$ is a compact subset o( . Then there $\dot{\boldsymbol{i}}S$ a continuous function f in S $\mathbf{\hat{G}}$ and there is a muli-index a such that (1) $$ \Lambda\phi=(-1)^{\vert x\vert}\int_{\Omega}f(x)(D^{x}\phi)(x)\,d x $$ for every $\phi\in{\mathcal{D}}_{K}\,,$ cube in $\textstyle R^{n},$ PRoor Assume, without loss of generality, that $K\subset{\mathcal{Q}}$ , where ${\mathcal Q}\,$ is the unit consisting of al ${\mathcal{X}}=(x_{1},\cdot\cdot\cdot,\,{\mathcal{X}}_{n})$ with $0\leq x_{i}\leq1$ for $i=1,\ldots,n.$ The mean value theorem shows that (2) $$ \left|\psi\right|\leq\operatorname*{max}_{x\,s\,Q}\left|(D_{i}\psi)(x)\right|\qquad(\psi\in\mathcal{D}_{Q}) $$ for $i=1,\ldots,$ n. Put $T=D_{1}D_{2}\cdot\cdot\cdot D_{n}.$ For $\nu\epsilon$ let Q(y) denote the subset of ${\cal Q}\,$ in which $x_{i}\leq y_{i}\,(1\leq i\leq n).$ Then (3) $$ \psi(y)=\int_{Q(y)}(T\psi)(x)\;d x\qquad(\psi\in{\mathcal{D}}_{Q}). $$ If ${\boldsymbol{N}}$ is a nonnegativeinter nd if (2 s plie toscesivederivatives o ${\mathcal{Y}},$ (3) eads to the inequalit (4) $$ \|\psi\|_{N}\leq\operatorname*{max}_{x\in Q}|(T^{N}\psi)(x)|\leq\int_{Q}|(T^{N+1}\psi)(x)|\,d x, $$ for every $\psi\in{\mathcal{D}}_{0}$ there exist ${\boldsymbol{N}}$ and ${\boldsymbol{C}}$ such that Sincc $\Lambda\in{\mathcal{D}}^{\prime}(\Omega),$ (5) $$ \big|\,\Lambda\phi\,\big|\leq C||\phi||_{N}\qquad(\phi\in\mathcal{D}_{K}). $$ Hence((4) shows tha $\mathbf{\Psi}(6)$ $$ |\operatorname{A}\phi|\leq C\int_{X}|(T^{N+1}\phi)(x)|\,\,d x\qquad(\phi\in{\mathcal{D}}_{R}). $$AND DISTRIBUTIONS 153 $\boldsymbol{\mathit{I}}$ By (3), ${\boldsymbol{T}}$ is one-to-one on ${\mathcal{D}}_{Q},$ hence on ${\mathcal{D}}_{K}.$ Consequenty $T^{N+1}{\mathrm{:}}$ of $\scriptstyle{T^{\operatorname{A}+1}}$ by sctin is one-to-one. A functiona $\Lambda_{1}$ can therefore be defined on the range ${\mathcal{D}}_{K}\to{\mathcal{D}}_{K}$ (7) $$ \Lambda_{1}T^{N+1}\phi=\Lambda\phi\qquad(\phi\in{\mathcal D}_{K}), $$ and (G) shows that (8) $$ \cdot|\Lambda_{1}\psi|\leq C\int_{\cal K}|\psi(x)|\;d x\qquad(\psi\in Y). $$ $\mathrm{on}\ L^{1}(K)$ The Hahn-Banach theorem thereforextends $\mathrm{A}_{1}$ to a bounded linear functiona such that ln other words, the sabounded Bore functon ${\mathcal{G}}$ on $\textstyle K$ (9) $$ \Lambda\phi=\Lambda_{1}T^{N+1}\phi=\int_{\cal K}g(x)(T^{N+1}\phi)(x)\,d x\qquad(\phi\in{\mathcal D}_{\cal K}). $$ Define $g(x)=0$ outside $K$ and put (10) $$ f(y)=\int_{-\infty}^{y_{1}}\cdot\cdot\cdot\int_{-\infty}^{y_{n}}g(x)\,d x_{n}\cdot\cdot\cdot d x_{1}\qquad(y\in R^{n}). $$ Then is coninus and nintgations b ats showthat ) giv (11) $$ \Lambda\phi=(-1)^{n}\int_{\Omega}f(x)(T^{N+2}\phi)(x)\,d x\qquad(\phi\in{\mathcal{D}}_{K}). $$ This is (1), with $x\doteq(N+2,\ast\cdot,\,N+2),$ except for a possible change in sign // global one: When as comoes uspt euol su proc a e udinto 6.27 Theorem Suppose $\textstyle K$ is compact, ${\mathbf{}}V$ and $\mathbb{Q}$ are open in $R^{n},$ and $K\subset V\subset\Omega.$ Then Suppose also tha $\Lambda\in\mathcal{P}(\Omega).$ that $\textstyle K$ is the spport of 流 $\Omega$ !(one for each multi-index A $\beta$ B with there exit itely many continous unctio $\mathrm{A,}$ , and thut A has order $N.$ $\beta_{i}\leq N+2J o r\ i=1,\cdot\cdot,n)$ with suports i $\ f_{\boldsymbol{\rho}}$ such that ${\mathit{V}},$ (1) $$ \Lambda=\sum_{\alpha}\ D^{\beta}f_{\beta}\,. $$ means thiat The ives are of ouse,to nersod th istritonsense:(0 (2) $$ \Lambda\phi=\sum_{\beta}\,(-1)^{1\beta1}{\int_{\Omega}f_{\beta}(x)(D^{\beta}\phi)(x)\,d x}\qquad[\phi\in{\mathcal{D}}(\Omega)]. $$ PROOF Choose an open set ${\mathcal{W}}$ with compact closure ${\overline{{W}}},$ such that $K\subset W$ and $W\subset V.$ Apply Theorem 6.26 withWin place of K Put $\chi=(N+2,\,\ast\,,\,N+2$ 2)154 DISTRIBUrroNs AND FoURIER TRANSFoRMs The proo f Theorem 6.26 shows that theisa continuous function in Q such that (3) $$ \Lambda\phi=(-1)^{|x|}{\int_{\Omega}f(x)(D^{\alpha}\phi)(x)\;d x}\qquad[\phi\in{\mathcal{A}}(W)]. $$ We may multiply by a continuous function which is l on ${\overline{{W^{\prime}}}}$ and whose support lies in ${\mathit{V}},$ without disturbing (3). Fix $\psi\in{\mathcal{D}}(\Omega),$ with support in $W,$ such that ${\ y}=1$ on some open set containing $K.$ Then (3) implies, for every $\phi\in{\mathcal{D}}(\Omega)_{;}$ that $$ \begin{array}{c}{{\Lambda\phi=\Lambda(\psi\phi)=(-1)^{|\alpha|}\int_{\Omega}f\cdot D^{x}(\psi\phi)}}\\ {{\phantom{\frac{}{\sim}}(-1)^{|\alpha|}\int_{\Omega}f\sum_{\beta\leq x}c_{\alpha\beta}\,D^{x-\beta}\psi D^{\beta}\phi.}}\end{array} $$ This is (2), with $$ f_{\beta}=(-1)^{|x-\beta|}c_{\alpha\beta}f\cdot D^{x-\beta}\psi\qquad(\beta\le\alpha). $$ $it\it i}l/{\it f$ Our next theorem describes the global structure of distributions 6.28 Theorem Suppose $\Lambda\in{\mathcal{D}}^{\prime}(\Omega).$ There exist continuous functions ${\mathcal{G}}_{\alpha}$ in $\Omega_{\mathrm{{,}}}$ ,one for each multi-index ${\mathcal{Q}},$ such that (a) each compact $K\subset\Omega$ intersectsthe suppors of only finitely many ${\mathcal{G}}_{\alpha}:$ and (b) $\Lambda=\sum_{\alpha}D^{\alpha}g_{\alpha}$ If A has finite order, then the functions ${\mathcal{I}}_{\alpha}$ ya can be chosen so that only funitely many are different from O. PROOF infinitely many $V_{i}.$ There are compact cubes $Q_{i}$ and open sets $V_{i}\left(i=1,2,3,\cdot\cdot\right)$ such that sequence $\scriptstyle\{\phi_{i}\}$ to construct a partition of unity $\{\psi_{i}\},$ ;, and no compact subset of $\phi_{i}=1$ on $\textstyle Q_{i}$ intersects $Q_{i}\subset V_{i}\subset\Omega,\mathsf{I}$ 2 is the union of thc ${\mathcal{O}}_{i}$ such that $\mathbb{Q}$ Use this There exist $\phi_{i}\in{\mathcal{D}}(V_{i})$ },as in Theorem 6.20; each $\textstyle\bigvee_{i}$ has its support in $V_{i}.$ $\textstyle\bigvee_{i}\Lambda$ It shows that there are finitely many Theorem $6.27$ applies to each continuous functions $f_{i,\alpha}$ with supports in $V_{i},$ such that (1) $$ \psi_{i}\Lambda=\sum_{\alpha}D^{\alpha}f_{i,\alpha}. $$ Define $\mathbf{\Psi}(2)$ $g_{\alpha}=\sum_{i=1}^{\infty}f_{i,\alpha}.$TEST FUNCTIONS AND DISTRIBrIoNs 155 (1) and (2) give (b). These sms arelocalyfite, in he sense that each compat $K\subset\Omega$ is continuous in S intersets the supports o oly finitely many $\phi\in{\mathcal{D}}(\Omega).$ we have $\Lambda=\sum\psi_{i}\Lambda,$ and therefore ${\mathcal{G}}_{\alpha}$ Since $\mathbb{Q}$ and that (a) holds $f_{i,x}$ .It follows that each ${\boldsymbol{\phi}}=\sum\psi_{i}\,\phi$ , for every The fina asertio follow from Theorem 6.27 // Convolutions (Theorem 6.37) Starting from convolutions of twofunctions w hal nowdefine the convolution o distiutnan tst fnton thn uneimciioS tecoiutono twodistiuons.The aeiportat in the platins o Fure tansforms diferta quions.A hactitipery fonvouins teycmt with translaionsand vith difetitions(Theoems 6.3063 637)ANso, i ferentons may be regard as convoluioswthderivtves o the Dra meau 1 i onen o mak sml hng ntio n o u eue 山。,..….for distibtos as well s for functions 6.29 Definitions In the rcst of this chapter, we shall write ', and $x\in R^{n}.$ $\tau_{x}{\mathcal{U}}$ and $\vec{\mathcal{4}}$ are the func- of ${\mathcal{D}}(R^{n})$ and ${\mathcal{D}}_{\cdot}^{\prime}(R^{n}),$ If ${\boldsymbol{u}}$ is a function in ${\mathcal{D}}$ and ${\mathcal{Q}}^{\prime}$ in place tions defined by $\textstyle R^{n}\!,$ (1) $$ (\tau_{x}u)(y)-u(y-x),\qquad\dot{u}(y)=u(-y)\qquad(y\in R^{n}). $$ Note that (2) $$ (\tau_{x}{\tilde{u}})(y)={\ddot{u}}(y-x)=u(x-y). $$ If and vare complex functions in $\textstyle R^{n}\!\!_{*}$ their convolution $u\approx v$ is defined by (3) $$ (u*v)(x)=\int_{R^{n}}u(y)x(x-y)\,d y, $$ sense. Because ot(2 providedththe negral exs oral r eas fralmost an $\phi\circ{\begin{array}{c}{\chi}\\ {\epsilon}\end{array}}\,$ in the Lebesgue $x\in R^{n},$ (4) $$ (u*v)(x)=\int_{R^{n}}u(y)(\tau_{x}\hat{v})(y)\,d y. $$ This makes it natural to definc (5) $$ (u*\phi)(x)=u(\tau_{x}\<)\;\;\;\;\;\;\;\;(u\in\mathcal{O}^{\prime},\,\phi\in\mathcal{D},\,x\in R^{n}), $$ for if us locly integrablefunction,(S)areswith 4). Note that u pis afunction156 DISTRuBrTIONs AND roURIER TRANSFORMs ) The relation $\ \}\left(\tau_{x}u\right)\cdot v=\textstyle\int u\cdot\left(\tau_{-x}v\right)\!\mathrm{;}$ valid for functions $u_{\textbf{u}}$ and v, makes i natural to define the translate $\tau_{x}\,u$ of $u\in{\mathcal{D}}^{\prime}$ by (6) $$ (\tau_{x}u)(\phi)=u(\tau_{-x}\phi)\qquad(\phi\in\mathcal{D},\,x\in R^{n}). $$ Then, for each $x\in R^{n},\,\tau_{x}u\in\mathcal{D}^{\prime}\,;$ we leave theveifction of the appropriate continuity requirement as an exercise 6.30 Theorem Suppose $u\in{\mathcal{D}}^{\prime}$ $\phi\in{\mathcal{D}},\,\forall\in\mathbb{Z}.\,T h$ en (a) $\tau_{x}(u*\phi)=(\tau_{x}u)*\phi=u*(\tau_{x}\phi)j o r\ a l l\ x\in R^{n};$ (b) $u*\phi\in C^{\infty}$ and $$ D^{\star}(u\ast\phi)=(D^{\star}u)\ast\phi=u\ast(D^{\star}\phi) $$ for every multi-index ${\mathcal{Q}};$ (c) $u*(\phi*\psi)=(u*\phi)*\psi$ PROOF For any $y\in R^{n},$ $$ \begin{array}{l l}{{(\tau_{x}(u*\phi))(y)=(u*\phi)(y-x)=u(\tau_{y-x}\tilde{\phi}),}}&{{\qquad.}}\\ {{((\tau_{x}u)*\phi)(y)=(\tau_{x}u)(\tau_{y}\tilde{\phi})=u(\tau_{y-x}\tilde{\phi}),}}&{{\qquad.}}\\ {{(u*(\tau_{x}\phi))(y)=u(\tau_{y}(\tau_{x}\phi)^{\sim})=u(\tau_{y}(\tau_{x}\phi),}}&{{\qquad.}}\end{array} $$ which gives (a); the relations $$ \tau_{y}\tau_{-x}=\tau_{y^{-}x}\qquad\mathrm{and}\qquad\left(\tau_{x}\phi\right)^{\ v}=\tau_{-x}\tilde{\phi} $$ were used. In the sequel. purely formal calculations sch as the preceding one will sometimes be omitted. If uis applicd to both sides of the identity (1) $$ \tau_{x}((D^{x}\phi)^{\vee})=(-\,1)^{|x|}D^{x}(\tau_{x}\tilde{\phi}) $$ one obtains part of $(b).$ ), namely, $$ (u\ast(D^{\alpha}\phi))(x)=((D^{\alpha}u)\ast\phi)(x). $$ To prove the rest of $(b),$ let $\scriptstyle{\mathcal{C}}$ be a unit vector in $\textstyle R^{n}\!\!_{*}$ ,and put (2) $$ \eta_{r}=r^{-1}(\tau_{0}-\tau_{r e})\qquad(r>0). $$ Then (a gives (3) $$ \eta_{r}(u\ast\phi)=u\ast(\eta_{r}\phi). $$ As $r\to c$ 0, $\eta_{r}\phi arrow D_{e}\phi$ in ${\mathcal{D}}_{s}$ , where $D_{e}$ denoles the directional derivative in the direction e. Hence $$ \tau_{x}((\eta_{r}\phi)^{\times})\to\tau_{x}(D_{e}\phi)^{\times}\operatorname{in}\operatorname{\mathcal{D}}, $$TEST FUNcrIONs AND DISTRIBUTIONs 157 . for each $\textstyle X\in N^{n}$ , So that (4) $$ \operatorname*{lim}_{r arrow0}\,(u*(\eta_{r}\phi))(x)=(u*(D_{c}\phi))(x). $$ By (3) and (4) we have (5) $$ D_{e}(u*\phi)=u*(D_{e}\phi), $$ and itcration of (S) gives (b). To prove (c), we begin with the identity (6) $$ {^\cdot}(\phi*\psi)^{\times}(t)=\int_{R^{n}}\hat{\mathcal V}(s)(\tau_{s}\tilde{\phi})(t)\,d s. $$ $\operatorname{tet}K,$ and $K_{2}$ be the supports of $\bar{\phi}$ and ${\tilde{\mathcal{V}}}.$ Put $K=K_{1}+K_{2}$ Then $$ .\cdot\cdot\mathrm{~\\\\}\cdot\qquad s\to\tilde{\psi}(s)\tau_{s}\tilde{\phi} $$ is a continuous mapping of $R^{n}$ into ${\mathcal{Q}}_{k}$ which is $\mathbf{0}$ outside $K_{2}$ .Therefore (6) may be writen as a $\sigma_{\kappa}\,\!.$ valued integral, namely, (7) $$ (\phi*\psi)^{\times}=\int_{\cal K_{2}}\tilde{\psi}(s)\tau_{s}\,\tilde{\phi}\,\,d s, $$ and now Theorem $3.27$ shows that $$ \begin{array}{l l}{{(u*(\phi*\psi))(0)=u((\phi*\psi)^{\sim})}}&{{}}\\ {{.}}&{{=\int_{R^{n}}\bar{\psi}(s)u(\tau_{s}\tilde{\psi})\;d s=\int_{R^{n}}\!\psi(-s)(u*\phi)(s)\;d s,}}\end{array} $$ or (8) $$ (u\ast(\phi\ast\psi))(0)=((u\ast\phi)\ast\psi)(0). $$ To obtain (8) with $\scriptstyle{\mathcal{X}}$ in place of O, apply (8) to $\tau_{-x}\psi$ in place of ${\mathcal{Y}},$ and appeal to (a). This proves (c) $///\hbar$ 6.31 Definition The term approxiate identity on $R^{n}$ will denote a scqucnce of functions $h_{j}$ ; of the form $$ h_{j}(x)=j^{n}h(_{j}(\,j x)\qquad(\,j-1,\,2,\,3,\dots). $$ where $h\in{\mathcal{D}}(R^{n}),\,h\geq0,$ and $\textstyle\bigcap_{R^{n}}h(^{*}\!x\!^{})\,d x=1.$ 6.32 Theorem Suppose $\textstyle|h_{j}\rangle$ is an approximate identiy o $R^{n},\,\phi\in{\mathcal A},\,a n d\,u\in\mathcal D^{\prime}.$ Then (a)lim 中* $h_{j}=\phi$ in 9, j→ $(b)\quad\mathop\mathrm{^\gamma_{\mathrm{i} arrow\infty}}~u*h_{j}=u\,i n\,\mathcal{D}^{\prime}.$158 usruorros MoD rou TRAsrow Note that $\mathbf{\nabla}(b)$ impliestat ever ditributon is lit in th tology o ${\mathcal{D}}^{\prime}{}_{i}$ of a sequence of ifinitelydifferentablefunctions sets,1 we see that some compact set sinc te supports of the $h_{j}$ Applying this to $D^{z}\phi$ uniformly on compac lie in PRO0F lt is a trivial exercise to check that $\mathbf{\Psi}(c)$ of Theorem 6.30 give (b), because $\{0\}.$ This gives $(a).$ $\hat{\Gamma}\int\mathbf{i}\mathbf{s}$ any continuous function on $f*h_{j}\to f$ in place of f $\textstyle R^{n}.$ $D^{x}(\phi*h_{j})\to D^{x}\phi$ uniformlyAso, the supports of al $\phi\star h_{j}$ shrink to Next, (a) and statcmcnt $$ \begin{array}{c}{{u(\tilde{\phi})=(u\ast\phi)(0)=\mathrm{i}\mathrm{i}\mathrm{Im}\ (u\ast(h_{j}\ast\phi))(0)}}\\ {{}}\\ {{=\mathrm{i}\mathrm{Im}\ ((u\ast h_{j})\ast\phi)(0)=\mathrm{i}\mathrm{i}\mathrm{i}\mathrm{m}\left(u\ast h_{j}\right)(\tilde{\phi}).}}\end{array} $$ $l/{\big/}{\big/}$ 6.33 Theorem (a) $I f u\in{\mathcal{D}}^{\prime}$ and (1) $$ L\phi=u*\phi\qquad(\phi\in\mathcal{D}), $$ then $\boldsymbol{\ L}$ .is a continuous linear mapping of $\mathcal{Q}$ D into $C^{\infty}$ which satisfies (b) (2),then there is a unique u $\scriptstyle\pi\,{\mathcal{O}}^{\prime}$ $$ \tau_{x}L=L\tau_{x}\qquad(x\in R^{n}). $$ $\mathbf{(}1)$ holds and if $\underline{{L}}$ saisie (2) Comersely,if Lis a coninuous linear mapping of $\mathcal{Q}$ into $C(R^{n}),$ such that Note that (b) implies that the range of ${\boldsymbol{L}}$ actually lies i $C^{*}.$ PROOF(a) Since continuous, we have to show that the restriction ot $\boldsymbol{\mathit{L}}$ $u*\phi_{i} arrow f$ in To prove that $\boldsymbol{\mathit{L}}$ is be applid. Suppose that ${\partial}_{i} arrow{\partial}$ in ${\mathcal{D}}_{K}$ and that (1) implies (2). to each ${\mathcal{D}}_{K}$ is a continuous $\tau_{x}(u\ast\phi)=u\ast(\tau_{x}\phi),$ mapping into $C^{\bullet},$ Sinc these are Fréchetspaces,th closed graph theorem can ; we have to $\textstyle{C^{*}}$ prove $\mathrm{that}\,f=u*\,\phi.$ $\tau_{x}\,\tilde{\phi}_{i}\to\tau_{x}\,\tilde{\phi}$ in ${\mathcal{Q}},$ so that $\operatorname{Fix}\,x\in R^{n}.$ Then $$ f(x)=\operatorname*{lim}\;(u*\phi_{i})(x)=\operatorname*{lim}u(\tau_{x}{\tilde{\phi}}_{i})=u(\tau_{x}{\tilde{\phi}})=(u*\phi)(x). $$ since evaluation at (b) Dcfinc udp) =(Lの)XO). Since $\phi arrow\bar{\phi}$ is a continuous operator on ${\mathcal{Q}},$ ,and $\mathbf{0}$ his a continuous linear functional on ${\cal{C}},$ u is continuous on ${\mathcal{Q}}.$ Thus $u\in{\mathcal{D}}^{\prime}$ Since $\boldsymbol{\ L}$ satisfies (2) $$ \begin{array}{l}{{(L\phi)(x)=(\tau_{-x}I\phi)(0)=(L\tau_{-x}\phi)(0)}}\\ {{{}=u(\tau_{-x}\phi)^{\sim})=u(\tau_{x}\[\phi)=(u*\phi)(x).\quad.}}\end{array} $$ The uniqueness of $\boldsymbol{u}$ is obvious for ir $u\in{\mathcal{D}}^{\prime}$ and $u*\phi=0$ for every $\phi~~~~~~~~~~~~~~~~~~~~~~~~~~~~~~~~~~~~~~~~~~~~~~~~~~~~~~~~~~~~~~~~~~~~~~~~~~~~~~~~~~~~~~~~~~~~~~~~~~~~~~~~~~~~~~~~~~~~~~~~~~~~~~~~~~~~~~~~~~~~~~~~~~~~~~~~~~~~~~~~~~~~~~~~~~~~~~~~~~~~~~~~~~~~~~~~~~~~~~~~~~~~~~~~~~~~~~~~~~~~~~~~~~~~~~~~~~~~~~~$ e 9, then $$ u(\bar{\phi})=(u*\phi)(0)=0 $$ for every pe 9;hence $u=0.$ $J/{\big/}$TEST FUNcnoNS AND DISTRIBUTIoNs 159 6.34 Definition Suppose now that $u\in{\mathcal{D}}^{\prime}$ and that ${\boldsymbol{u}}$ has compact support. By by thc samc formulasbefore, namely Theorem 6.24, uextends thn in a unique fashion to coninos inia rnctona o and any $\phi\in C^{\infty}$ C". One can therefore define the convolution of $\bar{u}$ $$ (u*\phi)(x)=u(\tau_{x}\tilde{\phi})\qquad(x\in R^{n}). $$ 6.35 Theorem Suppose $u\in{\mathcal{D}}^{\prime}$ has compae suort a $\phi\in C^{\infty}.$ $\scriptstyle{T_{h i n}}$ (a) $\tau_{x}(u*\phi)=(\tau_{x}u)_{,}*\phi=\dot{u}*(\tau_{x}\phi)~i f\,x\in R^{n},$ $$ D^{\star}(u\ast\phi)=(D^{\star}u)\ast\phi=u\ast(D^{\star}\phi). $$ (b)u $\star\phi\in C^{\infty}$ and 1/, in ditio, $\textstyle\bigcup\in{\mathcal{D}},$ then (c) $u*\psi\in\varnothing,$ and (d) $u*(\phi*\psi)=(u*\phi)*\psi=(u**\phi)*\psi=(u**\psi)*\phi.$ $\boldsymbol{\mathit{u}}$ and rRoor The proofs of a and (b re so silar to those given in Theorem 6.30 V is $x-H.$ K and ${\boldsymbol{H}}$ I be the supports of that they need not be repeated. To prove Cc), let $\textstyle K$ ${\mathit{l}}_{\cdot}$ respectively. The support of $\mathbf{r}_{x}\psi$ Therefore · $$ (u*\psi)(x)=u(\tau_{x}\hat{\psi})=0 $$ unless ${\boldsymbol{K}}$ intersects $x-H,$ thatis, unles $x\in K+H.$ The suport o $u*\psi$ thus lies in the compact set $K+H$ $\phi_{0}\in{\mathcal{D}}$ To prove $(d).$ ), let $\mathcal{W}$ be a bounded open set that contains $K_{\mathrm{{J}}}$ and choose so that ${\dot{\phi}}_{0}={\ddot{\phi}}$ in $W+H.$ Then $(\phi*\psi)^{\vee}=(\phi_{0}*\psi)^{\vee}$ in ${\mathcal{N}},$ so that (1) $$ (u*(\phi*\psi))(0)=(u*(\phi_{0}*\psi))(0). $$ If $\quad s-s\in H,$ then $\tau_{s}\phi=\tau_{s}\phi_{0}$ in $W\colon$ hence u $\iota\ast\phi=u\ast\phi_{0}\ \mathrm{in}-H.$ This give (2) $$ ((u*\phi)*\psi)(0)=((u*\phi_{0})*\psi)(0). $$ Since the support of $u*|/$ lies in $K+H,$ (3) $$ ((u*\psi)*\phi)(0)=((u*\psi)*\phi_{0})(0). $$ Theorem 6.30. The right sides of (I) to(3) are equal, by Theorem 6.30; hence so are are equal at the // theirlft sids. This proves that thethreeconvolutions in $(d)$ orign. The genera ase follows y translation, as at the cnd ot the proof“o 6.36 Definition If $u\in{\mathcal{D}}^{\prime},\ \ v\in{\mathcal{D}}^{\prime},$ and at least one of these two distributions has compact support, define (1 $L\phi=u*(v*\phi)\qquad(\phi\in\mathcal{D}).$160 pisrRumurioNs ANpD rouRuR TRANsrows $$ \begin{array}{l l l l l l l}{{\stackrel{}{\downarrow}}}&{{}}&{{}}&{{}}&{{}}&{{}}&{{}}&{{}}\\ {{\downarrow}}&{{}}&{{}}&{{}}&{{}}&{{}}&{{}}&{{}}\\ {{\downarrow}}&{{}}&{{}}&{{}}&{{\downarrow}}\\ {{}}&{{}}&{{}}&{{}}&{{}}&{{}}\end{array} $$ in ${\mathcal{D}}.$ By $v*{\tilde{\phi}}_{i}\to0$ in ${\mathcal{Q}}.$ Note that this s well defined Forif n has compact support,.then $v*\phi\in{\mathcal{D}},$ and then $\tau_{x}L=L\tau_{x}.$ , for all $x\in R^{n}$ has compact support, then again $L\phi\in C^{\infty},$ since $v*\phi\in C^{\infty}$ AIso, $L\phi\in C^{\infty};$ ; if ${\mathcal{U}}$ The asetons follow from Theorems 6.30 and 6.35 $\phi_{i}\to0$ $\lor\left(a\right)$ of Theorem lt follows in either case, that is in fact a distribution. To se this, suppose The functional $\phi arrow(L\tilde{\phi})(0)$ in addition,v hascompact support $6.33,\,v\ *{\tilde{\phi}}_{i} arrow0$ in $C^{\infty}:{\mathbb{I}},$ $(L{\tilde{\phi}_{i}})(0) arrow0.$ The roof o (6) ofTherem .3 no show thaths stibuton. which w shall denote by $\ ^{\prime}\,u\Vdash v,$ is related to $\boldsymbol{\mathit{L}}$ by the formula (2) $$ L\phi=(u*v)*\phi~~~~~~~(\phi\in\mathcal{D}). $$ ln other words, $u*v\in\varnothing$ is characterized by (3) $$ (u\ast v)\ast\phi=u\ast(v\ast\phi)\qquad(\phi\in\mathcal{D}). $$ 6.37 Theorem Suppose $u\in{\mathcal{D}}^{\prime},$ $v\in{\mathcal{D}}^{\prime},$ $w\in{\mathcal{D}}^{\prime}.$ (b) $U S_{s}$ and $\textstyle S_{v}$ lf a least one of u, has compac supor, then $u*v=v*u.$ s compact, then (a) , are thesppors ofuan ${\boldsymbol{v}},$ and i a last one f these $\dot{\boldsymbol{i}}{\boldsymbol{S}}$ $$ S_{u\ast v}\subset S_{u}+S_{v}\,. $$ (c)IJa least two of the supports $S_{u},\,S$ $S_{\mathrm{w}}$ are compact $t h e n\left(u\ast v\right)*w=u*(v*w).$ (d)I了 is the Dirc measure and a is a multi-idex, then $$ D^{x}u=(D^{x}\delta)*u. $$ In particular, $u=\delta*u.$ $S_{u}\,\colon$ $S_{v}$ iscompac, lhen (e)If at least one of the sets $$ D^{x}(u\ast v)=(D^{x}u)\ast v=u\ast(D^{x}v) $$ for every muli-index c Note: The associativelw (co dends surongly on the stated hypoteses; sc Exercise 24 $\left(c\right)$ PROOF $\mathbf{\Psi}(a)$ Pick $\phi\in{\mathcal{D}},$ $\psi\in{\mathcal{D}}.$ Sinceconvolution of functions is commutative of Theorem 6.30 implies that $$ \begin{array}{l}{{(u\ast v)\star(\phi\star\psi)=u\ast(v\ast(\phi\star\psi))}}\\ {{=u\ast((v\ast\phi)\star\psi)=u\ast(\psi\ast(v\ast\phi)).}}\end{array} $$ $\xi_{\epsilon}$ is compact,appy (e of Theore 6.30 once more i $\textstyle|S_{x}$ is compact, apply $(d)$ of Theorem $6.35;$ in either case () $$ (u\ast v)*(\phi*\psi)=(u*\psi)*(v*\phi). $$TEST FUNCTioNs AND DISTRuIurIoNs 161 Since $\phi*\vartheta\not rightarrow\not\psi\cong\not\psi*\not\phi,$ , the same computation gives (2) $$ (v*u)*(\phi*\psi)=(v*\phi)*(u*\psi). $$ in $\varnothing\varnothing.$ one in $C^{\infty});$ The two right members of $\mathbf{(}|\,\rangle$ and $\mathbf{\Psi}(2)$ are convolutions of functions (one hencc they are equal. Thus (3) $$ ((u\ast v)\ast\phi)\ast\psi=((v\ast u)\ast\phi)\ast\psi\,. $$ Theorem 6.33 now give Tvoapiso o uesarumn ue a h end o he pooro $u*v=v*u$ (b)If $\phi\in{\mathcal{D}},$ a simple computation gives (4) $$ (u*v)(\phi)=u((v*\bar{\phi})^{\vee}). $$ $(u\ast c)(\phi)=0$ unless $S_{u}$ By (ay we my assum,withoutloss of geraliy tat that ${\mathfrak{k}}_{x}$ unless ,is compact. The proof By (4), (c) of (c) of Theore 6.35 shows tht the upport o $S_{v}$ lies in $S_{v}-S_{\phi}\,.$ $S_{u}+S_{v}\,.$ $v*{\tilde{\phi}}$ intersects $S_{\phi}-S_{v}\,.$ $S_{\phi}$ intersects We conclude from (b) that both $$ (u\ast v)\ast w\qquad\mathrm{aId}\qquad u\ast(v\ast w) $$ are defined if at most one of the set $S_{u},$ $S_{\nu}$ $S_{\mathrm{w}}$ fals to be compact. 1f $\phi\in{\mathcal{D}},$ it folow drcty from Defnion 6.36 tha (5) $$ (u\ast(v\ast w))\ast\phi=u\ast((v\ast w)\ast\phi)=u\ast(v\ast(w\ast\phi)). $$ If $S_{\mathrm{w}}$ is compact, then (6) $$ ((u*v)*w)*\phi=(u*v)*(w*\phi)=u*(v*(w*\phi)) $$ because w * 4 $\phi~~~~~~~~~~~~~~~~~~~~~~~~~~~~~~~~~~~~~~~~~~~~~~~~~~~~~~~~~~~~~~~~~~~~~~~~~~~~~~~~~~~~~~~~~~~~~~~~~~~~~~~~~~~~~~~~~~~~~~~~~~~~~~~~~~~~~~~~~~~~~~~~~~~~~~~~~~~~~~~~~~~~~~~~~~~~~~~~~~~~~~~~~~~~~~~~~~~~~~~~~~~~~~~~~~~~~~~~~~~~~~~~~~~~~~~~~~~~~~~$ e 2, by Cc of Teorem 6.35. Comparison or $\mathbf{\nabla}(S)$ and (G) givcs (e) whenever $S_{\mathrm{w}}$ is compact If $S_{\mathrm{w}}$ is not compact, then $S_{u}$ is compact,and te peceding case, combined with the commutative law (a) gives $$ \begin{array}{l l}{{u*(v\ *w)\stackrel{-}{-}u\ *(w\ *v)=(w\ *v)*u}}\\ {{}}&{{=w\ *(v\ *v)=w\ *(u\ *v)=(u\ *v)\ *}}\\ {{}}&{{=}}&{{w\ *(v\ *v)=(u\ *v)\ *}}\end{array} $$ W. (d)If $\phi\in{\overrightarrow{X}},$ then $\delta\ast\phi=\phi.$ because $$ (\delta\ast\phi)(x)=\delta(\tau_{\star}\tilde{\phi})=(\tau_{x}\tilde{\psi})(0)\dots\tilde{\phi}(-x)=\phi(x). $$ Hence Ce) above and (b) of Theorem 6.30 giv $$ (J)^{x}\iota)*\phi=u*D^{x}\phi=u*D^{x}(\delta\ast\phi)=u*(D^{x}\delta)*\phi. $$ Finally, $\mathbf{\Psi}({\boldsymbol{e}})$ follows from Gd),(c), and GaJ: and $$ D^{x}(u*v)=(D^{x}\delta)*(u*v)=((D^{x}\delta)*_{i}u)*v=(D^{x}u)*v $$ (0D*)*0)*0=(u*D*8)*0=u*(D*6)*0)=u* D $J/{\big/}$162 DISTRIBUTIONS AND FOURIER TRANSFORMS Exercises Suppose $\boldsymbol{\f}$ f is a complex continuous function in $\textstyle R^{n}\!\!_{!}$ with compact support. Prove that $\psi P_{J} arrow f$ uniformly on $\textstyle R^{n}\!.$ , for some $\psi\in{\mathcal{D}}$ and for some sequence $\scriptstyle(p_{i})$ of polynomials Show that the metrizable topology for ${\mathcal{D}}(\Omega)$ that was rejected in Section $6.2$ is not com- plete for any Q2 3f Eis an arbitrary closed subset of $\textstyle R^{n},$ show that there is an fe $C^{\infty}(R^{n})$ such tha $\,\mathrm{t}f(x)=0$ for every $x\in E$ and $f(x)>0$ for every other $x\in R^{n}.$ Suppose $\Lambda\in{\mathcal{D}}^{\prime}(\Omega)$ and $\operatorname{A}\!\phi\cong0$ whenever $\phi\subset{\mathcal{D}}(\Omega)$ and $\scriptstyle\phi\geq0$ Prove that $\mathbf{A}$ is then a positive measure in $\Omega$ (which is finite on compact sets) Prove that the numbers $c_{\alpha\beta}$ in the Leibniz formula are $$ c_{x\beta}\:=\:\prod_{i=1}^{n}\:\frac{\alpha_{i}\:\mathrm{i}}{\beta_{i}\:\mathrm{l}\,(\alpha_{i}\:-\:\beta_{i})!} $$ 6(a) Suppose $c_{m}=\exp\left\{{}\right.$ -(m)!, $m=0,1,2,\dots.$ Does the series $$ \sum_{m=0}^{\infty}c_{m}(D^{m}\phi)(0) $$ converge for every $\phi\in C^{\alpha}(R)^{\gamma}$ (b) Let $\mathbb{Q}$ 2 be open in $\textstyle{R^{n}}_{;}$ suppose $\Lambda_{i}\in{\mathcal{D}}^{\prime}(\Omega),$ and suppose that al $\mathrm{A}_{i}$ have their supports in some fixed compact $K=\Omega_{\circ}$ Prove that the sequence {A} cannot converge in ${\mathcal{A}}^{\prime}(\Omega)$ unless the orders of the $\mathrm{\A}_{j}$ are bounded. Hint: Use the Banach-Steinhaus theorem. (c) Can the assumption about the supports be dropped in (b)? Let $\Omega=(0,$ co). Define $$ \Lambda\phi\,=\sum_{m=1}^{\infty}(D^{m}\phi)\biggl(\frac{1}{m}\biggr)\qquad[\phi\in\mathcal{D}(\Omega)]. $$ Prove that $\Lambda$ is a distribution of infinite order in $\Omega.$ Prove that $\Lambda$ cannot be extended 8 to a distribution in $R;$ that is, there exists no $\Lambda_{0}\in{\mathcal{D}}(R)$ such that $\Lambda_{0}=\Lambda$ in (0, oo). Characterize all distributions whose supports are finite sets O (a) Prove that a set $E\subset{\mathcal{D}}(\Omega)$ is bounded if and only if $$ \operatorname*{sup}\left\{|\operatorname{A}\!\phi|\!:\phi\in E\right\}<\infty $$ for every $\Lambda\in{\mathcal{D}}^{\prime}$ S2D. (b) Supposc $\scriptstyle(\phi_{3})$ is a sequence in $\scriptstyle{\mathcal{R}}(\mathbf{a})$ such that $\{\Lambda\phi_{j}\}$ is a bounded sequence of numbers for every ${\mathfrak{I}}\in{\mathcal{D}}^{\prime}(\Omega).$ Prove that some subsequence of $\scriptstyle(\phi_{i})$ converges, in the topology of ${\mathcal{A}}(\Omega).$ (c) Suppose $\scriptstyle\{\Lambda_{\lambda}\}$ is a sequence in ${\mathcal{D}}^{\prime}(\Omega)$ such that $\{\Lambda_{J}\,\phi\}$ is bounded, for every $\phi\in{\mathcal{P}}(\Omega).$ Prove that sorne subsequence of ${\mathcal{D}}_{K}$ are equicontinuous. Apply Ascoli's theorem and that the convergence is the restrictions of the A,to $\scriptstyle[\Lambda_{\mathcal{M}}]$ converges in ${\mathcal{D}}^{\prime}(\mathbf{d})$ uniform on every bounded subset of 9(D). Hin. By the Banach-Steinhaus theorem,TEST FUNCTIONS AND DISTRIBUrIoNs 163 $I O$ Suppose $\{f_{i}\}$ is a sequence of ocally ntegrable functions in $\Omega$ (an open set in ${\boldsymbol{R}}^{n}$ and $$ \operatorname*{lim}_{t arrow\infty}\;\int_{\cal K}\;|f_{i}(x)|\;d x=0 $$ index α. for every compact $K\subset\Omega.$ Prove that the $D^{\alpha}f_{i} arrow0\;\mathrm{in}\;\mathcal{D}^{\prime}(\Omega),$ as $i arrow\infty,$ , for every multi- ${\mathcal{L}}{\mathcal{I}}$ Suppose $\Omega$ is opcn in $\textstyle R^{2},$ and $\{f_{i}\}$ is a sequence of harmonic functions in $\mathbb{Q}$ that con- verges in the distribution sense to some $\Lambda\in{\mathcal{D}}^{\prime}(\Omega);$ explicitly, the assumption is that $$ \Lambda\phi=\operatorname*{lim}_{i arrow\infty}\int_{\Omega}f_{i}(x)\phi(x)\,d x\qquad[\phi\in\mathcal{D}(\Omega)]. $$ harmonic function. Hin: If $\boldsymbol{\mathit{f}}$ is harmonic converges uniformly on every compact subset of is the average of $\boldsymbol{\mathit{f}}$ over small circles $\mathrm{\A}$ is a Prove then that $\{f_{i}\}$ $\Omega$ and that ${\mathcal{F}}(x)$ centered at x ${\cal{I}}{\cal{2}}$ that a function $f\in C^{\infty}(R)$ Recallthat (the Dirac measurey isthe distributiondcfincd by $,\,\delta(\phi)=\phi(0),\,\mathrm{for}\,\phi\in\mathcal{D}(R).$ al- For which fe $C^{*}(R)$ is it true $\mathrm{that}f\delta^{\prime}=0^{\prime}$ Answer the same question for f8”. Conclude $\Lambda\in{\mathcal{D}}^{\prime}(R)$ though may vanish on the support of a distribution $\textstyle/\Lambda\neq0$ 13 If $\phi\in{\mathcal{B}}(\Omega)$ and $\Lambda\in{\mathcal{D}}^{\prime}(\Omega),$ does either of the statements · $$ \begin{array}{c c c}{{\phi\Lambda=0,}}&{{\ \ \ \Lambda\phi=0}}\end{array} $$ imply the other ? 14 in $K,$ vanishes on $K.$ $K\subset V\subset\Omega,$ $\textstyle K$ is compact, ${\mathbf{}}V$ $\textstyle{\mathcal{R}}^{n}\!,$ Find other sets $\boldsymbol{K}$ has its support in $K_{\mathrm{\scriptscriptstyle{J}}}$ and fe $C^{\infty}(R^{n})$ Suppose $\textstyle K$ is the closed unit ball in $\Lambda\in{\mathcal{D}}^{\prime}(R^{n})$ and $\{\phi_{i}\}\subset{\mathcal{D}}(\Omega)$ satisfie and $\mathbb{C}$ are open in ${\textstyle\bar{R}}^{n}.$ K for which this is true.(Compare has its support Prove t $\operatorname{hat}f\Lambda=0.$ 15 Suppos with Exercise 12.) $\Delta\in{\mathcal{D}}^{\prime}(\Omega)$ (a) $$ \operatorname*{lim}_{i arrow\infty}\;\left[\operatorname*{sup}_{s^{*}\textrm{c v}}|(D^{x}\phi_{i})(x)|\right]=0 $$ for every rmulti-index α. Prove that then $$ \operatorname*{lim}_{i arrow\infty}\;\Lambda(\phi_{i})=0. $$ such that $\textstyle\sum c_{J}<\alpha\,;$ 16 The prceding statement becomes false f is eplaced by $\boldsymbol{K}$ in the hypothesis $(a).$ Show this by means of the following example, in which $\scriptstyle\mathbf{\hat{e}}=R$ Choose $c_{1}>c_{2}>\cdot\cdot\cdot>0,$ define $$ \Lambda\phi={}_{\stackrel{\infty}{\sim}}^{\infty}(\phi(c_{j})-\phi(0))\qquad(\phi\in{\mathcal{D}}(R)); $$ and consider functions $\phi_{i}\in{\mathcal{D}}(R)$ such tha $:\phi_{i}(x)=0\mathrm{if}\ x\leq c_{i+1},\,\phi_{i}(x)=1/i\mathrm{if}\,c_{i}\leq x\leq c_{1}.$ Show also that this Aisa distrihution of order $\mathbf{I}.$ in the hypothesis a of Exercise However, for certain K, ${\mathbf{}}V$ can be replaced by $K$164 DISTRuBurIONs AND roURER TRANSFroRMs 15. Show that this s so when Kis the closed unitball o $\textstyle R^{n}\!,$ Find other sets T $\textstyle K$ K for which $I{\mathcal{B}}$ 21 T in this is true ${\boldsymbol{C}}.$ One may regard $\alpha.$ Prove that $\Delta\phi=0.$ Snggestion: D i frstfr distriutons wit $\mathbf{A}$ A and for If $\mathrm{\A}$ 20 Let $C^{\infty}(T)!$ is a distribution with compact support in as the subspace of $C^{\infty}(R)$ for some continuous function $f.$ ${\mathit{I}}{\mathit{I}}$ If $\Lambda=\delta,$ what are the possibities fo $f^{\gamma}$ $\Omega\,;$ $\Lambda=D^{N+2}f,$ for every $\scriptstyle{\mathcal{X}}$ in the support of where $\Lambda\in{\mathcal{D}}^{\prime}(R)$ has order $N,$ show that Exprcss $\delta\in{\mathcal{D}}^{\prime}(R^{2})$ Sin the frm given by Theorem 6.27, as xplicil s you ca ${\mathcal{I}}{\mathcal{I}}$ Suppose $\Lambda\in{\mathcal{D}}^{\prime}(\Omega),\,\phi\in{\mathcal{D}}(\Omega),$ and $(D^{x}\phi)(x)=0$ is of the form $f\to\Lambda f,$ every multi-index $C^{\infty}(\Omega)$ consisting of those functions compact support. by te method usedin Theorem 6.2 this is aconverse to d) of Theorem 6.24. Prove that every continuous lincar functional on be e sieoeietyieitnaicomope futon t int icl $C^{\infty}(T)$ that have period 2m. Suppose $$ f(z)=\sum_{n=0}^{\infty}a_{n}z^{n} $$ converges in the open unit disc $U$ in ${\boldsymbol{C}}$ Prove that ach ofthe following three roperties of fimplies the other two: and $\gamma<\infty$ such tha (a) There exist $p<\infty$ $$ \vert a_{n}\vert\leq\gamma\cdot n^{p}\qquad(n=1,\,2,\,3,\,\cdot\cdot). $$ () There exis $p<\,\infty$ and $\gamma<\infty$ such tha $$ |f(z)|\leq\gamma\cdot(1-|z|)^{-p}\qquad(z\in U). $$ (e) H $\mathrm{im}_{r\to1}\int_{-\pi}^{\pi}f(r e^{i\theta})\phi(e^{i\theta})\,d\theta$ exists (as a complex number) for every $\phi\in C^{\infty}(T).$ 22 For u $\in{\mathcal{P}}(R),$ show that $$ {\frac{n-\tau_{x}u}{x}}\to D u\qquad\textrm{i n}{\mathcal{D}}^{\prime}(R), $$ 23 Suppose $\{f_{i}\}$ (The derivative of ${\boldsymbol{u}}\,$ , may thustil eregarded as a limit ot quotients) ;, such that as $x\div0$ is a sequence of ocally intcgrable functions i ${\boldsymbol{R}}^{n}$ $$ \operatorname*{lim}_{t arrow\infty}\ \left({f_{t}}*\phi\right)\!(x) $$ 24 Let ${\boldsymbol{H}}$ exists, for each formy o compact ses forch mult-index $\textstyle\alpha.$ Prove that then $\{D^{\bullet}(f_{i}\bullet\phi)\}$ converges uni $\phi\in{\mathcal{D}}(R^{n})$ and each $x\in R^{n}.$ be the Heaniside function on ${\boldsymbol{R}},$ defined by $$ H(x)= \{\mathrm{I}\qquad\mathrm{if}\ x\geqslant0, $$ and let $\delta$ be the Dirac measure 4(s) ds, if $\phi\in{\mathcal{D}}(R)$ (a) Show t $\operatorname{lat}\left(H*\phi\right)(x)=\operatorname{li}_{-\infty}$ $\mathbf{\nabla}(b)$ Show that $\delta^{\prime}\ast H=\delta$ Ya Showthat .8*=0 Gere enot ecl ntrae fctowhs valei Tat every point and which is thought of as distribution)TEST FUNCToNs AND DBsSTRusurrous 165 (dO tflows that th sociative law fail $$ 1*(\delta^{\prime}*H)=1*\delta=1, $$ but $$ (1\ast\delta^{\prime})\ast H=0\ast H=0. $$ is a continuous linear mapping of $\mathcal{Q}$ into 2s Here i anoerhatio coltins aous o Theore .3S upse $C^{\infty}$ which commutes with every $D^{\alpha},$ that is, (a) $$ L D^{x}\phi=D^{x}L\phi~~~~~~(\phi\in\mathcal{D}). $$ Then thereis $u\in{\mathcal{D}}^{\prime}$ such that $$ L\phi=u*\phi. $$ Suggestion ${\mathrm{Fix}}\ \phi\in{\mathcal{D}},$ put $$ h(x)=(\tau_{-x}\,L\tau_{x}\,\phi)(0)=(L\tau_{x}\,\phi)(x)\qquad(x\in R^{n}), $$ let $D_{e}$ be th itona ervsus ro orTeorem 6. an show tha $$ (D_{e}h)(x)=(D_{e}L\tau_{x}\phi)(x)-(L\tau_{x}\,D_{e}\phi)(x), $$ 26 which is O if $\mathbf{\Psi}(a)$ holds. Thus $h(x)=h(0),$ which implies that $\tau_{x}L=L\tau_{x}$ Can the assumption that the range of c)) for every $\delta>0,$ definc its picpl vlue inegal to be $J\mathbf{f}f\in L^{1}((-\infty,-\delta)\cup(\delta,$ $\boldsymbol{\mathit{L}}$ is in $C^{\infty}$ be weakened ? $$ P V\int_{-x}^{\infty}f(x)\,d x=\operatorname*{lim}_{\delta\to0}\left(\int_{-\infty}^{x}+\int_{\delta}^{\infty}\right)f(x)\,d x, $$ if thc limit exis. For $\phi\in{\mathcal{D}}(R),$ put $$ \Lambda\phi=\int_{-\infty}^{\infty}\phi(x)\log\ |x|\ d x. $$ Show that $$ \Lambda^{\prime}\phi=P V\int_{-\infty}^{\infty}\phi(x)\,{\frac{d x}{x}}\;, $$ $$ \Lambda^{\prime\prime}\phi=-P V\int_{-\infty}^{\infty}\frac{\phi(x)\,\cdot\phi(0)}{x^{2}}\,d x. $$ (b) 27 Find all ditributions ue for ver $x\in R^{n},$ that atisy leaston of th following two conditions (a) ${\mathcal{D}}^{\prime}(R^{n})$ $\tau_{x}.u=u$ $D^{*}u=0$ for every $\textstyle\alpha$ with $|\alpha|=1.$