CHAPTER SEVENTEEN H'-SPACES This chapter is devoted to the study of certain subspaces of $H(U)$ which are defined by certain growth conditions; in fact, they are all contained in the class ${\mathbf{}}N$ defined in Chap. 15.These so-called $H^{p}.$ spaces (named for G. H. Hardy) have a large number of interesting properties concerning factorizations,boundary values, and Cauchy-type representations in terms of measures on the boundary of ${\boldsymbol{U}},$ We shall merely give some of the highlights, such as the theorem of F. and M. Riesz on measures p ${\boldsymbol{\mu}}$ whose Fourier coeficients AIn) are $\mathbf{0}$ for all $\scriptstyle n\,<0,$ Beurling's classification of the invariant subspaces of T $H^{2},$ and M. Riesz's theorem on conju- gate functions. A convenient approach to the subject is via subharmonic functions, and we begin with a brief outline of their properties. Subharmonic Functions 17.1 Definition A function u defined in an open set $\Omega$ in the plane is said to be subharmonic if it has the following four properties. (a $-\infty\leq u(z)<\infty{\mathrm{~for~all~}}z\in\Omega.$ (b)uis upper semicontinuous in Q2. (c)Whenever ${\tilde{D}}(a;r)<\Omega,$ then $$ u(a)\leq{\frac{1}{2\pi}}\,\int_{-\pi}^{\pi}u(a+r e^{i\theta})\,d\theta. $$ (d) None of the integrals in (c) is -00 335336 REAL AND coMPLEX ANALYsis Note that the integrals in (c) always exist and are not +00, since (a) and (b) n, or imply that $u_{u}$ is bounded above on every compact $\kappa\in\Omega$ 「Proof: If $K_{n}$ is the set of all $\scriptstyle{\varepsilon\in K}$ at which $u(z)\geq n,$ then $K\to K_{1}\to K_{2}\cdots,$ so either $K_{n}={\mathcal{D}}$ for some $\bigcap{}$ K。≠ 0, in which case $u(z)=\infty$ for some z ∈ K.] Hence (d) says that the integrands in (c) belong to $\scriptstyle U(T)$ Every real harmonic function is obviously subharmonic 17.2 Theorem If u is subharmonic in $\Omega,$ and if p is a monotonically increasing convex function on $R^{1},$ ", then (p。u is subharmonic. $\varphi$ [To have p。u defined at all points of $\Omega,$ we put $\varphi(-\infty)=\operatorname*{lim}\;\varphi(x)$ as $x\to-\infty,1$ PROOF First, $\varphi\circ u\lg$ s upper semicontinuous, since $\varphi$ is increasing and contin- uous. Next, if ${\tilde{D}}(a;r)<\Omega,$ we have $$ \varphi(u(a))\leq\varphi{\Biggl(}{\frac{1}{2\pi}}\int_{-\pi}^{\pi}u(a+r e^{i\theta})\,\,d\theta{\Biggr)}\leq{\frac{1}{2\pi}}\int_{-\pi}^{\pi}\varphi(u(a+r e^{i\theta}))\,\,d\theta. $$ The first of these inequalities holds since $\varphi$ is increasing and u is sub- harmonic; the second follows from the convexity of p, by Theorem 3.3. / 17.3 Theorem If Q is a region, fe H(Q2), and fis not identically O, then log lf is subharmonic in $\Omega,$ and so are logt if| andlfP (0< p < 0O). PROOF It is understood that log $|f(z)|=-\infty$ if $f(z)=0.$ Then log If」 is upper semicontinuous in $\Omega,$ and Theorem 15.19 implies that log lfl is sub- harmonic of u, with The other assertions follow if we apply Theorem $17.2$ to log $|f|$ in place $$ \varphi(t)=\operatorname*{max}\,(0,\,t)\quad{\mathrm{and}}\quad\varphi(t)=e^{p t}. $$ // 17.4 Theorem Suppose u is a continuous subharmonic function in $\Omega,$ ${\boldsymbol{K}}$ is a compact subset of $\Omega,$ ${\boldsymbol{h}}$ h is a continuous real function on K which is harmonic in the interior V of $K,$ and $u(z)\leq h(z)$ at all boundary points of $K.$ Then u(z)≤ h(z) for all $\scriptstyle{\varepsilon\in K}$ This theorem accounts for the term“subharmonic”Continuity of $\boldsymbol{\ u}$ u is not necessary here, but we shall not need the general case and leave it as an exercise. and since nonempty compact subset of $V.$ Let $\mathrm{z}_{0}$ and assume, to get a contradiction, that $u_{1}(z)>0$ for PROOF Put $u_{1}=u-h_{r}$ some $z\in V.$ Since $u_{1}$ is continuous on $\scriptstyle\kappa_{s}u_{1}$ attains its maximum $E.$ Then for $K_{\mathrm{{i}}}$ ${\mathfrak{m}}\,$ on $u_{*}\leq0$ on the boundary of $K,$ the set $E=\{z\in K:u_{1}(z)=m\}$ is a be a boundary point ofHP-SPACES 337 some $\scriptstyle{r\gg0}$ we have ${\tilde{D}}(z_{0};\ r)<V,$ but some subarc of the boundary of $\theta_{t o t}$ r) lies in the complement of $\displaystyle E.$ Hence $$ u_{1}(z_{0})=m>\frac{1}{2\pi}\left.\right]_{-\pi}^{\pi}u_{1}(z_{0}+r e^{i\theta})~d\theta, $$ and this means that $u_{1}$ is not subharmonic in $V.$ But if $\boldsymbol{\ u}$ is subharmonic,so is u-h, by the mean value property of harmonic functions, and we have our contradiction // 17.5 Theorem Suppose u is a continuous subharmonic function in ${\boldsymbol{U}},$ and $$ m(r)=\frac{1}{2\pi} \{-_{\pi}^{\pi}u(r e^{i\theta})\;d\theta\qquad(0\leq r<1). $$ (1) lfr <rz,then m(r)≤ m(rz) $17.4,u\leq h$ in $D(0;r_{2}).$ be the continuous function on $\scriptstyle{{\hat{D}}(0;r_{2})}$ which coincides with u PRoOF Let ${\boldsymbol{h}}$ on the boundary of ${\tilde{D}}(0;r_{2})$ and which is harmonic in $\scriptstyle{D(0;r_{2})}$ By Theorem Hence $$ m(r_{1})\leq\frac{1}{2\pi}\left|_{-\pi}^{\pi}h(r_{1}e^{i\theta})\,d\theta=h(0)=\frac{1}{2\pi}\int_{-\pi}^{\pi}h(r_{2}\,e^{i\theta})\,d\theta=m(r_{2}).\qquad H(r_{1})\right. $$ The Spaces $H^{p}$ and ${\boldsymbol{N}}$ 17.6 Notation As in Secs 11.15 and 11.19, we define $f_{r}$ on ${\mathbf{}}T$ by $$ f_{r}(e^{i\theta})=f(r e^{i\theta})\qquad(0\leq r<1) $$ (1) if $\boldsymbol{\f}$ measure on ${\boldsymbol{T}},$ is any continuous function with domain $U,$ and we let o denote Lebesguc so normalized that $\sigma(T)=1.$ Accordingly, $L^{p}.$ -norms will refer to $B(\sigma),\,\mathrm{In}$ particular $$ \|f_{r}\|_{p}=\left\{\left\{\right\}_{T}|f_{r}|^{p}\;d\sigma\right\}^{1/p}\;\;\;\;\;\;(0<p<\infty), $$ (2) $$ \|f_{r}\|_{\infty}=\operatorname*{sup}_{\theta}|f(r e^{i\theta})|, $$ (3) and we also introduce $$ \|f_{r}\|_{0}=\exp\bigcap_{T}\!\log^{+}|f_{r}|\,d\sigma. $$ (4) 17.7 Definition $\operatorname{If}f\in H(U)$ and $0\leq p\leq$ co, we put $$ \|f\|_{p}=\operatorname*{sup}\;\{\|f_{r}\|_{p};\;0\leq r<1\}. $$ (1)338 REAL AND coMPLEX ANALYSis f $0<p\leq\infty,$ $H^{p}$ is defined to be the class of al $f\in H(U)$ for which nology in the case $p=\,\infty.$ (Note that this coincides with our previously introduced termi- $\|f\|_{p}<_{\circ}\infty$ . The class ${\mathbf{}}N$ consists of $\operatorname{all}f\in H(U)$ for which $\|f\|_{0}<\infty.$ It is clear that $H^{\infty}\subset H^{p}\subset H^{\bullet}\subset N$ if $0<s<p<\infty$ 17.8 Remarks (a) When $p<\infty,$ Theorems 17.3 and 17.5 show that $\|f_{r}\|_{p}$ is a nondecreasing function of ${\boldsymbol{r}}_{\!_{J}}$ for every $f\in H(U);$ when $p=\infty,$ the same follows from the maximum modulus theorem. Hence $$ \|f\|_{p}=\operatorname*{lim}_{r\to1}\|f_{r}\|_{p}. $$ (1) (b) For $1\leq p\leq\varnothing,$ $\|f\|_{p}$ satisfies the triangle inequality, so that $H^{p}$ is a normed linear space. To see this, note that the Minkowski inequality gives $$ \|(f+g)_{r}\|_{p}=\|f_{r}+g_{r}\|_{p}\leq\|f_{r}\|_{p}+\|g_{r}\|_{p} $$ (2) if $0<r<1.\;\mathrm{As}\;r\to1,$ we obtain $$ \left\|f+g\right\|_{p}\leq\left\|f\right\|_{p}+\left\|g\right\|_{p}. $$ (3) suppose (c) Actually, $H^{p}$ is a Banach space, if $1\leq p\leq\infty:\mathrm{To}$ prove completeness center O Cauchy formula $\mathrm{to}f_{n}-f_{m},$ is a Cauchy sequence in HP $,\,\,|z\,|\Xi\,r<R<1,$ and apply the $\{f_{n}\}$ integrating around the circle of radius ${\boldsymbol{R}},$ This leads to the inequalities $$ (R-r)|f_{n}(z)-f_{m}(z)|\leq\|(f_{n}-f_{m})_{R}\|_{1}\leq\|(f_{n}-f_{m})_{R}\|_{p}\leq\|f_{n}-f_{m}\|_{p} $$ $U$ for all $n>m,$ from which we conclude tha $\{f_{n}\}$ converges uniformly on compact subsets of $\|f_{n}-f_{m}\|_{p}<\epsilon$ to a function fe $H(U).$ Given $\scriptstyle\epsilon\;>0.$ there is an m such that and then, for every $\gamma<1.$ $$ \|(f-f_{m})_{r}\|_{p}=\operatorname*{lim}_{n\to\infty}\|(f_{n}-f_{m})_{r}\|_{p}\leq\epsilon. $$ (4) This gives $\|f-f_{m}\|_{p}\to0$ as m→O (d) For $p<1,\,H^{p}$ is still a vector space, but the triangle inequality is no longer satisfied by $\|f\|_{p}.$ condition that the zeros of any f $\in H^{p}$ We saw in Theorem 15.23 that the zeros of any $f\in N$ satisfy the Blaschke $\Sigma(1-|\alpha_{n}|)<\infty.$ Hence the same is true in every $H^{p}.$ it is interesting can be divided out without increasing the norm: 17.9 Theorem Suppose $f\in N,f\neq0,$ and $\boldsymbol{B}$ is the Blaschke product formed with the zeros off. Put $g=f/B$ .Then $g\in N$ and $\left\|g\right\|_{0}=\left\|f\right\|_{0}.$ Moreover, iff∈ HP,then $g\in H^{p}$ and $\|g\|_{p}=\|f\|_{p}(0<p\leq\infty)$HP-SPACEs 339 PR0OF Note first that $$ |g(z)|\geq|f(z)|\qquad(z\in U). $$ (1) In fact, strict inequality holds for every $z\in U,$ unless $\boldsymbol{\mathsf{f}}$ has no zeros in ${\boldsymbol{U}},$ in which case $\scriptstyle{B=1}$ and $\scriptstyle{g=f_{\circ}}$ If s and t are nonnegative real numbers, the inequality $$ \log^{+}\left(s t\right)\leq\log^{+}\ s+\log^{+}\ t $$ (2) holds since the left side is $\mathbf{0}$ if $\scriptstyle x\,z\,v$ and is log $s+\log\ t$ if $\scriptstyle n\geq1$ Since $|g|=|f|/|B|,(2)$ gives $$ \log^{*}\left|g\right|\leq\log^{*}\left|f\right|+\log{\frac{1}{\left|B\right|}}. $$ (3) By Theorem 15.24,(3) implies that $\|g\|_{0}\leq\|f\|_{0}$ ,and since(1) holds, we actually have $\left\|g\right\|_{0}=\left\|f\right\|_{0}$ for some $\scriptstyle{p\;>\;0.}$ Let $B_{n}$ be the finite Blaschke Now suppose $f\in H^{p}$ product formed with the first n zeros of f(we arrange these zeros in some increases to|gl,so that sequence, taking multiplicities into account). Put $g_{n}=f/B_{n}.$ For each ${\boldsymbol{n}},$ $|B_{n}(r e^{i\theta})| arrow1$ uniformly,as $r\to1$ Hence $\left\|g_{n}\right\|_{p}=\left|\left|f\right|\right|_{p}$ As n→00,lgn1 $$ \|g_{r}\|_{p}=\operatorname*{lim}_{n\to\infty}\|(g_{n})_{r}\|_{p}\qquad(0<r<1), $$ (4)) for all from(1), as before. by the monotone convergence theorem. The right side of (4) is at most $\left\Vert g\right\Vert_{p}\leq\left\Vert f\right\Vert_{p}.$ Equality follows now $\|f\|_{p},$ $\gamma<1.$ If we let r→1, we obtain // 17.10 Theorem Suppose $0<p<\infty,\;f\in H^{p},\;$ $\scriptstyle f\neq\scriptstyle0.$ and B is the Blaschke $H^{2}$ product formed with the zeros of f. Then there is a zero-free function h ∈ such that $$ f=B\cdot h^{2t p}. $$ (1) In particular, every f ∈ $H^{1}$ is a product $$ \scriptstyle J=g h $$ (2) in which both factors are in $H^{2}.$ exp In fact, PROOF By Theorem 17.9,f/B ∈ $H^{p};$ in fact, $\|f_{,?}f_{\mathrm{}}||_{p}=|\mathrm{}j||_{p}.$ Since $f/B$ has no zero in ${\boldsymbol{U}}$ and U is simply connected, there exists $\varphi\in H(U)$ so that $|\,h\,|^{2}=\,|f/B\,|^{p}$ $|h|_{2}^{2}=|I|_{*}$ f IlR (Theorem 13.11).Put $h=\exp\,(p\varphi/2).$ Then $h\in H(U)$ and $(\varphi)=f/B$ , hence ${}^{k\;\epsilon\;m^{2}\,}$ and (1) holds To obtain (2), write (1) in the form f =(Bh)·h. /340 REAL AND coMPLEX ANALYSIs We can now easily prove some of the most important properties of the $H^{p}.$ spaces. 17.11 Theorem If0<p< Oo and fe H", then (a) the nontangential maximal functions N f are in LE(T), or all α <1; (c $\textstyle\operatorname*{lim}_{r\to1}\|f^{*}-f_{r}\|_{p}=0,$ (b)the nontangential limits f*(e") exist a.e. on T, and f* e LP(T); and (d) $\|f^{*}\|_{p}=\|f\|_{p}$ If f ∈ $H^{1}$ then fis the Cauchy integral as well as the Poisson integral of f* PRoOF We begin by proving (a) and (b) for the case $\scriptstyle p\gg1.$ Since holomorphic functions are harmonic, Theorem 11.30(b) shows that every f ∈ $H^{p}$ is then the $e^{i\theta}$ Poisson integral of a function (call it f*) in $L^{p}(T).$ Hence $N_{x}f\in D(T),$ by Theorem 11.25(b), and $f^{\star}(e^{i\theta})$ is the nontangential limit of f at almost every e T, by Theorem 11.23. If $0<p\leq1$ and fe $H^{p},$ use the factorization $$ f=B h^{2/p} $$ (1) given by Theorem 17.10, where $|f|\leq|h|^{2p}$ in $\scriptstyle V{\mathrm{~,~}}1$ follows that $h\in H^{2},$ and $\boldsymbol{h}$ h has $\boldsymbol{B}$ is a Blaschke product, no zero in U. Since $$ (N_{\alpha}f)^{p}\leq(N_{\alpha}h)^{2}, $$ (2) so that $N_{x}f\in D(T),$ because $N_{x}h\in L^{2}(T).$ and $h^{\oplus}$ a.e.on ${\mathbf{}}T$ implies that the non- wherever Similarly,the existence of $B^{\mathbb{N}}$ $|f^{\ast}|\leq N_{\alpha}f$ tangential limits of(call hem *) exist a.e. Obviously, $f^{\bullet}$ exists. Hence f* ∈ $\scriptstyle{|U(T)|}$ Since This proves (a) and (b), fo a.e. and $|f_{r}|<N_{s}f,$ the dominated convergence theorem ${\mathfrak{r}}\;0<p<\infty.$ gives (c) $f_{r}\to f^{*}$ If $p\geq1.$ (d) follows from (C),by the triangle inequality. If ${\mathfrak{p}}<1,$ use Exer- Finally, if fe cise 24, Chap. 3, to deduce (d) from (c) and $f_{r}(z)=f(r z),$ then $f_{r}\in H(D(0,1/r)$ ), and there- $H^{1},r<1,$ fore f, can be represented in $U$ by the Cauchy formula $$ f_{r}(z)={\frac{1}{2\pi}}\int_{-\pi}^{\pi}{\frac{f_{r}e^{i t}}{1-e^{-i t_{z}}}}\,d t $$ (3) and by the Poisson formula $$ f_{r}(z)={\frac{1}{2\pi}}\,\int_{-\pi}^{\pi}P(z,\,e^{i t})f_{r}(e^{i t})\,\,d t. $$ (4)HP-SPACES 341 For each z E ${\boldsymbol{U}},$ $|1-e^{-i t}z|$ and $P(z,\ e^{i t})$ are bounded functions on ${\boldsymbol{T}}.$ The case $\scriptstyle{p=1}$ of Cc) leads therefore from (3) and (4) to $$ f(z)={\frac{1}{2\pi}}\left.\right\}_{-\pi}^{\pi}{\frac{f^{\star}(e^{i t})}{1-e^{-i t}z}}\,d t\bar{t} $$ (5) and $$ f(z)={\frac{1}{2\pi}}\int_{-\pi}^{\pi}P(z,\,e^{i t})f^{*}(e^{i t})\,d t. $$ (6) // The space $H^{2}$ has a particularly simple characterization in terms of power series coefficients: 17.12 Theorem Suppose fe H(U and $$ f(z)=\sum_{0}^{\infty}a_{n}z^{n}. $$ Then fe H- if and only i $\sum_{0}^{\infty}$ |a,1< o PRoOF By Parseval's theorem, applied tof, with $\gamma_{<1}$ $$ \sum_{0}^{\infty}|\,a_{n}|^{2}=\operatorname*{lim}_{r\to1}\,\sum_{0}^{\infty}\,|\,a_{n}|^{2}r^{2n}=\operatorname*{lim}_{r\to1}\,\left[\right]_{T}|\,f_{r}|^{2}\,\,d\sigma=\|f\|_{2}^{2}\,. $$ // The Theorem of $\mathbf{F}.$ . and M. Riesz 17.13 Theorem Ifuis a complex Borel measure on the unit circle ${\mathbf{}}T$ " and $$ \iota^{*}e^{-i n t}\,d\mu(t)=0 $$ (1) for n = -1, -2,-3, .…, then p is absolutely continuous with respect to Lebesgue measure. PRoOF Putf = P[du]. Then f satisfies $$ \|f_{r}\|_{1}\leq\|\mu\|\qquad(0\leq r<1). $$ (2) (See Sec. 11.17.) Since,setting $z=r e^{\omega},$ $$ P(z,\,e^{i t})=\sum_{-\infty}^{\infty}r^{|n|}e^{i n\theta}e^{-i n t}, $$ (3)342 REAL AND cOMPLEX ANALYSIS as in Sec. 11.5, the assumption for all $\scriptstyle n\,<\,0,$ leads to the power series $(1),$ which amounts to saying that the Fourier coefficients A(n) are $\mathbf{0}$ $$ f(z)=\sum_{0}^{\infty}\hat{\mu}(n)z^{n}\qquad(z\in U). $$ (4) $f^{*}\in L^{1}(T).$ By(4)) and (2), $\textstyle f\in H^{\prime}$ Hence $f=P[f^{*}]$ by Theorem 17.11, where The uniqueness of the Poisson integral representation CTheorem 11.30) shows now that $d\mu=f^{*}\,d\sigma.$ // The remarkable feature of this theorem is that it derives the absolute contin uity of a measure from an apparently unrelated condition, namely, the vanishing of one-half of its Fourier coefficients. In recent years the theorem has been extended to various other situations. Factorization Theorems We already know from Theorem 17.9 that every $\scriptstyle f\in H^{\prime}$ (except $f=0)$ can be factored into a Blaschke product and a function $g\in H^{p}$ which has no zeros in ${\boldsymbol{U}}.$ There is also a factorization of g which is of a more subtle nature. It concerns, roughly speaking, the rapidity with which g tends to $\mathbf{0}$ along certain radii 17.14 Definition An inner function is a function $M\in H^{n}$ for which $|M^{*}|=1$ and if a.e. on T.(As usual, $M^{*}$ denotes the radial limits of M.) such that log $\varphi\in L^{1}(T),$ If p is a positive measurablefunction on ${\mathbf{}}T$ $$ Q(z)=c\,\exp\left\{{\frac{1}{2\pi}}\int_{-\pi}^{\pi}{\frac{e^{i t}+z}{e^{i t}-z}}\log\varphi(e^{i t})\ d t\right\} $$ (1) for $z\in U,$ then ${\boldsymbol{Q}}$ is called an outer function. Here c is a constant $|c|=1,$ Theorem 15.24 shows that every Blaschke product is an inner function but there are others. They can be described as follows. 17.15 Theorem Suppose c is a constant $|c|=1,$ $\boldsymbol{B}$ is a Blaschke product, p ${\boldsymbol{\mu}}$ u is a finite positive Borel measure on ${\mathbf{}}T$ which is singular with respect to Lebesgue measure, and $$ M(z)=c B(z)\exp\left\{-\ \int_{-\pi}^{\pi}{\frac{e^{i t}+z}{e^{i t}-z}}\,d\mu(t)\right\}\qquad(z\in U). $$ (1) Then M is an inner function, and every inner function is of this form PRoor If $^{*}(1)$ holds and $g=M/B,$ then log lgl is the Poisson integral of $M.$ Also $D\mu=0$ a.e., since ${\boldsymbol{\mu}}$ $|g|\leq0,$ so that $g\in H^{\infty},$ and the same is true of $-d\mu,$ hence log is singular (Theorem 7.13),and therefore the radial limits ofHP-sPACEs 343 loglgl are O a.e.(Theorem 11.22). Since $|B^{n}|=1$ a.e., we see that $\textstyle{M}$ is an inner function. Conversely, let $\boldsymbol{B}$ be the Blaschke product formed with the zeros of a given inner function $\textstyle{M}$ and put $g=M/B.$ Then log lgl is harmonic in ${\boldsymbol{U}}.$ log Thus log Theorems 15.24 and 17.9 show that $-d\mu,$ for some positive measure ${\boldsymbol{\mu}}$ is singular. Finally a.e. on ${\boldsymbol{T}}.$ $\scriptstyle(y\colon\leq1$ in $U$ and that $|g^{*}|=1$ $|g^{*}|=0$ a.e. on ${\boldsymbol{T}},$ 、 we have $\scriptstyle D_{\mu}=0$ We conclude from Theorem i1.30 that log lgl is the so p on ${\boldsymbol{T}}.$ Since $|g|\leq0.$ Poisson integral of a.e. on ${\boldsymbol{T}},$ log lglis the real part of $$ h(z)=\left.\right|_{-\pi}^{\pi}\frac{e^{i t}+z}{e^{i t}-z}\,d\mu(t), $$ and this implies that $g=c$ exp (h) for some constant c with $|c|=1,$ Thus ${\cal{M}}$ is of the form (1). This completes the proof. // The simplest example of an inner function which is not a Blaschke product is the following: Take $\scriptstyle\epsilon=1$ and $\scriptstyle B=1.$ and let $\boldsymbol{\mu}$ be the unit mass at $\scriptstyle t\;=\;0$ Then $$ M(z)=\exp{\left\{{\frac{z+1}{z-1}}\right\}}, $$ which tends to O very rapidly along the radius which ends at $z=1.$ 17.14.Then 17.16 Theorem Suppose ${\boldsymbol{Q}}$ is the outer function related to $\varphi$ as in Definition (a) log IQ| is the Poisson integral of log qp (c) (b)lim,-1 $|Q(r e^{\omega})|=\varphi(e^{\omega})$ a.e. on ${\boldsymbol{T}}.$ $Q\in H^{p}$ if and only if p e L(T). In this case, I Qll, = |ll, PROOF $\mathbf{\tau}_{(a)}$ is clear by inspection and $\mathbf{\tau}_{(a)}$ implies that the radial limits of $Q\in H^{p},$ Fatou's log |Q| are equal to log p a.e. on ${\boldsymbol{T}},$ which proves (b). If lemma implies that then $\|Q\|_{p}\leq\|Q\|_{p}$ ,SO $\|\varphi\|_{p}\leq\|Q\|_{p},$ by (b). Conversely, if $\varphi\in L^{p}(T),$ $$ |\,Q(r e^{i\theta})\,|^{p}=\exp\,\left\{\frac{1}{2\pi}\int_{-\pi}^{\pi}P_{,r}(\theta-t)\,\log\,\varphi^{p}(e^{i t})\,\,d t\right\} $$ $$ \leq{\frac{1}{2\pi}}\prod_{-\pi}^{\pi}P_{r}(\theta-t)\varphi^{p}(e^{i t})\;d t, $$ lll i $p<\infty$ , The case $p=\infty$ by the inequality between the geometric and arithmetic means CTheorem 3.3) $|0|,\leq$ // and if we integrate the last inequality with respect to we find that is trivial344 REAL AND cOMPLEX ANALYSIS log 17.17 Theorem Suppose $0<p\leq\infty,f\in H^{p},$ and $\boldsymbol{\mathsf{f}}$ is not identically O.Then $|f^{*}|\in L^{1}(T),$ the outer function $$ Q_{f}(z)=\exp\left\{{\frac{1}{2\pi}}\int_{-\pi}^{\pi}{\frac{e^{i t}+z}{e^{i t}-z}}\log\left|f^{*}(e^{i t})\right|\,d t\right\} $$ (1) is in $H^{p},$ , and there is an inner function $\textstyle{M_{f}}$ such that $$ f=M_{f}\,Q_{f}. $$ (2) Furthermore, $$ \log\mid f(0)\mid\leq{\frac{1}{2\pi}}\int_{-\pi}^{\pi}\log\mid f^{\star}(e^{i\imath})\mid d t. $$ (3) Equality holds in (3) if and only if $\textstyle{M_{f}}$ is constant tively; $Q_{f}$ The functions $M_{f}$ and $Q_{f}$ are called the inner and outer factors of ${\mathfrak{f}},$ respec- depends only on the boundary values of |fl. PRoOF We assume first that f∈ $H^{1}.$ If $\boldsymbol{B}$ is the Blaschke product formed with the zeros of $\boldsymbol{\mathsf{f}}$ and if $g=f/B,$ Theorem 17.9 shows that $\;_{a\,\in\,H^{\prime}}$ ;and since $|g^{*}|=|f^{*}|$ a.e. on ${\boldsymbol{T}},$ it suffices to prove the theorem with and thi ${\mathfrak{t r}}f(0)=1.$ in place of f. So let us assume that f has no zero in L 1 $\scriptstyle{\mathcal{G}}$ Then log If1 $U$ is harmonic in ${\boldsymbol{U}},$ log $|f(0)|=0,$ and since $\log=\log^{+}-\log^{-}$ ,the mean value property of harmonic functions implies tha $$ \frac{1}{2\pi}\left[-_{\pi}^{\pi}\log^{-}\vert f(r e^{i\theta})\vert\ d\theta=\frac{1}{2\pi}\int_{-\pi}^{\pi}\log^{+}\vert f(r e^{i\theta})\ d\theta\leq\vert f\vert_{0}\leq\vert\vert f\vert_{1}\right.\quad(4) $$ for $0<r<1.$ It now follows from Fatou's lemma that both log |f*| and $H^{1}.$ log |f*| are in $L^{1}(T),$ hence so is log $|f^{\ast}|$ $Q_{f}$ This shows that the definition (1) makes sense. By Theorem 17.16, Also, IQf|=|f*|≠ 0 a.e., since log $|f^{*}|\in L^{1}(T).$ if we can prove that $$ |f(z)|\leq|Q_{f}(z)|\qquad(z\in U), $$ (5) then Since log $\scriptstyle Q_{r}!$ will be an inner function, and we obtain the factorization (2) is equivalent to the $J Q_{t}$ is the Poisson integral of log $|f^{*}|,(5)$ inequality $$ \log\mid f\mid\leq P[\log\mid f^{*}\mid], $$ (6) which we shall now prove. Our notation is as in Chap. 11: P[h] is the Poisson integral of the function and $0<R<1,$ pu $\operatorname{tr}f_{R}(z)=f(R z).\operatorname{Fix}z\in U$ J. Then $h\in L^{1}(T).$ $\mathrm{For}\,|z|\leq1$ $$ \log\mid f_{R}(z)\mid=P[\log^{+}\mid f_{R}\mid\Im(z)-P[\log^{-}\mid f_{R}\mid\Im(z). $$ (7)HP-SPACES 345 Since |logt u - log t vl≤|u- v| for all real numbers u and v, and since $\|f_{R}-f^{*}\|_{1} arrow0$ aS $\scriptstyle R\, arrow\,1$ (Theorem 17.11), the the first Poisson integral in (T) converges to P[logt |/*|], as $\scriptstyle{R\to1{\frac{1}{2}}}$ Hence Fatou's lemma gives $$ P[\log^{-}\vert f^{*}\vert]\leq\operatorname*{lim}_{R\to1}\mathrm{im}\mathop{\operatorname*{inf}}P[\log^{-}\vert f_{R}\vert]=P[\log^{+}\vert f^{*}\vert]-\log\vert f\vert, $$ (8) which is the same as (6) We have now established the factorization(2). If we put $\scriptstyle z\neq0$ in (S) we if stant. obtain (3); equality holds in(3) if and only $\|M_{f}\|_{\infty}=1,$ this happens only when $\textstyle{M_{f}}$ is a con- $i{\bf f}\!\mid f(0)|=|{\cal Q}_{J}(0)|\,,$ i.e., if and only $|M_{f}(0)|=1;$ and since This completes the proof for the case $\scriptstyle p=1.$ If $1<p\leq\infty,$ then $H^{p}<H^{1}$ , hence all that remains to be proved is that $Q_{f}\in H^{p}$ But if f∈ $H^{p},$ then $|f^{*}|\in D(T),$ by Fatou's lemma; hence $Q_{f}\in H^{\prime},$ by Theorem 17.16(c) Theorem 17.10 reduces the case $\scriptstyle{p<1}$ to the case $p=2.$ / The fact that log $|f^{*}|\in L^{1}(T)$ has a consequence which we have already used in the proof but which is important enough to be stated separately: 17.18 Theorem $l f\,0<p\leq_{\infty}\circ_{r}f\in H^{p},$ and f is not identically O, then at almost all points of ${\mathbf{}}T$ we have $f^{**}(e^{\theta})\neq0.$ PROOF If $f^{\bullet}=0$ then log $|f^{*}|=-\infty,$ and if this happens on a set of posi- tive measure, then $$ \vert\sum_{-\pi}^{\pi}\log|f^{*}(e^{i t})|~d t=-\infty. $$ // Observe that Theorem 17.18 imposes a quantitative restriction on the loca- tion of the zeros of the radial limits of ${\mathfrak{a n}}\,f\in H^{p}$ Inside $U$ the zeros are also quantitatively restricted, by the Blaschke condition. As usual, we can rephrase the above result about zeros as a uniqueness theo- rem: If fe HP,ge H",and $f^{*}(e^{i\theta})=g^{*}(e^{i\theta})$ on some subset of ${\mathbf{}}T$ whose Lebesgue measure is positive, then f(z) = g(z) for all ze U. 17.19 Let us take a quick look at the class $N,$ with the purpose of determining how much of Theorems 17.17 and 17.18 is true here. If $\scriptstyle{\mathcal{G}}$ which has no zero in $\scriptstyle{\mathbb{N}}\mu\,0,$ we can divide by a Blaschke product and get a quotient $f\in N$ and and $U$ which is in ${\mathbf{}}N$ (Theorem 17.9). Then log lgl is harmonic, and since $$ |\log|g|\left.\right|=2\log^{*}|g|-\log|g| $$ (1)346 REAL AND coMPLEX ANALYSIS and $$ \frac{1}{2\pi}\left(\frac{\pi}{\pi}\right)_{-\pi}^{\pi}\log\left|g(r e^{i\theta})\right|\,d\theta=\log\left|\,g(0)\right|, $$ (2) we see that log lgl satisfies the hypotheses of Theorem 11.30 and is therefore the Poisson integral of a real measure $\mu.$ Thus $$ f(z)=c B(z)\exp\left\{ \{\begin{array}{c}{{e^{i t}+z}}\\ {{\bar{e}^{i t}-z}}\end{array}d\mu(t)\right\}, $$ (3) where c is a constant, $|c|=1,$ and $\boldsymbol{B}$ is a Blaschke product. Observe how the assumption that the integrals of logtlgl are bounded (which is a quantitative formulation of the statement that $|g|$ does not get too close to co) implies the boundedness of the integrals of log |g|(which says that lgldoes not get too close to $\mathbf{0}$ at too many places). f ${\boldsymbol{\mu}}$ is a negative measure, the exponential factor in (3) is in ${\cal H}^{\infty}.$ Apply the Jordan decomposition to ${\boldsymbol{\mu}}.$ This shows: no zero in $U$ and $f=b_{1}/b_{2}$ To everyfe N there correspond two functions $b_{1}$ , and $b_{2}\in H^{\infty}$ such that $b_{2}$ has Since $\scriptstyle b_{\mathrm{f}}\neq0$ a.e., tfollows that f has finite radial limits a.e. Also, $f^{*}\neq0$ a.e. Is log $|f^{*}|\in L^{1}(T)^{\gamma}$ Yes, and the proof is identical to the one given in Theorem 17.17 However, the inequality (3) of Theorem 17.17 need no longer hold. For example, if $$ f(z)=\exp{\left\{{\frac{1+z}{1-z}}\right\}}, $$ (4) then $\|f\|_{0}=e,|f^{*}|=1$ a.e, and $$ \log|f(0)|=1>0=\frac{1}{2\pi}\int_{-\pi}^{\pi}\log|f^{*}(e^{i t})|\;d t. $$ (5) The Shift Operator 17.20 Invariant Subspaces Consider a bounded linear operator $\boldsymbol{\mathsf{S}}$ on a Banach space $X{\mathrm{:}}$ that is to say, $\boldsymbol{\mathsf{S}}$ is a bounded linear transformation of $X$ into $X.$ If a closed subspace ${\mathbf{}}Y$ of $X$ has the property that ${\mathcal{S}}(Y)\in Y,$ we call ${\mathbf{}}Y$ an S ${\boldsymbol{S}}\cdot$ S-invariant subspace. Thus the S-invariant subspaces of $X$ are exactly those which are mapped into themselves by ${\dot{S}}.$ The knowledge of the invariant subspaces of an operator $\boldsymbol{\mathsf{S}}$ helps us to visual- ize its action.(This is a very general- -and hence rather vague -principle: In studying any transformation of any kind, it helps to know what the transform- ation leaves fixed.) For instance, if $\boldsymbol{\mathsf{S}}$ is a linear operator on an ${\boldsymbol{n}}.$ dimensional vector space $X$ and if $\boldsymbol{\mathsf{S}}$ has n linearly independent characteristic vectors $x_{1}\cdots$HP-SPACES 347 $x_{n},$ the one-dimensional spaces spanned by any of these ${\boldsymbol{x}}_{i}$ are ${\boldsymbol{S}}.$ invariant, and we obtain a very simple description of $\boldsymbol{\mathsf{S}}$ if we take $\{x_{1},\ldots,\ x_{n}\}$ as a basis of $X.$ We shall describe the invariant subspaces of the so-called “shift operator”S on $\ell^{2}.$ Here $\ell^{2}$ is the space of all complex sequences $$ x=\{\xi_{0},\,\xi_{1},\,\xi_{2},\,\xi_{3},\,\dots\} $$ (1) for which $$ \|x\|=\left\{\sum_{n=0}^{\infty}|\;\xi_{n}|^{2}\right\}^{1/2}<\infty, $$ (2) and $\boldsymbol{\mathsf{S}}$ S takes the element $x\in{\ell}^{2}$ given by (1) to $$ S x=\{0,\,\xi_{0},\,\xi_{1},\,\xi_{2},\,\ldots\}. $$ (3) It is clear that $\boldsymbol{\mathsf{S}}$ is a bounded linear operator on $\ell^{2}$ and that $\|{\vec{s}}\|=1.$ is the set of all A few S-invariant subspaces are immediately apparent: If ${\cal Y}_{k}$ $\in\ell^{2}$ whose first $\ \boldsymbol{k}$ coordinates are O, then ${\cal Y}_{k}$ is ${\boldsymbol{S}}.$ invariant $H^{2}$ To find others we make use of a Hilbert space isomorphism between to a multiplication operator on $H^{2}.$ The $\ell^{2}$ and which converts the shift operator $\boldsymbol{\mathsf{S}}$ point is that this multiplication operator is easier to analyze(because of the richer structure of $H^{2}$ as a space of holomorphic functions) than is the case in the original setting of the sequence space $\ell^{2}.$ We associate with each $x\in\ell^{2},$ given by (1), the function $$ f(z)=\sum_{n=0}^{\infty}\;\xi_{n}z^{n}\qquad(z\in U). $$ (4) By Theorem 17.12, this defines a linear one-to-one mapping of $\ell^{2}$ onto $H^{2}.$ If $$ y=\{\eta_{n}\},\qquad g(z)=\sum_{n=0}^{\infty}\eta_{n}z^{n} $$ (5) and if the inner product in $H^{2}$ is defined by $$ (f,g)=\frac{1}{2\pi}\left.\right>_{-\pi}^{\pi}f^{\star}(e^{i\theta})\overline{{{g^{\star}}}}(e^{i\theta})\ d\theta, $$ (6) isomorphism of $\ell^{2}$ onto $H^{2},$ and the shift operator $\boldsymbol{\mathsf{S}}$ Thus we have a Hilbert space the Parseval theorem shows that $(f,g)=(x,y)$ ' has turned into a multiplica- tion operator (which we still denote by ${\boldsymbol{S}})$ on $H^{2}.$ $$ (S f)(z)=z f(z)\qquad(f\in H^{2},\,z\in U). $$ (7) all For any finite set The previously mentioned invariant subspaces the space ${\mathbf{}}Y$ of all $\scriptstyle\epsilon\in H^{2}$ such $\operatorname{that}f(x_{1})=$ $\scriptstyle\int_{\epsilon}H^{2}$ which have a zero of order at least $\boldsymbol{k}$ ${\cal Y}_{k}$ are now seen to consist of at the origin. This gives a clue: $\{\alpha_{1},\ldots,\alpha_{k}\}\subset U,$ $\mathbf{\nabla}\cdot\mathbf{\nabla}=f(\alpha_{k})=0$ is ${\boldsymbol{S}}\cdot$ invariant. If $\boldsymbol{B}$ is the finite Blaschke product with zeros at $x_{1},$ .,&x, then f e ${\mathbf{}}Y$ if and only if $f/B\in H^{2}$ Thus $Y=B H^{2}$348 REAL AND COMPLEX ANALYSIS This suggests that infinite Blaschke products may also give rise to ${\boldsymbol{S}}\cdot$ invariant subspaces and, more generally, that Blaschke products might be replaced by arbi- trary inner functions $\varphi.$ It is not hard to see that each $\scriptstyle{\sigma H^{2}}$ is a closed ${\boldsymbol{S}}\cdot$ -invariant subspace of $H^{2}.$ , but that every closed S-invariant subspace of $H^{2}$ is of this form is a deeper result. 17.21 Beurling's Theorem (a) For each inner function cp the space $$ \varphi H^{2}=\{\varphi f\varepsilon f\in H^{2}\} $$ (1) is a closed ${\boldsymbol{S}}\cdot$ invariant subspace of $H^{2}$ (b)f $\varphi_{1}$ and $\varphi_{2}$ are inner functions and j $\varphi_{1}H_{2}^{2}=\varphi_{2}H^{2},$ then p:/92 is constant. (c) Every closed S-invariant subspace Y of $Y=\varphi H^{2}$ $H^{2},$ other than $\{0\}_{*}$ contains an inner function gp such that PROOF $H^{2}$ is a Hilbert space, relative to the norm $$ \|f\|_{2}=\sqrt{\frac{1}{2\pi}}\,\int_{-\pi}^{\pi}|f^{\star}(e^{i\theta})|^{2}\,\,d\theta\rangle^{1/2}. $$ (2) $h^{\oplus},$ Similarly, it follows that ${\boldsymbol{h}}$ is real in ${\boldsymbol{U}},$ is real a.e. on ${\boldsymbol{T}}.$ Since $\boldsymbol{h}$ a.e. The mapping f→ of is there $\scriptstyle{\sigma H^{2}}$ is a closed and since If $\varphi$ is an inner function, then $|\phi^{*}|=1$ then .{qpf,} is a Cauchy sequence, -invariance of $\scriptstyle{\sigma H^{2}}$ If $|\phi^{*}|=1$ a.e. on $\scriptstyle{T_{s}h^{\ast}}$ hence ${\boldsymbol{h}}$ ; being an isometry, its range The ${\boldsymbol{S}}\cdot$ $\varphi_{1}/\varphi_{2}\in H^{2}$ fore an isometry of $H^{2}$ into $H^{2}.$ subspace of $H^{2}.$ [Proof:If $\varphi f_{n}\to g_{.}$ in $H^{2},$ $g=\varphi f\in\varphi H^{2}.]$ $f\in H^{2},$ hence $h\in H^{2},$ hence so is $\scriptstyle\{f_{\mathrm{a}}\}$ ,hence $f_{n}\to f\in H^{2},$ so . Hence (a) holds. Then is also trivial, since $z\cdot\varphi f=\varphi\cdot z j$ for some is the Poisson integral of $\varphi_{1}H^{2}=\varphi_{2}H_{\;\;;}^{2}$ then $\varphi_{1}=\varphi_{2}f$ and $h=\varphi+(1/\varphi).$ $\varphi_{2}/\varphi_{1}\in H^{2}.$ Put $\varphi=\varphi_{1}/\varphi_{2}$ is constant. Then $\varphi$ must be con- stant, and (b) is proved. The proof of (c) will use a method originated by Helson and Lowdensla ger. Suppose ${\mathbf{}}Y$ is a closed S-invariant subspace of $H^{2}$ which does not consist of O alone. Then there is a smallest integer $\boldsymbol{k}$ such that ${\mathbf{}}Y$ contains a function f of the form $$ f(z)=\sum_{n=k}^{\infty}c_{n}z^{n},\qquad c_{k}=1. $$ (3) Then $\scriptstyle{\sqrt{\rho}}\;z\,Y$ where we write $z\,Y$ for the set of all g of the form ${\mathbf{}}Y$ 「closed $g(z)=z f(z),f\in$ Y. It follows that $z Y$ is a proper closed subspace of by the argument used in the proof of (a)],so ${\mathbf{}}Y$ contains a nonzero vector which is orthogonal to $z\,Y$ (Theorem 4.11)HP-SPACES 349 So there exists a p∈ Y such that $|\phi||_{2}=1$ and $\varphi\perp z Y.$ Then $\varphi\perp z^{n}\varphi,$ for $\scriptstyle n\;=\;1.$ 2, 3,.… By the definition of the inner product in $H^{2}$ [see 17.20(6)] this means that $$ \frac{1}{2\pi} [\O_{-\pi}^{\pi}|\varphi^{\star}\!(e^{i\theta})|^{2}e^{-i n\theta}\;d\theta=0\qquad(n=1,\,2,\,3,\dots). $$ (4) These equations are preserved if we replace the left sides by their complex function conjugates, ie., if we replace n by $-n.$ Thus all Fourier coefficients of the $n=0,$ which is $|\,\varphi^{\ast}\,|^{2}\,\in\,L^{1}(T)$ are O ${\boldsymbol{0}},$ , except the one corresponding to 1. Since ${\boldsymbol{L}}^{1}$ -functions are determined by their Fourier coefficients (Theorem Since $\varphi\in Y$ and ${\mathbf{}}Y$ $\textstyle|\phi^{*}|=1$ a.e. on ${\boldsymbol{T}}.$ But $\varphi\in H^{2};$ $\varphi$ is an inner function. hence 5.15), it follows that 、SO $\varphi$ p is the Poisson integral of $\varphi^{\mathbb{P}},$ for every polynomial ${\boldsymbol{P}}.$ and hence |cpl≤1. We conclude that for all $\scriptstyle n\geq0.$ $\varphi P\in Y$ is S-invariant, we have $\varphi z^{n}\in Y$ (the partial . The polynomials are dense in $H^{2}$ sums of the power series of any $\textstyle f\in H^{2}$ converge to $\boldsymbol{\f}$ in the $H^{2}.$ norm, by Parseval's theorem),and since ${\mathbf{}}Y$ is closed and $|\phi|\leq1,$ it follows that $_{\sigma H}^{2}\subset Y.$ We have to prove that this inclusion is not proper. Since $\sigma H^{2}$ is closed, it is enough to show that the assumptions h $*\circ Y$ and $h\,\perp\,\varphi H^{2}$ imply $\hbar=0.$ If $h\,\perp\,\varphi H^{2}$ , then h L qp2z” for n = 0, 1, 2,..…,or $$ \frac{1}{2\pi}\left|_{-\pi}^{\pi}h^{\star}(e^{i\theta})\overline{{{\varphi^{\star}(e^{i\theta})}}}e^{-i n\theta}\;d\theta=0\qquad(n=0,1,2,\ldots).\qquad\right. $$ (5) If $*\in Y$ ,then $s^{*}h\in z Y$ if $n=1,$ 2,3,.…., and our choice of $\varphi$ shows that $z^{*}h\perp\ \varphi,\,\mathrm{ot}$ $$ \frac{1}{2\pi}\left|_{-\pi}^{\pi}h^{*}(e^{i\theta})\overline{{{\varphi^{*}(e^{i\theta})}}}e^{-i\pi\theta}\;d\theta=0\qquad(n=-1,\;-2,\;-3,\ldots).\qquad\right. $$ (6) since Thus all Fourier coefficients of $h^{\ast}\varphi^{\ast}$ are ${\mathfrak{O}},$ hence $h^{*}{\overline{{\varphi^{*}}}}=0$ a.e. on ${\boldsymbol{T}}\,;$ and $|\phi^{*}|=1$ a.e.,we have $h^{*}=0$ a.e. Therefore $\hbar=0.$ and the proof is complete. // 17.22 Remark If we combine Theorems 17.15 and 17.21, we see that the S-invariant subspaces of $H^{2}$ are characterized by the following data:a $|\alpha_{n}|<1$ sequence of complex numbers $\{\alpha_{n}\}$ (possibly finite, or even empty) such that on ${\boldsymbol{T}},$ singular and $\Sigma(1-|\alpha_{n}|)<\infty,$ and a positive Borel measure ${\boldsymbol{\mu}}$ with respect to Lebesgue measure (so $H^{2}$ contains another. The partially ordered set of all 从、which ensure that one an exercise) to find conditions, in terms of $\{\alpha_{n}\}$ and ${\boldsymbol{\mu}},$ a.e.). It is easy (we leave this as $D\mu=0$ S-invariant subspace of S-invariant subspaces is thus seen to have an extremely complicated struc- ture, much more complicated than one might have expected from the simple definition of the shift operator on $\ell^{2}.$ We conclude the section with an easy consequence of Theorem 17.21 which depends on the factorization described in Theorem 17.17.350 REAL AND coMPLEX ANALYSis 17.23 Theorem Suppose ${\boldsymbol{S}}\cdot$ invariant subspace of $H^{2}$ which contains f. Then $f\in H^{2},$ and ${\bf Y_{\nu}}$ is the smallest closed $M_{f}$ is the inner factor of a function $$ \scriptstyle Y=M_{f}H^{2}, $$ (1) In particular, $Y=H^{2}$ if and only iff is an outer function PRoOF Let $f=M_{f,}Q_{f}$ be the factorization of $\boldsymbol{\f}$ into its inner and outer we have factors. It is clear that $f\in M_{f}H^{2};$ and since $M_{t}H^{2}$ is closed and S-invariant, $\scriptstyle Y\in M_{f}H^{2}$ On the other hand, Theorem 17.21 shows that there is an inner function o such that $Y=\varphi H^{2}$ Since fe Y, there exists an $h=M_{k}Q_{k}\in H^{2}$ such that $$ M_{f}\,\mathcal{Q}_{f}=\varphi M_{h}\,\mathcal{Q}_{h}. $$ (2) $Q_{h}.$ proof is complete. Since inner functions have absolute value $\mathbf{1}$ a.e. on ${\boldsymbol{T}},$ (2) implies that $\scriptstyle Q_{f}={\mathfrak{t}}$ ${\boldsymbol{S}}\cdot$ hence $M_{f}=\varphi M_{h}\in Y,$ and therefore ${\mathbf{}}Y$ $\textstyle{M_{f}}$ . Thus must contain the smallest , and the // invariant closed subspace which contains $M_{f}H^{2}\subset Y_{i}$ It may be of interest to summarize these results in terms of two questions to which they furnish answers. If $\scriptstyle f\in H^{\prime}$ , which functions $\;_{a\,\in\,H^{2}}$ can be approximated in the $H^{2},$ -norm by functions of the form ${\mathcal{P}},$ where ${\mathbf{}}P$ runs through the polynomials?Answer: Pre- cisely those $\scriptstyle{\mathcal{G}}$ for which $g/M_{f}\in H^{2},$ $\langle P|$ is dense in $\scriptstyle{\bar{u}}^{2}$ Answer: Precisely For whichfe $H^{2}$ is it true that the set for thosef for which $$ \log\mid f(0)\mid={\frac{1}{2\pi}}\int_{-\pi}^{\pi}\log\mid f^{**}(e^{i t})\mid d t. $$ Conjugate Functions 17.24 Formulation of the Problem Every real harmonic function u in the unit disc ${\boldsymbol{U}}$ is the real part of one and only one f ∈ $H(U)$ such that f(0) = u(0) $\operatorname{f}f=u+i v,$ the last requirement can also be stated in the form $v(0)=0.$ The function vis called the harmonic conjugate of u, or the conjugate function of u Suppose now that u satisfies $$ \operatorname*{sup}_{r<1}\left|u_{r}\right|_{p}<\infty $$ (1) for some p. Does it follow that (1) holds then with v in place of u? Equivalently, does it follow that $f\in H^{p}!$ $1<p<\infty.$ (For $\scriptstyle{p=1}$ and The answer (given by M. Riesz) is affirmative i $p=\propto\!\mathrm{{o}}\,\mathrm{{if}}$ is negative; see Exercise 24.) The precise statement is given by Theorem 17.26.HP-SPACES 351 Let us recall that every harmonic u that satisfies (1) is the Poisson integral of a function $u^{\bullet}\in D(T)$ (Theorem 11.30) if $1<p<\infty$ Theorem 11.11 suggests therefore another restatement of the problem $I/!<p<\infty,$ and $i f$ we associate to $e a c h\,h\in L(T)$ the holomorphic function $$ (\psi h)(z)=\frac{1}{2\pi}\int_{-\pi}^{\pi}\frac{e^{i t}+z}{e^{i t}-z}\;h(e^{i t})\;d t\qquad(z\in U), $$ (2) do all of these functions yh lie in $H^{p}{\dot{\gamma}}$ Exercise 25 deals with some other aspects of this problem. then 17.25 Lemma If $1<p\leq2,\;\delta=\pi/(1+p),\;\alpha=(\cos\delta)^{-1},$ and $\beta=\alpha^{p}(1+\alpha),$ $$ 1\leq\beta(\cos\,\varphi)^{p}-\alpha\,\cos\,p\varphi\qquad\left(-\,\frac{\pi}{2}\leq\varphi\leq\frac{\pi}{2}\right). $$ (1) PRoOF If $\delta\leq|\varphi|\leq\pi/2,$ then the right side $\mathfrak{s l}(x)$ is not less than -α cos pp ≥ -α cos $p\delta=\alpha$ cos $\delta=1.$ and it exceeds β(cos $\delta\mathsf{P}-\alpha=1\,\mathrm{if}\left|\,\varphi\right|\leq\delta.$ // 17.26 Theorem $f\vdash<p<\infty.$ then there is a constant $A_{p}<\infty$ such that the inequality $$ \|\psi h\|_{p}\leq A_{p}\|h\|_{p} $$ (1) holds for every $h\in D(T).$ More explicitly, the conclusion is that yh (defined in Sec.17.24) is in $H^{p},$ and that $$ \{_{T}|(\psi h)_{r}|^{p}\ d\sigma\leq A_{p}^{p}\prod_{T}h|^{p}\ d\sigma\quad.\ \ (0\leq r<1) $$ (2) where $d\sigma=d\theta/2\pi$ is the normalized Lebesgue measure on ${\boldsymbol{T}}.$ Note that $\boldsymbol{h}$ is not required to be a real function in this theorem, which asserts that $\psi\colon L^{p}\to H^{p}$ is a bounded linear operator. PROOF Assume first that $1<p\leq2,$ that $h\in D(T),$ h ≥ 0 $\hbar\neq0.$ and let u be ${\boldsymbol{U}}.$ If the real part o $f=\psi h.$ Formula 11.5(2) shows that $u=P[h]$ , hence $\scriptstyle w\gg0$ in Since $\alpha=\alpha_{p}$ and $g=f^{p},\ g(0)>0.$ Also, $\operatorname{ned}f$ has no zero in ${\boldsymbol{U}},$ there is a $g\in H(U)$ $U$ is simply connected such that ${\boldsymbol{U}}$ that satisfies $|\,\varphi\,|\,<\,\pi/2.$ cos p,where $\varphi$ is a real function with domain $u=|\,f|$ $\beta=\beta,$ are chosen as in Lemma 17.25,it follows tha $$ \left[\frac{1}{r}[f_{r}|^{p}\:d\sigma\leq\beta\:\prod_{T}^{}(u_{r})^{p}\:d\sigma-\alpha\:\right]_{T}^{}|f_{r}|^{p}\:\cos\:(p\varphi_{r})\:d\sigma $$ (3) for $0\leq r<1,$352 REAL AND coMPLEX ANALYSIs Note that |fP cos $p\varphi=\mathrm{Re}\,g$ The mean value property of harmonic Hence functions shows therefore that the last integral in (3) is equal to $\operatorname{Re}g(0)>0.$ $$ \{_{T}^{\ast}|f_{r}|^{p}\;d\sigma\leq\beta\ |_{\mathcal{T}}h^{p}\;d\sigma\qquad(0\leq r<1) $$ (4) because $u=P[h]$ implies $\|u_{r}\|_{p}\leq\|h\|_{p}.$ Thus $$ \|\psi h\|_{p}\leq\beta^{1/p}\|h\|_{p} $$ (5) if $h\in D(T),\,h\geq0.$ If ${\boldsymbol{h}}$ is an arbitrary (complex) function in $\scriptstyle{\bar{u}}(T)$ the preceding result applies to the positive and negative parts of the real and imaginary parts of $h_{\mathrm{.}}$ This proves (2), for $1<p\leq2,$ with $A_{p}=4\beta^{1/p}.$ $2<p<\infty.$ Let $w\in D(T),$ To complete the proof, consider the case where $\boldsymbol{\mathit{q}}$ is the exponent conjugate to p. Put $\tilde{w}(e^{i\theta})=w(e^{-i\theta}).$ A simple compu- tation, using Fubini's theorem, shows for any $h\in D(T)$ that $$ \left. (\frac\phi(\psi h)_{r}\,\tilde{w}\,\,d\sigma=\right)_{T}^{}(\psi w)_{r}\tilde{h}\,\,d\sigma\qquad(0\leq r<1). $$ (6) Since $q<2,(2)$ holds with w and $\boldsymbol{\mathit{q}}$ in place of ${\boldsymbol{h}}$ i and ${\boldsymbol{p}},$ so that (6) leads to $$ \left|\bigcup_{T}(\psi h)_{r}\tilde{w}\ d\sigma\right|\leq A_{q}\left|\left|w\right|_{q}\left|h\right|\right|_{p}. $$ (7) Now let w range over the unit ball of $\scriptstyle B(t)$ and take the supremum on the left side of (7). The result is $$ \left\{\oint_{T}|(\psi h)_{r}|^{p}\;d\sigma\right\}^{1/p}\leq A_{q}\biggl\{ \{\frac{}{}|h\,|^{p}\;d\sigma\biggr\}^{1/p}\qquad(0\leq r<1). $$ (8) Hence (2) holds again, with $A_{p}\leq A_{q}.$ // (If we take the smallest admissible values for $A_{p}=A_{q}.$ .) and $A_{q},$ the last calculation can be reversed, and shows that $A_{\mathfrak{p}}$ Exercises 1 Prove Theorems 17.4 and 17.5 for upper semicontinuous subharmonic functions $\boldsymbol{\mathit{U}}$ $\,\kappa\,=\,1$ 2 Assumefe H(Q) and prove that log for all $z\in U,$ Prove that if there is one such harmonic majorant u of $|f|^{p},$ 1s such that $|f(z)|^{p}\leq u(z)$ $(1+|f|)$ is subharmonic in Q 3 Suppose $0<p\leq\infty$ and fe H(U). Prove thatfe HP if and only if there is a harmonic function uin then there is at least one, say $u_{f}\leq u,\quad$ Prove that ${\boldsymbol{u}}_{f}.$ (Explicitly $|f|_{*}^{p}\leq u_{f}$ and $u_{f}$ is harmonic; and if $|f|^{p}\leq u$ and ${\mathcal{U}}$ harmonic, then $\|f\|_{p}=u_{f}(0)^{1/p}$ Hint: Consider the harmonic functions in D(0; R), with boundary values |f|P, and le $\scriptstyle n\sim1$ 4 Prove likewise thatfe N if and only if logt If| has a harmonic majorant in $\boldsymbol{\mathit{U}}$HP-SPACES 353 s Suppose e $H^{p},$ $\varphi\in H(U),$ and $\varphi(U)\subset U.$ Does it follow $\operatorname{that}f\circ\varphi\in H^{p\neq\gamma}$ Answer the same ques- tion with ${\mathbf{}}N$ in place o $H^{p}.$ ${\mathfrak{G P}}\cup<r<s\leq\infty.$ show that ${\boldsymbol{H}}^{*}$ is a proper subclass of $H^{r}.$ T Show that $H^{\omega}$ is a proper subclass of the intersection of all $H^{p}$ with $p<\cdots$ 8 1 $\{f\in H^{1}$ and $\Im f^{*}\in D(T),$ prove that fe $H^{p}.$ 9.Suppos $f\in H(U)$ and f(U) is not dense in the plane. Prove that f has finite radial limits at almost all points of ${\boldsymbol{T}}.$ 10 Fix $x\in U,$ Prove that the mapping $f\to f(x)$ is a bounded linear functional on ${\textstyle H}^{2}$ Since $H^{2}$ is a Hilbert space, this functional can be represented as an inner product with some $g\in H^{2}.$ Find this $g.$ 11 Fix $\alpha\in U.$ How large can $|f^{\prime}(\alpha)|$ be if $\|f\|_{2}\leq17$ Find the extremal functions. Do the same for $f^{(n)}(x).$ 12 Suppose $p\geq1,f\in H^{p},$ and f* is real ae. on T. Prove that fis then constant. Show that this result is false for every $p<1.$ $\textstyle{\mathit{\epsilon}}\times{\mathit{1}}$ 13 Suppose $f\in H(U),$ and suppose there exists an $M<\infty$ such that $\boldsymbol{\mathit{f}}$ maps every circle of radius and center O $\mathbf{\partial}$ O onto a curve $\gamma_{r}$ whose length is at most $M.$ Prove that f has a continuous extension to $\bar{U}$ and that the restriction of f to ${\mathbf{}}T$ is absolutely continuous. 14 Suppose ${}^{\mu}$ is a complex Borel measure on ${\mathbf{}}T$ such that $$ \{r_{T}^{\mathrm{(int}\ d)}(t)=0\qquad(n=1,\,2,\,3,\,\ldots).\qquad\qquad\qquad\qquad\qquad(n=1,\,2,\,3,\,\ldots). $$ Prove that then either $\mu=0$ or the support of pis all of ${\boldsymbol{T}}.$ 15 Suppose $\scriptstyle{\mathcal{K}}$ is a proper compact subset of the unit circle ${\boldsymbol{T}}.$ Prove that every continuous function on ${\bar{\boldsymbol{K}}}$ can be uniformly approximated on ${\cal K}$ by polynomials. Hint: Use Exercise 14. 16 Complete the proof of Theorem 17.17 for the case $0<p<1.$ 17 Let p be a nonconstant inne function with no zero in ${\boldsymbol{U}},$ (a) Prove that 1/9 生 $H^{p}\operatorname{if}p>0.$ ${\mathrm{lim}}_{r\to1}\varphi(r e^{i\theta})=0.$ (b) Prove that there is at least one e"e T such that Hint: log lolis a negative harmonic function. 18 Suppose $\textstyle\varnothing\quad$ is a nonconstant inner function, $\scriptstyle w\backslash v.$ and αs q(U). Prove that lim,- $\varphi(r e^{i\theta})=\alpha$ for at least one $e^{i\theta}\in T.$ 19 Suppose ${\textstyle H^{1}}$ and $1/f\in H^{1}.$ Prove that fis then an outer function 20 Suppose f ∈ $H^{1}$ and Re $[f(z)]>0$ for al $z\in U$ . Prove that fis an outer function. 21 Prove tha $1f\in N$ if and only $\mathrm{f}f=g/h,$ where $\scriptstyle{\mathcal{G}}$ and $h\in H^{\infty}$ and h has no zero in ${\boldsymbol{U}}.$ 22 Prove the following converse of Theorem 15.24: $\operatorname{If}f\in H(U)$ and if $$ \operatorname*{lim}_{r\to1}\int_{-\pi}^{\pi}|\log|f(r e^{i\theta})||\,d\theta=0, $$ (*) then fis a Blaschke product $H i n t:({\overset{*}{*}})$ implies $$ \operatorname*{lim}_{r\to1}\int_{-\pi}^{\pi}\log^{+}\ |f(r e^{i\theta})|\ d\theta=0. $$ $f=B g,\,\lbrace$ g has no zeros, $|g\,|\leq1,$ it follows from Theorems $17.3$ and 17.5 that log $|f|=0,$ so $|f|\leq1.$ Now Since log $|f|\geq0,$ and (") holds with 1/g in place of f. By the first argument,|1/gl≤ 1. Hence $|g|=1.$ 23 Find the conditions mentioned in Sec. 17.22354 REAL AND coMPLEX ANALYSis 24 The conformal mapping of $\boldsymbol{\mathit{U}}$ onto a vertical strip shows that M. Riesz's theorem on conjugate either. functions cannot be extended to $p=\infty.$ Deduce that it cannot be extended to $p=1$ $\mathbf{25}$ Suppose $1<p<\infty,$ and associate with each f∈ $D(T)$ its Fourier coefficients $$ \hat{f}(n)=\frac{1}{2\pi}\left|_{-\pi}^{\pi}f(e^{i n})e^{-i n t}\,d t\qquad(n=0,\pm1,\;\pm2,\,...).\nonumber\,\right|^{\pi}, $$ Deduce the following statements from Theorem 17.26: ${\hat{g}}(n)=0$ (a) To each f $\in{\mathcal{D}}(T)$ there corresponds a function $g\in L^{p}(T)$ such that ${\hat{g}}(n)={\hat{f}}(n)$ for $n\geq0$ but for al $n<0.$ In fact, there is a constant ${\boldsymbol{C}},$ depending only on ${\mathfrak{p}},$ D, such that $$ \|g\|_{p}\leq C\|f\|_{p}. $$ The mapping $f\to g$ is thus a bounded linear projection of $n<0.$ into ${\cal D}(T).$ The Fourier series of $\scriptstyle{\mathcal{G}}$ g is obtained from that of f by deleting the terms with $D(T)$ (b) Show that the same is true if we delete the terms with $n<k,$ where $\boldsymbol{k}$ is any given integer. sequence in (c) Deduce from ${\mathfrak{(}}b{\mathfrak{)}}$ that the partial sums. ${\mathfrak{s}}_{n}$ 。 of the Fourier series of any fe $D(T)$ form a bounded $D(T).$ Conclude further that we actually have $$ \operatorname*{lim}_{n\to\infty}\left\|f-s_{n}\right\|_{p}=0. $$ (d) Iff ∈ ${\boldsymbol{D}}(T)$ and if $$ F(z)=\sum_{n=0}^{\infty}\hat{f}(n)z^{n}, $$ then $F\in H^{p},$ and every $F\in H^{p}$ is so obtained. Thus the projection mentioned in ${\mathbf{}}(a)$ may be regarded as a mapping ot $D(T)$ onto $H^{p}.$ 26 Show that there is a much simpler proof of Theorem 17.26 $\quad\Gamma\rho=2,$ and find the best value of $A_{2}.$ $27$ Suppose f(z)= $\textstyle\sum_{i=0}^{\infty}$ a,z" in $\boldsymbol{\mathit{U}}$ and $\scriptstyle\sum|a_{n}|<\infty$ Prove that $$ \bigcap_{0}^{1}|f^{\prime}(r e^{i\beta})|\ d r<\infty $$ for all e $\mathrm{28}$ Prove that the following statements are correct if $\{n_{k}\}$ is a sequence of positive integers which tends to oo sufficiently rapidly. If $$ f(z)=\sum_{k=1}^{\infty}{\frac{z^{n}}{k}} $$ then IJ(2)I > n,/(10k for all such that $$ 1-{\frac{1}{n_{k}}}<|z|<1-{\frac{1}{2n_{k}}}. $$ Hence $$ \bigcap_{0}^{1}|f^{\prime}(r e^{i\theta})|\ d r=\infty $$ for every e,although $$ \quad,\qquad\qquad\operatorname*{lim}_{R\to1}\prod_{\vartheta_{0}}^{R}f^{\prime}(r e^{i\beta})\ d r $$HP-SPACES 355 exists (and is finite for almost al 1 ${\boldsymbol{\theta}}.$ lnterpret this geometricaly, in terms of the lengths of the images under fof the radif in ${\boldsymbol{U}}.$ $\mathbf{29}$ Use Theorem 1711 to obtain the following characterization of the boundary values of $H^{p}.$ functions, for $1\leq p\leq\infty;$ A function $g\in L^{p}(T){\mathrm{~is~}}f^{*}\left(a.\epsilon.\right)$ for somefe HP if and only if $$ \frac{1}{2\pi}\prod_{-\pi}^{\pi}g(e^{i t})e^{-i n t}~d t=0 $$ for all negative integers n