CHAPTER FOUR ELEMENTARY HILBERT SPACE THEORY Inner Products and Linear Functionals 4.1 Definition A complex vector space ${\boldsymbol{H}}$ is called an inner product space (or unitary space) if to each ordered pair of vectors $\scriptstyle{\mathcal{X}}$ and $\nu\in H$ there is associ- ated a complex number (x,y),the so-called“inner product”(or“scalar product ”) of $\scriptstyle{\mathcal{X}}$ and $y_{\circ}$ , such that the following rules hold: (b) (a)v, $x)=({\overline{{x}}},{\overline{{y}}}$ 八).(The bar denotes complex conjugation.)) $\varepsilon\in H.$ (e)(x, $\scriptstyle v_{i}=0$ $(x+y,z)=(x,z)+(y,z)$ if ${\mathcal{X}},$ y, and is a scalar. (c) (αx, y) = α(x, y)if b for all $\mathrm{*{rer}}\,n$ and $\scriptstyle{\dot{\mathbf{x}}}$ (a(x, $\scriptstyle v\geq V$ only if $\scriptstyle{\mathcal{X}}$ c and $\textstyle v\in H$ $\scriptstyle x\;=\;0$ Let us list some immediate consequences of these axioms: (c implies that $[0,y)=0$ for al $\textstyle v\in H$ (b) and(c) may be combined into the statement: For every $\nu\in H$ the mapping $x\to(x,y)$ is a linear functional on $\textstyle H.$ (a) and $\mathbf{(c)}$ show that $(x,\alpha y)={\tilde{\alpha}}(x,y).$ (a) and (b) imply the second distributive law: $$ (z,\,x+y)=(z,\,x)+(z,\,y). $$ By (d), we may define |xl|, the norm of the vector $x\in H.$ to be the non- negative square root of (x,x). Thus (J) $$ \|x\|^{2}=(x,\,x). $$ 76ELEMENTARY HILBERT SPACE THEORY 77 4.2 The Schwarz Inequality The properties 4.1 (a)to (d) imply that $$ |(x,\,y)|\leq\|x\|\ \|y\| $$ for all x and $\textstyle{\epsilon\,\epsilon\,n}$ PROOF Put $A=\|x\|^{2}$ $B=|(x,y)|$ and $C=\|y\|^{2}$ There is a complex number α such that $\scriptstyle x\,=\,1$ and cy, $x)=B.$ For any real ${\boldsymbol{r}}_{\!_{J}}$ we then have $$ (x-r\alpha y,\;x-r\alpha y)=(x,\,x)-r\alpha(y,\;x)-r\bar{\alpha}(x,\,y)+r^{2}(y,\,y) $$ (1) The expression on the left is real and not negative. Hence $$ 4-2B r+C r^{2}\geq0 $$ (2) positive for every real r. If . If $\ C>0,$ take $r=B/C$ in (2), and obtain $B^{2}\leq4C$ / $\scriptstyle C=0,$ we must have $\theta=0;$ otherwise (2) is false for large ${\boldsymbol{r}}.$ 4.3 The Triangle Inequality For $\scriptstyle{\mathcal{X}}$ and $\nu\epsilon\,\eta,$ we have $$ \|x+y\|\leq\|x\|+\|y\|. $$ PROOF By the Schwarz inequality $$ \begin{array}{c}{{|x+y||^{2}=(x+y,\,x+y)=(x,\,x)+(x,\,y)+(y,\,x)+(y,\,y)}}\\ {{\leq\,\|x\|^{2}+2\|x\|\,\|y\|+\|y\|^{2}=(\|x\|+\|y\|)^{2}.}}\end{array} $$ / 4.4 Definition It follows from the triangle inequality that $$ \|x-z\|\leq\|x-y\|+\|y-z\|\quad\quad(x,\,y,\,z\in H). $$ (1) If we define the distance between $\scriptstyle{\mathcal{X}}$ and $\mathbf{\vec{y}}$ to be $\|x-y\|,$ all the axioms for a .metric space are satisfied; here, for the first time, we use part (e) of Definition 4.1. Thus ${\boldsymbol{H}}$ is now a metric space. If this metric space is complete, ie., if every Cauchy sequence converges in $\textstyle H,$ I, then ${\boldsymbol{H}}$ is called a Hilbert space Throughout the rest of this chapter, the letter ${\boldsymbol{H}}$ will denote a Hilbert space 4.5 Examples (a)For any fixed ${\boldsymbol{n}}_{\mathsf{J}}$ the set $C^{n}$ of all ${\boldsymbol{\pi}}$ -tuples $$ x=(\xi_{1},\dots,\ \xi_{n}), $$ where $\zeta_{1},\cdot\cdot\cdot,\ \zeta_{n}$ are complex numbers, is a Hibert space if addition and scalar multiplication are defined componentwise, as usual, and if $$ (x,\,y)=\sum_{j=1}^{n}\,\xi_{j}\,\tilde{\eta}_{j}\qquad(y=(\eta_{1},\,\ldots,\,\eta_{m}). $$$78$ REAL AND COMPLEX ANALYSIS (b)f ${\boldsymbol{\mu}}$ is any positive measure, $L^{2}(\mu)$ is a Hilbert space, with inner product $$ (f,g)=\int_{X}f{\bar{g}}\;d\mu. $$ The integrand on the right is in $\scriptstyle I(\mu).$ by Theorem 3.8,so that $(f,\,g)$ is well defined. Note that $$ \|f\|=(f,f)^{1/2}=\left\{\int_{X}|f|^{2}\ d\mu\right\}^{1/2}=\|f\|_{2}\,. $$ The completeness of $\scriptstyle{E(n)}$ (Theorem 3.1) shows that $\scriptstyle{E(\theta)}$ is indeed a Hilbert space.[We recall that $\scriptstyle{E(n)}$ should be regarded as a space of equivalence classes of functions; compare the discussion in Sec. 3.10.1 For $H=E^{2}(\mu),$ the inequalities 4.2 and 4.3 turn out to be special cases of the inequalities of H6lder and Minkowski Note that Example (a) is a special case of (b). What is the measure in $(a)^{\gamma}$ (c)The vector space of all continuous compiex functions on [O,1] is an inner product space if $$ (f,g)=\prod_{0}^{1}f(t){\overline{{g(t)}}}\;d t $$ but is not a Hilbert space 4.6 Theorem For any fixed ye H, the mappings $$ x\longrightarrow(x,\,y),\quad x\longrightarrow(y,\,x),\quad x\longrightarrow\|x\| $$ are continuous functions on $\textstyle H.$ PxoOF The Schwarz inequality implies that $$ |(x_{1},y)-(x_{2},y)|=|(x_{1}-x_{2},y)|\leq\|x_{1}-x_{2}\|\,\|y\|, $$ true for which proves that $x\to(x,y){\mathrm{~if~}}$ s, in fact, uniformly continuous, and the same is yields $x\to(y,\,x)$ The triangle inequality $\|x_{1}\|\leq\|x_{1}-x_{2}\|+\|x_{2}\|$ $$ \|x_{1}\|=\|x_{2}\|\leq\|x_{1}-x_{2}\|, $$ and if we interchange $x_{1}$ and $\scriptstyle{X_{2}}$ we see that $$ \bigstar||x_{1}||-\|x_{2}||\ |\leq\|x_{1}-x_{2}\| $$ for all $x_{1}$ and $x_{*}\in H$ Thus $x\to\left|x\right|$ is also uniformly continuous // 4.7 Subspaces A subset ${\cal M}$ of a vector space ${\mathbf{}}V$ is called a subspace of ${\mathbf{}}V$ if $\textstyle{M}$ is itself a vector space, relative to the addition and scalar multiplication which are defined in $V.$ $\mathbf{A}$ necessary and sufficient condition for a set $M\subset V$ to be a sub- spac is that $x+y\in M$ and $x x\in M$ whenever x $\scriptstyle{\mathcal{X}}$ and $y\in M$ and $\scriptstyle{\dot{\mathbf{x}}}$ is a scalar.ELEMENTARY HLBERT SPACE THEORY 79 In the vector space context, the word“subspace”will always have this meaning. Sometimes, for emphasis, we may use the term“linear subspace”in place of subspace. For example, if ${\mathbf{}}V$ is the vector space of all complex functions on a set ${\boldsymbol{S}},$ S, the set of all bounded complex functions on $\boldsymbol{\mathsf{S}}$ is a subspace of ${\mathit{V}},$ but the set of al fe V with |f(x)|≤1 for all $x\in{\mathcal{S}}$ is not. The real vector space $R^{3}$ has the follow ing subspaces, and no others: (a) $R^{3},$ (b) all planes through the origin O,(c) all straight lines through O, and (d) {0} A closed subspace of ${\boldsymbol{H}}$ I is a subspace that is a closed set relative to the topol- ogy induced by the metric of $H.$ Note that i M is a subspace of $\textstyle H,$ so is its closure ${\bar{M}}.$ To see this, pick xand y in $\bar{M}$ and let α be a scalar. There are sequences $\{x_{n}\}$ and $\{y_{n}\}$ in ${\cal M}$ that converge tO $\scriptstyle{\mathcal{X}}$ and y, respectively. It is then easy to verify that $x+y\in{\bar{M}}$ and αx ∈ M. $x_{\underline{{{n}}}}+y_{n}$ and $\alpha X_{n}$ converge to $x+y$ and αx, respectively. Thus 4.8 Convex Sets A set $\boldsymbol{E}$ in a vector space ${\mathbf{}}V$ is said to be convex if it has the following geometric property: Whenever x $\mathbf{\beta}\in E,y\in E,$ and $\ 0<t<1$ the point $$ z_{t}=(1-t)x+t y $$ line segment in ${\boldsymbol{V}},$ ,from $\scriptstyle{\mathcal{X}}$ also lies in E. As t runs from O to 1, one may visualize $\mathrm{z}_{t}$ , as describing a straight to y. Convexity requires that $\boldsymbol{E}$ contain the segments between any two of its points. It is clear that every subspace of ${\mathbf{}}V$ is convex Aso, if $\boldsymbol{E}$ is convex, so is each of its translates $$ E+x=\{y+x\colon y\in E\}. $$ to ${\ y},$ 4.9 Orthogonality If $(x,y)=0$ for some $\scriptstyle{\mathcal{X}}$ and $\gamma\in H,$ we say that $\scriptstyle{\mathcal{X}}$ is orthogonal $\underline{{\boldsymbol{L}}}$ and sometimes write x l y. Since (x, $\scriptstyle y\;=\;0$ implies Gy, $\scriptstyle x\;=\;0,$ the relation is symmetric. subspace of $\textstyle H,$ let $M^{\bot}$ be the set of all $\nu\in H$ which are orthogonal to $x;$ and if $\textstyle{M}$ is a Let $x^{\perp}$ denote the set of all $\nu\in H$ which are orthogonal to every $x\in M.$ Note that $x^{\perp}$ l is a subspace of $\textstyle H,$ since $x\perp y$ and xl y' implies $x\perp(y+y)$ and xl αy. Also, $x^{\perp}$ is precisely the set of points where the continuous function $y\to(x,y)$ is O. Hence $\ x^{\perp}$ is a closed subspace of $H.$ Since $$ M^{\perp}=\bigcap_{x\in M}x^{\perp}, $$ $M^{\bot}$ is an intersection of closed subspaces, and it follows that $M^{\bot}$ is a closed subspace o $H.$ 4.10 Theorem Every nonempty, closed, convex set $\boldsymbol{E}$ in a Hilbert space ${\boldsymbol{H}}$ con- tains a unique element of smallest norm.80 REAL AND cOMPLEX ANALYSIs every $x\in E$ In other words, there is one and only one $v_{\mathrm{e}}\in E$ such that $\|x_{0}\|\leq\|x\|$ for PR0oF An easy computation, using only the properties listed in Definition 4.1, establishes the identity $$ \|x+y\|^{2}+\|x-y\|^{2}=2\|x\|^{2}+2\|y\|^{2}\qquad(x{\mathrm{~and~}}y\in H). $$ (1) This is known as the parallelogram law:If we interpret llxll to be the length of the vector x,(1) says that the sum of the squares of the diagonals of a parallelogram is equal to the sum of the squares of its sides, a familiar propo and obtain sition in plane geometry. For any $\scriptstyle{\mathcal{X}}$ and $\nu\in E_{t}$ we apply(1) to yx and y Let $\delta=\operatorname*{inf}\left\{\|x\|\colon x\in E\right\}.$ $$ {\frac{1}{4}}\|x-y\|^{2}={\frac{1}{2}}\|x\|^{2}+{\frac{1}{2}}\|y\|^{2}-\left|{\frac{x+y}{2}}{\right|} |^{2}. $$ (2) Since $\boldsymbol{E}$ is convex,(x + y)/2 ∈ E. Hence $$ \|x-y\|^{2}\leq2\|x\|^{2}+2\|y\|^{2}-4\delta^{2}\qquad(x{\mathrm{~and~}}y\in E). $$ (3) If also $\|x\|=\|y\|=\delta,$ then (3) implies $x=y,$ and we have proved the unique- ness assertion of the theorem. $\|y_{n}\|\to\delta$ as $n\to G\to$ Replace $\scriptstyle{\mathcal{X}}$ x and $\mathbf{\vec{y}}$ $y_{n}\in E$ shows that there is a sequence $\{y_{n}\}$ in $\boldsymbol{E}$ so that The definition of $\delta$ $m\to\infty$ sequence. Since ${\boldsymbol{H}}$ as n→00. Since in (3) by $y_{n}$ ,and $y_{m}$ . Then, as $n\to\varnothing.$ and $\|y_{n}-x_{0}\|\to0,$ , the right side of (3) will tend to 0. This shows that and $\boldsymbol{E}$ is closed, $v_{\mathrm{e}}\in E$ Since the i.e., $\{y_{n}\}$ is a Cauchy is complete, there exists an $v_{n}\in H$ so that $y_{n}\to x_{0}\,,$ norm is a continuous function on ${\boldsymbol{H}}$ (Theorem 4.6), it follows that $$ \|x_{0}\|=\operatorname*{lim}_{n\to\infty}\|y_{n}\|=\delta. $$ // 4.11 Theorem Let M be a closed subspace of a Hilbert space $\textstyle H.$ (a)Every xe ${\boldsymbol{H}}$ has then a unique decomposition $$ x=P x+Q x $$ into a sum of $P x\in M$ and $Q x\in M^{1}$ (c) (b)Px and Qx are the nearest points to M and $Q\colon H\to M^{4}$ in $\textstyle{M}$ and in $M^{\bot}\colon$ , respectively The mappings P: $\scriptstyle{\mathcal{X}}$ are linear. $\scriptstyle n\, arrow$ (d) $\|x\|^{2}=\|P x\|^{2}+\|Q x\|^{2}.$ Corollary I $f M\neq H,$ then there exists $\operatorname{ver}n$ $\scriptstyle y^{n}\,0.$ such that $y\perp\ M.$ ${\mathbf{}}P$ and $Q_{\mathbf{\delta}}Q$ are called the orthogonal projections of ${\boldsymbol{H}}$ onto $\textstyle{M}$ and $M^{\bot}.$ELEMENTARY HILBERT SPACE THEORx 81 some vectors $x^{\prime},\,x^{\prime\prime}$ in $\textstyle{M}$ PROOF As regards the uniqueness in (a),suppose that $x^{\prime}+y^{\prime}=x^{\prime\prime}+y^{\prime}$ for and $y^{\prime},y^{\prime\prime}$ in $M^{\dagger}.$ Then $$ x^{\prime}-x^{\prime\prime}=y^{\prime\prime}-y^{\prime}. $$ Since $x^{\prime}-x^{\prime}$ e M, $y^{\prime\prime}\subset y^{\prime}\in M^{\perp}.$ , and ${\cal M}\,\cap\,{\cal M}^{\bot}=\{0\}$ [an immediate conse- quence of the fact that $(x,x)=0$ implies $x=0|.$ we have $x^{\prime\prime}=x^{\prime},\,y^{\prime\prime}=y^{\prime}.$ To prove the existence of the decomposition, note that the set $$ x+M=\{x+y\colon y\in M\} $$ is closed and convex. Define ${\boldsymbol{Q}}{\boldsymbol{x}}$ to be the element of smallest norm in Since To prove that $\boldsymbol{\mathit{Q}}$ ; this exists, by Theorem 4.10. Define $\ P x=x-Q x$ ${\mathbf{}}P$ P maps ${\boldsymbol{H}}$ into $M.$ $y\in M.$ $x+M$ $Q x\in x+M,$ it is clear that $P x\in M.$ Thus Assume maps ${\boldsymbol{H}}$ into $M^{\bot}$ we show that $(Q x,y)=0$ for all $\|y\|=1,$ without loss of generality, and put $\scriptstyle z\ge Q x.$ The minimizing property of Qx shows that $$ (z,z)=\|z\|^{2}\leq\|z-\alpha y\|^{2}=(z-\alpha y,z-\alpha y) $$ for every scalar α.This simplifies to $$ 0\leq-\alpha(y,z)-\bar{\alpha}(z,y)+\alpha\bar{\alpha}. $$ With $x=(z,y),$ this gives $$ 0\leq-\,|(z,\,y)|^{2},\,\mathrm{so~that}\,(z,\,y)=0. $$ Thus $Q x\in M^{4}$ We have already seen that Pxe M. If ye M, it follows that $$ \|x-y\|^{2}=\|Q x+(P x-y)\|^{2}=\|Q x\|^{2}+\|P x-y\|^{2} $$ which is obviously minimized when $y=P x.$ We have now proved (a) and (6b).If we apply (a) to x,t0 ${\mathfrak{y}},$ and to $\alpha x+\beta y,$ we obtain $$ P(\alpha x+\beta y)-\alpha P x-\beta P y=\alpha Q x+\beta Q y-Q(\alpha x+\beta y). $$ linear The left side is in $\textstyle{M},$ the right side in $M^{\bot}.$ Hence both are ${\boldsymbol{0}},$ so ${\mathbf{}}P$ and $\textstyle{\cal Q}$ are $P x\in M,x\neq P x,$ Since PxL Qx, (d) follows from (a) $x\in H,\;x\notin M,$ and put y= Qx. Since // To prove the corollary, take $y=x-P x\neq0.$ hence We have already observed that $H.$ It is a very important fact that $\ a l l$ continuous linear $x\to(x,\ y)$ is, for each $\vee\epsilon\,{\cal H},$ a continuous linear functional on 1 ${\boldsymbol{H}}$ are of this type functionals on 4.12 Theorem If $\boldsymbol{\ L}$ is a continuous linear functional on $\textstyle H,$ then there is a unique y e H such that $$ L x=(x,y)\qquad(x\in H). $$ (1)82 REAL AND COMPLEX ANALYSIs PROOF If $\scriptstyle L\scriptstyle z\times=\mathbf{0}$ for all x, take $y=0.$ Otherwise, define $$ M=\{x\colon L x=0\}. $$ (2) The linearity of ${\boldsymbol{L}}$ shows that $\textstyle{M}$ is a subspace. The continuity of ${\boldsymbol{L}}$ shows that ${\cal M}$ is closed. Since $L x\neq0$ for some $x\in R.$ Theorem 4.11 shows that $M^{\bot}$ does not consist of $\mathbf{0}$ alone Hence there exists $\kappa\in M^{4},$ with $|z|=1.$ Put $$ u=(L x)z-(L z)x. $$ (3) Since $L u=(L x)(L z)-(L z)(L x)=0,$ we have ue $M.$ Thus (w $z)=0.$ This gives $$ L x=(L x)(z,\,z)=(L z)(x,\,z). $$ (4) Thus (1) holds with then $(x,z)=0$ $\scriptstyle{\mathcal{I}}$ is easily proved, for if $(x,y)=(x,$ y) for an $x\in R,$ set $y=\alpha z,$ where $\bar{\alpha}=L z.$ The uniqueness of $z=y-y^{\prime};$ for all $x\in H$ ; in particular, (Z, $z)=0,$ hence $\scriptstyle z=0,$ // Orthonormal Sets The set 4.13 Definitions If ${\mathbf{}}V$ is a vector space, if $x_{1},\ldots,x_{k}\in V,$ and if $c_{1},\,\cdot\cdot\cdot,\,c_{k}$ are scalars, then $c_{1}x_{1}\dots\cdot\cdot+c_{k}x_{k}$ is called a linear combination of $x_{1},\ \ldots,\ x_{k}$ $\boldsymbol{\mathsf{S}}$ is $\{x_{1},\ldots,x_{k}\}$ is called independent if $c_{1}x_{1}+\mathbf{\cdot\cdot\cdot}+c_{k}x_{k}=0$ implies that $c_{1}=\cdot\cdot\cdot=c_{k}=0.$ A set $\operatorname{se}V$ is independent if every finite subset of independent. The set [S] of all linear combinations of all finite subsets of $\boldsymbol{\mathsf{S}}$ (also called the set of all finite linear combinations of members of ${\boldsymbol{S}}\}$ is clearly a vector space; [S]is the smallest subspace of ${\mathbf{}}V$ which contains ${\boldsymbol{S}};$ [S] is called the span of S, or the space spanned by ${\dot{S}}.$ index set A set of vectors $u_{\alpha}$ in a Hilbert space $\textstyle H,$ where α runs through some $A,$ is called orthonormal if it satisfies the orthogonality relations $(u_{x},u_{\beta})=0$ for all $\alpha\neq\beta,\;\alpha\in A,$ and $\beta\in A,$ and if it is normalized so that $|u_{x}|=1$ for each α∈ A. In other words, $\{u_{\alpha}\}$ is orthonormal provided that $$ (u_{\alpha},u_{\beta})= \{{1\atop0}\;\;\mathrm{if}\;\alpha=\beta,\quad $$ (1) If $\{u_{x}\colon x\in A\}$ is orthonormal, we associate with each $x\in H$ a complex function ${\hat{x}}$ on the index set $A,$ defined by $$ {\hat{x}}(x)=(x,u_{\alpha})\qquad(\alpha\in A). $$ (2) One sometimes calls the numbers ${\hat{x}}(\alpha)$ the Fourier coefficients of x, relative to the set $\scriptstyle(u_{n}).$ We begin with some simple facts about finite orthonormal setsELEMENTARY HLBERT SPACE THEORY 83 4.14 Theorem Suppose that $\{u_{x}:\alpha\in A\}$ is an orthonormal set in ${\boldsymbol{H}}$ and that ${\mathbf{}}F$ is a finite subset of A. Let $M_{F}$ be the span of $\{u_{x}:\alpha\in F\}$ (a)If p is a complex function on A that is $\mathbf{0}$ outside $F_{\circ}$ 、then there is a vector y ∈ $M_{r},$ namely $$ y=\sum_{\alpha\in F}\varphi(\alpha)u_{\alpha} $$ (1) that has Nc) = q(c) for every αe A. Also, $$ \|y\|^{2}=\sum_{\alpha\,\in\,F}|\,\varphi(\alpha)|^{2}. $$ (2) (b)Ifx∈ H and $$ s_{F}(x)=\sum_{\alpha\in F}{\hat{x}}(\alpha)u_{x} $$ (3) then $$ \|x-s_{F}(x)\|<\|x-s\| $$ (4) for every se M,,except for $s=s_{F}(x),a$ nd $$ \sum_{\alpha\,\in\,F}|\,\hat{x}(\alpha)\,|^{2}\,\leq\,\|x\|^{2}. $$ (5) PROOF Part (a) is an immediate consequence of the orthogonality relations 4.13(1) In the proof of (b), let us write $s_{F}$ in place of s(x), and note that $\scriptstyle j_{i}(i)=$ ${\hat{x}}(x)$ for all $\textstyle{\hat{\mathbf{r}}}\in F$ F.This says that $(x-s_{F})\perp u_{n}$ if $\propto\epsilon,$ hence $(x-s_{r})\perp$ $(s_{F}-s)$ for every se $\scriptstyle M_{r},$ and therefore $$ \|x-s\|^{2}=\|(x-s_{F})+(s_{F}-s)\|^{2}=\|x-s_{F}\|^{2}+\|s_{F}-s\|^{2}. $$ (6) This gives (4). With $\scriptstyle s=0.$ (6) gives $\|s_{F}\|^{2}\leq\|x\|^{2}$ , which is the same as(5), / because of (2) The inequality (4) states that the “partial sum”s,(x) of the “Fourier series” Exc)u。 of $\scriptstyle{\mathcal{X}}$ is the unique best approximation to $\scriptstyle{\mathcal{X}}$ in $M_{F},$ relative to the metric defined by the Hilbert space norm. 4.15 We want to drop the finiteness condition that appears in Theorem 4.14 (thus obtaining Theorems 4.17 and 4.18) without even restricting ourselves to sets that are necessarily countable. For this reason it seems advisable to clarify the meaning of the symbol $\Sigma_{\circ}$ 4p(c) when a ranges over an arbitrary set $A.$ Assume $0\leq\varphi(x)\leq\infty$ for each $\scriptstyle x\in A.$ Then $$ \sum_{x\in A}\varphi(x) $$ (1) $\alpha_{1},\dots,\alpha_{n}$ denotes the supremum of the set of all finite sums $\varphi(\alpha_{1})+\cdot\cdot\cdot+\varphi(\alpha_{n}),$ where are distinct members of $A.$84 REAL AND COMPLEX ANALYSiIs A moment's consideration will show that the sum (1) is thus precisely the Lebesgue integral of gp relative to the counting measure ${\boldsymbol{\mu}}$ L On $A.$ domain $\scriptstyle A$ is thus in In this context one usually writes $\ell^{p}(A)$ for $L^{p}(\mu).$ $\mathbf{A}$ complex function $\varphi$ with $\ell^{2}(A)$ if and only if $$ \sum_{\alpha\,\in\,A}|\,\varphi(\alpha)|^{2}<\,\infty. $$ (2) Example 4.5(b) shows that $\ell^{2}(A)$ is a Hilbert space, with inner product $$ (\varphi,\,\psi)=\sum_{\alpha\,\in\,A}\,\varphi(\alpha){\overline{{\psi(\alpha)}}}. $$ (3) counting measure; note that py e $\ell^{1}(A)$ stands for the integral of $\varphi{\bar{\psi}}$ with respect to the Here, again, the sum over $\scriptstyle A$ because $\varphi$ p and $\psi$ l are in $\ell^{2}(A).$ is the set of all α where Theorem 3.13 shows that the functions then {α∈ A: $\varphi(x)\neq0\}$ is at most countable. For if ${\boldsymbol{A}}_{n}$ subset of A are dense in $\ell^{2}(A).$ $\varphi$ that are zero except on some finite Moreover, if $\varphi\in\ell^{2}(A),$ then the number of elements of Ais at most $|\varphi(\alpha)|>1/n,$ $$ \sum_{\alpha\,\in\,A_{n}}|\,n\varphi(\alpha)\,|^{2}\,\leq\,n^{2}\,\sum_{\alpha\,\in\,A}\,|\,\varphi(\alpha)\,|^{2}. $$ Each $A_{n}\left(n=1,2,3,\ldots\right)$ is thus a finite set The following lemma about complete metric spaces will make it easy to pass from finite orthonormal sets to infinite ones. 4.16 Lemma Suppose that (a) $X$ and ${\mathbf{}}Y$ are metric spaces, $X_{0}$ on which f is an isometry, and $X$ is complete (の $f\colon X\to Y$ is continuous, (c) $X$ has a dense subset (d) $f(X_{0})$ is dense in ${\boldsymbol{Y}}.$ ${\boldsymbol{Y}}.$ Then f is an isometry of X onto The most important part of the conclusion is that f maps $\textstyle X{\ ~}$ onto all of Y Reall that an isometry is simply a mapping that preserves distances. Thus $x_{1}$ and ${\boldsymbol{x}}_{2}$ , in $X,$ by assumption, the distance between $\scriptstyle J(x_{s})$ and f(xz) in ${\mathbf{}}Y$ is equal to that between for all points $x_{1},x_{2}$ in $X_{0}$ Pick PROOF The fact that $X$ implies now that is dense in $X.$ is an immediate consequence of in $X_{0}$ such $\mathrm{that}f(x_{n}) arrow y$ as $n\to G\to$ Thus is an isometry on $\left\{x_{,n}\right\}_{,0}$ converges to some $\scriptstyle x\in X\,;$ and the con- // isometry on $X_{0}.$ $\boldsymbol{\f}$ $X_{0}$ is dense in Y, there is a sequence $\{x_{n}\}$ Since f is an $X$ the continuity of , since $\scriptstyle y\in Y.$ Since $f(X_{0})$ pleteness of $\{f(x_{n})\}_{$ is a Cauchy sequence in ${\boldsymbol{Y}}.$ it follows tha {, is also a Cauchy sequenc. The com- tinuity of f shows tha $\mathrm{i}\,f(x)=\mathrm{lim}\,f(x_{n})=y.$ELEMENTARY HLBERT SPACE THEORy 85 4.17 Theorem Let {u.:α∈ A} be an orthonormal set in ${\boldsymbol{H}}$ and let ${\mathbf{}}P$ be the space of all finite linear combinations of the vectors $u_{\alpha}$ The inequality $$ \sum_{\alpha\in A}|{\hat{\mathcal{X}}}(\alpha)|^{2}\leq\|x\|^{2} $$ (1) holds then for every $x\in R.$ and $x\to{\hat{x}}$ is a continuous linear mapping of ${\boldsymbol{H}}$ onto $\ell^{2}(A)$ whose restriction to the closure $\bar{\boldsymbol{P}}$ of ${\mathbf{}}P$ is an isometry of $\bar{\boldsymbol{P}}$ P onto $\ell^{2}(A).$ PROOF Since the inequality 4.14(5) holds for every finite set $F\subset A,$ we have (1), the so-called Bessel inequality $\ell^{2}(A).$ Define $\boldsymbol{\f}$ on ${\boldsymbol{H}}$ by $f(x)={\bar{x}}.$ Then(1) shows explicitly that $\boldsymbol{\mathsf{f}}$ maps ${\boldsymbol{H}}$ into The linearity of is obvious. If we apply (1) to $x-y$ we see that $$ \|f(y)-f(x)\|_{2}=\|{\tilde{y}}-{\hat{x}}\|_{2}\leq\|y-x\|. $$ Thus $\boldsymbol{\mathsf{f}}$ is continuous. Theorem $\ell^{2}(A)$ consisting of those functions whose support is a ${\mathbf{}}P$ onto $4.14(a)$ shows that fis an isometry of the dense subspace of finite set $\scriptstyle{F\in{\mathcal{A}}}$ . The theorem follows therefore from Lemma 4.16,applied with complete metric spate $\textstyle H,$ $Y=\ell^{2}(A);$ note that ${\bar{P}},$ being a closed subset of the / $\scriptstyle X=P$ $X_{0}=P,$ is itself complete. The fact that the mapping $x\to{\hat{x}}$ carries ${\boldsymbol{H}}$ onto $\ell^{2}(A)$ is known as the Riesz- Fischer theorem 4.18 Theorem Let $\{u_{x};x\in A\}$ be an orthonormal set in $\textstyle H.$ Each of the follow- ing four conditions on $\{u_{\alpha}\}$ implies the other three: (i) (i) The set ${\mathbf{}}P$ is a maximal orthonormal set in ${\boldsymbol{H}}$ is dense in $\textstyle H.$ $\{u_{\alpha}\}$ (ii)The equality of all finite linear combinations of members of $\left\{u_{\alpha}\right\}$ $$ \sum_{\alpha\in A}|\hat{\hat{x}}(\alpha)|^{2}=\|x\|^{2} $$ holds for every $\textstyle{\epsilon\,\epsilon\,m}$ (iv) The equality $$ \sum_{\alpha\in A}{\hat{\hat{x}}}(\alpha){\overline{{{\hat{y}}(\alpha)}}}=(x,\,y) $$ holds for all $\textstyle{\boldsymbol{x}}\in H$ and $\nu\in H.$ The last formula is known as Parseval's identity. Observe that xand p are in $\ell^{2}(A),$ hence ${\hat{x}}{\hat{y}}$ is in $\ell^{1}(A),$ so that the sum in (iv) is well defined. Of course,(ii is the special case $x=y$ of (iv). Maximal orthonormal sets are often called complete orthonormal sets or orthonormal bases.86 REAL AND COMPLEX ANALYSIs PROOF To say that $\{u_{\alpha}\}$ in such a way that the resulting set is still orthonormal. This ${\boldsymbol{H}}$ can be $\{u_{\alpha}\}$ is maximal means simply that no vector of adjoined to happens precisely when there is no $x\neq0$ bin ${\boldsymbol{H}}$ that is orthogonal to every $u_{\alpha}$ If ${\mathbf{}}P$ We shall prove that $(\mathbf{i}) arrow(\mathbf{\hat{n}}) arrow(\mathbf{\hat{n}}) arrow(\mathbf{\hat{n}}) arrow(\mathbf{\hat{n}}) arrow(\mathbf{\hat{l}}).$ , and the corollary is not dense in $\textstyle H,$ then its closure P is not all of $\textstyle H,$ maximal when to Theorem 4.11 implies that I $P^{\perp}$ contains a nonzero vector. Thus $\{u_{\alpha}\}$ is not $\boldsymbol{\mathit{P}}$ is not dense, and (i) implies Gi) If (i) holds, so does Gii),by Theorem 4.17. The implication $({\mathrm{in}})\to({\mathrm{iv}})$ follows from the easily proved Hilbert space identity (sometimes called the“ polarization identity ” $$ 4(x,\,y)=\|x+y\|^{2}-\|x-y\|^{2}+i\|x+i y\|^{2}-i\|x-i y\|^{2} $$ which expresses the inner product (x, $\scriptstyle X_{,}$ ,y, simply because $\ell^{2}(A)$ is also a Hilbert equally valid with戈,yin place of $y\quad$ in terms of norms and which is space.(See Exercise 19 for other identities of this type.) Note that the sums in (ii) and (iv) are $\|{\hat{x}}\|_{2}^{2}$ and (戏,), respectively. in ${\boldsymbol{H}}$ so that $(u,u_{x})=0$ for all Finally, if $\mathbf{(i)}$ ) is false, there exists $u\neq0$ α ∈ A. If $x=y=u,$ then ( $x,y)=\|u\|^{2}>0$ but ${\tilde{x}}(x)=0$ for all $\propto\in A,$ hence (iv) fails. Thus (iv) implies G), and the proof is complete / 4.19 Isomorphisms Speaking informally,two algebraic systems of the same nature are said to be isomorphic if there is a one-to-one mapping of one onto the other which preserves all relevant properties. For instance, we may ask whether two groups are isomorphic or whether two fields are isomorphic. Two vector spaces are isomorphic if there is a one-to-one linear mapping of one onto the other. The linear mappings are the ones which preserve the relevant concepts in a vector space, namely, addition and scalar multiplication. In the same way, two Hilbert spaces of $H_{1}$ onto $\textstyle H_{2}$ which also preserves inner products $\textstyle H_{1}$ and $H_{2}$ , are isomorphic if there is a one-to-one linear mapping $\mathrm{\A}$ $(\operatorname{A}\!x,\,\operatorname{A}\!y)=(x,\,y)$ specifically, a Hilbert space isomorphism) of $\mathbf{\hat{\Pi}}H_{1}$ 1onto Such a A is an isomorphism (or, more for all $\scriptstyle{\mathcal{X}}$ and $\gamma\in H_{1}.$ 1. Using this terminology, $H_{2}$ Theorems 4.17 and 4.18 yield the following statement: (x,uz), then the mapping $x\to{\hat{x}}$ is a maximal orthonormal set in a Hilbert space $\textstyle H,$ and i ${\mathrm{sta}}={\hat{\mathbf{\alpha}}}$ $I f\{u_{x}\colon\alpha\in A\}$ is a Hilbert space isomorphism of ${\boldsymbol{H}}$ H onto $\ell^{2}(A).$ One can prove (we shall omit this) that $\ell^{2}(A)$ and $\ell^{2}(B)$ are isomorphic if and only if the sets ${\cal A}$ and $\boldsymbol{B}$ have the same cardinal number. But we shall 'prove that every nontrivial Hilbert space (this means that the space does not consist of O alone) is isomorphic to some $\ell^{2}(A),$ by proving that every such space contains a maximal orthonormal set (Theorem 4.22). The proof will depend on a property of partially ordered sets which is equivalent to the axiom of choice 4.20 Partially Ordered Sets A set $\mathcal{P}$ P is said to be partially ordered by a binary relation ≤ ifELEMENTARY HILBERT SPACE THEORY 87 (a) $a\leq b$ and ${\mathfrak{s}}\leq{\mathfrak{c}}$ implies $a\leq c.$ b a ≤a for every αe io implies $a=b.$ (c)a ≤b and $b\leq a$ A subset of a partially ordered set $\mathcal{P}$ is said to be totally ordered (or lin- early ordered if every pair a, $b\in{\mathcal{B}}$ satisfies either $\alpha\leq b$ or $b\leq a.$ For example, every collection of subsets of a given set is partially ordered by the inclusion relation c To give a more specific example, let E $\mathcal{P}$ be the collection of all open subsets of the plane, partially ordered by set inclusion, and let Q be the collection of al open circular discs with center at the origin. Then ${\mathcal{A}}\subset{\mathcal{P}},{\mathcal{R}}$ is totally ordered by C,and $\scriptstyle{\mathcal{D}}$ is a maximal totally ordered subset of ${\mathcal{P}}.$ This means that if any member of ${\mathcal{P}}$ not in $\textstyle{\mathcal{A}}$ is adjoined to 2,the resulting collection of sets is no longer totally ordered by C 4.21 The Hausdorff Maximality Theorem Every nonempty partially ordered set contains a maximal totally ordered subset. This is a consequence of the axiom of choice and is, in fact, equivalent to it another (very similar) form of it is known as Zorn's lemma. We give the proof in the Appendix. If now ${\boldsymbol{H}}$ is a nontrivial Hilbert space, then there exists a ${\underset{}{{\mathrm{te}}}}\,m$ with $|w|=1,$ so that there is a nonempty orthonormal set in $H.$ The existence of a maximal orthonormal set is therefore a consequence of the following theorem: 4.22 Theorem Every orthonormal set $\boldsymbol{B}$ in a Hilbert space ${\boldsymbol{H}}$ is contained in a maximal orthonormal set in $\textstyle H.$ PROOF Let $\textstyle{\mathcal{P}}$ be the class of all orthonormal sets in ${\boldsymbol{H}}$ which contain the $\mathcal{P}$ given set ${\boldsymbol{B}}.$ Partially order $\textstyle{\mathcal{P}}$ by set inclusion. Since $\boldsymbol{B}$ e 9, ${\mathcal{P}}\neq{\mathcal{Q}}$ Hence contains a maximal totally ordered class Q. Let $\boldsymbol{\mathsf{S}}$ be the union of all members of Q. It is clear that $\scriptstyle{B\in{S}}$ We claim that $\boldsymbol{\mathsf{S}}$ is a maximal orthonor- $\boldsymbol{\mathsf{S}}$ mal set: and $u_{x}\in S_{x}$ then $u_{1}\in A_{1}$ and $u_{x}\in A_{x}$ for some $A_{1}$ and $A_{2}\in\Omega.$ Since ifw $\stackrel{u_{1}}{\sim}$ is total ordered, $4_{1}\subset A_{2}$ (or $A_{2}\subset A_{1}),$ so that $u_{1}\in A_{2}$ if $\scriptstyle u_{1}=u_{2}$ .Thus Since $A_{2}$ is orthonormal, $(u_{1},u_{2})=0$ if and $u_{2}\in A_{2}$ $\Omega$ is an orthonormal set. $u_{1}\neq u_{2},(u_{1},u_{2})=1$ set adjoin $S^{\bullet}$ Clearly, $S^{*}\notin\Omega,$ is not maximal. Then $S^{\bullet}$ contains every member of Q. Hence we may Suppose $\boldsymbol{\mathsf{S}}$ $\boldsymbol{\mathsf{S}}$ is a proper subset of an orthonormal $S^{\mathbb{N}}.$ to $\Omega$ and and still have a total order. This contradicts the maximality of $\Omega.$ //88 REAL AND cOMPLEX ANALYSis Trigonometric Series 4.23 Definitions Let ${\mathbf{}}T$ be the unit circle in the complex plane, i.e., the set of all complex numbers of absolute value 1. If ${\mathbf{}}F$ is any function on ${\mathbf{}}T$ and iff is defined on ${\boldsymbol{R}}^{1}$ by $$ f(t)=F(e^{t t}), $$ (1) all real t. Conversely, if then fis a periodic function of period 2n. This means $R^{1},$ with period 2z, then there is a for $\boldsymbol{\f}$ is a function on $\mathrm{that}\,f(t+2\pi)=f(t)$ function ${\mathbf{}}F$ on ${\mathbf{}}T$ such that ((1) holds. Thus we may identify functions on ${\mathbf{}}T$ with 2z-periodic functions on $R^{1}.$ }; and, for simplicity of notation, we shall sometimes write f(t) rather than $f(e^{i t}),$ even if we think of f as being defined on ${\mathbf{}}T$ With these conventions in mind, we define $L^{p}(T),$ for $1\leq p<\infty,$ to be the class of all complex, Lebesgue measurable,、2z-periodic functions on $R^{1}$ for which the norm $$ \|f\|_{p}=\not=\underbrace{1}_{2\pi}\ {\LARGE\int_{-\pi}^{\pi}}_{n}\left|f(t)\right|^{p}\,d t \rangle^{1/p} $$ (2) is finite. members of $L^{\infty}(R^{1}),$ In other words, we are looking at $P(n)_{i}$ where ${\boldsymbol{\mu}}$ is Lebesgue measure on consists of [0,2元](or on T) divided by 2元. $L^{\infty}(T)$ will be the class of all z-periodic with the essential supremum norm, and ${\mathfrak{c}}(n)$ all continuous complex functions on ${\mathbf{}}T$ (or, equivalently, of all continuous complex,2元-periodic functions on $R^{1}$ ), with norm $$ \|f\|_{\infty}=\sup_{t}|f(t)|, $$ (3) The factor $1/(2\pi)$ in (2) simplifes the formalism we are about to develop For instance, the ${\boldsymbol{D}}^{\prime}$ P-norm of the constant function 1is 1 A trigonometric polynomial is a finite sum of the form $$ f(t)=a_{0}+\sum_{n=1}^{N}\,(a_{n}\cos\,n t+b_{n}\sin\,n t)\qquad(t\in R^{1}) $$ (4) where $a_{0}\,,\,a_{1},\,\dots,\,a_{N}$ and $b_{1},\ldots,b_{N}$ are complex numbers. On account of the Euler identities, (4 can also be written in the form $$ f(t)=\sum_{n=-N}^{N}c_{n}e^{i n t} $$ (5) which is more convenient for most purposes. It is clear that every trigono- metric polynomial has period 2元.ELEMENTARY HILBERT SPACE THEORY 89 We shall denote the set of all integers (positive, zero, and negative) by Z, and put $$ u_{n}(t)=e^{i n t}\quad\quad(n\in\mathbb{Z}). $$ (6) If we define the inner product in $L^{2}(T)$ by $$ (f,\,g)={\frac{1}{2\pi}}\,\biggr)_{-\pi}^{\pi}\,f(t){\overline{{g(t)}}}\,d t $$ (7) [note that this is in agreement with (2)], an easy computation shows that $$ (u_{n},\;u_{m})=\frac{1}{2\pi}\left\{\O_{-\pi}^{\pi}e^{i(n-m)t}\;d t= \{0\right.\mathrm{~\if~}n=m, $$ (8) Thus $\left\{u_{n}\colon n\in\mathbb{Z}\right\}$ is an orthonormal set in $L^{2}(T),$ usually called the trigono- metric system. We shall now prove that this system is maximal, and shall then derive concrete versions of the abstract theorems previously obtained in the Hilbert space context. 4.24 The Completeness of the Trigonometric System Theorem 4.18 shows that the maximality (or completeness) of the trigonometric system will be proved as soon Since $\operatorname{c}(n)$ is dense in as we can show that the set of all trigonometric polynomials is dense in $L^{2}(T).$ nomial ${\mathbf{}}P$ to show that to every $L^{2}(T),$ by Theorem 3.14 (note that $\scriptstyle x\;{\sim}\;0$ there is a trigonometric poly- the esti- ${\mathbf{}}T$ is compact), it suffices mate such that $f\in C(T)$ and to every for every $g\in C(T),$ $\|f-P\|_{2}<\epsilon$ $\|f-P\|_{2}<\epsilon.$ Since $\|g\|_{2}\leq\|g\|_{x^{n}}$ and it is this estimate which will follow from $\|f-P\|_{\infty}<\epsilon,$ we shall prove. Suppose we had trigonometric polynomials $Q_{1},$ $Q_{2},\,Q_{3},\,\ldots,$ with the follow- ing properties: (b) $$ {\frac{Q_{k}(t)\geq0~{\mathrm{for~}}t\in R^{1}.}{2\pi}} $$ (a) (c) $I f\eta_{k}(\delta)=\operatorname*{sup}\;\{Q_{k}(t)\colon\delta\leq|t|\leq\pi\},$ then $$ \operatorname*{lim}_{k arrow\infty}\eta_{k}(\delta)=0 $$ for every $\scriptstyle{6>0}$ Another way of stating (c) is to say: for every $\delta>0$ $Q_{k}(t)\to0$ uniformly on $[-\pi,-\delta]\cup[\delta,\pi].$ we associate the functions $\mathbf{}P_{k}$ defined by To each f ∈ $\operatorname{c}(n)$ $$ P_{k}(t)=\frac{1}{2\pi}\left.\right\}_{-\pi}^{\pi}f(t-s)Q_{k}(s)\;d s\qquad(k=1,\,2,\,3,\,...). $$ (1)90 REAL AND coMPLEX ANALYSIs If we replace s by -s (using Theorem 2.20(e)) and then by $s-t,$ the periodicity of $\boldsymbol{\mathit{f}}$ and $Q_{k}$ shows that the value of the integral is not affected. Hence $$ P_{k}(t)=\frac{1}{2\pi}\left\{_{-\pi}^{\pi}f(s)Q_{k}(t-s)\;d s\qquad(k=1,\,2,\,3,\,...).\qquad\qquad(k=1,\,2,\,3,\,...\right) $$ (2) Since each $Q_{k}$ is a trigonometric polynomial, $Q_{k}$ is of the form $$ Q_{k}(t)=\sum_{n=-N_{k}}^{N_{k}}a_{n,k}\,e^{i n t}, $$ (3) and if we replace t byt-s in (3) and substitute the result in (2) we see that each $P_{k}$ Let is a trigonometric polynomial $|t-s|<\delta|<\delta|$ By (b), we have ${\boldsymbol{T}},$ there exists a $\delta>0$ $\scriptstyle\epsilon\;>0$ be given. Since f is uniformly continuous on such that $|f(t)-f(s)|<\epsilon$ whenever $$ P_{k}(t)-f(t)={\frac{1}{2\pi}}\left|\O_{-\pi}^{\pi}\{f(t-s)-f(t)\}Q_{k}(s)\;d s,\nonumber\right. $$ and (a) implies, for all ${\mathfrak{t}},$ that $$ \left|P_{k}(t)-f(t)\right|\leq{\frac{1}{2\pi}}\int_{-\pi}^{\pi}\left|f(t-s)-f(t)\right|Q_{k}(s)~d s=A_{1}+A_{2}\,, $$ where $A_{1}$ is the integral over $[-\delta,\;\delta]$ and $A_{2}$ is the integral over [-元,-6] u [6,元]. In $A_{1},$ the integrand is less than $\scriptstyle A_{0}(s)$ so $A_{1}<\epsilon,\;\mathrm{by}\;(b).\;\mathrm{In}\;A_{2}\,,$ we have $Q_{k}(s)\leq\eta_{k}(\delta)$ , hence $$ A_{2}\leq2\|f\|_{\alpha}\cdot\eta_{k}(\delta)<\epsilon $$ (4) for sufficiently large $k,$ by (c). Since these estimates are independent of ${\hat{L}}_{3}$ we have proved that $$ \operatorname*{lim}_{k arrow\infty}\left\|f-P_{k}\right\|_{\infty}=0. $$ (5) simple one. Put It remains to construct the $Q_{k}$ Q,. This can be done in many ways. Here is a $$ Q_{k}(t)=c_{k}\left\{{\frac{1+\cos\,t}{2}}\right\}^{k}, $$ (6) where c, is chosen so that (b) holds. Since (a) is clear, we only need to show (c) Since $Q_{k}$ is even,(b) shows that $$ 1=\frac{c_{k}}{\pi}\left\{\right\}_{0}^{\pi}\left\{\frac{1+\cos\;t}{2}\right\}^{k}\!\!d t>\frac{c_{k}}{\pi}\int_{0}^{\pi}\sqrt{\frac{1+\cos\;t}{2}} \}^{k}\sin\;t\;d t=\frac{2c_{k}}{\pi(k+1)}. $$ Since $Q_{k}$ is decreasing on [O, 元], it follows that $$ Q_{k}(t)\leq Q_{k}(\delta)\leq{\frac{\pi(k+1)}{2}}\left({\frac{1+\cos\delta}{2}}\right)^{k}\qquad(0<\delta\leq\vert\,t\,\vert\leq\pi). $$ (7)ELEMENTARY HILBERT SPACE THEORY 91 This implies (c), since 1 + cos $\delta<2\,\mathrm{if}\,0<\delta\leq\pi.$ We have proved the following important result: 4.25 Theorem If fe C(T) and $\mathbf{e\!>0}.$ there is a trigonometric polynomial P such that $$ |f(t)-P(t)|<\epsilon $$ for every real t. A more precise result was proved by Fejer (1904):The arithmetic means of the partial sums of the Fourier series of any $f\in C(T)$ converge uniformly to f For a proof (quite similar to the above) see Theorem 3.1 of [45], or p. 89 of [36], vol. I. 4.26 Fourier Series For any fe $\scriptstyle{I(T)}$ we define the Fourier coefficients of f by the formula $$ \hat{f}(n)=\frac{1}{2\pi}\left|\right.^{\pi}f(t)e^{-i n t}~d t\qquad(n\in Z), $$ (1) where, we recall, $\underline{{Z}}$ is the set of all integers. We thus associate with eachfe $\scriptstyle{D(T)}$ a function fon Z. The Fourier series of f is $$ \sum_{-\infty}^{\infty}\ {\hat{f}}(n)e^{i n t}, $$ (2)) and its partial sums are $$ s_{N}(t)=\sum_{-N}^{N}\hat{f}(n)e^{i n t}\qquad(N=0,\,1,\,2,\,\ldots). $$ (3) Since $L^{2}(T)\subset L^{1}(T),(1)$ can be applied to every f∈ $\displaystyle\colon L^{2}(T).$ Comparing the defi- nitions made in Secs.4.23 and 4.13,we can now restate Theorems 4.17 and 4.18 in concrete terms: The Riesz-Fischer theorem asserts that if {c,}is a sequence of complex numbers such that $$ \sum_{n=-\infty}^{\infty}|c_{n}|^{2}<\infty, $$ (4) then there exists $\mathrm{an}f\in L^{2}(T)$ such that $$ c_{n}={\frac{1}{2\pi}}\int_{-\pi}^{\pi}f(t)e^{-i n t}\,d t\qquad(n\in Z). $$ (5) The Parseval theorem asserts that $$ \sum_{n=-\infty}^{\infty}\hat{f}(n)\overline{{{\hat{g}(n)}}}=\frac{1}{2\pi}\,\int_{-\pi}^{\pi}f(t)\overline{{{g(t)}}}\,d t $$ (6)92 REAL AND COMPLEX ANALYSIS whenever $f\in L^{n}(T)$ and $g\in L^{2}(T);$ the series on the left of (6) converges absolu- tely; and if ${\boldsymbol{S}}_{N}$ is as in $({\mathbf{3}}),$ then $$ \operatorname*{lim}_{N\to\infty}\left\|S-s_{N}\right\|_{2}=0, $$ (7) since a special case of (6) yields $$ \|f-s_{N}\|_{2}^{2}=\sum_{|n|>_{N}}\|{\hat{f}}(n)\|^{2}. $$ (8) Note that (T) says that every $f\in L^{2}(T)$ is the $L^{2}.$ -limit of the partial sums of its Fourier series; i.e., the Fourier series of f converges to ${\boldsymbol{f}},$ in the $L^{2}.$ -sense. Pointwise convergence presents a more delicate problem, as we shall see in Chap. 5. The Riesz-Fischer theorem and the Parseval theorem may be summarized by saying that the mapping f→fis a Hilbert space isomorphism of $L^{2}(T)$ onto $\ell^{2}(Z)$ The theory of Fourier series in other function spaces, for instance in $\scriptstyle{I(\mathbf{r}_{1})}$ is much more difficult than in $L^{2}(T),$ and we shall touch only a few aspects of it Observe that the crucial ingredient in the proof of the Riesz-Fischer theorem is the fact that ${\boldsymbol{L}}^{2}$ is complete. This is so well recognized that the name“Riesz- Fischer theorem $9\Re$ is sometimes given to the theorem which asserts the complete- ness of $L^{2},$ or even of any ${\boldsymbol{D}},$ Exercises In this set of exercises, ${\boldsymbol{H}}$ always denotes a Hilbert space 1 If ${\cal{M}}$ is a closed subspace of ${\boldsymbol{H}},$ prove that $M=(M^{\bot})^{\bot}$ . Is there a similar true statement for sub- spaces ${\cal{M}}$ which are not ncessarily closed ? span for al $N.$ $\operatorname{Put}u_{1}=x_{1}/\|x_{1}\|.$ Having $u_{1},\cdot\cdot\cdot,\;u_{n-1}$ be a linearly independent set of vectors in ${\boldsymbol{H}}.$ Show that thefollowing 2 Let $\{x_{n}\colon n=1,$ 2, $\mathbf{3},\ldots\}$ $\left\{u_{n}\right\}$ such that $\{x_{1},\ \ldots,x_{N}\}$ and $\{u_{1},\ \ldots,u_{N}\}$ have the same construction yields an orthonormal set define $$ v_{n}=x_{n}-\sum_{i=1}^{n-1}\,(x_{n},\,u_{i})u_{i},\qquad u_{n}=v_{n}/\|v_{n}\|. $$ Note that this leads to a proof of the existence of a maximal orthonormal set in separable it contains a countable dense subset.) Hilbert spaces which makes no appeal to the Hausdorff maximality principle.(A space is separable if 3 Show that $D(T)$ is separable if $|\leq p<\infty.$ but that $L^{\infty}(T)$ is not separable 4 Show that $\boldsymbol{H}$ is separable if and only if ${\boldsymbol{H}}$ contains a maximal orthonormal system which is at most countable S If $M=\{x\colon L x=0\}$ , where ${\cal L}$ is a continuous linear functional on ${\boldsymbol{H}},$ prove that $M^{\bot}$ is a vector space of dimension 1(unless $M=H).$ 6 Let $\left\{ .M_{n}\right\}$ $(n=1,$ $\downarrow_{*,{\bf\Delta}}\Lambda_{\mu_{\nu}{\bf\Delta}}\ast{\bf\cdot}\ast\ast\ast\Big)$ be an orthonormal set in ${\boldsymbol{H}}.$ Show that this gives an example of a closed and bounded set which is not compact. Let $\scriptstyle{Q}$ 2 be the set of all $x\in H$ of the form $$ x=\sum_{1}^{\infty}c_{n}u_{n}\qquad\left({\mathrm{where~}}|c_{n}|\leq{\frac{1}{n}}\right). $$ Prove that $\scriptstyle{\mathcal{Q}}$ is compact. ${\boldsymbol{Q}}$ is called the Hilbert cube.)ELEMENTARY HLBERT SPACE THEORy 93 More generally, let {6.} be a sequence of positive numbers, and let S be the set of al $x\in H$ of the form $$ x=\sum_{1}^{\infty}c_{n}u_{n}\qquad({\mathrm{where~}}|\,c_{n}|\leq\delta_{n}). $$ Prove that $\boldsymbol{\mathsf{S}}$ is compact if and only if $\textstyle{\sum_{1}^{\infty}\delta_{n}^{2}<\infty.}$ Prove that ${\boldsymbol{H}}$ is not locally compact. E 7 Suppose $\left\{a_{n}\right\}$ is a sequence of positive numbers such that $\sum a_{n}b_{n}<\infty$ whenever $b_{n}\geq0$ and $b_{n}^{2}<\infty.$ Prove that E $a_{n}^{2}<\infty.$ so tha Suggestion: IfEa. = o then there are disjoint sets $E_{k}\left(k=1,2,3,\ldots\right)$ $$ \sum_{n\,e\,\in\,E_{n}}a_{n}^{2}>1. $$ Define b ${\mathfrak{b}}_{n}$ D。so that $b_{n}=c_{k}a_{n}$ for n ∈ ${\underline{{\mu}}}_{k}$ For suitably chosen $c_{k},\boldsymbol{\Sigma}$ a,b,= 0o although $\sum b_{n}^{2}<\infty.$ 8 If $\textstyle H_{1}$ and $\textstyle H_{2}$ are two Hilbert spaces, prove that one of them is isomorphic to $\underline{{\land}}$ subspace of the other. (Note that every closed subspace of a Hilbert space is a Hilbert space.) 9 If $4\subset[0,2\pi]$ and $\scriptstyle A\quad\quad A$ is measurable, prove that $$ \operatorname*{lim}_{n\to\infty}\left[\operatorname{cos}\,n x\,d x=\operatorname*{lim}_{n\to\infty}\, (\operatorname*{sin}_{n\to\infty}\,\right)_{A}^{n}\sin\,n x\,d x=0. $$ 10 Let $n_{1}<n_{2}<n_{3}<\cdots$ be positive integers, and let $\bar{E}$ be the set of all so sin $n_{i}x\to\pm1/{\sqrt{2}}$ a.e. on ${\boldsymbol{E}},$ {sin $|P_{k}\ X\rangle$ converges. Prove that $m(E)=0.$ Hint: 2 sin- $x\in[0,$ 2元]at which $\alpha=1-\cos2\alpha,$ by Exercise 9. 11 Find a nonempty closed set $\bar{E}$ in $L^{2}(T)$ that contains no element of sallest norm ${\boldsymbol{1}}{\boldsymbol{2}}$ The constants $c_{k}$ in Sec.4.24 were shown to be such that $k^{-1}c_{k}$ is bounded. Estimate the relevant integral more precisely and show that $$ 0<\operatorname*{lim}_{k arrow\infty}k^{-1/2}c_{k}<\infty. $$ 13 Suppose fis a continuous function on $R^{1},$ with period 1. Prove that $$ \operatorname*{lim}_{N\to\infty}{\frac{1}{N}}\sum_{n=1}^{N}\ f(n\alpha)=\int_{0}^{1}f(t)\ d t $$ for every irrational real number α. Hint: Doit first for $$ f(t)=\exp\,(2\pi i k t),\qquad k=0,\,\pm1,\,\pm2,\,\ldots $$ 14 Compute $$ \operatorname*{min}_{a,b,c}\prod_{i=1}^{1}|x^{3}-a-b x-c x^{2}|^{2}\ d x $$ and find $$ \mathrm{max}\,\prod_{-1}^{1}x^{3}g(x)\,\,d x, $$ $\scriptstyle{\mathcal{G}}$ where g is subject to the restrictions $$ \bigcap_{-1}^{1}g(x)\;d x=\left\{_{-1}^{1}x g(x)\;d x= \{_{-1}^{1}x^{2}g(x)\;d x=0;\qquad .\right|_{-1}^{1}|g(x)|^{2}\;d x=1. $$94 REAL AND cMPLEX ANALYSIs 15 Compute $$ \operatorname*{min}_{a,b,\epsilon}\prod_{0}^{a_{0}}|x^{3}-a-b x-c x^{2}|^{2}e^{-x}\,d x. $$ State and solve thecorresponding maximum problem,as in Exercise 4 16 $|\mathbf{r}_{x}\circ m$ and $M{\mathrm{~is~a}}$ closed linear subspace of $\textstyle H,$ prove that m $$ \mathrm{in}\ \{\|x-x_{0}\|\colon x\in M\}=\operatorname*{max}\left\{|(x_{0},y)|:y\in M^{\perp},\;\|y\|=1\right\}. $$ ${\mathfrak{I}}{\mathfrak{I}}$ orthogonal to $\gamma(d)-\gamma(c)$ Show that there is a continuous one-to-one mapping $\gamma$ of [O,1] into $\boldsymbol{H}$ such that $\gamma(b)-\gamma(a)$ is whenever $0\leq a\leq b\leq c\leq d\leq1.(\gamma$ may be called a“curve with orthogonal increments.") Hint: Take $H=L^{2},$ and consider characteristic functions of certain subsets of [0, 1] 18 Define linear combinatons of these functions $u_{s},\operatorname{If}f\in X$ and $g\in X,$ show that be the complex vector space consisting of all finite $u_{s}(t)=e^{i s t}$ for all $s\in R^{1},t\in R^{1}$ Let $\scriptstyle{\mathcal{X}}$ $$ (f,g)=\operatorname*{lim}_{A arrow\infty}{\frac{1}{2A}}\int_{-A}^{A}f(t){\overline{{g(t)}}}\,d t $$ exists. Show that this inner product makes $\scriptstyle{\mathcal{X}}$ into a unitary space whose completion is a non- separable Hilbert space ${\boldsymbol{H}}.$ Show also that {u,:s e $R^{1}\}$ is a maximal orthonormal set in $H.$ 19 Fix a positive integer ${\boldsymbol{N}},$ put o $\textstyle\langle=e^{2\pi i/N},$ prove the orthogonality relations $$ {\frac{1}{N}}\sum_{n=1}^{N}\,\omega^{n k}={\binom{1}{0}}\ {\mathrm{~if~}}\quad k=0 $$ and use them to derive the identities $$ (x,y)={\frac{1}{N}}\sum_{n=1}^{N}\Vert x+\omega^{n}y\Vert^{2}\omega^{n} $$ that hold in every inner product space if $N\geq3.$ Show also that $$ (x,y)=\frac{1}{2\pi}\prod_{-\pi}^{\pi}\|x+e^{i\theta}y\|^{2}e^{i\theta}\;d\theta. $$