CHAPTER TWO POSITIVE BOREL MEASURES Vector Spaces 2.1 Definition A complex vector space (or a vector space over the complex field is a set ${\boldsymbol{V}},$ whose elements are called vectors and in which two oper- ations, called addition and scalar multiplication, are defined, with the follow- ing familiar algebraic properties: To every pair of vectors $\scriptstyle{\mathcal{X}}$ and $\mathbf{\vec{y}}$ there corresponds a vector $x+y,$ in $x\in V;$ such a way that $x+y=y+x$ and $x+(y+z)=(x+y)+z;$ ${\mathbf{}}V$ contains a and to each unique vector O (the zero vector or origin of ${\cal{V}}\}$ such that $x+0=x$ for every $x\in V$ there corresponds a unique vector $-\,x$ such that $x+(-x)=0.$ $\scriptstyle(n,\,x).$ where x∈ ${\mathbf{}}V$ and α is a scalar (in this context,the To each pair word scalar means complex number),there is associated a vector $x x\in V.$ in such a way that $1x=x,\;\alpha(\beta x)=(\alpha\beta)x,$ and such that the two distributive laws $$ a(x+y)=\alpha x+\alpha y,(\alpha+\beta)x=\alpha x+\beta x $$ (1) hold. A linear transformation of a vector space ${\mathbf{}}V$ into a vector space $V_{\mathrm{i}}$ is a mapping A of ${\mathbf{}}V$ into $V_{\mathrm{i}}$ such that $$ \Lambda(\alpha x+\beta y)=\alpha\Lambda x+\beta\Lambda y $$ (2) for all $\scriptstyle{\mathcal{X}}$ and $y\in V$ and for all scalars $\scriptstyle{\dot{\alpha}}$ and ${\boldsymbol{\beta}}.$ In the special case in which $V_{\mathrm{i}}$ is the field of scalars this is the simplest example of a vector space, except for the trivial one consisting of $\mathbf{0}$ alone), $\mathrm{\A}$ is called a linear functional. A linear functional is thus a complex function on ${\mathbf{}}V$ which satisfies (2) Note that one often writes $\Lambda x,$ rather than A(x) if $\Lambda$ is linear. 3334 REAL AND cOMPLEX ANALYSIs The preceding definitions can of course be made equally well with any field whatsoever in place of the complex field. Unless the contrary is xplicitly Stated. however, all vector spaces occurring in this book will be complex, with one notable exception: the euclidean spaces ${\boldsymbol{R}}^{k}$ are vector spaces over the real field. 2.2 Integration as a Linear Functional Analysis is full of vector spaces and linear transformations, and there is an especially close relationship between integration on the one hand and linear functionals on the other. For instance,Theorem 1.32 shows that $\scriptstyle T(\mu)$ is a vector space, for any posi- tive measure ${\boldsymbol{\mu}},$ and that the mapping $$ f arrow\int_{X}f\,d\mu $$ (1) is a linear functional on $L^{1}(\mu),$ Similarly, if $\scriptstyle{\mathcal{G}}$ is any bounded measurable function the mapping $$ f{ arrow}\int_{X}f g\ d\mu $$ (2) is a linear functional on $\scriptstyle{U(p)}$ ; we shall see in Chap. 6 that the functionals (2) are in a sense, the only interesting ones on $L^{1}(\mu).$ For another example, let ${\boldsymbol{C}}$ be the set of all continuous complex functions on the unit interval 1 = [0,1]. The sum of the two continuous functions is contin- uous, and so is any scalar multiple of a continuous function. Hence ${\boldsymbol{C}}$ is a vector space, and if $$ \Lambda f= \vert_{0}^{+1}f(x)\;d x\qquad(f\in C), $$ (3) the integral being the ordinary Riemann integral, then $\mathrm{\A}$ is clearly a linear func- tional on ${\boldsymbol{C}}\,;$ A has an additional interesting property: it is a positive linear func- tional. This means that $\scriptstyle N\geq0$ whenever $f\geq0.$ One of the tasks which is still ahead of us is the construction of the Lebesgue measure. The construction can be based on the linear functional (3)by the fol- lowing observation: Consider a segment $(a,b)\subset I$ and consider the class of all $f\in C$ such that $0\leq f\leq1$ on ${\mathbf I}$ and $f(x)=0$ for all $\scriptstyle{\mathcal{X}}$ not in (a, b). We have as $\Lambda f<b-a$ for all such ${\mathfrak{f}},$ but we can choose $\boldsymbol{\mathsf{f}}$ so that $\Lambda f$ f is as close to $b-a$ desired. Thus the length (or measure) of (α, b) is intimately related to the values of the functional A. The preceding observation, when looked at from a more general point of view, leads to a remarkable and extremely important theorem of F. Riesz To every positive linear functional $\mathrm{\A}$ A on ${\mathbf{C}}$ corresponds a finite positive Borel measure $\boldsymbol{\mu}$ L on ${\mathbf I}$ such that $$ \Lambda f=\int_{I}f\,d\mu\qquad(f\in C). $$ (4)POSITTVE BOREL MEASURES 35 [The converse is obvious: if $\boldsymbol{\mu}$ is a finite positive Borel measure on ${\mathbf I}$ and if $\Lambda$ is defined by (4), then $\mathrm{\A}$ is a positive linear functional on C.] It is clearly of interest to replace the bounded interval ${\mathbf I}$ T by $R^{1}$ We can do this by restricting attention to those continuous functions on ${\boldsymbol{R}}^{1}$ which vanish outside some bounded interval.(These functions are Riemann integrable,_for instance.) Next, functions of several variables occur frequently in analysis. Thus we ought to move from $R^{1}$ l to ${\boldsymbol{R}}^{n}.$ . It turns out that the proof of the Riesz theorem still goes through,with hardly any changes. Moreover, it turns out that the euclidean properties of ${\boldsymbol{R}}^{n}$ (coordinates, orthogonality, etc.) play no role in the proof; in fact, if one thinks of them too much they just get in the way. Essential to the proof are certain topological properties of $\textstyle R^{n}.$ .(Naturally. We are now dealing with continuous functions.)The crucial property is that of local com- pactness: Each point of ${\boldsymbol{R}}^{n}$ has a neighborhood whose closure is compact. We shall therefore establish the Riesz theorem in a very general setting (Theorem 2.14). The existence of Lebesgue measure follows then as a special case. Those who wish to concentrate on a more concrete situation may skip lightly over the following section on topological preliminaries (Urysohn's lemma is the item of greatest interest there; see Exercise 3) and may replace locally compact Hausdorff spaces by locally compact metric spaces, or even by euclidean spaces, without missing any of the principal ideas It should also be mentioned that there are situations, especially in probability theory, where measures occur naturally on spaces without topology, or on topo- logical spaces that are not locally compact. An example is the so-called Wiener measure which assigns numbers to certain sets of continuous functions and which is a basic tool in the study of Brownian motion. These topics will not be dis- cussed in this book. Topological Preliminaries 2.3 Definitions Let $X$ be a topological space, as defined in Sec. 1.2 (a)A set $\scriptstyle{E\in X}$ is closed if its complement $E^{\mathrm{c}}$ is open. (Hence Z and $\textstyle X$ are closed, finite unions of closed sets are closed, and arbitrary intersections of closed sets are closed.) (b)The closure $\overline{{E}}$ of a set $\scriptstyle{E\in{\mathcal{X}}}$ is the smallest closed set in $X$ which con tains $E.$ (The following argument proves the existence of ${\overrightarrow{E}}{\mathrm{:}}$ The collec- tion $\Omega$ of all closed subsets of $X$ which contain $\boldsymbol{E}$ is not empty, since (c) A set $X\in\Omega\colon\mathbb{R}\in\mathbb{Z}$ be the intersection of all members of Q2.) contains a finite sub $\kappa\in X$ is compact if every open cover of $\textstyle K$ cover. More explicitly, the requirement is that if{V} is a collection of open sets whose union contains $K_{\mathrm{{,}}}$ then the union of some finite sub collection of $\{V_{\alpha}\}$ also contains $K.$ is called a compact space. In particular, if $X$ K is itself compact, then $\textstyle X{\ ~}$ (d)A neighborhood of a point ${\mathrm{p}}\in X$ is any open subset of $X$ X which contains pP.(The use of this term is not quite standardized; some use36 REAL AND coMPLEX ANALYSIs “ neighborhood of p”for any set which contains an open set containing p.) (e) $X$ is a Hausdorff space if the following is true: If and $\boldsymbol{\mathit{q}}$ has a neighborhood ${\mathbf{}}V$ such that then p has a neighborhood ${\mathrm{pe}}\,X_{*}$ ${\mathrm{;~}}x,$ and $p\neq q,$ $U$ $U\,\cap\,V=\mathcal{D}.$ (f) $\textstyle X{\ ~}$ is locally compact if every point of $X$ has a neighborhood whose closure is compact Obviously, every compact space is locally compact. We recall the Heine-Borel theorem: The compact subsets of a euclidean space ${\boldsymbol{R}}^{n}$ are precisely those that are closed and bounded ([26],t Theorem 2.41). From this it follows easily that ${\boldsymbol{R}}^{n}$ is a locally compact Hausdorff space.Also, every metric space is a Hausdorff space. 2.4 Theorem Suppose ${\cal K}$ is compact and ${\mathbf{}}F$ is closed, in a topological space $X.$ $I F\subset K,$ then ${\mathbf{}}F$ is compact. PROOF If $\scriptstyle V_{\mathrm{sl}}$ is an open cover of ${\mathbf{}}F$ F and $W=F^{\circ}.$ then $W\cup\emptyset\setminus\emptyset_{\alpha}$ V covers X; hence there is a finite collection $\scriptstyle V_{x}|$ such that $$ K\subset W\cup V_{\alpha_{1}}\cup\ *\cdot\cup V_{\alpha_{n}}. $$ Then $F\subset V_{\alpha_{1}}\cup\mathbf{\alpha}\cdot\mathbf{\nabla}\cup V_{\alpha_{n}}$ // Corollary If $\cdot A\subset B$ and jf $\boldsymbol{B}$ has compact closure, so does $\scriptstyle A$ 2.5 Theorem Suppose $X$ is a Hausdorff space, $\kappa\in X;$ $\textstyle K$ is compact, and $p\in K^{c}.$ T hen there are open sets $U$ and W such tha $p\in U,\;K\subset W,$ and $U\,\cap\,W={\mathcal{D}}.$ PRoor If $q\in K,$ the Hausdorff separation axiom implies the existence of dis- $q\in V_{q}.$ Since ${\boldsymbol{K}}$ is compact, joint open sets $U_{q}$ and $V_{q}\,,$ such that $p\in U_{q}$ and there are points $q_{1},\dots,q_{n}\in K$ such that $$ K\subset V_{q_{1}}\cup\cdot\cdot\cdot\cup\ V_{q_{n}}. $$ Our requirements are then satisfied by the sets $$ U=U_{q_{1}}\;\cap\cdot\cdot\cdot\cdot\top U_{q_{n}}\;\;\;\mathrm{and}\;\;\;\;W=V_{q_{1}}\;\cup\;\cdot\cdot\cdot\cup\;V_{q_{n}}. $$ / Corollaries (a) Compact subsets of Hausdorff spaces are closed. (b)If F is closed and ${\cal K}\,\,$ is compact in a Hausdorff space,then $F\,\cap\,K$ is compact. Corollary $\mathbf{(}b)$ follows from (a and Theorem 2.4. t Numbers in brackets refer to the Bibliography.POSITIVE BOREL MEASURES $37$ and if 2.6 Theorem If $\textstyle|k_{x}\rangle$ is a collection of compact subsets of a Hausdorff space also has empty inter $\bigcap_{\alpha}K_{\alpha}=\varnothing$ , then some finite subcollection of $\textstyle|K_{x}\rangle$ section. to every some finite collection $\scriptstyle V_{n}|$ Fix a member $K_{1}$ of $\scriptstyle[K_{x}]$ Since no point of $K_{1}$ belongs for PROOF Put $V_{n}=K_{x}^{\circ}$ is an open cover of $K_{1}.$ Hence $K_{1}\subset V_{\alpha_{1}}\cap\cdots\sharp\ V_{\alpha_{n}}$ $K_{x},\;\{V_{x}\}$ This implies that $$ K_{1}\;\cap\;K_{\alpha_{1}}\;\cap\;\cdots\;\cap\;K_{\alpha_{n}}=\mathcal{D}. $$ // 2.7 Theorem Suppose $U$ is open in a locally compact Hausdorff space $X,$ $K\subset U,$ and $\textstyle K$ is compact. Then there is an open set ${\mathbf{}}V$ with compact closure such that $$ K\subset V\subset{\bar{V}}\subset U. $$ PROOF Since every point of ${\cal K}\,\,\,$ has a neighborhood with compact closure and since ${\boldsymbol{K}}$ is covered by the union of finitely many of these neighborhoods, ${\cal K}\,\,$ lies in an open set ${\boldsymbol{G}}$ with compact closure. If $U=X,$ take $V=G.$ Otherwise, let ${\mathbf{C}}$ be the complement of ${\boldsymbol{U}}.$ Theorem 2.5 shows that to each $\scriptstyle{p\in C}$ there corresponds an open set $W_{p}$ such that $K\subset W_{p}$ and $p\notin W_{p}.$ Hence $\{C\cap_{..}\tilde{G}\cap W_{p}\}.$ where $~~~~~~~~~~~~~~~~~~~~~~~~~~~~~~~~~~~~~~~~~~~~~~~~~~~~~~~~~~~~~~~~~~~~~~~~~~~~~~~~~~~~~~~~~~~~~~~~~~~~~~~~~~~~~~~~~~~~~~~~~~~~~~~~~~~~~~~~~~~~~~~~~~~~~~~~~~~~~~~~~~~$ p ranges over ${\cal{C}},$ is a collection of compact sets such that with empty intersection. By Theorem $2.6$ there are points $p_{1},\ldots,p_{n}\in C$ $$ C\cap\vec{G}\cap\overline{{{W}}}_{p_{1}}\cap\hat{\bf\omega\cdot\cdot\cdot\cdot\Gamma}\overline{{{W}}}_{p_{n}}=\mathcal{D}. $$ The set $$ V=G\,\cap\,W_{p_{1}}\,\cap\,\cdots\,\cap\,W_{p_{n}} $$ then has the required properties, since $$ \bar{V}\subset\bar{G}\bigcap\overline{{W}}_{p_{1}}\cap\cdots\cap\overline{{W}}_{p_{n}}. $$ // space. If 2.8 Definition Let $\boldsymbol{\mathsf{f}}$ be a real (or extended-real) function on a topological $$ \{x:f(x)>x\} $$ is open for every real α,fis said to be lower semicontinwous. If $$ \{x:f(x)<\alpha\} $$ is open for every real c,fis said to be upper semicontinuous. A real function is obviously continuous if and only if it is both upper and lower semicontinuous. The simplest examples of semicontinuity are furnished by characteristic func- tions:38 REAL AND coMPLEX ANALYSis (a)Characteristic functions of open sets are lower semicontinuous (b)Characteristic functions of closed sets are upper semicontinuous. The following property is an almost immediate consequence of the defini tions: (c)The supremum of any collection of lower semicontinuous functions is lower semicontinuous. The infimum of any collection of upper semicontinuous func tions is upper semicontinuous. 2.9 Definition The support of a complex function f on a topological space $X$ is the closure of the set $$ \{x;f(x)\neq0\}. $$ The collection of all continuous complex functions on $X$ whose support is compact is denoted by C.(X). Observe that $\scriptstyle C_{i}(X)$ is a vector space. This is due to two facts (a) The support of f + g lies in the union of the support of f and the support of g, and any finite union of compact sets is compact. (b)The sum of two continuous complex functions is continuous,as are scalar multiples of continuous functions (Statement and proof of Theorem 1.8 hold verbatim if“measurable function”i replaced by“continuous function,”“ measurable space”by “topological space”; take $\Phi(s,\,t)=s+t,$ or Q(s, $t)=s t,$ , to prove that sums and products of contin- uous functions are continuous.) 2.10 Theorem Let $X$ and ${\bf Y_{\nu}}$ be topological spaces, and let f: X→Y be contin- uous. If $\textstyle K$ is a compact subset of X, then f(K) is compact. PRoOF If $\scriptstyle\{\nu_{\mathrm{sl}}\}$ is an open cover of f(K), then $\{f^{-1}(V_{\alpha})\}$ $\alpha_{1},$ …,α,,and therefore $K,$ is an open cover of hence $K\subset f^{-1}(V_{\alpha_{1}})\cup\cdot\cdot\cdot\cup f^{-1}(V_{\alpha_{n}})$ for some $f(K)\subset V_{\alpha_{1}}\cup\mathbf{\bf\nabla\cdot\cdot\cdot}\cup\ V_{\alpha_{n}}.$ / Corollary The range of any fe C(X) is a compact subset of the complex plane. In fact, if $\textstyle K$ is the support of fe C(X), then $f(X)\subset f(K)\cup\{0\}.$ If X is not compact, then O e f(X), but O need not lie in f(K), as is seen by easy examples 2.11 Notation In this chapter the following conventions will be used. The notation $$ \kappa{\mathrel{<J}} $$ (1)PoSITIVE BOREL MEASUREs 39 for all will mean that $K_{\mathbf{\delta}}K$ is a compact subset of $X.$ , that f∈ $C_{c}(X),$ that $0\leq f(x)\leq1$ $x\in X,$ and that f(x) = 1 for all $x\in K$ The notation $$ \scriptstyle{\int_{}{\sqrt{}}\;\mathbf{\sqrt{}}} $$ (2) will mean that ${\mathbf{}}V$ is open, that fe C.(X), 0 ≤f≤1, and that the support of J lies in $V.$ The notation $$ K{\mathcal{L}}{\mathcal{L}}{\mathcal{V}} $$ (3) will be used to indicate that both (1) and (2) hold 2.12 Urysohn's Lemma Suppose $X$ is a locally compact Hausdorf space, ${\mathbf{}}V$ is open in $X.$ $K\subset V,$ and ${\boldsymbol{K}}$ is compact. Then there exists an fe C,(X), such that $$ K\prec f<V. $$ (1) In terms of characteristic functions, the conclusion asserts the existence of a continuous function f which satisfies the inequalities $\chi_{K}\leq f\leq\chi_{V}$ Note that it is easy to find semicontinuous functions which do this;examples are $\chi_{K}$ and $\gamma_{V}$ PROOF Put $\scriptstyle\nu_{i}=0$ $r_{2}=1,$ and let $r_{3}\,,\;r_{4}$ , rs,.…. be an enumeration of the and then $V_{\mathrm{i}}$ such rationals in (0,1). By Theorem $2.7,$ we can find open sets $V_{\mathrm{0}}$ that ${\overline{{V}}}_{0}$ is compact and $$ K\subset V_{1}\subset{\bar{V}}_{1}\subset V_{0}\subset{\bar{V}}_{0}\subset V. $$ (2) Suppose $n\geq2$ and $V_{r_{1}},\dots,$ $V_{r_{n}}$ have been chosen in such a manner that ${\boldsymbol{r}}_{j}$ ,will be the smal- so that $r_{i}<r_{j}$ implies ${\tilde{V}}_{r_{j}}\subset V_{r_{i}}$ Then one of the numbers $r_{1},\cdot\cdot\cdot,r_{n},$ S $\pm y\;r_{i}\,,$ will be the largest one which is smaller than $\,r_{n+1},$ and another, say $V_{r_{n+1}}$ lest one larger than ${\boldsymbol{r}}_{n+1}.$ Using Theorem 2.7 again, we can find $$ \bar{V}_{r_{j}}\subset V_{r_{n+1}}\subset\bar{V}_{r_{n+1}}\subset V_{r_{i}}. $$ rational Continuing,we obtain a collection $\scriptstyle\{v_{s}\}$ of open sets, one for every ,each ${\tilde{V}}_{r}$ is ${\mathbf{}}r$ e [0,1], with the following properties: $K\subset V_{1},\,V_{0}\subset V$ compact, and $$ s>r~i m p l i e s~\bar{V}_{s}\subset V_{r}. $$ (3) Define $$ f_{r}(x)={\binom{r}{0}}\cdot\quad{\mathrm{if~}}x\in V_{r},\quad\quad g_{s}(x)={\binom{1}{s}}\quad\quad{\mathrm{if~}}x\in{\bar{V}}_{s}, $$ (4) and $$ f=\operatorname*{sup}_{r}f_{r},\qquad g=\operatorname*{inf}_{s}g_{s}. $$ (5) The remarks following Definition 2.8 show that $\boldsymbol{\f}$ is lower semi- continuous and that $\scriptstyle{\mathcal{G}}$ is upper semicontinuous. It is clear that $0\leq f\leq1,$ that40 REAL AND coMPLEX ANALYSIs But $f(x)=1$ if $x\in\kappa.$ and that $\boldsymbol{\mathit{f}}$ has its support in ${\bar{V}}_{0}\,.$ The proof will be com $x\notin{\bar{V}}_{s},$ $r>s$ pleted by showing that f = 9. is possible only if $r>s,\,x\in V_{r},$ and The inequality $f_{*}(x)>g_{*}(x)$ for all ${}^{T}$ and s, sof≤ g implies $V_{r}\subset V_{s}$ Hence $f_{r}\leq g_{s}$ Suppose $f(x)<g(x)$ for some x. Then there are rationals ${}^{T}$ and s such that $f(x)\ {\underline{{<}}}\ r<s<g(x).$ Since $f(x)<r,$ we have $x\notin V_{r};$ since $g(x)>s,$ we havexe D,.By (3), this is a contradiction. Hence f = . // 2.13 Theorem Suppose $V_{1},\dots,\ V_{n}$ are open subsets of a locally compact Haus- dorff space X, ${\boldsymbol{K}}$ is compact, and $$ K\subset V_{1}\cup\cdot\cdot\ .\ \cup V_{n}. $$ Then there exist functions $$ :h_{i}<V_{i}(i=1,\dots,n)\,s u c h\,t h a t $$ $$ h_{1}(x)+\cdot\cdot\cdot+h_{n}(x)=1\qquad(x\in K). $$ (1) Because of (1), the collection $\{h_{1},\,\ldots,\,h_{n}\}$ is called a partition of unity on $K,$ subordinate to the cover $\{V_{1},\ldots,V_{n}\}.$ that $W_{x_{1}}\cup\cdot\cdot\cdot\ \cup\ W_{x_{m}}\Rightarrow K$ If $1\leq i\leq n$ By Urysohn's lemma, there are functions $g_{i}$ with compact such closure PROOF By Theorem 2.7, each for some $\overline{{}}~~~~\underline{{~i}}$ (depending on $x).$ There are points $X_{\ {1}_{\cdot}}\ \cdot\cdot\cdot\cdot\ ,\ X_{m}$ ${\tilde{W}}_{s}\subset V_{i}$ $x\in K$ let has a neighborhood $W_{x}$ $\textstyle H_{i}$ be the union of those W, which lie in $V_{i}.$ such that $H_{i}<g_{i}<V_{i}.$ Define $$ \begin{array}{l}{{h_{1}=g_{1}}}\\ {{h_{2}=(1-g_{1})g_{2}}}\\ {{\cdot\cdot\cdot\cdot\cdot\cdot\cdot}}\\ {{h_{n}=(1-g_{1})(1-g_{2})\cdots(1-g_{n-1})g_{n}.}}\end{array} $$ (2) Then $h_{i}\prec V_{i},\operatorname{II}$ is easily verified, by induction, that $$ h_{1}+h_{2}+\cdot\cdot\cdot+h_{n}=1-(1-g_{1})(1-g_{2})\cdot\cdot\cdot(1-g_{n}). $$ (3) Since $K\subset H_{1}\cup\cdots\cup H_{n},$ at least one $g_{i}(x)=1$ at each point $x\in K$ hence (3) shows that (1) holds. // The Riesz Representation Theorem 2.14 Theorem Let X be a locally compact Hausdorff space, and let A be a $X$ contains all Borel sets in $X,$ ,and there exists $\bar{a}$ Then there exists a o-algebra OD in $X$ which positive linear functional on $C_{c}(X).$ unique positive measure ${\boldsymbol{\mu}}$ u on D which represents $\Lambda$ in the sense that (a)Af =Jxf dp for everyfe $C_{c}(X),$ and which has the following additional properties:POSIrTTVE BOREL MEASURES 4 (b)p(K)<oo for every compact set $\kappa\in X$ (c) For every $E\in{\mathcal{M}}$ ,we have $$ \mu(E)=\operatorname*{inf}\;\{\mu(V)\colon E\subset V,\;V\;\mathrm{{\;open}}\}. $$ (d)The relation $$ \mu(E)=\operatorname*{sup}\;\{\mu(K);\;K\subset E,\;K\;\operatorname{compact}\} $$ holds for every open set ${\boldsymbol{E}},$ and for every E e OD with u(E) < (e)If E∈ D, A C E, and $\mu(E)=0.$ , then $A\in\mathfrak{M}$ For the sake of clarity, let us be more explicit about the meaning of the word “ positive”in the hypothesis: A is assumed to be a linear functional on the complex vector space $C_{c}(X),$ with the additional property that Af is a nonnegative real number for every $\boldsymbol{\mathsf{f}}$ whose range consists of nonnegative real numbers. Briefly, $\operatorname{lf}f(X)\subset\left[0,\right.$ o) then Af e[O,00) Property a is of course the one of greatest interest. After we define OD and 山 that ${\boldsymbol{\mu}}$ “ reasonable”spaces $X$ ) will be established in the course of proving that W is a o-algebra and (b) to $\mathbf{\nabla}(d)$ is countably additive. We shall see later(Theorem 2.18)that in every Borel measure which satisfies (b) also satisfies (c and (d and that (d) actually holds for every $\boldsymbol{E}$ e D, in those cases. Property (e) merely says that (X,D,p) is a complete measure space, in the sense of Theorem 1.36. Throughout the proof of this theorem, the letter ${\boldsymbol{K}}$ will stand for a compact subset of $X,$ and ${\mathbf{}}V$ will denote an open set in $X.$ that prove that Let us begin by proving the uniqueness of for all $K,$ whenever $\mu_{1}$ and $\mu_{2}$ are measures for which with ${\boldsymbol{\mu}}$ ${\boldsymbol{\mu}}.$ ${\boldsymbol{\mu}}$ 1.If u satisfies $\mathbf{\Psi}(c)$ and (d, it is clear is determined on by its values on compact sets. Hence it suffices to $\mu_{1}(K)=\mu_{2}(K)$ K and $\scriptstyle x\;{\ _{3}}$ By (b) and (c), there exists a $V\to K$ the theorem holds. So, fix ${\cal K}$ hence $\mu_{2}(V)<\mu_{2}(K)+\epsilon;$ by Urysohn's lemma, there exists an $\boldsymbol{\f}$ so that K $<f<V;$ $$ \mu_{1}(K)=\left|_{X}^{}\chi_{K}\,d\mu_{1}\leq [_{X}f\,d\mu_{1}=\Lambda f= (\frac\star\right)_{X}f\,d\mu_{2} $$ $$ \leq\ \int_{X}\chi_{V}\ d\mu_{2}=\mu_{2}(V)<\mu_{2}(K)+\epsilon. $$ Thus $\mu_{1}(K)\leq\mu_{2}(K).$ If we interchange the roles of $\mu_{1}$ and $\mu_{2}\,$ ,the opposite inequality is obtained, and the uniqueness of ${\boldsymbol{\mu}}$ is proved Incidentally the above computation shows that (a forces (b) Construction of p and D For every open set ${\mathbf{}}V$ in $X,$ define $$ \mu(V)=\operatorname*{sup}\;\{\Lambda f;f<V\}. $$ (1)$42$ REAL AND cOMPLEX ANALYSiS If $V_{1}\in V_{2},{\mathrm{if}}$ is clear that $\mathbf{(1)}$ implies $\mu(V_{1})\leq\mu(V_{2}).$ Hence $$ \mu(E)=\operatorname*{inf}\left\{\mu(V);\;E\subset V,\;V\;\mathrm{open}\right\}\!, $$ (2) if $\boldsymbol{E}$ is an open set, and it is consistent with(1) to define $\scriptstyle{i(k)}$ by (2), for every $\scriptstyle{E\in X}$ Note that although we have defined $\scriptstyle{\theta(E)}$ for every $\scriptstyle{E\in X}$ the countable additivity of ${\mathfrak{M}}_{F}$ be the class of all $\scriptstyle{E\in X}$ which satisfy two conditions: $X$ and Let ${\boldsymbol{\mu}}$ will be proved only on a certain o-algebra の in $\mu(E)<\infty,$ $$ \mu(E)=\operatorname*{sup}\;\{\mu(K);\,K\subset E,\,K\;\operatorname{compact}\}. $$ (3) Finally, let ODt be the class of all $\scriptstyle{F arrow X}$ such that $E\cap K\in{\mathfrak{M}}_{F}$ for every compact $K.$ Proof that ${\boldsymbol{\mu}}$ and Ot have the required properties $\mu(E)=0$ It is evident that $\boldsymbol{\mu}$ is monotone, i.e.,that $\mu(A)\leq\mu(B)$ if $\scriptstyle A\leq B$ and that implies $E\in\mathbb{N}_{F}$ and E∈ OD. Thus (e) holds, and so does (c), by defini- tion Since the proof of the other assertions is rather long, it will be convenient to divide it into several steps. Observe that the positivity of $\Lambda$ implies that A is monotone: $\scriptstyle J\leq\theta$ implies $\Lambda f\leq\Lambda g.$ onicity will be used in Steps $\mathbf{I}$ and $\mathrm{X}.$ $\Lambda g=\Lambda f+\Lambda(g-f)$ and $g-f\geq0.$ This monot- This is clear, since STEP I $I f\ E_{1},E_{2},E_{3},\dots a r e$ arbitrary subsets of $X.$ then $$ \mu{\Bigg(}\bigcup_{i=1}^{\infty}E_{i}{\Bigg)}\leq\sum_{i=1}^{\infty}\mu(E_{i}). $$ (4) PRoOF We first show that $$ \mu(V_{1}\cup\ V_{2})\leq\mu(V_{1})+\mu(V_{2}), $$ (5) if ${\mathit{V}}_{1}$ and ${\mathit{V}}_{2}$ are open. Choose $g<V_{1}\cup V_{2}$ By Theorem 2.13 there are func- tions $h_{1}$ and $h_{2}$ such that $h_{i}\prec V_{i}$ and $h_{1}(x)+h_{2}(x)=1$ for all x in the support of ${\mathfrak{g}}.$ Hence $h_{i}g\prec V_{i},g=h_{1}g+h_{2}\,g,$ and so $$ \Lambda g=\Lambda(h_{1}g)+\Lambda(h_{2}g)\leq\mu(V_{1})+\mu(V_{2}). $$ (6) that If Since (6) holds for every $g<V_{1}\ \cup\ V_{2},(5)$ follows. such $\mu(E_{i})=\infty$ for some $i,\mathbf{\alpha}$ then $\mathbf{\tau}(\lambda)$ is trivially true. Suppose therefore that $\mu(E_{i})<\mathbf{x}$ for every i. Choose $\scriptstyle\epsilon\;>_{0},$ By (2) there are open sets $V_{i}\to E_{i}$ $$ \mu(V_{i})<\mu(E_{i})+2^{-i}\epsilon\qquad(i=1,\,2,\,3,\,\ldots). $$POSITIVE BOREL MEASURES 43 Put $V=\left(\right)_{1}^{\infty}\,V_{i},$ and choose $f\prec V.$ Since $\boldsymbol{\f}$ has compact support, we see that $f\prec V_{1}\ \cup\ \cdots\ \cup\ \ V_{n}$ for some ${\mathfrak{n}}.$ Applying induction to (5), we therefore obtain $$ \Lambda f\leq\mu(V_{1}\ \sim\ \cdot\ \cdot\ \cup\ V_{n})\leq\mu(V_{1})+\ \cdot\cdot\cdot+\mu(V_{n})\leq\sum_{i=1}^{\infty}\mu(E_{i})+\epsilon. $$ Since this holds for every $f<V,$ and since $\bigcup E_{i}\subset V,{\mathrm{it~follows~that}}$ $$ \mu{\bigg(}\bigcup_{i=1}^{\infty}E_{i}{\bigg)}\leq\mu(V)\leq\sum_{i=1}^{\infty}\mu(E_{i})+\epsilon, $$ which proves (4), since e was arbitrary. // STEP I If ${\cal K}\,\,$ is compact, then $K\in\mathfrak{M}_{F}a n d$ $$ \mu(K)=\operatorname*{inf}\left\{\Lambda f\colon K-f\right\}. $$ (7) This implies assertion $\mathbf{(}b\mathbf{)}$ of the theorem. PRoor If $K<f$ and $0<\alpha<1.$ let $V_{\alpha}=\{x\colon f(x)>\alpha\}.$ Then $K\subset V_{a},$ and αg ≤f whenever $g\prec V_{\alpha}.$ Hence $$ \mu(K)\leq\mu(V_{\alpha})=\operatorname*{sup}\;\{\mathrm{A}g\!:\,g\prec V_{\alpha}\}\leq\alpha^{-1}\Lambda f. $$ Let α→1, to conclude that $$ \mu(K)\leq\Lambda f. $$ (8) Thus $\mu(K)<\alpha$ Since $\displaystyle K$ evidently satisfies (3), $K\in\mathfrak{M}_{P}.$ By Urysohn's lemma, If $\scriptstyle\epsilon\;>0,$ there exists $V\to K$ with $\mu(V)<\mu(K)+\epsilon.$ 人 ${\mathfrak{X}}{\ <}f{\ <}V$ for some f. Thus $$ \Lambda f\leq\mu(V)<\mu(K)+\epsilon, $$ which, combined with (8), gives (7) // STEP m Every open set satisfies (3). Hence ${\mathfrak{M}}_{F}$ contains every open set ${\mathbf{}}V$ with $\mu(V)<\infty$ $x<\Lambda f.$ If $\Lambda f\leq\mu(W).$ Thus be a real number such that $\alpha<\mu(V).$ There exists ${\mathfrak{a n}}f\ll V$ with PR00F Let $\scriptstyle{\dot{\mathbf{x}}}$ is any open set which contains the support $\textstyle K$ of , then <W with ${\boldsymbol{W}}$ hence $\Lambda f\leq\mu(K),$ This exhibits a compact $\kappa\in V$ / $x<\mu(K),$ so that (3) holds for $V.$ STEP Iv Suppose $E=\bigcup_{i=1}^{\infty}E_{i}$ ,where $E_{1},E_{2},E_{3},\dots$ are pairwise disjoint members of DR, . Then $$ \mu({\cal E})=\sum_{i=1}^{\infty}\mu({\cal E}_{i}). $$ (9) ${\mathit{I}}{\mathit{f}},$ in addition, $\mu(E)<\infty,$ then also $E\in\mathfrak{M}_{P}.$$44$ REAL AND cOMPLEX ANALYSIS PROOF We first show that $$ \mu(K_{1}\cup K_{2})=\mu(K_{1})+\mu(K_{2}) $$ (10) if $K_{1}$ and $K_{2}$ are disjoint compact sets. Choose $\scriptstyle\epsilon\;>0,$ By Urysohn's lemma there exists $f\in C_{*}(X)$ $\Pi$ there exists $\scriptstyle{\mathcal{G}}$ such that on $K_{1},\;f(x)=0$ on $K_{2},$ and $0\leq f\leq1.$ such that $f(x)=1$ By Step $$ K_{1}\cup K_{2}\prec g\quad{\mathrm{and}}\quad\Lambda g<\mu(K_{1}\cup K_{2})+\epsilon. $$ that Note that $K_{1}\triangleleft f g$ and $K_{2}<(1-f)g.$ Since $\mathrm{\A}$ is linear, it follows from(8) $$ u(K_{1})+\mu(K_{2})\leq\Lambda(f g)+\Lambda(g-f g)=\Lambda g<\mu(K_{1}\cup K_{2})+\epsilon. $$ Since e was arbitrary,(10) follows now from Step I. If $\mu(E)=\infty,$ (9) follows from Step I. Assume therefore that $\mu(E)<\infty,$ and choose $\scriptstyle\epsilon\;>0,$ Since $E_{i}\in\mathfrak{M}_{P},$ there are compact sets $H_{i}\in E_{i}$ with $$ \mu(H_{i})>\mu(E_{i})-2^{-i}\epsilon\qquad(i=1,2,\,3,\dots). $$ (11) Puttin $K_{n}=H_{1}\cup\cdot\cdot\sim\cup\,H_{n}$ and using induction on(10), we obtain $$ \mu(E)\geq\mu(K_{n})=\sum_{i=1}^{n}\mu(H_{i})>\sum_{i=1}^{n}\mu(E_{i})-\epsilon. $$ (12) Since (12) holds for every $\;n$ and every $\scriptstyle\epsilon\;>0$ the left side of ((9) is not smaller than the right side, and so (9) follows from Step $\mathbf{I}.$ But if $\mu(E)<\infty$ and $\scriptstyle\epsilon\;>_{0},$ (9) shows that $$ \mu(E)\leq\sum_{i=1}^{N}\mu(E_{i})+\epsilon $$ (13) for some N. By (12), it follows that $E\in{\mathfrak{M}}_{F}$ $\mu(E)\leq\mu(K_{N})+2\epsilon,$ and this shows that $\boldsymbol{E}$ satisfies (3); hence // STEP v If $E\in\mathbb{N}_{F}$ and $\scriptstyle\epsilon\;>0,$ there is a compact ${\cal K}$ and an open ${\mathbf{}}V$ such that $K\subset E\subset V$ and $\mu(V-K)<\epsilon.$ PROOF Our definitions show that there exist $\scriptstyle{\kappa\in E}$ and $V\to E$ so that $$ \mu(V)-{\frac{\epsilon}{2}}<\mu(E)<\mu(K)+{\frac{\epsilon}{2}}. $$ Since $V-K$ is open, $V-K\in\Re_{F}.$ by Step II. Hence Step IV implies that $$ \mu(K)+\mu(V-K)=\mu(V)<\mu(K)+\epsilon. $$ // STEP vi If $4\in\mathfrak{M}_{r}$ and $B\in\mathfrak{M}_{F},t h e n~A-B,A\ \cup~B,\iota$ and A n B belong to ${\mathfrak{M}}_{F}$POSITIVE BOREL MEASURES $45$ PRoor If $\scriptstyle\epsilon\;>0,$ Step $\mathbf{V}$ shows that there are sets $K_{i}$ and ${\mathit{V}}_{i}$ such that $K_{1}\subset A\subset V_{1},\,K_{2}\subset B\subset V_{2},$ and $\mu(V_{i}-K_{i})<\epsilon,\mathrm{for}:$ = 1,2. Since $$ A-B\subset V_{1}-K_{2}\subset(V_{1}-K_{1})\cup(K_{1}-V_{2})\cup(V_{2}-K_{2}) $$ Step I shows that $$ \mu(A-B)\leq\epsilon+\mu(K_{1}-V_{2})+\epsilon. $$ (14) Since $K_{1}-V_{2}$ is a compact subset of A - B,(14) shows that $\scriptstyle4\,-\,B$ satisfies (3), so that A - B ∈ 领, Since $A\ \cup\ B=(A-B)\cup B,$ an application of Step $\mathbf{I}\mathbf{V}$ shows that A $\{\sim B\in{\mathfrak{M}}_{F}$ Since $A\cap B=A-(A-B),$ we also have $A\frown B\in{\mathfrak{M}}_{F}$ // STEP VIi のR is a o-algebra in $X$ which contains all Borel sets. PRO0F Let ${\cal K}\,\,\,\,\,\,\,\,\,\,{\cal K}$ be an arbitrary compact set in $X.$ If A∈ D, then $A^{c}\frown K=K-(A\cap K),$ so that $4^{c}\cap K$ is a difference of two members of ${\mathcal{M}}_{F}$ . Hence $A^{c}\,\cap\,K\in\Re_{F}.$ , and we conclude: A ∈ の implies $A^{c}\in\Re.$ $A=\left(\right)_{1}^{\infty}\,A_{i},$ where each $A_{i}$ ∈ D. Put $B_{1}=A_{1}\cap K,$ and Next, suppose $$ B_{n}=(A_{n}\cap K)-(B_{1}\cup\mathbf{\partial}\cdot\cdot\cdot\cup\ B_{n-1})\qquad(n=2,\,3,4,\,...). $$ (15)) Then $\scriptstyle(b_{a}|$ is a disjoint sequence of members of Dt,, by Step VI, and $A\ \cap K=\bigcup_{1}^{\infty}B_{n}.{\mathrm{It}}$ follows from Step IV that $A\cap K$ ∈ D .Hence A ∈ D. so F'inally,if ${\boldsymbol{C}}$ is closed, then $c\cap K$ is compact, hence $C\cap K\in\mathfrak{M}_{F},$ C ∈ D. In particular, X ∈の. We have thus proved that の is a ${\boldsymbol{\sigma}}.$ -algebra in $X$ which contains all closed subsets of X. Hence DR contains all Borel sets in X // STEP VII ${\mathfrak{M}}_{F}$ consists of precisely those sets E e Dt for which $\mu(E)<\alpha.$ This implies assertion (d) of the theorem. compact $K,$ hence $\boldsymbol{E}$ e 咖 ,Steps II and VI imply that $E\cap K\in{\mathfrak{M}}_{F}$ for every with PRoOF If $E\in\mathbb{N}_{F}$ $\mu(V-K)<\epsilon.$ Conversely, suppose $\scriptstyle E\in{\mathfrak{M}}$ and $\mu(E)<\infty,$ and choose $\scriptstyle\epsilon\;>0$ There is an open set $V\to E$ with $\mu(V)<\alpha;$ by III and V, there is a compact $\kappa\in V$ with Since $E\cap K\in{\mathfrak{M}}_{F}$ , there is a compact set $H\subset E\cap K$ $$ \mu(E\cap K)<\mu(H)+\epsilon. $$ Since $E\subset(E\cap K)\cup(V-K),$ it follows that $$ \mu(E)\leq\mu(E\,\cap\,K)+\mu(V-K)<\mu(H)+2\epsilon, $$ which implies that $E\in\mathbb{N}_{F}.$ // STEP IX ${\boldsymbol{\mu}}$ is a measure on ${\mathfrak{M}}.$46 REAL AND COMPLEX ANALYSIS PR0OF The countable additivity of ${\boldsymbol{\mu}}$ on De follows immediately from Step IV and VIII // STEP x For every fe C(X)X $\Lambda f=\textstyle\int_{X}f\,d\mu$ L. This proves (a), and completes the theorem PRoOF Clearly, it is enough to prove this for real ${\boldsymbol{f}}.$ Also,it is enough to prove the inequality $$ \Lambda f\leq\int_{x}f\,d\mu $$ (16) for every $\operatorname{real}f\in C_{c}(X).$ For once (16) is established, the linearity of $\Lambda$ shows that $$ -\Lambda f=\Lambda(-f)\leq\int_{x}(-f)~d\mu=-\int_{x}f\,d\mu, $$ which, together with (16), shows that equality holds in (16) and choose $y_{i}.$ be the support of a real $f\in C_{c}(X),$ let [a,b] be an interval which $\scriptstyle\epsilon\;>_{0},$ Let ${\cal K}\,\,\,\,\,$ ,, for $\scriptstyle{i=0},$ 1, .….,, so that (note the Corollary to Theorem 2.10), choose and contains the range of $\boldsymbol{\f}$ $y_{i}-y_{i-1}<\epsilon$ $$ y_{0}<a<y_{1}<\cdots<y_{n}=b. $$ (17) Put $$ E_{i}=\{x;\,y_{i-1}<f(x)\leq y_{i}\}\,\cap\,K\qquad(i=1,\,\ldots,\,n). $$ (18) Since $\boldsymbol{\mathsf{f}}$ disjoint Borel sets whose union is $K.$ is Borel measurable, and the sets $\textstyle E_{i}$ are therefore is continuous, $\boldsymbol{\f}$ There are open sets $V_{i}\supset E_{i}$ such that $$ \mu(V_{i})<\mu(E_{i})+{\frac{\epsilon}{n}}\;\;\;\;\;\;\;\;(i=1,\;...\,,n) $$ (19) that and such that $f(x)<y_{i}{\underline{{+}}}\in$ for all $x\in V_{*}$ By Theorem 2.13, there are func tions $\mathbb{N}_{n}\times V_{n}$ such that $\textstyle\sum h_{i}=1$ on $K.$ Hence $f=\sum\,h_{i}f,$ and Step II shows $$ \mu(K)\leq\Lambda(\sum h_{i})=\sum\Lambda h_{i}. $$POSITTVE BOREL MEASURES $47$ Since $h_{i}f\leq(y_{i}+\epsilon)h_{i},$ and since $y_{i}-\epsilon<f(x)$ on $E_{i},$ we have $$ \begin{array}{c}{{\frac{\vert\beta\vert}{\vert\varepsilon\vert}}}\\ {{\frac{\vert}{\vert\alpha\vert}}}\end{array} $$ Jx" Since e was arbitrary, (16) is established, and the proof of the theorem is complete // Regularity Properties of Borel Measures 2.15 Definition A measure $\boldsymbol{\mu}$ defined on the o-algebra of all Borel sets in a locally compact Hausdorff space $X$ is called a Borel measure on X. If $\boldsymbol{\mu}$ is positive, a Borel set $\scriptstyle{E\cdot x}$ is outer regular or inner regular, respectively, i $\boldsymbol{E}$ has property (c or (d of Theorem 2.14. If every Borel set in $X$ is both outer and inner regular, ${\boldsymbol{\mu}}$ is called regular. was In our proof of the Riesz theorem, outer regularity of every set $\boldsymbol{E}$ built into the construction, but inner regularity was proved only for the open sets and for those $E\in\mathbb{N}$ for which $\mu(E)<\infty{\mathrm{~it}}$ turns out that this flaw is in the nature of things One cannot prove regularity of $\boldsymbol{\mu}$ under the hypothesis of Theorem 2.14; an example is described in Exercise 17. However, a slight strengthening of the hypotheses does give us a regular measure. Theorem 2.17 shows this. And if we specialize a little more, Theorem 2.18 shows that all regularity problems neatly disappear. 2.16 Definition A set $\boldsymbol{E}$ in a topological space is called o-compact if $\boldsymbol{E}$ is a countable union of compact sets. A set $\boldsymbol{E}$ in a measure space (with measure with $\mu(E_{0})<\infty$ O- measure if $\boldsymbol{E}$ is a countable union of sets $\boldsymbol{\mu}$ is said to have o-finite $E_{i}$ For example,、in the situation described in Theorem 2.14,every compact set has ${\boldsymbol{\sigma}}\cdot$ finite measure. Also, it is easy to sce that if $E\in\mathbb{N}$ and $\boldsymbol{E}$ has ${\boldsymbol{\sigma}}^{*}$ finite measure, then $\boldsymbol{E}$ is inner regular.48 REAL AND coMPLEX ANALYSIs 2.17 Theorem Suppose $\textstyle X$ is a locally compact, o-compact Hausdorff space.If o and p are as described in the statement of Theorem 2.14, then O and p have the following properties: (b) (a)r ${\boldsymbol{\mu}}$ is a regular Borel measure on $\scriptstyle A$ and $\boldsymbol{B}$ such that A is an F。,B is a such that $G_{\delta}$ $E\in\mathbb{N}$ and $\scriptstyle\mathbf{\hat{e}}\;>0,$ there is a closed set ${\mathbf{}}F$ and an open set ${\mathbf{}}V$ $F\subset E\subset V$ and $\mu(V-F)<\epsilon.$ $X.$ (c)if $E\in{\mathfrak{M}}$ ,there are sets $A\subset E\subset B.$ and $\mu(B-A)=0.$ of measure O As a corollary of (c) we see that every $E\in{\mathfrak{M}}$ is the union of an $F_{\sigma}$ and a set that PRoOF Let $X=K_{1}\ \cup\ K_{2}\ \cup\ K_{3}\ \cup\ \cdots$ ,where each $K_{n}$ is compact. I $F\in{\mathfrak{M}}$ and $\scriptstyle\epsilon\;>0.$ then $\mu(K_{n}\cap E)<\infty,$ and there are open sets $V_{n}\supset K_{n}\cap E$ such $$ \mu(V_{n}-(K_{n}\cap E))<\frac{\epsilon}{2^{n+1}}\qquad(n=1,\,2,\,3,\,\ldots). $$ (1) If $V=\bigcup V_{n},$ then $V-E\subset{\big\lfloor}\backslash\{V_{n}-(K_{n}\cap E)),$ so that $$ \mu(V-E)<{\frac{\epsilon}{2}}. $$ follows. Apply this to $E^{\mathrm{c}}$ in place of $E{\mathrel{:}}$ There is an open set $W\supset E^{*}$ such that $\mu(W-E^{c})<\epsilon/2.$ If $F=W^{\circ},$ then $F\in E.$ and $E-F=W-E^{\circ}$ Now(a) Every closed set $\scriptstyle{F\in X}$ is ${\boldsymbol{\sigma}}.$ -compact, because $F=\bigcup_{k\leq1}\cap K_{n}).$ Hence (a) implies that every set $E\in{\mathfrak{D}}$ is inner regular. This proves (b). ${\cal B}=\bigcap{V}_{j}$ If we apply (a) with e= 1 $~j~(j=1,\,2,\,3,\,\ldots),$ we obtain closed sets $F_{j}$ and open sets ${\mathit{V}}_{j}$ such that $F_{j}\subset E\subset V_{j}$ and $\mu(V_{j}-F_{j})<1/j.$ $G_{\partial},$ and $\mu(B-A)=0$ 'since . Then $A\subset E\subset B,\;A$ is an 2,3, …… This proves Cc) Put $A=\bigcup F_{j}$ and $F_{\sigma},$ $\boldsymbol{B}$ is a $B-A\subset V_{j}-F_{j}\mathrm{for}\,j=1,$ // 2.18 Theorem Let $X$ be a locally compact Hausdorff space in which every open set is G-compact. Let $\lambda$ be any positive Borel measure on $X$ such tha $\lambda(K)<\infty$ for every compact set K. Then Ais regular. Note that every euclidean space ${\boldsymbol{R}}^{k}$ satisfies the present hypothesis, since every open set in $R^{k}$ is a countable union of closed ballsPOsrrTVE BOREL MEASURES 49 $\Lambda$ PROOF Put ${\mathsf{N}}/=\int_{X}f\,d{\lambda},\,{\mathrm{for}}\,f\in C_{c}(X)$ Since $\lambda(K)<\infty$ for every compact $K,$ is a positive linear functional on $C_{c}(X),$ and there is a regular measure ${\boldsymbol{\mu}},$ satisfying the conclusions of Theorem 2.17, such that $$ \bigcap_{x}f\,d\lambda=\int_{x}f\,d\mu\qquad(f\in C_{c}(X)). $$ (1) We will show that $\lambda=\mu.$ point Let ${\mathbf{}}V$ be open in Then ${\mathit{g_{n}}}$ ∈ C.(X) and $\scriptstyle y.(x)$ increases to $\scriptstyle\epsilon_{i}=\epsilon_{i}$ at every Let $X.$ Then $V=\bigcup K_{i}$ where $K_{i}$ is compact, i = 1,2, 3,.……. By Urysohn's lemma we can choose ${\dot{f}}_{i}$ so that $K_{i}<f_{i}<V.$ $g_{n}=\operatorname*{max}\,(f_{1},\dots,f_{n}).$ . Hence((1) and the monotone convergence theorem imply $\operatorname{xe}{\mathcal{X}}$ $$ \lambda(V)=\operatorname*{lim}_{n arrow\infty}\,\left[\right)_{x}^{}g_{n}\,d\lambda=\operatorname*{lim}_{n arrow\infty}\,\int_{x}g_{n}\,d\mu=\mu(V)\;. $$ (2) Now let $\boldsymbol{E}$ be a Borel set in $X,$ and choose $\scriptstyle\epsilon\;>0$ Since ${\boldsymbol{\mu}}$ satisfies Theo- and rem 2.17, there is a closed set ${\mathbf{}}F$ and an open set ${\mathbf{}}V$ such that $F\prec E\subset V$ $\mu(V-F)<\epsilon.$ Hence $\mu(V)\leq\mu(F)+\epsilon\leq\mu(E)+\epsilon.$ hence $\lambda(V)\leq\lambda(E)+\epsilon$ Since $V-F$ is open,(2) shows that $\lambda(V\div F)<\epsilon,$ Consequently and so that|A(E) - p(E)1 $$ \begin{array}{c}{{\lambda(E)\leq\lambda(V)\leq\mu(E)+\epsilon}}\\ {{\mu(E)\leq\mu(V)\leq\lambda(E)+\epsilon\,,}}\\ {{\mu(E)\leq\phi(\forall)\leq\lambda(E)+\epsilon\,,}}\\ {{<\epsilon{\mathrm{~for~everv}}\,\epsilon>0.~{\mathrm{Hence}}\,\lambda(E)=u l I}}\end{array} $$ 从ED) // In Exercise 18 a compact Hausdorff space is described in which the com- plement of a certain point fails to be ${\boldsymbol{\sigma}}\cdot$ -compact and in which the conclusion of the preceding theorem is not true. Lebesgue Measure 2.19 Euclidean Spaces Euclidean $k\!\cdot\!$ -dimensional space $R^{k}$ is the set of all points $x=(\xi_{1},\dots,\xi_{k})$ whose coordinates $\xi_{i}$ are real numbers, with the following alge- braic and topological structure: If $x=(\xi_{1},\ldots,\ \xi_{k}),\,y=(\eta_{1},\ldots,\eta_{k}),$ and $\scriptstyle{\vec{\alpha}}$ is a real number $x+y$ and αx are defined by $$ {}^{x}+y=(\xi_{1}+\eta_{1},\dots,\,\xi_{k}+\eta_{k}),\qquad\alpha x=(\alpha\xi_{1},\dots,\,\alpha\xi_{k})\;. $$ (1) This makes ${\boldsymbol{R}}^{k}$ into a real vector space. If $x\cdot\O y=\sum\xi_{i}\eta,$ , and $|x|=(x\cdot x)^{1/2};$ the Schwarz inequality $|x\cdot y|\leq|x|\ |y|\operatorname{li}$ eads to the triangle inequality $$ |x-y|\leq|x-z|+|z-y|\,; $$ (2) hence we obtain a metric by setting $\rho(x,\,y)=|x-y|$ . We assume that these facts are familiar and shall prove them in greater generality in Chap. 4.${\mathsf{S O}}$ REAL AND COMPLEX ANALYSiS If $E\subset R^{k}{\mathrm{~and~}}x\in R^{k},$ the translate of $\boldsymbol{E}$ by $\scriptstyle{\mathcal{X}}$ x is the se $$ E+x=\{y+x\colon y\in E\}. $$ (3) A set of the form $$ W=\{x\colon\alpha_{i}<\xi_{i}<\beta_{i},\;1\leq i\leq k\}, $$ (4) or any set obtained by replacing any or all of the < signs in (4) by ≤,is called a $k{\mathrm{-}}c e l l\,;$ its volume is defined to be $$ \operatorname{vol}\left(W\right)=\prod_{i=1}^{k}\left(\beta_{i}-\alpha_{i}\right). $$ (5) If a e $\hat{z}\ R^{k}$ and $\scriptstyle\delta>0$ we shall call the set $$ Q(a;\delta)=\left\{x;\alpha_{i}\leq\xi_{i}<\alpha_{i}+\delta,\;1\leq i\leq k\right\} $$ (6) the ${\hat{\boldsymbol{\delta}}}.$ -box with corner at a. Here we let $P_{n}$ be the set of all $x\in R^{4}$ whose coordinates are boxes with The For integral multiples of $2^{-n},$ and we let $a=(\alpha_{1},\ \ldots,\,\alpha_{k})$ be the collection of all $2^{-n}$ $|\Omega_{n}|$ $\textstyle n=1$ ,2 $\mathbb{S}\ldots$ corners at points of $\Omega_{n}$ $P_{n}$ . We shall need the following four properties of first three are obvious by inspection (bif (a)If n is fixed, each ${\boldsymbol{Q}}.$ lies in one and only one member of $Q\subset Q^{\prime}$ or $Q^{\prime}\cap Q^{\prime}=\mathcal{D}.$ $x\in R^{4}$ $\Omega_{n}.$ (c)If $Q^{\prime}\in\Omega_{n},\,Q^{\prime}\in\Omega_{r},$ and $r<n,$ then either , the set $P_{n}$ has exactly 2"-* $Q\in\Omega,$ then vol $(Q)=2^{-n}$ ; and i $n>r.$ points in (d) Every nonempty open set in ${\boldsymbol{R}}^{k}$ is a countable union of disjoint boxes belonging to $\Omega_{1}\cup\Omega_{2}\cup\Omega_{3}$ U· hence PRo0F OF(d) If ${\mathbf{}}V$ is open, every $\textstyle{\mathcal{Q}}$ belonging to some $\Omega_{n}.$ In other words, $\Omega_{n}$ . From this $V;$ $x\in V$ lies in an open ball which lies in union of all boxes which lie in ${\mathbf{}}V$ ${\mathbf{}}V$ is the $x\in Q\in V$ for some and which belong to some collection of boxes, select those which belong to $\Omega_{1},$ and remove those in $\Omega_{2}$ $\mathbf{a}_{1},\ldots$ which lie in any of the selected boxes. From the remaining collection, select those boxes of $\Omega_{2}$ which lie in ${\mathit{V}},$ and remove those in $\Omega_{3}$ $\Omega_{4}$ which lie in any of the selected boxes. If we proceed in this way,(a) and (b show that (d) holds. /// 2.20 Theorem There exists a positive complete measure m defined on a o- alaebra 领 in $R^{k},$ R*, with the following properties: (a) $m(W)=\operatorname{vol}\,(W).$ for every k-cell $\textstyle W.$ (b)のt contains all Borel sets in $\;R^{k}\,;$ more precisely, Ee OR if and only if there are sets $\scriptstyle A$ and $\scriptstyle S\,=\,R^{4}$ such that $4\subset E\subset B.$ A is an $F_{\sigma}$ $\boldsymbol{B}$ is a $G_{\delta},$ ,and $m(B-A)=0.$ Also, m is regularPOSITVE BOREL MEASURES 51 (c)m is translation-invariant, i.e., $$ m(E+x)=m(E) $$ for every $\boldsymbol{E}$ ∈ DR and every $x\in R^{4}$ (d)If p is any positive translation-invariant Borel measure on ${\boldsymbol{R}}^{k}$ such that (e) $\mu(K)<\infty$ for every compact set $K,$ then there is a constant $\scriptstyle{\mathcal{C}}$ such that $\mu(E)=c m(E)J$ for all Borel sets $E\subset R^{k}.$ $R^{k}$ into ${\boldsymbol{R}}^{k}$ corresponds a real number To every linear transformation ${\mathbf{}}T$ 0f A(T) such that $$ m(T(E))=\Delta(T)m(E) $$ for every E e D. In particular, $m(T(E))=m(E)$ when ${\mathbf{}}T$ is a rotation. The members of S ${\mathfrak{M}}$ are the Lebesgue measurable sets in $R^{k};$ m is the ${\boldsymbol{L}},$ ebesgue measure on $R^{k}.$ When clarity requires it, we shall write $m_{k}$ in place of $m.$ PROOF Iffis any complex function on $R^{k},$ with compact support, define $$ \Lambda_{n}\,f=2^{-n}\sum_{x\in F_{n}}f(x)\qquad(n=1,\,2,\,3,\,\ldots), $$ (1) where $P_{n}$ is as in Sec. 2.19. $f\in C_{c}(R^{k}),$ f is real, $\textstyle W$ is an open $\boldsymbol{k}$ -cell which contains the Now suppose f, and $\scriptstyle\epsilon\;>0$ The uniform continuity of f([26], Theorem 4.19) support of ${\mathfrak{f}},$ support in $W,$ shows that there is an integer and $({\operatorname{in}})\,h-g<\epsilon.$ and that there are functions $\scriptstyle{\mathcal{G}}$ and $\boldsymbol{h}$ with ${\boldsymbol{N}}$ $\Omega_{N},(\mathbf{i})\,g\leq f\leq h,$ such that () $\scriptstyle{\mathcal{G}}$ z and ${\boldsymbol{h}}$ are constant on each box belonging to If $n>N.$ , Property 2.19(c) shows that $$ \Lambda_{N}\,g=\Lambda_{n}\,g\leq\Lambda_{n}f\leq\Lambda_{n}h=\Lambda_{N}\,h. $$ (2) Thus the upper and lower limits of $\{\Lambda_{n}f\}$ differ by at most e vol (W), and since $\textstyle{\epsilon}$ was arbitrary, we have proved the existence of $$ \Lambda f=\operatorname*{lim}_{n arrow\infty}\Lambda_{n}f\qquad(f\in C_{c}(R^{k})). $$ (3) It is immediate that $\Lambda$ is a positive linear functional on $C_{c}(R^{k}).$ (In fact, Af is precisely the Riemann integral off over $R^{k}.$ We went through the preceding construction in order not to have to rely on any theorems about Riemann integrals in several variables.) We define m and O to be the measure and o-algebra associated with this A as in Theorem 2.14. Since Theorem 2.14 gives us a complete measure and since ${\boldsymbol{R}}^{k}$ is o- compact, Theorem 2.17 implies assertion (b) of Theorem 2.20.${\mathsf{s}}2$ REAL AND COMPLEX ANALYSIS $f_{r}<W,$ To prove (a), let $\textstyle W$ be the open cell 2.19(4), let $W,$ choose $f_{r}$ so that $\scriptstyle{E_{r}\times}$ boxes belonging to $\Omega_{r}$ whose closures lie in $E_{r}$ be the union of those and put g,= max $\{f_{1},\cdot\cdot,f_{r}\}.$ Our construction of $\Lambda$ shows that $$ \operatorname{vol}\left(E_{r}\right)\leq\Lambda f_{r}\leq\Lambda g_{r}\leq\operatorname{vol}\ W. $$ (4) As r→, vol $\scriptstyle(t_{s})\, arrow$ vol $(W),$ and $$ \Lambda g_{r}=\left\{\epsilon_{r}\,d m\to m(W)\right\} $$ (5) Thus $m(W)=\operatorname{vol}\operatorname{t}$ by the monotone convergence theorem, since $g_{t}(x)\to\chi_{t r}(x)$ for all $x\in R^{4}$ (W) for every open cell $W,$ and since every $k{\mathrm{:}}$ -cell is the intersection of a decreasing sequence of open $k.$ cells, we obtain (a) The proofs of (c),(a), and (e) will use the following observation:If $\lambda$ . is a positive Borel measure on ${\boldsymbol{R}}^{k}$ and $\lambda(E)=m(E)$ for all boxes ${\boldsymbol{E}},$ then the same equality holds for all open sets $E,$ by property 2.19(d), and therefore for all Borel sets E, since $\lambda$ and m are regular (Theorem 2.18). To prove (c),fix $x\in R^{n}$ and define $\lambda(E)=m(E+x).$ It is clear that Ais then a measure; by (a) $\textstyle E.$ . The same equality holds for every $\boldsymbol{E}$ ∈ D, because of (b) for all Borel sets $\lambda(E)=m(E)$ for all boxes, hence $m(E+x)=m(E)$ $c=\mu(Q_{0}).$ Suppose next that ${\boldsymbol{\mu}}$ satisfies the hypotheses of (d). Let $2^{-n}$ boxes that are translates Since $Q_{0}$ is the union of $2^{n k}$ disjoint $Q_{0}$ be a 1-box, put of each other, we have $$ 2^{n k}\mu(Q)=\mu(Q_{0})=c m(Q_{0})=c\cdot2^{n k}m(Q) $$ for every $2^{-n}$ -box Q. Property 2.19(d) implies now that $\mu(E)=c m(E)$ for all open sets $E\subset R^{k}.$ This proves (d) To prove (e), let $T\colon R^{k}\to R^{k}$ be linear. If the range of ${\mathbf{}}T$ is a subspace Y $\Delta(T)=0.$ of lower dimension, then ${\boldsymbol{R}}^{k}$ onto ${\boldsymbol{R}}^{k}$ whose inverse is also linear. Thus ${\mathbf{}}T$ is a $m(Y)=0$ and the desired conclusion holds with ${\mathbf{}}T$ is a In the other case, elementary linear algebra tells us that one-to-one map of ${\boldsymbol{E}},$ homeomorphism of ${\boldsymbol{R}}^{k}$ onto and we can therefore define a positive Borel measure ${\boldsymbol{\mu}}$ is a Borel set for every Borel set by $R^{k},$ so that $\scriptstyle T(E)$ l on ${\boldsymbol{R}}^{k}$ $$ \mu(E)=m(T(E)). $$ The linearity of ${\boldsymbol{T}},$ combined with the translation-invariance of $m,$ gives A(E +x)= m(T(E + x)) = m(T(E) + Tx) = m(T(E)) = A(E). Thus L ${\boldsymbol{\mu}}$ is translation-invariant, and the first assertion of $\mathbf{\Psi}({\boldsymbol{e}})$ follows from (d), first for Borel sets ${\boldsymbol{E}},$ , then for all $E\in\mathfrak{M}\ .{\mathfrak{M}}\left(b\right).$ and To find $\Delta(T),$ we merely need to know $m(T(E))/m(E)$ for one set $\boldsymbol{E}$ with $0<m(E)<\infty$ 、If ${\mathbf{}}T$ is a rotation, let $\boldsymbol{E}$ be the unit ball of $R^{k},$ ; then $T(E)=E,$ // $\Delta(T)=1.$PosrTIVE BOREL MEASURES ${\mathsf{s3}}$ in place of to the measurable subsets of ${\boldsymbol{E}},$ is the Lebesgue measure on $R^{k}{}_{;}$ it is customary to write $L^{1}(R^{k})$ 2.21 Remarks Ifr ${\mathfrak{m}}$ $\scriptstyle I(m)$ If $\boldsymbol{E}$ is a Lebesgue measurable subset of $R^{k},$ , and if ${\mathfrak{m}}\,$ is restricted , a new measure space is obtained in an obvious fashion. The phrase ${}^{\circ}f\in L^{1}$ on $E^{'\;\;\mathrm{or}\;\;^{\mathrm{*}{}}}f\in L^{1}(E)^{\gamma}$ is used to indicate that j $\boldsymbol{\f}$ f is integrable on this measure space. If $k=1,$ if ${\mathbf I}$ is any of the sets (a,b), (α,b],[a,b), [a,b], and iff e $\scriptstyle U(t)_{k}$ it is customary to write $$ \bigcap_{a}^{\prime b}f(x)\;d x\quad{\textrm{i n p l a c e}}\;\bigcup_{}f\;d m. $$ Since the Lebesgue measure of any single point is ${\boldsymbol{0}},$ it makes no difference over which of these four sets the integral is extended. Everything learned about integration in elementary Calculus courses is still useful in the present context, for if is a continuous complex function on [a,b], then the Riemann integral of f and the Lebesgue integral of f over [a, b] coincide This is obvious from our construction $\operatorname{iff}(a)=f(b)=0$ andi $\operatorname{if}f(x)$ is defined to be $\mathbf{0}$ for $x<a$ and for $\scriptstyle x\,>b$ The general case follows without difficulty. Actually the same thing is true for every Riemann integrable $\boldsymbol{\f}$ on [a,b]. Since we shal have no occasion to discuss Riemann integrable functions in the sequel, we omit the proof and refer to Theorem 11.33 of [26] Two natural questions may have occurred to some readers by now: Is every Lebesgue measurable set a Borel set? Is every subset of F $R^{k}$ Lebesgue measurable? The answer is negative in both cases, even when $k=1.$ The first question can be settled by a cardinality argument which we sketch briefly. Let c be the cardinality of the continuum (the real line or, equivalently the collection of all sets of integers). We know that ${\boldsymbol{R}}^{k}$ t has a countable base (open balls with rational radii and with centers in some countable dense subset of $R^{k}),$ and that ${\mathcal{B}}_{k}$ (the collection of all Borel sets of T $R^{k}{}_{.}$ is the ${\boldsymbol{\sigma}}.$ -algebra generated by this base. It follows from this (we omit the proof) that $E=R^{\prime}$ $2^{\mathfrak{c}}$ C subsets of $\boldsymbol{E}$ is Lebesgue measur the other hand,there exist Cantor sets with ${\mathcal{R}}_{k}$ has cardinality c. On completeness of m implies that each of the $m(E)=0.$ (Exercise 5.) The able. Since $2^{c}>c,$ most subsets of $\boldsymbol{E}$ are not Borel sets. The following theorem answers the second question. 2.22 Theorem If $\scriptstyle4\,\epsilon\,R^{\prime}$ and every subset of A is Lebesgue measurable then $m(A)=0$ Corollary Every set of positive measure has nonmeasurable subsets. PR0OF We shall use the fact that $R^{1}$ is a group, relative to addition. Let $\boldsymbol{E}$ be a set that $\textstyle{\mathcal{Q}}$ D be the subgroup that consists of the rational numbers, and let contains exactly one point from each coset of $Q_{\mathbf{\delta}}Q$ in $R^{1}.$ (The assertion that54 REAL AND coMPLEX ANALYSis there is such a set is a direct application of the axiom of choice.) Then $\boldsymbol{E}$ has the following two properties. (a) ${\frac{(E+r)\cap(E+s)=\emptyset}{-}}\operatorname{if}r\in Q,s\in Q,r\neq s$ for some $\scriptstyle\tau\in Q$ (b)Every xe R lies in $\scriptstyle{E+r}$ some To prove (a), suppose $x\in(E+r)\cap(E+s).$ Then $x=y+r=z+s$ for $y\in E,\ z\in E,\ .$ y ≠ z. But $y-z=s-r\in Q,$ so that $\mathbf{\vec{y}}$ and z lie in the same coset of ${\boldsymbol{Q}},$ a contradiction To prove (b), let $\scriptstyle{y}$ be the point of $\boldsymbol{E}$ that lies in the same coset as x, put $r=x-y$ Fix $\scriptstyle*\varrho$ for the moment, and put $A_{t}=A\cap(E+t).$ By hypothesis $A_{t}$ is measurable. Let $\kappa\in A,$ be compact, let ${\boldsymbol{H}}$ be the union of the translates $K+r.$ where r ranges over $Q\cap[0,1]$ Then ${\boldsymbol{H}}$ is bounded, hence $m(H)<\infty.$ Since $K\subset E+t,$ (a) shows that the sets $\kappa+\prime$ are pairwise disjoint. Thus $m(H)=\sum_{r}m(K+r).$ But $m(K+r)=m(K).$ It follows that $m(K)=0.$ This holds for every compact $K\subset A_{t}.$ Hence $m(A_{i})=0.$ ${\boldsymbol{Q}}$ Since Q is // Finally,(b) shows that $A=\bigcup A_{t}.$ where t ranges over countable, we conclude that $m(A)=0.$ 2.23 Determinants The scale factors ${\mathrm{d}}(t)$ that occur in Theorem $2.20(e)$ can be interpreted algebraically by means of determinants Let $\{e_{1},\ldots,e_{k}\}$ If $T\colon R^{k}\to R^{k}$ is linear and $R^{k}{\mathrm{:}}$ the ith coordinate of $\scriptstyle e_{j}$ is 1 if be the standard basis for $i=j,0\operatorname{if}i\neq j.$ $$ T e_{j}=\sum_{i=1}^{k}\alpha_{i j}e_{i}\qquad(1\leq j\leq k) $$ (1) then det ${\mathbf{}}T$ is, by definition, the determinant of the matrix [T] that has $x_{i j}$ in row i and columnj We claim that $$ \Delta(T)=|\operatorname*{det}\:T|. $$ (2) If $T=T_{1}T_{2},$ it is clear that $\Delta(T)=\Delta(T_{1})\Delta(T_{2}).$ ${\boldsymbol{R}}^{k}$ is a product of finitely many linear $T_{2},$ then (2) also The multiplication theorem holds for for determinants shows therefore that if (2) holds for $T_{1}$ and ${\boldsymbol{T}}.$ Since every linear operator on operators of the following three types, it suffices to establish (2) for each of these: (I) $\{T e_{1},\ldots,\;T e_{k}\}$ is a permutation of $\{e_{1},\ldots,e_{k}\}.$ (iD Tre,= αe, $t e_{i}=e_{i}$ for i = 2,.…., k. (iII Te, = + e,, Te,= e, for i =2,..,、 Let Q be the cube consisting of all $x=(\xi_{1},\dots,\ \xi_{k})$ with O≤ $\scriptstyle{\varepsilon_{k}<1}$ for $i=1,\ldots,l$ 大. If ${\mathbf{}}T$ is of type (D), then [T] has exactly one l in each row and each column and has O in all other places. So det $T=\pm1$ Also,T(Q) = Q. So $\Delta(T)=1=\left|\,\mathrm{det}\,\,T\,\right|.$POSITIVE BOREL MEASURES ${\mathsf{S S}}$ If ${\mathbf{}}T$ is of type (II), then clearly $\Delta(T)=|\alpha|=|\operatorname*{det}\,\,T|$ If ${\mathbf{}}T$ is of type (II), then det $\scriptstyle T\;=\;1$ and $r(0)$ is the set of all points $\sum{}$ 长;e, whose coordinates satisfy 民 $$ {}_{1}\leq\xi_{2}<\xi_{1}+1,\qquad0\leq\xi_{i}<1\quad\mathrm{if}\quad i\neq2. $$ (3) If $S_{1}$ is the set of points in $\scriptstyle T(\vartheta)$ that have $\xi_{z}<\rangle$ and if ${\boldsymbol{S}}_{2}$ is the rest of $\scriptstyle T(0)$ then $$ S_{1}\cup(S_{2}-e_{2})=Q, $$ (4) and $S_{1}\cap(S_{2}-e_{2})$ is empty. Henc $$ :\Delta(T)=m(S_{1}\cup S_{2})=m(S_{1})+m(S_{2}-e_{2})= $$ $m(Q)=1,$ so that we again have $\Delta(T)=|c|$ det $T\,|$ Continuity Properties of Measurable Functions Since the continuous functions played such a prominent role in our construction of Borel measures, and of Lebesgue measure in particular, it seems reasonable to expect that there are some interesting relations between continuous functions and measurable functions. In this section we shall give two theorems of this kind. We shall assume, in both of them, that ${\boldsymbol{\mu}}$ L is a measure on a locally compact Hausdorff space $\textstyle X$ which has the properties stated in Theorem 2.14. In particular ${\boldsymbol{\mu}}$ could be Lebesgue measure on some $\textstyle R^{k}.$ 2.24 Lusin's Theorem Suppose $\int\,i s\,\,\,a$ complex measurable function on X $\mu(A)<\infty,f(x)=0$ if x生 A, and $\scriptstyle\epsilon\;>0$ Then there exists a ge C.(X) such that $$ \mu(\{x;f(x)\neq g(x)\})<\epsilon. $$ (1) Furthermore, we may arrange $i t$ so that $$ \operatorname*{sup}_{x\circ x}|g(x)|\leq\operatorname*{sup}_{x\circ x}|f(x)|. $$ (2) $\{s_{n}\}:\mathbb{Q}f,$ PROOF Assume first that $0\leq f<1$ and that $\scriptstyle A$ is compact. Attach a sequence $t_{n}=s_{n}-s_{n-1}$ and as in the proof of Theorem 1.17, and put $t_{1}=s_{1}$ and for n = 2,3,4,.……. Then $2^{n}t_{n}$ is the characteristic function of a set $\scriptstyle I_{\bullet}=A,$ $$ f(x)=\sum_{n=1}^{\infty}t_{n}(x)\qquad(x\in X). $$ (3) Fix an open set ${\mathbf{}}V$ such that $A\subset V$ and $\bar{V}$ is compact. There are compact sets $K_{n}$ and open sets ${\mathit{V}}_{n}$ such that $K_{n}\subset V_{n}\subset V_{n}\subset V$ and $\mu(V_{n}-K_{n})<2^{-n}\epsilon.$ Define By Urysohn's lemma, there are functions $h_{n}$ such that $K_{n}\subset h_{n}\subset V_{n}.$ $$ g(x)=\sum_{n=1}^{\infty}2^{-n}h_{n}(x)\qquad(x\in X). $$ (4) This series converges uniformly on $X,$ so $\scriptstyle{\mathcal{G}}$ is continuous. Also, the support of $\scriptstyle{\mathcal{G}}$ lies in ${\bar{V}}.$ Since $2^{-n}h_{n}(x)=t_{n}(x)$ except in $V_{n}-K_{n};$ we have $g(x)=f(x)$56 REAL AND coMPLEX ANALYSis holds if $\scriptstyle A$ is compact and $0\leq f\leq1.$ and this latter set has measure less than e. Thus(1) except in $\bigcup{(V_{n}-K_{n})},$ It follows that(1) holds if $\scriptstyle A$ is compact and $\boldsymbol{\f}$ is a bounded measurable $\{x;|f(x)|>n\}$ function. The compactness of $\scriptstyle A$ is easily removed, for if $\mu(A)<\infty$ then $\scriptstyle A$ $\boldsymbol{\f}$ contains a compact set ,then $\bigcap B_{n}={\mathcal{D}}.$ so $\mu(B_{n})\to0,$ by Theorem 1.19(e). Since $\scriptstyle{B_{s}=e}$ tive number. Next, if $\boldsymbol{\mathsf{f}}$ $\textstyle K$ with $\mu(A-K)$ smaller than any preassigned posi is a complex measurable function and if coincides with the bounded function $(1-\chi_{B_{n}})\cdot f$ except on $B_{n},\,(1)$ follows in the general case $\varphi(z)=R z/|z|$ plane onto the disc of radius ${\boldsymbol{R}}.$ $\{\,|f(x)|\,:\,x\in X\}$ and define $\varphi(z)=z\,{\mathrm{~if~}}\;|z\,|\leq R$ then ${\mathfrak{g}}_{1}$ Finally,let R = sup if $|z|>R.$ Then $\varphi$ is a continuous mapping of the complex $g_{1}=\varphi\circ g,$ // If $\scriptstyle{\mathcal{G}}$ satisfies (1) and satisfies (1) and (2) Corollary Assume that the hypotheses of Lusin's theorem are satisfied and that $|f|\leq1.$ Then there is a sequence $\{g_{n}\}$ such that $g_{n}\in C_{c}(X),|\,g_{n}|\leq1,$ and $$ f(x)=\operatorname*{lim}_{n\to\infty}g_{n}(x)\qquad{\mathrm{a.e.}} $$ (5) with many of the sets $E_{n}$ PRoOF The theorem implies that to each $\mu(E_{n})\leq2^{-n}.$ where $r_{\mathit{l}}$ there corresponds a $g_{n}\in C_{c}(X),$ // $f(x)\neq g_{n}(x).$ For almost every (Theorem 1.41). For any such ${\mathcal{X}},$ is the set of all $\scriptstyle{X}$ at which $|g_{s}|\leq1,$ such that $E_{n}$ $\scriptstyle{\mathcal{X}}$ it is then true that $\scriptstyle{\mathcal{X}}$ lies in at most finitely gA(x) for all large enough ${\mathfrak{n}}\,$ it follows ${\mathrm{that}}\,f(x)=$ This gives (5) $\scriptstyle\epsilon\;>0,$ 2.25 The Vitali-Carathéodory Theorem Suppose $X$ such that u $\leq f\leq\iota$ D,u is upper Then there exist functions u and t on $f\in L^{1}(\mu),f$ is real-valued, and semicontinuous and bounded above,v is lower semicontinuous and bounded below, and $$ \bigcap_{x}(v-u)\;d\mu<\epsilon. $$ (1) PR0OF Assume first that $f\geq0$ and that $\boldsymbol{\f}$ is not identically O. Since $s_{n},f$ is the sum $\boldsymbol{\mathit{f}}$ is the pointwise limit of an increasing sequence of simple functions of the simple functions $t_{n}=s_{n}-s_{n-1}$ (taking $s_{0}=0).$ and since $t_{n}$ is a linear combination of characteristic functions, we see that there are measurable sets $E_{i}$ (not necessarily disjoint) and constants $\scriptstyle{\varepsilon_{1}\;\leq\;0}$ such that $$ f(x)=\sum_{i=1}^{\infty}c_{i}\,\gamma_{E_{i}}(x)\qquad(x\in X). $$ (2)POSITIVE BOREL MEASURES ${\mathsf{S I I}}$ Since $$ \bigcap_{x}f\,d\mu=\ \stackrel{\circ}{_\mathrm{=}}\,_{i=1}^{\circ}c_{i}\,\mu(E_{i}), $$ (3) the series in (3) converges. There are compact sets $K_{i}$ and open sets ${\mathit{V}}_{i}$ such that $K_{i}\subset E_{i}\subset V_{i}$ and $$ c_{i}\mu(V_{i}-K_{i})<2^{-i-1}\epsilon\qquad(i=1,2,\,3,\ldots). $$ (4) Put $$ \nu=\sum_{i=1}^{\infty}c_{i}\,\gamma_{\nu_{i}},\qquad u=\sum_{i=1}^{N}c_{i}\,\chi_{K_{i}}, $$ (5) where ${\mathbf{}}N$ is chosen so that $$ \sum_{N+1}^{\infty}c_{i}\,\mu(E_{i})<{\frac{\epsilon}{2}}. $$ (6) Then v is lower semicontinuous, L $\boldsymbol{\ u}$ is upper semicontinuous, u $\leq f\leq v,$ and $$ \begin{array}{c}{{v-u=\sum_{i=1}^{N}c_{i}(\chi_{\nu_{i}}-\chi_{K})+\sum_{N+1}^{\infty}c_{i}\chi_{\nu_{i}}}}\\ {{\leq\sum_{i=1}^{\infty}c_{i}(\chi_{\nu_{i}}-\chi_{K})+\sum_{N+1}^{\infty}c_{i}\chi_{E_{i}}}}\end{array} $$ so that (4) and (6) imply (1) and ${\boldsymbol{v}}_{2}$ In the general case, write $f=f^{+}-f^{-}$ ,attach $u_{1}$ and $v_{\mathrm{I}}$ to ft, attach $u_{2}$ tof", as above, and put $u=u_{1}-v_{2}$ $v=v_{1}-u_{2}$ .Since $\scriptstyle-\varepsilon_{2}$ is upper semicontinuous and since the sum of two upper semicontinuous functions is proof of this as an exercise), $\boldsymbol{\u}_{}^{}$ upper semicontinuous (similarly for lower semicontinuous; we leave the / and v have the desired properties. Exercises ments: (b) $\underline{{\mathbb{F}_{1}}}f_{1}$ be a sequence of real nonnegative functions on $f_{1}+f_{2}$ is per semicontinuous 1 Let $\{f_{n}\}$ $R^{1},$ and consider the following four state- (a Iff ${\textrm{i}}\operatorname{and}_{3}f_{2}$ are upper semicontinuous, then $f_{1}$ + f, is lower semicontinuous. and , are lower semicontinuous, then (c) If each f, is upper semicontinuous, then ZYJAis upper semicontinuous (a) $|{\mathrm{freach}}f_{n}$ is lower semicontinuous, then $\textstyle\sum_{i}^{\infty}f_{n}$ is lower semicontinuous. is omitted ? Is the truth of the statements affected if Show that three of these are true and that one is false. What happens if the word “nonnegative” is replaced by a general topological space? ${\boldsymbol{R}}^{1}$ 2 Letf be an arbitrary complex function on $R^{1},$ and define p $$ (x,\,\delta)=\operatorname*{sup}\,\{\,|\,f(s)-f(t)\,|\,:s,\,t\in(x-\delta,\,x+\delta)\}, $$ $$ \varphi(x)=\operatorname*{inf}{\big\{}\varphi(x,\,\delta)\!:\,\delta>0{\big\}}. $$ Prove that $\varphi$ is upper semicontinuous, that fis continuous at a point $\textstyle X$ if and only if $\varphi(x)=0,$ and hence that the set of points of continuity of an arbitrary complex function is a ${\cal G}_{\delta}$ $R^{1}.$ Formulate and prove an analogous statement for general topological spaces in place of$58$ REAL AND cOMPLEX ANALYSIS 3 Let $\scriptstyle{\mathcal{X}}$ be a metric space, with metric ${\boldsymbol{\rho}}.$ For any nonempty $E\subset X,$ define $$ \rho_{E}(x)=\operatorname*{inf}\left\{\rho(x,y)\colon y\in E\right\}. $$ Show that $\rho_{E}$ is a uniformly continuous function on $X.$ If $\scriptstyle A\quad}$ and $\bar{\boldsymbol{B}}$ are disjoint nonempty closed subsets of $X,$ examine the relevance of the function $$ f(x)={\frac{\rho_{A}(x)}{\rho_{A}(x)+\rho_{B}(x)}} $$ to'Urysohn's lemma. (b) If $E\in{\mathfrak{M}}_{F}$ , then E $\mathbf{\hat{\Pi}}=N$ 4 Examine the proof of the Riesz theorem and prove the following two statements: ${\cal{V}}_{2}$ are disjoint open sets, then $\mu(E_{1}\cup E_{2})=$ (a) If $E_{1}\subset V_{1}$ and $E_{2}\subset V_{2}\,,$ where V and is a disjoint countable collection of $\mu(E_{1})+\mu(E_{2}),\mathrm{eve}$ n if ${\boldsymbol{E}}_{1}$ and ${\boldsymbol{E}}_{2}$ are not in のt $\{K_{i}\}$ u K u K,0…, where compact sets and $\mu(N)=0.$ In Exercises $\scriptstyle{\hat{S}}$ to 8, ${\bar{m}}\,$ stands for Lebesgue measure on $R^{1}$ $\scriptstyle{\bar{5}}$ Let $\boldsymbol{E}$ be Cantor's familiar“middle thirds”set. Show that $m(E)=0,$ even though $\bar{E}$ and $R^{1}$ have the same cardinality. 6 Construct a totally disconnected compact set $\scriptstyle{\kappa\cdot{\kappa}^{\prime}}$ such that $m(K)>0.$ $\mathbf{\nabla}(K)$ is to have no con- nected subset consisting of more than one point.) $v\leq\chi_{K},$ show that actually $v\leq0.$ Hence $\chi_{\scriptscriptstyle K}$ cannot be approx- If v is lower semicontinuous and imated from below by lower semicontinuous functions, in the sense of the Vitali-Carathéodory theorem. T If $0<\epsilon<1.$ construct an open set $\scriptstyle E\in[0,\,1]$ which is dense in [0, 1], such that $m(E)=\epsilon.$ (To say that A is dense in $\bar{\boldsymbol{B}}$ means that the closure of $\scriptstyle A\quad\quad A\quad\quad$ contains ${\boldsymbol{B}}\cdot$ 8 Construct a Borel set $\scriptstyle t\in\mathbb{R}^{\prime}$ such that $$ 0<m(E\cap I)<m(I) $$ for every nonempty segment $\scriptstyle{I_{\circ}}$ I. Is it possible to have $m(E)<\infty$ for such a set ?? 9 Construct a sequence of continuous function f,on [0, 1] such that $0\leq f_{n}\leq$ 1, such that $$ \operatorname*{lim}_{n arrow\infty}\ \;\prod_{0}^{1}f_{n}(x)\;d x=0, $$ but such that the sequence $\{f_{n}(x)\}$ converges for no $x\in[0,1].$ $10$ If $\{f_{n}\}$ is a sequence of continuous functions on $\mathbb{D},$ $1\!\!\!\!1\!\!\!\rfloor$ such that O $\leq f_{n}\leq1$ and such that fAx)→0 as n→o,for every xe[0, 1], then $$ \operatorname*{lim}_{n arrow\infty}\ \; \vert{\bf{\alpha}}^{1}f_{n}(x)\;d x=0. $$ Try to prove this without using any measure theory or any theorems about Lebesgue integration nice proof was given by W. F (This is to impress you with the power of the Lebesgue integral $\mathrm{A}$ pps 35-30.*1953 Eberlein in Communications on Pure and Applied Mathematics, vol. $\scriptstyle X,$ compare Exercise 18. Show that 11 Let p be a regular Borel measure on a compact Hausdorff space ${\cal K}$ be the intersection of all compact $\mu(K)=1$ bu $\textstyle:\mu(H)<1$ for every $X{\dot{\cdot}}$ ; assume $\mu(X)=1.$ Prove that there is a compact set $\kappa={\mathfrak{X}}$ (the carrier or support of $\scriptstyle{\mathcal{K}}$ also contains some $K_{\alpha}$ Regularity of with $\mu(K_{\alpha})=1;$ proper compact subset $\boldsymbol{H}$ of $K.$ Hint: Let ${\boldsymbol{\mu}}_{}^{}\}$ such that $\textstyle K_{\alpha}$ is needed; show that every open set ${\boldsymbol{y}}$ which contains ${}_{\mu}$ $K^{\mathrm{c}}$ is the largest open set in $\textstyle X$ whose measure is O 12 Show that every compact subset of $R^{1}$ is the support of a Borel measure.POSrTIVE BOREL MEASUREs 59 13 s it true that every compact subset of $R^{1}$ is the support of a continuous function?? If not, can you describe the class of all compact sets in $R^{1}$ which are supports of continuous functions? Is your description valid in other topological spaces ? 14 Let f be a real-valued Lebesgue measurable function on ${\boldsymbol{R}}^{k}$ Prove that there exist Borel functions g and ${\boldsymbol{h}}$ such that $g(x)=h(x)$ a.e. [m], and gx) ≤f(x) ≤ A(x) for every $\scriptstyle{x\neq K^{\mathrm{{T}}}}$ 15 It is easy to guess the limits of $$ \left|\stackrel{n}{\right)}_{0}^{n}\left(1-\frac{x}{n}\right)^{n}e^{x/2}\ d x\quad\mathrm{and}\quad\left\{1+\frac{x}{n}\right\}^{n}e^{-2x}\,d x, $$ as $n\to\infty.$ Prove that your guesses are correct. 16 Why is $m(Y)=0$ in the proof of Theorem 2.20e)? 17 Define the distance between points $\scriptstyle w_{v,Y}$ ) and $(x_{2},\,y_{2})$ in the plane to be $$ |y_{1}-y_{2}|\quad{\mathrm{if~}}x_{1}=x_{2},\qquad1+|y_{1}-y_{2}|\quad{\mathrm{if~}}x_{1}\neq x_{2} $$ Show that this is indeed a metric, and that the resulting metric space $\scriptstyle{\mathcal{X}}$ is locally compact. $\operatorname{If}f\in C_{c}(X),$ let $X_{1},\cdot\cdot\cdot,\ X_{n}$ be those values of $\scriptstyle{\dot{\boldsymbol{x}}}$ for ${\mathrm{which}}\,f(x,y)\neq0$ for at least one y (there are only finitely many such x!), and define $$ \Lambda f=\sum_{j=1}^{n} \{\sum_{-\infty}^{\infty}f(x_{j},\,y)\,\,d y. $$ Let ${}_{\!\mu}$ be the measure associated with this A by Theorem 2.14. I $\scriptstyle{\vec{E}}$ is the x-axis, show that $\mu(E)=\infty$ although $\mu(K)=0$ for every compact $\scriptstyle{\kappa\in E}$ 18 This exercise requires more set-theoretic skill than the preceding ones. Let $\scriptstyle{\mathcal{X}}$ be a well-ordered uncountable set which has a last element $\omega_{1},$ such that every predecessor of ${\boldsymbol{\omega}}_{1}$ has at most countably many predecessors. (" Construction": Take any well-ordered set which has elements with uncountably α E many predecessors, and let ${\boldsymbol{\omega}}_{1}$ be the first of these; ${\boldsymbol{\sigma}}_{1}$ is called the first uncountable ordinal.) For open if it is a $P_{\alpha}$ $X,$ let $P_{a}[S_{a}]$ be the set of all predecessors (successors) of α, and call a subset of $\scriptstyle{\mathcal{X}}$ or an ${\boldsymbol{S}}_{\boldsymbol{\beta}}$ or a $P_{\alpha}\cap S_{\beta}$ or a union of such sets. Prove that $\scriptstyle{\mathcal{X}}$ is then a compact Hausdorff space.(Hint: No well-ordered set contains an infinite decreasing sequence.) Prove that the complement of the point ${\boldsymbol{\omega}}_{1}$ is an open set which is not ${\boldsymbol{\sigma}}.$ T -compact. Prove that to every fe C(X)there corresponds an $\textstyle x\neq\omega_{1}$ such that fis constant on $S_{\alpha}.$ Prove that the intersection of every countable collection $\{K_{n}\}$ of uncountable compact subsets of $\scriptstyle{\mathcal{X}}$ is uncountable. (Hint: Consider limits of increasing countable sequences in $\scriptstyle{\mathcal{X}}$ which intersect each ${\mathit{K}}_{n}$ in infinitely many points.) $\varepsilon\cup\{\omega_{1}\}$ or $\:F\cup\;\{\omega_{1}\}\:$ contains an Let Ot be the collection of all $E\subset X$ such that either in the second case, define $\lambda(E)=0.$ Prove uncountable compact set; in the first case, define $\lambda(E)=1\,;$ that D is a g-algebra which contains all Borel sets in $X,$ that $\lambda$ is a measure on D which is not regular (every neighborhood of ${\boldsymbol{\omega}}_{1}$ has measure ${\mathfrak{I}}_{}^{},$ and that $$ f(\omega_{1})=\left|\vdots_{x}(x_{1})\right|_{x}=\left|\ 2_{x}- |\ldots\right|^{r} $$ for every fe C(X). Describe the regular ${}^{\mu}$ which Theorem 2.14 associates with this inear functional. 19 Go through the proof of Theorem 2.14, assuming $\scriptstyle{X}$ to be compact (or even compact metric) rather than just locally compact, and see what simplifications you can find. J8,(x) 20 Find continuous functions $\oint_{m}\int_{m}$ is not in $f_{n}\colon$ [0,1]→[0, ) such that $f_{n}(x)\to0$ for all xe [0,1] as n→ $d x{ arrow}0,$ but sup。 $\boldsymbol{L}$ (This shows that the conclusion of the dominated convergence theorem may hold even when part of its hypothesis is violated.) 21 If $\scriptstyle{\mathcal{X}}$ is compact and f: $x\to(-\infty.$ oo) is upper semicontinuous, prove that f attains its maximum at some point of $X.$60 REAL AND cOMPLEX ANALYSis $22$ Suppose that $\scriptstyle{\mathcal{X}}$ is a metric space, with metric ${\mathit{d}},$ and $\operatorname{har}1/z X\to[0,$ c] is lower semicontinuous, $f(p)<\infty$ for at least one $p\in X.$ For $n=1,$ $2,3,\ldots,x\in X,$ define $$ g_{n}(x)=\operatorname*{inf}\left\{f(p)+n d(x,\,p)\colon p\in X\right\} $$ and prove that 6) $\left|\,g_{n}(x)-g_{n}(y)\,\right|\leq n d(x,\,y),$ (i) $0\leq g_{1}\leq g_{2}\leq\cdots\leq f,$ $n\to\infty,{\mathrm{for~all~}}x\in X.$ ii gAx)→(Xx) as verse is almost trivial) Thus fis the pointwise limit of an increasing sequence of continuous functions.(Note that the con- to 23 Suppose $\mathcal{V}$ is open in ${\boldsymbol{R}}^{k}$ and $\mu$ is a finite positive Borel measure on $R^{k},$ Is the function that sends x $\mu(V+x)$ necessarily continuous? lower semicontinuous? upper semicontinuous? 24 A step function is, by definition, a finite linear combination of characteristic functions of bounded intervals in $R^{1}$ . Assume f∈ $L^{1}(R^{1}),$ and prove that there is a sequence $\{g_{n}\}$ of step functions so that $$ \operatorname*{lim}_{n arrow\infty}\left.\right\}_{-\infty}^{\infty}|f(x)-g_{n}(x)|\ d{x}=0. $$ 25i) Find the smallest constant $\bar{C}$ such that $$ \log\,(1+e^{\iota})<c+t\quad\quad(0<t<\infty). $$ Gi) Does $$ \operatorname*{lim}_{n\to\infty}{\frac{1}{n}}\prod_{0}^{1}\log\left\{1+e^{n f(x)}\right\}\,d x $$ exist for every realfe $L^{1}{\mathcal{V}}$ If it exists, what is it?