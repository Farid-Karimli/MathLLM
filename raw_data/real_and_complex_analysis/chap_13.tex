CHAPTER IHIRITEDN APPROXIMATIONS BY RATIONAL FUNCTIONS Preparation 13.1 The Riemann Sphere It is often convenient in the study of holomorphic functions to compactify the complex plane by the adjunction of a new point called ${\mathcal{O}}.$ The resulting set ${\boldsymbol{S}}^{2}$ to be open if and only if it is the union of discs $D(\infty\,;r)=D^{\prime}(\infty\,;r)\cup\{\infty\},$ l and $\scriptstyle(\infty)$ is ${\boldsymbol{S}}^{2}$ (the Riemann sphere, the union of $R^{2}$ topologized in the following manner. For any $|z|>r,$ put let $D(\infty;r)$ be the set of all and $\scriptstyle r\gg0,$ complex numbers z such that where the a's are arbitrary points of $S^{2}$ $D(a;r),$ declare a subset of and the r's are arbitrary positive numbers On $S^{2}-\{\infty\},$ this gives of course the ordinary topology of the plane. It is easy to see that $S^{2}$ is homeomorphic to a sphere (hence the notation). In fact, a homeo- morphism $\varphi$ of ${\boldsymbol{S}}^{2}$ onto the unit sphere in $R^{3}$ can be explicitly exhibited: Put $\varphi(\mathcal{C})=(0,$ 0,1), and put $$ \varphi(r e^{i\theta})=\left(\frac{2r\;\mathrm{cos}\;\theta}{r^{2}+1},\frac{2r\;\mathrm{sin}\;\theta}{r^{2}+1},\frac{r^{2}-1}{r^{2}+1}\right) $$ (1) for all complex numbers $r e^{i\theta}.$ We leave it to the reader to construct the geometric picture that goes with (1) Thus if $\boldsymbol{\mathit{f}}$ If f is holomorphic in $D(c\cdot r),$ ${\boldsymbol{0}}.$ then lim,-f(C) exists and is a complex defined in we say that f has an isolated singularity at oo. $D^{\prime}(0;\;1/r)\;\mathrm{by}\;{\tilde{f}}(z)=f(1/z),$ has at The nature of this singularity is the same as that which the function ${\tilde{f}},$ is bounded in $D(\infty;r),$ number (as we see if we apply Theorem 10.20 to 九, we define $\scriptstyle I(\infty)$ to be this limit, and we thus obtain a function in $D(\infty;r)$ which we call holomorphic: note that this is defined in terms of the behavior of f $\tilde{f}$ near O,and not in terms of differentiability of f at ${\mathcal{O}}$ 266APPROxIMA TIONS BY RATIONAL FUNCTIONs 267 If fhas a pole of order ${\mathfrak{m}}\,$ at O, then fis said to have a pole of order ${\mathfrak{m}}\,$ at oo the principal part of f at oo is then an ordinary polynomial of degree m (compare Theorem 10.21), and if we subtract this polynomial from $f,$ we obtain a function with a removable singularity at oo. Finally, iff has an essential singularity at O, then f is said to have an essential singularity at oo. For instance, every entire function which is not a polynomial has an essential singularity at $\mathbb{Q}.$ Later in this chapter we shall encounter the condition ${\mathcal{S}}^{2}-\Omega$ is connected,' where $\Omega$ is an open set in the plane. Note that this is not equivalent to the condition “the complement of Q elative to the plane is connected.”For example, if $\Omega$ consists of all complex $z=x+i y$ with $0<y<1,$ , the complement of $\Omega$ rela- tive to the plane has two components, but $\scriptstyle S^{2}\,{\mathrm{-(}}$ is connected. 13.2 Rational Functions A rational function $\boldsymbol{\f}$ is, by definition, a quotient of two polynomials ${\mathbf{}}P$ and $Q:f=P/Q$ It follows from Theorem 10.25 that every noncon stant polynomial is a product of factors of degree 1. We may assume that ${\mathbf{}}P$ and ${\boldsymbol{Q}}$ have no such factors in common. Then f has a pole at each zero of $\textstyle{\mathcal{Q}}$ (the pole of $\boldsymbol{\mathit{f}}$ has the same order as the zero of Q). If we subtract the corresponding prin- cipal parts,we obtain a rational function whose only singularity is at $\mathbb{Z}\mathbb{O}$ and which is therefore a polynomial. Every rational function $\scriptstyle f\;=\;P/Q$ has thus a representation of the form $$ f(z)=A_{0}(z)+\sum_{j=1}^{k}A_{j}(z-a_{j})^{-1}) $$ (1) where $A_{0},\,A_{1},\,\ldots,\,A_{k}$ are polynomials, $Q;(1)$ is called the partial fractions decomposition $a_{1},$ ${\boldsymbol{a}}_{k}$ are the distinct zeros of $A_{1},\ldots,A_{k}$ have no constant term, and O $\textstyle{\mathfrak{f}}f.$ We turn to some topological considerations. We know that every open set in the plane is a countable union of compact sets (closed discs,for instance) However, it will be convenient to have some additional properties satisfied by these compact sets: 13.3 Theorem Every open set $\Omega$ in the plane is the union of a sequence {K,} $n=1,2,3,\ldots,$ of compact sets such that (a)K, lies in the interior of K.+1,for $n=1,2,3,\ldots.$ (b)Every compact subset of $\Omega$ lies in some $K_{n}.$ 2, (c)Every component of $S^{2}-K_{s}$ contains $\bar{a}$ component of $S^{2}-\Omega,f o r\ n=1,$ 3,.. Property Cc) is, roughly speaking, that $K_{n}$ , has no holes except those which are forced upon it by the holes in Q2. Note that $\Omega$ is not assumed to be connected. The interior of a set $\boldsymbol{E}$ is, by definition, the largest open subset of $\textstyle E.$268 REAL AND coMPLEX ANALYSIs PRoOF For $n=1,2,3,\ldots,\operatorname{put}$ $$ V_{n}=D(\infty;n)\cup\bigcup_{a\neq\Omega}D\!\left(a;\frac{1}{n}\right) $$ (1) and put $K_{n}=S^{2}-\,V_{n}.$ [Of course, $a\neq\infty$ in (1).] Then $K_{n}$ 1 $\varepsilon\in K_{s}$ and is a closed and Hence $\Omega$ bounded (hence compact) subset of $\Omega,$ and $\Omega=\bigcup K_{n}$ This gives (a) $\Omega,$ then $r=n^{-1}-(n+1)^{-1},$ one verifies easily that $D(z;r)\subset K_{n+1}.$ is the union of the interiors for some $N.$ hence $K\subset K_{N}.$ is a compact subset of $W_{n}\ \mathrm{of}\ K_{n}.$ If $K$ $K\subset W_{1}\ \cup\ \cdots\ \cup\ W_{N}$ ponent of Finally, each of the discs in $\mathbf{(1)}$ intersects $S^{2}-\Omega;$ each disc is connected; no com // hence each component of ${\mathit{V}}_{n}$ intersects $S^{2}-\Omega;$ since $V_{n}\ {\ {\ {S}}^{2}}-\Omega,$ $\scriptstyle S^{2}\,-\,\Omega$ can intersect two components of ${\mathit{V}}_{n}$ . This gives Cc) 13.4 Sets of Oriented Intervals Let $\Phi$ be a finite collection of oriented intervals in the plane. For each point ${\boldsymbol{p}},$ let m-(p[mA(p)] be the number of members of $\Phi$ that is have initial point [end point] ${\boldsymbol{p}}.$ p. If $m_{I}(p)=m_{E}(p)$ for every ${\boldsymbol{p}},$ we shall say that $\mathbf{\Phi}\Phi$ balanced If $\Phi$ is balanced (and nonempty), the following construction can be carried out. $\textstyle\cdot\cdot\cdot\gamma_{k}$ Pick $\gamma_{1}=\left[a_{0}\,,\,a_{1}\right]\in\Phi$ Assume $k\geq1.$ and assume that distinct members for $1\leq i\leq k.$ If $a_{k}=a_{0}\,,$ contains at least one other interval, say $\gamma_{k+1},$ whose initial point is $a_{k}\,.$ Since $\mathbf{\Phi}\Phi$ $\gamma_{1},$ is of end point, then only $r-1$ have been chosen in such a way that $\gamma_{i}=[a_{i-1},$ $a_{i}{\Big]}$ $\Phi$ is balanced, o ${\boldsymbol{a}}_{k}$ as $\Phi$ stop. I $a_{k}\neq a_{0}\,,$ and if precisely ${}^{T}$ of the intervals $\gamma_{1},\,\cdot\cdot\cdot,\,\gamma_{k}$ have of them have ${\boldsymbol{a}}_{k}$ a, as initial point; since finite, we must return to ${\boldsymbol{a}}_{0}$ eventually, say at the nth step. Then $\gamma_{1},\cdot\cdot\cdot,\gamma_{n}$ join (in this order) to form a closed path The remaining members of $\Phi$ still form a balanced collection to which the above construction can be applied. It follows that the members of $\Phi$ can be so numbered that they form finitely many closed paths. The sum of these paths is a cycle. The following conclusion is thus reached. $I f\Phi=\{\gamma_{1},\ldots,\gamma_{N}\}$ is a balanced collection of oriented intervals, and i $$ \Gamma=\gamma_{1}\div\cdot\cdot\mp\ \gamma_{N} $$ then ${\Gamma}$ T is a cycle 13.5 Theorem If ${\cal K}\,\,\,\,\,\,\,\,{\cal K}$ is a compact subset of a plane open set $\Omega$ (≠0), then there is a cycle $\boldsymbol{\Gamma}$ in $\alpha-\kappa$ such that the Cauchy formula $$ f(z)={\frac{1}{2\pi i}}\prod_{\Gamma}^{s}{\frac{f(\zeta)}{\zeta-z}}\;d\zeta $$ (1) holds for every fe H(Q) and for every z e K PROOF Since $\textstyle K$ is compact and $\textstyle K$ to any point outside $\Omega$ is at least 2n,. Construct distance from any point of $\Omega$ is open, there exists an $\scriptstyle n\,>0$ such that theAPPROXIMATiONs BY RATIONAL FUNCTiONs 269 a grid of horizontal and vertical lines in the plane, such that the distance If oriented interval between any two adjacent horizontal lines is and $a_{r}+b$ is one of its vertices, let $\gamma_{r k}$ be the $a_{r}$ formed by this grid and which intersect $K.$ $\eta,$ 、and likewise for the vertical which are lines. Let is the center of ${\mathcal{Q}}_{r}$ be those squares (closed 2-cells) of edge $\textstyle\eta$ 1. $Q_{1},\,\ldots,\,Q_{m}$ Then $Q_{r}\subset\Omega\operatorname{for}r=1,\ldots,m$ $$ \gamma_{r k}=[a_{r}+i^{k}b,\,a_{r}+i^{k+1}b] $$ (2) and define $$ \partial Q_{r}=\gamma_{r1}\stackrel{\star}{\+}\gamma_{r2}\stackrel{\star}{\downarrow}\gamma_{r3}\stackrel{\uparrow}{\downarrow}\gamma_{r4}\qquad(r=1,\dots,m). $$ (3) ${\mathrm{It}}$ is then easy to check for example, as a special case of Theorem 10.37, or by means of Theorems 10.11 and 10.40) that $$ \mathrm{Ind}_{\mathrm{sQ}},(x)= \{{\frac{1}{0}}\quad\mathrm{~if~a~is~in~the~interior~of~}}\scriptstyle\mathrm{~of~} $$ $Q_{r}$ (4) Let $\Sigma_{}^{}$ be the collection of al $\gamma_{r k}\left(1\leq r\leq m,\,1\leq k\leq4\right).$ It is clear that $\Sigma$ is balanced. Remove those members of $\Sigma$ whose opposites (see Sec. 10.8) also belong to $\Sigma_{}^{}$ . Let $\Phi$ be the collection of the remaining members of $\Sigma.$ Then $\Phi$ is balanced. Let $\boldsymbol{E}$ lies intersect $K.$ Hence $\Sigma_{}^{}$ contains two oriented intervals which If an edge ${\boldsymbol{E}}$ of some ${\mathcal{Q}}_{r}$ be the cycle constructed from $K,$ then the two squares in whose $\boldsymbol{\Gamma}$ $\Phi,$ as in Sec. 13.4 intersects boundaries are each other's opposites and whose range is $\textstyle E.$ These intervals do not occur in D. Thus ${\boldsymbol{\Gamma}}$ is a cycle in $\Omega-K.$ The construction of $\Phi$ from E shows also that $$ \mathrm{Ind}_{T}\left(x\right)=\sum_{r=1}^{m}\mathrm{Ind}_{8Q},\left(x\right) $$ (5) if c is not in the boundary of any $Q_{r}.$ Hence (4) implies $$ \mathrm{Ind_{F}\left(x\right)=} \{{\frac{1}{0}}\quad\mathrm{if~e~is~in~terior~of~some}\;Q_{r}, $$ (6) If $\varepsilon\in K.$ then $z\not\in\Gamma^{*}.$ and $\mathbb{Z}$ is a limit point of the interior of some $Q_{r}$ Since the left side of (6) is constant in each component of the complement of $\Gamma^{\Phi},(6)$ gives $$ \lceil\mathrm{nd_{F}}\left(z\right)= \{\frac{1}{0}\quad\mathrm{if~}z\in K, $$ (7) Now (1) follows from Cauchy's theorem 10.35 //270 REAL AND coMPLEX ANALYSis Runge's Theorem The main objective of this section is Theorem 13.9. We begin with a slightly different version in which the emphasis is on uniform approximation on one compact set. 13.6 Theorem Suppose ${\cal K}\,\,\,\,$ is a compact set in the plane and $\{\alpha_{j}\}$ is a set which contains one point in each component of $S^{2}-K.$ ui Q is open, $\Omega\supset K,f\in H(\Omega),$ and $\scriptstyle\epsilon\;>0,$ there exists a rational function ${\boldsymbol{R}},$ all of whose poles lie in the pre- scribed set $(x_{i}).$ such that $$ |f(z)-R(z)|<\epsilon $$ (1) for every z e K Note that $\scriptstyle{g^{2}-K}$ has at most countably many components. Note also that the preassigned point in the unbounded component of $\scriptstyle g^{2}\,-\,K$ may very well be Oo;in fact, this happens to be the most interesting choice. PROOF We consider the Banach space $\mathbb{C}(k)$ whose members are the contin- be the sub- uous complex functions on $K,$ with the supremum norm. Let $\textstyle{M}$ space of $C(K)$ which consists of the restrictions to ${\cal K}\,\,$ of those rational the closure of $\textstyle{M}.$ functions which have all their poles in $\scriptstyle(x_{i})$ The theorem asserts that f is in By Theorem 5.19(a consequence of the Hahn-Banach theorem), this is equivalent to saying that every bounded linear functional on $\scriptstyle C(k)$ which vanishes on $\textstyle{M}$ also vanishes at $f,$ and hence the Riesz represent- ation theorem(Theorem 6.19) shows that we must prove the following asser- tion: If p is a complex Borel measure on ${\boldsymbol{K}}$ K such that $$ \bigcap_{R}R\ d\mu=0 $$ (2)) for every rational function ${\boldsymbol{R}}$ with poles only in the set $\scriptstyle(x_{n}|$ and if fe H(Q2), then we also have $$ \bigcap_{K}f\,d\mu=0. $$ (3) So let us assume that ${\boldsymbol{\mu}}$ u satisfies (2). Define $$ h(z)=\int_{K}{\frac{d\mu(\zeta)}{\zeta-z}}\quad\quad(z\in S^{2}-K). $$ (4) By Theorem 10.7 (with $X=K,\,\varphi(\zeta)=\zeta),\,h\in H(S^{2}-K).$APPROXIMATiONS BY RATIONAL FUNCTIONS $271$ Let ${\mathit{V}}_{j}$ be the component of $\scriptstyle{S^{2}-K}$ which contains $\alpha_{j},$ and suppose $D(x_{j};r)\subset V_{j},\operatorname{If}x_{j}\neq\infty$ and if z is fixed in $D(\alpha_{j};r),$ then $$ {\frac{1}{\zeta-z}}=\operatorname*{lim}_{N\to\infty}\sum_{n=0}^{N}{\frac{(z-\alpha_{j})^{n}}{(\zeta-\alpha_{j})^{n+1}}} $$ (5) (2) applies. Hence $\hbar(z)=0$ for all $z\in D(x_{j};r)$ Each of the functions on the right of (S) is one to which for all uniformly for $\zeta\in K.$ z ∈ V, by the uniqueness theorem 10.18. ). This implies that $h(z)=0$ $\operatorname{P}x_{j}=\infty,(5)$ is replaced by $$ \frac{1}{\zeta-z}=-\operatorname*{lim}_{N arrow\infty}\,\sum_{n=0}^{N}\,z^{-n-1}\zeta^{n}\qquad(\zeta\in{\cal K},\,\vert z\,\vert>r), $$ (6) which implies again that $h(z)=0$ in $D(\infty;\,r),$ hence in $V_{j}.$ We have thus proved from (2) that $$ h(z)=0\qquad(z\in S^{2}-K). $$ (7) Now choose a cycle ${\Gamma}$ in $\Omega-K.$ as in Theorem 13.5, and integrate this Cauchy integral representation of $\boldsymbol{\f}$ with respect toA、An application o Fubini's theorem ((legitimate, since we are dealing with Borel measures and continuous functions on compact spaces) combined with (T), gives $$ \begin{array}{c}{{\left\{J\,d\mu=\displaystyle\int_{K}d\mu(\zeta)\displaystyle [{\frac{1}{2\pi i}}\int_{\Gamma}{\frac{f(w)}{w-\zeta}}\,d w\right]}}\\ {{={\frac{1}{2\pi i}}\displaystyle\int_{\Gamma}f(w)\,d w\displaystyle\int_{K}{\frac{d\mu(\zeta)}{w-\zeta}}}}\end{array} $$ The last equality depends on the fact that $\Gamma^{*}\subset\Omega-K,$ where h(w) = 0. // Thus (3) holds, and the proof is complete. The following special case is of particular interest. 13.7 Theorem Suppose ${\cal K}\,\,\,\,\,\,\,\,$ is a compact set in the plane $\scriptstyle{S^{2}-K}$ is connected and fe H(Q), where $\Omega$ p is some open set containing K. Then there is a sequence {P,} of polynomials such that P,(z)→f(z) uniformly on $\textstyle K$ PR0OF Since $\scriptstyle{g^{2}-K}$ has now only one component, we need only one point ${\mathfrak{x}}_{j}$ to apply Theorem 13.6, and we may take $\alpha_{j}=\infty.$ // 13.8 Remark The preceding result is false for every compact $\textstyle K$ in the bounded component $V.$ Choose α∈ ${\cal{V}},$ is not connected. For in that case $\scriptstyle{S^{2}-K}$ has a plane such that $\scriptstyle{g^{2}-K}$ put $f(z)=(z-\alpha)^{-1},$ and put$272$ REAL AND cOMPLEX ANALYSIS $m=\operatorname*{max}\,\{\,|z-\alpha\,|\,\colon\ z\in K\}.$ Suppose ${\mathbf{}}P$ is a polynomial,、such that $|P(z)-f(z)|<1/m$ for all z e K. Then $$ |(z-\alpha)P(z)-1|<1\qquad(z\in K). $$ (1) In particular, (1) holds if $\mathbb{Z}$ is in the boundary of $V;$ since the closure of ${\mathbf{}}V$ is $z\in V;$ taking $z=\alpha,$ we obtain $\vdash\bigstar\bigstar1$ compact, the maximum modulus theorem shows that (1) holds for every Hence the uniform approximation is not possible. Theorem 13.6. The same argument shows that none of the ${\mathfrak{x}}_{j}$ can be dispensed with in We now apply the preceding approximation theorems to approximation in open sets. Let us emphasize that ${\cal K}\,\,$ was not assumed to be connected in Theorems 13.6 and 13.7 and that $\Omega$ will not be assumed to be connected in the theorem which follows. 13.9 Theorem Let S $\Omega$ 2 be an open set in the plane, let A be a set which has one point in each component of $S^{2}-\Omega,$ and assume fe H(Q2).Then there is a sequence $|R_{\alpha}|$ of rational functions, with poles only in A, such that $R_{n}{\xrightarrow{}}f$ uni- formly on compact subsets of $\Omega.$ In the special case in which $\scriptstyle S^{2}\,{\mathrm{-d}}$ is connected, we may take $A=\{\infty\}$ and thus obtain polynomials $P_{n}$ such that $P_{n}$ →f uniformly on compact subsets of Q. Observe that $\scriptstyle S^{2}\,{\mathrm{-Omega}}$ may have uncountably many components; for instance, we may have $S^{2}-\Omega=\{\infty\}\cup C,$ where C ${\boldsymbol{C}}$ is a Cantor set. PROOF Choose a sequence of compact sets $K_{n}$ 1n $\Omega,$ with the properties speci fied in Theorem 13.3. Fix n,for the moment. Since each component of in $\scriptstyle A$ tains a point of $A,$ contains a component of $S^{2}-\Omega,$ each component of $S^{2}-K,$ con- $S^{2}-K_{n}$ such that $R_{n}$ with poles so Theorem 13.6 gives us a rational function $$ |R_{n}(z)-f(z)|<\frac{1}{n}~~~~~(z\in K_{n}). $$ (1) all If now $\textstyle K$ is any compact set in $\Omega,$ there exists an ${\mathbf{}}N$ such that $K\subset K_{s}$ for $n\geq N.$ It follows from (1) that $$ \vert R_{n}(z)-f(z)\vert<\frac{1}{n}~~~~~(z\in K,\,n\geq N), $$ (2) which completes the proof. //APPROXIMATIONS BY RATIONAL FUNCTIONS $273$ The Mittag-Leffler Theorem Runge's theorem will now be used to prove that meromorphic functions can be constructed with arbitrarily preassigned poles 13.10 Theorem Suppose $\Omega$ is an open set in the plane $A\subset\Omega,$ A has no limit point in ${\boldsymbol{\Omega}},$ and to each αe A there are associated a positive integer $\scriptstyle m(x)$ and a rational function $$ P_{\alpha}(z)=\sum_{j=1}^{m(x)}c_{j,\alpha}(z-\alpha)^{-j}. $$ Then there exists a meromorphic function f in $\Omega,$ whose principal part at each αe A is $P_{\alpha}$ and which has no other poles in S2. PROOF We choose a sequence $K_{n}$ lies in the interior of $K_{n+1},$ every compact subset of $\Omega$ For n= 1, 2,3,… $\left\{K_{x}\right\}$ of compact sets in $\Omega,$ as in Theorem 13.3: Since $4_{n}\subset K_{n}$ and $\scriptstyle A$ and every component of $S^{2}-K_{s}$ contains a component of 3,4, ··· lies in some $K_{n},$ and $A_{n}=A\cap(K_{n}-K_{n\mp1})$ for $\textstyle n=2,$ each $A_{n}$ is a $s^{2}-\Omega$ Put $A_{1}=A\cap K_{1},$ has no limit point in $\Omega$ (hence none in $K_{n},$ finite set. Put $$ Q_{n}(z)=\sum_{\alpha\in A_{n}}P_{\alpha}(z)\qquad(n=1,\,2,\,3,\,\ldots). $$ (1) Since each $A_{n}$ is finite, each $Q_{n}$ is a rational function. The poles of $Q_{n}$ lie in $K_{n}-K_{n-1},$ for $n\geq2.$ In particular, $Q_{n}$ is holomorphic in an open set con- taining $K_{n-1}$ .It now follows from Theorem 13.6 that there exist rational such that functions $R_{n},$ all of whose poles are in $S^{2}-\Omega,$ $$ \vert R_{n}(z)-Q_{n}(z)\vert<2^{-n}\qquad\left(z\in K_{n-1}\right). $$ (2) We claim that $$ f(z)=Q_{1}(z)+\sum_{n=2}^{\infty}(Q_{n}(z)-R_{n}(z))\qquad(z\in\Omega) $$ (3) has the desired properties. Fix $N.$ On $K_{N},$ we have $$ f=Q_{1}+\sum_{n=2}^{N}\left(Q_{n}-R_{n}\right)+\sum_{N+1}^{\infty}\left(Q_{n}-R_{n}\right). $$ (4) By (2), each term in the last sum in (4) is less than $K_{N}{\mathrm{:}}$ to a function which is holomorphic in the $K_{\mathrm{{V}}};$ hence this last $2^{-n}$ on series converges uniformly on interior of $K_{N}\,.$ Since the poles of each $R_{n}$ are outside $\Omega,$ $$ f-(Q_{1}+\cdots+Q_{N}) $$$274$ REAL AND coMPLEX ANALYSis is holomorphic in the interior of $K_{N}$ .Thus $\boldsymbol{\f}$ has precisely the prescribed principal parts in the interior of $K_{N}.$ ,and hence in $\Omega,$ since ${\mathbf{}}N$ was arbitrary // Simply Connected Regions We shall now summarize some properties of simply connected regions (see Sec. 10.38)) which illustrate the important role that they play in the theory of holomorphic functions. Of these properties,(a) and (b) are what one might call internal topological properties of Q;(c) and (d) refer to the way in which $\Omega$ )1S embedded in $S^{2}{\mathrm{:}}$ properties (e) to (h) are analytic in character; O) is an algebraic statement about the ring H(2). The Riemann mapping theorem 14.8 is another very important property of simply connected regions. In fact, we shall use it to prove the implication $(\jmath)\to(a).$ 13.11 Theorem For a plane region $\Omega,$ each of the following nine conditions implies all the others. (a)Q is homeomorphic to the open unit disc ${\boldsymbol{U}}.$ (b)Q is simply connected. (c)Ind, $\scriptstyle(n )\;=\;0$ ) for every closed path y in $\Omega$ and for every $x\in S^{2}-\Omega.$ (d) $\scriptstyle S^{2}\,-\,\Omega$ is connected. can be approximated by polynomials, uniformly on compact (e)Every f ∈ $\scriptstyle{H_{\mathrm{({D}})}}$ subsets of Q2 (f)For every fe H(Q2) and every closed path $\scriptstyle{\mathcal{Y}}$ in $\Omega.$ $$ \ T_{\gamma}^{*}(z)\ d z=0. $$ (g)To everyfe $m_{\mathrm{to}}$ corresponds an $F\in H(\Omega)$ such that $F^{\prime}=f.$ (h)Iff∈ H(Q) and 1/f ∈ H(Q), there exists a $g\in H(\Omega)$ such that f= exp (g). O))Iffe H(Q) and 1/f ∈ H(Q), there exists a $\varphi\in H(\Omega)$ such that f = p-. $\boldsymbol{\f}$ The assertion of (h) is that $\boldsymbol{\mathsf{f}}$ has a“holomorphic logarithm” and $\mathbf{\nabla}(f)$ says that the (O) $\scriptstyle{\mathcal{G}}$ in $\Omega\colon$ asserts that has a“holomorphic square root” $\varphi$ p in $\Omega{\mathrm{;}}$ Cauchy theorem holds for every closed path in a simply connected region. In Chapter 16 we shall see that the monodromy theorem describes yet another characteristic property of simply connected regions. PRoOF (a) implies (b).To say that $\Omega$ is homeomorphic to $U$ means that there is a continuous one-to-one mapping $\psi$ ofS $\Omega$ onto $U$ whose inverse $\psi^{-1}$ is also continuous. If y is a closed curve in S ${\boldsymbol{\Omega}},$ with parameter interval [O,1], put $$ H(s,\,t)=\rlap/\psi^{-1}(t\psi(\gamma(s))). $$ Then $H\colon I^{2}\to\Omega$ is continuous; ${\iiint}\langle{\mathbf{S}},\rangle$ $0)=\textstyle\psi^{-1}(0),$ $\Omega$ is simply connected. $H(s,\;1)=\gamma(s);$ a constant; and $\scriptstyle{m0}$ $t)=H(1,t)$ because $\gamma(0)=\gamma(1).$ ThusAPPROxIMATIONS BY RATIONAL FUNCTIONS $\textstyle{\bigwedge}a{\overset{\underset{\mathrm{a}}{}}{\longleftrightarrow}}a$ (b) implies (c).If(b) holds and yis a closed path in Q, then $\scriptstyle{\mathcal{Y}}$ y is(by definition of“simply connected”) Q-homotopic to a constant path. Hence Cc holds, by Theorem 10.40. (c) implies (d). Assume (d is false. Then $\scriptstyle S^{2}\,{\mathrm{-=}}$ is a closed subset of ${\boldsymbol{S}}^{2}$ which is not connected. As noted in Sec. 10.1, it follows that $\scriptstyle S^{2}\,{\mathrm{-a}}$ is the union of two nonempty disjoint closed sets ${\boldsymbol{H}}$ and $K.$ K. Let ${\boldsymbol{H}}$ be the one that contains o. Let $\textstyle W$ be the complement of $\textstyle H,$ relative to the plane. Then $W=\Omega~\cup~K.$ Since ${\cal K}$ is compact, Theorem 13.5 (with f = 1) shows that there is a cycle T in $W-K=\Omega$ such that Ind- $(z)=1$ for z ∈ K. Since $K\neq{\mathcal{D}},\,(c,$ fails (d implies (e). This is part of Theorem 13.9. (e) implies $(f).$ Choose $f\in H(\Omega),$ let $\scriptstyle{\mathcal{Y}}$ y be a closed path in $\Omega,$ and choose polynomials $P_{n}$ which converge to $f,$ uniformly on $\gamma^{\star}$ . Since $\textstyle\int_{\gamma}P_{n}(z)\;d z=0$ for all ${\mathfrak{n}},$ we conclude that $(f)$ holds. and put $\mathbf{\nabla}(f)$ implies $(g).$ Assume $(f)$ holds, fix $\scriptstyle z_{\mathrm{e}}\in\Omega,$ $$ F(z)=\bigcap_{Y(z)}f(\zeta)\;d\zeta\qquad(z\in\Omega) $$ (1) ${\Gamma}_{1}$ where T(z) is any path in $\Omega$ from $\mathrm{z}_{0}$ to z $\mathbb{Z}.$ This defines a function ${\mathbf{}}F$ in $\Omega.$ . For if $\Gamma_{1}(z)$ is another path from $\mathrm{z}_{0}$ to $\mathbb{Z}$ (in Q), then ${\boldsymbol{\Gamma}}$ followed by the opposite of is a closed path in $\Omega,$ the integral of f over this closed path is O,so (1) is not affected if F(z) is replaced by $\Gamma_{1}(z).$ We now verify that $F^{\prime}=f.$ Fix $\overset{a\epsilon\Omega}{\hookrightarrow}$ $F(z)$ There exists an $\scriptstyle\gamma\simeq0$ such that $D(a;r)<\Omega.$ For z ∈ ${\mathfrak{b l}}(a;r)$ we can compute by integrating $\boldsymbol{\mathsf{f}}$ over a path T(a), followed by the interval [a, z]. Hence, for ze D'(a; r), $$ {\frac{F(z)-F(a)}{z-a}}={\frac{1}{z-a}}\left.\right|_{(a,z)}f(\zeta)\,d\zeta, $$ (2) and the continuity ${\mathfrak{o l}}r$ at a implies now that $F(a)=f(a),$ as in the proof of Theorem 10.14 the derivative of fe""is O in $\Omega,$ hence $f e^{-\theta}$ is constant (since $\Omega$ $f^{\prime}/f\in H(\Omega),$ and (g) (g) implies (h). If fe H(Q) and f has no zero in $\Omega,$ then implies that there exists a $g\in H(\Omega)$ so that $g^{\prime}=f^{\prime}/f.$ We can add a constant shows that to g, so that exp $\{g(z_{0})\}=f(z_{0})$ for some $\mathbb{Z}_{0}$ e Q. Our choice of $\scriptstyle{\mathcal{G}}$ is connected), and it follows that f = e*. (h) implies O). By $(h),f=e^{\theta}$ . Put $\varphi=\exp\,(\mathrm{i}$ hg) is homeomorphic to ${\boldsymbol{U}}\,,$ () implies (a)If $\Omega$ is the whole plane, then $\Omega$ map z to $z/(1+|z|).$ If $\Omega$ is a proper subregion of the plane which satisfies O), then there actually exists a holomorphic homeomorphism of $\Omega$ onto ${\boldsymbol{U}}$ (a conformal mapping). This assertion is the Riemann mapping theorem, which is the main objective of the next chapter. Hence the proof of Theorem 13.11 will be com- plete as soon as the Riemann mapping theorem is proved.(See the note following the statement of Theorem 14.8.) ///276 REAL AND coMPLEX ANALYSIs The fact that (h) holds in every simply connected region has the following consequence (which can also be proved by quite elementary means): 13.12 Theorem Iff e H(Q2), where $\Omega$ is any open set in the plane, and iff has no zero in Q, then log lfl is harmonic in S2 f = e in PROOF To every disc $\scriptstyle D\in\Omega$ there corresponds a function $g\in H(D)$ such that is $D.$ harmonic in every disc in $\Omega,$ , then uis harmonic in $D\!\!\!\!/$ D, and $|f|=e^{s}.$ Thus log $|f|$ // . If $u=\mathrm{Re}\,g$ and this gives the desired conclusion. Exercises 1 Prove that every meromorphic function on S $S^{2}$ is rational 2 Let $\Omega=\{z:\iota z\}<1$ and $|2z-1|>1\rangle,$ and suppose fe H(Q2). such that $P_{n}\to f$ uniformly on compact subsets of 2? (a) Must there exist a sequence of polynomials ${\mathbf{}}P_{n}$ (b) Must there exist such a sequence which converges to funiformly in Q2? namely, that f be holomorphic in some (c Is the answer to b) changed if we require more $\operatorname{of}f,$ open set which contains the closure of $\Omega^{\gamma}$ 3 Is there a sequence of polynomials ${\mathbf{}}P_{n}$ such that $P_{n}(0)=1$ for n = 1,2,3, ……,but P,.z)→0 for every $z\neq0,$ as $n{ arrow}\infty{\ }$ 4 Is there a sequence of polynomials ${\mathbf{}}P_{n}$ such that $$ \operatorname*{lim}_{n arrow\infty}P_{n}(z)={\left\{\begin{array}{l l}{1}&{{\mathrm{if~}}\ ~{\mathsf{I m}}~z>0,}\\ {0}&{{\mathrm{if~}}z{\mathrm{~is~real}},}\\ {-1}&{{\mathrm{if~}}\mathrm{~Im~}}z<07}\end{array}\right.} $$ on S For $n=1,$ 2, $3,\ldots,$ let ${\boldsymbol{\Delta}}_{n}$ be a closed disc in ${\boldsymbol{U}},$ and let $\underline{{\mathbf{f}}}_{m}$ be an arc (a homeomorphic image of In other words. [0,1) in $U-\Delta_{n}$ which intersects every radius of $L_{n},$ Show that {A,}, $\{L_{n}\},$ and $\{P_{n}\}$ can be so chosen that the series $f=\Sigma\ P_{n}$ for no real does lim,-. f(rel")exist. ${\boldsymbol{U}}.$ There are polynomials ${\mathbf{}}P_{n}$ which are very small $\Delta_{n}$ and more or less arbitrary on which has no radial limit at any point of ${\boldsymbol{T}}.$ defines a function $f\in H(U)$ and 6 Here is another construction of such a function. Let $\{n_{k}\}$ be a sequence of intersuch that $n_{1}>1$ $n_{k+1}>2k n_{k}.$ Define $$ h(z)=\sum_{k=1}^{\infty}\;5^{k}z^{n_{k}}. $$ Prove that the series converges if $|z|<1$ and prove that there is a constant $c>0$ such that $|h(z)|>c\cdot5^{n}$ for all $\mathbb{Z}$ with $|z|=1-(1/n_{m}).$ [Hint: For such z the mth term in the series defining h(z) is much larger than the sum of all the others.] Hence Prove also that ${\boldsymbol{h}}$ h has no finite radial limits. . (Compare with Exercise 15, Chap.12.) ${\boldsymbol{h}}$ must have infinitely many zeros in ${\boldsymbol{U}}.$ In fact, prove that to every complex number α there correspond infinitely many $z\in U$ at which $h(z)=\alpha$ 7 Show that in Theorem 13.9 we need not assume that $\scriptstyle A\quad\quad A$ intersects each component of $\scriptstyle A\quad}$ intersects each component of $\scriptstyle S^{n}=\Omega$ It is enough to assume that the closure of $\scriptstyle{\theta-\alpha}$ 8 Prove the Mittag-Lefler theorem for the case in which $\underline{{\Omega}}$ is the whole plane, by a direct argument which makes no appeal to Runge's theorem. 9 Suppose $\underline{{\Omega}}$ is a simply connected region, fe H(Q), f has no zero in $\Omega,$ and $\scriptstyle{\vec{n}}$ is a positive integer. Prove that there exists ${\mathfrak{a}}\ g\in H(\Omega)$ such that $g^{n}=f.$APPROXIMA TIONs BY RATIONAL FUNCTIONS $277$ 10 Suppose S is a region, fe H(Q), and f≠ 0. Prove that f has a holomorphic logarithm in $\underline{{\Omega}}$ if and only if has holomorphic nth roots in $\Omega_{\mathbf{\Lambda}}$ for every positive integer n every $\scriptstyle{\varepsilon\circ\Omega}$ 11 Suppose that f,e H(Q) Prove that $\underline{{\Omega}}$ has a dense open subset ${\boldsymbol{y}}$ on which $\underline{{\Omega}}$ contains a disc on which $\scriptstyle{\varphi}$ is $\varphi=\operatorname*{sup}\;|f_{n}|\,.$ $(n=1,$ 2,3, ……)f is a complex function in $\Omega,$ , and $f(z)=\operatorname*{lim}_{n\to\infty}\;f_{n}(z)$ for Use Baire's theorem to prove that every disc in $\scriptstyle{\mathcal{f}}$ s holomorphic. Hint: Put bounded. Apply Exercise 5, Chap. 10. (In general, $V\neq\Omega.$ Compare Exercises 3 and 4.) 12 Suppose, however, that fis any complex-valued measurable function defined in the complex plane, and prove that there is a sequence of holomorphic polynomials ${\mathbf{}}P_{n}$ such that $\operatorname*{lim}_{n arrow\infty}P_{n}(z)=f(z)$ for almost every z (with respect to two-dimensional Lebesgue measure).