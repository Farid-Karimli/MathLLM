\documentclass[10pt]{article}
\usepackage[utf8]{inputenc}
\usepackage[T1]{fontenc}
\usepackage{amsmath}
\usepackage{amsfonts}
\usepackage{amssymb}
\usepackage[version=4]{mhchem}
\usepackage{stmaryrd}

\begin{document}
\section{COMPLEX MEASURES}
\section{Total Variation}
6.1 Introduction Let $\mathfrak{M}$ be a $\sigma$-algebra in a set $X$. Call a countable collection $\left\{E_{i}\right\}$ of members of $\mathfrak{M}$ a partition of $E$ if $E_{i} \cap E_{j}=\varnothing$ whenever $i \neq j$, and if $E=$ $\bigcup E_{i}$. A complex measure $\mu$ on $\mathfrak{M}$ is then a complex function on $\mathfrak{M}$ such that

$$
\mu(E)=\sum_{i=1}^{\infty} \mu\left(E_{i}\right) \quad(E \in \mathfrak{M})
$$

for every partition $\left\{E_{i}\right\}$ of $E$.

Observe that the convergence of the series in (1) is now part of the requirement (unlike for positive measures, where the series could either converge or diverge to $\infty$ ). Since the union of the sets $E_{i}$ is not changed if the subscripts are permuted, every rearrangement of the series (1) must also converge. Hence ([26], Theorem 3.56) the series actually converges absolutely.

Let us consider the problem of finding a positive measure $\lambda$ which dominates a given complex measure $\mu$ on $\mathfrak{M}$, in the sense that $|\mu(E)| \leq \lambda(E)$ for every $E \in \mathfrak{M}$, and let us try to keep $\lambda$ as small as we can. Every solution to our problem (if there is one at all) must satisfy

$$
\lambda(E)=\sum_{i=1}^{\infty} \lambda\left(E_{i}\right) \geq \sum_{1}^{\infty}\left|\mu\left(E_{i}\right)\right|
$$

for every partition $\left\{E_{i}\right\}$ of any set $E \in \mathfrak{M}$, so that $\lambda(E)$ is at least equal to the supremum of the sums on the right of (2), taken over all partitions of $E$. This suggests that we define a set function $|\mu|$ on $\mathfrak{M}$ by

$$
|\mu|(E)=\sup \sum_{i=1}^{\infty}\left|\mu\left(E_{i}\right)\right| \quad(E \in \mathfrak{M})
$$

the supremum being taken over all partitions $\left\{E_{i}\right\}$ of $E$.

This notation is perhaps not the best, but it is the customary one. Note that $|\mu|(E) \geq|\mu(E)|$, but that in general $|\mu|(E)$ is not equal to $|\mu(E)|$.

It turns out, as will be proved below, that $|\mu|$ actually is a measure, so that our problem does have a solution. The discussion which led to (3) shows then clearly that $|\mu|$ is the minimal solution, in the sense that any other solution $\lambda$ has the property $\lambda(E) \geq|\mu|(E)$ for all $E \in \mathfrak{M}$.

The set function $|\mu|$ is called the total variation of $\mu$, or sometimes, to avoid misunderstanding, the total variation measure. The term "total variation of $\mu$ " is also frequently used to denote the number $|\mu|(X)$.

If $\mu$ is a positive measure, then of course $|\mu|=\mu$.

Besides being a measure, $|\mu|$ has another unexpected property: $|\mu|(X)<\infty$. Since $|\mu(E)| \leq|\mu|(E) \leq|\mu|(X)$, this implies that every complex measure $\mu$ on any $\sigma$-algebra is bounded: If the range of $\mu$ lies in the complex plane, then it actually lies in some disc of finite radius. This property (proved in Theorem 6.4) is sometimes expressed by saying that $\mu$ is of bounded variation.

6.2 Theorem The total variation $|\mu|$ of a complex measure $\mu$ on $\mathfrak{M}$ is a positive measure on $\mathfrak{M}$.

ProOF Let $\left\{E_{i}\right\}$ be a partition of $E \in \mathfrak{M}$. Let $t_{i}$ be real numbers such that $t_{i}<|\mu|\left(E_{i}\right)$. Then each $E_{i}$ has a partition $\left\{A_{i j}\right\}$ such that

$$
\sum_{j}\left|\mu\left(A_{i j}\right)\right|>t_{i} \quad(i=1,2,3, \ldots)
$$

Since $\left\{A_{i j}\right\}(i, j=1,2,3, \ldots)$ is a partition of $E$, it follows that

$$
\sum_{i} t_{i} \leq \sum_{i, j}\left|\mu\left(A_{i j}\right)\right| \leq|\mu|(E) .
$$

Taking the supremum of the left side of (2), over all admissible choices of $\left\{t_{i}\right\}$, we see that

$$
\sum_{i}|\mu|\left(E_{i}\right) \leq|\mu|(E) .
$$

To prove the opposite inequality, let $\left\{A_{j}\right\}$ be any partition of $E$. Then for any fixed $j,\left\{A_{j} \cap E_{i}\right\}$ is a partition of $A_{j}$, and for any fixed $i,\left\{A_{j} \cap E_{i}\right\}$ is a partition of $E_{i}$. Hence

$$
\begin{aligned}
\sum_{j}\left|\mu\left(A_{j}\right)\right| & =\sum_{j}\left|\sum_{i} \mu\left(A_{j} \cap E_{i}\right)\right| \\
& \leq \sum_{j} \sum_{i}\left|\mu\left(A_{j} \cap E_{i}\right)\right| \\
& =\sum_{i} \sum_{j}\left|\mu\left(A_{j} \cap E_{i}\right)\right| \leq \sum_{i}|\mu|\left(E_{i}\right) .
\end{aligned}
$$

Since (4) holds for every partition $\left\{A_{j}\right\}$ of $E$, we have

$$
|\mu|(E) \leq \sum_{i}|\mu|\left(E_{i}\right)
$$

By (3) and (5), $|\mu|$ is countably additive.

Note that the Corollary to Theorem 1.27 was used in (2) and (4).

That $|\mu|$ is not identically $\infty$ is a trivial consequence of Theorem 6.4 but can also be seen right now, since $|\mu|(\varnothing)=0$.

6.3 Lemma If $z_{1}, \ldots, z_{N}$ are complex numbers then there is a subset $S$ of $\{1, \ldots, N\}$ for which

$$
\left|\sum_{k \in S} z_{k}\right| \geq \frac{1}{\pi} \sum_{k=1}^{N}\left|z_{k}\right|
$$

ProOF Write $z_{k}=\left|z_{k}\right| e^{i \alpha_{k}}$. For $-\pi \leq \theta \leq \pi$, let $S(\theta)$ be the set of all $k$ for which $\cos \left(\alpha_{k}-\theta\right)>0$. Then

$$
\left|\sum_{S(\theta)} z_{k}\right|=\left|\sum_{S(\theta)} e^{-i \theta} z_{k}\right| \geq \operatorname{Re} \sum_{S(\theta)} e^{-i \theta} z_{k}=\sum_{k=1}^{N}\left|z_{k}\right| \cos ^{+}\left(\alpha_{k}-\theta\right)
$$

Choose $\theta_{0}$ so as to maximize the last sum, and put $S=S\left(\theta_{0}\right)$. This maximum is at least as large as the average of the sum over $[-\pi, \pi]$, and this average is $\pi^{-1} \sum\left|z_{k}\right|$, because

$$
\frac{1}{2 \pi} \int_{-\pi}^{\pi} \cos ^{+}(\alpha-\theta) d \theta=\frac{1}{\pi}
$$

for every $\alpha$.

6.4 Theorem If $\mu$ is a complex measure on $X$, then

$$
|\mu|(X)<\infty
$$

Proof Suppose first that some set $E \in \mathfrak{M}$ has $|\mu|(E)=\infty$. Put $t=\pi(1+|\mu(E)|)$. Since $|\mu|(E)>t$, there is a partition $\left\{E_{i}\right\}$ of $E$ such that

$$
\sum_{i=1}^{N}\left|\mu\left(E_{i}\right)\right|>t
$$

for some $N$. Apply Lemma 6.3 , with $z_{i}=\mu\left(E_{i}\right)$, to conclude that there is a set $A \subset E$ (a union of some of the sets $E_{i}$ ) for which

$$
|\mu(A)|>t / \pi>1
$$

Setting $B=E-A$, it follows that

$$
|\mu(B)|=|\mu(E)-\mu(A)| \geq|\mu(A)|-|\mu(E)|>\frac{t}{\pi}-|\mu(E)|=1
$$

We have thus split $E$ into disjoint sets $A$ and $B$ with $|\mu(A)|>1$ and $|\mu(B)|>1$. Evidently, at least one of $|\mu|(A)$ and $|\mu|(B)$ is $\infty$, by Theorem 6.2.

Now if $|\mu|(X)=\infty$, split $X$ into $A_{1}, B_{1}$, as above, with $\left|\mu\left(A_{1}\right)\right|>1$, $|\mu|\left(B_{1}\right)=\infty$. Split $B_{1}$ into $A_{2}, B_{2}$, with $\left|\mu\left(A_{2}\right)\right|>1,|\mu|\left(B_{2}\right)=\infty$. Continuing in this way, we get a countably infinite disjoint collection $\left\{A_{i}\right\}$, with $\left|\mu\left(A_{i}\right)\right|>1$ for each $i$. The countable additivity of $\mu$ implies that

$$
\mu\left(\bigcup_{i} A_{i}\right)=\sum_{i} \mu\left(A_{i}\right)
$$

But this series cannot converge, since $\mu\left(A_{i}\right)$ does not tend to 0 as $i \rightarrow \infty$. This contradiction shows that $|\mu|(X)<\infty$.

6.5 If $\mu$ and $\lambda$ are complex measures on the same $\sigma$-algebra $\mathfrak{M}$, we define $\mu+\lambda$ and $c \mu$ by

$$
\begin{aligned}
(\mu+\lambda)(E) & =\mu(E)+\lambda(E) \\
(c \mu)(E) & =c \mu(E)
\end{aligned}
$$

for any scalar $c$, in the usual manner. It is then trivial to verify that $\mu+\lambda$ and $c \mu$ are complex measures. The collection of all complex measures on $\mathfrak{M}$ is thus a vector space. If we put

$$
\|\mu\|=|\mu|(X)
$$

it is easy to verify that all axioms of a normed linear space are satisfied.

6.6 Positive and Negative Variations Let us now specialize and consider a real measure $\mu$ on a $\sigma$-algebra $\mathfrak{M}$. (Such measures are frequently called signed measures.) Define $|\mu|$ as before, and define

$$
\mu^{+}=\frac{1}{2}(|\mu|+\mu), \quad \mu^{-}=\frac{1}{2}(|\mu|-\mu)
$$

Then both $\mu^{+}$and $\mu^{-}$are positive measures on $\mathfrak{M}$, and they are bounded, by Theorem 6.4. Also,

$$
\mu=\mu^{+}-\mu^{-}, \quad|\mu|=\mu^{+}+\mu^{-} .
$$

The measures $\mu^{+}$and $\mu^{-}$are called the positive and negative variations of $\mu$, respectively. This representation of $\mu$ as the difference of the positive measures $\mu^{+}$ and $\mu^{-}$is known as the Jordan decomposition of $\mu$. Among all representations of $\mu$ as a difference of two positive measures, the Jordan decomposition has a certain minimum property which will be established as a corollary to Theorem 6.14 .

\section{Absolute Continuity}
6.7 Definitions Let $\mu$ be a positive measure on a $\sigma$-algebra $\mathfrak{M}$, and let $\lambda$ be an arbitrary measure on $\mathfrak{M}$; $\lambda$ may be positive or complex. (Recall that a complex measure has its range in the complex plane, but that our usage of the term "positive measure" includes $\infty$ as an admissible value. Thus the positive measures do not form a subclass of the complex ones.)

We say that $\lambda$ is absolutely continuous with respect to $\mu$, and write

$$
\lambda \ll \mu
$$

if $\lambda(E)=0$ for every $E \in \mathfrak{M}$ for which $\mu(E)=0$.

If there is a set $A \in \mathfrak{M}$ such that $\lambda(E)=\lambda(A \cap E)$ for every $E \in \mathfrak{M}$, we say that $\lambda$ is concentrated on $A$. This is equivalent to the hypothesis that $\lambda(E)=0$ whenever $E \cap A=\varnothing$.

Suppose $\lambda_{1}$ and $\lambda_{2}$ are measures on $\mathfrak{M}$, and suppose there exists a pair of disjoint sets $A$ and $B$ such that $\lambda_{1}$ is concentrated on $A$ and $\lambda_{2}$ is concentrated on $B$. Then we say that $\lambda_{1}$ and $\lambda_{2}$ are mutually singular, and write

$$
\lambda_{1} \perp \lambda_{2} .
$$

Here are some elementary properties of these concepts.

6.8 Proposition Suppose, $\mu, \lambda, \lambda_{1}$, and $\lambda_{2}$ are measures on a $\sigma$-algebra $\mathfrak{M}$, and $\mu$ is positive.

(a) If $\lambda$ is concentrated on $A$, so is $|\lambda|$.

(b) If $\lambda_{1} \perp \lambda_{2}$, then $\left|\lambda_{1}\right| \perp\left|\lambda_{2}\right|$.

(c) If $\lambda_{1} \perp \mu$ and $\lambda_{2} \perp \mu$, then $\lambda_{1}+\lambda_{2} \perp \mu$.

(d) If $\lambda_{1} \ll \mu$ and $\lambda_{2} \ll \mu$, then $\lambda_{1}+\lambda_{2} \ll \mu$.

(e) If $\lambda \ll \mu$, then $|\lambda| \ll \mu$.

(f) If $\lambda_{1} \ll \mu$ and $\lambda_{2} \perp \mu$, then $\lambda_{1} \perp \lambda_{2}$.

(g) If $\lambda \ll \mu$ and $\lambda \perp \mu$, then $\lambda=0$.

\section{ProOF}
(a) If $E \cap A=\varnothing$ and $\left\{E_{j}\right\}$ is any partition of $E$, then $\lambda\left(E_{j}\right)=0$ for all $j$. Hence $|\lambda|(E)=0$.

(b) This follows immediately from $(a)$.

(c) There are disjoint sets $A_{1}$ and $B_{1}$ such that $\lambda_{1}$ is concentrated on $A_{1}$ and $\mu$ on $B_{1}$, and there are disjoint sets $A_{2}$ and $B_{2}$ such that $\lambda_{2}$ is concentrated on $A_{2}$ and $\mu$ on $B_{2}$. Hence $\lambda_{1}+\lambda_{2}$ is concentrated on $A=A_{1} \cup$ $A_{2}, \mu$ is concentrated on $B=B_{1} \cap B_{2}$, and $A \cap B=\varnothing$.

(d) This is obvious.

(e) Suppose $\mu(E)=0$, and $\left\{E_{j}\right\}$ is a partition of $E$. Then $\mu\left(E_{j}\right)=0$; and since $\lambda \ll \mu, \lambda\left(E_{j}\right)=0$ for all $j$, hence $\sum\left|\lambda\left(E_{j}\right)\right|=0$. This implies $|\lambda|(E)=0$.
(f) Since $\lambda_{2} \perp \mu$, there is a set $A$ with $\mu(A)=0$ on which $\lambda_{2}$ is concentrated. Since $\lambda_{1} \ll \mu, \lambda_{1}(E)=0$ for every $E \subset A$. So $\lambda_{1}$ is concentrated on the complement of $A$.

(g) By $(f)$, the hypothesis of $(g)$ implies, that $\lambda \perp \lambda$, and this clearly forces $\lambda=0$.

We come now to the principal theorem about absolute continuity. In fact, it is probably the most important theorem in measure theory. Its statement will involve $\sigma$-finite measures. The following lemma describes one of their significant properties.

6.9 Lemma If $\mu$ is a positive $\sigma$-finite measure on a $\sigma$-algebra $\mathfrak{M}$ in a set $X$, then there is a function $w \in L^{1}(\mu)$ such that $0<w(x)<1$ for every $x \in X$.

Proof To say that $\mu$ is $\sigma$-finite means that $X$ is the union of countably many sets $E_{n} \in \mathfrak{M}(n=1,2,3, \ldots)$ for which $\mu\left(E_{n}\right)$ is finite. Put $w_{n}(x)=0$ if $x \in$ $X-E_{n}$ and put

$$
w_{n}(x)=2^{-n} /\left(1+\mu\left(E_{n}\right)\right)
$$

if $x \in E_{n}$. Then $w=\sum_{1}^{\infty} w_{n}$ has the required properties.

The point of the lemma is that $\mu$ can be replaced by a finite measure $\tilde{\mu}$ (namely, $d \tilde{\mu}=w d \mu$ ) which, because of the strict positivity of $w$, has precisely the same sets of measure 0 as $\mu$.

6.10 The Theorem of Lebesgue-Radon-Nikodym Let $\mu$ be a positive $\sigma$-finite measure on a $\sigma$-algebra $\mathfrak{M}$ in a set $X$, and let $\lambda$ be a complex measure on $\mathfrak{M}$.

(a) There is then a unique pair of complex measures $\lambda_{a}$ and $\lambda_{s}$ on $\mathfrak{M}$ such that

$$
\lambda=\lambda_{a}+\lambda_{s}, \quad \lambda_{a} \ll \mu, \quad \lambda_{s} \perp \mu .
$$

If $\lambda$ is positive and finite, then so are $\lambda_{a}$ and $\lambda_{s}$.

(b) There is a unique $h \in L^{1}(\mu)$ such that

$$
\lambda_{a}(E)=\int_{E} h d \mu
$$

for every set $\boldsymbol{E} \in \mathfrak{M}$.

The pair $\left(\lambda_{a}, \lambda_{s}\right)$ is called the Lebesgue decomposition of $\lambda$ relative to $\mu$. The uniqueness of the decomposition is easily seen, for if $\left(\lambda_{a}^{\prime}, \lambda_{s}^{\prime}\right)$ is another pair which satisfies (1), then

$$
\lambda_{a}^{\prime}-\lambda_{a}=\lambda_{s}-\lambda_{s}^{\prime}
$$

$\lambda_{a}^{\prime}-\lambda_{\alpha} \ll \mu$, and $\lambda_{s}-\lambda_{s}^{\prime} \perp \mu$, hence both sides of (3) are 0 ; we have used 6.8(c), $6.8(d)$, and $6.8(g)$.

The existence of the decomposition is the significant part of $(a)$.

Assertion (b). is known as the Radon-Nikodym theorem. Again, uniqueness of $h$ is immediate, from Theorem $1.39(b)$. Also, if $h$ is any member of $L^{1}(\mu)$, the integral in (2) defines a measure on $\mathfrak{M}$ (Theorem 1.29) which is clearly absolutely continuous with respect to $\mu$. The point of the Radon-Nikodym theorem is the converse: Every $\lambda \ll \mu$ (in which case $\lambda_{a}=\lambda$ ) is obtained in this way.

The function $h$ which occurs in (2) is called the Radon-Nikodym derivative of $\lambda_{a}$ with respect to $\mu$. As noted after Theorem 1.29, we may express (2) in the form $d \lambda_{a}=h d \mu$, or even in the form $h=d \lambda_{a} / d \mu$.

The idea of the following proof, which yields both $(a)$ and $(b)$ at one stroke, is due to von Neumann.

Proof Assume first that $\lambda$ is a positive bounded measure on $\mathfrak{M}$. Associate $w$ to $\mu$ as in Lemma 6.9. Then $d \varphi=d \lambda+w d \mu$ defines a positive bounded measure $\varphi$ on $\mathfrak{M}$. The definition of the sum of two measures shows that

$$
\int_{X} f d \varphi=\int_{X} f d \lambda+\int_{X} f w d \mu
$$

for $f=\chi_{E}$, hence for simple $f$, hence for any nonnegative measurable $f$. If $f \in L^{2}(\varphi)$, the Schwarz inequality gives

$$
\left|\int_{X} f d \lambda\right| \leq \int_{X}|f| d \lambda \leq \int_{X}|f| d \varphi \leq\left\{\int_{X}|f|^{2} d \varphi\right\}^{1 / 2}\{\varphi(X)\}^{1 / 2}
$$

Since $\varphi(X)<\infty$, we see that

$$
f \rightarrow \int_{X} f d \lambda
$$

is a bounded linear functional on $L^{2}(\varphi)$. We know that every bounded linear functional on a Hilbert space $H$ is given by an inner product with an element of $H$. Hence there exists a $g \in L^{2}(\varphi)$ such that

$$
\int_{X} f d \lambda=\int_{X} f g d \varphi
$$

for every $f \in L^{2}(\varphi)$.

Observe how the completeness of $L^{2}(\varphi)$ was used to guarantee the existence of $g$. Observe also that although $g$ is defined uniquely as an element of $L^{2}(\varphi), g$ is determined only a.e. $[\varphi]$ as a point function on $X$.

Put $f=\chi_{E}$ in (6), for any $E \in \mathfrak{M}$ with $\varphi(E)>0$. The left side of (6) is then $\lambda(E)$, and since $0 \leq \lambda \leq \varphi$, we have

$$
0 \leq \frac{1}{\varphi(E)} \int_{E} g d \varphi=\frac{\lambda(E)}{\varphi(E)} \leq 1
$$

Hence $g(x) \in[0,1]$ for almost all $x$ (with respect to $\varphi$ ), by Theorem 1.40. We may therefore assume that $0 \leq g(x) \leq 1$ for every $x \in X$, without affecting (6), and we rewrite (6) in the form

$$
\int_{X}(1-g) f d \lambda=\int_{X} f g w d \mu
$$

Put

$$
A=\{x: 0 \leq g(x)<1\}, \quad B=\{x: g(x)=1\}
$$

and define measures $\lambda_{a}$ and $\lambda_{s}$ by

$$
\lambda_{a}(E)=\lambda(A \cap E), \quad \lambda_{s}(E)=\lambda(B \cap E)
$$

for all $E \in \mathfrak{M}$.

If $f=\chi_{B}$ in (8), the left side is 0 , the right side is $\int_{B} w d \mu$. Since $w(x)>0$ for all $x$, we conclude that $\mu(B)=0$. Thus $\lambda_{s} \perp \mu$.

Since $g$ is bounded, (8) holds if $f$ is replaced by

$$
\left(1+g+\cdots+g^{n}\right) \chi_{E}
$$

for $n=1,2,3, \ldots, E \in \mathfrak{M}$. For $\operatorname{such} f,(8)$ becomes

$$
\int_{E}\left(1-g^{n+1}\right) d \lambda=\int_{E} g\left(1+g+\cdots+g^{n}\right) w d \mu
$$

At every point of $B, g(x)=1$, hence $1-g^{n+1}(x)=0$. At every point of $A$, $g^{n+1}(x) \rightarrow 0$ monotonically. The left side of (11) converges therefore to $\lambda(A \cap E)=\lambda_{a}(E)$ as $n \rightarrow \infty$.

The integrands on the right side of (11) increase monotonically to a nonnegative measurable limit $h$, and the monotone convergence theorem shows that the right side of (11) tends to $\int_{E} h d \mu$ as $n \rightarrow \infty$.

We have thus proved that (2) holds for every $E \in \mathfrak{M}$. Taking $E=X$, we see that $h \in L^{1}(\mu)$, since $\lambda_{a}(X)<\infty$.

Finally, (2) shows that $\lambda_{a} \ll \mu$, and the proof is complete for positive $\lambda$.

If $\lambda$ is a complex measure on $\mathfrak{M}$, then $\lambda=\lambda_{1}+i \lambda_{2}$, with $\lambda_{1}$ and $\lambda_{2}$ real, and we can apply the preceding case to the positive and negative variations of $\lambda_{1}$ and $\lambda_{2}$.

If both $\mu$ and $\lambda$ are positive and $\sigma$-finite, most of Theorem 6.10 is still true. We can now write $X=\bigcup X_{n}$, where $\mu\left(X_{n}\right)<\infty$ and $\lambda\left(X_{n}\right)<\infty$, for $n=1,2,3$, The Lebesgue decompositions of the measures $\lambda\left(E \cap X_{n}\right)$ still give us a Lebesgue decomposition of $\lambda$, and we still get a function $h$ which satisfies Eq. 6.10(2); however, it is no longer true that $h \in L^{1}(\mu)$, although $h$ is "locally in $L^{1}$," i.e., $\int_{X_{n}} h d \mu<\infty$ for each $n$.

Finally, if we go beyond $\sigma$-finiteness, we meet situations where the two theorems under consideration actually fail. For example, let $\mu$ be Lebesgue measure on $(0,1)$, and let $\lambda$ be the counting measure on the $\sigma$-algebra of all Lebesgue
measurable sets in $(0,1)$. Then $\lambda$ has no Lebesgue decomposition relative to $\mu$, and although $\mu \ll \lambda$ and $\mu$ is bounded, there is no $h \in L^{1}(\lambda)$ such that $d \mu=h d \lambda$. We omit the easy proof.

The following theorem may explain why the word "continuity" is used in connection with the relation $\lambda \ll \mu$.

6.11 Theorem Suppose $\mu$ and $\lambda$ are measures on a $\sigma$-algebra $\mathfrak{M}, \mu$ is positive, and $\lambda$ is complex. Then the following two conditions are equivalent:

(a) $\lambda \ll \mu$.

(b) To every $\epsilon>0$ corresponds a $\delta>0$ such that $|\lambda(E)|<\epsilon$ for all $E \in \mathfrak{M}$ with $\mu(E)<\delta$.

Property $(b)$ is sometimes used as the definition of absolute continuity. However, $(a)$ does not imply $(b)$ if $\lambda$ is a positive unbounded measure. For instance, let $\mu$ be Lebesgue measure on $(0,1)$, and put

$$
\lambda(E)=\int_{E} t^{-1} d t
$$

for every Lebesgue measurable set $E \subset(0,1)$.

Proof Suppose $(b)$ holds. If $\mu(E)=0$, then $\mu(E)<\delta$ for every $\delta>0$, hence $|\lambda(E)|<\epsilon$ for every $\epsilon>0$, so $\lambda(E)=0$. Thus $(b)$ implies $(a)$.

Suppose $(b)$ is false. Then there exists an $\epsilon>0$ and there exist sets $E_{n} \in$ $\mathfrak{M}(n=1,2,3, \ldots)$ such that $\mu\left(E_{n}\right)<2^{-n}$ but $\left|\lambda\left(E_{n}\right)\right| \geq \epsilon$. Hence $|\lambda|\left(E_{n}\right) \geq \epsilon$. Put

$$
A_{n}=\bigcup_{i=n}^{\infty} E_{i}, \quad A=\bigcap_{n=1}^{\infty} A_{n}
$$

Then $\mu\left(A_{n}\right)<2^{-n+1}, A_{n} \supset A_{n+1}$, and so Theorem 1.19(e) shows that $\mu(A)=0$ and that

$$
|\lambda|(A)=\lim _{n \rightarrow \infty}|\lambda|\left(A_{n}\right) \geq \epsilon>0
$$

since $|\lambda|\left(A_{n}\right) \geq|\lambda|\left(E_{n}\right)$.

It follows that we do not have $|\lambda| \ll \mu$, hence $(a)$ is false, by Proposition 6.8(e).

\section{Consequences of the Radon-Nikodym Theorem}
6.12 Theorem Let $\mu$ be a complex measure on a $\sigma$-algebra $\mathfrak{M}$ in $X$. Then there is a measurable function $h$ such that $|h(x)|=1$ for all $x \in X$ and such that

$$
d \mu=h d|\mu|
$$

By analogy with the representation of a complex number as the product of ts absolute value and a number of absolute value 1, Eq. (1) is sometimes referred $\mathrm{o}$ as the polar representation (or polar decomposition) of $\mu$.

Proof It is trivial that $\mu \ll|\mu|$, and therefore the Radon-Nikodym theorem guarantees the existence of some $h \in L^{1}(|\mu|)$ which satisfies (1).

Let $A_{r}=\{x:|h(x)|<r\}$, where $r$ is some positive number, and let $\left\{E_{j}\right\}$ be a partition of $A_{r}$. Then

$$
\sum_{j}\left|\mu\left(E_{j}\right)\right|=\sum_{j}\left|\int_{E_{j}} h d\right| \mu|| \leq \sum_{j} r|\mu|\left(E_{j}\right)=r|\mu|\left(A_{r}\right)
$$

so that $|\mu|\left(A_{r}\right) \leq r|\mu|\left(A_{r}\right)$. If $r<1$, this forces $|\mu|\left(A_{r}\right)=0$. Thus $|h| \geq 1$ a.e.

On the other hand, if $|\mu|(E)>0$, (1) shows that

$$
\left|\frac{1}{|\mu|(E)} \int_{E} h d\right| \mu||=\frac{|\mu(E)|}{|\mu|(E)} \leq 1
$$

We now apply Theorem 1.40 (with the closed unit disc in place of $S$ ) and conclude that $|h| \leq 1$ a.e.

Let $B=\{x \in X:|h(x)| \neq 1\}$. We have shown that $|\mu|(B)=0$, and if we redefine $h$ on $B$ so that $h(x)=1$ on $B$, we obtain a function with the desired properties.

6.13 Theorem Suppose $\mu$ is a positive measure on $\mathfrak{M}, g \in L^{1}(\mu)$, and

$$
\lambda(E)=\int_{E} g d \mu \quad(E \in \mathfrak{M})
$$

Then

$$
|\lambda|(E)=\int_{E}|g| d \mu \quad(E \in \mathfrak{M})
$$

Proof By Theorem 6.12, there is a function $h$, of absolute value 1, such that $d \lambda=h d|\lambda|$. By hypothesis, $d \lambda=g d \mu$. Hence

$$
h d|\lambda|=g d \mu
$$

This gives $d|\lambda|=\bar{h} g d \mu$. (Compare with Theorem 1.29.)

Since $|\lambda| \geq 0$ and $\mu \geq 0$, it follows that $\bar{h} g \geq 0$ a.e. $[\mu]$, so that $\bar{h} g=|g|$ a.e. $[\mu]$.

6.14 The Hahn Decomposition Theorem Let $\mu$ be a real measure on a $\sigma$ algebra $\mathfrak{M}$ in a set $X$. Then there exist sets $A$ and $B \in \mathfrak{M}$ such that
$A \cup B=X, A \cap B=\varnothing$, and such that the positive and negative variations $\mu^{+}$and $\mu^{-}$of $\mu$ satisfy

$$
\mu^{+}(E)=\mu(A \cap E), \quad \mu^{-}(E)=-\mu(B \cap E) \quad(E \in \mathfrak{M})
$$

In other words, $X$ is the union of two disjoint measurable sets $A$ and $B$, such that " $A$ carries all the positive mass of $\mu$ " [since (1) implies that $\mu(E) \geq 0$ if $E \subset A]$ and " $B$ carries all the negative mass of $\mu$ " [since $\mu(E) \leq 0$ if $E \subset B]$. The pair $(A, B)$ is called a Hahn decomposition of $X$, induced by $\mu$.

ProOf By Theorem 6.12, $d \mu=h d|\mu|$, where $|h|=1$. Since $\mu$ is real, it follows that $h$ is real (a.e., and therefore everywhere, by redefining on a set of measure 0 ), hence $h= \pm 1$. Put

$$
A=\{x: h(x)=1\}, \quad B=\{x: h(x)=-1\} .
$$

Since $\mu^{+}=\frac{1}{2}(|\mu|+\mu)$, and since

$$
\frac{1}{2}(1+h)= \begin{cases}h & \text { on } A \\ 0 & \text { on } B\end{cases}
$$

we have, for any $E \in \mathfrak{M}$,

$$
\mu^{+}(E)=\frac{1}{2} \int_{E}(1+h) d|\mu|=\int_{E \cap A} h d|\mu|=\mu(E \cap A)
$$

Since $\mu(E)=\mu(E \cap A)+\mu(E \cap B)$ and since $\mu=\mu^{+}-\mu^{-}$, the second half of (1) follows from the first.

Corollary If $\mu=\lambda_{1}-\lambda_{2}$, where $\lambda_{1}$ and $\lambda_{2}$ are positive measures, then $\lambda_{1} \geq \mu^{+}$ and $\lambda_{2} \geq \mu^{-}$.

This is the minimum property of the Jordan decomposition which was mentioned in Sec. 6.6.

Proof Since $\mu \leq \lambda_{1}$, we have

$$
\mu^{+}(E)=\mu(E \cap A) \leq \lambda_{1}(E \cap A) \leq \lambda_{1}(E)
$$

\section{Bounded Linear Functionals on $L^{p}$}
6.15 Let $\mu$ be a positive measure, suppose $1 \leq p \leq \infty$, and let $q$ be the exponent conjugate to $p$. The Hölder inequality (Theorem 3.8) shows that if $g \in L^{q}(\mu)$ and if $\Phi_{g}$ is defined by

$$
\Phi_{g}(f)=\int_{X} f g d \mu
$$

then $\Phi_{g}$ is a bounded linear functional on $L^{p}(\mu)$, of norm at most $\|g\|_{q}$. The question naturally arises whether all bounded linear functionals on $L^{p}(\mu)$ have this form, and whether the representation is unique.

For $p=\infty$, Exercise 13 shows that the answer is negative: $L^{1}(m)$ does not furnish all bounded linear functionals on $L^{\infty}(m)$. For $1<p<\infty$, the answer is affirmative. It is also affirmative for $p=1$, provided certain measure-theoretic pathologies are excluded. For $\sigma$-finite measure spaces, no difficulties arise, and we shall confine ourselves to this case.

6.16 Theorem Suppose $1 \leq p<\infty, \mu$ is a $\sigma$-finite positive measure on $X$, and $\Phi$ is a bounded linear functional on $L^{p}(\mu)$. Then there is a unique $g \in L^{q}(\mu)$, where $q$ is the exponent conjugate to $p$, such that

$$
\Phi(f)=\int_{x} f g d \mu \quad\left(f \in L^{p}(\mu)\right)
$$

Moreover, if $\Phi$ and $g$ are related as in (1), we have

$$
\|\Phi\|=\|g\|_{q} \text {. }
$$

In other words, $L^{q}(\mu)$ is isometrically isomorphic to the dual space of $L^{p}(\mu)$, under the stated conditions.

Proof The uniqueness of $g$ is clear, for if $g$ and $g^{\prime}$ satisfy (1), then the integral of $g-g^{\prime}$ over any measurable set $E$ of finite measure is 0 (as we see by taking $\chi_{E}$ for $f$ ), and the $\sigma$-finiteness of $\mu$ implies therefore that $g-g^{\prime}=0$ a.e.

Next, if (1) holds, Hölder's inequality implies

$$
\|\Phi\| \leq\|g\|_{q} .
$$

So it remains to prove that $g$ exists and that equality holds in (3). If $\|\Phi\|=0$, (1) and (2) hold with $g=0$. So assume $\|\Phi\|>0$.

We first consider the case $\mu(X)<\infty$.

For any measurable set $E \subset X$, define

$$
\lambda(E)=\Phi\left(\chi_{E}\right) .
$$

Since $\Phi$ is linear, and since $\chi_{A \cup B}=\chi_{A}+\chi_{B}$ if $A$ and $B$ are disjoint, we see that $\lambda$ is additive. To prove countable additivity, suppose $E$ is the union of countably many disjoint measurable sets $E_{i}$, put $A_{k}=E_{1} \cup \cdots \cup E_{k}$, and note that

$$
\left\|\chi_{E}-\chi_{A_{k}}\right\|_{p}=\left[\mu\left(E-A_{k}\right)\right]^{1 / p} \rightarrow 0 \quad(k \rightarrow \infty)
$$

the continuity of $\Phi$ shows now that $\lambda\left(A_{k}\right) \rightarrow \lambda(E)$. So $\lambda$ is a complex measure. [In (4) the assumption $p<\infty$ was used.] It is clear that $\lambda(E)=0$ if $\mu(E)=0$,
since then $\left\|\chi_{E}\right\|_{p}=0$. Thus $\lambda \ll \mu$, and the Radon-Nikodym theorem ensures the existence of a function $g \in L^{1}(\mu)$ such that, for every measurable $E \subset X$,

$$
\Phi\left(\chi_{E}\right)=\int_{E} g d \mu=\int_{X} \chi_{E} g d \mu
$$

By linearity it follows that

$$
\Phi(f)=\int_{X} f g d \mu
$$

holds for every simple measurable $f$, and so also for every $f \in L^{\infty}(\mu)$, since every $f \in L^{\infty}(\mu)$ is a uniform limit of simple functions $f_{i}$. Note that the uniform convergence of $f_{i}$ to $f$ implies $\left\|f_{i}-f\right\|_{p} \rightarrow 0$, hence $\Phi\left(f_{i}\right) \rightarrow \Phi(f)$, as $i \rightarrow \infty$.

We want to conclude that $g \in L^{q}(\mu)$ and that (2) holds; it is best to split the argument into two cases.

CASE $1 p=1$. Here (5) shows that

$$
\left|\int_{E} g d \mu\right| \leq\|\Phi\| \cdot\left\|\chi_{E}\right\|_{1}=\|\Phi\| \cdot \mu(E)
$$

for every $E \in \mathfrak{M}$. By Theorem 1.40, $|g(x)| \leq\|\Phi\|$ a.e., so that $\|g\|_{\infty} \leq\|\Phi\|$.

CASE $21<p<\infty$. There is a measurable function $\alpha,|\alpha|=1$, such that $\alpha g=|g|$ [Proposition 1.9(e)]. Let $E_{n}=\{x:|g(x)| \leq n\}$, and define $f=$ $\chi_{E_{n}}|g|^{q-1} \alpha$. Then $|f|^{p}=|g|^{q}$ on $E_{n}, f \in L^{\infty}(\mu)$, and (6) gives

$$
\int_{E_{n}}|g|^{q} d \mu=\int_{X} f g d \mu=\Phi(f) \leq\|\Phi\|\left\{\int_{E_{n}}|g|^{q}\right\}^{1 / p},
$$

so that

$$
\int_{X} \chi_{E_{n}}|g|^{q} d \mu \leq\|\Phi\|^{q} \quad(n=1,2,3, \ldots)
$$

If we apply the monotone convergence theorem to (7), we obtain $\|g\|_{a} \leq\|\Phi\|$.

Thus (2) holds and $g \in L^{q}(\mu)$. It follows that both sides of (6) are continuous functions on $L^{p}(\mu)$. They coincide on the dense subset $L^{\infty}(\mu)$ of $L^{p}(\mu)$; hence they coincide on all of $L^{p}(\mu)$, and this completes the proof if $\mu(X)<\infty$.

If $\mu(X)=\infty$ but $\mu$ is $\sigma$-finite, choose $w \in L^{1}(\mu)$ as in Lemma 6.9. Then $d \tilde{\mu}=w d \mu$ defines a finite measure on $\mathfrak{M}$, and

$$
F \rightarrow w^{1 / p} F
$$

is a linear isometry of $L^{p}(\tilde{\mu})$ onto $L^{p}(\mu)$, because $w(x)>0$ for every $x \in X$. Hence

$$
\Psi(F)=\Phi\left(w^{1 / p} F\right)
$$

defines a bounded linear functional $\Psi$ on $L^{p}(\tilde{\mu})$, with $\|\Psi\|=\|\Phi\|$.

The first part of the proof shows now that there exists $G \in L^{q}(\tilde{\mu})$ such that

$$
\Psi(F)=\int_{X} F G d \tilde{\mu} \quad\left(F \in L^{p}(\tilde{\mu})\right)
$$

Put $g=w^{1 / q} G$. (If $p=1, g=G$.) Then

$$
\int_{X}|g|^{q} d \mu=\int_{X}|G|^{q} d \tilde{\mu}=\|\Psi\|^{q}=\|\Phi\|^{q}
$$

if $p>1$, whereas $\|g\|_{\infty}=\|G\|_{\infty}=\|\Psi\|=\|\Phi\|$ if $p=1$. Thus (2) holds, and since $G d \tilde{\mu}=w^{1 / p} g d \mu$, we finally get

$$
\Phi(f)=\Psi\left(w^{-1 / p} f\right)=\int_{X} w^{-1 / p} f G d \tilde{\mu}=\int_{X} f g d \mu
$$

for every $f \in L^{p}(\mu)$.

6.17 Remark We have already encountered the special case $p=q=2$ of Theorem 6.16. In fact, the proof of the general case was based on this special case, for we used the knowledge of the bounded linear functionals on $L^{2}(\mu)$ in the proof of the Radon-Nikodym theorem, and the latter was the key to the proof of Theorem 6.16. The special case $p=2$, in turn, depended on the completeness of $L^{2}(\mu)$, on the fact that $L^{2}(\mu)$ is therefore a Hilbert space, and on the fact that the bounded linear functionals on a Hilbert space are given by inner products.

We now turn to the complex version of Theorem 2.14.

\section{The Riesz Representation Theorem}
6.18 Let $X$ be a locally compact Hausdorff space. Theorem 2.14 characterizes the positive linear functionals on $C_{c}(X)$. We are now in a position to characterize the bounded linear functionals $\Phi$ on $C_{c}(X)$. Since $C_{c}(X)$ is a dense subspace of $C_{0}(X)$, relative to the supremum norm, every such $\Phi$ has a unique extension to a bounded linear functional on $C_{0}(X)$. Hence we may as well assume to begin with that we are dealing with the Banach space $C_{0}(X)$.

If $\mu$ is a complex Borel measure, Theorem 6.12 asserts that there is a complex Borel function $h$ with $|h|=1$ such that $d \mu=h d|\mu|$. It is therefore reasonable to lefine integration with respect to a complex measure $\mu$ by the formula

$$
\int f d \mu=\int f h d|\mu|
$$

The relation $\int \chi_{E} d \mu=\mu(E)$ is a special case of (1). Thus

$$
\int_{X} \chi_{E} d(\mu+\lambda)=(\mu+\lambda)(E)=\mu(E)+\lambda(E)=\int_{X} \chi_{E} d \mu+\int_{X} \chi_{E} d \lambda
$$

whenever $\mu$ and $\lambda$ are complex measures on $\mathfrak{M}$ and $E \in \mathfrak{M}$. This leads to the addition formula

$$
\int_{X} f d(\mu+\lambda)=\int_{X} f d \mu+\int_{X} f d \lambda
$$

which is valid (for instance) for every bounded measurable $f$.

We shall call a complex Borel measure $\mu$ on $X$ regular if $|\mu|$ is regular in the sense of Definition 2.15. If $\mu$ is a complex Borel measure on $X$, it is clear that the mapping

$$
f \rightarrow \int_{X} f d \mu
$$

is a bounded linear functional on $C_{0}(X)$, whose norm is no larger than $|\mu|(X)$. That all bounded linear functionals on $C_{0}(X)$ are obtained in this way is the content of the Riesz theorem:

6.19 Theorem If $X$ is a locally compact Hausdorff space, then every bounded linear functional $\Phi$ on $C_{0}(X)$ is represented by a unique regular complex Borel measure $\mu$, in the sense that

$$
\Phi f=\int_{X} f d \mu
$$

for every $f \in C_{0}(X)$. Moreover, the norm of $\Phi$ is the total variation of $\mu$ :

$$
\|\Phi\|=|\mu|(X)
$$

Proof We first settle the uniqueness question. Suppose $\mu$ is a regular complex Borel measure on $X$ and $\int f d \mu=0$ for all $f \in C_{0}(X)$. By Theorem 6.12 there is a Borel function $h$, with $|h|=1$, such that $d \mu=h d|\mu|$. For any sequence $\left\{f_{n}\right\}$ in $C_{0}(X)$ we then have

$$
|\mu|(X)=\int_{X}\left(\bar{h}-f_{n}\right) h d|\mu| \leq \int_{X}\left|\bar{h}-f_{n}\right| d|\mu|
$$

and since $C_{c}(X)$ is dense in $L^{1}(|\mu|)$ (Theorem 3.14), $\left\{f_{n}\right\}$ can be so chosen that the last expression in (3) tends to 0 as $n \rightarrow \infty$. Thus $|\mu|(X)=0$, and $\mu=0$. It is easy to see that the difference of two regular complex Borel measures on $X$ is regular. This shows that at most one $\mu$ corresponds to each $\Phi$.

Now consider a given bounded linear functional $\Phi$ on $C_{0}(X)$. Assume $\|\Phi\|=1$, without loss of generality. We shall construct a positive linear functional $\Lambda$ on $C_{c}(X)$, such that

$$
|\Phi(f)| \leq \Lambda(|f|) \leq\|f\| \quad\left(f \in C_{c}(X)\right)
$$

where $\|f\|$ denotes the supremum norm.

Once we have this $\Lambda$, we associate with it a positive Borel measure $\lambda$, as in Theorem 2.14. The conclusion of Theorem 2.14 shows that $\lambda$ is regular if $\lambda(X)<\infty$. Since

$$
\lambda(X)=\sup \left\{\Lambda f: 0 \leq f \leq 1, f \in C_{c}(X)\right\}
$$

and since $|\Lambda f| \leq 1$ if $\|f\| \leq 1$, we see that actually $\lambda(X) \leq 1$.

We also deduce from (4) that

$$
|\Phi(f)| \leq \Lambda(|f|)=\int_{X}|f| d \lambda=\|f\|_{1} \quad\left(f \in C_{c}(X)\right)
$$

The last norm refers to the space $L^{1}(\lambda)$. Thus $\Phi$ is a linear functional on $C_{c}(X)$ of norm at most 1 , with respect to the $L^{1}(\lambda)$-norm on $C_{c}(X)$. There is a normpreserving extension of $\Phi$ to a linear functional on $L^{1}(\lambda)$, and therefore Theorem 6.16 (the case $p=1$ ) gives a Borel function $g$, with $|g| \leq 1$, such that

$$
\Phi(f)=\int_{X} f g d \lambda \quad\left(f \in C_{c}(X)\right)
$$

Each side of (6) is a continuous functional on $C_{0}(X)$, and $C_{c}(X)$ is dense in $C_{0}(X)$. Hence (6) holds for all $f \in C_{0}(X)$, and we obtain the representation (1) with $d \mu=g d \lambda$.

Since $\|\Phi\|=1,(6)$ shows that

$$
\int_{X}|g| d \lambda \geq \sup \left\{|\Phi(f)|: f \in C_{0}(X),\|f\| \leq 1\right\}=1
$$

We also know that $\lambda(X) \leq 1$ and $|g| \leq 1$. These facts are compatible only if $\lambda(X)=1$ and $|g|=1$ a.e. [ $\lambda$ ]. Thus $d|\mu|=|g| d \lambda=d \lambda$, by Theorem 6.13, and

$$
|\mu|(X)=\lambda(X)=1=\|\Phi\|
$$

which proves (2).

So all depends on finding a positive linear functional $\Lambda$ that satisfies (4). If $f \in C_{c}^{+}(X)$ [the class of all nonnegative real members of $\left.C_{c}(X)\right]$, define

$$
\Lambda f=\sup \left\{|\Phi(h)|: h \in C_{c}(X),|h| \leq f\right\} .
$$

Then $\Lambda f \geq 0, \Lambda$ satisfies (4), $0 \leq f_{1} \leq f_{2}$ implies $\Lambda f_{1} \leq \Lambda f_{2}$, and $\Lambda(c f)=c \Lambda f$ if $c$ is a positive constant. We have to show that

$$
\Lambda(f+g)=\Lambda f+\Lambda g \quad\left(f \text { and } g \in C_{c}^{+}(X)\right)
$$

and we then have to extend $\Lambda$ to a linear functional on $C_{c}(X)$.

Fix $f$ and $g \in C_{c}^{+}(X)$. If $\epsilon>0$, there exist $h_{1}$ and $h_{2} \in C_{c}(X)$ such that $\left|h_{1}\right| \leq f,\left|h_{2}\right| \leq g$, and

$$
\Lambda f \leq\left|\Phi\left(h_{1}\right)\right|+\epsilon, \quad \Lambda g \leq\left|\Phi\left(h_{2}\right)\right|+\epsilon .
$$

There are complex numbers $\alpha_{i},\left|\alpha_{i}\right|=1$, so that $\alpha_{i} \Phi\left(h_{i}\right)=\left|\Phi\left(h_{i}\right)\right|, i=1,2$. Then

$$
\begin{aligned}
\Lambda f+\Lambda g & \leq\left|\Phi\left(h_{1}\right)\right|+\left|\Phi\left(h_{2}\right)\right|+2 \epsilon \\
& =\Phi\left(\alpha_{1} h_{1}+\alpha_{2} h_{2}\right)+2 \epsilon \\
& \leq \Lambda\left(\left|h_{1}\right|+\left|h_{2}\right|\right)+2 \epsilon \\
& \leq \Lambda(f+g)+2 \epsilon,
\end{aligned}
$$

so that the inequality $\geq$ holds in (10).

Next, choose $h \in C_{c}(X)$, subject only to the condition $|h| \leq f+g$, let $V=\{x: f(x)+g(x)>0\}$, and define

$$
\begin{aligned}
& h_{1}(x)=\frac{f(x) h(x)}{f(x)+g(x)}, \quad h_{2}(x)=\frac{g(x) h(x)}{f(x)+g(x)} \quad(x \in V), \\
& h_{1}(x)=h_{2}(x)=0 \quad(x \notin V) .
\end{aligned}
$$

It is clear that $h_{1}$ is continuous at every point of $V$. If $x_{0} \notin V$, then $h\left(x_{0}\right)=0$; since $h$ is continuous and since $\left|h_{1}(x)\right| \leq|h(x)|$ for all $x \in X$, it follows that $x_{0}$ is a point of continuity of $h_{1}$. Thus $h_{1} \in C_{c}(X)$, and the same holds for $h_{2}$.

Since $h_{1}+h_{2}=h$ and $\left|h_{1}\right| \leq f,\left|h_{2}\right| \leq g$, we have

$$
|\Phi(h)|=\left|\Phi\left(h_{1}\right)+\Phi\left(h_{2}\right)\right| \leq\left|\Phi\left(h_{1}\right)\right|+\left|\Phi\left(h_{2}\right)\right| \leq \Lambda f+\Lambda g .
$$

Hence $\Lambda(f+g) \leq \Lambda f+\Lambda g$, and we have proved (10).

If $f$ is now a real function, $f \in C_{c}(X)$, then $2 f^{+}=|f|+f$, so that $f^{+} \in$ $C_{c}^{+}(X)$; likewise, $f^{-} \in C_{c}^{+}(X)$; and since $f=f^{+}-f^{-}$, it is natural to define

$$
\Lambda f=\Lambda f^{+}-\Lambda f^{-} \quad\left(f \in C_{c}(X), f \text { real }\right)
$$

and

$$
\Lambda(u+i v)=\Lambda u+i \Lambda v .
$$

Simple algebraic manipulations, just like those which occur in the proof of Theorem 1.32, show now that our extended functional $\Lambda$ is linear on $C_{c}(X)$. This completes the proof.


\end{document}