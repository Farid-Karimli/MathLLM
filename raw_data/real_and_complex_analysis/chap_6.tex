CHAPTER SIX COMPLEX MEASURES Total Variation 6.1 Introduction Let o be a g-algebra in a set on ${\mathfrak{M}}$ is then a complex function on whenever $i\neq j,$ and if $\scriptstyle{E=}$ U E;. A complex measure ${\boldsymbol{\mu}}$ $\boldsymbol{E}$ $X.$ Call a countable collection $(t_{n}|$ of members of のa partition of E if $E_{i}\cap E_{j}=\varnothing$ such that ${\mathfrak{M}}$ $$ \mu({\cal E})=\sum_{i=1}^{\infty}\mu({\cal E}_{i})\qquad({\cal E}\in\mathfrak{O}) $$ (1) for every partition $(E_{n})$ of $\textstyle E.$ is now part of the require- Observe that the convergence of the series in $\mathbf{(1)}$ ment (unlike for positive measures, where the series could either converge or diverge to o). Since the union of the sets $\textstyle E_{i}$ is not changed if the subscripts are permuted, every rearrangement of the series (1) must also converge. Hence ([26], Theorem 3.56) the series actually converges absolutely Let us consider the problem of finding a positive measure $\lambda$ which dominates a given complex measure ${\boldsymbol{\mu}}$ on D ${\mathfrak{M}}.$ in the sense that $|\mu(E)|\leq\lambda(E)$ for every $E\in{\mathfrak{M}},$ and let us try to keep $\lambda$ as small as we can. Every solution to our problem $(\mathrm{iff}$ there is one at all) must satisfy $$ \lambda(E)=\sum_{i=1}^{\infty}\lambda(E_{i})\geq\sum_{1}^{\infty}\left|\,\mu(E_{i})\right|, $$ (2) for every partition $\{E_{i}\}$ of any set $E\in{\mathfrak{M}},$ so that $\scriptstyle{d E}$ is at least equal to the supremum of the sums on the right of (2), taken over all partitions of $\textstyle E.$ This suggests that we define a set function |p| on O by $$ |\mu|(E)=\operatorname*{sup}\sum_{i=1}^{\infty}\;|\,\mu(E_{i})|\qquad(E\in{\mathfrak{M}}), $$ (3) the supremum being taken over all partitions $(E_{n})$ of E. 116CoMPLEX MEASUREs 117 This notation is perhaps not the best, but it is the customary one. Note that $|\mu|(E)\geq|\mu(E)|\,.$ It turns out, as will be proved below,that $\left|\,\mu\right|$ is not equal to|p(E)I but that in general $\mu|(E)$ actually is a measure, so that our problem does have a solution. The discussion which led to (3) shows then the property $\lambda(E)\geq\left|\mu\right|(E)$ for all $\scriptstyle E\in\mathbb{N}$ is the minimal solution, in the sense that any other solution 2 has or sometimes, to avoid clearly that $\left|\,\mu\right|$ The set function $|\,\mu\,|\,$ is called the total variation of ${\boldsymbol{\mu}},$ misunderstanding, the total variation measure. The term “total variation of $\mu^{\nu}\/$ is also frequently used to denote the number |p|(X) If ${\boldsymbol{\mu}}$ is a positive measure, then of course $|\mu|=\mu,$ $|\mu|(X)<\infty.$ Besides being a measure, Ipl has another unexpected property: Since $|\,\mu(E)|\leq|\,\mu\,|\,(E)\leq|\,\mu\,|\,(X),$ this implies that every complex measure ${\boldsymbol{\mu}}$ on any o-algebra is bounded: If the range of ${\boldsymbol{\mu}}$ lies in the complex plane,then it actually lies in some disc of finite radius. This property (proved in Theorem 6.4) is sometimes expressed by saying that $\boldsymbol{\mu}$ is of bounded variation. measure on D 6.2 Theorem The total variation $\left|\,\mu\right|$ of a complex measure $\boldsymbol{\mu}$ on ${\mathfrak{M}}$ is a positive PROOF Let $\scriptstyle\{E_{n}\}$ be a partition of $\;E\in{\mathfrak{M}}$ Let $t_{i}$ be real numbers such that $t_{i}<\mid\mu\mid(E_{i})$ Then each $E_{i}$ has a partition $\{A_{i j}\}$ such that $$ \sum_{j}|\mu(A_{i j})|>t_{i}\qquad(i=1,\,2,\,3,\,\ldots){\mathrm{.}} $$ (1) Since $\{A_{i j}\}\;(i,j=1,\,2,\,3,\dots)$ is a partition of ${\boldsymbol{E}},$ it follows that $$ \sum_{i}t_{i}\leq\sum_{i,j}|\mu(A_{i j})|\leq|\mu|(E). $$ (2) Taking the supremum of the left side of (2), over all admissible choices of {t;}, we see that $$ \sum_{i}|\mu|(E_{i})\leq|\mu|(E). $$ (3) partition of $E_{i}.$ Hence To prove the opposite inequality, let $\{A_{j}\}$ be any partition of $E.$ Then for is a any fixed $j,$ $\{A_{j}\cap E_{i}\}$ is a partition of $A_{j},$ and for any fixed i $\{A_{j}\cap E_{i}\}$ $$ \begin{array}{c c}{{\sum_{j}\left|\mu(A_{j})\right|=\sum_{j}\left|\sum_{i}\mu(A_{j}\cap E_{i})\right|}}&{{\qquad}}\\ {{\leq\sum_{j}\sum_{i}\left|\mu(A_{j}\cap E_{i})\right|\leq\sum_{i}\left|\mu\right|(E_{i}).}}\end{array} $$ (4)118 REAL AND coMPLEX ANALYsis Since (4) holds for every partition $\{A_{j}\}\circ\mathbb{E},$ we have $$ |\,\mu\,|\,(E)\leq\sum_{i}|\,\mu\,|\,(E_{i}). $$ (5) By (3) and (5),|ul is countably additive Note that the Corollary to Theorem 1.27 was used in (2) and (4) Thatlplis not identically $\infty$ is a trivial consequence of Theorem 6.4 but can also be seen right now, since $|\mu|(\emptyset)=0.$ /// 6.3 Lemma If $z_{1},~\ldots,$ $z_{N}$ are complex numbers then there is a subset S of $\{1,\cdot\cdot,N\}$ for which $$ \left|\sum_{k\in S}z_{k}\right|\geq{\frac{1}{\pi}}\sum_{k=1}^{N}\,\left|\,z_{k}\right|. $$ PROOF Write $z_{k}=|z_{k}|\,e^{i\alpha_{k}}.$ For $-\pi\leq\theta\leq\pi,$ let $S(\theta)$ be the set of all $\boldsymbol{k}$ for which cos $(x_{k}-\theta)>0.$ Then $$ \left|\sum_{S(\theta)}z_{k}\right|=\left|\sum_{S(\theta)}e^{-i\theta}z_{k}\right|\geq\mathrm{Re~\sum_{S(\theta)}e^{-i\theta}}z_{k}=\sum_{k=1}^{N}|z_{k}|\cos^{+}\left(\alpha_{k}-\theta\right). $$ Choose $\theta_{0}$ so as to maximize the last sum, and put $S=S(\theta_{0}).$ This maximum is at least as large as the average of the sum over[一元,z], and this average is $\pi^{-1}\sum|z_{k}|,$ because $$ \frac{1}{2\pi}\left.\right|_{-\pi}^{\pi}\cos^{+}\left(\alpha-\theta\right)\,d\theta=\frac{1}{\pi} $$ for every α // 6.4 Theorem If p is a complex measure on $X,$ then $$ |\mu|(X)<\infty. $$ PRoOF Suppose first that Since|pl(E) >t, there is a partition $\{E_{i}\}$ of $|\mu|(E)=\infty.$ Put some set $E\in{\mathfrak{N}}$ has $t=\pi(1+|\mu(E)|).$ $\boldsymbol{E}$ such that $$ \sum_{i=1}^{N}|\mu(E_{i})|>t $$ for some $N.$ Apply Lemma 6.3, with $z_{i}=\mu(E_{i}),$ to conclude that there is a set $\scriptstyle A\in E$ (a union of some of the sets $E_{i})$ for which $$ |\mu(A)|>t/\pi>1. $$ Setting $B=E-A,{\mathrm{i}}$ t follows that $$ |\,\mu(B)|=|\,\mu(E)-\mu(A)|\geq|\,\mu(A)|-|\,\mu(E)|>\frac{t}{\pi}-|\,\mu(E)|=1. $$cOMPLEX MEASURES 119 We have thus split $\boldsymbol{E}$ into disjoint sets $\scriptstyle A$ and $\boldsymbol{B}$ with $|\mu(A)|>1$ and $\textstyle|\mu(B)|>1.$ Evidently, at least one of split $\textstyle X{\ ~}$ into ${\mathrm{cl}}_{n},B_{+}$ as above, with $|\mu(A_{1})|>1,$ Now if $\mu|(A)$ and $\mu|(B)$ is o,by Theorem 6.2. $|\mu|(X)=\infty$ $|\mu|(B_{1})=\infty.$ Split $B_{1}$ into ${\vec{u}}_{2},\,{\vec{u}}_{3}$ , with $|\mu(A_{2})|>1,|\mu|(B_{2})=\infty.$ Contin- $|\mu(A_{\partial}|>1$ uing in this way, we get a countably infinite disjoint collection u implies that $\scriptstyle\{4_{i}\}$ , with for each i. The countable additivity of ${\boldsymbol{\mu}}$ $$ \mu{\Big(}\bigcup_{i}A_{i}{\Big)}=\sum_{i}\mu(A_{i}). $$ But this series cannot converge, since $\scriptstyle{\mu(A)}$ does not tend to O asi→ 00. This // contradiction shows that $|\mu|(X)<\infty$ 6.5 f ${\boldsymbol{\mu}}$ and $\lambda$ are complex measures on the same o-algebra 奶, we define $\mu+\lambda$ and cp by $$ \begin{array}{l l}{{(\mu+\lambda)(E)=\mu(E)+\lambda(E)~~~~~~(E\in{\mathfrak{M}})}}&{{~~~~~(E\in{\mathfrak{M}})}}\end{array} $$ (1) for any scalar c, in the usual manner. It is then trivial to verify that $\mu+\lambda$ and cu are complex measures. The collection of all complex measures on ${\mathfrak{M}}$ is thus a vector space. If we put $$ \|\mu\|=|\mu|(X), $$ (2) it is easy to verify that all axioms of a normed linear space are satisfied 6.6 Positive and Negative Variations Let us now specialize and consider a real measure $\boldsymbol{\mu}$ on a ${\boldsymbol{\sigma}}\cdot$ algebra D.(Such measures are frequently called signed mea- sures.) Define $\mid\mu\mid$ as before, and define $$ \mu^{+}={\textstyle{\frac{1}{2}}}(|\mu|+\mu),\qquad\mu^{-}={\textstyle{\frac{1}{2}}}(|\mu|-\mu). $$ (1) Then both $\mu^{+}$ and $\mu^{-}$ are positive measures on ${\mathfrak{M}},$ and they are bounded, by Theorem 6.4. Also, $$ \mu=\mu^{+}-\mu^{-},\qquad|\,\mu|=\mu^{+}+\mu^{-}. $$ (2) The measures $\mu^{+}$ and $\mu^{-}$ are called the positive and negative variations of ${\boldsymbol{\mu}},$ respectively. This representation of ${\boldsymbol{\mu}}$ as the difference of the positive measures $\mu^{+}$ and $\mu^{-}$ is known as the Jordan decomposition of ${\boldsymbol{\mu}}.$ Among all representations of ${\boldsymbol{\mu}}$ as a difference of two positive measures,the Jordan decomposition has a certain minimum property which will be established as a corollary to Theorem 6.14.120 REAL AND coMPLEX ANALYSIS Absolute Continuity 6.7 Definitions Let $\boldsymbol{\mu}$ be a positive measure on a o-algebra O, and let $\lambda$ be an arbitrary measure on の;A may be positive or complex.(Recall that a complex measure has its range in the complex plane, but that our usage of the term“positive measure”includes oo as an admissible value. Thus the positive measures do not form a subclass of the complex ones.) We say that $\lambda$ is absolutely continuous with respect top ${\boldsymbol{\mu}},$ L, and write $$ \scriptstyle\varepsilon\,{\overset{\underset{\mathrm{a}}{}}}\,\!\cdot\,4\,\mu $$ (1) if $\lambda(E)=0$ for every $E\in{\mathfrak{M}}$ for which $\mu(E)=0.$ for every $E\in{\mathfrak{D}},$ we If there is a set Ae DR such that $\lambda(E)=\lambda(A\cap E)$ say that is concentrated on A. This is equivalent to the hypothesis that $\lambda(E)=0$ whenever $E\cap A=\varnothing.$ is concentrated on ${\boldsymbol{A}}$ and $\lambda_{2}$ is concen- Suppose $\lambda_{1}$ and $\lambda_{2}$ are measures on D, and suppose there exists a pair of disjoint sets $\scriptstyle A$ and $\boldsymbol{B}$ such that $\lambda_{1}$ trated on ${\boldsymbol{B}}.$ Then we say that $\lambda_{1}$ and $\lambda_{2}$ are mutually singular, and write $$ \lambda_{1}\perp\lambda_{2}\,. $$ (2) Here are some elementary properties of these concepts 6.8 Proposition Suppose,从、入 $\lambda_{1},$ and $\lambda_{2}$ are measures on a o-algebra D, and ${\boldsymbol{\mu}}$ is positive. (b)ir (a)If Ais concentrated on and $s_{2}\perp\mu$ , then $\lambda_{1}+\lambda_{2}\perp\mu.$ $A,$ so is|入|. $\lambda_{1}\:\perp\ \lambda_{2}\:,t h e n\:|\lambda_{1}|\;\perp\;|\lambda_{2}|\,.$ (c) 1 $\textstyle l\lambda_{1}\perp\mu$ and $\lambda_{2}\leqslant\mu,$ then $\lambda_{1}+\lambda_{2}\ll\mu.$ (a if $\lambda_{1}\ll\mu$ and $s_{2}\perp$ L, then $\lambda_{1}\perp\lambda_{2}$ (e)if $\lambda\ll\mu,\,t h e n\,|\,\lambda|\ll\mu.$ (f)If $q_{1}\ll\mu$ and 入工从 then $\scriptstyle a=0,$ (o if $\lambda\ll\mu$ PROOF (a)I $E\,\cap\,A=\mathcal{D}$ and $(E_{n})$ is any partition of $\textstyle E,$ then $\lambda(E_{j})=0$ for al $j.$ (b) Hence $|\lambda|(E)=0.$ This follows immediately from (a) (c) ${\boldsymbol{\mu}}$ $\scriptstyle{A_{1},\mu}$ There are disjoint sets and ${\boldsymbol{\mu}}$ on $B_{2}$ Hence $\lambda_{1}+\lambda_{2}$ is concentrated on $\lambda_{2}$ is concen- and $A_{1}$ and $B_{1}$ such that $\lambda_{1}$ is concentrated on $A_{1}$ trated on $A_{2}$ ,and there are disjoint sets $A_{2}$ and $B_{2}$ such that $A=A_{1}\ \cup$ on $B_{1}.$ is concentrated on $B=B_{1}\cap B_{2}\,,$ and $A\cap B=\varnothing.$ (e) (d)This is obvious. and $\{E_{j}\}$ is a partition of ${\boldsymbol{E}}.$ Then $\mu(s_{j})=0;$ and since Suppose $\mu(E)=0,$ for all j hence $\sum\left|\,\lambda(E_{j})\,\right|=0.$ This implies $\lambda|\langle E\rangle=0.$ $\lambda\ll\mu,\,\lambda(E_{i})=0$cOMPLEX MEASURES 121 (f)Since $\lambda_{2}\perp\mu,$ there is a set A with for every $\scriptstyle{E\in{A}}$ So $\lambda_{1}$ is concentrated on the Since 忒<从 $\lambda_{1}(E)=0$ $\mu(A)=0$ on which $\lambda_{2}$ is concentrated complement of $A.$ (g)By (J), the hypothesis of $\mathbf{\Psi}(g)$ implies, that $\lambda\perp\lambda,$ and this clearly forces $\lambda=0.$ / We come now to the principal theorem about absolute continuity. In fact, it is probably the most important theorem in measure theory. Its statement will involve ${\boldsymbol{\sigma}}\cdot$ finite measures. The following lemma describes one of their significant properties there is a function 6.9 Lemma If p is a positive o-finite measure on a o-algebra ${\mathfrak{M}}$ in a set $X,$ then $w\in L^{l}(\mu)$ such that $0<w(x)<1$ for every $\operatorname{xe}X;$ $X-E_{n}$ PRooF To say that ${\boldsymbol{\mu}}$ is c-finite means that $X$ is the union of countably many if xe sets $E_{n}\in{\mathfrak{M}}$ (n = 1, 2, 3, ..) for which $\mu(E_{n})$ is finite. Put $w_{n}(x)=0$ and put $$ w_{n}(x)=2^{-n}/(1\ +\mu(E_{n})) $$ ifxe $:E_{n}\,.$ Then $w=\sum_{i}^{\infty}\:w_{i}$ , has the required properties. // (namely, $d{\tilde{\mu}}=w\ d\mu)$ The point of the lemma is that ${\boldsymbol{\mu}}$ can be replaced by a finite measure $\boldsymbol{\vec{\mu}}$ which, because of the strict positivity of w, has precisely the same sets of measure $\mathbf{0}$ as ${\boldsymbol{\mu}}.$ 6.10 The Theorem of Lebesgue-Radon-Nikodym Let p be a positive o-finite measure on a o-algebra OR in a set $X_{\cdot}$ and let $\lambda$ be a complex measure on ${\mathfrak{M}}$ (a)There is then a unique pair of complex measures $\lambda_{a}$ ,and 入on OR such that $$ \lambda=\lambda_{a}+\lambda_{s},\qquad\lambda_{a}\ll\mu,\qquad\lambda_{s}\ \perp\ \mu $$ (1) If 入 is positive and finite, then so are $\lambda_{a}$ and $\lambda_{s}.$ (b))There is a unique $h\in L^{n}(\mu)$ such that $$ \lambda_{a}(E)=\left.\right|_{E}h\,d\mu $$ (2) for every set $F\in{\mathfrak{M}}$ The pair $(\lambda_{a},\,\lambda_{s})$ is called the Lebesgue decomposition of $(\lambda_{a}^{\prime},\,\lambda_{s}^{\prime})$ is another pair which ${\boldsymbol{\mu}},$ The $\lambda$ 2 relative to uniqueness of the decomposition is easily seen, for if satisfies (1), then $$ \lambda_{a}^{\prime}-\lambda_{a}=\lambda_{s}-\lambda_{s}^{\prime}, $$ (3) $\lambda_{a}^{\prime}-\lambda_{a}\leqslant\mu,$ and $\lambda_{x}-\lambda_{x}^{\prime}\perp\ \mu,$ hence both sides of (3) are ${\mathfrak{O}};$ we have used 6.8(c, 6.8(4), and 6.8(g)122 REAL AND coMPLEX ANALYsis The existence of the decomposition is the significant part of (a). Assertion (b) is known as the Radon-Nikodym theorem. Again, uniqueness of h is immediate, from Theorem 1.39(b).Also,if h is any member of $\scriptstyle T_{\theta\mathrm{{a}}}$ the integral in (2) defines a measure on 领(Theorem 1.29) which is clearly absolutely continuous with respect to $\scriptstyle\epsilon_{\cdot}\!\cdot\!\cdot\!\mu$ (in which case $\lambda_{a}=\lambda\rangle$ is obtained in this way ${\boldsymbol{\mu}}$ . The point of the Radon-Nikodym theorem is the converse: Every The function ${\boldsymbol{h}}$ which occurs in (2) is called the Radon-Nikodym derivative of $\lambda_{a}$ with respect to ${\boldsymbol{\mu}}.$ As noted after Theorem 1.29, we may express (2) in the form $d\lambda_{a}=h$ dp, or even in the form $h=d\lambda_{a}/d\mu.$ due to von Neumann The idea of te folloing proot which yields both (ay and (b) at one stroke,is PROOF Assume first that $\lambda$ l is a positive bounded measure on W. Associate w to $\boldsymbol{\mu}$ as in Lemma 6.9. Then $d\varphi=d\lambda+w\,d\mu$ defines a positive bounded measure $\varphi$ on 9の. The definition of the sum of two measures shows that $$ \bigcap_{X}f\,d\varphi=\bigcup_{X}f\,d\lambda+\bigcup_{X}^{}f w\,\,d\mu $$ (4) for $f_{=}\oslash\chi_{E},$ hence for simple $f_{}^{}$ hence for any nonnegative measurable $f.$ If fe $L^{2}(\varphi),$ the Schwarz inequality gives $$ \left|\int_{X}f\,d\lambda\right|\leq\left|_{x}|f|\ d\lambda\leq\left\{\right|_{X}|f|\ d\varphi\leq \{\bigcap_{X}|f\,|^{2}\ d\varphi\right\}^{1/2}\{\varphi(X)\}^{1/2}. $$ Since $\varphi(X)<\infty,$ we see that $$ f{ arrow}\int_{X}f\,d\lambda $$ (5) is a bounded linear functional on $L^{2}(\varphi).$ We know that every bounded linear functional on a Hilbert space ${\boldsymbol{H}}$ is given by an inner product with an element of H. Hence there exists a $g\in L^{2}(\varphi)$ such that $$ \ D_{x}^{\bullet}f\,d\lambda=\ {\stackrel{\bullet}{\bigcup_{X}}}f g\ d\varphi $$ (6) for every f∈ $L^{2}(\varphi).$ tence of ${\mathfrak{g}}.$ Observe how the completeness of $L^{2}(\varphi)$ was used to guarantee the exis Observe also that although $\scriptstyle{\mathcal{G}}$ is defined uniquely as an element of $L^{2}(\varphi),\,_{!}$ g is determined only a.e. [p] as a point function on $\varphi(E)>0.$ The left side of (6) is then $X.$ $\mathrm{Put}f=\chi_{E}$ in (6), for any $\boldsymbol{E}$ ∈ D with A(E), and since $0\leq\lambda\leq\varphi,$ we have $$ 0\leq\frac{1}{\varphi(E)} \vert^{\circ}g\,\,d\varphi=\frac{\lambda(E)}{\varphi(E)}\leq1. $$ (7cOMPLEX MEASURES 123 Hence g(x) ∈ [0,1] for almost all x(with respect to p),by Theorem 1.40. We may therefore assume tha $0\leq g(x)\leq1$ for every xe X, without affecting (6), and we rewrite (6) in the form $$ \bigcap_{x}(1-g)f\,d\lambda=\int_{X}^{\circ}f g w\,\,d\mu. $$ (8) Put $$ A=\{x\colon0\leq g(x)<1\},\qquad B=\{x\colon g(x)=1\}, $$ (9) and define measures $\lambda_{a}$ and $\lambda_{s}$ by $$ \lambda_{a}(E)=\lambda(A\cap E),\qquad\lambda_{s}(E)=\lambda(B\cap E), $$ (10) for all $E\in{\mathfrak{M}}$ in (8), the left side is O, the right side is Thus $\lambda_{s}\perp$ A w du. Since $w(x)>0$ Tif $f=\chi_{B}$ $\mu(B)=0.$ ${\mathfrak{f}}_{B}$ for all ${\mathfrak{X}},$ we conclude that Since $\scriptstyle{\mathcal{G}}$ is bounded,(8) holds if f is replaced by $$ (1+g+\cdots+g^{n})\chi_{E} $$ for $n=1,2,3,\ldots,E\in\Re.$ For such f,(8) becomes $$ \bigcap_{E}(1-g^{n+1})\,d\lambda=\bigcap_{\textstyle\mathbb{E}}g(1+g+\cdots\cdot+g^{n})w\,d\mu. $$ (11) At every point of ${\boldsymbol{B}},$ $\theta(x)=1,$ hence $1-g^{n+1}(x)=0.$ At every point of $A,$ $g^{n+1}(x)\to0$ monotonically. The left side of(11) converges therefore to $\lambda(A\cap E)=\lambda_{a}(E)$ as $n\to\varnothing.$ The integrands on the right side of (11) increase monotonically to a non negative measurable limit $h_{\mathrm{,}}$ , and the monotone convergence theorem shows see that that the right side of $(11)$ tends to $\mathbf{tr}\,h$ du as $\textstyle n\!\to\!\infty$ $\scriptstyle{E (\scriptstyle{\frac{1}{R}}\in{\mathfrak{M}}}$ Taking $E=X.$ we $h\in L^{n}(\mu),$ We have thus proved that $(2)$ holds for every since $\lambda_{\alpha}(X)<\alpha.$ Finally, (2) shows that $\lambda_{a}\ll\mu,$ and the proof is complete for positive with $\lambda_{1}$ and $\lambda_{2}$ p real $\lambda_{*}$ If Ais a complex measure on D, then $\lambda=\lambda_{1}+i\lambda_{2}$ and we can apply the preceding case to the positive and negative variations of $\lambda_{1}$ and $\lambda_{2}$ // If both ${\boldsymbol{\mu}}$ and $\lambda$ are positive and ${\boldsymbol{\sigma}}\cdot$ -finite, most of Theorem 6.10 is still true. $\boldsymbol{\mathit{h}}$ which satisfies 2,3, We can now write $X_{_{\circ}}=\bigcup X_{n,\circ}$ where $\mu(X_{n})<\,\infty$ and $\lambda(X_{n})<\infty,$ for $n=1,$ . The Lebesgue decompositions of the measures $\lambda(E arrow X_{n})$ still give us a Lebesgue decomposition of ${\boldsymbol{\lambda}},$ and we still get a function Eq.6.10(2); however, it is no longer true that $h\in L(\mu),$ although $\boldsymbol{h}$ is “locally in $L^{1}{}_{\!\cdot\!}$ i.e., $\textstyle\bigcap_{x_{n}}h_{..}d\mu<\infty_{!}$ for each n. -finiteness, we meet situations where the two theo Finally, if we go beyond ${\boldsymbol{\sigma}}.$ on (0,1),and let $\lambda$ 元 rems under consideration actually fail. For example, let ${\boldsymbol{\sigma}}.$ -algebra of all Lebesgue be the counting measure on the ${\boldsymbol{\mu}}$ be Lebesgue measure124 REAL AND coMPLEX ANALYSiS measurable sets in (0,1).Then A has no Lebesgue decomposition relative to ${\boldsymbol{\mu}},$ and although $\mu\ll\lambda$ and p is bounded,there is no $h\in L^{n}(\lambda)$ such that $d\mu=h$ dA.We omit the easy proof. The following theorem may explain why the word“continuity”is used in connection with the relation $\scriptstyle\epsilon_{\cdot}\!\cdot\!\cdot\!\mu$ and $\lambda$ 6.11 Theorem Suppose ${\boldsymbol{\mu}}$ and $\lambda$ are measures on a o-algebra D 以 is positive is complex. Then the following two conditions are equivalent: $\boldsymbol{\mu}$ (a)入<从 corresponds a $\scriptstyle\delta>0$ such that $\lambda(E)\vert<\epsilon$ for al $\scriptstyle E\in{\mathfrak{M}}$ (b)To every $\scriptstyle{\epsilon\;>0}$ with $\mu(E)<\delta$ Propertyb)is sometimes used as the definition of absolute continuity However,(a) does not imply (b) ifAis a positive unbounded measure. For instance, let ${\boldsymbol{\mu}}$ be Lebesgue measure on (0,1), and put $$ \lambda(E)=\left[{}_{E}^{t}{}^{-1}~d t\right. $$ for every Lebesgue measurable set $E\subset(0,\;1).$ $\lambda(E)\vert<\epsilon$ PROOF Suppose (b) holds. If $\mu(E)=0,$ then $\mu(E)<\delta$ for every $\delta>_{0}$ hence for every $\scriptstyle\epsilon\;>0,$ so $\lambda(E)=0.$ Thus (b) implies (a_ Put Suppose (b) is false. Then there exists an $\scriptstyle\epsilon\;>0$ and there exist sets $E_{n}$ E D $(n=1,\,2,\,3,\,\ldots)$ such that $\mu(E_{n})<2^{-n}$ but $|\lambda(E_{n})|\geq\epsilon.$ Hence $|\lambda|(E_{n})\geq\epsilon.$ $$ A_{n}=\bigcup_{i=n}^{\infty}E_{i},\qquad A=\bigcap_{n=1}^{\infty}A_{n}. $$ (1) Then $\mu(A_{n})<2^{-n+1},$ $A_{n}\supset A_{n+1},$ and so Theorem 1.19(e) shows that $\mu(A)=0$ and that $$ |\lambda|(A)=\operatorname*{lim}_{n arrow\infty}|\lambda|(A_{n})\geq\epsilon>0, $$ since $|\lambda|(A_{n})\geq|\lambda|(E_{n}).$ hence (a) is false, by Proposition / It follows that we do not have $|\lambda|\ll\mu,$ 6.8(e). Consequences of the Radon-Nikodym Theorem 6.12 Theorem Let p be $\bar{a}$ complex measure on a o-algebra OR in $X.$ Then there is a measurable function $\boldsymbol{h}$ such that $|\,h(x)|\,=\,1$ for all $\scriptstyle{\mathcal{X}}$ E $X$ and such that $$ d\mu=h\,d|\mu|\,. $$ (1)cOMPLEX MEASUREs 125 By analogy with the representation of a complex number as the product of ts absolute value and a number of absolute value 1, Eq.(1) is sometimes referred o as the polar representation (or polar decomposition) of ${\boldsymbol{\mu}},$ Let $A_{r}=\{x\colon|h(x)|<r\},$ where ${\boldsymbol{r}}$ and therefore the Radon-Nikodym theorem be PR0OF ${\mathrm{It}}$ is trivial that Then $\mu\ll|\mu|,$ which satisfies (1) $\left(E_{n}\right)$ guarantees the existence of some $h\in L^{1}(\mid\mu\mid)$ is some positive number, and let a partition of $A_{r}$ $$ \sum_{j}|\mu(E_{j})|=\sum_{j}\left|\[\right|_{E_{j}}h\left|\mu\right| |\leq\sum_{j}r\left|\mu\right|(E_{j})=r\left|\mu\right|(4_{r}), $$ so that $\left|\,\mu\right|(A_{r})\leq r\,|\,\mu\,|\,(A_{r}).$ If $\scriptstyle\gamma=1$ ,this forces $|\mu|\left(A_{r}\right)=0.$ Thus $|h|\geq1$ a.e. On the other hand, f $\left|\,\mu\right|(E)>0,(1)$ shows that $$ \left|\frac{1}{\mid\mu\mid(E)}\right|_{E}h\left.\right|=\frac{\mid\mu(E)\mid}{\mid\mu\mid(E)}\leq1. $$ We now apply Theorem 1.40(with the closed unit disc in place of S) and conclude that $|h|\leq1$ a.e. Let $B=\{x\in X\colon|\,h(x)|\neq1\}.$ $h(x)=1$ on ${\boldsymbol{B}},$ we obtain a function with the desired and if we redefine $\boldsymbol{h}$ on $\boldsymbol{B}$ We have shown that $|\mu|(B)=0,$ so that properties // 6.13 Theorem Suppose $\boldsymbol{\mu}$ is a positive measure on D, $g\in L^{(}|\mu),a n$ id $$ \lambda(E)= [{g}~d\mu\qquad(E\in\mathfrak{M}). $$ (1) Then $$ |\lambda\,|\,(E)=\int_{E}|\,g\,|\,d\mu\qquad(E\in\Re). $$ (2) PRoOF By Theorem 6.12, there is a function $h_{\mathrm{,}}$ of absolute value $\mathbf{I},$ such that $d\lambda=h$ $|i\lambda|.$ By hypothesis, $d\lambda=g$ du. Hence $$ h~d|\lambda|=g~d\mu. $$ This gives $d|\lambda|=\hbar g\cdot$ du.(Compare with Theorem 1.29.) a.e.[u],so that $\hbar g=\left|g\right|$ // Since $\scriptstyle{|{\mathcal{A}}|\leq0}$ and $\mu\geq0,$ it follows that $\scriptstyle h y\geq0$ a.e. [L] algebra の in a set 6.14 The Hahn Decomposition Theorem Let ${\boldsymbol{\mu}}$ be a real measure on ao- $X.$ Then there exist sets A and Be DR such that126 REAL AND cOMPLEX ANALYSIS $\mu^{+}$ and $A\ \cup\ B=X,\ A\cap B=\emptyset,$ and such that the positive and negative variations $\mu^{-}$ of ${\boldsymbol{\mu}}$ satisy $$ \mu^{+}(E)=\mu(A\;\cap\;E),\;\;\;\mu^{-}(E)=\;-\,\mu(B\;\cap\;E)\;\;\;\;\;\;\;(E\in\Re|). $$ (1) In other words, $X$ is the union of two disjoint measurable sets ${\boldsymbol{A}}$ and ${\boldsymbol{B}},$ such $E\subset A]$ that“A carries all the positive mass of u”[since(1))implies that $\mu(E)\geq0$ if and“B carries all the negative mass of $\mu^{\nu}\!^{\nu}$ [since $\mu(E)\leq0$ if $E=B_{1}$ ] The pair (A, B) is called a Hahn decomposition of $X.$ induced by ${\boldsymbol{\mu}}.$ PRoOF By Theorem 6.12 $\scriptstyle d\mu=h$ d|p, where $|h|=1.$ Since p ${\boldsymbol{\mu}}$ is real, it fol- lows that $\boldsymbol{\mathit{h}}$ is real (a.e., and therefore everywhere, by redefining on a set of measure O), hence $h=\pm1.$ Put $$ A=\{x\colon h(x)=1\},\qquad B=\{x\colon h(x)=-1\}. $$ (2) Since $\mu^{+}={\frac{1}{2}}(|\mu|+\mu),$ and since $$ {\textstyle\frac{1}{2}}(1+h)={\binom{h}{0}}\quad{\mathrm{on~}}A, $$ (3) we have, for any $E\in{\mathfrak{M}},$ $$ \mu^{+}(E)=\frac{1}{2}\left[_{E}(1+h)\,d\,|\,\mu|=\right]_{E\,\cdot\,A}^{\bullet}h\,d|\,\mu|=\mu(E\,\cap A). $$ (4) Since $\mu(E)=\mu(E\cap A)+\mu(E\cap B)$ and since $\mu=\mu^{+}-\mu^{-},$ the second half of // (1) follows from the first and Corollary $U/\mu=\lambda_{1}-\lambda_{2}\,,$ where $\lambda_{1}$ and $\lambda_{2}$ are positive measures, then $\lambda_{1}\geq\mu^{*}$ $\lambda_{2}\geq\mu$ This is the minimum property of the Jordan decomposition which was men- tioned in Sec. 6.6. PROOF Since $\mu\leq\lambda_{1},$ we have $$ \mu^{+}(E)=\mu(E\cap A)\leq\lambda_{1}(E\cap A)\leq\lambda_{1}(E). $$ // Bounded Linear Functionals on ${\boldsymbol{L}}^{p}$ $\Phi_{g}$ 6.15 Let ${\boldsymbol{\mu}}$ be a positive measure, suppose $1\leq p\leq\infty.$ and let q $\boldsymbol{\mathit{q}}$ be the exponent and if is defined by conjugate to p. The H6lder inequality (Theorem 3.8) shows that if $g\in D(\mu)$ $$ \Phi_{g}(f)= \{_{x}^{}f g\;d\mu, $$ (1)COMPLEX MEASURE $127$ then $\Phi_{g}$ is a bounded linear functional on $P_{(i)i}$ of norm at most $\|g\|_{q}.$ . The ques- tion naturally arises whether all bounded linear functionals on $\scriptstyle T_{(i,j)}$ have this form, and whether the representation is unique For $p=\,\infty,$ Exercise 13 shows that the answer is negative: $\scriptstyle I_{\theta\theta}^{1}$ does not furnish all bounded linear functionals on LP(m). For $1<p<\infty,$ the answer is affirmative. It is also affirmative for $\scriptstyle p=1,$ provided certain measure-theoretic pathologies are excluded. For ${\boldsymbol{\sigma}}\cdot$ -finite measure spaces, no dificulties arise, and we shall confine ourselves to this case. 6.16 Theorem Suppose $1\leq p<\infty,$ ${\boldsymbol{\mu}}$ is a o-finite positive measure on $X,$ and $\ \Phi$ is a bounded linear functional on $\scriptstyle{H_{(i)}}$ Then there is a unique $g\in E(\mu),$ where $\boldsymbol{\mathit{q}}$ is the exponent conjugate to ${\boldsymbol{p}},$ such that $$ \Phi(f)= \{_{X}^{}f g\;d\mu\ {\qquad(f\in D(\mu)),} $$ (1) Moreover, if D and $\scriptstyle{\mathcal{G}}$ are related as in $(1),$ we have $$ \|\Phi\|=\|g\|_{q}. $$ (2) In other words, $\scriptstyle R(y)$ is isometrically isomorphic to the dual space of $\scriptstyle P_{(0)}$ under the stated conditions. PxOOF The uniqueness of $\scriptstyle{\mathcal{G}}$ is clear, for if $\scriptstyle{\mathcal{G}}$ and ${\mathfrak{g}}^{\prime}$ satisfy (1), then the integral of g - g' over any measurable set finitess o ${\boldsymbol{\mu}}$ implies therefore that $g-g^{\prime}=0$ a.e. xgfor ) and the $\boldsymbol{E}$ of finite measure is $\mathbf{0}$ (as we see by taking ${\boldsymbol{\sigma}}.$ Next, if (1) holds, HBlder's inequality implies $$ \|\Phi\|\leq\|g\|_{q}. $$ (3) So it remains to prove that $\scriptstyle{\mathcal{G}}$ exists and that equality holds in (3). 1 $|\Phi||=0.$ (1) and (2) hold with $\scriptstyle g=0$ So assume $|\Phi|>0.$ We first consider the case $\mu(X)<\infty$ For any measurable set $E\in X.$ define $$ \lambda(E)=\Phi(\chi_{E}). $$ Since $\mathbf{\Phi}\Phi$ is additive. To prove countable additivity, suppose $\boldsymbol{E}$ are disioint, we see that $\lambda$ is linear, and since $\chi_{A}\cup B=\chi_{A}+\chi_{B}$ if $\scriptstyle A$ and $\boldsymbol{B}$ is the union of countably many disjoint measurable sets $E_{i},$ put $A_{k}=E_{1}\;\cup\;\cdots\;\cup\;E_{k},$ and note that $$ \|\chi_{E}-\chi_{A_{k}}\|_{p}=[\mu(E-A_{k})]^{1/p} arrow0\qquad(k arrow\infty); $$ (4) $\left[\mathbf{In}~(4)\right]$ the assumption $p<\infty$ was used.] ${\mathrm{It}}$ is clear that $\lambda(E)=0$ is a complex measure $\mu(E)=0,$ the continuity of $\ \Phi$ shows now that $\lambda(A_{k})\to\lambda(E).$ So $\lambda$ if128 REAL AND coMPLEX ANALYSIs since then $\|{\boldsymbol{\tau}}_{E}\|_{p}=0.$ Thus $\scriptstyle\lambda\smallsetminus\mu,$ and the Radon-Nikodym theorem ensures $E\in X.$ the existence of a function $g\in L^{n}(\mu)$ such that, for every measurable $$ \Phi(\chi_{E})=\left\{_{E}^{}g\;d\mu=\right\}_{x}^{}\gamma_{E}g\;d\mu. $$ (5) By linearity it follows that $$ \Phi(f)=\int_{X}f g\ d\mu $$ (6) $\operatorname{every}f\in L^{\infty}(\mu)$ holds for every simple measurable $f,$ f, and so also for every $f\in L^{n}(\mu),$ since is a uniform limit of simple functions f. Note that the uniform convergence of f, tof implies $\|f_{i}-f\|_{p}\to0,$ hence $\Phi(f_{i})\to\Phi(f),\,\mathbf{as}\,i.$ 0o We want to conclude that $g\in D(\mu)$ and that (2) holds; it is best to split the argument into two cases CAsE 1 $p=1.$ Here (5) shows that $$ \left|\int_{E}g\ d\mu\right|\leq||\Phi|\cdot||\chi_{E}||_{1}=\|\Phi|\ \cdot\ \mu(E) $$ for every $\boldsymbol{E}$ e D. By Theorem $1.40,|\,g(x)|\leq\|\Phi\|$ a.e., so that $\|g\|_{\alpha}\leq\|\Phi\|.$ CASE $\chi_{E_{n}}|\,g\,|^{q-1}\alpha.$ Then $|f|^{p}=|g|^{*}$ There is a measurable function α $|\alpha|=1,$ such that $f=1$ $2\ 1<p<\infty.$ $E_{n}=\{x;\ |\ g(x)|\leq n\},$ and define cαg =l9l [Proposition 19()]、Let on $E_{n},f\in L^{\alpha}(\mu),\,{\mathrm{al}}$ nd (6) gives $$ \left|\O_{E_{n}}\right|_{\displaystyle B|}g\left|^{q}\,d\mu=\oplus(f)\leq\left|\bar{\Phi}\right|\!\left\{\right|_{E_{n}}\right|_{\displaystyle B^{+}} \}\!\!\!\!\!\!-1_{\displaystyle E_{n}}\left|\vphantom{D}_{\displaystyle B}\right|^{\!1}\!\!,\bot\cdots $$ so that $$ \int_{x}\gamma_{E_{n}}\vert g\vert^{q}\,d\mu\leq\vert\Phi\vert^{q}\qquad(n=1,2,3,\ldots). $$ (7) If $\mu(X)=\infty$ but ${\boldsymbol{\mu}}$ is ${\boldsymbol{\sigma}}\cdot$ If we apply the monotone convergence theorem to (7), we obtain $\|g\|_{q}\leq\|\Phi\|.$ $\scriptstyle{H_{(i)}}$ hence they coincide on all of $\scriptstyle I_{(i,j)}$ finite, choose w ∈ $\scriptstyle T(\omega)$ . It follows that both sides of (6) are contin- of Thus (2) holds and $g\in G^{n}(\mu)$ $\scriptstyle{E(y_{i})}$ uous functions on $\scriptstyle{P(\omega)}$ They coincide on the dense subset and this completes the proof $\operatorname{lf}\mu(X)<\infty.$ as in Lemma 6.9. Then dp = w du defines a finite measure on D, and $$ F arrow w^{1/p}F $$ (8) Hence is a linear isometry of $\scriptstyle R(z)$ onto $\scriptstyle H_{\mathrm{(g)}}$ because $w(x)>0$ for every xe X. $$ \Psi(F)=\Phi(w^{1/p}F) $$ (9) defines a bounded linear functional $\Psi$ on $\scriptstyle I(\theta).$ with $\|\Psi\|=\|\Phi\|.$coMPLEX MEASUREs 129 The first part of the proof shows now that there exists $G\in E(\#)$ such that $$ \Psi(F)=\int_{X}\!F G\ d{\tilde{\mu}}\qquad(F\in L^{p}({\tilde{\mu}})). $$ (10) Put $g=w^{1/q}G.(\mathrm{ff}\,p=1,g=G.)\,]$ Then $$ \{\oint_{X}\vert g\,\vert^{q}\,\,d\mu=\int_{X}\vert G\,\vert^{q}\,\,d\tilde{\mu}=\vert\Psi\vert\vert^{q}=\vert\Phi\vert\vert^{q} $$ (11) if $p>1.$ , whereas $\|g\|_{\infty}=\|G\|_{\infty}=\|\Psi\|=\|\Phi\|$ if $\scriptstyle{p=1}$ Thus (2) holds, and since ${\boldsymbol{G}}$ $d{\tilde{\mu}}=w^{1/p}g$ du, we finally get $$ \Phi(f)=\Psi(w^{-1/p}f)=\left\{_{X}w^{-1/p}f G\ d\tilde{\mu}= (\tilde{h}_{X}f g\ d\mu\right) $$ (12) for every f∈ LE(u). // 6.17 Remark We have already encountered the special case $p=q=2$ of Theorem 6.16. In fact, the proof of the general case was based on this special case, for we used the knowledge of the bounded linear functionals on $\scriptstyle{\vec{c}}(t_{i0})$ in the proof of the Radon-Nikodym theorem, and the latter was the key to the completeness of proof of Theorem 6.16. The special case $p=2,$ in turn, depended on the $\scriptstyle{E(\omega)}$ on the fact that $\scriptstyle{E(n)}$ is therefore a Hilbert space, and on the fact that the bounded linear functionals on a Hilbert space are given by inner products We now turn to the complex version of Theorem 2.14. The Riesz Representation Theorem 6.18 Let $\textstyle X{\mathrm{~}}$ be a locally compact Hausdorff space. Theorem 2.14 characterizes the positive linear functionals on $\scriptstyle C_{i}(X)$ We are now in a position to characterize the bounded linear functionals $\Phi$ on $C_{c}(X).$ Since $C_{c}(X)$ is a dense subspace of $\mathbb{C}_{d}X_{1}$ relative to the supremum norm, every such p has a unique extension to a bounded linear functional on $\scriptstyle C_{i}|X\rangle$ Hence we may as well assume to begin with If $\boldsymbol{\mu}$ that we are dealing with the Banachrspace $\scriptstyle C_{d}|X\rangle$ is a complex Borel measure, Theorem 6.12 asserts that there is a complex Borel function $\boldsymbol{h}$ with $|\,h\,|\,=\,1$ such that $d\mu=h$ $d|\mu|\cdot\mathbf{H}$ is therefore reasonable to iefine integration with respect to a complex measure ${\boldsymbol{\mu}}$ by the formula $$ \{\,f\,d\mu= \{\,f h\,d\,|\,\mu\,|\,. $$ (1) The relationG $\chi_{E}\,d\mu=\mu(E)$ is a special case of (1). Thus $$ \Rightarrow_{x}^{}\chi_{E}~d(\mu+\lambda)=(\mu+\lambda)(E)=\mu(E)+\lambda(E)=\int_{x}\chi_{E}~d\mu+\left\{\gamma_{E}~d\lambda\right. . $$ (2)130 REAL AND coMPLEX ANALYSIs whenever ${\boldsymbol{\mu}}$ and 入are complex measures on ${\mathfrak{M}}$ and $\boldsymbol{E}$ ∈ D. This leads to the addition formula $$ \bigcap_{x}f\,d(\mu+\lambda)=\bigcap_{x}f\,d\mu+\prod_{x}^{^{\prime}}f\,d\lambda, $$ (3) which is valid (for instance) for every bounded measurable f We shall call a complex Borel measure $\boldsymbol{\mu}$ u on $X$ regular if $\left|\,\mu\right|$ is regular in the sense of Definition 2.15. If ${\boldsymbol{\mu}}$ is a complex Borel measure on $X,$ it is clear that the mapping $$ f arrow\int_{x}f\,d\mu $$ (4) is a bounded linear functional on $C_{0}(X),$ whose norm is no larger than lpl(X) That all bounded linear functionals on $C_{0}(X)$ are obtained in this way is the content of the Riesz theorem: 6.19 Theorem If X is a locally compact Hausdorff space, then every bounded linear functional $\Phi$ On $C_{0}(X)$ is represented $b y$ a unique regular complex Borel measure ${\boldsymbol{\mu}},$ in the sense that $$ \Phi f= [\int_{X}f\,d\mu $$ (1) for every fe Co(X). Moreover, the norm of $\Phi$ is the total variation of u $$ |\Phi|=|\mu|(X). $$ (2) 6.12 there is a Borel function PROOF We first settle the uniqueness question. Suppose and $[f\,d\mu=0$ for a $\|1f\in C_{\mathrm{o}}(X).$ ${\boldsymbol{\mu}}$ is a regular complex Borel measure on $\textstyle X$ By Theorem $h_{\mathrm{,}}$ , with $|h|=1,$ such that $d\mu=h\,d|\mu|.$ For any sequence $\{f_{n}\}$ in $C_{0}(X)$ we then have $$ |\,\mu\,|\,(X)=\int_{X}(\overline{{{h}}}-f_{n})h\;d\,|\,\mu\,|\leq \{\,\!\!_{X}|\,\overline{{{h}}}-f_{n}\,|\,d\,|\,\mu\,|\,, $$ (3) and since $C_{c}(X)$ is dense in $L^{1}(\mid\mu\mid)$ (Theorem 3.14) $\{f_{n}\}$ can be so chosen that $\mu=0.$ It the last expression in (3) tends to $\mathbf{0}$ as n→O0. Thus $|\mu|(X)=0,$ and is easy to see that the difference of two regular complex Borel measures on $X$ is regular. This shows that at most one pu corresponds to each D Now consider a given bounded linear functional $\Phi$ on $C_{0}(X).$ Assume $|\Phi||=1,$ without loss of generality. We shall construct a positive linear func- tional $\Lambda$ on $C_{c}(X),$ such that $$ |\Phi(f)|\leq\Lambda(|f|)\leq\|f\|\qquad(f\in C_{c}(X)), $$ (4) where $\|f\|$ denotes the supremum norm.cOMPLEX MEASURES 131 Once we have this A, we associate with it a positive Borel measure $\lambda,$ as in Theorem 2.14. The conclusion of Theorem 2.14 shows that $\lambda$ is regular if $\lambda(X)<\infty.$ Since $$ \lambda(X)=\operatorname*{sup}\;\{\Lambda f;\,0\leq f\leq1,f\in C_{c}(X)\} $$ and since $|N1|\leq1$ if $\|f\|\leq1,$ we see that actually $\lambda(X)\leq1$ We also deduce from (4) that $$ |\Phi(f)|\leq\Lambda(|f|)=\left.\right|_{x}^{\circ}|f|\ d\lambda=\|f\|_{1}\qquad(f\in C_{c}(X)). $$ (5) that The last norm refers to the space $\scriptstyle{I(\partial)}$ Thus $\Phi$ is a linear functional on $C_{c}(X)$ of norm at most 1, with respect to the $\Phi$ to a linear functional on -norm on $C_{c}(X).$ There is a norm- such preserving extension of $\scriptstyle{T(s)}$ $\scriptstyle{T(\partial)}$ and therefore Theorem 6.16(the case $p=1)$ gives a Borel function ${\mathfrak{g}},$ with $|g|\leq1,$ $$ \Phi(f)=\left\{_{X}^{\prime}f g\;d\lambda\qquad(f\in C_{c}(X)).\qquad\right. $$ (6) Each side of (6) is a continuous functional on C(X), and C.(X)is dense in with $d\mu=g\,$ Co(X). Hence (6) holds for ${\mathrm{all}}f\in C_{0}(X),$ and we obtain the representation(1) d元. Since $\|\Phi\|=1,(6)$ shows that $$ \left\{{}_{x}|g|\,d\lambda\geq\operatorname*{sup} \{|\Phi(f)|\,;f\in C_{0}(X),\,|f|\,\leq1\right\}=1. $$ (7) $\lambda(X)=1$ We also know that $\langle s(X)\leq1$ and $|g|\leq1.$ These facts are compatible only if and $|g|=1$ a.e. [2]. Thus $d\mid\mu\mid=\mid g\mid d\lambda=d\lambda,$ by Theorem 6.13, and $$ |\mu|(X)=\lambda(X)=1=\|\Phi\|, $$ (8) which proves (2) If fe So all depends on finding a positive linear functional A that satisfies (4) define $C_{c}^{+}(X)$ [the class of all nonnegative real members of $C_{c}(X)],$ $$ \Lambda f=\operatorname*{sup}\left\{|\Phi(h)|:h\in C_{c}(X),|h|\leq f\right\}. $$ (9) Then $\scriptstyle N\geq0,$ A satisfies (4) $0\leq f_{1}\leq f_{2}$ implies $\Lambda f_{1}\leq\Lambda f_{2},$ , and $\Lambda(c f)=c\Lambda f$ if c is a positive constant. We have to show that $$ \Lambda(f+g)=\Lambda f+\Lambda g\qquad(f\operatorname{and}g\in C_{c}^{*}(X)), $$ (10) Fix $\boldsymbol{\f}$ and we then have to extend $g\in C_{c}^{+}(X).$ If $\scriptstyle\epsilon\;>0,$ there exist $h_{1}$ and $h_{2}\in C_{c}(X)$ such that and $\Lambda$ to a linear functional on $C_{c}(X).$ $|h_{1}|\leq f,|h_{2}|\leq g,$ and $$ \Lambda f\leq |\,\Phi(h_{1})\,|\,+\,\epsilon,\qquad\Lambda g\leq|\,\Phi(h_{2})\,|\,+\,\epsilon. $$ (11)132 REAL AND COMPLEX ANALYSIs There are complex numbers $x_{i},\ |\alpha_{i}|=1,$ so that α,D(h) =|D(h)|,i= 1,2. Then $$ \begin{array}{l}{{\Lambda f+\Lambda g\leq|\Phi(h_{1})|+|\Phi(h_{2})|+2\epsilon}}\\ {{\,}}\\ {{\,}}\\ {{\,}}\\ {{\,}}\\ {{\le\Lambda(x_{1}h_{1}+\alpha_{2}h_{2})+2\epsilon}}\\ {{\le\Lambda(f+g)+2\epsilon,}}\end{array} $$ so that the inequality ≥ holds in (10). Next, choose $h\in C_{c}(X),$ subject only to the condition $|h|\leq f+g,$ let $V=\{x;f(x)+g(x)>0\},$ and define $$ h_{1}(x)=\frac{f(x)h(x)}{f(x)+g(x)},\qquad h_{2}(x)=\frac{g(x)h(x)}{f(x)+g(x)}\qquad(x\in V), $$ (12) $$ h_{1}(x)=h_{2}(x)=0\qquad(x\not\in V). $$ ${\mathrm{It}}$ $\scriptstyle x_{0}$ since ${\boldsymbol{h}}$ is a point of continuity of $h_{1}.$ is continuous at every point of $V.$ If $x_{0}\notin V,$ then $\hbar(x_{0})=0;$ $h_{2}$ is clear that $h_{\mathrm{1}}$ for all $\operatorname{xe}\,X;$ it follows that is continuous and since $|h_{1}(x)|\leq|h(x)|$ and the same holds for Thus $h_{1}\in C_{c}(X),$ Since $h_{1}+h_{2}=h$ and $|h_{1}|\leq f,|h_{2}|\leq g,$ we have $$ |\Phi(h)|=|\Phi(h_{1})+\Phi(h_{2})|\leq|\Phi(h_{1})|+|\Phi(h_{2})|\leq\Lambda f+\Lambda g. $$ Hence $\Lambda(f+g)\leq\Lambda f+\Lambda g,$ and we have proved (10 then $2f^{*}=|f|+f,$ so that $f^{+}$ E If $\boldsymbol{\mathsf{f}}$ is now a real function, $f\in C_{c}(X),$ it is natural to define $C_{c}^{+}(X);$ likewise, f Ct(X); and since $f=f^{+}-f^{-},$ $$ \Lambda f=\Lambda f^{+}-\Lambda f^{-}\qquad(f\in C_{c}(X),f\operatorname{real}, $$ ) (13) and $$ \Lambda(u+i v)=\Lambda u+i\Lambda v. $$ (14) Simple algebraic manipulations, just like those which occur in the proof of Theorem 1.32, show now that our extended functional $\Lambda$ is linear on C,(X) This completes the proof // Exercises 1 Ifpis a complex measure on a o-algebra_ ${\mathfrak{M}},$ , and if $\scriptstyle{\vec{E}}$ e 奶, define $$ \lambda(E)=\operatorname*{sup}\;\sum\left|\,\mu(E_{i})\right|, $$ the supremum being taken over all finite partitions $\{E_{i}\}$ of E. Does it follow that ${\boldsymbol{\lambda}}=\left|\mu\right|\,{\boldsymbol{\gamma}}$ 2 Prove that the example given at the end of Sec. 6.10 has the stated properties. 3 Prove that the vector space $M(X)$ of all complex regular Borel measures on a locally compact Hausdorff space $\scriptstyle{\mathcal{X}}$ is a Banach space if ll $M(X)$ is in M(X) was used in the first paragraph of the proof of difference of any two members of $\|=|\mu|(X).$ Hint: Compare Exercise 8, Chap. 5. [That the Theorem 6.19 supply a proof of his fact.]coMPLEX MEASUREs 133 and 4 Suppose $1\leq p\leq\infty,$ and $\scriptstyle{\mathcal{A}}$ is the exponent conjugate to p. Suppose ${}_{\mu}$ is a positive o-finite measure $\scriptstyle{\mathcal{G}}$ is a measurable function such that fg e $L^{1}(\mu)$ for every fe LE(ul). Prove that then $g\in L^{g}(\mu).$ sS Suppose $\scriptstyle{X}$ consists of two points $\underset{^{\sim}}{a}$ and ${\mathfrak{b}};$ b; define $\mu(\{a\})=1,\,\mu(\{b\})=\mu(X)=\infty,\,\mathrm{and}\;\mu(\mathcal{O})=0$ . Is it true, for this ${\boldsymbol{\mu}},$ , that $L^{\sigma}(\mu)$ is the dual space of $L^{1}(\mu)^{\gamma}$ 6 Suppose $1<p<\infty$ and prove that ${\mathcal{B}}(\mu)$ is the dual space of $D(\mu)$ even if p is not o-finite. (As usual, $1/p+1/q=1.)$ 7 Suppose ${}_{\!\mu}$ is a complex Borel measure on [O,2元) (or on the unit circle $T),$ and define the Fourier coefficients of ${}_{\mu}$ by $$ \hat{\mu}(n)= \{e^{-i\omega t}\;d\mu(t)\qquad(n=0,\;\pm1,\;\pm2,\;\ldots). $$ Assume that ${\hat{\mu}}(n)\to0$ as $n\to+\infty$ and prove that then ${\hat{\boldsymbol{\mu}}}(n)\to0$ as $n\to-\infty.$ Hint: The assumption also holds with f dp in place of dp if f is any trigonometric polynomial, hence iff is continuous, hence if f is any bounded Borel function, hence if ${\mathit{d\mu}}$ is replaced by $d|\mu|.$ ${\hat{\mu}}(n+k)={\hat{\mu}}(n)$ 8 In the terminology of Exercise 7, fnd all ${\boldsymbol{n}};$ of course, k is also assumed to be an integer.] $\boldsymbol{\hat{\mu}}$ is periodic, with period $k_{\circ}$ [This means that for al integers ${}_{\!\mu}$ such that 9 Suppose that $\{g_{n}\}$ is a sequence of positive continuous functions on $\scriptstyle t^{n}=1^{n}$ 1], that ${}_{\mu}$ is a positive Borel measure on I, and that (ifi (i) lim, $g_{n}(x)=0$ for all a.e. [m] for every fe C(I) (ii) ${\mathfrak{f}}_{t}$ g。 dm = 1 J。dm = J, Jdu $\operatorname*{lim}_{n\to\infty}\left.\int_{I}\right.$ $\mu\perp\ m\gamma$ Does it follow that 10 Let (X,肌,p) be a positive measure space. Call a set $\Phi\in L^{\left(\mu\right)}$ uniformly integrable if to each $\epsilon>0$ corresponds a 8> 0 such that $$ \left|\int_{E}f\,d\mu\right|<\epsilon $$ whenever fe O and $\mu(E)<\delta.$ (a) Prove that every finite subset o $L^{1}(\mu)$ is uniformly integrable (b) Prove the following convergence theorem of Vitali $\operatorname{If}\left(\mathbf{i}\right)\mu(X)<\infty,\left(\mathbf{i}\right)\,\left\{f_{n}\right\}$ is uniformly inegrabl,(ii,fAx)→f(x) a.e. as n→o,and iv) |J(x)1 $<\infty\ a\epsilon,t h e n f\in L^{l}(\mu)$ and $$ \operatorname*{lim}_{n\to\infty} \{_{X}|f_{n}-f|\ d\mu=0. $$ Suggestion: Use Egoroff's theorem fails i ${}_{\!\mu}$ is Lebesgue measure on(-o,Go), ven if ${}_{\mu}$ (for instance, for Lebesgue measure is assumed to be (c) Show that ${\ \ \left(b\right)}$ $\{\|f_{n}\|_{1}\}$ bounded. Hypothesis $\mathbf{(}1)$ can therefore not be omitted in (b). (4) Show that hypothesis Giv is redundant in (b) for some on a bounded interval), but that there are finite measures for which the omission of (iv) would make (b) false (e Show that Vitalis therem implies Lebesgues dominated convergence theorem, for fnit for every x,J $f_{n} arrow0$ , but $\{f_{n}\}$ is not Lebesgue's theorem do not hold. measire spaces Construct an example in which Vitali's theorem applies alitough the hypotheses o $\mathbf{(f)}$ Construct a sequence $\{f_{n}\}_{*}$ say on [0, 1], so tha $\operatorname{t}f_{n}(x) arrow0$ uniformly integrable (with respect to Lebesgue measure) and (g) Howeve,the following converse of Vitalis theorem is true: $J f\mu(X)<\infty{,}f_{n}\in L^{1}(\mu),$ $$ \operatorname*{lim}_{n\to\infty}\ {\overline{{\bigcup_{E}}}}f_{n}\,d\mu $$ exists for every $\underline{{E}}$ ∈奶, then $\{f_{n}\}$ is uniformly integrable.134 REAL AND COMPLEX ANALYSIS Prove this by completing the following outline. that Define $_{*}^{\rho}(A,B)=\int|\chi_{A}-\chi_{B}|\ d_{\mu}.$ Then (D, p) is a complete metric space (modulo sets of measure 员, N (Exercise 13, Chap ${\mathsf{5}})$ SO 0), and E→Jef,dp is continuous for each n.If $\epsilon>0,$ there exist ${\cal E}_{0},$ $$ \left|\int_{E}^{\infty}(f_{n}-f_{N})\;d\mu\right|<\epsilon\mathrm{\boldmath~\if~}\;\;\rho(E,E_{0})<\delta,\quad\quad n>N. $$ (*) I of $\Gamma\mu(A)<\delta,$ (*) holds with $\scriptstyle B=E_{0}-A$ and $C=E_{0}\cup A$ in place of ${\boldsymbol{E}}.$ Thus $(^{\circ})$ holds with $\scriptstyle A$ in place $\boldsymbol{E}$ and $\mathbf{z}\epsilon$ in place of e. Now apply (a) to $\{f_{1},\ldots,f_{N}\};$ There exists $\scriptstyle{\theta}\ \;{\overset{\underset{\mathrm{p}}{}}{=}}\;$ such that $$ \left|\left|\right|_{A}^{\varepsilon}f_{n}\,d\mu\right|<3\varepsilon\mathrm{\boldmath~\if~\}\mu(A)<\delta^{\prime},\mathrm{\boldmath~\Omega~}\mu=1,\,2,\,3,\,..... $$ 11 Suppose ${}_{\mu}$ is a positive measure on $\scriptstyle X,\;\mu(X)<\infty,f_{n}\in L^{1}(\mu)\;{\mathrm{for~}}n=\mathbf{i}$ ,2,3, ……,f(x)→f(x) a.e, and there exists $p>1$ and $c{\mathrm{-}}\alpha$ such that $\textstyle{\int_{X}|f_{n}|^{p}\,d\mu<C}$ for al . Prove that $$ \operatorname*{lim}_{n arrow\infty}\ {\ ^{\circ}}_{X}|f-f_{n}|\,d\mu=0. $$ Hint: $\{f_{n}\}$ is uniformly integrable_ show that $\scriptstyle{\mathcal{G}}$ 12 Let OD be the collection of all sets $\bar{E}$ in the unit interval [O $1\!\!\!\!1$ such that either $\scriptstyle{\vec{E}}$ or its complement is at most countable. Let ${}_{\!\mu}$ be the counting measure on this ${\sigma}.$ -algebra D. If $g(x)=x$ for $0\leq x\leq1.$ is not ${\mathfrak{M}}.$ -measurable, although the mapping $$ f{ arrow}\sum x f(x)=\int f g\;d\mu $$ makes sense for every f∈ $L^{1}(\mu)$ and defines a bounded linear functional on Eu). Thus (L)* ≠ DP" in this situation. $13$ Let $L^{\infty}=L^{\infty}(m),$ where ${\mathfrak{m}}\,$ is Lebesgue measure on $\scriptstyle t\equiv[0]$ 1]. Show that there is a bounded linear functional $\Lambda\neq0$ on $L^{\infty}$ that is ${\mathbf{0}}$ on ${\boldsymbol{C}}(I)$ , and that therefore there is no $g\in L^{1}(m)$ that satisfies $\Lambda f=$ 「,f dm for every fe $L^{\infty},$ Thus $(L^{\alpha})^{\alpha}\neq L^{1}.$