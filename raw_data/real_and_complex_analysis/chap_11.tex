CHAPTER ELEVEN HARMONIC FUNCTIONS The Cauchy-Riemann Equations 11.1 The Operators $\scriptstyle{\hat{O}}$ and é Suppose $\boldsymbol{\f}$ is a complex function defined in a plane $R^{2},$ and assume that open set Q2. Regard fas a transformation which maps S $\Omega$ into $\boldsymbol{\f}$ has a differential at some point $z_{0}=f(z_{0})=0.$ Our differentiability assumption is then equiv- simplicity, suppose $z_{0}\in\Omega,$ in the sense of Definition 7.22. For alent to the existence of two complex numbers $\scriptstyle{\dot{\mathbf{x}}}$ and $\beta$ (the partial derivatives of f with respect to x and $\scriptstyle{\mathcal{I}}$ y at $z_{0}=0)$ such that $$ f(z)=\alpha x+\beta y+\eta(z)z\qquad(z=x+i y), $$ (1) where $\eta(z) arrow0$ as $\scriptstyle x\to0.$ and $2i y=z-\bar{z},(1)$ can be rewritten in the form Since $2x=z+{\bar{z}}$ $$ f(z)=\frac{\alpha-i\beta}{2}\,z+\frac{\alpha+i\beta}{2}\,\bar{z}+\eta(z)z. $$ (2) This suggests the introduction of the differential operators $$ \partial=\frac{1}{2}\left(\frac{\partial}{\partial x}-i\frac{\partial}{\partial y}\right),\qquad\bar{\partial}=\frac{1}{2}\left(\frac{\partial}{\partial x}+i\frac{\partial}{\partial y}\right). $$ (3) Now (2) becomes $$ {\frac{f(z)}{z}}=(\hat{\mathcal{O}})(0)+(\tilde{\partial}f)(0)\cdot{\frac{\bar{z}}{z}}+\eta(z)\qquad(z\neq0). $$ (4) at For real z $\mathbb{Z}/z=_{-}1\,;$ for pure imaginary $z,{\bar{z}}/z=-1.$ Hence $f(z)/z$ has a limit $\mathbf{0}$ if and only if $(\partial f)(0)=0,$ and we obtain the following characterization of holomorphic functions: 231232 REAL AND COMPLEX ANALYSiS 11.2 Theorem Suppose $\boldsymbol{\f}$ is a complex function in Q that has a differential at every point of Q. Then fe H(Q2) if and only if the Cauchy-Riemann equation $$ ({\hat{G}})(z)=0 $$ (1) holds for every z e Q2. In that case we have $$ f^{\prime}(z)=(\partial f)(z)\qquad(z\in\Omega). $$ (2) $\operatorname{If}f=u+i v,u$ and v real, (1) splits into the pair of equations $$ u_{x}=v_{y},\qquad u_{y}=-v_{x} $$ where the subscripts refer to partial differentiation with respect to the indicated variable. These are the Cauchy-Riemann equations which must be satisfied by the real and imaginary parts of a holomorphic function. 11.3 The Laplacian Let f be a complex function in a plane open set $\Omega,$ such that $f_{x x}{\mathrm{and}}f_{y}$ exist at every point of Q2. The Laplacian of fis then defined to be $$ \Delta f=f_{x x}+f_{y y}. $$ (1) If f is continuous in $\Omega$ and if $$ \scriptstyle M=0 $$ (2)) at every point of Q, then fis said to be harmonic in Q Since the Laplacian of a real function is real (if it exists),it is clear that a complex function is harmonic in $\Omega$ if and only if both its real part and its imaginary part are harmonic in Q2. Note that $$ \Delta f=4\partial\bar{\phi}f $$ (3) provided that $f_{x y}=f_{y x}\,,$ and that this happens for all $\boldsymbol{\f}$ which have continuous second-order derivatives If f is holomorphic, then ${\bar{\mathcal{I}}}=0,{\mathcal{I}}$ has continuous derivatives of all orders, and therefore (3) shows: 11.4 Theorem Holomorphic junctos are harmonic. We shall now turn our attention to an integral representation of harmonic functions which is closely related to the Cauchy formula for holomorphic func tions. It will show, among other things,that every real harmonic function is locally the real part of a holomorphic function, and it will yield information about the boundary behavior of certain classes of holomorphic functions in open discs.HARMONiC FUNCTiONs 233 The Poisson Integral 11.5 The Poisson Kernel This is the function $$ P_{r}(t)=\sum_{-\infty}^{\infty}r^{|n|}e^{i m}\qquad(0\leq r<1,\;t\ \mathrm{real}). $$ (1) We may regard $\scriptstyle{p,i j}$ as a function of two variables $\textstyle\gamma^{*}$ and t or as a family of functions of t, indexed by r If $z=r e^{i\theta}$ $0\leq r<1.$ G real), a simple calculation, made in Sec. 5.24, shows that $$ P_{r}(\theta-t)=\mathrm{Re}\left[{\frac{e^{i t}+z}{e^{i t}-z}}\right]={\frac{1-r^{2}}{1-2r\cos\left(\theta-t\right)+r^{2}}}. $$ (2) From $\mathbf{(1)}$ we see that $$ \frac{1}{2\pi} [\O_{-\pi}^{\pi}P_{r}(t)\;d t=1\qquad(0\leq r<1). $$ (3) From (2) it follows that $P_{r}(t)>0,P_{r}(t)=P_{r}(-t),\mathrm{t}\mathrm{t}$ hat $$ P_{r}(t)<P_{r}(\delta)\qquad(0<\delta<|t|\leq\pi), $$ (4) and that $$ \operatorname*{lim}_{r\to1}P_{r}(\delta)=0\qquad(0<\delta\le\pi). $$ (5) were discussed in Sec. 4.24. These properties are reminiscent of the trigonometric polynomials $\scriptstyle(a^{2}/b$ that The open unit disc $\mathrm{boi}$ 1) will from now on be denoted by ${\boldsymbol{U}}.$ ${\boldsymbol{T}}.$ Whenever it is The unit circle - the boundary of $U$ in the complex plane - will be denoted by convenient to do so,we shall identify the spaces $L^{p}(T)$ and $\operatorname{c}(n)$ with the corre- sponding spaces of $2\pi.$ z-periodic fuhctions on $R^{1},$ as in Sec. 4.23 becomes One can also regard $p_{*}(\theta-t)$ as a function of $z=r e^{i\theta}$ and $e^{i t},$ Then(2) $$ P(z,\,e^{i t})=\frac{1-|z|^{2}}{|e^{i t}-z|^{2}} $$ (6) for $z\in U,e^{i t}\in T.$ 11.6 The Poisson Integral $\operatorname{If}f\in L^{1}(T)$ and $$ F(r e^{i\theta})=\frac{1}{2\pi}\int_{-\pi}^{\pi}P_{r}(\theta-t)f(t)\,\,d t, $$ (1) then the function ${\mathbf{}}F$ so defined in $U$ is called the $P_{\bar{\nu}}$ oisson integral of f. We shall sometimes abbreviate the relation (1) to $$ F=P[f]. $$ (2)234 REAL AND coMPLEX ANALYSiS If f is real, formula 11.5(2) shows that $\scriptstyle{|y|,|}$ is the real part of $$ {\frac{1}{2\pi}} |_{-\pi}^{\pi}{\frac{e^{i t}+z}{e^{i t}-z}}\,f(t)\,d t, $$ (3) is harmonic in ( which is a holomorphic function of $z=r e^{i\theta}$ in ${\boldsymbol{U}},$ by Theorem 10.7. Hence P[f] ${\boldsymbol{U}}.$ Since linear combinations (with constant coefficients) of harmo nic functions are harmonic, we see that the following is true: 11.7 Theorem If f e $\scriptstyle U(T)$ then the Poisson integral P[f]is a harmonic func tion in ${\boldsymbol{U}},$ The following theorem shows that Poisson integrals of continuous functions behave particularly well near the boundary of ${\boldsymbol{U}}.$ 11.8 Theorem $\;U f\in C(T)$ and if H is deined on the closed unit dis $\bar{U}$ by $$ (H f)(r e^{i\theta})=\left\{\!\!\begin{array}{c}{{f(e^{i\theta})\!}}\\ {{(P[f](r e^{i\theta})\!}}\end{array}\right.\quad\mathrm{if}\ r=1, $$ (1) then Hf ∈ $C({\bar{U}}).$ PRooF Since $P_{r}(t)>0,$ formula 11.5(3) shows, for every $g\in C(T),$ that $$ \mid{\cal P}[g](r e^{i\theta})\mid\leq\mid\mid g\mid_{T}\quad\quad(0\leq r<1), $$ (2)) so that $$ \|H g\|_{\bar{U}}=\|g\|_{T}\qquad(g\in C(T)). $$ (3) the set $E_{\circ},$ (As in Sec. 5.22, we use the notation $\|g\|_{E}$ to denote the supremum of $|\,g\,|$ on If $$ g(e^{i\theta})=\sum_{n=-N}^{N}c_{n}e^{i n\theta} $$ (4) is any trigonometric polynomial, it follows from 11.5(1) that $$ (H g)(r e^{i\theta})=\sum_{n=-N}^{N}c_{n}\,r^{|n|}e^{i n\theta}, $$ (5) as so that $H g\in C({\bar{U}}).$ such that $\|g_{k}-f\|_{T}\to$ 0 $k\to\varnothing.$ Finally, there are trigonometric polynomials ${\mathfrak{g}}_{k}$ (See Sec. 4.24.) By (3), it follows that $$ \|H g_{k}-H f\|_{U}=\|H(g_{k}-f)\|_{U}\to0 $$ (6) to $H f.$ Hence $H f\in C({\bar{U}}).$ This says that the functions $H g_{k}\in C({\bar{U}})$ converge, uniformly on ${\bar{U}},$ as $k\to\varnothing.$ ///HARMONiC FUNCTiONs 235 Note: This theorem provides the solution of a boundary value problem (the Dirichlet problem): A continuous function f is given on ${\mathbf{}}T$ and it is required to find a harmonic functiori ${\mathbf{}}F$ in $U$ J“whose boundary values are f” The theorem exhibits a solution, by means of the Poisson integral of f $f,$ , and it states the relation between $\boldsymbol{\f}$ and ${\mathbf{}}F$ more precisely. The uniqueness theorem which corresponds to this existence theorem is contained in the following result. 11.9 Theorem Suppose u is a continuous real function on the closed unit disc ${\tilde{U}},$ and suppose u is harmonic in ${\boldsymbol{U}}.$ J. Then (in U)u is the Poisson integral of its restriction to ${\boldsymbol{T}},$ and $\boldsymbol{u}$ is the real part of the holomorphic function $$ f(z)={\frac{1}{2\pi}}\int_{-\pi}^{\pi}{\frac{e^{i t}+z}{e^{i t}-z}}\,u(e^{i t})\,d t\qquad(z\in U). $$ (1) $u_{1}$ PRooF Theorem 10.7 shows $\mathrm{that}\,f\in H(U).$ If $u_{1}=\mathrm{Re}\,f,$ then(1) shows that is the Poisson integral of the boundary values of $u,$ and the theorem will Put be proved as soon as we show that Then $\boldsymbol{\mathit{h}}$ is continuous on $\bar{U}$ (apply Theorem 11.8 to u) $\boldsymbol{h}$ $u=u_{1}.$ $h=u-u_{1}.$ is harmonic in ${\boldsymbol{U}},$ and $h=0$ at all points of ${\boldsymbol{T}}.$ . Assume (this will lead to a define contradiction) that $\hbar(z_{0})>0$ for some $z_{0}\in U.$ Fix e so that $0<\epsilon<h(z_{0}),$ and $$ g(z)=h(z)+\epsilon\mid z\mid^{2}\qquad(z\in\bar{U}). $$ (2) Then $g(z_{0})\geq h(z_{0})>\epsilon$ .Since $g\in C({\mathcal{U}})$ $\scriptstyle{\mathcal{G}}$ g has a local maximum. This implies that at all points of ${\boldsymbol{T}},$ there exists a point $z_{1}\in U$ and since $g=\epsilon$ at which $\scriptstyle y_{\alpha}\leq0$ and $\scriptstyle g_{\eta}\leq0$ at $z_{1}.$ But (2) shows that the Laplacian of $\scriptstyle{\mathcal{G}}$ is $\scriptstyle4\times5.$ and we have a contradiction. Thus $u-u_{1}\leq0.$ The same argument shows that $u_{1}-u\leq0.$ Hence $u=$ $u_{1},$ and the proof is complete. ${\it f}/{\it f}/{\it f}$ 11.10 So far we have considered only the unit disc $U=D(0;1).$ It is clear that the preceding work can be carried over to arbitrary circular discs,by a simple change of variables. Hence we shall merely summarize some of the results: If $\boldsymbol{u}$ is a continuous real function on the boundary of the disc $D(a;R)$ and if u is defined in $\scriptstyle{D(\omega).}$ R) by the Poisson integral $$ u(a+r e^{i\theta})=\frac{1}{2\pi}\left|_{-\pi}^{\pi}\frac{R^{2}-r^{2}}{R^{2}-2R r\cos (\theta-t\right)+r^{2}}\,u(a+R e^{i t})\,d t $$ (1) t then u is continuous on $D(a;R)$ ${\bar{D}}(a;\,R)$ and harmonic in $D(a;\,R).$ ${\tilde{D}}(a;R)<\Omega,$ then u $\boldsymbol{u}$ whose real part is u. This $\boldsymbol{\f}$ is harmonic (and rea) in an open set $\Omega$ and if defined in $D(a;\,R)$ satisfies(1) in and there is a holomorphic function $\boldsymbol{\f}$ is uniquely defined, up to a pure imaginary additive constant. For if two functions, holomorphic in the same region, have the same real part, their difference must be constant (a corollary of the open mapping theorem, or the Cauchy-Riemann equations)236 REAL AND coMPLEX ANALYSIs We may summarize this by saying that every real harmonic function is locall the real part of a holomorphic function. Consequently, every harmonic function has continuous partial derivatives of all orders The Poisson integral also yields information about sequences of harmonic functions: 11.11 Harnack's Theorem Let{u,}be a sequence of harmonic functions in a region Q. (a)If u。→u uniformly on compact subsets of Q2, then $\boldsymbol{u}$ is harmonic in Q (b)If $u_{1}\leq u_{2}\leq u_{3}\leq\cdot\cdot,$ then either {u,} converges uniformly on compact subsets of Q2, or $u_{n}(z)\to$ oo for every z ∈ $\Omega.$ PROOF To prove (a), assume ${\tilde{D}}(a;R)<\Omega,$ and replace $\boldsymbol{\mathit{u}}$ by $\textstyle u_{n}$ in the Poisson integral 11.10(1). Since $u_{n}\to u$ uniformly on the boundary of ${\tilde{D}}(a;R),$ we con- clude that uitself satisfies 11.10(1) in $D(a;\,R).$ $\underline{{{u}}}_{n}-u_{1}.)$ Put u = sup $u_{n\,,}$ In the proof of (b),we may assume that $u_{1}\geq0$ (If not, replace $\textstyle u_{n}$ by ${\tilde{D}}(a;R)<\Omega.$ ,let $A=\{z\in\Omega\colon u(z)<\infty\}.$ and $B=\Omega-A.$ Choose The Poisson kernel satisfies the inequalities $$ {\frac{R-r}{R+r}}\leq{\frac{R^{2}-r^{2}}{R^{2}-2r R\cos\left(\theta-t\right)+r}}\leq{\frac{R+r}{R-r}} $$ for $0\leq r<R.$ Hence $$ \frac{R-r}{R+r}\,u_{n}(a)\leq u_{n}(a+r e^{i\theta})\leq\frac{R+r}{R-r}\,u_{n}(a). $$ $u(z)=\,\infty$ Thus both The same inequalities hold with $\boldsymbol{u}$ in place of $\Omega$ is connected, we have either for all $z\in D(a;R){\mathrm{~or~}}u(z)<\infty$ for all $\textstyle u_{n}$ $\mathbf{I}\!:$ follows that either $z\in D(a;R)$ $\scriptstyle A$ and $\boldsymbol{B}$ are open; and since $A={\mathcal{D}}$ (in which case there is nothing to prove) or $A=\Omega.$ In the latter case, the monotone convergence theorem shows that the Poisson formula holds for u in every disc in Q.Hence $\boldsymbol{\ u}$ is harmonic in Q.Whenever a sequence of continuous functions converges monotonically to a continuous limit, the con- vergence is uniform on compact sets ([26], Theorem 7.13). This completes the proof. //HARMONIC FUNCTIONS $237$ The Mean Value Property 11.12 Definition We say that a continuous function u in an open set $\Omega$ has the mean value property if to every $\varepsilon\in\Omega$ there corresponds a sequence $ \lbrace r_{n} \rbrace$ such that $r_{n}>0,r_{n}\to0$ as $n\to\mathbf{o},$ and $$ u(z)=\frac{1}{2\pi}\left.\int_{-\pi}^{\pi}u(z+r_{n}e^{i t})\,d t\qquad(n=1,\,2,\,3,\,\dots).\qquad\right. $$ (1) In other words, $i d\!\!\!/(\mathbb{Z})$ is to be equal to the mean value of $\boldsymbol{\ u}$ on the circles of radius ${\boldsymbol{r}}_{n}$ and with center at z Note that the Poisson formula shows that (1) holds for every harmonic function $u,$ , and for every r such that ${\tilde{D}}(z;r)<\Omega.$ Thus harmonic functions satisfy a much stronger mean value property than the one that we just defined. The following theorem may therefore come as a surprise: 11.13 Theorem If a continuous function u has the mean value property in an open set $\Omega,$ then $\boldsymbol{\ u}$ is harmonic in $\Omega.$ PR00F It is enough to prove this for real u. Fix ${\tilde{D}}(a;R)<\Omega$ The Poisson $D(a;R),$ $\boldsymbol{E}$ integral gives us a continuous function $\iota_{1}\ \mathrm{on}\ \tilde{D}(a;R)$ which is harmonic in Put that is a compact subset of and which coincides with u on the boundary of Assume $m>0.$ and let $\boldsymbol{E}$ be $D(a;\,R)$ and let m = sup $\{v(z)\colon z\in{\bar{D}}(a;\,R)\}.$ Since $v=0$ on the boundary of such $D(a;\,R).$ $v=u-h,$ the set of all $z\in{\bar{D}}(a;R)$ at which $v(z)=m.$ Hence there exists a $\scriptstyle z_{\mathrm{e}}\in E$ $D(a;\,R).$ $$ |z_{0}-a|\geq|z-a| $$ than tion. Thus $\scriptstyle m=0.$ so $\scriptstyle x\,<\,0.$ small enough ${\boldsymbol{r}}_{\!_{J}}$ at least half the circle with center $-v.$ Hence $v=0,$ for all $z\in E.$ For $\ a l l$ ${\boldsymbol{E}},$ so that the corresponding mean values of v are all less $z_{\mathrm{0}}$ and radius $\scriptstyle{\mathcal{I}}$ lies outside or $u=h$ in $D(a;\,R),$ and since ${\bar{D}}(a;\,R)$ But v has the mean value property, and we have a contradic u is $m=v(z_{0}).$ The same reasoning applies to was an arbitrary closed disc in $\Omega,$ harmonic in $\Omega.$ // Theorem 11.13 leads to a reflection theorem for holomorphic functions. By half plane the upper half plane ${\boldsymbol{\Pi}}^{+}$ we mean the set of all $z=x+i y$ with $\scriptstyle y\,>0;$ the lower $\Pi^{-}$ consists of all z whose imaginary part is negative 11.14 Theorem (The Schwarz reflection principle) Suppose ${\boldsymbol{L}}$ is a segment of the real axis, $\Omega^{+}$ is a region in $\mathbf{n}^{\prime}.$ and every $\overset{*}{\in}L$ is the center of an open disc $D_{t}$ such that $\Pi^{*}\frown D_{i}$ lies in $\Omega^{+}.$ Let $\Omega^{*}$ be the reflection of $\Omega^{+}$ $$ \Omega^{-}=\{z\colon\Xi\in\Omega^{+}\}. $$ (1)238 REAL AND coMPLEX ANALYSIs Suppose f = u + iv is holomorphic in $\Omega^{+},$ and $$ \operatorname*{lim}_{n\to\infty}v(z_{n})=0 $$ (2) for every sequence $\{z_{n}\}i n\,\Omega^{+}$ which converges to a point of $L.$ such that Then there is a function ${\boldsymbol{F}},$ holomorphic in $\Omega^{*}\sim L\cup\Omega^{-},$ $F(z)=f(z)$ in $\mathbf{a}^{*}\!:$ this ${\mathbf{}}F$ satisfies the relation $$ F(\bar{z})=\overline{{{F(z)}}}\;\;\;\;\;\;\;(z\in\Omega^{+}\;\cup\;L\cup\Omega^{-}). $$ (3) The theorem asserts that $\boldsymbol{\f}$ f can be extended to a function which is holo- morphic in a region symmetric with respect to the real axis, and (3) states that ${\mathbf{}}F$ preserves this symmetry. Note that the continuity hypothesis(2)is merely imposed on the imaginary part off $\scriptstyle\epsilon\in L$ and that v has the mean value property in $\Omega,$ We extend vto Q by defining $v(z)=0$ for PROOF Put $\Omega=\Omega^{+}\cup L\cup\Omega^{-}$ and $v(z)=\,-v({\bar{z}})$ for $\varepsilon\in\Omega$ It is then inmediate that vis continuous by so that vis harmonic in $\Omega,$ Theorem 11.13 Hence vis locally the imaginary part of a holomorphic function. This that means that to each of the discs ${\hat{f}}_{t}$ is determined by v up to a real additive constant. If this ${\mathrm{an}}\,f_{i}\in H(D_{i})$ such that We $D_{t}$ there corresponds Im $f_{t}=v$ Each on $L,$ so that all derivatives of for some $z\in D_{t}\cap\Pi^{*}$ ,the same will cients, since constant is chosen so that $f_{t}(z)=f(z)$ is constant in the region $D_{i}\sim\Pi^{*}$ hold for all $z\in D_{t}\cap\Pi^{+}$ since $f-f_{t}$ in powers of $z-t$ has only real coeffi- assume that the functions ${\hat{f}}_{t}$ are so adjusted are real at t. It follows The power series expansion $\operatorname{of}f_{t}$ $v=0$ ${f}_{t}$ $$ f_{t}(\bar{z})=\overline{{{f_{t}(z)}}}\qquad(z\in D_{t}). $$ (4) Next, assume that $D_{s}\cap D_{t}\neq\varnothing.$ Then $f_{t}=f=f_{s}$ in $D_{t}\sim D_{s}\sim\Pi^{+}$ ;and since $D_{t}\sim D_{s}$ is connected, Theorem 10.18 shows that $$ f_{t}(z)=f_{s}(z)\qquad(z\in D_{t}\cap D_{s}). $$ (5) Thus it is consistent to define $$ F(z)={\sqrt{f(z)}}\qquad{\mathrm{for~}}z\in\Omega^{+} $$ (6) and it remains to show that ${\mathbf{}}F$ is holomorphic in $\Omega^{-}.$ If $D(a;r)<\Omega$ ,then $D(\bar{a};r)\subset\Omega^{+},$ so for every z $\in D(a;r)$ we have $$ f({\bar{z}})=\sum_{n=0}^{\infty}c_{n}({\bar{z}}-{\bar{a}})^{n}. $$ (T)HARMONiC FUNCroNs 239 Hence $$ F(z)=\sum_{n=0}^{\infty}\bar{c}_{n}(z-a)^{n}\qquad(z\in D(a;r)). $$ (8) This completes the proof. // Boundary Behavior of Poisson Integrals 11.15 Our next objective is to find analogues of Theorem 11.8 for Poisson inte- grals of $L^{p}.$ functions and measures on ${\boldsymbol{T}}.$ Let us associate to any function $\boldsymbol{\u}$ in $U$ a family of functions $u_{r}$ on ${\boldsymbol{T}},$ defined by $$ u_{r}(e^{i\theta})=u(r e^{i\theta})\qquad(0\le r<1). $$ (1) Thus $u_{r}$ is essentially the restriction of u to the circle with radius r, center O, but we shift the domain of $u_{r}$ to ${\boldsymbol{T}}.$ Using this terminology, Theorem 11.8 can be stated in the following form ${\boldsymbol{J}}$ fe $\operatorname{c}(\eta)$ and ${\boldsymbol{F}}=P[f]$ ,then $F_{r}{ arrow}f$ uniformly on ${\boldsymbol{T}},$ as $\scriptstyle{r\to1}$ In other words, $$ \operatorname*{lim}_{r\to1}\|F_{r}-f\|_{\infty}=0, $$ (2) which implies of course that $$ \operatorname*{lim}_{r arrow1}F_{r}(e^{i\theta})=f(e^{i\theta}) $$ (3) at every point o ${\boldsymbol{T}}.$ As regards (2),we shall now see(Theorem 11.16) that the corresponding norm-convergence result is just as easy in $L^{p}.$ Instead of confining ourselves to investigating radial limits,as in (3), we shall then study nontangential limits of Poisson integrals of measures and $L^{p}.$ -functions; the differentiation theory devel- oped in Chap. 7 will play an essential role in that study. 11.16 Theorem $$ I f\mid\leq p\leq\infty,f\in D(T),a n d\,u=P[f]\},t h e n $$ $$ \|u_{r}\|_{p}\leq\|f\|_{p}\qquad(0\leq r<1). $$ (1) $I f\vdash\leq p<\infty,t h e n$ $$ \operatorname*{lim}_{r\to1}\|u_{r}-f\|_{p}=0. $$ (2) PRoor If we apply Jensen's inequality (or Holder's) t $$ u_{r}(e^{i\theta})=\frac{1}{2\pi}\int_{-\pi}^{\pi}f(t)P_{r}(\theta-t)\ d t $$ 3)240 REAL AND COMPLEX ANALYSIS we obtain $$ |\,u_{r}(e^{i\theta})|^{p}\leq{\frac{1}{2\pi}}\, |\vphantom{\int}_{-\pi}^{\pi}|f(t)\,|^{p}P_{r}(\theta-t)\,\,d t. $$ (4) If we integrate (4) with respect to 0,over [-元,z] and use Fubini's theorem, we obtain (1) Note that formula 11.5(3) was used twice in this argument To prove ((2), choose $\scriptstyle\epsilon\;>_{0},$ and choose ge $\operatorname{c}(n)$ so tha $\|g-f\|_{p}<\epsilon$ (Theorem 3.14). Let v = P[g]. Then $$ u_{r}-f=(u_{r}-v_{r})+(v_{r}-g)+(g-f). $$ (5) By (1), ||u, - D, $$ _{p}=\|(u-v)_{r}\|_{p}\leq\|f-g\|_{p}<\epsilon.{\mathrm{Thus}} $$ $$ \left\|u_{r}-f\right\|_{p}\leq2\epsilon+\left\|v_{r}-g\right\|_{p} $$ (6) for all $\gamma_{<1}$ Also, $\left\|v_{r}-g\right\|_{p}\leq\left\|v_{r}-g\right\|_{\alpha}$ ,and the latter converges to O as // $r\to1,$ by Theorem 11.8. This proves (2) 11.17 Poisson Integrals of Measures If ${\boldsymbol{\mu}}$ is a complex measure on ${\boldsymbol{T}},$ and if we want to replace integrals over ${\mathbf{}}T$ by integrals over intervals of length 2z in $R^{1}$ these intervals have to be taken half open, because of the possible presence of point masses in u. To avoid this (admittedly very minor) problem, we shall keep integration on the circle in what follows, and will write the Poisson integral $u=P[d\mu]$ of ${\boldsymbol{\mu}}$ in the form $$ u(z)=\left|_{T}P(z,\,e^{i t})\,d\mu(e^{i t})\quad\quad(z\in U)\,\right. $$ (1) where $P(z,\,e^{i t})=(1-|z\,|^{2})/|\,e^{i t}-z\,|^{2},$ as in formula 11.5(6) The reasoning that led to Theorem 11.7 applies without change to Poisson Settin integrals of measures. Thus u, defined by (1), is harmonic in ${\cal U}.$ $\|\mu\|=|\mu|(T),$ the analogue of the first half of Theorem 11.16 is $$ \|u_{r}\|_{1}={\frac{1}{2\pi}}\int_{-\pi}^{\pi}|u(r e^{i\theta})|\ d\theta\leq\|\mu\|. $$ (2) 11.5(3). To see this, replace p ${\boldsymbol{\mu}}$ by lp| in (1), apply Fubini's theorem, and refer to formula $D(0;\,\alpha)$ and the line segments from $z=1$ to points of $\overline{{\mathfrak{b o}}};$ α) to be the union of the disc and 11.18 Approach Regions For $0<\alpha<1.$ we define $\Omega_{\alpha}$ In other words, $\Omega_{\alpha}$ is the smallest convex open set that contains $D(0;\,\alpha)$ of has the point l in its boundary. Near $z=1,\Omega_{x}$ is an angle, bisected by the radius $U$ that terminates at 1, of opening 20, where $x=\sin\,\theta$ . Curves that approach 1 within $\Omega_{x}$ cannot be tangent to T. Therefore $\Omega_{\alpha}$ is called a nontangential approach region, with vertex 1.HARMONIC FUNCTiONS 241 The fregions $\Omega_{\alpha}$ expand when α increases. Their union is ${\boldsymbol{U}},$ their intersection is the radius [0,1). Rotated copies of $\Omega_{\alpha},$ with vertex at $e^{i t},$ will be denoted by $r^{\alpha}\mathbf{u}.$ 11.19 Maximal Functions If $0<\alpha<1$ and u is any complex function with domain U, its nontangential maximal function $\scriptstyle N_{s}u$ is defined on ${\mathbf{}}T$ by $$ (N_{\alpha}u)(e^{i t})=\operatorname*{sup}\;\{\;|u(z)|\!:z\in e^{i t}\Omega_{\alpha}\}. $$ (1) Similarly, the radial maximal function of uis $$ (M_{r a d}\,u)(^{i t})=\operatorname*{sup}\;\{\,|\,u(r^{i t})|\,;\,0\leq r<1\}. $$ (2) If u is continuous and $\lambda$ is a positive number, then the set where either of these maximal functions is $\leq\lambda$ is a closed subset of ${\boldsymbol{T}}.$ Consequently $\scriptstyle\mathbf{y}_{s}u$ and $M_{\mathrm{rad}}\,l$ i are lower semicontinuous on $\scriptstyle r:\mathbb{h}$ n particular, they are measurable. ${\mathbf{}}T$ by $\sigma=m/2\pi$ Then $\textstyle{\boldsymbol{\sigma}}$ is tion Clearly, 11.20 will show that the size of $\scriptstyle\mathbf{y}_{s}u$ and the latter increases with α.If $u=P[d\mu],$ Theorem $M_{\mathrm{rad}}u\leq N_{\alpha}u,$ is, in turn, controlled by the maximal func- $\textstyle{M\mu}$ that was defined in Sec. 7.2 taking $k=1)$ . However, it will simplify the notation if we replace ordinary Lebesgue measure ${\mathfrak{m}}\,$ on a rotation-invariant positive Borel measure on ${\boldsymbol{T}},$ so normalized that $\sigma(T)=1.$ Accordingly, $M\mu$ is now defined by $$ (M\mu)(e^{i\theta})=\mathrm{sup}\,\frac{|\mu|\,(I)}{\sigma(I)}. $$ (3) The supremum is taken over all open arcs $\scriptstyle{t\in T}$ whose centers are at $e^{i\theta}.$ includ- ing ${\mathbf{}}T$ itself (even though ${\mathbf{}}T$ is of course not an arc) Similarly, the derivative ${\cal{D}}\mu$ of a measure $\boldsymbol{\mu}$ on ${\mathbf{}}T$ is now $$ ({\cal D}\mu)(e^{i\theta})=\operatorname*{lim}_{\mathcal{O}(I)}, $$ (4) as the open arcs $\scriptstyle{I\in T}$ shrink to their center $e^{i\theta},$ and $e^{i\theta}$ is a Lebesgue point of $f\in L^{n}(T)$ if $$ \textstyle\operatorname*{lim}\frac{1}{\sigma(I)} \{_{I}|f-f(e^{i\theta})|\ d\sigma=0, $$ (5) where $\{I\}$ is as in(4). is the Lebesgue decomposition of a complex Borel that If $d\mu=f\,d\sigma+d\mu_{s}$ where $f\in L^{n}(T)$ and $\mu_{s}~\underline{{{1}}}$ α, Theorems 7.4, 7.7, and 7.14 assert measure ${\boldsymbol{\mu}}$ on ${\boldsymbol{T}},$ $$ \sigma\{M\mu>\lambda\}\leq{\frac{3}{\lambda}}\;\|\mu\|, $$ (6) that almost every point of ${\mathbf{}}T$ is a Lebesgue point ${\mathfrak{o l}}/{\mathfrak{z}}$ and that $D\mu=f,D\mu_{s}=0$ a.e. [o]242 REAL AND COMPLEX ANALYSIS We will now see, for any complex Borel measure ${\boldsymbol{\mu}}$ On ${\boldsymbol{T}},$ that the non- tangential and radial maximal functions of the harmonic function $\scriptstyle\Gamma(d)1$ are con- trolled by Mu. In fact, if any one of them is finite at some point of ${\boldsymbol{T}},$ so are the others; this can be seen by combining Theorem 11.20 with Exercise 19. 11.20 Theorem Assume $0<\alpha<1.$ Then there is a constani $\scriptstyle c_{n}>0$ with the following property:If p is a positive finite Borel measure on ${\mathbf{}}T$ and $u=P[d\mu]$ is its Poisson integral, then the inequalities $$ c_{\alpha}(N_{\alpha}u)(e^{i\theta})\leq(M_{\mathrm{rad}}u)(e^{i\theta})\leq(M\mu)(e^{i\theta}) $$ (1) hold at every point $e^{i\theta}\in T.$ PRoor We shall prove(1)for $\theta=0.$ The general case follows then if the can show that special case is applied to the rotated measure $\mu_{\theta}(E)=\mu(e^{i\theta}E).$ Since $u(z)=\int_{T}\,P(z,\,e^{i t})\,d\mu(e^{i t}),$ the first inequality in(1) will follow if we $$ c_{\alpha}\,P(z,\,e^{i t})\leq P(\mid z\mid,\,e^{i t}) $$ (2) holds for all $\varepsilon\in\Omega_{s}$ and all $x^{n}\in T_{i}$ By formula 11.5(6),(2) is the same as $$ c_{\alpha}|\,e^{i t}-r\,|^{2}\leq|\,e^{i t}-z\,|^{2} $$ (3) where $r=|z|\,,$ shows that $|z-r|/(1-r)$ is bounded in $\Omega_{\alpha}.$ say by A. Hence The definition of $\Omega_{\alpha}$ 「e" $$ \begin{array}{l}{{-\;r\mid\leq\mid e^{t t}-z\mid+\mid z-r\mid}}\\ {{\leq\mid e^{t t}-z\mid+\gamma_{\alpha}(1-r)\leq(1+\gamma_{\alpha})\mid e^{i t}-z\mid}}\end{array} $$ so that(3) holds with $c_{\alpha}=(1+\gamma_{\alpha})^{-2}.$ This proves the first half of (1) For the second half, we have to prove that $$ \{\bar{P}_{T}(t)\;d\mu(e^{i t})\leq(M\mu)(1)\qquad(0\leq r\leq1). $$ (4) put $I_{n}=T,$ Fix r. Choose open arcs $\scriptstyle I_{\prime}=T,$ centered at 1,So that on ${\boldsymbol{T}}.$ Define $\,I_{j},$ ,and let $h_{j}$ For $1\leq j\leq n,\operatorname{let}\chi_{j}$ be the characteristic function of $I_{1}\subset I_{2}\subset\cdots\subset I_{n-1},$ be the largest positive number for which $h_{j}\chi_{j}\leq P_{r}$ $$ K=\sum_{j=1}^{n}(h_{j}-h_{j+1})\chi_{j} $$ (5) where $h_{n+_{1}}=0.$ Since $\scriptstyle{p,(q)}$ is an even function of ${\mathbf{}}t$ that decreases as $\hat{\boldsymbol{k}}$ increases from $\mathbf{0}$ to 元,we see that $K\leq P_{r}.$ The definition of $M\mu$ shows that on $I_{j}-I_{j-1}$ (putting $I_{0}={\mathcal{O}}),$ and that ${\underline{{h}}}_{j}-h_{j+1}\geq0.$ that $K=h_{j}$ $$ \mu(I_{j})\leq(M\mu)(1)\sigma(I_{j}). $$ (6)HARMONIC FUNCTIONs 243 Hence, setting $(M\mu)(1)=M,$ $$ [\sum_{T}K\ d\mu=\sum_{j=1}^{n}(h_{j}-h_{j+1})\mu(I_{j})\leq M\sum_{j=1}^{n}(h_{j}-h_{j+1})\sigma(I_{j}) $$ $$ =M\left.\right|_{T}K\,d\sigma\leq M\stackrel{ |}{\to}P_{r}\,d\sigma=M. $$ (7) partition of Finally, if we choose the arcs ${\cal I}_{j}$ so that their endpoints form a suficiently fine ,uniformly on // ${\boldsymbol{T}},$ we obtain step functions ${\boldsymbol{K}}$ that converge to $P_{r},$ T. Hence (4) follows from (7) tangential limit $\lambda$ at 11.21 Nontangential Limits A function 'e $T_{\mathbf{\delta}}$ if, for each $x<1.$ ${\boldsymbol{F}},$ defined in ${\boldsymbol{U}},$ is said to have non $e^{i\theta}$ $$ \operatorname*{lim}_{j arrow\infty}F(z_{j})=\lambda $$ for every sequence $\{z_{j}\}$ that converges to $e^{i\theta}$ and that lies in $\scriptstyle{\varepsilon^{\prime}\Omega_{s}}$ 11.22 Theorem If p is a positive Borel measure on ${\mathbf{}}T$ and $(D\mu)(e^{i\theta})=0.$ for some 8, then its Poisson integral $u=P[d\mu]$ has nontangential limit $\mathbf{0}$ at $e^{i\theta}.$ PRoOF By definition, the assumption $(D\mu)(e^{i\theta})=0$ means that $$ \operatorname*{lim}\ \mu(I)/\sigma(I)=0 $$ (1) as the open arcs $\scriptstyle{t\in T}$ shrink to their center $e^{i\theta},$ Pick $\scriptstyle\epsilon\;>0$ One of these arcs, say $I_{\mathrm{0}},$ is then small enough to ensure that $$ \mu(I)<\epsilon\sigma(I) $$ (2) for every $\scriptstyle t\in I_{\mathrm{o}}$ that has $\mu_{i}$ $(i=0,\;1).$ Suppose $z_{j}$ converges to $e^{i\theta}$ within some $e^{i\theta}$ as center. Let $\mu_{0}\,$ be the restriction of ${\boldsymbol{\mu}}$ to $I_{\mathrm{0}}$ , put $\mu_{1}=\mu-\mu_{0}\,.$ and let $u_{i}$ be the Poisson integral of region $\scriptstyle{}^{\mu}\!\Omega.$ Then $z_{j}$ stays at a positive distance from $T-I_{0}.$ The inte- grands in $$ u_{1}(z_{j})=\left.\right|_{T-I_{0}}P(z_{j},\,e^{i t})\,d\mu(e^{i t}) $$ (3) converge therefore to $\mathbf{0}$ as $j\to\mathbb{G},$ uniformly on $T-I_{0}\,.$ Hence $$ \operatorname*{lim}_{j arrow\infty}u_{1}(z_{j})=0. $$ (4) Next, use (2) together with Theorem 11.20 to see that $$ c_{\alpha}(N_{x}u_{0})(e^{i\theta})\leq(M\mu_{0})(e^{i\theta})\leq\epsilon. $$ (5)244 REAL AND coMPLEX ANALYSIS In ei*Q2。,uo(z)≤(N。uoXe"), Hence (5) implies that $$ \operatorname*{lim}_{j arrow\infty}\mathrm{gup}\;u_{0}(z_{j})\leq\epsilon/c_{\alpha}. $$ (6) Since $u=u_{0}+u_{1}$ and e was arbitrary, $\mathbf{\Psi}(\lambda)$ and (6) give $$ \operatorname*{lim}_{j arrow\infty}u(z_{j})=0. $$ (T) // 11.23 Theorem Iffe $\scriptstyle{I(T)_{k}}$ then P[] has nontangential limit $f(e^{i\theta})$ at every Lebesgue point $e^{i\theta}$ of f. PRoor Suppose $e^{i\theta}$ is a Lebesgue point of f. By subtracting a constant from $\boldsymbol{\f}$ we may assume, without loss of generality, $\mathrm{that}f(e^{i\theta})=0.$ Then $$ \operatorname*{lim}{\frac{1}{\sigma(I)}}\int_{I}|f|\ d\sigma=0 $$ (1) as the open arcs $\scriptstyle{t\in{\tau}}$ shrink to their center $e^{i\theta}.$ Define a Borel measure $\boldsymbol{\mu}$ u on ${\mathbf{}}T$ by $$ \mu(E)=\int_{E}|f|\;d\sigma $$ (2) Then (1) says tha $(D\mu)(e^{i\theta})=0\,;$ hence $\scriptstyle{p(d)1}$ has nontangential limit $\mathbf{0}$ at $e^{i\theta},$ by Theorem 11.22. The same is true of P[f], because $$ |P[f]|\leq P[1]f[]=P[d\mu]. $$ (3) // The last two theorems can be combined as follows 11.24 Theorem If $d\mu=f\,d\sigma+d\mu,$ is the Lebesgue decomposition of a complex Borel measure ${\boldsymbol{\mu}}$ on T,where fe $v(r),$ $\mu_{s}\,$ l o, then P[du] has nontangential $l i m i t f(e^{i\theta})$ at almost all points of ${\boldsymbol{T}}.$ PROOF Apply Theorem 11.22 to the positive and negative variations of the real and imaginary parts of $\mu_{s}\,,$ , and apply Theorem 11.23 to f // Here is another consequence of Theorem 11.20 11.25 Theorem For $0<\alpha<1$ and 1≤p ≤0,there are constants $A(x,\,p)<\infty$ with the following propertiesHARMONIC FUNCTIONs 245 (a)If p is a complex Borel measure on ${\boldsymbol{T}},$ and u = P[du], then $$ \sigma\{N_{\alpha}u>\lambda\}\leq\frac{A(\alpha,1)}{\lambda}\ \|\mu\|\qquad(0<\lambda<\infty). $$ (b) $I f\vdash<p\leq\infty,f\in B(T),$ and $u=P[f],t h e n$ $$ \|N_{\alpha}u\|_{p}\leq A(\alpha,\,p)\|f\|_{p}. $$ Pxoor Combine Theorem 11.20 with Theorem 7.4 and the inequality (7) in the proof of Theorem 8.18 // and they are in $\scriptstyle v(\tau)$ The nontangential maximal functions $N_{\alpha}u$ are thus in weak ${\boldsymbol{L}}^{1}$ l if $u=P[d\mu],$ if $u=P[f]$ for some $f\in D(T),\;p>1.$ This later result may be regarded as a strengthened form of the first part of Theorem 11.16. Representation Theorems 11.26 How can one tell whether a harmonic function u in $U$ is a Poisson integral or not? The preceding theorems (11.16 to 11.25) contain a number of necessary family conditions. It turns out that the simplest of these, the $L^{p}.$ boundedness of the $\{u_{r}\colon0\leq r<1\}$ is also sufficient! Thus, in particular, the boundedness of ${\boldsymbol{T}},$ , since, as we $|u_{i}\rangle|_{1},\ a s\ r\to1,$ implies the existence of nontangential limits a.e. on will see in Theorem 11.30, u can then be represented as the Poisson integral of a measure This measure will be obtained as a so-called“weak limit”of the functions $u_{r}\,.$ Weak convergence is an important topic in functional analysis. We will approach it through another important concept, called equicontinuity, which we will meet again later, in connection with the so-called “normal families”of holomorphic functions. 11.27 Definition Let J $\textstyle{\mathcal{F}}$ be a collection of complex functions on a metric space $X$ with metric $\boldsymbol{\rho}$ is equicontinuous if to every $\scriptstyle x\;{\sim}\;0$ corresponds a $\scriptstyle{\delta\,>\,0}$ such We say that $\mathcal{F}$ that $|f(x)-f(y)|<\epsilon$ for every $\boldsymbol{\f}$ e $\textstyle{\mathcal{F}}$ and for all pairs of points x,y with p(x,y)< 6.(In particular, every f∈ ${\mathcal{F}}$ is then uniformly continuous. We say that $\textstyle{\mathcal{F}}$ is pointwise bounded if to every xe $\textstyle X{\ ~}$ corresponds an $M(x)<\protect$ < oo such that $|f(x)|\leq M(x)$ for every f∈ F. 11.28 Theorem(Arzela-Ascon) Suppose that J is a pointwise bounded equi- ,and that X contains continuous collection of complex functions on a metric space $X,$ a countable dense subset $\textstyle E,$ Every sequence $\{f_{n}\}$ in ${\mathcal{F}}$ has then a subsequence that converges uniformly on every compact subset of $X$246 REAL AND COMPLEX ANALYSIs PxOOF Let $x_{1},$ set of all positive integers. Suppose $\mathbb{K}\geq1$ be an enumeration of the points of $\boldsymbol{E}$ . Let $S_{0}$ be the $x_{2},x_{3},\dotsc$ and an infinite set $S_{k-1}\subset S_{0}$ has been chosen. Since $\{f_{n}(x_{k});$ $n\in S_{k-1}\}$ is a bounded sequence of complex numbers, it has a convergent subsequence. In other words, there is an infinite set $S_{k}\subset S_{k-1}$ so that lim $\scriptstyle j d x_{3}$ exists as $n{ arrow}\infty$ within ${\mathbf{}}S_{k}\,.$ Continuing in this way, we obtain infinite sets $S_{0}\gg S_{1}\gg S_{2}\perp\cdot\cdot$ with Let the property that lim $f_{n}(x_{j})$ exists, for 1 $\leq j\leq k,$ if $n{ arrow}\circ\sigma$ within $\ {\boldsymbol{S}}_{k}$ ${\boldsymbol{r}}_{k}$ be the kth term of ${\boldsymbol{S}}_{k}$ (with respect to the natural order of the positive integers) and put $$ S=\{r_{1},\,r_{2},\,r_{3},\,\ldots\}. $$ Now let $\kappa\in X$ there are then at most ${\boldsymbol{k}}-1$ terms of $\boldsymbol{\mathsf{S}}$ that are not in ${\boldsymbol{S}}.$ ${\boldsymbol{S}}_{k}$ For each $\boldsymbol{k}$ Hence lim $\scriptstyle{j_{\mathrm{e}}v}$ exists, for $e v e r y\ x\in E.$ as $\textstyle n\!\to\!G$ within (The construction of 、 $\boldsymbol{\mathsf{S}}$ from $\{S_{k}\}$ is the so-called diagonal process be compact, pické> 0.By equicontinuity, there is a $\scriptstyle\delta>0$ so that $\rho(p,q)<\delta$ implies $|f_{n}(p)-f_{n}(q)|<\epsilon,$ for all ${\mathfrak{n}}.$ Cover ${\boldsymbol{K}}$ with ${\dot{S}}.$ $p_{i}\in B_{i}\cap E$ Hence there is an integer ${\mathbf{}}N$ of radius $\delta/2.$ Since $\boldsymbol{E}$ is dense in $X$ there are points open balls $B_{1},\ldots,B_{M}$ Since $p_{i}\in E,$ lim $\scriptstyle{\sqrt{M}}\scriptstyle\theta\rangle$ exists, as $n\to\infty$ within for $1\leq i\leq M.$ such that $$ |f_{m}(p_{i})-f_{n}(p_{i})|<\epsilon $$ for $i=1,\ldots,M,$ if $m>N,n>N,$ and ${\mathfrak{m}}$ and $\scriptstyle n$ are in ${\boldsymbol{S}}.$ $\rho(x,\,p_{i})<\delta.$ Our To finish, pick $x\in K$ Then $x\in B_{i}$ for some i, and choice of $\delta$ and ${\mathbf{}}N$ shows that l fmx)-f,(x)|≤|f(x))-fm(p)| +|Jmp))一J,Ap)| +|J.(p)一f.(x)1 <e+ + = 3= i $m>N,n>N.$ m ∈ S,n e S. // 11.29 Theorem Suppose that (a $X$ is a separable Banach space, $X,$ (b) $\scriptstyle\{\lambda_{3}\}$ is a sequence of linear functionals on (c)su $\mathsf{p|I A}_{n}|=M<\mathcal{D}$ n Then there is a subsequence $\{\Lambda_{n_{i}}\}$ such that the limit $$ \Lambda x=\operatorname*{lim}_{i\to\infty}\Lambda_{n_{i}}x $$ (1) exists for everyx∈ $X.$ Moreover, A is linear, and $\|\Lambda\|\le M.$ $\{\Lambda_{m}\}{\mathrm{:}}$ ; see Exercise 18.) (In this situation, A is said to be the weak limit ofHARMONIC FUNCTIONS $247$ PR0OF To say that $\textstyle X$ is separable means, by definition, that $\textstyle X$ has a count- X able dense subset. The inequalities $$ |\operatorname{A}_{n}x|\leq M\|x\|,\qquad|\operatorname{A}_{n}x^{\prime}-\operatorname{A}_{n}x^{\prime\prime}|\leq M\|x^{\prime}-x^{\prime\prime}\| $$ show that $\scriptstyle\{\lambda_{n}\}$ is pointwise bounded and equicontinuous. Since each point of $X$ is a compact set, Theorem 11.28 implies that there is a subsequence $\{\Lambda_{n_{i}}\}$ A by (1). It is then clear that $\mathrm{\A}$ converges, for every $x\in X;$ as i→00. To finish, define ${\it f}//{\it f}$ such that $\{\Lambda_{m_{i}}x\}$ is linear and that $\|\Lambda\|\le M.$ Let us recall, for the application that follows, that $\operatorname{c}(\eta)$ and $D(T)\left(p<\infty\right)$ are separable Banach spaces,because the trigonometric polynomials are dense in them, and because it is enough to confine ourselves to trigonometric polynomials whose coefficients lie in some prescribed countable dense subset of the complex field. 11.30 Theorem Suppose uis harmonic in $~~~~~~~~~U,1\leq p\leq\infty,$ and $$ \operatorname*{sup}_{0<r<1}\|u_{r}\|_{p}=M<\infty. $$ (1) (a)If $p=1,$ it follows that there is a unique complex Borel measure ${\boldsymbol{\mu}}$ on ${\mathbf{}}T$ 'so that w = P[du]. (b)If ${\mathfrak{p}}\succ1,$ it follows that there is a unique f ∈ $\scriptstyle{M(T)}$ so that $u=P[f].$ (c)Every positive harmonic function in $U$ is the Poisson integral of a unique U positive Borel measure on ${\boldsymbol{T}}.$ PROOF Assume first that $p=1.$ Define linear functionals A, on C(T) by $$ \Lambda_{r}g=\int_{T}g u_{r}\;d\sigma\;\;\;\;\;\;\;\;(0\leq r<1). $$ (2) with $\|\mu\|\leq M,$ and a sequence $r_{j} arrow1$ By Theorems 11.29 and 6.19 there is a measure ${\boldsymbol{\mu}}$ on ${\boldsymbol{T}},$ By (1), $\|\land_{r}\|\leq M.$ ,so that $$ \operatorname*{lim}_{j\to\infty}\;\left[_{\scriptstyle T}{g u_{r_{j}}}\,d\sigma=\right]_{T}{g}\;d\mu $$ (3) for every $g\in C(T).$ Then $h_{j}$ is harmonic in ${\boldsymbol{U}},$ continuous on ${\bar{U}},$ and is Put $h_{j}(z)=u(r_{j}z).$ ${\mathbf{}}T$ (Theorem 11.9). Fix therefore the Poisson integral of its restriction to $z\in U,$ , and apply (3) with $$ g(e^{i t})=P(z,\,e^{i t}). $$ (4)248 REAL AND cOMPLEX ANALYSIS Since $h_{j}(e^{i t})=u_{r_{j}}(e^{i t}),$ we obtain u $$ \begin{array}{c}{{(z)=\operatorname*{lim}_{\mu}u(r_{j}z)=\operatorname*{lim}_{j}k_{j}(z)}}\\ {{\l}}\\ {{=\operatorname*{lim}_{\ j}\end{array}\displaystyle [_{\cal T}P(z,\,e^{i t})h_{j}(e^{i t})\,d\sigma(e^{i t})}}\\ {{=\displaystyle\operatorname*{lim}_{P}(z,\,e^{i t})\,d\mu(e^{i t})=P[d\mu](z)}}\end{array} $$ If $1<p\leq\infty$ ,let $\boldsymbol{\mathit{q}}$ be the exponent conjugate to ${\boldsymbol{p}}$ . Then $\scriptstyle B(t)$ is separa- with ble. Define $\Lambda_{r}$ as in (2), but for all $g\in E(T).$ Again, $\|\land_{r}\|\leq M.$ Refer to Theo- $\|f\|_{p}\leq M$ rems 6.16 and 11.29 to deduce, as above, that there is an $f\in D(T),$ The , so that (3) holds, with f do in place of dp, for every $g\in E(T).$ rest of the proof is as it was in the case $p=1$ Pick $\boldsymbol{\f}$ This establishes the existence assertions in $\mathbf{\tau}_{(a)}$ and ((b). To prove unique- ness, t suffices to show that $P[d\mu]=0$ implies $\mu=0.$ e C(T), put $u=f\,\lbrack f\rbrack,$ $v=P[d\mu].$ By Fubini's theorem, and the symmetry P(re ${}^{i\theta},\,e^{i\theta}\}=P(r e^{i\theta},\,e^{i\theta}),$ $$ \left[{\frac{}{}}u_{r}\,d\mu=\frac{}{}\right]_{T}\!r\,f\,d{\sigma}\qquad(0\leq r<1). $$ (5) When $v=0$ then $\scriptstyle y_{r}=0,$ and since $u_{r}\to f$ uniformly, as $r\to1,$ we conclude that $$ \ D_{T}^{}f\,d\mu=0 $$ (6) for every $\boldsymbol{\f}$ ∈ $\operatorname{c}\!n$ pi $P[d\mu]=0.$ By Theorem 6.19,(6) implies that $\mu=0.$ Finally $\mathbf{\Psi}(c)$ is a corollary of (a), since ${\boldsymbol{u}}>0$ implies (1) with $\scriptstyle{p=1}$ $$ \left\{\bar{D}_{T}^{\phantom{\dagger}} |u_{r}\right|\,d\sigma= .\int_{T}^{}u_{r}\,d\sigma=u(0)\,\qquad(0\leq r<1) $$ (7) by the mean value property of harmonic functions. The functionals $\Lambda_{r}$ used in the proof of $\mathbf{\tau}_{(a)}$ are now positive, hence $\mu\geq0,$ // 11.31 Since holomorphic functions are harmonic, all of the preceding results (of which Theorems 11.16,11.24,11.25, 11.30 are the most significant) apply to holo- morphic functions in U. This leads to the study of the T $H^{p}.$ P-spaces, a topic that wil be taken up in Chap. 17. At present we shall only give one application, to functions in the space $H^{\infty}$ This, by definition, is the space of all bounded holomorphic functions in $U;$ the norm $$ \|f\|_{\infty}=\operatorname*{sup}\left\{|f(z)|:z\in U\right\} $$ turns $H^{\infty}$ into a Banach space.HARMONIC FUNCTIONs 249 As before, $L^{\infty}(T)$ is the space of all (equivalence classes of) essentially bounded functions on ${\boldsymbol{T}},$ normed by the essential supremum norm, relative to Lebesgue measure. For $g\in L^{\infty}(T),$ lll stands for the essential supremum of|gl. 11.32 Theorem ${\boldsymbol{T}}o$ every $f\in H^{\infty}$ corresponds a function $f^{*}\in L^{*}(T),$ defined almost everywhere by $$ f^{**}(e^{i\theta})=\operatorname*{lim}_{r\to1}f(r e^{i\theta}). $$ (1) The equality $\|f\|_{\alpha}=\|f^{*}\|_{\alpha}$ holds. on some arc $\scriptstyle{I\in{\frac{}{I}}}$ then $f(z)=0$ for every $I f^{\bullet}(e^{i\theta})=0$ for almost all $e^{i\theta}$ z ∈ ${\cal U}.$ (A considerably stronger uniqueness theorem will be obtained later,in Theorem 15.19. See also Theorem 17.18 and Sec. 17.19. 2元/n. Let PRooF By Theorem 11.30, there is a unique $g\in L^{\infty}(T)$ such that ${\mathbf I}$ is larger than By Theorem 11.23,(1) holds with $f^{*}=g$ The inequality $f=P[g]$ In particular, from Theorem 11.16(1); the opposite inequality is obvious $\|f\|_{\infty}\leq\|f^{*}\|_{*}$ 。 follows $i f f^{*}=0\;a.e.,$ then $\|f^{\ast}\|_{\infty}=0,$ hence $\|f\|_{\infty}=0,h e n c e f=0.$ Now choose a positive integer $\scriptstyle n$ so that the length of $x=\exp{\{2\pi i/n\}}$ and define $$ F(z)=\prod_{k=1}^{n}f(x^{k}z)\qquad(z\in U). $$ (2) Then $F\in H^{\infty}$ and $F^{*}=0$ a.e. on ${\boldsymbol{T}},$ hence $F(z)=0$ for all z e U. If $\mathrm{z}(p)$ the since $\scriptstyle{2(F)}$ Hence is the union of $\scriptstyle n$ by Theorem 10.18. were at most countable, the same would be true of $\scriptstyle{2(J)}$ by rotations. But $\mathbb{Z}(t)_{3}$ zero set of $\boldsymbol{\f}$ in ${\boldsymbol{U}},$ sets obtained from $\scriptstyle Z(F)=U$ $f=0,$ // Exercises harmonic? functions.) Show that $u^{2}$ are real harmonic functions in a plane region Q. Under what conditions is w is |/P 1 Suppose ${\mathbf{}}^{u}$ and $\underline{{{\mathcal{U}}}}$ harmonic?(Note that the answer depends strongly on the fact that the question is one about rea cannot be harmonic in Q,unless u is constant. For which $f\in H(\Omega)$ 2 Suppose fis a complex function in a region $\Omega,$ and both $f{\mathrm{~and~}}f^{2}$ are harmonic in $\Omega.$ Prove that either f or fis holomorphic in Q 3 If uis a harmonic function in a region $\Omega,$ what can you say about the set of points at which the gradient of uis O0?(This is the set on which $u_{x}=u_{y}=0,\quad$ 4 Prove that every partial derivative of every harmonic function is harmonic. Borel measure $\mathcal{J}$ Verify, by direct computation,that $P_{r}(\theta-t)$ is, for each fixed t,a harmonic function of $r e^{i\theta}.$ Deduce (without referring to holomorphic functions) that the Poisson integral P[du] of every finite on ${\mathbf{}}T$ is harmonic in $U_{\mathrm{,}}$ ,by showing that every partial derivative of P[dp] is equal to the integral of the corresponding partial derivative of the kernel S Suppose $f\in H(\Omega)$ $\boldsymbol{\mathit{f}}$ and f has no zero in $\Omega.$ Prove that log $|f|$ is harmonic in Q, by computing its Laplacian. Is there an easier way?250 REAL AND cOMPLEX ANALYSIS 6 Suppose fe $H(U),$ where $\boldsymbol{\mathit{U}}$ is the open unit disc, f is one-to-one in $U,\Omega=f(U),\;\mathrm{and}\,f(z)=\sum c_{n}z$ Prove that the area of Q is $$ \pi\;\sum_{n=1}^{\infty}n\left|c_{n}\right|^{2}. $$ Hint: The Jacobian offi $|f^{\prime}|^{2}.$ 70 $\;\operatorname{If}f\in H(\Omega),f(z)\neq0\;{\mathrm{for~}}z\in\Omega,$ and $-\infty<\alpha<\alpha.<\alpha.$ , prove that $$ \Delta(|f|^{a})=\alpha^{2}|f|^{a-2}|f^{\prime}|^{2}, $$ by proving the formula $$ \hat{\sigma}\hat{\bar{o}}(\psi\circ(\mathcal{\bar{f}}))=(\varphi\circ|f|^{2})\cdot|f^{\prime}|^{2}, $$ in which $\vartheta$ is twice differentiable on $\textstyle(0,$ oo) and $$ \varphi(t)=t\psi^{\prime\prime}(t)+\psi^{\prime}(t). $$ (b) Assume $f\in H(\Omega)$ and $\Phi$ is a complex function with domain $f(\Omega),$ , which has continuous second-order partial derivatives. Prove that $$ \Delta[\Phi\circ f]=[(\Delta\Phi)\circ f]\cdot|f^{\prime}|^{2}. $$ Show that this specializes to the result of (a) if $\Phi(w)=\Phi(\mid w\mid).$ 8 Suppose $\underline{{\Omega}}$ is a region, $f_{n}\in H(\Omega)$ for $n=1,$ 2, $3,\,\dots,\,u_{n}$ is the real part of $f_{n}.$ $\left\{u_{n}\right\}$ converges uni- $\{f_{n}\}$ formly on compact subsets of $\Omega,$ and $\{f_{n}(z)\}$ converges for at least one $z\in\Omega.$ Prove that then converges uniormly on compact subsets of 9 Suppose uis a Lebesgue measurable function in a region $\Omega,$ and ${}^{4}$ is locally in ${\boldsymbol{L}}^{1}$ This means that the integral of lul over any compact subset of $\underline{{\Omega}}$ is finite. Prove that u ${\mathcal{U}}$ is harmonic if it satisfies the following form of the mean value property: $$ u(a)=\frac{1}{\pi^{2}}\sqrt{ (\frac{}{}|\psi\langle f\rangle\rangle\ d X\ d\rangle} $$ whenever ${\tilde{D}}(a;r)\subset\Omega.$ 10 Suppose $\scriptstyle I=[a,b]$ is an interval on the real axis, $\scriptstyle{\varphi}$ is a continuous function on ${\mathit{l}}_{\mathrm{,}}$ ,and $$ f(z)={\frac{1}{2\pi i}}\int_{a}^{b}{\frac{\varphi(t)}{t-z}}\,d t\qquad(z\not\in I). $$ Show that $$ \operatorname*{lim}_{\epsilon arrow0}\left[f(x+i\epsilon)-f(x-i\epsilon)\right]\qquad(\epsilon>0) $$ exists for every real $x_{*}$ and find it in terms of $\textstyle\varnothing\quad$ What happens then at points x at How is the result affected if we assume merely that $\varphi\in L^{1}{\dot{Y}}$ which $\varphi$ has right- and left-hand limits? 11 Suppose that $I=[a,b],\,\Omega$ is a region, $I\subset\Omega,f$ is continuous in $\Omega,$ , and $f\in H(\Omega-I).$ Prove that actu $\mathrm{\lally}\,f\in H(\Omega).$ by some other sets for which the same conclusion can be drawn Replace ${\mathbf{}}I$ 12 (Harnack's Inequalities) Suppose $\;\underline{{\Omega}}$ (depending on $z_{0},K,$ and is a compact subset of QL $,\,z_{0}\in\Omega.$ Prove that is a region, $\scriptstyle{\mathcal{K}}$ there exist positive numbers $^{\alpha}$ and $\textstyle\beta$ $\Omega)$ such that $$ \alpha u(z_{0})\leq u(z)\leq\beta u(z_{0}) $$ for every positive harmonic function ${\mathcal{1}}$ in s $\underline{{\Omega}}$ and for ${\mathbf{a}}||z\in K.$HARMONIC FUNCTIONs 251 of $\left\{\mathcal{M}_{\scriptscriptstyle H}\right\}$ in the rest of $\Omega.$ is a sequence of positive harmonic functions in $u_{n}(z_{0})\to\varnothing.$ Show that the assumed positivity of $\left\{\mathcal{M}_{m}\right\}$ is If $\left\{\mathcal{M}_{m}\right\}$ Do the same if $\underline{{\Omega}}$ Q and if $u_{n}(z_{0})\to0,$ describe the behavior essential for these results 13 Suppose ${\mathbf{}}^{u}$ is a positive harmonic function in U and $u(0)=1.$ How large can u) be? How small $U_{\mathbf{\delta}}$ Get the best possible bounds. 14 For which pairs of lines $L_{1}\cup L_{2}$ $\textstyle{\bar{\mathbf{C}}}$ such that do there exist real functions, harmonic in the whole plane, that are for every $e^{i\theta}\neq1.$ Prove 0 at all points of $L_{\mathrm{1}}$ ${\cal L}_{2}$ without vanishing identically? 15 Suppose u is a positive harmonic function in ${\boldsymbol{U}},$ and $u(r e^{i\theta}) arrow0$ as $r\to1.$ that there is a constant $$ u(r e^{i\theta})=c P_{r}(\theta). $$ 16 Here is an example of a harmonic function in $U_{\mathbf{\delta}}$ which is not identically $\mathbf{\partial}$ but all of whose radial limits are ${\mathfrak{O}}\colon$ $$ u(z)=\operatorname{Im}\left[\left({\frac{1+z}{1-z}}\right)^{2}\right]. $$ Prove that this uis not the Poisson integral of any measure on ${\mathbf{}}T$ " and that i $\mathbf{i}$ it is not the difference of two positive harmonic functions in ${\boldsymbol{U}},$ $17$ Let $\Phi$ be the set of all positive harmonic functions ${\mathcal{U}}$ in $\boldsymbol{\mathit{U}}$ such that $u(0)=1.$ Show that D is a convex set and find the extreme points of o.(A point $\textstyle X$ in a convex set $\Phi$ is called an extreme point of O if x lies on no segment both of whose end points lie in $\Phi$ D and are different from x.) Hint: If ${\mathbf{}}C$ is the convex set whose members are the positive Borel measures on ${\boldsymbol{T}},$ of total variation 1, show that the extreme points of $\scriptstyle{\vec{C}}$ are precisely those $\mu\in C$ whose supports consist of only one point of ${\boldsymbol{T}}.$ to $\Lambda$ ∈ $X^{\bullet}$ if $\Lambda_{n}x\to\Lambda x$ be the dual space of the Banach space for every $\scriptstyle x\circ X\,$ Note that $\Lambda_{n}\to\Lambda$ weakly whenever $\Lambda_{n} arrow\Lambda$ in the 18 Let $X^{\bullet}$ $X.$ A sequence $\{\Lambda_{n}\}$ in $X^{\bullet}$ is said to converge weakly norm of as $\quad n\to\infty,$ $X^{\bullet}.$ (See Exercise 8, Chap. 5S)) The converse need not be true. For example, the functionals $f\to{\hat{f}}(n)$ on $L^{2}(T)$ tend to O weakly (by the Bessel inequality), but each of these functionals has norm 1. Prove that $\{\|\Lambda_{n}\|\}$ must be bounded if $\{\Lambda_{n}\}$ converges weakly. 19(a) Show that $\beta P_{i}(0)>1$ if $\delta=1-r,$ is the arc with center 1 and length $2\delta,$ show that (b)1 ${\textsf{f}}\mu\geq0,u=P[d\mu],$ and $I_{\delta}\subset T$ $$ \mu(I_{\delta})\leq\delta u(1-\delta) $$ and that therefore $$ (M\mu)(1)\leq\pi(M_{r a d}\,u)(1). $$ (c) If,furthermore, $\mu\perp\pi$ n, show tha $$ u(r e^{|\beta|})\to\emptyset\qquad\mathrm{a.e.}\,\,[\mu], $$ Hint: Use Theorem 7.15 20 Suppose $E\subset T,m(E)=0.$ Prove that there is $\mathbf{a}\mathbf{z}=H^{\alpha},{\mathrm{with}}f(0)=1,\operatorname{thathas}$ $$ \operatorname*{lim}_{r arrow1}f(r e^{i\theta})=0 $$ at every $e^{i\theta}\in E.$ whose real part is $P[\psi].{\bf L e t}f=1/g.$ $\psi\in L^{1}(T),$ $\psi>0,\,\psi=\,+\,\alpha.$ at every point of E. There Sugestion: Find alower semicontinuous is a holomorphic $\scriptstyle{\mathcal{G}}$ 21 Define fe $H(U)$ and $g\in H(U)$ by $f(z)=\exp{\left\{(1+z)/(1-z)\right\}},$ $g(z)=(1-z)$ exp {-f(z)}. Prove that $$ g^{k l}(e^{l\theta})=\operatorname*{lim}_{r arrow1}g(r e^{i\theta}) $$ exists at every $e^{i\theta}$ e T, that g*e C(T), but that $\scriptstyle{\mathcal{G}}$ is not in $H^{\omega}.$252 REAL AND cOMPLEX ANALYSIs Suggestion: Fix s, put $$ z_{t}={\frac{t+i s-1}{t+i s+1}}\qquad(0<t<\infty). $$ For certain values of s $|\langle(z_{\prime})|\to$ oo ast→ 00. $\mathrm{22}$ Suppose u is harmonic in ${\boldsymbol{U}},$ and $\{u,\cdot0\leq r<1\}$ is a uniformly integrable subset of $L^{1}(T).$ (See Exercise 10, Chap. 6.) Modify the proof of Theorem 11.30 to show that $u=P[f]$ for somefe $L^{l}(T).$ $23$ Put $\theta_{n}=2^{-n}$ and define $$ \L_{l}\Big(\bar{\zeta}\Big)\equiv\sum_{\bf p\atop{\hbar=1}}^{\bar{\alpha}_{0}}\Big|^{*}\mathcal{L}\Big\{\bar{\cal{P}}\Big(\bar{\zeta}_{i},\,\bar{\varrho}^{ |\bar{\beta}_{i} \rangle}-\bar{\cal{P}}\Big(\bar{\zeta}_{i},\,\varrho^{- |\bar{\beta}_{n} \rangle}\Big\}, $$ for $z\in U$ . Show that uis the Poisson integral of a measure on ${\boldsymbol{T}},$ that $u(x)=0$ if $-1<x<1.$ but that $$ u(1-\epsilon+i\epsilon) $$ is unbounded, as e decreases to O.(Thus uhas a radial limit, but no nontangential limit, at 1.) Hint: If∈ = sin $\theta\qquad\theta$ P is small and $z=1-\epsilon+i\epsilon$ , then $$ P(z,\,e^{i\theta})-P(z,\,e^{-i\theta})>1/\epsilon. $$ 24 Let D,t) be the Dirichlet kernel, as in Sec. 5.11, define the Fejer kernel by $$ K_{N}=\frac{1}{N+1}\,(D_{0}+D_{1}+\cdots+D_{N}), $$ put $L_{N}(t)=\operatorname*{min}\;(N,\,\pi^{2}/N t^{2}).$ Prove that $$ K_{N-1}(t)={\frac{1}{N}}\cdot{\frac{1-\cos\,N t}{1-\cos\,t}}\leq L_{N}(t) $$ and that Jr ${\cal L}_{N}$ do $\leq2.$ Use ths to prove that the arihmetic means $$ \sigma_{N}={\frac{S_{0}+S_{1}+\cdots+S_{N}}{N+1}} $$ point of the partial sums ${\mathfrak{s}}_{n}$ of the Fourier series of a function fe $M f,$ then proceed as in the proof of Theorem 11.23.) $f(e^{i\theta})$ at every Lebesgue $L^{1}(T)$ converge to ${\mathrm{of}}f.$ (Show that sup $|\sigma_{N}|$ is dominated by $\mathbb{Z}\mathbb{S}$ If $\cdot1\le p\le\infty\ \mathrm{and}\,f\in D(R^{1}),$ prove that $(f*\;h_{2})(x)$ is a harmonic function of $x+i\lambda$ in the upper half plane. $(h_{\lambda}$ is defined in Sec $9.7;\mathrm{it}$ is the Poisson kernel for the half plane.)