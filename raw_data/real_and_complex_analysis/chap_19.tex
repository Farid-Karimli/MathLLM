CHAPTER NINETEEN HOLOMORPHIC FOURIER TRANSFORMS Introduction function $\hat{f}$ on $R^{1}.$ 19.1 In Chap. 9 the Fourier transform of a function $\hat{f}$ f can be extended to a function which is holo- was defined to be a $\boldsymbol{\mathsf{f}}$ on ${\boldsymbol{R}}^{1}$ Frequently $(1+x^{2})^{-1},$ morphic in a certain region of the plane. For instance, if $f(t)=e^{-|t|},$ $\operatorname{hen}f(x)=$ a rational function. This should not be too surprising. For each real t, the kernel $\!\,e^{i t z}$ sis an entire function of z,so one should expect that there are conditions onfunder which fwill be holomorphic in certain regions. We shall describe two classes of holomorphic functions which arise in this manner For the first one,let ${\mathbf{}}F$ be any function in ${\mathcal{L}}(-\infty,$ o) which vanishes on (-00,0) [i.e., take $F\in L^{2}(0,$ o)] and define $$ f(z)=\int_{0}^{\infty}F(t)e^{i t z}\;d t\;\;\;\;\;\;\;\;(z\in\Pi^{+}), $$ (1) where $\Pi^{+}$ is the set of all $z=x+i y$ with $\scriptstyle{\nu\gg0}$ If $z\in\Pi^{+}$ ,then $|e^{i t}|=e^{-t y},$ which shows that the integral in(1) exists as a Lebesgue integral If Im $z>\delta>0.$ Im $z_{n}>\delta,$ and $z_{n} arrow z,$ the dominated convergence theorem shows that $$ \operatorname*{lim}_{n arrow\infty}\,\left[\operatorname*{su}_{0}^{(\infty}|\exp\,(i t z_{n})-\exp\,(i t z)|^{2}\,\,d t=0\right] $$ because the integrand is bounded by the ${\boldsymbol{L}}^{1}$ -function 4 exp $(-2\delta t)$ and tends to 0 $\Pi^{+}$ for every $\scriptstyle t\,>0,$ The Schwarz inequality implies therefore that $\boldsymbol{\mathit{f}}$ is continuous in The theorems of Fubini and Cauchy show that $\textstyle\int_{\gamma}\,f(z)\,\,d z=0$ for every closed path y in II*. By Morera's theorem, fe H(II+) 371372 REAL AND coMPLEX ANALYSis Let us rewrite (1) in the form $$ f(x+i y)=\int_{0}^{e_{0}}F(t)e^{-t y}e^{i t x}\,d t, $$ (2) regard $\scriptstyle{\mathcal{I}}$ as fixed, and apply Plancherel's theorem. We obtain $$ {\frac{1}{2\pi}}\int_{-\infty}^{\infty}|f(x+i y)|^{2}\,\,d x=\int_{0}^{\infty}|F(t)|^{2}e^{-2t y}\,\,d t\leq\int_{0}^{\infty}|F(t)|^{2}\,\,d t $$ (3) for every $\scriptstyle y\geq0$ [Note that our notation now differs from that in Chap. 9. There $\sqrt{2\pi}.$ Here we just use the underlying measure was Lebesgue measure divided by Lebesgue measure. This accounts for the factor 1/(2元) in (3).] This shows: (a)Iff is of the form(1), then f is holomorphic in $\Pi^{+}$ and its restrictions to horizon- tal lines in $\Pi^{+}$ form a bounded set in ${\mathcal{L}}(-\infty,$ oO). Our second class consists of all f of the form $$ f(z)=\prod_{i=1}^{A}F(t)e^{i t z}~d t $$ (4) where $\ 0<A<\infty$ and $F\in L^{2}(-A,\ A).$ These functions $\boldsymbol{\mathit{f}}$ are entire (the proof is the same as above), and they satisfy a growth condition: $$ |\,f(z)|\leq\int_{-A}^{|A}|F(t)|\,e^{-t y}\,d t\leq e^{A|y|}\ \int_{-A}^{|A}|F(t)|\,d t. $$ (5) If C is this last integral, then $c<\alpha,$ and (5) implies that $$ |f(z)|\leq C e^{A|z|}. $$ (6) [Entire functions which satisfy (6) are said to be of exponential type.] Thus: (b)Every f of the form(4))is an entire function which satisfies (6) and whose restriction to the real axis lies in $L^{2}$ (by the Plancherel theorem) It is a remarkable fact that the converses of (a) and (b) are true. This is the content of Theorems 19.2 and 19.3. Two Theorems of Paley and Wiener 19.2 Theorem Suppose fe H(II+)and $$ \operatorname*{sup}_{0<y<\alpha}\frac{1}{2\pi}\int_{-\alpha}^{\infty}|f(x+i y)|^{2}~d x=C<\infty. $$ (1) Then there exists an $F\in L^{2}(0,\infty)$ such that $$ f(z)=\int_{0}^{\infty}F(t)e^{i t z}~d t\qquad(z\in\Pi^{+}) $$ (2)HOLoMORPHIC FOURIER TRANSFORMS $373$ and $$ \bigcap_{0}^{\infty}|F(t)|^{2}\ d t=C. $$ (3) Note: The function ${\mathbf{}}F$ we are looking for is to have the property that f(x + iy) is the Fourier transform of $F(t)e^{-y t}$ (we regard $\scriptstyle{\mathcal{I}}$ as a positive constant). Let us apply the inversion formula (whether or not this is correct does not matter; we are trying to motivate the proof that follows): The desired ${\mathbf{}}F$ should be of the form $$ F(t)=e^{t y}\cdot{\frac{1}{2\pi}}\int_{-\infty}^{\infty}f(x+i y)e^{-i t x}\,d x={\frac{1}{2\pi}} |f(z)e^{-i z}\,d z. $$ (4) The last integral is over a horizontal line in $\Pi^{+}\!,$ and if this argument is correct at al] the integral will not depend on the particular line we happen to choose. This suggests that the Cauchy theorem should be invoked. PROOF Fix y, $\scriptstyle0<y<\omega,$ For each $\scriptstyle\alpha>0$ let $\Gamma_{\alpha}$ be the rectangular path with vertices at ±α. +iand $\pm\alpha+i y.\,\mathrm{By}$ Cauchy's theorem $$ \bigcap_{T_{x}}f(z)e^{-i t z}\,d z=0. $$ (5) $I=[1,y]$ if $1<y.$ Then $\beta+i$ to $\beta+i y\,(\beta$ real). Put be the integral of f(z)e- i" over 1] if $y<1,$ We consider only real values of t. Let $\Re\theta\theta$ $\scriptstyle I=\mathbb{D},$ the straight line interval from $$ |\,\Phi(\beta)|^{2}=\left|\,\right|\!\int_{I}f(\beta+i u)e^{-i u(\beta+i u)}\,d u |^{2}\leq\left|\,\!\mid\!f(\beta+i u)\!\mid^{2}\,d u\,\right|_{I}e^{2u}\,d u $$ (6) Pu $$ \Lambda(\beta)= \{_{I}f(\beta+i u){\big|}^{2}\ d u. $$ (7D) Then (1) shows, by Fubini's theorem, that $$ {\frac{1}{2\pi}}\int_{-\infty}^{\infty}\Lambda(\beta)\ d\beta\leq C m(I). $$ (8) Hence there is a sequence $\{\alpha_{j}\}$ such that αj→ Oo and $$ \Lambda(\alpha_{j})+\Lambda(-\alpha_{j})\to0\qquad(j\to\infty). $$ (9) By (6), this implies that $$ \Phi(\alpha_{j})\to0,\qquad\Phi(-\alpha_{j})\to0\qquad\mathrm{as~j\to\infty.} $$ (10)374 REAL AND coMPLEX ANALYSIs Note that this holds for every ${\hat{\mathbf{r}}}$ and that the sequence $\{\alpha_{j}\}$ does not depend on t. Let us define $$ g_{j}(y,t)={\frac{1}{2\pi}}\int_{-x_{j}}^{x_{j}}f(x+i y)e^{-i t x}\;d x. $$ (11) Then we deduce from (5) and (10) that $$ \operatorname*{lim}_{j arrow\infty}\left[e^{\sigma}g_{j}(y,t)-e^{t}g_{j}(1,t)\right]=0\qquad(-\infty<t<\infty). $$ (12) Write $\scriptstyle{\sqrt{a x}}$ for $f(x+i y).$ Then $f_{\gamma}\in L^{2}(-\infty,$ 0O), by hypothesis, and the Plancherel theorem asserts that $$ \operatorname*{lim}_{j\to\infty}\;\ \;\;\stackrel{\longleftarrow}{\mapsto}|\hat{J}_{y}(t)-g_{j}(y,\,t)|^{2}\;d t=0, $$ (13) where f ${\hat{f}}_{y}$ , is the Fourier transform $\scriptstyle{f_{i}\theta_{i}}$ for almost all t (Theorem 3.12). If we define $\{g_{i}(y,t)\}_{\sim}$ converges therefore pointwise to $\operatorname{of}_{r},$ A subsequence of $$ F(t)=e^{i\!\!\!{\hat{f}}_{1}(t)}, $$ (14) it now follows from (12) that $$ F(t)=e^{\imath t}{\hat{J}}_{y}(t). $$ (15) Note that (14) does not involve $\scriptstyle{\mathcal{I}}$ and that (15) holds for every $y\in(0,\infty).$ Plancherel's theorem can be applied to (15): $$ |\stackrel{\infty}{\ldots}e^{-2t y}|F(t)|^{2}\;d t= |\stackrel{\cdots}{\ldots}|\hat{J}_{y}(t)|^{2}\;d t=\frac{1}{2\pi}\int_{-\infty}^{\infty}|f_{y}(x)|^{2}\;d x\leq C. $$ (16) If we let $y arrow\infty,(16)$ shows that $F(t)=0\ a.\mathrm{e},\,\mathrm{in}\,(-\infty,\,0).$ If we let y→0,(16) shows that $$ \bigcap_{0}^{\infty}|F(t)|^{2}\;d t\leq C. $$ (17) It now follows from (15) that f,e Ll if $\scriptstyle v\gg0$ Hence Theorem 9.14 gives $$ f_{y}(x)=\left\{\O_{-\infty}^{\infty}{\hat{f}}_{y}(t)e^{i t x}\,d t\right. $$ (18) or $$ f(z)=\int_{0}^{+\infty}F(t)e^{-y t}e^{i t x}\,d t= (\sum_{0}^{+\infty}F(t)e^{i t z}\,d t\qquad(z\in\Pi^{+}). $$ (19) This is (2), and now (3) follows from (17) and formula_19.!:3) //HOLOMORPHIC FOURIER TRANSFORMs 375 19.3 Theorem Suppose $\scriptstyle A$ and ${\boldsymbol{C}}$ are positive constants and f is an entire func- tion such that $$ |f(z)|\leq C e^{A|z|} $$ (1) for all z, and $$ \bigcap_{-\infty}^{\infty}|f(x)|^{2}\ d x<\infty. $$ (2) Then there exists an $F\in L^{2}(-A,\,A)$ such that $$ f(z)=\int_{-A}^{A}F(t)e^{i t z}~d t $$ (3) for all z PRoor ${\mathrm{Put}}\,f_{e}(x)=f(x)e^{-c|x|},$ for ∈> 0 and $\scriptstyle{\mathcal{X}}$ real. We shall show that $$ \operatorname*{lim}_{\epsilon arrow0}\bigg\{\O_{-\infty}^{\epsilon\infty}f_{\epsilon}(x)e^{-i t x}\,d x=0\qquad(t\mathrm{~real},\,|t|>4). $$ (4) Since $\|f_{\epsilon}-f\|_{2}\to0$ as $\scriptstyle\epsilon\to0.$ the Plancherel theorem implies that the of f(more Fourier transforms ${\mathfrak{o l}}i$ converge in ${\boldsymbol{L}}^{2}$ to the Fourier transform ${\mathbf{}}F$ precisely, of the restriction of f to the real axis). Hence (4) will imply that ${\mathbf{}}F$ vanishes outside [-A, A], and then Theorem 9.14 shows that (3) holds for almost every real z. Since each side of (3) is an entire function, it follows that (3) holds for every complex z Thus (4) implies the theorem. For each real $\alpha,$ , let $\Gamma_{\alpha}$ be the path defined by $$ \Gamma_{\alpha}(s)=s e^{i\alpha}\qquad(0\le s<\infty), $$ (5) put $$ \Pi_{\alpha}=\{w\colon\mathrm{Re}\;(w e^{i\alpha})>A\}, $$ (6) and if $w\in\Pi_{\alpha},$ define $$ \Phi_{\alpha}(w)=\int_{\Gamma_{\alpha}}f(z)e^{-\,w z}\,d z=e^{i\alpha}\int_{0}^{\infty}f(s e^{i\alpha})\,\exp\left(-\,w s e^{i\alpha}\right)\,d s. $$ (7) By (1) and (5), the absolute value of the integrand is at most $$ {\cal C}\,\exp\,\{-[{\mathrm R e}\,(w e^{i x})-A\}s\}, $$ and it follows (as in Sec.19.1) that $\Phi_{\alpha}$ is holomorphic in the half plane $\Pi_{\alpha}.$ However, more is true if $\scriptstyle x\;=\;0$ and if $\alpha=\pi\colon$ We have $$ \Phi_{0}(w)=\bigcap_{w}^{\infty}f(x)e^{-\,w x\,}d x\qquad(\mathrm{Re}\ w>0), $$ (8)376 REAL AND CoMPLEX ANALYSIs $$ \Phi_{\pi}(w)=\bigcap_{-\infty}^{\infty}f(x)e^{-w x}\,d x\qquad(\mathrm{Re~w<0}). $$ (9) $\Phi_{0}$ and $\Phi_{\pi}$ are holomorphic in the indicated half planes because of (2) The significance of the functions $\Phi_{\alpha}$ to (4) lies in the easily verified rela_ tion $$ \left.\stackrel{\rightharpoonup}{\to}_{-\infty}f_{\epsilon}(x)e^{-i x}\,d x=\Phi_{0}(\epsilon+i t)-\Phi_{\pi}(-\epsilon+i t)\qquad(t\ \mathrm{real}). $$ (10) Hence we have to prove that the right side of (10) tends to $\mathbf{0}$ as $\scriptstyle\epsilon\to0.$ if $t>A$ and if $t<-A.$ We shall do this by showing that any two of our functions $\Phi_{\alpha}$ agree in in (10) f $t<-A,$ and by $\Phi_{-\pi/2}$ the intersection of their domains of definition, i.e., that they are analytic con- and $\Phi_{\pi}$ 。by $\Phi_{\pi/2}$ tinuations of each other. Once this is done, we can replace if $t>A,$ and it is then obvious that the $\Phi_{0}$ difference tends to $\mathbf{0}$ as $\scriptstyle\epsilon\to0.$ So suppose $0<\beta-\alpha<\pi.$ Put $$ \gamma={\frac{\alpha+\beta}{2}},\qquad\eta=\cos{\frac{\beta-\alpha}{2}}>0. $$ (11) If $w=|w|\,e^{-i\gamma},$ then $$ \mathrm{Re}\,(w e^{i x})=\eta\mid w\mid=\mathrm{Re}\,(w e^{i\beta}) $$ (12) so that $w\in\Pi_{\alpha}\cap\Pi_{\beta}$ as soon as $\vert w\vert>A/\eta.$ Consider the integral $$ \bigcap_{T}f(z)e^{-\omega z}\,d z $$ (13) over the circular arc ${\Gamma}$ given by $\Gamma(t)=r e^{t t},\alpha\leq t\leq\beta.$ Since $$ \mathrm{Re}\,(-\,w z)=\,-\,|\,w\,|\,r\,\cos\,(t-\gamma)\leq\,-\,|\,w\,|\,r\eta, $$ (14) the absolute value of the integrand in (13) does not exceed $$ C\exp{\bf\{}(A-|{\boldsymbol{w}}|{\boldsymbol{\eta}}){\boldsymbol{r}}\}. $$ If $|w|>A/\eta$ it follows that (13) tends to O as $r arrow\infty.$ We now apply the Cauchy theorem. The integral of $f(z)e^{-\Phi t}$ over the interval [O, re'门] is equal to the sum of $(13)$ and the integral over $[0,\,r^{(n)}].$ Since (13) tends to O as r→0o,we conclude that $\Phi_{\alpha}(w)=\Phi_{\beta}(w)$ if $w=|w|\,e^{-i\gamma}$ and $|w|>A/\eta,$ and then Theorem 10.18 shows that $\Phi_{x}$ and $\Phi_{\beta}$ coincide in the intersection of the half planes in which they were originally defined This completes the proof. // 19.4 Remarks Each of the two preceding proofs depended on a typical appli- cation of Cauchy's theorem. In Theorem 19.2 we replaced integration over one horizontal line by integration over another to show that 19.2(15) wasHOLOMORPHIC FOURIER TRANSFORMS $377$ independent of y. In Theorem 19.3, replacement of one ray by another was used to construct analytic continuations; the result actually was that the functions $\Phi_{\alpha}$ are restrictions of one function $\Phi$ which is holomorphic in the complement of the interval[- Ai, Ai]. The class of functions described in Theorem 19.2 is the half plane ana- logue of the class I $H^{2}$ discussed in Chap. 17. Theorem 19.3 will be used in the proof of the Denjoy-Carleman theorem (Theorem 19.11) Quasi-analytic Classes 19.5 If $\Omega$ is a region and if $\scriptstyle t_{\mathrm{a}}\in\Omega_{\mathrm{b}}$ every fe H(Q) is uniquely determined by the numbers $f(z_{0}),f^{\prime}(z_{0}),f_{.}^{\prime}(z_{0}),$ ..On the other hand, there exist infinitely differen- tiable functions on $R^{1}$ which are not identically O but which vanish on some interval. Thus we have here a uniqueness property which holomorphic functions possess but which does not hold in $C^{\infty}$ (the class of all infinitely differentiable complex functions on R $R^{1}$ If fe H(Q), the growth of the sequence $\left\{\left|\int^{(m)}(z_{0})\right|\right\}$ is restricted by Theorem 10.26.It is therefore reasonable to ask whether the above uniqueness property holds in suitable subclasses of $C^{\infty}$ in which the growth of the derivatives is subject to some restrictions. This motivates the following definitions; the answer to our question is given by Theorem 19.11. 19.6 The Classes $C\{M_{n}\}$ If $M_{0},\,M_{1},\,M_{2},\,\dots$ are positive numbers, we let $C\{M_{n}\}$ be the class of all f e $C^{\infty}$ which satisfy inequalities of the form $$ |D^{n}f\,||_{\infty}\leq\beta_{f}B_{f}^{n}\,M_{n}\qquad(n=0,\,1,\,2,\,...). $$ (1) norm over $R^{1},$ and $\beta_{f}$ D"f is the nth derivative of $\boldsymbol{\mathsf{f}}$ if $n\geq1,$ the norm is the supremum but not on Here $D^{0}f=f,\ l$ and $B_{f}$ are positive constants (depending $\scriptstyle{\mathrm{onf.}}$ $n\mathbf{)}.$ If f satisfies (1), then $$ \operatorname*{lim}_{n arrow\infty}\L\diamond\sqrt\frac{\|D^{n}f\|_{\infty}}{M_{n}}\bigg>^{1/n}\leq B_{f}\,. $$ (2) omitted in(1), the case $\textstyle n=0$ is a more significant quantity than $\beta_{f}.$ However, if $\beta_{f}$ were This shows that $B_{f}$ would imply $\|f\|_{\alpha}\leq M_{\mathrm{o}},$ an undesirable Each restriction. The inclusion of $\beta_{f}$ makes ${\cal C}\{M_{n}\}$ into a vector space. ${\mathit{C}}\{M_{n}\}$ is inariant under affine transformations. More explicitly, suppose $f\in C\{M_{n}\}$ and $g(x)=f(a x+b).$ Then g satisfies (1), with $\beta_{g}=\beta_{f}$ and $B_{g}=a B_{f}$ We shall make two standing assumptions on the sequences $\{M_{n}\}$ under con- sideration: $$ \begin{array}{l l}{{{\cal M}_{0}=1.}}\\ {{{\cal M}_{n}^{2}\leq{\cal M}_{n-1}{\cal M}_{n+1}}}&{{(n=1,\,2,\,3,\,...).}}\end{array} $$ (4) (3) Assumption (4) can be expressed in the form: {log $\scriptstyle M_{\mathrm{sl}}$ is a convex sequence.378 REAL AND coMPLEX ANALYSIs These assumptions will simplify some of our work, and they involve no los of generality. [One can prove, although we shall not do so,that every class $C\{M_{n}\}$ is equal to a class $C\{M_{n}\}.$ , where $\{M_{n}\}$ satisfies (3) and (4).] The following result illustrates the utility of (3) and (4) tion. 19.7 Theorem Each $C\{M_{n}\}$ is an algebra, with respect to pointwise multiplica PR0OF Suppose f and $g\in C\{M_{n}\},$ and $\beta_{f},B_{f},\beta_{g},$ and $B_{g}$ are the correspond- ing constants. The product rule for differentiation shows that $$ D^{n}(f g)=\sum_{j=0}^{n}{\binom{n}{j}}(D^{j}f)\cdot(D^{n-j}g). $$ (1) Hence $$ |D^{n}(f g)|\leq\beta_{f}\beta_{g}\sum_{j=0}^{n}{\binom{n}{j}}B_{f}^{j}B_{g}^{n-j}M_{j}M_{n-j}. $$ (2) The convexity of {log $\scriptstyle M_{\mathrm{sl}}$ combined with $M_{0}=1,$ shows that $M_{j}M_{n-j}\leq$ $M_{n}\operatorname{for}0\leq j\leq n.$ Hence the binomial theorem leads from (2) to $$ |D^{n}(f g)||_{\infty}\leq\beta_{f}\,\beta_{g}(B_{f}+B_{g})^{n}M_{n}\qquad(n=0,\;1,\,2,\,...), $$ (3) so $\operatorname{that}f g\in C\{M_{n}\}.$ // 19.8 Definition A class $C\{M_{n}\}$ is said to be quasi-analytic if the conditions $$ \begin{array}{l l}{{f\in C\{M_{n}\},}}&{{\ \ \ (D^{n}f)(0)=0\qquad(\mathrm{for}\ n=0,1,\,2,\,...)}}\end{array} $$ (1) imply $\operatorname{that}f(x)=0$ for all $x\in R^{\dagger}$ The content of the definition is of course unchanged if(D"fXO) is replaced by $(D^{i}f)(x_{0}),$ where $\scriptstyle{X_{0}}$ is any given point. The quasi-analytic classes are thus the ones which have the uniqueness property we mentioned in Sec. 19.6. One of these classes is very intimately related to holomorphic functions: 19.9 Theorem The class $\scriptstyle C[n]\!\!1$ consists of all f to which there corresponds a $\scriptstyle\delta>0$ such that $\boldsymbol{\mathit{f}}$ can be extended to a bounded holomorphic function in the strip defined by |Im(z)|< 6. Consequently $\mathbb{C}[n]\!]$ is a quasi-analytic class PROOF Suppose $f\in H(\Omega)$ and $|f(z)|<\beta$ for all z ∈ Q2, where $\Omega$ consists of al $z=x+i y$ with $|y|<\delta.$ It follows from Theorem 10.26 that $$ \lfloor(D^{n}f)(x)\rfloor\leq\beta\delta^{-n}n!\qquad(n=0,\;1,\;2,\ldots) $$ (1) for all real x. The restriction of fto the real axis therefore belongs to $\scriptstyle C(n){\big|}$HOLoMORPHIC FoURIER TRANSFORMs 379 Conversely, suppose $\boldsymbol{\mathsf{f}}$ is defined on the real axis and fe C{n!}. In other words, $$ \|D^{n}f\|_{\infty}\leq\beta B^{n}n!\qquad(n=0,\ 1,\ 2,\ldots). $$ (2) We claim that the representation $$ f(x)=\sum_{n=0}^{\infty}{\frac{(D^{n}f)(a)}{n!}}\,(x-a)^{n} $$ (3) is valid for all $\alpha\in R^{\prime}$ if $a-B^{-1}<x<a+B^{-1}$ . This follows from Taylor's formula $$ f(x)=\sum_{j=0}^{n-1}{\frac{(D^{j}f)(a)}{j!}}\,(x-a)^{j}+{\frac{1}{(n-1)!}}\int_{a}^{x}(x-t)^{n-1}(D^{n}f)(t)\ d t, $$ (4) which one obtains by repeated integrations by part. By (2) the last term in (4) (the“remainder ”) is dominated by $$ n\beta B^{n}\left|\right|_{a}^{x}(x-t)^{n-1}~d t |=\beta\left|B(x-a)\right|^{n}. $$ (5) 1 $\mathbb{f}\,|\,B(x-a)|<1,$ this tends to $\mathbf{0}$ as $n\to G,$ and (3) follows. We can now replace x in(3)by any complex number z such that at a and radius $1/B,$ and This defines a holomorphic function if $\scriptstyle{\mathcal{X}}$ is real and $F_{a}$ in the disc with center The $|z-a|<1/B.$ various functions $F_{a}(x)=f(x)$ $|x-a|<1/B.$ $F_{a}$ are therefore analytic continuations of each other; they If form a holomorphic extension F and $z=a+i y,|y|<\delta,$ then $|y|<1/B.$ 中 'of f in the strip ${\mathbf{}}F$ $0<\delta<1/B$ $$ |F(z)|=|F_{a}(z)|=\left|\sum_{n=0}^{\infty}\frac{(D^{n}f)(a)}{n!}\left(i y\right)^{n}\right|\leq\beta\sum_{n=0}^{\infty}\left(B\delta\right)^{n}=\frac{\beta}{1-B\delta}. $$ This shows that ${\mathbf{}}F$ is bounded in the strip $|y|<\delta,$ and the proof is complete // 19.10 Theorem The class ${\cal C}\{M_{n}\}$ is quasi-analytic if and only i $C\{M_{n}\}$ con- tains no nontrivial function with compact support PRoOF If $C\{M_{n}\}$ $\boldsymbol{\mathit{f}}$ and all its derivatives vanish at some point, hence and $g(x)=0$ for $x<0,$ then g ∈ // $h(x)=0$ trivial member of $C\{M_{n}\}$ is quasi-analytic, $\operatorname{iff}\in C\{M_{n}\},$ and if f has compact support, $f(x)=0$ then evidently for all x. $C\{M_{n}\}$ is not quasi-analytic. Then there exists an $f\in C\{M_{n}\}$ $C\{M_{n}\}.$ Suppose for $n=0,$ 1, 2,.……,but f(xo) ≠ 0 for some $\scriptstyle x_{0}$ . We may such that $(D^{i j})(0)=0$ If $g(x)=f(x)$ for $x\geq0$ $h(x_{0})=f^{2}(x_{0})\neq0.$ Thus $\ \boldsymbol{h}$ is a non- assume $v_{n}>0$ $h(x)=g(x)g(2x_{0}-x).$ But By Theorem 19.7, $h\in C\{M_{n}\}.$ Also, Put if $x<0$ and if $x>2x_{0}$ with compact support.380 REAL AND coMPLEX ANALYSIS We are now ready for the fundamental theorem about quasi-analytic clases The Denjoy-Carleman Theorem 19.11 Theorem Suppose $M_{0}=1$ ,M, ≤ M,-1Mm+1 for n = 1, 2,3,.…,and $$ Q(x)=\sum_{n=0}^{\infty}\,{\frac{x^{n}}{M_{n}}},\qquad q(x)=\operatorname*{sup}_{n\geq0}{\frac{x^{n}}{M_{n}}}, $$ for $\scriptstyle x\;>0$ Then each of the following five conditions implies the other four: (a) $C\{M_{n}\}$ is not quasi-analytic (6) Jo log Q(x) dx < OO 1 +x (c) /C log q(x dx oo 1 +x 1/元 (d M, oO (e) .A M,-1 < 0 M, Note::If $M_{n}\to\mathbb{c}_{n}$ very rapidly as $n\to\infty,$ then $\scriptstyle(\Omega^{\circ})$ tends to infinity slowly as $M_{n}\to\infty$ x→O0. Thus each of the five conditions says, in its own way, that rapidly. Note also that $Q(x)\geq1$ and $q(x)\geq1.$ The integrals in $\mathbf{(}b\mathbf{)}$ and $\mathbf{\Psi}_{(C)}$ are thus always defined. It may happen that $Q(x)=\infty$ for some $x<\infty.$ In that case, the integral (b) is $\scriptstyle+\infty,$ and the theorem asserts that hence $\mathbf{\Psi}({\boldsymbol{e}})$ is violated,and the theorem °it $M_{n}=n^{1},$ then $M_{n-1}/M_{n}=1/n,$ ${\cal C}\{M_{n}\}$ is quasi-analytic asserts that $\scriptstyle C[n]!\quad$ is quasi-analytic, in accordance with Theorem 19.9 $C\{M_{n_{s-}}\}$ PROOF THAT (a) IMPLIES (b)Assume that $\ C(M_{a})$ is not quasi-analytic. Then contains a nontrivial function with compact support (Theorem 19.10) An affine change of variable gives a function $F\in C\{M_{n}\},$ with support in some interval [O, A], such that $$ \|D^{n}F\|_{\alpha}\leq2^{-n}M_{n}\qquad(n=0,\,1,\,2,\,\ldots) $$ (1) and such that ${\mathbf{}}F$ is not identically zero. Define $$ f(z)=\prod_{0}^{A}F(t)e^{i t z}~d t $$ (2) and $$ g(w)=f{\biggl(}{\frac{i-i w}{1+w}}{\biggr)}. $$ (3)HOLOMORPHIC FOURIER TRANSFORMS 381 Then $\boldsymbol{\mathsf{f}}$ $|F(s)|.$ Hence $\boldsymbol{\mathit{f}}$ is continuous on ${\bar{U}},$ the absolute value of the integrand in (2) is $w=-1$ Since is is entire. If Im $\scriptstyle x\;\gg\;0,$ at most is bounded in the upper half plane; therefore $\scriptstyle{\mathcal{G}}$ bounded in ${\boldsymbol{U}}.$ Also, $\scriptstyle{\mathcal{G}}$ except at the point $\boldsymbol{\f}$ is not identically $\mathbf{0}$ (by the uniqueness theorem for Fourier transforms) the ${\mathfrak{g}}_{!}$ same is true of g, and now Theorem 15.19 shows that $$ \frac{1}{2\pi}\left.\right|_{-\pi}^{\pi}\log\left|g(e^{i\theta})\right|\,d\theta>\,-\,\infty. $$ (4) If $x=i(1-e^{i\theta})/(1+e^{i\theta})=\tan\left(\theta/2\right),$ then $d\theta=2(1+x^{2})^{-1}~d$ Ix, so(4) is the same as $$ {\frac{1}{\pi}}\mathop{\bigcup_{-\infty}}^{\prime\infty}\log\mid f(x)\mid{\frac{d x}{1+x^{2}}}>-\infty. $$ (5) On the other hand, partial integration of (2) gives $$ f(z)=(i z)^{-n}\displaystyle\int_{0}^{x}(D^{n}F)(t)e^{i t z}~d t\qquad(z\neq0) $$ (6) since ${\mathbf{}}F$ and all its derivatives vanish at $\mathbf{0}$ and at A. It now follows from (1) and (6) that $$ |x^{n}f(x)|\leq2^{-n}A M_{n}\qquad(x\,\mathrm{real},\,n=0,\,1,\,2,\,\ldots). $$ (7) Hence $$ Q(x)\,|\,f(x)|=\sum_{n=0}^{\infty}\,\frac{x^{n}|\,f(x)|}{M_{n}}\leq2A\qquad(x\geq0), $$ (8) and (5) and (8) imply that $\mathbf{(}b\mathbf{)}$ holds // PRoOF THAT (b) MPLIEs (C) q(x)≤ Q(×) // $M_{n-1}M_{n+1}$ PROOF THAT (c) IMPLIES (d) Put $a_{n}=M_{n}^{1/n}.$ Since $M_{\alpha}=1$ and since $M_{x}^{*}\leq$ it is easily verified that $a_{n}\leq a_{n+1},$ for $\scriptstyle n\gg0$ Tf $x\geq e a_{n},$ then $x^{n}/M_{n}\geq e^{n},$ so $$ \log~q(x)\geq\log~\frac{x^{n}}{M_{n}}\geq\log~e^{n}=n. $$ (9) Hence $$ e\left.\int_{e a_{1}}^{\infty}\log\,q(x)\cdot{\frac{d x}{x^{2}}}\geq e\,\sum_{n=1}^{N}\,n\,\prod_{e a_{n}}^{e a_{n+1}}x^{-2}\,\,d x+e\,\int_{e a_{N}+1}^{\infty}(N+1)x^{-2}\,\,d x\right. $$ $$ =\sum_{n=1}^{N}n\Biggl(\frac{1}{a_{n}}-\frac{1}{a_{n+1}}\Biggr)+\frac{N+1}{a_{N+1}}=\sum_{n=1}^{N+1}\frac{1}{a_{n}} $$ (10) for every $N.$ This shows that (c) implies (d)_ //382 REAL AND cOMPLEX ANALYSiS PRoOF THAT (d) IMPLIES $\mathbf{\Psi}({\boldsymbol{e}})$ Put $$ \lambda_{n}=\frac{M_{n-1}}{M_{n}}\qquad(n=1,\,2,\,3,\,\ldots). $$ (11) Then $s_{1}\simeq s_{2}\simeq s_{3}\pm s\cdot$ and if $a_{n}=M_{n}^{1/n},$ as above, we have $$ (a_{n}\lambda_{n})^{n}\leq M_{n}\cdot\lambda_{1}\lambda_{2}\cdot\cdot\cdot\lambda_{n}=1. $$ (12) Thus $\lambda_{n}\leq1/a_{n},$ and the convergence of $\ \Sigma(1/a_{n})$ implies that of $\Sigma\lambda_{n}.$ // PR0OF THAT $\mathbf{\Psi}({\boldsymbol{e}})$ IMPLIES (a) The assumption now is that $\Sigma\lambda_{n}<\infty,$ where $\lambda_{n}$ is given by (11). We claim that the function $$ f(z)=\left({\frac{\sin z}{z}}\right)^{2}\prod_{n=1}^{\infty}{\frac{\sin\lambda_{n}z}{\lambda_{n}z}} $$ (13)) is an entire function of exponential type, not identically zero, which satisfies the inequalities $$ |x^{k}\!f(x)|\leq M_{k}\biggl({\frac{\sin\,x}{x}}\biggr)^{2}\qquad(x\,\mathrm{real},\,k=0,\,1,\,2,\,...\biggr). $$ (14) Note first that $1-z^{-1}$ sin $\widetilde{\mathbb{Z}}$ has a zero at the origin. Hence there is a constant $\boldsymbol{B}$ such that $$ \left|1-{\frac{\sin z}{z}}\right|\leq B|z|\qquad(|z|\leq1). $$ (15) It follows that $$ \left|1-{\frac{\sin\lambda_{n}z}{\lambda_{n}z}}\right|\leq B\lambda_{n}|z|\ \ \ \ \ \left(|z|\leq{\frac{1}{\lambda_{n}}}\right), $$ (16) so that the series $$ \sum_{n=1}^{\infty}\left|1-{\frac{\sin\;\lambda_{n}z}{\lambda_{n}z}}\right| $$ (17) converges uniformly on compact sets. (Note that $1/\lambda_{n}\to\infty$ as $\quad n\to\infty,$ since $\Sigma{\lambda_{n}}<\infty$ .) The infinite product (13) therefore defines an entire function $\boldsymbol{\mathsf{f}}$ which is not identically zero. Next, the identity $$ {\frac{\sin z}{z}}={\frac{1}{2}}\,\int_{-1}^{1}e^{i t z}\;d t $$ (18) shows that $|z^{-1}$ sin $z\vert\leq e^{\nu\vert}$ if $z=x+i y.$ Hence $$ |f(z)|\le e^{A|z|},\qquad\mathrm{with}\ A=2+\sum_{n=1}^{\infty}\lambda_{n}. $$ (19)HOLOMORPHIC FOURIER TRANSFORMS 383 For real x, we have|si ${\textbf{n}}x\vert\leq\vert x\vert{\underset{\mathrm{and}}{\operatorname{and}}}\vert{\ \mathrm{sin~}}x\vert\leq1.{\textbf{I}}$ lence $$ |x^{k}\!f(x)|\leq|x^{k}|{\bigg(}{\frac{\sin x}{x}}{\bigg)}^{2}\prod_{n=1}^{k}{\bigg|}{\frac{\sin\lambda_{n}x}{\lambda_{n}x}}{\bigg|} $$ $$ \leq\left({\frac{\sin\,x}{x}}\right)^{2}(\lambda_{1}\cdot\cdot\cdot\lambda_{k})^{-1}=M_{k}{\Big(}{\frac{\sin\,x}{x}}{\Big)}^{2}. $$ (20) This gives (14), and if we integrate (14) we obtain $$ \frac1\pi \vert_{-\infty}^{\mathrm{\scriptsize~*co}}\vert x^{k}f(x)\vert\ d x\leq M_{k}\qquad(k=0,1,2,\ldots).\qquad\qquad\qquad\qquad\qquad\qquad\qquad\qquad\qquad\qquad\qquad\qquad\qquad\qquad\qquad\qquad\qquad\qquad\qquad\qquad\qquad\qquad\qquad\qquad\qquad\qquad\qquad\qquad\qquad\qquad\qquad\qquad\qquad\qquad\qquad\qquad\qquad\qquad\qquad\qquad\qquad\qquad\qquad\qquad(\qquad\qquad\qquad\qquad\qquad(\qquad\qquad(1,1) $$ (21) We have proved that $\boldsymbol{\mathit{f}}$ f satisfies the hypotheses of Theorem 19.3. The Fourier transform of f, $$ F(t)={\frac{1}{2\pi}}\int_{-\infty}^{\infty}f(x)e^{-i t x}\;d x\qquad{\mathrm{(f~real)}} $$ (22 is therefore a function with compact support, not identically zero, and(21) shows that $F\in C^{*}$ and that $$ (D^{k}F)(t)=\frac{1}{2\pi}\int_{-\infty}^{+\infty}(-i x)^{k}f(x)e^{-i t x}\;d x, $$ (23) Hence $\ C(M_{a})$ by repeated application of Theorem 9.2(f). Hence $\|D^{k}F\|_{\infty}\leq M_{k},$ by (21), // which shows that $F\in C\{M_{n}\}.$ is not quasi-analytic, and the proof is complete Exercises 1 Suppose fis an entire function of exponential type and $$ \varphi(y)=\left[\!\!\right]_{-\infty}^{\infty}|f(x+i y)|^{2}\ d x. $$ Prove that either $\varphi(y)=\alpha\alpha$ for all real $\mathbf{\vec{y}}$ or $\varphi(y)<\alpha\alpha$ for all real y. Prove that f = 0 if p is a bounded function. 2 Supposef is an entire function of exponential type such that the restriction of f to two nonparalle lines belongs to $L^{2}.$ Prove that f= 0. 3 Suppose $\boldsymbol{\mathit{f}}$ is an entire function of exponential type whose restriction to two nonparallel lines is bounded. Prove that fis constant.(Apply Exercise of Chap. 12.) 4 Suppose fis entire, $\scriptstyle J(x)|<C$ texp (alz|), and $f(z)=\Sigma a_{n}z^{n}\circ\mathrm{Pu}^{\prime}$ $$ \Phi(w)=\sum_{n=0}^{\infty}\frac{n!\,a_{n}}{w^{n+1}}. $$ Prove thathe seres converges $\mathbb{F}\left|w\right|>A,$ that $$ f(z)={\frac{1}{2\pi i}}\int_{\Gamma}\Phi(w)e^{w z}~d w $$384 REAL AND COMPLEX ANALYSIs if $\Gamma(t)=(A+\epsilon)e^{i t},\;0\leq t\leq2\pi,$ and that $\Phi$ is the function which occurred in the proof of Theorem 19.3.(See also Sec. 19.4.) S Suppose f satisfies the hypothesis of Theorem 19.2. Prove that the Cauchy formula $$ f(z)={\frac{1}{2\pi i}} [\stackrel{ rightarrow}{\omega}_{-\alpha}^{\quad\zeta\in\chi\quad\zeta\qquad}\ d{\mathfrak{c}}\qquad(0<\epsilon<y) $$ (*) holds; here $\scriptstyle z\;=\;x\,+\,y,$ Prove that $$ f^{*}(x)=\operatorname*{lim}_{y\to0}f(x+\operatorname{i}y) $$ exists for almost al $\textstyle{\mathcal{X}}$ What is the relation between $f^{\bullet}$ and the function ${\mathbf{}}F$ which occurs in Theorem 19.2? 1s (") true with $\epsilon=0$ and with f* in place of fin the integrand? 6 Suppose $\varphi\in L^{2}(-\infty,\infty)$ and $\varphi>0.$ Prove that there exists an f with $|f|=\varphi$ such that the Fourier transform off vanishes on a half line if and only if $$ \=_{-\infty}^{\infty}\log\varphi(x)\,{\frac{d x}{1+x^{2}}}>-\infty. $$ Suggestion: Consider f*, as in Exercise ${\bf5},\;$ where f= exp $\boldsymbol{u+}$ ip) and $$ u(z)={\frac{1}{\pi}}\left.\right\}_{-\infty}^{\infty}{\frac{y}{(x-t)^{2}+y^{2}}}\log\varphi(t)\;d t. $$ T Let f be a complex function on a closed set $\bar{E}$ in the plane. Prove that the following two conditions on f are equivalent : (a) There is an open set $x\in E$ there corresponds a neighborhood $V_{\alpha}\,$ of α and a function $F_{a}\in H(V_{a})$ such that (b) To each $\Omega\supset E$ and a function $F\in H(\Omega)$ such that $F(z)=f(z)\operatorname{for}z\in E$ $F_{a}(z)=f(z)$ in $V_{a}\cap E.$ (A special case of this was proved in Theorem 19.9.) 8 Prove that $C\{n!\}=\mathbb{C}\{n^{n}\}.$ $C\{n!\},$ 9 Prove that there are quasi-analyticlss which are larger than 10 Put $\lambda_{n}=M_{n-1}/M_{n},$ as in the proof of Theorem 19.11. Pick $g_{0}\in C_{c}(R^{1}),$ and define $$ g_{n}(x)=(2\lambda_{n})^{-1}\,\int_{-\lambda_{n}}^{\lambda_{n}}g_{n-1}(x-t)\;d t\qquad(n=1,\,2,\,3,\,\dots). $$ Prove directly (without using Fourier transforms or holomorphic functions) that $g=\ln\,g_{n}$ is a func tion which demonstrates that (e) implies (a in Theorem 19.11. You may choose any ${\mathfrak{g}}_{0}$ that is conve- nient.) 1 Find an explicit formula for a function $\varphi\in C^{\infty},$ with support in[-2, $2],$ such that $\varphi(x)=1$ if $-1\leq x\leq1.$ that β。x"9(x), and if 12 Prove that to every sequence for $n=0,$ 1,2 *…… Sugestion: I $\textstyle\phi$ is as in Exercise 11, if $\beta_{n}=\alpha_{n}/n!,$ if g,(x) = such $\left\{\alpha_{n}\right\}$ of complex numbers there corresponds a function fe $C^{\alpha}$ $(D^{n}f)(0)=\alpha_{n}$ $$ \textstyle\int_{n}(x)=\lambda_{n}^{-n}g_{n}(\lambda_{n}x)=\oint_{n}x^{n}\varphi(\lambda_{n}x), $$ then $\|D^{k}f_{n}\|_{\infty}<2^{-n}$ for $k=0,\ldots,n-1,$ provided that $\scriptstyle{\dot{\lambda}}_{n}$ 。is arge enough. Take f= Z。 13 Construct afunction fe $C^{\infty}$ such that the power series $$ \sum_{n=0}^{\infty}{\frac{(D^{n}f)(a)}{n!}}\,(x-a)^{n} $$HOLoMORPHIC FoURIER TRANSFORMs 385 has radius of convergence O for every $a\in R^{1}.$ Suggestion: Put $$ f(z)=\sum_{k=1}^{\infty}c_{k}\,e^{i\lambda_{k}x}, $$ where $\{c_{k}\}$ and $\{\lambda_{k}\}$ are sequences of positive numbers, chosen so that $\Sigma c_{k}\lambda_{k}^{n}<\infty$ for $n=0,$ 1,2,.… and so that $c_{n}$ A, increases very rapidly and is much larger than the sum of all the other terms in the series $\Sigma c_{k}\lambda_{k}^{n}.$ and choose $\{\lambda_{k}\}$ so that For instance, put $c_{k}=\lambda_{k}^{1-k},$ $$ \lambda_{k}>2\sum_{j=1}^{k-1}c_{j}\lambda_{j}^{k}\quad\mathrm{and}\quad\lambda_{k}>k^{2k}. $$ 14 Suppose ${\cal C}\{M_{n}\}$ is quasi-analytic $f\in C\{M_{n}\};$ , and $f(x)=0$ for infinitely many $x\in[0,1]$ 1] What follows? 15 Let $X$ be the vector space of allentire functions f that satisfy $|f(z)|\leq C e^{e^{i|z|}}$ for some $c\varepsilon\omega,$ and whose restriction to the real axis is in $L^{2}.$ Associate with each fe $\scriptstyle{\mathcal{X}}$ its restriction to the integers. Prove that f→{f(n)} is a linear one-to-one mapping of $\scriptstyle{\mathcal{X}}$ onto ${\mathcal{L}}^{2},$ 16 Assume fis a measurable function on $(-\infty,$ oo) such tha $|f(x)|<e^{-|x|}$ for all x. Prove that its Fourier transform fcannot have compact support, unless $f(x)=0.$ a.e.