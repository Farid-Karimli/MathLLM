CHAPTER TWELVE THE MAXIMUM MODULUS PRINCIPLE Introduction 12.1 The maximum modulus theorem (10.24) asserts that the constants are the only homomorphic functions in a region $\Omega$ whose absolute values have a local maximum at any point of $\Omega.$ Here is a restatement: If ${\cal K}\,\,\,\,\,$ is the closure of a bounded region Q,iff is contin- uous on $\textstyle K$ and holomorphic in $\Omega,$ then $$ |f(z)|\leq\|f\|_{\mathrm{on}} $$ (1) For if for every z ∈ S2. If equality holds at one point $z\in\Omega,t h e n f i$ is constant $|f|$ on ${\cal K}\,\,$ (which is [The right side of (1) is the supremum ${\mathfrak{A}}|{\mathfrak{J}}|$ on the boundary aQ of Q.] $\mathbf{\hat{|}}f(z)\mathbf{|}\geq\left\|f\right\|_{\mathbf{b}\mathbf{a}}$ at some $\varepsilon\in\Omega$ then the maximum of attained at some point of $K,$ since $\textstyle K$ is compact) is actually attained at some point of Q, sof is constant, by Theorem 10.24. The equality $\|f\|_{\infty}=\|f^{*}\|_{\infty}.$ , which is part of Theorem 11.32, implies that $$ |f(z)|\leq\|f^{\bullet}\|_{\alpha}\qquad(z\in U,f\in H^{\infty}(U)). $$ (2) This says (roughly speaking) that $\scriptstyle J(z){\big|}\;\psi(z){\big|}\;$ is no larger than the supremum of the boundary values ${\mathfrak{o f}}\,j,$ a statement similar to (1). But this time boundedness on $U$ is enough; we do not need continuity on $\bar{U}$ This chapter contains further generalizations of the maximum modulus theorem, as well as some rather striking applications of ${\mathrm{it}},$ and it concludes with a theorem which shows that the maximum property“almost”characterizes the class of holomorphic functions. 253254 REAL AND coMPLEX ANALYSIs The Schwarz Lemma This is the name usually given to the following theorem. We use the notation established in Sec. 11.31. 12.2 Theorem Suppose f∈ $H^{\infty}.$ , /l ≤ 1, and f(0) = 0.Then $$ |f(z)|\leq|z|\qquad(z\in U), $$ (1) $$ |f^{\prime}(0)|\leq1; $$ (2) if equality holds in(1) for one $z\in U-\{0\},$ or if equalit y holds in (2),then $f(z)=\lambda z,$ where Xis a constant, $\scriptstyle\lambda\lambda=1$ In geometric language, the hypothesis is that f is a holomorphic mapping of ${\boldsymbol{U}}$ into $U$ which keeps the origin fixed; part of the conclusion is that either f is a rotation or f moves each $z\in U-\{0\}$ closer to the origin than it was PR0OF Since $f(0)=0,f(z)/z$ has a removable singularity at $\scriptstyle z=0.$ Hence there exists $g\in H(U)$ such that $f(z)=z g(z),\operatorname{If}z\in U$ and $|z|<r<1,$ then $$ |g(z)|\leq\operatorname*{max}_{\theta}\frac{|f(r e^{i\theta})|}{r}\leq\frac{1}{r}. $$ Letting $r\to1,$ we see that $|g(t)|\leq1$ at every $z\in U.$ This gives(1). Since $f^{\prime}(0)=g(0),$ (2) follows. If $|g(z)|=1$ for some $z\in U,$ then $\scriptstyle{\mathcal{G}}$ is constant, by another application of the maximum modulus theorem ${\it j}/j{\it j}$ Many variants of the Schwarz lemma can be obtained with the aid of the following mappings of $U$ J onto ${\boldsymbol{U}}\,;$ 12.3 Definition For any $\alpha\in U,$ define $$ \varphi_{\alpha}(z)={\frac{z-\alpha}{1-{\bar{\alpha}}z}}. $$ 12.4 Theorem $F i x\ \alpha\in U.$ Then $\varphi_{\alpha}$ is a one-to-one mapping which carries have $T_{\mathbf{\delta}}$ onto T, $U$ onto U, and α to O. The inverse of $\varphi_{\alpha}$ 。is $\varphi_{-x},W e$ $$ \varphi_{x}^{\prime}(0)=1-\left|\alpha\right|^{2},\qquad\varphi_{x}^{\prime}(\alpha)=\frac1{1-\left|\alpha\right|^{2}}. $$ (1) lies outside is holomorphic in the whole plane, except for a pole at 1/α which PR00F $\varphi_{x}$ ${\widetilde{U}}.$ Straightforward substitution shows that $$ \varphi_{-\alpha}(\varphi_{\alpha}(z))=z. $$ (2)THE MAXIMUM MODULUS PRINCIPLE 255 Thus $\varphi_{\alpha}$ is one-to-one, and $\varphi_{-\alpha}$ is its inverse. Since, for real t $$ \left|\frac{e^{i t}-\alpha}{1-\bar{\alpha}e^{i t}}\right|=\frac{|e^{i t}-\alpha|}{|e^{-i t}-\bar{\alpha}|}=1 $$ (3) (z and $\overline{{\mathbb{Z}}}$ have the same absolute value), $\varphi_{\alpha}$ maps ${\mathbf{}}T$ into T; the same is true of ${\it j}/j{\it j}$ that $\varphi_{\alpha}(U)\subset U$ , and consideration of $\phi_{-\alpha}$ It now follows from the maximum modulus theorem $\varphi_{\alpha}(U)=U.$ Q-a; hence $\varphi_{x}(T)=T.$ shows that actually 12.5 An Extremal Problem Suppose α and $\beta$ are complex numbers, $|\alpha|<1,$ and To solve this, put Iβ|<1. How large can lf(α)| be iff is subject to the conditions f∈ $H^{\infty}$ $\|f\|_{\infty}\leq1,$ $a n d f(x)=\beta^{\prime}$ $$ g=\varphi_{\beta}\circ f\circ\varphi_{-\alpha}. $$ (1) Since $\varphi_{-\alpha}$ The passage from map ${\boldsymbol{U}}$ onto ${\boldsymbol{U}},$ we see that $g\in H^{*}$ and $|g|_{\alpha}\leq1;$ ; also, $g(0)=0.$ and $\varphi_{\beta}$ to $\mathbf{\Omega}^{g}$ has reduced our problem to the Schwarz lemma, which gives $\boldsymbol{\mathsf{f}}$ the chain rule gives $|g^{\prime}(0)|\leq1.89\;(1),$ $$ {\mathcal{G}}^{\prime}(0)=\varphi_{\beta}^{\prime}(\beta)f^{\prime}(x)\varphi_{-\alpha}^{\prime}(0). $$ (2)) If we use Eqs. 12.41), we obtain the inequality $$ |f^{\prime}(x)|\leq{\frac{1-|\beta|^{2}}{1-|\alpha|^{2}}}. $$ (3) This solves our problem, since equality can occur in (3). This happens if and only if $|g(0)|=1,$ in which case $\scriptstyle{\mathcal{G}}$ is a rotation (Theorem 12.2), so tha $$ f(z)=\varphi_{-\beta}(\lambda\varphi_{a}(z))\qquad(z\in U) $$ (4) for some constant A with $|\lambda|=1.$ A remarkable feature of the solution should be stressed. We imposed no smoothness conditions (such as continuity on L ${\bar{U}},$ 厂 for instance) on the behavior of $\boldsymbol{\f}$ near the boundary of ${\boldsymbol{U}}.$ Nevertheless, it turns out that the functions $\boldsymbol{\mathsf{f}}$ which maximize $|f^{\prime}(\alpha)|$ under the stated restrictions are actually rational functions. (not just into) and that Note also that these extremal functions map $U$ onto ${\boldsymbol{U}}$ they are one-to-one. This observation may serve as the motivation for the proof of the Riemann mapping theorem in Chap. 14. At present, we shall merely show how this extremal problem can be used to characterize the one-to-one holomorphic mappings of ${\boldsymbol{U}}$ J onto $U.$ 12.6 Theorem Suppose fe H(U), fis one-to-one $f(U)=U,\ \alpha\in U,$ and $f(x)=0.$ Then there is a constant 入,|入|= 1, such that $$ f(z)=\lambda\varphi_{\alpha}(z)\qquad(z\in U). $$ (1) In other words, we obtain f by composing the mapping $\varphi_{\alpha}$ with a rotation.256 REAL AND cOMPLEX ANALYSIs rule, PROOF Let $\scriptstyle{\mathcal{G}}$ be the inverse of f, defined by $g(f(z))=z,\,z\in U.$ Since fis one- to-one, $f^{\prime}$ has no zero in ${\boldsymbol{U}},$ so $g\in H(U),$ by Theorem 10.33. By the chain $$ g(0)f^{\prime}(x)=1. $$ (2) The solution of 12.5, applied to f and to ${\mathfrak{g}},$ yields the inequalities $$ |f^{\prime}(x)|\leq{\frac{1}{1-|\alpha|^{2}}},\qquad|g^{\prime}(0)|\leq1-|\alpha|^{2}. $$ (3) By (2),equality must hold in (3).As we observed in the preceding problem (with ${\boldsymbol{\beta}}=0),$ this forces f to satisfy (1) // The Phragmen-Lindelf Method 12.7 For a bounded region $\Omega,$ we saw in Sec.12.1 that if f is continuous on the closure of Q and $\operatorname{iff}\in H(\Omega),$ the maximum modulus theorem implies $$ \|f\|_{\Omega}=\|f\|_{\otimes\Omega}. $$ (1) For unbounded regions, this is no longer true. To see an example, let $$ \Omega=\left\{z=x+i y\colon-{\frac{\pi}{2}}<y<{\frac{\pi}{2}}\right\}; $$ (2) $\Omega$ is the open strip bounded by the parallel lines $y=\pm\pi/2;$ its boundary aQ is the union of these two lines. Put $$ f(z)=\exp\,(\exp\,(z)). $$ (3) For real x $$ f{\biggl(}x\pm{\frac{\pi i}{2}}{\biggr)}=\exp\left(\pm i e^{x}\right) $$ (4) since exp $(\pi i/2)=i,$ so $|f(z)|=1$ for z ∈ OQ. But f(z)→OO very rapidly as $x\to\infty$ along the positive real axis, which lies in QL SVery”is the key word in the preceding sentence. A method developed by Phragmen and Lindel6f makes it possible to prove theorems of the following kind:If $f\in H(\Omega)$ and if $|f|<g,$ where $\theta(z) arrow\infty$ “ slowly”as $z\to\infty$ in Q(just what“slowly”means depends on Q2), then $\boldsymbol{\mathsf{f}}$ is actually bounded in $\Omega,$ 2,and this usually implies further conclusions about $f,$ by the maximum modulus theorem Rather than describe the method by a theorem which would cover a large number of situations, we shall show how it works in two cases. In both, Q2 will be a strip. In the first, $\boldsymbol{\mathsf{f}}$ will be assumed to be bounded, and the theorem wil improve the bound; in the second, a growth condition will be imposed on f which just excludes the function (3). In view of later applications, $\Omega$ will be a vertica strip in Theorem 12.8.THE MAXIMUM MODULUS PRINCIPLE 257 First, however, let us mention another example which also has this general flavor: Suppose f is an entire function and $$ |f(z)|<1+|z|^{1/2} $$ (5) for all z. Then f is constant. that f("(0) = 0 for $n=1,$ 2,3 …… This follows immediately from the Cauchy estimates 10.26, since they show 12.8 Theorem Suppose $$ \Omega=\{x+i y\colon a<x<b\},\,\,\,\,\,\,\,\,\,\,\bar{\Omega}=\{x+i y\colon a\leq x\leq b\}, $$ (1) $\boldsymbol{\mathsf{f}}$ is continuous on $\Omega,f\in H(\Omega),$ and suppose that $|f(z)|<B$ for al $\varepsilon\in\Omega$ and some fixed $B<\infty.1f$ $$ M(x)=\operatorname*{sup}\;\{\,|f(x+i y)|:-\;\infty\,<y<\infty\}\qquad(a\leq x\leq b) $$ (2) then we actually have $$ M(x)^{b-a}\leq M(a)^{b-x}M(b)^{x-a}\qquad(a<x<b). $$ (3) on the boundary of $\Omega.$ Note: The conclusion (3) implies that the inequality is no larger in $\Omega$ than the supremum of $|f|$ $|f|\leq\operatorname*{max}\left(M(a),\,M(t)\right.\leq$ b) so that $|f|<B$ can be replaced by $|f|$ If we apply the theorem to strips bounded by lines $x=\alpha$ and $x=\beta,$ where $a\leq\alpha<\beta\leq b,$ the conclusion can be stated in the following way: Corollary Under the hypotheses of the theorem, log $\textstyle{M}$ is a convex function on (a,b). that For each PRoOF We assume first that $M(a)=M(b)=1.$ In this case we have to prove $|f(z)|\leq1$ for all $\varepsilon\in\Omega$ $\scriptstyle\epsilon>0,$ we define an auxiliary function $$ h_{\epsilon}(z)=\frac{1}{1+\epsilon(z-a)}\qquad(z\in\bar{\Omega}). $$ (4) Since Re $\{1+\epsilon(z-a)\}=1+\epsilon(x-a)\geq1$ in ${\bar{\Omega}},$ we have $|h_{i}|\leq1$ in ${\bar{\Omega}},$ so that $$ |f(z)h_{\epsilon}(z)|\leq1\qquad(z\in\partial\Omega). $$ (5) AIso, $|\,1\,+\,\epsilon(z-a)\,|\,\geq\,\epsilon\,|\,y\,|\,.$ ,so that $$ |f(z)h_{\epsilon}(z)|\leq{\frac{B}{\epsilon\,|y|}}\qquad(z=x+i y\in\Omega). $$ (6) (6) Let ${\boldsymbol{R}}$ be the rectangle cut off from $\bar{\Omega}$ by the lines $y=\pm B/\epsilon.$ By (S) and $|f h_{e}|\leq1$ on OR, hence $|f h_{c}|\leq1$ on R,by the maximum modulus $|f(z)h_{\varepsilon}(z)|\leq1$ theorem. But (6) shows that $|f h_{c}|\leq1$ on the rest of Q.Thus258 REAL AND cOMPLEX ANALYSIS for all $\varepsilon\in\Omega$ and all $\scriptstyle\epsilon\;>0$ If we fix z ∈ Q and then let $\epsilon\to0,$ we obtain the desired result $|f(z)|\leq1.$ We now turn to the general case. Put $$ g(z)=M(a)^{(b-z)/(b-a)}M(b)^{(z-a)/(b-a)}, $$ (7) where, for $M>0$ and $w_{\nu}$ complex, $M^{\nu}$ is defined by $$ M^{w}=\exp\,(w\,\log\,M), $$ (8)) and log ${\cal{M}}$ is real. Then g $\scriptstyle{\mathcal{G}}$ g is entire, $\mathbf{\Omega}^{g}$ has no zero $1/g$ is bounded in ${\bar{\Omega}},$ $$ |g(a+i y)|=M(a),\qquad|g(b+i y)|=M(b), $$ (9) and hence $f/g$ satisfies our previous assumptions. Thus $|f g|\leq1$ in $\Omega,$ and this gives (3).(See Exercise 7.) // 12.9 Theorem Suppose $$ \Omega=\left\{x+i y\colon|y|<\frac{\pi}{2}\right\},\qquad\bar{\Omega}=\left\{x+i y\colon|y|\leq\frac{\pi}{2}\right\}. $$ (1) that Suppose fis continuous on $\Omega,f\in H(\Omega),$ there are constants α $<1,\,4<\infty,$ such $$ |f(z)|<\exp\left\{4\not\exp (\alpha\,|\,x\,|\,\right)\rangle\qquad(z=x+i y\in\Omega), $$ (2) and $$ \left|f\Bigl(x\pm{\frac{\pi i}{2}}\Bigr)\right|\leq1\qquad(-\infty<x<\infty). $$ (3) Then lf(z)|≤1 for all z ∈ Q Note that the conclusion does not follow i $\scriptstyle x\;=\;1.$ as is shown by the function exp (exp z). PRoOF Choose $\beta>0$ so that o $c<\beta<1.$ Foré> 0, define $$ h_{\varepsilon}(z)=\exp\ \{-\epsilon(e^{\beta z}+e^{-\beta z})\}. $$ (4) For zeQ, $$ \mathrm{Re}\,\left[e^{\beta z}+e^{-\beta z}\right]=(e^{\beta x}+e^{-\beta x})\,\cos\,\beta y\geq\delta(e^{\beta x}+e^{-\beta x}) $$ (5) where 6= cos (βz/2) > 0, since $|\beta|<1.$ Hence $$ |\,h_{\epsilon}(z)\,|\leq\exp\,\{-\epsilon\delta(e^{\beta x}+e^{-\beta x})\}<1\qquad(z\in\bar{\Omega}). $$ (6) It follows that|fhal≤1on OQ and that $$ |f(z)h_{e}(z)|\leq\exp\left\{A e^{\alpha|x|}-\epsilon\delta(e^{\beta x}+e^{-\beta x})\right\}\qquad(z\in\tilde{\Omega}). $$ (7)THE MAXIMUM MODULUS PRINCIPLE 259 Fix $\scriptstyle x\,>0$ Since $\epsilon\delta>0$ and $\delta>x.$ the exponent in(T) tends to 一OO as for all $x>x_{0}.$ Since Hence there exists an $x_{0}$ so that the right side of(T) is less than 1 $x\to\pm\infty.$ $|f h_{c}|\leq1$ on the boundary of the rectangle whose ver- tices ar ${\mathfrak{C}}\ \pm x_{0}\pm(\pi i/2).$ the maximum modulus theorem shows that actually for every // $|f h_{c}|\leq1$ on this rectangle. Thus $|f h_{c}|\leq1$ at every point of $\Omega,$ for all $\scriptstyle\epsilon\;>\;0$ As e→0, $h_{\varepsilon}(z)\to1$ for every z, so we conclude that $|f(z)|\leq1$ z ∈ Q2. Here is a slightly different application of the same method. It will be used in the proof of Theorem 14.18. 12.10 Lindel6f 's Theorem Suppose $\Gamma$ is a curve, with parameter interval [0, 1], such that $|\Gamma(t)|<1~i f t<1$ and $\Gamma(1)=1.1f g\in H^{\infty}$ and $$ \operatorname*{lim}_{t arrow1}g(\Gamma(t))=L, $$ (1) limit ${\boldsymbol{L}}$ at 1.) has radial limit ${\boldsymbol{L}}$ at 1 actually has nontangential then $\scriptstyle{\mathcal{G}}$ $\scriptstyle{\mathcal{G}}$ (It follows from Exercise 14, Chap.14, that PROOF Assume $|g|<1,\,L=0,$ without loss of generality. Let we have be given. $\scriptstyle\epsilon\;>0$ There exists $t_{0}<1$ so that, setting $r_{0}=\mathrm{Re}\;\Gamma(t_{0}),$ $$ |g(\Gamma(t))|<\epsilon\mathrm{\boldmath~\and~\partial~Re~}\Gamma(t)>r_{0}>\frac{1}{2} $$ (2) as soon as $t_{0}<t<1.$ Pick $r,r_{0}<r<1.$ by Define F $\boldsymbol{h}$ in $\Omega=D(0;1)\cap D(2r;1)$ $$ h(z)=g(z)\overline{{{g(\bar{z})}}}g(2r-z)\overline{{{g(2r-\bar{z})}}}. $$ (3) Then $h\in H(\Omega){\mathrm{~and~}}|\,h|<1.$ We claim that $$ |h(r)|<\epsilon. $$ (4) Since $h(r)=|g(r)|^{4},$ the theorem follows from (4) 1]), where $t_{\mathrm{1}}$ is the largest ${\mathbf{}}t$ for which Re To prove(4)。let $E_{1}=\Gamma([t_{1},$ in the real axis, and let $\boldsymbol{E}$ be the that $\Gamma(t)=r,$ let $E_{2}$ be the reflection of $E_{1}$ $x=r.$ Then (2) and (3) imply union of $E_{1}\cup E_{2}$ and its reflection in the line $$ |h(z)|<\epsilon\quad\mathrm{if}\quad z\in\Omega\cap E. $$ (5) Pick $\epsilon>_{0},$ define $$ h_{c}(z)=h(z)(1-z)^{c}(2r-1-z)^{c} $$ (6)260 REAL AND COMPLEX ANALYSIs on the boundary of $K.$ $h_{c}(1)=h_{c}(2r-1)=0.$ If ${\cal K}\,\,\,\,\,\,$ is the union of $\boldsymbol{E}$ and the the for z ∈ Q2, and put tinuous on bounded components of the complement of ${\boldsymbol{E}},$ then $\boldsymbol{K}$ is compact, $h_{\mathrm{{c}}}$ is con- $K_{\mathrm{\scriptscriptstyleJ}}$ holomorphic in the interior of $K,$ and $({\mathsf{5}})$ implies that $|h_{\epsilon}|<\epsilon$ Since the construction of $\boldsymbol{E}$ shows that $\scriptstyle\epsilon\in K.$ Letting maximum modulus theorem implies that $|h_{\epsilon}(r)|<\epsilon$ // $\scriptstyle c\to0.$ we obtain (4) An Interpolation Theorem 12.11 The convexity theorem 12.8 can sometimes be used to prove that certain linear transformations are bounded with respect to certain $L^{p}.$ -norms. Rather than discuss this in full generality, let us look at a particular situation of this kind Let $X$ be a measure space, with a positive measure ${\boldsymbol{\mu}}_{s}$ and suppose $\{\psi_{n}\}$ $(n=1,\,2,\,3,\,\ldots)$ is an orthonormal set of functions in $\scriptstyle T(n)$ ; we recall what this means: $$ \bigcap_{x}\psi_{n}{\bar{\psi}}_{m}\,d\mu= \{_{0}^{1}\qquad{\mathrm{if}}\ m=n, $$ (1) $M<\infty$ Let us also assume that $\{\psi_{n}\}$ is a bounded sequence in $L^{\infty}(\mu).$ : There exists an such that $$ \vert\psi_{n}(x)\vert\leq M\qquad(n=1,\,2,\,3,\,\ldots\,;\,x\in X) $$ (2) Then for any fe $\scriptstyle{m_{\mathrm{th}}}$ where $1\leq p\leq2,$ the integrals $$ \hat{f}(n)= |_{X}^{*}f\bar{\psi}_{n}\,d\mu\qquad(n=1,\,2,\,3,\,...) $$ (3) exist and define a function fon the set of all positive integers. There are now two very easy theorems: ${\mathrm{For}}f\in L^{1}(\mu),(2)$ gives $$ \|{\hat{f}}\|_{\infty}\leq M\|f\|_{1}, $$ (4) and $\mathrm{for}f\in L^{2}(\mu),$ the Bessel inequality gives $$ \|{\hat{f}}\|_{2}\leq\|f\|_{2}\,, $$ (5) where the norms are defined as usual: $$ \|f\|_{p}={\biggl[}\bigcap\vert f\mid^{p}d\mu\bigcap^{-1/p},\qquad\vert l\rangle\|_{q}=[\sum\vert{\hat{f}}(n)\vert^{q}]^{1/q}, $$ (6) and $\|{\hat{f}}\|_{\alpha}=\operatorname*{sup}_{n}\|{\hat{f}}(n)\|.$ Since(1, o) and (2,2) are pairs of conjugate exponents, one may conjecture that $\|{\hat{f}}\|_{q}$ is finite whenever $f\in D(\mu)$ and $1<p<2,\,q=p/(p-1).$ This is indeed true and can be proved by“interpolation”between the preceding trivial cases $p=1$ and $p=2.$THE MAXIMUM MODULUS PRINCIPLE 261 12.12 The Hausdorff-Young Theorem Under the above assumptions,the inequality $$ \|{\hat{f}}\|_{q}\leq M^{(2-p)/p}\|f\|_{p} $$ (1) holds i $1\leq p\leq2$ and iff∈ E(u) Fix $p,\;1<p<2$ Let PRooF We first prove a reduced form of the theorem be a simple complex function such that $\textstyle\sum|b_{n}|^{p}=1.$ Our objective and let $\boldsymbol{\mathit{f}}$ $\|f\|_{p}=1,$ $b_{1},\ldots,b_{N}$ be complex numbers such that is the inequality $$ \left|\sum_{n=1}^{N}b_{n}{\hat{f}}(n)\right|\leq M^{(2-p)/p}. $$ (2) Put $F=\mid f\mid^{p},$ and put $B_{n}=\mid b_{n}\mid^{p}$ Then there is a function $\varphi$ and there are complex numbers $\beta_{1},\ldots,\beta_{N}$ such that $$ f=F^{1/p}\varphi,\qquad|\varphi|=1,\qquad\int_{X}F\,d\mu=1, $$ (3) and $$ b_{n}=B_{n}^{1/p}\beta_{n},\qquad|\beta_{n}|=1,\qquad\sum_{n=1}^{N}B_{n}=1. $$ (4) If we use these relations and the definition off(n) given in Sec. 12.11, we obtain $$ \sum_{n=1}^{N}\,b_{n}\hat{f}(n)=\sum_{n=1}^{N}\,B_{n}^{1/p}\beta_{n}\,\Bigg|^{\,*}F^{1/p}\varphi\bar{\psi}_{n}\,d\mu. $$ (5) Now replace $1/p\;\mathrm{by}\;z\;\mathrm{in}\left(5\right),$ and define $$ \Phi(z)=\sum_{n=1}^{N}\,B_{n}^{z}\,\beta_{n}\,\Bigg|_{X}^{*}F^{z}\varphi\bar{\psi}_{n}\,d\mu $$ (6) we agree that $\scriptstyle A^{n}\;=\;0$ Since ${\mathbf{}}F$ Recall that $A^{z}=\exp{(z)}$ log A) i ${\underset{4>0;}{\sim}}$ if $A=0,$ for any complex number $\mathbb{Z}.$ is simple, since $F\geq0,$ and since $s_{n}\geq0,$ we see that p is a finite linear combination of such exponentials,so $\Phi$ is an entire function which is bounded on $$ \{z\colon a\leq\operatorname{Re}(z)\leq b\} $$ For for any finite a and b. We shall take define and $b=1.$ shall estimate $\Phi$ on the $\mathbf{\omega}_{n}={\frac{1}{2}}$ $\Phi(1/p).$ edges of this strip, and shall then apply Theorem 12.8 to estimate $-\infty<y<\alpha,$ $$ c_{n}(y)=\int_{X}F^{1/2}F^{i j}\varphi\bar{\psi}_{n}\,d\mu. $$ (7262 REAL AND coMPLEX ANALYSIS The Bessel inequality gives $$ \sum_{n=1}^{N}\,|\,c_{n}(y)\,|^{2}\leq\left\{{}_{X}|F^{1/2}F^{i,\nu}\varphi\,|^{2}\,\,d\mu= .\right\}_{X}^{}|F\,|\,\,d\mu=1, $$ (8) and then the Schwarz inequality shows that $$ |\Phi_{({\tt2}}^{(1}+i y)|=\left|\sum_{n=1}^{N}B_{n}^{1/2}B_{n}^{i\nu}\beta_{n}c_{n}\right|\leq\left\{\sum_{n=1}^{N}B_{n}\cdot\sum_{n=1}^{N}|c_{n}|^{2}\right\}^{1/2}\leq1. $$ (9) The estimate $$ |\Phi(1+i y)|\leq M\qquad(-\infty<y<\infty. $$ (10) follows trivially from (),(4), and (6), since $\|\psi_{n}\|_{\infty}\leq M.$ We now conclude from (9), (10), and Theorem 12.8 that $$ \vert\Phi(x+i y)\vert\leq M^{2x-1}\qquad(\frac{4}{2}\leq x\leq1,\,-\infty<y<\infty). $$ (11) With $x=1/p{\mathrm{~and~}}y=0,$ this gives the desired inequality (2) The proof is now easily completed. Note first that $$ \left\{\sum_{n=1}^{N}\,|\,\hat{f}(n)|^{q}\right\}^{1/q}=\mathrm{sup}\left|\sum_{n=1}^{N}\,b_{n}\,\hat{f}(n)\right|, $$ (12) the supremum being taken over all $\{b_{1},\ldots,b_{N}\}$ with $\textstyle\sum|b_{n}|^{p}=1,$ since the ${\cal L}^{q}$ norm of any function on any measure space is equal to its norm as a linear functional on ${\boldsymbol{D}}.$ Hence (2) shows that $$ \left\{\sum_{n=1}^{N}|\,{\hat{J}}(n)|^{q}\right\}^{1/q}\leq M^{(2-p)/p}\|f\|_{p} $$ (13) $j\to\operatorname{co}.$ Then for every simple complex f ∈ $\scriptstyle H_{\mathrm{(g)}}$ because $\psi_{n}\in L^{q}(\mu).$ Thus since (13) holds as ${\hat{J}}_{j}(n){ arrow}{\hat{J}}(n)$ If now fe LE(u), there are simple functions ${\mathfrak{n}},$ ${\mathbf{}}N$ was arbitrary, we finally obtain (1) /// $\|f_{j}-f\|_{i<j\cdot i}\to0$ for every $f_{j}$ ; such that for each f, it also holds for f. Since A Converse of the Maximum Modulus Theorem We now come to the theorem which was alluded to in the introduction of the present chapter The letter j will denote the identity function:j(z) = 2. The function which assigns the number l to each z ∈ 厂 $\bar{U}$ will be denoted by 1. 12.13 Theorem Suppose ${\cal{M}}$ is a vector space of continuous complex functions on the closed unit disc ${\bar{U}},$ with the following properties. (a)1∈ M. (b)Iff∈ M, then also if e M. (c)Iff e M, then $\|f\|_{U}=\|f\|_{T}.$ Then every fe M is holomorphic in U.THE MAXIMUM MODULUS PRINCIPLE 263 Note that (c) is a rather weak form of the maximum modulus principle;(c) asserts only that the overall maximum of|fl on $\bar{U}$ is attained at some point of the boundary ${\boldsymbol{T}},$ but (c) does not a priori exclude the existence of local maxima of 1flin ${\boldsymbol{U}}.$ PROOF By (a) and (b), $\textstyle{M}$ contains all polynomials. In conjunction with (c), this shows that A $\textstyle{M}$ satisfies the hypotheses of Theorem 5.25. Thus every fe M is harmonic in ${\boldsymbol{U}}.$ We shall use (b) to show that every fe $\textstyle{M}$ actually Satisfie the Cauchy-Riemann equation Let $\scriptstyle{\hat{O}}$ and $\tilde{\partial}$ be the differential operators introduced in Sec. 11.1. The product rule for differentiation gives $$ (\bar{\partial})(f g)=f\cdot(\bar{\partial}g)+(\bar{\partial}f)\cdot(\bar{\partial}g)+(\bar{\partial}f)\cdot(\bar{\partial}g)+(\bar{\partial}\bar{\partial}f)\cdot g. $$ F $\pm\,f\in M,$ and take $\scriptstyle g=j,$ Then $\hat{\mathcal{J}}\in M.$ Hence $\boldsymbol{\f}$ and $\hat{\mathcal{I}}$ are harmonic, so // ${\partial}{\bar{\partial}}f=0$ and $(\bar{\omega}\phi|(\beta)=0$ Also, ${\bar{\partial}}j=0$ and ${\partial}j=1.$ The above identity therefore reduces to ${\bar{\partial}}f=0.$ Thus fe $H(U)$ This result will be used in the following proof. 12.14 Rad6's Theorem Assume f∈ C(U), $\Omega$ 2 is the set of $a l l z\in U$ at which f(z)≠ 0, and fis holomorphic in 2. Then fis holomorphic in ${\cal U}.$ In particular, the theorem asserts that $v-\alpha$ is at most countable, unless $\Omega=\mathcal{D}.$ PROOF Assume hence $h(z)=0.\mathrm{If}\,z\in T\subset1$ aQ2, then is dense in U. If not, there so that exist αe QL and $\Omega\neq{\mathcal{O}}.$ We shall first prove that $\Omega$ Choose $\scriptstyle n$ , then $\beta\in U-\Omega$ such that $2\vert\,\beta-\alpha\vert<1-{\frac{1}{2}}\,\beta\vert$ $2^{n}|f(\alpha)|>\|f\|_{T}.$ Define $h(z)=(z-\beta)^{-n}f(z),$ for z ∈ Q. f $z\in U\cap\partial\Omega$ $f(z)=0,$ $$ |\,h(z)|\leq(1-|\,\beta\,|)^{-n}||f\,||_{T}<|\,\alpha-\beta\,|^{-n}|f(\alpha)|=|\,h(\alpha)| $$ This contradicts the maximum modulus theorem Thus $\Omega$ is dense in ${\boldsymbol{U}}.$ Next, let ${\cal M}$ be the vector space of all $g\in C({\mathcal{U}})$ that are holomorphic in Q. Fix $g\in M.$ For $\mathbf{\partial}\cdot\mathbf{\partial}n=1,\ \mathbf{\partial}2,\ \ 3,\ \ldots,\ f g^{n}=0$ on $U\cap\tilde{c}\Omega.$ The maximum modulus theorem implies therefore, for every $\mathrm{*e}\,\Omega.$ that $$ |f(\alpha)|\ |g(\alpha)|^{n}\leq\|f g^{n}\|_{\delta\Omega}=\|f g^{n}\|_{T}\leq\|f\|_{T}\|g\|_{T}^{n} $$ If we take nth roots and then let $\textstyle n\!\to$ 00, we see th $\operatorname{at}|g(x)|\leq\|g\|_{T},$ for every $\scriptstyle x\in\Omega$ Since $\Omega$ is dense in ${\boldsymbol{U}},$ $\|g\|_{U}=\|g\|_{T}.$ $f\in M,$ It follows that $\textstyle{M}$ satisfies the hypotheses of Theorem 12.13. Since fis holomorphic in ${\boldsymbol{U}}.$ //264 REAL AND CoMPLEX ANALYSIs Exercises 1 Suppose △ is a closed equilateral triangle in the plane,、with verticesα、b,c.Find max $(|z-a|\left|z-b\right|\left|z-c\right|)$ as z ranges over $\Delta.$ 2 Suppose $f\in H(\Pi^{+}),$ where $\Pi^{*}$ is the upper half plane, and $|f|\leq1.$ How large can $|f^{\prime}(i)|$ be? Find the extremal functions. (Compare the discussion in Sec. 12.5.) 3 Suppose fe H(Q). Under what conditions can lflhave a local minimum in $\Omega{\dot{r}}$ 4(a) Suppose $\underline{{\Omega}}$ is a region, ${\boldsymbol{D}}$ is a disc, ${\bar{D}}\subset\Omega,f\in H(\Omega),f{\mathrm{~is~}}$ ${\boldsymbol{D}}.$ is constant on the boundary of ${\boldsymbol{D}}.$ not constant, and $|f|$ Prove that fhas at least one zero in (b) Find all entire functions f such th ${\mathfrak{a t}}\,|f(z)|=1$ whenever $|z|=1.$ morphic in $\Omega,$ and is a bounded region, $\{f_{n}\}$ is a sequence of continuous functions on $\bar{\Omega}$ which are holo S Suppose $\underline{{\Omega}}$ formly on ${\bar{\Omega}}.$ $\{f_{n}\}$ converges uniformly on the boundary of Q. Prove that $\{f_{n}\}$ converges uni- and ind- 6 Suppose fe H(Q), F is a cycle in $\underline{{\Omega}}$ such that Ind, $(\alpha)=0$ for allzc Q, If(OI ≤S1 for every $\zeta\in\Gamma^{*},$ $(z)\neq0.$ Prove that $|f(z)|\leq1.$ and $M(b)>0.$ Show that the theorem is true if $M(a)=0,$ 7 In the proof of Theorem 12.8 it was tacitly assumed that $M(a)>0$ and that then f(z) = 0 for al $z\in\Omega.$ 8 I $0<R_{1}<R_{2}<\infty,\operatorname{let}A(R_{1},R_{2})$ )denote the annulus $$ \{z\colon R_{1}<|z|<R_{2}\}. $$ There is a vertical strip which the exponential function maps onto $^{\alpha_{B_{v}}}$ $R_{2}).$ Use this to prove Hadamard's three-circle theorem: Iff $\in H(A(R_{1},R_{2})),$ if $$ M(r)=\operatorname*{max}_{\theta}|f(r e^{i\theta})|\quad\quad(R_{1}<r<R_{2}), $$ and if $R_{1}<a<r<b<R_{2},$ then $$ \log\,M(r)\leq{\frac{\log\,{(b/r)}}{\log\,{(b/a)}}}\log\,M(a)+{\frac{\log\,{(r/a)}}{\log\,{(b/a)}}}\log\,M(b). $$ [In other words, log $M(r)$ is a convex function of log $r_{\mathrm{\scriptsize~z}}$ For which f does equality hold in this inequality ? closure of TI (Re 9 Let Il be the open right half plane $(z\in\Pi$ if and only i $\scriptstyle\mathbf{Re}\,t>0)$ Suppose fis continuous on the $z\geq0),f\in H(\Pi),$ and there are constants $A<\alpha_{0}$ and $\scriptstyle{x\cdot{1}}$ such that $$ |f(z)|<A\ \exp\left(|z|^{x}\right) $$ for al $z\in\Pi.$ Furthermore, $|f(i y)|\leq1$ for all real y. Prove that |f(z)|≤lin I. Show that the conclusion is false for $\scriptstyle{x=1}$ How does the result have to be modified f $\Pi$ is replaced by a region bounded by two rays through the origin, at an angle not equal to r? 10 Let Il be the open right half plane. Suppose that fe H(II), that |f(2)|<1 for al $z\in\Pi,$ and that there exists $\alpha,-\pi/2<\alpha<\pi/2,$ such that $$ {\frac{\log\vert f(r e^{i x})\vert}{r}} arrow-\infty\quad{\mathrm{as}}\quad r arrow\infty. $$ Prove that f = 0 Hint: Put g,(z) =/(z)e",,n = 1, 2,3,.… Apply Exercise 9 to the two angular regions defined by all n $-\pi/2<\theta<\alpha,\,\alpha<\theta<\pi/2.$ Conclude that each ${\mathfrak{g}}_{n}$ is bounded in II, and hence that $|g_{n}|<1$ in I, for $\mathrm{i}{\mathrm{I}}$ Suppose T is the boundary of an unbounded region such that $|f|\leq M$ on ${\Gamma}$ and $|f|\leq B$ in $\Omega.$ Prove that we and there are constants $B<\cdots$ and $\Omega,f\in H(\Omega),f$ is continuous on $\Omega\cup\Gamma,$ then actually have $|f|\leq M$ in Q. $M<\infty$THE MAXIMUM MODULUS PRINCIPLE 265 let n be a large integer, let Suggestion: Show that it involves no loss of generality to assume that $U\,\cap\,\Omega=\mathcal{D}$ Fix $z_{0}\in\Omega,$ ${\mathbf{}}V$ be a large disc with center at O, and apply the maximum modulus theorem to the function f"(z)/z in the component of $V\cap\Omega$ which contains $\mathbf{z_{0}}$ $\mathbf{1}\mathbf{2}$ Let f be an entire function. If there is a continuous mapping $\gamma$ of [O, ${\mathfrak{I}}\}$ into the complex plane such that $y(t)\to\infty$ and f(r()→α as $t\to1,$ we say that a is an asymptotic value of f. [In the complex plane, $|\gamma(t)|>R$ if ${}^{\circ}\gamma(t)\to\infty$ as $t\to1^{\circ}$ means that to each $\,\kappa\,{\stackrel{\exp}{\operatorname{a}}}$ there corresponds ${\underline{{\mathbf{F}}}}_{\mathrm{{}}}$ is unbounded (proof?) and contains $t_{R}<t<1.\}$ Suggestion: Let Each component of ${\textbf{a}}t_{R}<1$ such that Prove that every nonconstant entire function has co as an asymptotic value. a component of $E_{n+1},\mathrm{by}$ $E_{n}=\{z:|f(z)|>n\}.$ Exercise 11 13 Show that exp has exactly two asymptotic values: $\mathbf{\sigma}_{0}$ and $\infty.$ How about sin and cos? Note: sin and cos z are defined,forall complex $\mathbb{Z}_{3}$ by $$ \sin z={\frac{e^{i z}-e^{-i z}}{2i}},\qquad\cos z={\frac{e^{i x}+e^{-i z}}{2}}. $$ 14 If f is entire and if a is not in the range of f, prove that $\scriptstyle{\vec{\mathbf{x}}}$ is an asymptotic value of funless f is constant. 15 Suppose $\textstyle f\in H(U)$ Prove that there is a sequence $\left\{z_{n}\right\}$ in $U_{\mathbf{\delta}}$ such that $|z_{n}|\to1$ and $\{f(z_{n})\}$ is bounded 16 Suppose $\scriptstyle\Omega$ is a bounded region, f∈ $\scriptstyle n_{\mathrm{th}}$ and $$ \operatorname*{lim}_{n\to\infty}\operatorname*{sup}_{n\to\infty}|f(z_{n})|\leq M $$ for every sequence $\left\{z_{n}\right\}$ in $\underline{{\Omega}}$ h which converges to a boundary point of Q. Prove that $|f(z)|\leq M$ for al $z\in\Omega.$ 17 Let $\Phi$ be the set of all f e $H(U)$ such that O <|J(z)|<1 for z e U, and let $\Phi_{c}$ be the set of all f ∈ 中 that have $f(0)=c.$ Define $$ M(c)=\operatorname*{sup}\;\{\;|f^{\prime}(0)|:f\in\Phi_{c}\},\qquad M=\operatorname*{sup}\;\{\;|f^{\prime}(0)|:f\in\Phi $$ D}. Find ${\cal{M}},$ and $M(c)\mathrm{~for~}0<c<1.$ Find anfe O with f"(0) = M, or prove that there is no such f Suggestion: log f maps $\boldsymbol{\mathit{U}}$ into the left half plane. Compose log f with a properly chosen map that takes this half plane to ${\boldsymbol{U}}.$ Apply the Schwarz lemma.