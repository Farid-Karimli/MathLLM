CHAPTER EIGHTEEN ELEMENTARY THEORYOF BANACH ALGEBRAS Introduction 18.1 Definitions A complex algebra is a vector space A over the complex field in which an associative and distributive multiplication is defined, i.e., $$ {\mathfrak{x}}(y z)=(x y)z,\qquad(x+y)z=x z+y z,\qquad x(y+z)=x y+x z $$ (1) for $x,y$ y, and ze A, and which is related to scalar multiplication so that $$ x(x y)=x(\alpha y)=(\alpha x)y $$ (2) for $\scriptstyle{\mathcal{X}}$ and $y\in A,\alpha$ a scalar. which makes $A\!\!\!/$ into a normed linear space If there is a norm defined in $\scriptstyle A$ and which satisfies the multiplicative inequality $$ \|x y\|\leq\|x\|\ \|y\|\qquad(x{\mathrm{~and~}}y\in A), $$ (3) then ${\boldsymbol{A}}$ is a normed complex algebra.If, in addition, A is a complete metric space relative to this norm, ie., if A is a Banach space, then we call A a Banach algebra. means that if $x_{n}\to x$ and $y_{n}\to y,$ The inequality (3) makes multiplication a continuous operation. This which follows from (3) and then $x_{n}y_{n} arrow x y_{;}$ the identity $$ x_{n}y_{n}-x y=(x_{n}-x)y_{n}+x(y_{n}-y). $$ (4) Note that we have not required that $\scriptstyle A$ be commutative, i.e., that $x y=y x$ for all x and y e A, and we shall not do so except when explicitly stated 356ELEMENTARY THEORY OF BANACH ALGEBRAs 357 However, we shall assume that $\scriptstyle A$ A has a unit. This is an element $\scriptstyle{\mathcal{C}}$ such that $$ x e=e x=x\qquad(x\in A). $$ (5) It is easily seen that there is at most one such e $(e^{\prime}=e^{\prime}e=e)$ and that $|e|\geq1,$ by (3). We shall make the additional assumption that $$ |e|=1. $$ (6) An element $x\in A$ will be called invertible if $\scriptstyle{\mathcal{X}}$ has an inverse in A, i.e., if there exists an element $x^{-1}\in,$ A such that $$ x^{-1}x=x x^{-1}=e. $$ (7 Again, it is easily seen that no $x\in A$ has more than one inverse If x and $\scriptstyle{\mathcal{I}}$ are invertible in $A,$ so are $x^{-1}$ and xy, since $(x y)^{-1}=y^{-1}x^{-1}.$ The invertible elements therefore form a group with respect to multiplication. The spectrum of an element $x\in A$ is the set of all complex numbers 入 such that x- ke is not invertible. We shall denote the spectrum of x by o(x) 18.2 The theory of Banach algebras contains a great deal of interplay between algebraic properties on the one hand and topological ones on the other. We already saw an example of this in Theorem 9.21, and shall see others. There are also close relations between Banach algebras and holomorphic functions The easiest proof of the fundamental fact that o(x) is never empty depends on Liouville's theorem concerning entire functions, and the spectral radius formula follows naturally from theorems about power series. This is one reason for re stricting our attention to complex Banach algebras. The theory of real Banach algebras (we omit the definition, which should be obvious) is not so satisfactory The Invertible Elements In this section, A will be a complex Banach algebra with unit e, and ${\boldsymbol{G}}$ G will be the set of all invertible elements of $\scriptstyle A$ 18.3 Theorem Ifxe A and $\|x\|<1,t h e n\,e+x\in G,$ $$ (e+x)^{-1}=\sum_{n=0}^{\infty}(-1)^{n}x^{n}, $$ (1) and $$ \|(e+x)^{-1}-e+x\|\leq{\frac{\|x\|^{2}}{1-\|x\|}}. $$ (2) PROOF The multiplicative inequality 18.1(3) shows that $\|x^{n}\|\leq\|x\|^{n}.$ If $$ s_{N}:=e-x+x^{2}-\cdots+(-1)^{N}x^{N}, $$ (3)358 REAL AND CoMPLEX ANALYsis it follows that $\scriptstyle(s_{n}|$ is a Cauchy sequence in A, hence the series in (1) con- verges (with respect to the norm of $A|$ to an element $y\in A.$ Since multiplica- tion is continuous and $$ (e+x)s_{N}=e+(-1)^{N}x^{N+1}=s_{N}e+x), $$ (4) we see that $(e+x)y=e=y(e+x).$ This gives (1), and (2) follows from $$ \left|\left|\sum_{n=2}^{\infty}(-1)^{n}x^{n}\right|\right|\leq\sum_{n=2}^{\infty}\|x^{n}\|\leq\sum_{n=2}^{\infty}\|x\|^{n}={\frac{\|x\|^{2}}{1-\|x\|}}. $$ (5) // 18.4 Theorem Suppose xe G, $\|x^{-1}\|=1/\alpha,\ h\in A,$ and $\|h\|=\beta<\alpha.$ Then $x+h\in G,$ and $$ \|(x+h)^{-1}-x^{-1}+x^{-1}h x^{-1}\|\leq{\frac{\beta^{2}}{\alpha^{2}(\alpha-\beta)}}. $$ (1) Px0OF $\|x^{-1}h\|\leq\beta/\alpha<1,$ hence $e+x^{-1}h\in G,$ by Theorem 18.3; and since $x+h=x(e+x^{-1}h),$ we have $x+h\in G$ and $$ (x+h)^{-1}=(e+x^{-1}h)^{-1}x^{-1}. $$ (2) Thus $$ (x+h)^{-1}-x^{-1}+x^{-1}h x^{-1}=\left[(e+x^{-1}h)^{-1}-e+x^{-1}h\right]x^{-1}, $$ (3) and the inequality (1) follows from Theorem 18.3, with $x^{-1}h$ in place of x./ Corollary 1 G is an open set, and the mapping $x\to x^{-1}$ is a homeomorphism of ${\boldsymbol{G}}$ onto G. For if $x\in G$ and $|h|\to0,$ (1) implies that $\|(x+h)^{-1}-x^{-1}\|\to0.$ Thus $x\to x^{-1}$ is continuous; it clearly maps ${\boldsymbol{G}}$ onto G, and since it is its own inverse, it is a homeomorphism Corollary The mapping x→x- is diferentiable. Its differential at an x ∈ ${\boldsymbol{G}}$ is the linear operator which takes $h\in A\ t o\ -x^{-1}h x^{-1}$ This can also be read off from (1). Note that the notion of the differential of a transformation makes sense in any normed linear space, not just in $R^{k},$ as in Definition 7.22. If ${\cal A}_{\bf d}$ is commutative,the above differential takes h to $-x^{-2}h,$ which agrees with the fact that the derivative of the holomorphic function $z^{-1}$ is -z-3 Corollary 3 For every x∈ A,o(x) is compact, and $|\lambda|\leq\|x\|$ 矿入e otx).ELEMENTARY THEORY OF BANACH ALGEBRAs 359 of $x-\lambda e=-\lambda(e-\lambda^{-1}x);$ For if|A| > |xl,then e-2-1xe G, by Theorem 18.3, and the same is tre is closed, observ $(a)\lambda\in\sigma(x)$ A, by Corollary 1; and (c) the mapping hence入生α(x). To prove that $\scriptstyle{(n)}$ is a closed subset of if and only if $x-\lambda e\not\in G;(b)$ the complement of ${\boldsymbol{G}}$ $\lambda\to x-\lambda e$ is a continuous mapping of the complex plane into $A.$ 18.5 Theorem Let D be a bounded linear functional on A,fixxe A, and define $$ f(\lambda)=\Phi[(x-\lambda e)^{-1}]\qquad(\lambda\not rightarrow\sigma(x)). $$ (1) Thenf is holomorphic in the complement of ox) and f()→O as $\lambda{\mathfrak{D}}.$ PRoOF Fix $\lambda\neq\sigma(x)$ and apply Theorem 18.4 with $x-\lambda e$ in place of $\scriptstyle{\mathcal{X}}$ and with $({\lambda}-\mu)e$ in place of h. We see that there is a constant ${\boldsymbol{C}},$ depending on x and ${\boldsymbol{\lambda}}_{*}$ such that $$ \|(x-\mu e)^{-1}-(x-\lambda e)^{-1}+(\lambda-\mu)(x-\lambda e)^{-2}\|\leq C|\,\mu-\lambda|^{2} $$ (2) for all ${\boldsymbol{\mu}}$ which are close enough to $\lambda.$ .Thus $$ {\frac{(x-\mu e)^{-1}-(x-\lambda e)^{-1}}{\mu-\lambda}}\to(x-\lambda e)^{-2} $$ (3) $\Phi$ as $\mu\to\lambda,$ and if we apply $\Phi$ to both sides of (3), the continuity and linearity of show that $$ {\frac{f(\mu)-f(\lambda)}{\mu-\lambda}}\to\Phi[(x-\lambda e)^{-2}]. $$ (4) So $\boldsymbol{\f}$ is differentiable and hence holomorphic outside c(x). Finally, as $\lambda{ arrow}~\mathrm{o}$ we have $$ \lambda f(\lambda)=\Phi[\lambda(x-\lambda e)^{-1}]=\Phi\bigg[\bigg(\frac{x}{\lambda}-e\bigg)^{-1}\bigg] arrow\Phi(-e), $$ (5) by the continuity of the inversion mapping in ${\cal G}.$ // 18.6 Theorem For everyxe A, o(x) is compact and not empty Then PRoOF We already know that $\scriptstyle{e(x)}$ were empty, Theorem 18.5 would imply that and fix 入o生 o(x) is of a bounded linear functional $\ \Phi$ $\scriptstyle{v(x)}$ is compact. Fix $x\in A,$ where fis defined $\boldsymbol{\f}$ $(x-\lambda_{0}e)^{-1}\neq0,$ on A such and the Hahn-Banach theorem implies the existence as in Theorem 18.5. If $\mathrm{that}f(\lambda_{0})\neq0,$ an entire function which tends to O at o, hence $(\lambda_{0})\neq0.$ So ox)is not empty:// Liouville's theorem, and this contradicts f $f(\lambda)=0$ for every入,by 18.7 Theorem (Gelfand-Mazur)If A is a complex Banach algebra with unit in which each nonzero element is invertible, then $\scriptstyle A$ is (isometrically isomorphic to) the complex field.360 REAL AND coMPLEX ANALYSIs An algebra in which each nonzero element is invertible is called a division algebra. Note that the commutativity of $\scriptstyle A$ is not part of the hypothesis; it is part of the conclusion. $x-\lambda_{2}\,e$ PR0OF Ifx∈ A and $\lambda_{1}\neq\lambda_{2},$ at least one of the elements $x-\lambda_{1}e$ and must be invertible, since they cannot both be O. It now follows from Since Theorem 18.6 that $\scriptstyle{\sigma_{(k)}}$ consists of exactly one point, say $\scriptstyle A$ onto the complex field, which is for each $x\in A.$ /// $x\to\lambda(x)$ $x-\lambda(x)e$ is not invertible, it must be O, hence ${\underset{i}{\operatorname{a}(z)}}.$ The mapping is therefore an isomorphism of $x=\lambda(x)e.$ also an isometry, since $|\lambda(x)|=\|\lambda(x)e\|=\|x\|$ for all $x\in A.$ 18.8 Definition For any $x\in A,$ the spectral radius p(x) of xis the radius of the smallest closed disc with center at the origin which contains o(x) (sometimes this is also called the spectral norm of x; see Exercise 14): $$ \rho(x)=\operatorname*{sup}\;\{\;|\,\lambda|\!:\,\lambda\in\sigma(x)\}. $$ 18.9 Theorem (Spectral Radius Formula) For every $\scriptstyle{\mathcal{X}}$ e A, $$ \operatorname*{lim}_{n arrow\infty}\|x^{n}\|^{1/n}=\rho(x). $$ (1) (This existence of the limit is part of the conclusion.) PROOF Fix x∈ A,let $r_{\mathit{l}}$ be a positive integer,入a complex number, and assume $\lambda^{n}\notin\sigma(x^{n}).$ We have $$ (x^{n}-\lambda^{n}e)=(x-\lambda e)(x^{n-1}+\lambda x^{n-2}+\cdot\cdot\cdot+\lambda^{n-1}e). $$ (2)) Multiply both sides of (2) by $(x^{n}-\lambda^{n}e)^{-1}$ This shows that $x-\lambda e$ is invert- ible, hence $\lambda\notin\sigma(x).$ then $\lambda^{n}\in\sigma(x^{n})$ for n = 1,2,3,.……. Corollary 3 to Theorem This gives So if $\lambda\in\sigma(x),$ and therefore $|\,\lambda\,|\,\leq\,\|x^{n}\|^{1/n}.$ 18.4 shows that $|\,\lambda^{n}\,|\leq\,\|x^{n}\|,$ $$ \rho(x)\leq\operatorname*{lim}_{n\to\infty}\operatorname*{inf}_{\alpha}\left\|x^{n}\right\|^{1/n}. $$ (3) Now if $|\lambda|>\|x\|,{\mathrm{it}}$ is easy to verify that $$ \left(\lambda e-x\right)\sum_{n=0}^{\infty}\lambda^{-n-1}x^{n}=e. $$ (4) tional on $\scriptstyle A$ The above series is therefore $-(x-\lambda e)^{-1}$ . Let $\Phi$ be a bounded linear func- and define f as in Theorem 18.5. By (4) the expansion $$ f(\lambda)=-\sum_{n=0}^{\infty}\Phi(x^{n})\lambda^{-n-1} $$ (5)ELEMENTARY THEORY OF BANACH ALGEBRAS 361 $({\mathfrak{f}})$ is valid for all $\lambda$ such that $|\lambda|>\|x\|.$ By Theorem 18.5, $\boldsymbol{\mathsf{f}}$ is holomorphic outside o(x), hence in the set $\{\lambda\colon|\lambda|>\rho(x)\}.$ It follows that the power series converges if $|\lambda|>\rho(x).$ In particular, $$ \begin{array}{c l}{{\mathrm{sup}\ |\Phi(\lambda^{-n}x^{n})|<\infty}}&{{\quad(|\lambda|>\rho(x))}}\end{array} $$ (6) for every bounded linear functional $\Phi$ D on $A.$ It is a consequence of the Hahn-Banach theorem (Sec. 5.21)that the norm of any element of A is the same as its norm as a linear functional on the dual space of $A.$ Since (6) holds for every , we can now apply the corresponds a real number Banach-Steinhaus theorem and conclude that to each Awith such that there $|\lambda|>\rho(x)$ $\scriptstyle(c_{i})$ $$ \|\lambda^{-n}x^{n}\|\leq C(\lambda)\qquad(n=1,\,2,\,3,\,\ldots). $$ (7) Multiply (7) by |入|" and take nth roots. This gives $$ \|x^{n}\|^{1/n}\leq|\lambda|\,[C(\lambda)]^{1/n}\qquad(n=1,\,2,\,3,\,\ldots) $$ (8) if $|\lambda|>\rho(x),$ and hence $$ \operatorname*{lim}_{n arrow\infty}\operatorname*{sup}_{\theta}\,\,\|x^{n}\|^{1/n}\leq\rho(x). $$ (9) The theorem follows from (3) and (9) // 18.10 Remarks (a) Whether an element of A is or is not invertible in $\scriptstyle A$ is a purely algebraic property. Thus the spectrum of ${\boldsymbol{x}},$ and likewise the spectral radius p(x) are defined in terms of the algebraic structure of A, regardless of any metric (or topological)) considerations. The limit in the statement of Theorem 18.9,on the other hand, depends on metric properties of A. This is one of the remarkable features of the theorem:It asserts the equality of two quantities which arise in entirely different ways (b)Our algebra may be a subalgebra of a larger Banach algebra $\boldsymbol{B}$ (an example follows), and then it may very well happen that some $x\in A$ is not invertible in $\scriptstyle A$ but is invertible in B. The spectrum of $\scriptstyle{\mathcal{X}}$ therefore depends on the algebra; using the obvious notation, we have $\sigma_{A}(x)\beth$ og(x), and the inclusion may be proper. The spectral radius of ${\boldsymbol{x}},$ however, is unaffected by this, since Theorem 18.9 shows that it can be expressed in terms of metric properties of powers of x, and these are independent of anything that happens outside A. 18.11 Example Let $\operatorname{c}(\eta)$ be the algebra of all continuous complex functions on the unit circle T(with pointwise addition and multiplication and the supremum norm), and let ${\boldsymbol{A}}$ be the set of all fe C(T) which can be extended F is to a continuous function ${\mathbf{}}F$ on the closure of the unit disc ${\boldsymbol{U}}_{\cdot}$ J, such that ${\mathbf{}}F$ holomorphic in U. It is easily seen that $\scriptstyle A$ is a subalgebra of $\operatorname{c}(n)$ $\operatorname{If}f_{n}$ e A362 REAL AND coMPLEX ANALYSis and $\{f_{n}\}$ converges uniformly on $\{F_{n}\}$ to converge uniformly on the closure of L ${\boldsymbol{U}}.$ This ${\boldsymbol{T}}.$ the maximum modulus theorem forces the associated sequence shows that A is a closed subalgebra of $\operatorname{c}(n)$ and so A is itself a Banach algebra Define the function f $\scriptstyle A$ consists of the closed unit disc; with respect to $F_{o}(z)=z$ The spectrum of fo the as an element of $f_{0}$ ,by $f_{0}(e^{i\theta})=e^{i\theta}.$ Then $\operatorname{c}(t),$ spectrum of $f_{\mathrm{0}}$ consists only of the unit circle. In accordance with Theorem 18.9, the two spectral radii coincide. Ideals and Homomorphisms From now on we shall deal only with commutative algebras 18.12 Definition A subset ${\mathbf I}$ of a commutative complex algebra A is said to be an ideal if (a) ${\mathbf I}$ is a subspace of $\scriptstyle A$ (in the vector space sense) and ((b) $\scriptstyle x\lor\in I$ whenever x∈ A and $\nu\in I,$ I. If $I\neq A,\,I$ is a proper ideal. Maximal ideals are proper ideals which are not contained in any larger proper ideals. Note that no proper ideal contains an invertible element. lt homomorphism if $\varphi$ is another complex algebra, a mapping $\varphi$ $\scriptstyle A$ into $\boldsymbol{B}$ is called a $\boldsymbol{B}$ 9 of is a linear mapping which also preserves multiplication: of all $\varphi(x)\varphi(y)=\varphi(x y)$ for all $\scriptstyle{\mathcal{X}}$ and $\mathbf{\vec{y}}$ ∈ A. The kernel (or null space) of $\varphi$ is the set $x\in A$ such that $\varphi(x)=0.$ It is trivial to verify that the kernel of a homomorphism is an ideal. For the converse, see Sec. 18.14. 18.13 Theorem If A is a commutative complex algebra with unit, every proper ideal of A is contained in a maximal ideal. ${\bar{I}}{\bar{f}},$ f,in addition, $\scriptstyle A$ is a Banach algebra,every maximal ideal of A is closed. PRoOF The first part is an almost immediate consequence of the Hausdorf maximality principle (and holds in any commutative ring with unit). Let ${\mathbf I}$ be a proper ideal of A. Partially order the collection ${\mathcal{P}}$ of all proper ideals of $\scriptstyle A$ which contain I(by set inclusion), and let A $\ M$ be the union of the ideals in some maximal linearly ordered subcollection Q of ${\mathcal{P}}.$ Then ${\cal{M}}$ is an ideal (being the union of a linearly ordered collection of ideals), $I\subset M,$ , and $M\neq$ A, since no member of ${\mathcal{P}}$ P contains the unit of A. The maximality of $\underline{{\mathcal{A}}}$ implies that N $\textstyle{M}$ is a maximal ideal of $A.$ If A is a Banach algebra, the closure T $\overline{{M}}$ 4 of ${\cal{M}}$ is also an ideal (we leave the details of the proof of this statement to the reader). Since ${\cal{M}}$ I contains no invertible element of $\scriptstyle A$ and since the set of all invertible elements is open, we have ${\overline{{M}}}\neq A,$ and the maximality of ${\cal{M}}$ therefore shows that ${\widetilde{M}}=M.$ ${\it j}/j{\it j}$ 18.14 Quotient Spaces and Quotient Algebras Suppose $\boldsymbol{J}$ is a subspace of a vector space A, and associate with each $x\in A$ the coset $$ \varphi(x)=x+J=\{x+y\colon y\in J\}. $$ (1)ELEMENTARY THEORY OF BANACH ALGEBRAs 363 If $x_{1}-x_{2}\in J,$ then $\varphi(x_{1})=\varphi(x_{2}).$ If ${\bf\hat{\ }}\cdot x_{1}-x_{2}\notin J,\,\varphi(x_{1})\cap\varphi(x_{2})=\mathcal{D}.$ The set of all cosets of $\boldsymbol{J}$ is denoted by $A/J;{\mathfrak{i}}$ t is a vector space if we define 4 $$ \rho(x)+\varphi(y)=\varphi(x+y),\qquad\lambda\varphi(x)=\varphi(\lambda x) $$ (2) for $\scriptstyle{\mathcal{X}}$ and $\scriptstyle\nu\in A$ "and scalars A. Since J $\boldsymbol{J}$ is a vector space, the operations (2) are well defined; this means that if $\varphi(x)=\varphi(x^{\prime})\,a$ and py) = p(y'), then $$ \varphi(x)+\varphi(y)=\varphi(x^{\prime})+\varphi(y^{\prime}),\qquad\lambda\varphi(x)=\lambda\varphi(x^{\prime}). $$ (3) Also, $\varphi$ is clearly a linear mapping of $\scriptstyle A$ onto $A/J;$ the zero element of $A/J$ is $\varphi(0)=J.$ Suppose next that A is not merely a vector space but a commutative algebra and that ${\mathbf{}}J$ is a proper ideal of $A.$ If $x^{\prime}-x:$ e J and $y^{\prime}-y\in J,$ the identity $$ x^{\prime}y^{\prime}-x y=(x^{\prime}-x)y^{\prime}+x(y^{\prime}-y) $$ (4) shows that $x^{\prime}y^{\prime}-x y\in J.$ Therefore multiplication can be defined in $A/J$ in a consistent manner : $$ \varphi(x)\varphi(y)=\varphi(x y)\qquad(x{\mathrm{~and~}}y\in A). $$ (5) It is then easily verified that $A/J$ is an algebra, and $\varphi$ is a homomorphism of A onto $A/J$ whose kernel is ${\boldsymbol{J}}.$ is the unit of $A/J,$ and $A/J$ is a field if and If $\scriptstyle A$ has a unit element e, then $\varphi(e)$ only j J is a maximalsleal To see this, suppose xe $\scriptstyle A$ and x生J, and put $$ I=\{a x+y;a\in A,y\in J\}. $$ (6) Then I ${\mathbf I}$ is an ideal in A which contains for some $\scriptstyle{\mathcal{X}}$ as above so that $I\neq A,$ hence $\scriptstyle\epsilon\not\in I,$ and then and $I=A,$ is not maximal, we can choose ${\mathbf{}}J$ properly, since $x\in I$ If ${\mathbf{}}J$ is maximal, hence $a x+y=e$ $a\in A$ and $\nu\in J,$ hence $\varphi(a)\varphi(x)=\varphi(e);$ this says that every nonzero element of $A/J$ is invertible, so that $A/J$ is a field. If J cp(x) is not invertible in $A/J.$ 18.15 Quotient Norms Suppose $\scriptstyle A$ is a normed linear space, $\boldsymbol{J}$ is a closed subspace of A, and $\varphi(x)=x+J,$ as above. Define $$ \|\varphi(x)\|=\operatorname*{inf}\left\{\|x+y\|\colon y\in J\right\}. $$ (1) Note that Ilp(x)| is the greatest lower bound of the norms of those elements which lie in the coset $\scriptstyle{\theta(x){\mathrm{:}}}$ ; this is the same as the distance from $A/J.$ It has the following We call $\scriptstyle{\mathcal{X}}$ x to ${\boldsymbol{J}}.$ the norm defined in $A/J$ by $\mathbf{(1)}$ the quotient norm of properties : (a)A/J is a normed linear space. (c) (b)If A is a Banach space, so is A/J. is a proper closed ideal, then A/J If A is a commutative Banach algebra and $\boldsymbol{J}$ is a commutative Banach algebra. These are easily verified:364 REAL AND COMPLEX ANALYSIS and If $x\in J,\;\|\varphi(x)\|=0.$ I $x\notin J.$ the fact that $\boldsymbol{J}$ is closed implies that $\scriptstyle\epsilon\;>0,$ there exist $y_{\mathrm{1}}$ $y_{i}\in J$ so that It is clear that Iixo0(x) = 2110). 1f $x_{1}$ and $x_{2}\in A$ and $|\psi(x)|>0.$ $$ \|x_{i}+y_{i}\|<\|\varphi(x_{i})\|+\epsilon\qquad(i=1,2). $$ (2) Hence llox + ×2)| ≤ |lx1 + ×2 + y1 + P2ll < ||lo(x,)| |le(x2)| + 2 (3) which gives the triangle inequality and proves (a) Suppose $\scriptstyle A$ is complete and $\{\varphi(x_{n})\}$ is a Cauchy sequence in A/J. There is a subsequence for which $$ \|\varphi(x_{n})-\varphi(\dot{x}_{n_{i+1}})\|<2^{-i}\qquad(i=1,\,2,\,3,\,\ldots), $$ (4) a Cauchy sequence in $A\,\cdot$ 4; and since $\scriptstyle A$ $z_{i}-x_{n_{i}}\in J$ and $\|z_{i}-z_{i+1}\|<2^{-i}.$ Thus $\left\{z_{i}\right\}$ is and there exist elements $z_{i}$ so that is complete, there exists $z\in A$ such that $\|z_{i}-z\|\to0.$ It follows that $\varphi(x_{n})$ converges to $\varphi(z)$ in A/.J. But if a Cauchy sequence has a convergent subsequence, then the full sequence converges. Thus $A/J$ is complete, and we have proved (b) and $x_{2}\in A$ and $\scriptstyle x\;\simeq\;0.$ and choose $y_{\mathrm{I}}$ and $y_{i}\in J$ so To prove (c), choose $x_{1}$ that (2) holds. Note tha $$ \operatorname{t}\left(x_{1}+y_{1}\right)\left(x_{2}+y_{2}\right)\in x_{1}x_{2}+J,\sin $$ that $$ \|\varphi(x_{1}x_{2})\|\leq\|(x_{1}+y_{1})(x_{2}+y_{2})\|\leq\|x_{1}+y_{1}\||x_{2}+y_{2}\| $$ l| (5) Now (2) implies $$ \|\varphi(x_{1}x_{2})\|\leq\|\varphi(x_{1})\|\,\|\varphi(x_{2})\|. $$ (6) Finally, if ${\mathcal{C}}$ is the unit element of $A,$ take $x_{1}$ 生 $\boldsymbol{J}$ and $\scriptstyle v_{2}=e$ in (6); this gives $|\phi(e)|\geq1.$ But ee p(e)and the definition of the quotient norm shows that $\|\varphi(e)\|\le\|e\|=1.80\ \|\varphi(e)\|=1$ , and the proof is complete. 18.16 Having dealt with these preliminaries, we are now in a position to derive some of the key facts concerning commutative Banach algebras. Suppose,as before, that $\scriptstyle A$ is a commutative complex Banach algebra with unit element e. We associate with ${\boldsymbol{A}}$ the set $\underline{{\land}}$ of all complex homomorphisms of $A{\mathrm{i}}$ these are the homomorphisms of $\scriptstyle A$ onto the complex field, or, in different terminology, the multiplicative linear functionals on $\scriptstyle A$ which are not identically O As before, o(x) denotes the spectrum of the element x $\in A,$ and $\scriptstyle{\rho(x)}$ is the spectra radius of x Then the following relations hold: 18.17 Theorem (a) Every maximal ideal M of A is the kernel of some $h\in\Delta.$ (e) (b)入∈ o(x)if and only if $h(x)=\lambda$ for some $h\in\Delta.$ (c)x is invertible in A ijf and only if $|t(x)\neq0$ for every he A (d)h(x)e o(x) for every $\mathrm{e.e\;}A$ and h e A and $h\in\Delta.$ $|h(x)|\leq\rho(x)\leq\|x\|$ for every $x\in A$ELEMENTARY THEORY OF BANACH ALGEBRAS 365 of ${\boldsymbol{h}}$ PRoor If $\textstyle{M}$ is a maximal ideal of $A,$ then A/M is a field; and since 。Q, where $\varphi$ is the is $\textstyle{M}$ closed (Theorem 18.13) $A/M$ is a Banach algebra. By Theorem 18.7 there is and the kernel homomorphism of $\scriptstyle A$ 4 onto $A/M$ onto the complex field. If $h=j$ $h\in\Delta$ an isomorphism $\dot{\boldsymbol{\jmath}}$ of $A/M$ whose kernel is M, then is M. This proves (a) If $\lambda\in\sigma(x),$ then $x-{\overline{{x}}}x$ is not invertible; hence the set of all elements (x- Ae)y, where $y\in A,$ is a proper ideal of $A,$ , which lies in a maximal ideal (by Theorem 18.13),and(a) shows that there exists an $h\in\Delta$ such that $h(x-\lambda e)=0.$ Since $h(e)=1,$ this gives $h(x)=\lambda.$ for some $y\in A.$ It follows that On the other hand, if $\lambda\notin\sigma(x),$ then $(x-\lambda e)y=e$ $h(x-\lambda e)\neq0,$ or $h(x-\lambda e)h(y)=1$ for every $h\in\Delta,$ so that Since $\scriptstyle{\mathcal{X}}$ This proves (b) follows from (b). // $h(x)\neq\lambda.$ is invertible if and only if $0\not\in\sigma(x),(c)$ Finally, (d) and ${\boldsymbol{\epsilon}})$ are immediate consequences of (b) Note that (e) implies that the norm of $h_{\mathrm{,}}$ as a linear functional, is at most 1. In 9.21) particular, each $h\in\Delta$ is continuous. This was already proved earlier (Theorem Applications We now give some examples of theorems whose statements involve no algebraic concepts but which can be proved by Banach algebra techniques. 18.18 Theorem Let A(U) be the set of $\ a l l$ continuous functions on the closure $\overline{{U}}$ of the open unit disc $U$ whose restrictions to ${\boldsymbol{U}}$ are holomorphic. Suppose f, … ${\mathfrak{f}}_{n}$ are members of $A(U),$ such that $$ |f_{1}(z)|+\cdot\cdot\cdot+|f_{n}(z)|>0 $$ (1) for every $z\in{\overrightarrow{U}}$ . Then there exist $g_{1},\ldots,g_{n}\in A(U)\,s u c h\,t h a t$ $$ \sum_{i=1}^{n}f_{i}(z)g_{i}(z)=1\qquad(z\in\bar{U}). $$ (2) PROOF Since sums, products, and uniform limits of holomorphic functions set ideal of $A(U).$ By Theorem 18.13 this happens if and only if ${\mathbf{}}J$ is a Banach algebra, with the supremum norm. The $A(U),$ is an $\boldsymbol{J}$ are holomorphic, $A(U)$ $\Sigma f_{i}\,g_{i},$ where the $g_{i}$ contains the unit element l of $A(U).$ of all functions are arbitrary members of We have to prove that $\boldsymbol{J}$ lies in no maximal ideal of $A(U).$ homomorphism F $\boldsymbol{h}$ h of By Theorem 18.17(a) it is therefore enough to prove that there is no for every $A(U)$ onto the complex field such that $h(f_{i})=0$ $i\left(1\leq i\leq n\right)$366 REAL AND COMPLEX ANALYSIS Before we determine these homomorphisms, let us note that the poly since $\boldsymbol{\f}$ is uniformly continuous on ${\bar{U}},$ To see this, suppose f∈ A(U) and $\scriptstyle{r\,<\,1}$ such that nomials form a dense subset of $A(U).$ there exists an $\scriptstyle\epsilon\;\!>0;\;$ verges if $|f(z)-f(r z)|<\epsilon$ for all $z\in{\tilde{U}};$ ; the expansion of f(rz) in powers of z con and this gives $\lceil r z\rceil<1,$ hence converges to f(rz) uniformly for $z\in{\overline{{U}}},$ the desired approximation. $\alpha\in{\overline{{U}}}$ such that $h(f_{0})=x.$ be a complex homomorphism of $A(U).$ Put $f_{0}(z)=z.$ Then Now let ${\boldsymbol{h}}$ $f_{0}\in{\frac{2}{4}}(U).$ It is obvious that $\sigma(f_{0})=U.$ By Theorem 18.17(d) there exists an 2,3, …, so Hence $h(f_{0}^{a})=\alpha^{n}=f_{0}^{a}(x),$ for $\scriptstyle n\;=\;1.$ ${\mathsf{I}}\leq i\leq r$ We have proved that to each $h\in\Delta$ for every polynomial P. Since his continuous and since the poly- $A(U).$ $h(P)=P(x)$ . Thus $h(f_{i})\neq0.$ there corresponds at least one of the nomials are dense in $A(U)_{i}$ it follows that hMG $^{*}h\,{\bar{r}}\,{\stackrel{_\ *}}\,=\,{\bar{r}}\,{\bar{\neq}}$ ) for every f∈ Our hypothesis(1) implies that $|f_{i}(x)|>0$ for at least one index i, given functions ${\boldsymbol{f}}_{i}$ such that $h(f_{j}\neq\{0,$ and this, as we noted above, is enough to prove the theorem. // all maximal ideals of $A(U)$ Note: We have also determined all maximal ideals of $A(U),$ in the course of 。is the preceding proof, since each is the kernel of some then $M_{\mathrm{e}}$ is a maximal ideal of $A(U),$ and $h\in\Delta\colon I f\alpha\in{\bar{U}}$ and if $M_{\alpha}$ the set of all f∈ A(U) such that $f(x)=0,$ are obtained in this way ${\mathit{A}}(U)$ is often called the disc algebra 18.19 The restrictions of the members of $A(U)$ to the unit circle ${\mathbf{}}T$ form a closed subalgebra of $\operatorname{c}(\eta)$ This is the algebra A discussed in Example 18.11. In fact, $\scriptstyle A$ is a maximal subalgebra of $\operatorname{c}(n)$ More explicitly, i $A\subset B\subset C(T)$ and $\boldsymbol{B}$ is a closed (relative to the supremum norm) subalgebra of $\operatorname{c}(r)$ then either $\scriptstyle{B=A}$ $\scriptstyle A$ consists pre- It is easy to see (compare with Exercise 29,Chap. 17) that or $B=C(T).$ cisely of those f ∈ $\operatorname{c}(n)$ for which $$ \hat{f}(n)=\frac{1}{2\pi} |_{-\pi}^{\pi}f(e^{i\theta})e^{-i n\theta}~d\theta=0~~~~~~(n=-1,\,-2,\,-3,\,...).\nonumber $$ (1) Hence the above-mentioned maximality theorem can be stated as an approx imation theorem: f ∈ $\operatorname{c}(n)$ 18.20 Theorem Suppose $g\in C(T)$ and ${\hat{\sigma}}(n)\neq0$ for some $\scriptstyle n\,<0.$ Then to every and to every $\scriptstyle x\,>0$ there correspond polynomials $$ P_{n}(e^{i\theta})=\sum_{k=0}^{m(n)}a_{n,k}\,e^{i k\theta}\qquad(n=0,\ldots,N) $$ (1) such that $$ |f(e^{i\theta})-\sum_{n=0}^{N}P_{n}(e^{i\theta})g^{n}(e^{i\theta})|<\epsilon\qquad(e^{i\theta}\in T). $$ (2)ELEMENTARY THEORY OF BANACH ALGEBRAS 367 PRoOF Let $\boldsymbol{B}$ be the closure in $C=C(T)$ of the set of all functions of the form $$ \mathbf{\Sigma}_{n=0}^{N}P_{n}g^{n}. $$ (3) The theorem asserts that $\scriptstyle B=C.$ Let us assume $\beta\neq C,$ The set of all functions (3)(note that ${\boldsymbol{N}}$ is not fixed) is a complex algebra Its closure $\boldsymbol{B}$ is a Banach algebra which contains the function $f_{\mathrm{0}}$ o,where $f_{0}(e^{i\theta})=e^{i\theta}.$ Our assumption that $\beta\neq C$ implies that $1/5_{0}\notin B,$ for otherwise $\boldsymbol{B}$ would contain $f_{0}^{n}$ for all integers n, hence all trigonometric polynomials would be in ${\boldsymbol{B}}{\boldsymbol{\cdot}}$ and since the trigonometric polynomials are dense in ${\boldsymbol{C}}$ (Theorem 4.25) we should have $\scriptstyle B=\mathbb{C}.$ So $m_{_{0}}$ is not invertible in ${\boldsymbol{B}}.$ By Theorem 18.17 there is a complex homo- morphism ${\boldsymbol{h}}$ of $\boldsymbol{B}$ such that $h(f_{0})=0.$ Every homomorphism onto the complex feld satisfies $h(1)=1;$ and since $h(f_{0})=0,$ we also have $$ h(f_{0}^{n})=[h(f_{0})]^{n}=0\qquad(n=1,2,\,3,\,...). $$ (4) We know that ${\boldsymbol{h}}$ is a linear functional on B, of norm at most 1. The Hahn-Banach theorem extends ${\boldsymbol{h}}$ to a linear functional on ${\boldsymbol{C}}$ (still denoted by real for real $f\colon$ hence 5.22 shows that h is a positive linear functional on ${\boldsymbol{C}}.$ the argument used in Sec ${}^{n}/{}\quad$ is h) of the same norm. Since $h(1)=1$ and $|h|\leq1,$ In particular, $h({\bar{f}})={\overline{{h(f)}}},$ Since f " is the complex conjugate of f8, it follows that (4) also holds for $n=-1.$ 一2,-3,.….Thus $$ h(f_{\,0}^{n})=\left\{\frac{1}{0}\begin{array}{\;\;\mathrm{if}\;n=0,}\\ {\mathrm{if}\;n\neq0.}\end{array}\right. $$ (5) Since the trigonometric polynomials are dense in ${\boldsymbol{C}},$ there is only one bounded linear functional on ${\boldsymbol{C}}$ which satisfies (5). Hence ${\boldsymbol{h}}$ h is given by the formula $$ h(f)=\frac{1}{2\pi}\left|\sum_{-\pi}^{\pi}f(e^{i\theta})\,d\theta\right.\qquad(f\in C). $$ (6) Now if $\scriptstyle n$ is a positive integer, $d l_{0}^{\prime}\in B;$ and since ${\boldsymbol{h}}$ is multiplicative on ${\boldsymbol{B}},$ (6) gives $$ \hat{g}(-n)=\frac{1}{2\pi}\int_{-\pi}^{\pi}g(e^{i\theta})e^{i n\theta}~d\theta=h(g f_{0}^{n})=h(g)h(f_{0}^{n})=0, $$ (7) by (5). This contradicts the hypothesis of the theorem. // We conclude with a theorem due to Wiener 18.21 Theorem Suppose $$ f(e^{i\theta})=\sum_{-\infty}^{\infty}\,c_{n}e^{i n\theta},\qquad\sum_{-\infty}^{\infty}\,|\,c_{n}|<\infty, $$ (1)368 REAL AND coMPLEX ANALYSIS and f(e') ≠ 0 for every real 8.Then $$ \frac{1}{f(e^{i\theta})}=\sum_{-\infty}^{\infty}\,\gamma_{n}\,e^{i n\theta}\;\;\;\omega i t h\;\;\;\sum_{-\infty}^{\infty}\,|\gamma_{n}|<\infty. $$ (2) PROOF We let ${\cal A}$ be the space of all complex functions $\boldsymbol{\mathit{f}}$ on the unit circle which satisfy (1), with the norm $$ \|f\|=\sum_{-\infty}^{\infty}\left|c_{n}\right|. $$ (3) $\mathrm{I}\!t$ is clear that $\scriptstyle A$ is a Banach space. In fact, $\scriptstyle A$ is isometrically isomorphic to $\ell^{1}.$ the space of all complex functions on the integers which are integrable with respect to the counting measure. But $\scriptstyle A$ is also a commutative Banach algebra, under pointwise multiplication. For i $\scriptstyle g\in A$ and $g(e^{i\theta})=\Sigma b_{n}e^{i n\theta},$ then $$ f(e^{i\theta})g(e^{i\theta})=\sum_{n}\left(\sum_{k}\,c_{n-k}\,b_{k}\right)e^{i n\theta} $$ (4) and hence $$ \|f g\|=\sum_{n}\left|\sum_{k}c_{n-k}b_{k}\right|\leq\sum_{k}|b_{k}|\sum_{n}|c_{n-k}|=\|f\|\cdot\|g\|. $$ (5) that lhl Also, the function $\mathbf{1}$ is the unit of ${\bar{A}},$ and $\|1\|=1.$ and $\;|f_{\,0}^{n}||=1$ for $\scriptstyle n\;=\;0,$ $\mathrm{Put}f_{0}(e^{i\theta})=e^{i\theta},$ f $\boldsymbol{h}$ is any complex homomorphism of $\scriptstyle A$ and $h(f_{0})=\lambda,$ the fact as before ${\mathrm{Then}}f_{0}\in A,\,1/j_{0}\in A,$ $\pm1,\;{\pm2,}\ldots$ implies that $\leq1$ $$ |\lambda^{n}|=|h(f_{0}^{n})|\leq\|f_{0}^{n}\|=1\qquad(n=0,\,\pm1,\,\pm2,\,...). $$ (6) that $h(f_{0})=e^{i s},$ so In other words, to each ${\boldsymbol{h}}$ corresponds a point $e^{i x}\in T$ such Hence $|\lambda|=1.$ $$ h(f_{\;\;0}^{n})=e^{i n x}=f_{\;\;0}^{n}e^{i x})\qquad(n=0,\;\pm1,\;\pm2,\,\ldots). $$ (7 ${\boldsymbol{h}}$ If fis given by (1), then $f=\Sigma c_{n}f_{0}^{n}$ .This series converges in $A{\mathrm{;}}$ and since is a continuous linear functional on $A,$ we conclude from $\left(7\right)$ that $$ h(f)=f(e^{i x})\qquad(f\in A). $$ (8) Our hypothesis th ${\mathfrak{a r f}}$ vanishes at no point of ${\mathbf{}}T$ says therefore that $\boldsymbol{\mathsf{f}}$ is not in the kernel of any complex homomorphism of $A,$ and now Theorem 18.17 implies that fis invertible in $A.$ But this is precisely what the theorem asserts. //ELEMENTARY THEORY OF BANACH ALGEBRAS 369 Exercises 1 Suppose BX)is the algebra of al bounded linear operators on the Banach space $X,$ with $$ (A_{1}+A_{2})(x)=A_{1}x+A_{2}x,\;\;\;(A_{1}A_{2})(x)=A_{1}(A_{2}\,x),\;\;\;\|A\|=\ s u p\;{\frac{\|A x\|}{\|x\|}}, $$ if A, $A_{1},$ and $A_{2}\in B(X).$ Prove that $B(X)$ is a Banach algebra 2 Let n be a positive integer, let $\scriptstyle{\mathcal{X}}$ be the space of all complex n-tuples (normed in any way, as long spectrum of each member of as the axioms for a normed linear space are satisfied), and let $B(X)$ be as in Exercise 1. Prove that the $B(X)$ consists of at most n complex numbers. What are they? 3 Take $X=L^{2}(-\infty,$ $\infty),$ suppose $\varphi\in L^{\alpha}(-\infty,$ 00), and let ${\cal{M}}$ be the multiplication operator which is takes $f\in L^{2}$ to opf. Show that ${\cal{M}}$ is a bounded linear operator on ${\boldsymbol{L}}^{2}$ P and that the spectrum of ${\cal{M}}$ equal to the essential range of g(Chap. 3, Exercise 19) 4 What is the spectrum of the shift operator on $\ell^{2}\gamma$ (See Sec. $17.20$ for the definition.) S Prove that the closure of an ideal in a Banach algebra is an ideal 6 If $\scriptstyle{\mathcal{X}}$ is a compact Hausdorff space, find all maximal ideals in C(X) T Suppose $\scriptstyle A\quad\quad\quad\quad$ is a commutative Banach algebra with unit, which is generated by a single element $\mathbf{X}.$ This means that the polynomials in $\textstyle X$ are dense in ${\boldsymbol{A}}.$ Prove that the complement o $\sigma(x)$ is a con- nected subset of the plane. $H i n t:\mathbb{F}\lambda\not\in\sigma(x),$ there are polynomials ${\mathbf{}}P_{n}$ such that $P_{n}(x)\to(x-\lambda e)^{-1}$ in that A. Prove that $P_{n}(z)\to(z-\lambda)^{-1}$ uniformly for $z\in\sigma(x).$ $z\in{\bar{U}},$ and $1/f(z)=\sum_{o}^{\infty}\,a_{n}\,z^{n}.$ Prove 8 Suppose $\textstyle\sum_{\l}^{\infty}\mid c_{n}\mid<\infty,\,f(z)=\sum_{0}^{\l\alpha}$ CAz", If()| > 0 for every $\textstyle\sum_{\alpha}^{\alpha}\mid a_{n}\mid<\alpha.$ 9 Prove that a closed linear subspace of the Banach algebra $L^{1}(R^{1})$ (see Sec. 9.19) is translation invariant if and only if it is an ideal $10$ Show that $L^{1}(T)$ is a commutative Banach algebra (without unit) if multiplication is defined by $$ (f*g)(t)={\frac{1}{2\pi}}\int_{-\pi}^{\pi}f(t-s)g(s)\;d s. $$ set of Find all complex homomorphisms of $L^{1}(T),$ as in Theorem 9.23. 1 $\boldsymbol{E}$ is a set of integers and if $L^{1}(T),$ and prove ${\mathrm{all}}\,f\in L^{1}(T)$ such $\operatorname{that}{\hat{f}}(n)=0$ for all $n\in E,$ prove that $I_{E}$ is a closed ideal in $I_{E}$ z is the that every closed ideal in $L^{1}(T)$ is obtained in this manner. 11 The resolvent $R(\lambda^{-})={\frac{1}{\lambda^{-1}}}-{\frac{1}{\lambda^{-1}}}-{\frac{1}{\lambda^{-1}}}-{\frac{1}{\lambda^{-1}}}-{\frac{1}{\lambda^{-1}}}-{\frac{1}{\lambda^{-1}}}+{\frac{1}{\lambda^{-1}}}+{\frac{1}{\lambda^{-1}}}+{\frac{1}{\lambda^{-1}}}+{\frac{1}{\lambda^{-1}}}+{\frac{1}{\lambda^{-1}}}$ ${\boldsymbol{x}})$ of an element $\mathbf{X}$ in a Banach algebra with unit is defined as $$ R(\lambda,\,x)=(\lambda e-x)^{-1} $$ for all complex Afor which this inverse exists. Prove the identity $$ R(\lambda,\,x)-R(\mu,\,x)=(\mu-\lambda)R(\lambda)R(\mu) $$ and use it to give an alternative proof of Theorem 18.5 ${\boldsymbol{1}}{\boldsymbol{2}}$ Let $\scriptstyle A\quad\quad A$ be a commutative Banach algebra with unit. The radical of $\scriptstyle A\quad}$ is defined to be the intersec- tion of al maximal ideals of A. Prove that the following three statements about an element $x\in A$ are equivalent: (a) $\textstyle{\mathcal{X}}$ is in the radical of ${\boldsymbol{A}}.$ (b)lim Ix"|1/" = 0 n→α (c) $h(x)=0$ for every complex homomorphism of ${\boldsymbol{A}}.$ space) such that $x^{n}\neq0$ for al $n>0,$ but im.- $\|x^{n}\|^{1/n}=0.$ (for instance, a bounded linear operator on a Hilbert 13 Find an element $\scriptstyle{\dot{\boldsymbol{x}}}$ in a Banach algebra $\scriptstyle A\quad}$370 REAL AND coMPLEX ANALYSIs 14 Suppose $\scriptstyle A\quad\quad A\quad\quad$ is a commutative Banach algebra with unit, and let $\underline{{\land}}$ be the set of al complex homo- morphisms o $A_{*}$ 4, as in Sec. 18.16.Associate with each $x\in A{\textbf{a}}$ function $\textstyle{\hat{x}}$ on $\underline{{\Delta}}$ by the formula $$ {\hat{x}}(h)=h(x)\qquad(h\in\Delta). $$ Ris called the Gelfand transform of $X.$ $\tilde{A}$ of complex functions Prove that the mapping $x\to{\hat{x}}$ is a homomorphism of $\scriptstyle A\quad}$ onto an algebra 五 on A, with pointwise multiplication. Under what condition on $\scriptstyle A\quad}$ is this homomorphism an iso- morphism? (See Exercise 12. Prove that the spectral radius p(x) is equal to $$ \|{\hat{x}}\|_{x_{0}}=\operatorname*{sup}\left\{|{\hat{x}}(h)|;h\in\Delta\right\}, $$ Prove that the range of the function $\textstyle{\hat{x}}$ is exactly the spectrum o(x) and mapping $x\to(x,0)$ is a commutative Banach algebra without unit, let 1 n be the algebra of all ordered pairs (x, ), This is a standard 15 If ${\mathbf{}}A$ $A_{1}$ $A_{1}.$ with $\|(x,\,\lambda)\|=\|x\|+|\lambda|$ is an isometric isomorphism of $\scriptstyle A\quad}$ a complex number; addition and multiplication are defned in the “obvious" way, $x\in A$ and $\dot{\boldsymbol{A}}$ Prove that $A_{1}$ is a commutative Banach algebra with unit and that the onto a maximal ideal of embedding of an algebra without unit in one with unit. 16 Show that $H^{\omega}$ is a commutative Banach algebra with unit, relative to the supremum norm and is a complex homomorphism of $H^{\omega}$ pointwise addition and multiplication. The mapping $f\to f(x)$ whenever $\scriptstyle n:c.$ Prove that there must be others. 17 Show that the set of alfunctions $(z-1)^{2}f,\mathrm{where}\,f\in H^{\infty},$ is an ideal in $H^{\alpha}$ which is not closed. Hint: $$ |(1-z)^{2}(1+\epsilon-z)^{-1}-(1-z)|<\epsilon\qquad\mathrm{if~}|z|<1,\epsilon>\mathbb{C} $$ D. 18 Suppose $\mathcal{\varphi}$ is an inner function. Prove that $\{\varphi f\colon f\in H^{\infty}\}$ is a closed ideal in $H^{\omega},$ In other words prove that i $\{f_{n}\}$ is a sequence in $H^{\infty}$ such that $\sigma f_{n}\to g$ uniformly in ${\boldsymbol{U}},$ then $g/\varphi\in H^{\infty}.$