CHAPTER SEVEN DIFFERENTIATION In elementary Calculus we learn that integration and differentiation are inverses of each other. This fundamental relation persists to a large extent in the context of the Lebesgue integral. We shall see that some of the most important facts about differentiation of integrals and integration of derivatives can be obtained with a minimum of effort by first studying derivatives of measures and the associ ated maximal functions. The Radon-Nikodym theorem and the Lebesgue decom position will play a prominent role. Derivatives of Measures We begin with a simple theorem whose main purpose is to motivate the defini- tions that follow. 7.1 Theorem Suppose $\boldsymbol{\mu}$ is a complex Borel measure on $R^{1}$ and $$ f(x)=\mu(-\infty,\,x))\qquad(x\in R^{1}). $$ (1) $I f\times\in R^{1}$ and $\scriptstyle A$ is a complex number, each of the following two statements implies the other: (b)To every (a) fis differentiable at x and $\textstyle{f(x)=A}$ such that $\scriptstyle x\,>0$ corresponds a $\scriptstyle\delta>0$ $$ \left|\frac{\mu(I)}{m(I)}-A\right|<\epsilon $$ (2) 135136 REAL AND coMPLEX ANALYSIS for every open segment ${\mathbf I}$ that contains $\scriptstyle{\mathcal{X}}$ and whose length is less than . Here m denotes Lebesgue measure on $R^{1}.$ 7.2 Definitions Theorem 7.1 suggests that one might define the derivative of ${\boldsymbol{\mu}}$ at $\scriptstyle{\mathcal{X}}$ to be the limit of the quotients pu(ID/m(I), as the segments ${\mathbf I}$ shrink to ${\boldsymbol{x}},$ and that an analogous definition might be appropriate in several variables, i.e., in ${\boldsymbol{R}}^{k}$ rather than in $R^{1},$ Accordingly, let us fix a dimension $k,$ denote the open ball with center $x\in R^{n}$ and radius $\scriptstyle\gamma\simeq0$ by $$ B(x,\,r)=\{y\in R^{k}\colon|y-x|<r\} $$ (1) (the absolute value indicates the euclidean metric, as in Sec. 2.19), associate to any complex Borel measure ${\boldsymbol{\mu}}$ i on ${\boldsymbol{R}}^{k}$ the quotients $$ (Q_{r},\mu)(x)={\frac{\mu(B(x,r))}{m(B(x,r))}}, $$ (2) ofp ${\boldsymbol{\mu}}$ at $m=m_{k}$ is Lebesgue measure on $R^{k},$ and define the symmetric derivative where $\scriptstyle{\mathcal{X}}$ x to be $$ (D\mu)(x)=\operatorname*{lim}_{r\to0}\,(Q_{r}\mu)(x) $$ (3) at those points $x\in R^{4}$ at which this limit exists We shall study ${\cal{D}}\mu$ by means of the maximal function Mu. For $\mu\geq0,$ this is defined by $$ (M\mu)(x)=\operatorname*{sup}_{0<r<\alpha}(Q_{r}\mu)(x), $$ (4) and the maximal function of a complex Borel measure p is, by definition, that able. of its total variation $\mid\mu\mid.$ o] are lower semicontinuous, hence measur- and fix $x\in E$ The functions $M\mu\colon R^{k}\to[0,$ To see this, assume $\scriptstyle\nu\simeq0$ such that pick $\scriptstyle\lambda\,>0$ let $E=\{M\mu>\lambda\},$ Then there is an $\mu\geq0,$ $$ \mu(B(x,\,r))=t m(B(x,\,r)) $$ (5) for some $t>\lambda_{\mathrm{{,}}}$ and there is a $\scriptstyle\delta>0$ that satisfies $$ (r+\delta)^{k}<r^{k}t/\lambda. $$ (6) If $|y-x|<\delta,$ then $B(y,r+\partial)\supseteq B(x,r),$ and therefore A(B(y, r + の))≥ tm(B(x, r)) = t[r/(r + )]*m(B(y, r + 6))> 九m(B(y, r + 6)). Thus $B(x,\vartheta)\subset E.$ This proves that $\boldsymbol{E}$ is open. Our frst objective is the“maximal theorem”7.4. The following covering lemma will be used in its proof.DIFFERENTIATION 137 7.3 Lemma If $\bar{a}$ set $S\subset\{1,\ldots,N\}$ so that $B(x_{i},\,r_{i}),\,1\leq i\leq N,$ $\textstyle W$ is the union of a finite collection of balls then there is (a) theballs $B(x_{i},\,r_{i})\,w i t h\,i\in S$ are disjoint, (b) $W\subset\bigcup_{i\in S}$ B(x;,3r), and (c) m(W)≤ 3* EmB(x, r) ieS PRooF Order the balls $B_{i}=B(x_{i},\,r_{i})$ so that $r_{1}\geq r_{2}\geq\mathbf{\nabla}\cdot\mathbf{\nabla}\geq r_{N}.$ let $B_{i_{3}}$ be the first if Discard all there are any. Discard al $B_{j}$ with $j>i_{2}$ that intersect $B_{i_{2}}.$ Put $i_{1}=1.$ $B_{j}$ that intersect $B_{i_{1}}.$ Let $B_{i_{2}}$ be the first of the remaining $B_{j,\cdot}$ of the remaining ones, and so on, as long as possible. This process stops after some a finite number of steps and gives $S=\{i_{1},i_{2},\ldots\}.$ is a subset of $\scriptstyle{B x_{*}}$ $3r_{i})$ for $\scriptstyle i\in S_{*}$ It is clear that $\mathbf{\tau}_{(a)}$ holds. Every discarded intersects $B(x,\,r),$ then $B(x^{\prime},\,r^{\prime})\subset B(x,\,3,$ r) for if $\gamma_{\leq}\gamma$ and $B(x^{\prime},\,r^{\prime})$ $B_{j}$ This proves (b), and (c) follows from (b) because in $R^{k}.$ $$ m(B(x,\,3r))=3^{k}m(B(x,\,r)) $$ // The following theorem says, roughly speaking, that the maximal function of a measure cannot be large on a large set. 7.4 Theorem If p is $\bar{a}$ complex Borel measure on ${\boldsymbol{R}}^{k}$ and $\lambda$ is a positive number then $$ m\{M\mu>\lambda\}\leq3^{k}\lambda^{-1}\|\mu\|. $$ (1) Here $\|\mu\|=|\mu|(R^{k}),$ and the left side of (1) is an abbreviation for the more cumbersome expression $$ m(\{x\in R^{k}\colon(M\mu)(x)>\lambda\}). $$ (2) We shall often simplify notation in this way PROOF Fix ${\boldsymbol{\mu}}$ and $\lambda_{*}$ Let ${\cal K}\,\,\,\,$ be a compact subset of the open set $\{M\mu>\lambda\}$ Each $x\in K$ is the center of an open ball $\boldsymbol{B}$ for which $$ |\mu|(B)>\lambda m(B). $$ Some finite collection of these $B^{\mathrm{g}}$ covers $K,$ and Lemma $7.3$ gives us a dis joint subcollection, say $\{B_{1},\dots,B_{n}\};$ that satisfes $$ m(K)\leq3^{k}\sum_{1}^{n}m(B_{i})\leq3^{k}\lambda^{-1}\sum_{1}^{n}|\,\mu\,|\,(B_{i})\leq3^{k}\lambda^{-1}\|\,\mu\|. $$ The disjointness of $\{B_{1},\dots,B_{n}\}$ was used in the last inequality over all compact // Now(1)follows by taking the supremum $K\subset\{M\mu>\lambda\}$138 REAL AND coMPLEX ANALYsIs 7.5 Weak ${\boldsymbol{L}}^{1}$ If f ∈ $L^{1}(R^{k})$ and $\scriptstyle\lambda\,>0,$ then $$ m\{\left|\ f\right|>\lambda\}\leq\lambda^{-1}\left\|f\right\|_{1} $$ (1) because, putting $E=\{|f|>\lambda\},$ we have $$ \lambda m(E)\leq [{\frac{1}{\kappa}}f\mid d m\leq\int_{R^{k}}^{\infty}\mid f\mid d m=\Vert f\Vert_{1}. $$ (2) Accordingly, any measurable function f for which $$ \lambda\cdot m\{|f|>\lambda\} $$ (3) is a bounded function of A on (0, co)is said to belong to weak $L^{1}.$ Thus weak ${\boldsymbol{L}}^{1}$ contains $L^{1}.$ That it is actually larger is shown most simply by the function 1/x on (0,1). setting We associate to each $f\in L^{n}(R^{k})$ its maximal function Mf: $R^{k}{\overset{\underset{\mathrm{a}}{}}}\to[0,$ o0],by $$ (M f)(x)=\operatorname*{sup}_{0<r<\infty}{\frac{1}{m(B_{r})}}\bigcap_{B(x,r)}^{\circ}|f|\ d m. $$ (4) TWe wrote $B_{r}$ in place of $B(x,r)$ because $m(B(x,\,r))$ depends only on the radius $r_{\cdot}\rfloor$ If we identify $\boldsymbol{\f}$ with the measure ${\boldsymbol{\mu}}$ given by $d\mu=f\,d m,$ we see that $\mathbf{(4)}$ agrees with operator” the previously defined sends ${\boldsymbol{L}}^{1}$ to weak ${\boldsymbol{L}}^{1}$ , with a bound (namely 3") that depends only on $M\mu$ Theorem 7.4 states therefore that the“maximal $\textstyle{M}$ the space $R^{k}{\mathrm{:}}$ and every $\lambda>0.$ For every f∈ $L^{1}(R^{k})$ $$ m\{M f>\lambda\}\leq3^{k}\lambda^{-1}\|f\|_{1}. $$ (5) 7.6 Lebesgue points $\mathrm{If}f\in L^{1}(R^{k}),$ any $x\in R^{\prime}$ for which it is true tha $$ \operatorname*{lim}_{r\to0}\frac{1}{m(B_{r})}\left.\right|_{B(x,r)}|f(y)-f(x)|\ d m(y)=0 $$ (1) is called a Lebesgue point of f For example,(1) holds iff is continuous at the point x. In general, (1) means that the averages of $|f-f(x)|$ are small on small balls centered at $\scriptstyle X.$ The Lebesgue points of f are thus points where $\boldsymbol{\f}$ does not oscillate too much, in an average sense It is probably far from obvious that every $f\in L^{1}$ has Lebesgue points. But the following remarkable theorem shows that they always exist. (See also Exercise 23.) 7.7 Theorem Ife L(R"),.then almost every xe ${\boldsymbol{R}}^{k}$ is a Lebesgue point of fDIFFERENTIATION 139 PR0OF Define $$ (T_{r}\,f)(x)={\frac{1}{m(B_{r})}}\bigcap_{B(x,r)}|\,f-f(x)\,|\,d m $$ (1) for $x\in R^{k},r>0,$ and put $$ (T f)(x)=\operatorname*{lim}_{r\to0}\operatorname*{sup}\,(T_{r}f)(x). $$ (2) Pick We have to prove that $T f=0$ a.e.[m] $\scriptstyle{v\gg0}$ Let n be a positive integer. By Theorem 3.14,there exists $g\in C(R^{n})$ so that $\|f-g\|_{1}<1/n;{\mathrm{Put}}\,h=f-g.$ Since $\scriptstyle{\mathcal{G}}$ is continuous, $T g=0.$ Since $$ (T_{r}h)(x)\leq{\frac{1}{m(B_{r})}}\prod_{B(x,r)}|h|\ d m+|h(x)| $$ (3) we have $$ T h\leq M h+|h|\,. $$ (4) Since T, f≤ T, g + T, h,it follows that $$ T f\leq M h+|h|. $$ (5) Therefore $$ \{T f>2y\}\subset\{M h>y\}\cup\{|h|>y\}. $$ (6) Theorem Denote the union on the right of(6)by E(y, n) Since $\|h\|_{1}<1/n,$ $7.4$ and the inequality 7.5(1) show that $$ m(E(y,\,n))\leq(3^{k}+1)/(y n). $$ (7) The left side $\Phi({\mathfrak{g}})$ is independent of n. Hence $$ \{{\cal T}f>2y\}\subset\bigcap_{n=1}^{\infty}E(y,n). $$ (8) This intersection has measure ${\boldsymbol{0}},$ 0。 by (T), so that $\{T f>2y\}$ is a subset of a set of measure O. Since Lebesgue measure is complete $\{T f>2y\}$ is Lebesgue measurable, and has measure O. This holds for every positive y. Hence $T f=0$ a.e. [m] // Theorem 7 yields interesting information, with very little effort, about topics such as (a) differentiation of absolutely continuous measures, (b)differentiation using sets other than balls (c) differentiation of indefinite integrals in $R^{1}.$ (d) metric density of measurable sets140 REAL AND coMPLEX ANALYSIS We shall now discuss these topics the Radon-Nikodym derivative of ${\boldsymbol{\mu}}$ is a complex Borel measure on $R^{k},$ and $\mu\ll m.$ Let f be 7.8 Theorem Suppose $\boldsymbol{\mu}$ with respect to m.Then $D\mu=f$ a.e.[m], and $$ \mu(E)=\int_{E}(D\mu)\ d m $$ (1) for all Borel sets $E\in R^{n}$ In other words, the Radon-Nikodym derivative can also be obtained as a limit of the quotients $\scriptstyle0,\mu$ PROOF The Radon-Nikodym theorem asserts that (1) holds with fin place of DA. At any Lebesgue point $\scriptstyle{\mathcal{X}}$ of f, it follows that $$ f(x)=\operatorname*{lim}_{r\to0}{\frac{1}{m(B_{r})}}\int_{B(x,r)}^{\epsilon}f\,d m=\operatorname*{lim}_{r\to0}{\frac{\mu(B(x,\,r))}{m(B(x,\,r))}}. $$ (2) Thus (Dpu)Xx) exists and equals $\scriptstyle{I(x)}$ at every Lebesgue point of $f,$ hence a.e [m] // 7.9 Nicely shrinking sets Suppose $x\in R^{4}$ A sequence $\left\{\widehat{M}_{\widehat{k}}\right\}$ of Borel sets in $R^{k}$ is said to shrink to x nicely if there is a number $\scriptstyle\alpha\;>0$ with the following property There is a sequence of balls $B(x,\,r_{i}),$ with lim $r_{i}=0,$ such that $E_{i}\subset B(x,\,r_{i})$ and $$ m(E_{i})\geq\alpha\cdot m(B(x,\,r_{i})) $$ (1) for $i=1,2,3,\ldots$ is not required that $x\in E_{i},$ nor even that $\scriptstyle{\mathcal{X}}$ x be in the closure of Note that $\operatorname{it}$ $E_{i}.$ Condition (1) is a quantitative version of the requirement that each $\textstyle E_{i}$ must occupy a substantial portion of some spherical neighborhood of x. For example a nested sequence of k-cells whose longest edge is at most 1,000 times as long as its shortest edge and whose diameter tends to $\mathbf{0}$ shrinks nicely. A nested sequence of rectangles (in R) whose edges have lengths 1/i and $(1/i)^{2}$ does not shrink nicely. 7.10 Theorem Associate to each $x\in R^{4}$ a sequence {E(x)} that shrinks to $\scriptstyle{\mathcal{X}}$ nicely, and let f ∈ $L^{1}(R^{k}).$ Then $$ f(x)=\operatorname*{lim}_{i arrow\infty}\frac{1}{m(E_{i}(x))}\end{array}\biggr)_{E_{i}(x)}^{*}f\,d m $$ (1) at every Lebesgue point off, hence a.e. [m]DIFFERENTIATION 141 PxOOF Let x be a Lebesgue point of f and let α(x) and $B(x,r_{i})$ be the positive number and the balls that are associated to the sequence{E(x)}. Then, because $E_{i}(x)\subset B(x,\,r_{i}),$ $$ {\frac{\alpha(x)}{m(E_{i}(x))}}\prod_{\textstyle{E_{i}(x)}}^{\cdot}\left|f-f(x)\right|\,d m\leq{\frac{1}{m(B(x,\,r_{i}))}}\left[\bigcup_{B(x,\,r_{i})}f-f(x)\right|\,d m. $$ The right side converges to $\mathbf{0}$ ) as $i\to\infty,$ because $r_{i}\to0$ and $\scriptstyle{\mathcal{X}}$ is a Lebesgue point of f. Hence the left side converges to ${\boldsymbol{0}},$ D, and (1) follows. / {E(y), for different points x and $y.$ Note that no relation of any sort was assumed to exist between $\{E_{i}(x)\}$ and Note also that Theorem 7.10 leads to a correspondingly stronger form of Theorem 7.8. We omit the details 7.11 Theorem Iff ∈ $L^{1}(R^{1})$ and $$ F(x)=\int_{-\infty}^{x}f\,d m\ \ \ \ \ (-\infty<x<\infty), $$ then $F(x)=f(x)$ at every Lebesgue point off, hence a.e. [m] (This is the easy half of the fundamental theorem of Calculus, extended to Lebesgue integrals.) Theorem 7.10, with $E_{i}(x)=[x,\,x+\delta_{i}],$ be a sequence of positive numbers that converges to O. PR0OF Let $\{\delta_{i}\}$ shows then that the right-hand deriv- ative of ${\mathbf{}}F$ exists at all Lebesgue points of x of f and that it is equal to f(x) at these points. If we let $\scriptstyle{i k k}$ be $[x-\partial_{i},\cdot$ x] instead, we obtain the same result for the left-hand derivative of ${\mathbf{}}F$ at $\scriptstyle{X.}$ / 7.12 Metric density Let $\boldsymbol{E}$ be a Lebesgue measurable subset of $R^{k}.$ The metric density of $\boldsymbol{E}$ at a point $x\in R^{\prime}$ is defined to be $$ \operatorname*{lim}_{r\to0}{\frac{m(E\cap B(x,\,r))}{m(B(x,\,r))}} $$ (1) provided, of course, that this limit exists. If we let $\boldsymbol{\mathsf{f}}$ be the characteristic function of $\boldsymbol{E}$ and apply Theorem 7.8 or Theorem 7.10, we see that the metric density of $\boldsymbol{E}$ is $\mathbf{1}$ at almost every point of ${\boldsymbol{E}},$ and that it is $\mathbf{0}$ at almost every point of the complement of $\textstyle E.$ Here is a rather striking consequence of this, which should be compared with Exercise 8 in Chap $2\colon$ $E=R^{\prime}$ that satisfies $\;I\in>0,$ there is no set for every segment $$ \epsilon<\frac{m(E\cap I)}{m(I)}<1-\epsilon $$ (2) ${\boldsymbol{\jmath}}.$142 REAL AND COMPLEX ANALYSIS Having dealt with differentiation of absolutely continuous measures, we now turn to those that are singular with respect to m 7.13 Theorem Associate to each $x\in R^{4}$ a sequence {E(x)} that shrinks to $\scriptstyle{\mathcal{X}}$ nicely. $I f~\mu$ is a complex Borel measure and ${\boldsymbol{\mu}}$ L m, then $$ \operatorname*{lim}_{i arrow\infty}\frac{\mu(E_{i}(x))}{m(E_{i}(x))}=0\quad\mathrm{a.e.}\,\,[m]. $$ (1) PRoor The Jordan decomposition theorem shows that i sffces to prove (1) under the additional assumption that $\scriptstyle u\geq0$ . In that case, arguing as in the proof of Theorem 7.10, we have $$ {\frac{x(x)\mu(E_{i}(x))}{m(E_{i}(x))}}\leq{\frac{\mu(E_{i}(x))}{m(B(x,\,r_{i}))}}\leq{\frac{\mu(B(x,\,r_{i}))}{m(B(x,\,r_{i}))}}. $$ Hence $\mathbf{(1)}$ is a consequence of the special case $$ ({\cal D}\mu)(x)=0\quad\mathrm{a.e.~}[m], $$ (2) which will now be proved The upper derivative ${\tilde{D}}\mu,$ defined by $$ (\bar{D}\mu)(x)=\operatorname*{lim}_{n arrow\infty}\left[\operatorname*{sup}_{0<r<1/n}(Q_{r}\mu)(x)\right]~~~~~(x\in R^{k}) $$ (3) is a Borel function, because the quantity in brackets decreases as n increases and is, for each ${\mathfrak{n}},$ a lower semicontinuous function of $X{\dot{\boldsymbol{r}}}$ ; the reasoning used in Sec. 7.2 proves this. Choose $\lambda>0,\epsilon>0.$ Since ${\boldsymbol{\mu}}$ (Theorem 2.18) shows therefore that there is a measure O. The regularity of $\mu\perp\ m$ , $\boldsymbol{\mu}$ is concentrated on a set of Lebesgue Then compact set $K,$ with $m(K)=0,\mu(K)>\|\mu\|-\epsilon.$ $K,$ and put $\mu_{2}=\mu-\mu_{1}.$ Define $\mu_{1}(E)=\mu(K\cap E).$ for any Borel set $E\subset R^{k},$ $\|\mu_{2}\|<\epsilon,$ and,for every $\scriptstyle{\mathcal{X}}$ outside $$ (\bar{D}\mu)(x)=(\bar{D}\mu_{2})(x)\leq(M\mu_{2})(x). $$ (4) Hence $$ \{\bar{D}\mu>\lambda\}\subset K\;\cup\;\{M\mu_{2}>\lambda\}, $$ (5) and Theorem $7.4$ shows that $$ m\{\tilde{D}\mu>\lambda\}\leq3^{k}\lambda^{-1}\|\mu_{2}\|<3^{k}\lambda^{-1}\epsilon. $$ (6) Since(6) holds for every $\scriptstyle\epsilon\;>0$ and for every 入>0,we conclude that $\scriptstyle D_{\mu}=0$ a.e.[m], ie., that (2) holds. // Theorems 7.10 and 7.13 can be combined in the following wayDIFERENTIATION 143 7.14 Theorem Suppose that to each $x\in R^{4}$ is associated some sequence $\textstyle R^{k}.$ $(k_{i}(x))$ that shrinks to xnicely, and that $\boldsymbol{\mu}$ is a complex Borel measure on Let dp =f dm + $d\mu_{s}$ be the Lebesgue decomposition of p with respect to m. Then $$ \operatorname*{lim}_{i arrow\infty}\frac{\mu(E_{i}(x))}{m(E_{i}(x))}=f(x)\quad\mathrm{a.e.}\ [m]. $$ In particular, A L m if and only $i f(D\mu)(x)=0$ a.e.[m] The following result contrasts strongly with Theorem 7.13: 7.15 Theorem Ifp is a positive Borel measure on $R^{k}$ and p L m, then $$ (D\mu)(x)=\infty\quad\quad\alpha,\mathrm{e.tr}\quad\lceil\mu\rceil. $$ (1) radii $r_{i}=r_{i}(x),$ PRooF There is a Borel set $\textstyle s\in R^{\prime}$ with $m(S)=0$ and $\mu(R^{k}-S)=0,$ and For there are open sets $V_{j}\to S$ with $m(V_{j})<1/j,\mathrm{for}\,j=1,2,3,\ldots.$ to which correspond $N=1,\,2,\,3,\,\ldots,$ let $E_{N}$ be the set of all $x\in S$ with lim $\scriptstyle\nu_{i}=0,$ such that $$ \mu(B(x,\,r_{i}))<N m(B(x,\,r_{i})). $$ (2) Then (1) holds for every $x\in S-\bigcup_{N}E_{N}.$ tains $E_{N}$ and lies in $V_{j}.$ for the moment. Every $x\in E_{N}$ is then the center of a ball that con- $1/3$ Fix ${\boldsymbol{N}}$ and $j,$ be the open ball with center x whose radius is $w_{i,\cdot}$ $B_{x}\in V_{j}$ of that of $B_{x}.$ that satisfies (2). Let $\beta_{x}$ $\beta_{x}$ is an open set The union of these balls We claim that $$ \mu(W_{j,N})<3^{k}N/j. $$ (3) To prove (3),let $K\subset W_{i,N}$ be compact. Finitely many $F\subset E_{N}$ with the follow- $K.$ Lemma 7.3 shows therefore that there is a finite set $\beta_{x}$ cover ing properties: (a) $\{\beta_{x}\colon x_{\in}F\}$ is a disjoint collection, and 6 $K\subset\bigcup_{x\in F}B_{x}$ Thus Now put $\Omega_{N}=\bigcap_{i}W_{j,N}$ . Then $$ \begin{array}{c}{{\mu(K)\leq\sum_{x\not\in F}\mu(B_{x})<N\sum_{x\not\in F}m(B_{x})}}\\ {{\ }}\\ {{=3^{k}N\sum_{x\not\in F}m(\beta_{x})\leq3^{k}N m(V_{j})<3^{k}N/i.}}\end{array} $$ is a G。, $\mu(\Omega_{N})=0,$ and This proves (3). $E_{N}\subset\Omega_{N},$ $\Omega_{N}$ $(D\mu)(x)=\infty$ at every point o $S-\bigcup_{N}\Omega_{N}.$ //144 REAL AND coMPLEX ANALYSIS The Fundamental Theorem of Calculus 7.16 This theorem concerns functions defined on some compact interval [a, b] in R'. It has two parts. The first asserts, roughly speaking, that the derivative of the indefinite integral of a function is that same function. We dealt with this in Theorem 7.11. The second part goes the other way: one returns to the original function by integrating its derivative. More precisely $$ f(x)-f(a)=\left.\right|_{a}^{x}f^{\prime}(t)\;d t\qquad(a\leq x\leq b). $$ (1) In the elementary version of this theorem, one assumes that f is differentiable at every point of [a, b] and that $f^{\prime}$ is a continuous function. The proof of (1) is then easy. In trying to extend(1) to the setting of the Lebesgue integral, questions such as the following come up naturally: Is it enough to assume that f'e L, rather than that f' is continuous? If $\boldsymbol{\f}$ is continuous and differentiable at almost all points of [a, b], must (1) then hold? Before proving any positive results, here are two examples that show how (1) can fail (a)Pu $f(x)=x^{2}$ sin $\left(x^{-2}\right)$ if $x\neq0,\,f(0)=0.$ Then $\boldsymbol{\mathsf{f}}$ is differentiable at every point, but $$ \bigcap_{0}^{1}|f^{\prime}(t)|\;d t=\infty, $$ (2) SO $f^{\prime}\notin L^{1}.$ If we interpret the integral in(1)(with [0,1] in place of [a,b]) as the limit, as $\scriptstyle\epsilon\to0.$ of the integrals over [e, 1], then (1) still holds for this f. More complicated situations can arise where this kind of passage to the limit is of no use. There are integration processes, due to Denjoy and Perron (see [18] [28]), which are so designed that (1) holds whenever f is differentiable at every point. These fail to have the property that the integrability of $\boldsymbol{\mathit{f}}$ implies that of lf|,and therefore do not play such an important role in analysis. (b)Suppose $\boldsymbol{\mathit{f}}$ is continuous on [a,b], fis differentiable at almost every point of [α,b], and $f^{\prime}\in L$ on [a, b]. Do these assumptions imply that (1) holds? Answer: No. Choose $\{\delta_{n}\}$ so that $1=\delta_{0}>\delta_{1}>\delta_{2}>\cdots$ $\delta_{n} arrow0.$ Put $E_{0}=[0,$ 1]. Suppose $\scriptstyle n\,>0$ and $E_{n}$ is constructed so that $E_{n}$ is the union of $2^{n}$ disjoint of these $2^{n}$ intervals, so that each of the remaining $2^{n+1}$ Delete a segment in the center of each closed intervals, each of length $2^{-n}\delta_{n}.$ intervals has lengthDIFFERENTIATON 145 $2^{-n-1}\delta_{n+1}$ (this is possible, since $\delta_{n+1}<\delta_{n}),$ and let $E_{n+1}$ be the union of these $2^{n+1}$ intervals. Then $E_{1}\supset E_{2}\to\cdots,m$ (E,) = 6,,and if $$ E=\bigcap_{n=1}^{\infty}E_{n}, $$ (3) then $\boldsymbol{E}$ is compact and $m(E)=0.(\ln\operatorname{fact},E$ is perfect.) Put $$ g_{n}=\delta_{n}^{-1}\chi_{E_{n}}\quad\mathrm{and}\quad f_{n}(x)= \{_{0}^{x}g_{n}(t)\ d t\qquad(n=0,1,\,2,\,\ldots). $$ (4) Then $f_{n}(0)=0,f_{n}(1)=1,$ and each $\ f_{n}$ is a monotonic function which is con- intervals stant on each segment in the complement of $E_{n},$ .If ${\mathbf I}$ is one of the $2^{n}$ whose union is $E_{n},$ then $$ \int_{I}g_{n}(t)\ d t=\int_{I}g_{n+1}(t)\ d t=2^{-n}. $$ (5) It follows from (5) that $$ f_{n+1}(x)=f_{n}(x)\qquad(x\not\in E_{n}) $$ (6) and that $$ |f_{n}(x)-f_{n+1}(x)|\leq \vert_{I}|g_{n}-g_{n+1}|<2^{-n+1}\quad\quad(x\in E_{n}). $$ (7 Hence $\{f_{n}\}$ converges uniformly to a continuous monotonic function $m(E)=0,$ we have $\scriptstyle{f^{\prime}=0}$ ${\mathfrak{f}},$ with $f(0)=0,f(1)=1,$ and ${f^{\prime}}(x)=0$ for all x生E. Since a.e. Thus (1) fails If $\delta_{n}=(2/3)^{n};$ the set $\boldsymbol{E}$ is Cantor's“middle thirds”set. Having seen what can go wrong, assume now that $f^{\prime}\in L^{1}$ and that (1) does hold. There is then a measure从defined by $d\mu=f^{\prime}$ dm.Since $\mu\ll m,$ Theorem 6.11 shows that there corresponds to each $\epsilon>0\;\;\mathrm{a}\;\;\delta>0$ so that than $|\mu|(E)<\epsilon$ whenever $\boldsymbol{E}$ $f(y)-f(x)=\mu((x,\,y))$ if is a union of disjoint segments whose total length is less it follows that the absolute ${\hat{\delta}}.$ Since $a\leq x<y\leq b,$ continuity of f, as defined below, is necessary for(1). Theorem 7.20 will show that this necessary condition is also sufficient. said to be absolutely continuous on ${\mathbf I}$ (briefy, $\boldsymbol{\f}$ defined on an interval $I=[a,b],$ is 7.17 Definition A complex function ${\mathfrak{f}},$ is AC on ${\boldsymbol{\mathit{I}}}\}$ if there corre- sponds to every $\scriptstyle\epsilon\;>0$ a $\scriptstyle\delta>0$ so that $$ \sum_{i=1}^{n}|f(\beta_{i}^{j}-f(x_{i})|<\epsilon $$ (1)146 REAL AND COMPLEX ANALYSIS for any $r_{\mathit{l}}$ and any disjoint collection of segments (α,,β,) .……,(α,, β,)in ${\mathbf I}$ whose lengths satisfy $$ \sum_{i=1}^{n}\,(\beta_{i}-\alpha_{i})<\delta. $$ (2) Such an fis obviously continuous: simply take $\scriptstyle n\;=\;1.$ interesting. That In the following theorem, the implication $(b)\to(c)$ is probably the most $(a)\to(c)$ without assuming monotonicity of f is the content of Theorem 7.20. 7.18 Theorem Let I= [a,b], let f: $\scriptstyle I\to R^{\dagger}$ be continuous and nondecreasing. Each of the following three statements about f implies the other two. (a) fis AC on ${\mathit{I}}.$ (b) f maps sets of measure $\mathbf{0}$ to sets of measure O. (c)f is differentiable a.e. on $I,f^{\prime}\in L.$ and $$ f(x)-f(a)=\left.\right|_{a}^{x}f^{\prime}(t)\;d t\qquad(\alpha\leq x\leq b). $$ (1) Note that the functions constructed in Example 7.16(b) map certain compact sets of measure O onto the whole unit interval Exercise 12 complements this theorem PROOF We will show that $(a)\to(b)\to(c)\to(a).$ Let D denote the o-algebra of all Lebesgue measurable subsets of $R^{1},$ Assume fis AC on ${\boldsymbol{I}},$ pick $\scriptstyle{E\in I}$ so that $E\in{\mathfrak{M}}$ and $m(E)=0.$ We have to show that f(E) ∈ ODt and $m(f(E))=0.$ Without loss of generality, assume that neither a nor E $\boldsymbol{\ b}$ lie in $\textstyle E.$ Choose $\scriptstyle\epsilon\;>0.$ Associate $\mathfrak{g}>0$ to $\boldsymbol{\f}$ and e, as in Definition 7.17. There is be the then an open set ${\mathbf{}}V$ with $m(V)<\delta,$ so that $E\subset V<I.$ Let (α, $,\,\,\beta_{i})$ disjoint segments whose union is $V.$ Then $\sum\left(\beta_{i}-\alpha_{i}\right)<\delta,$ and our choice of & shows that therefore $$ \sum_{i}\,(f(\beta_{i})-f(\alpha_{i}))\leq\epsilon. $$ (2) [Definition 7.17 was stated in terms of finite sums; thus (2) holds for every partial sum of the (possibly) infinite series, hence (2) holds also for the sum of the whole series, as stated.] Since $E\subset V,f(E)\subset(\oint\mathbb{C}f(x_{i}),f(\beta_{i})]$ The Lebesgue measure of this union is the left side of (). This says that f(E) is a subset of Borel sets of arbitrarily small measure. Since Lebesgue measure is complete, it follows that f(E) ∈ 领 and $m(f(E))=0.$ We have now proved that (a) implies (b)DIFFERENTIA TION 147 Assume next that (b) holds. Define $$ g(x)=x+f(x)\qquad(a\leq x\leq b). $$ (3) $\mathrm{If}$ the f-image of some segment of length $\eta+\eta^{\prime}.$ From this it follows easily that g $\textstyle\eta$ has length ${\boldsymbol{\eta}}^{\prime},$ then the g-image of that same segment has length satisfies (b), since f does. is an $F_{\sigma}$ Now suppose $E\subset I,\,E\in\mathbb{N}.$ Then $E=E_{1}\cup E_{0}$ where $m(E_{\alpha})=0$ and $E_{1}$ (Theorem 2.20). Thus $E_{1}$ is a countable union of compact sets, and so is $g(E_{1}),$ because $\scriptstyle{\mathcal{G}}$ is continuous. Since $\scriptstyle{\mathcal{G}}$ satisfies $(b),\,m(g(E_{0}))=0.$ Since $g(E)=g(E_{1})\cup g(E_{0}),$ we conclude: $\scriptstyle{\theta(t)}$ e 领R. Therefore we can define $$ \mu(E)=m(g(E))\qquad(E\subset I,\,E\in\Re). $$ (4) Since $\scriptstyle{\mathcal{G}}$ is one-to-one (this is our reason for working with $\scriptstyle{\mathcal{G}}$ rather than $f{\mathfrak{h}},$ disjoint sets in ${\mathbf I}$ have disjoint ${\mathfrak{g}}.$ -images. The countable additivity of ${\mathfrak{m}}$ shows therefore that ${\boldsymbol{\mu}}$ is a (positive, bounded) measure on DR. Also, ${\boldsymbol{\mu}}$ 《m, because g satisfies b). Thus $$ d\mu=h\;d m $$ (5) for some $h\in L^{l}(m),$ by the Radon-Nikodym theorem ) gives If $E=[a,x],$ then $g(E)=[g(a),g(x)],\mathrm{and}\ (5$ $$ g(x)-g(a)=m(g(E))=\mu(E)= .{\overline{{\mu}}}_{E}^{}\,d m=\ <_{a}^{x}h(t)\,\,d t. $$ If we now use (3), we conclude that $$ f(x)-f(a)= .{\frac{c}{a}}\,[h(t)-1]\ d{t}\quad\quad(x\leq x\leq b). $$ (6) Thus $f^{\prime}(x)=h(x)-1$ a.e. [m], by Theorem 7.11 We have now proved that (b) implies (c) The discussion that preceded Definition 7.17 showed that Cc) implies (a) /// 7.19 Theorem Suppose f $\scriptstyle I rightarrow R^{\prime}$ is AC, 1 = [a,b]. Define $$ F(x)=\operatorname*{sup}\sum_{t=1}^{N}|f(t)-f(t_{t-1})|\qquad(a\leq x\leq b) $$ (1) where the supremum is taken over all N ${\mathbf{}}N$ and over all choices of {t;} such that $$ a=t_{0}<t_{1}<\cdot\cdot\cdot<t_{N}=x. $$ (2) The functions F, $F+f,F$ -f are then nondecreasing and AC on I.148 REAL AND coMPLEX ANALYSIs [F is called the total variation function of f. Iffis any (complex) function on ${\boldsymbol{I}},$ AC or not, and $F(b)<\infty,$ then f is said to have bounded variation on ${\boldsymbol{I}},$ and $\scriptstyle{\vec{F}}(s)$ is the total variation of fon I. Exercise 13 is relevant to this.] PR0OF If (2) holds and $x<y\leq b$ , then $$ F(y)\geq|f(y)-f(x)|+\sum_{i=1}^{N}|f(t_{i})-f(t_{i-1})|. $$ (3) Hence $F(y)\geq{\bigl|}f(y)-f(x){\bigl|}+F(x).$ In particular $$ F(y)\geq f(y)-f(x)+F(x)\quad{\mathrm{and}}\quad F(y)\geq f(x)-f(y)+F(x). $$ (4) This proves that $F,F+f,F-J$ rare nondecreasing Since sums of two AC functions are obviously AC, it only remains to be proved that ${\mathbf{}}F$ F is $\mathrm{AC}$ on ${\boldsymbol{I}}.$ If $(a,\beta)\in I$ then $$ {\cal F}(\beta)-{\cal F}(x)=\mathrm{sup}\;\sum_{1}^{n}|f(t_{i})-f(t_{i-1})|, $$ (5) Now pick the supremum being taken over al $\{t_{i}\}$ that satisfy $\alpha=t_{0}<\cdot\cdot<t_{n}=\beta$ Note that $\sum\left(t_{i}-t_{i-1}\right)=\beta-\alpha.$ to f and cas in Definition 7.17, choose $\scriptstyle\epsilon\;:\;0.$ associate $\delta>0$ $(\alpha_{j},\,\beta_{j}).$ disjoint segments $(x_{j},\,\beta_{j})\subset I$ with $\sum\left(\beta_{j}-\alpha_{j}\right)<\delta,$ and apply(5)to each It follows that $$ \sum_{j}\left(F(\beta_{j})-F(x_{j})\right)\leq\epsilon, $$ (6) by our choice of 6. Thus ${\mathbf{}}F$ is AC on ${\boldsymbol{\mathit{I}}}.$ // We have now reached our main objective: 7.20 Theorem Iffis a complex function that is AC on $I=[a,b],$ then $\boldsymbol{\f}$ f is differentiable at almost all points of $I,f^{\prime}\in L^{1}(m),$ and $$ f(x)-f(a)=\left.\right|_{a}^{x}f^{\prime}(t)\;d t\qquad(a\leq x\leq b). $$ (1) PROOF It is of course enough to prove this for real ${\boldsymbol{f}}.$ Let ${\mathbf{}}F$ be its total variation function, as in Theorem 7.19, define $$ f_{1}={\frac{1}{2}}(F+f),\qquad f_{2}={\frac{1}{2}}(F-f), $$ (2) and apply the implication $(a)\to(c)$ of Theorem 7.18 to $f_{1}$ and $f_{2}\,.$ Since $$ f=f_{1}-f_{2} $$ (3) this yields (1) //DIFFERENTIATON 149 The next theorem derives (1) from a different set of hypotheses, by an entirely different method of proof. $f^{*}\operatorname{e}L$ 7.21 Theorem UJ f: [L, then $b]\to R^{1}$ is differentiable at every point of [a,b] and on [a, $,\,b\,],$ $$ f(x)-f(a)=\left.\right._{a}^{x}f^{\prime}(t)\,d t\qquad(a\leq x\leq b). $$ (1) Note that differentiability is assumed to hold at every point of [a,b]. PR0OF ${\mathrm{It}}$ is clear that it is enough to prove this for $x=b.$ Fixe> 0. [a,b] such that $\scriptstyle g>f$ and Theorem 2.25 ensures the existence of a lower semicontinuous function g on $$ \operatorname{li}_{a}^{(b)}g(t)\ d t<\int_{a}^{t_{b}}f^{\prime}(t)\ d t+\epsilon. $$ (2) Actually, Theorem 2.25 only gives $\scriptstyle g\leq f_{\cdot}$ but since $m([a,b])<\infty,$ define we can add a small constant to $\scriptstyle{\mathcal{G}}$ without affecting (2). For any $\scriptstyle n\;>\;0,$ $$ F_{\eta}(x)=\sum_{a}^{e x}g(t)\ d t-f(x)+f(a)+\eta(x-a)\qquad(a\leq x\leq b). $$ (3) Keep $\textstyle\eta$ fixed for the moment. To each $x\in[a,b)$ there corresponds a $\scriptstyle a_{\gg}\;0$ such that $$ g(t)>f^{\prime}(x)\quad{\mathrm{and}}\quad{\frac{f(t)-f(x)}{t-x}}<f^{\prime}(x)+\eta $$ (4) for all t ∈ (x,X+ 6,)since $\scriptstyle{\mathcal{G}}$ is lower semicontinuous and $g(x)>f^{\prime}(x).$ For any such t we therefore have $$ F_{\eta}(t)-F_{\eta}(x)=\left.\int_{x}^{t}g(s)\;d s-[f(t)-f(x)]+\eta(t-x)\right.\nonumber $$ $$ \Im\left(t-x\right)f^{\prime}(x)-(t-x)[f^{\prime}(x)+\eta]+\eta(t-x)=0 $$ Since $F_{\eta}(a)=0$ and $F_{\eta}$ is continuous, there is a last point $x\in[a,b]$ at which for $F_{\mu}(x)=0.$ If In any case, $F_{n}(b)\geq0.$ the preceding computation implies that $F_{\nu}(t)>0$ $t\in(x,b].$ $x<b,$ Since this holds for every $\scriptstyle n\geq0,$ (2) and (3) now give $$ f(b)-f(a)\leq\int_{a}^{b}g(t)\ d t<\int_{a}^{b}\!f^{\prime}(t)\ d t+\epsilon, $$ (5) and since e was arbitrary, we conclude that $$ f(b)-f(a)\leq\int_{a}^{b}f^{\prime}(t)\;d t. $$ (6)150 REAL AND coMPLEX ANALYSIS If $\boldsymbol{\f}$ satisfies the hypotheses of the theorem,so does -f; therefore (6) // holds with -fin place of f, and these two inequalities together give (1). Differentiable Transformations 7.22 Definitions Suppose $V\!$ is an open set in $R^{k}{}_{,}$ ${\mathbf{}}T$ maps ${\mathbf{}}V$ into $R^{k},$ and xeV. If there exists a linear operator $\scriptstyle A$ on ${\boldsymbol{R}}^{k}$ (i.e., a linear mapping of $R^{k}$ into $R^{k},$ as in Definition 2.1) such that $$ \operatorname*{lim}_{h\to0}{\frac{|T(x+h)-T(x)-A h|}{|h|}}=0 $$ (1) (where, of course, he R'), then we say that ${\mathbf{}}T$ is differentiable at x $x,$ , and define $$ T^{\prime}(x)=A. $$ (2) The linear operator $T^{\prime}(x)$ is called the derivative of ${\mathbf{}}T$ at x.(One shows easily that there is at most one linear A that satisfies the preceding require- ments; thus it is legitimate to talk about the derivative of T.) The term differ- ential is also often used for T"(x) The point of(1) is of course that the difference $h.$ is approximated by T"(x)h, a linear function of $T(x+h)-T(x)$ Since every real number $\scriptstyle{\dot{\alpha}}$ gives rise to a linear operator on $R^{1}$ (mapping When h to αh), our definition of $\scriptstyle T(x)$ coincides with the usual one when $k=1.$ $A\colon R^{k}\to R^{k}$ is linear, Theorem 2.20(e) shows that there is a number $\scriptstyle A(A)$ such that $$ m(A(E))=\Delta(A)m(E) $$ (3) for all measurable sets $E\subset R^{k}.$ Since $$ A^{\prime}(x)=A\qquad(x\in R^{k}) $$ (4) and since every differentiable transformation ${\mathbf{}}T$ can be locally approximated by a constant plus a linear transformation, one may conjecture that $$ {\frac{m(T(E))}{m(E)}}\sim\Delta(T^{\prime}(x)) $$ (5) for suitable sets $\boldsymbol{E}$ E that are close to $\scriptstyle{X_{\circ}}$ This will be proved in Theorem 7.24 and furnishes the motivation for Theorem 7.26. denoted by $\scriptstyle J_{t}(x)$ $\Delta(A)=|\operatorname*{det}\,A|$ was proved in Sec. 2.23. When ${\mathbf{}}T$ is differen- Recall that tiable at ${\boldsymbol{x}},$ the determinant of $T^{\prime}(x)$ is called the Jacobian of ${\mathbf{}}T$ at ${\boldsymbol{x}},$ and is Thus $$ \Delta(T^{\prime}(x))=|J_{T}(x)|\,. $$ (6) The following lemma seems geometrically obvious. Its proof depends on the Brouwer fxed point theorem. One can avoid the use of this theorem by imposingDIFERENTIATION 151 stronger hypotheses on ${\boldsymbol{F}},$ for example, by assuming that ${\mathbf{}}F$ is an open mapping But this would lead to unnecessarily strong assumptions in Theorem 7.26. 7.23 Lemma Let $S=\{x:|x|=1\}$ be the sphere in ${\boldsymbol{R}}^{k}$ that is the boundary of the open unit ball B = B(0,1) If F:B→ Rt is continuous, $0<\epsilon<1$ ,and $$ |F(x)-x|<\epsilon $$ (1) for al $x\in{\mathcal{S}}.$ then $F(B)\supset B(0,1-\epsilon).$ fore $a\neq F(x),$ PROOF Assume, to reach a contradiction, that some point if $x\epsilon\,s$ . Thus a is not in $\scriptstyle{P(3)}$ and there- is $a\in B(0,\;1-\epsilon)$ not in F(B).By (1), $|F(x)|>{\underline{{1}}}-\epsilon$ for every xe B. This enables us to define a continuous map $G\colon{\bar{B}} arrow{\bar{B}}$ by $$ G(x)={\frac{a-F(x)}{|a-F(x)|}}. $$ (2) Ifxe S, then $x\cdot x=|x|^{2}=1,$ so that $$ x\cdot(a-F(x))=x\cdot a+x\cdot(x-F(x))-1<|a|+\epsilon-1<0 $$ (3) This shows that $x\cdot G(x)<0,$ hence $x\neq G(x).$ If x∈B, then obviously $x\not\equiv G(x),$ simply because $G(x)\in S.$ Thus G fixes no point of ${\bar{B}},$ contrary to Brouwer's theorem which states that every continuous map of E $\bar{B}$ into $\bar{B}$ B has at least one fixed point. /// A proof of Brouwer's theorem that is both elementary and simple may be found on pp- 38-40 of“Dimension Theory”by Hurewicz and wallman Princeton University Press, 1948 7.24 Theorem If (a)V is open in R b)T: $V{\boldsymbol{\to}}R^{k}$ is continuous, and (c)T is diferentiable at some point xe V, then $$ \operatorname*{lim}_{r arrow0}\frac{m(T(B(x,\,r)))}{m(B(x,\,r))}=\Delta(T^{\prime}(x)). $$ (1) Note that T(B(x, r)) is Lebesgue measurable; in fact, it is ${\boldsymbol{\sigma}}\cdot$ -compact, because B(x, r) is o-compact and ${\mathbf{}}T$ is continuous PRoOF Assume, without loss of generality, that $\scriptstyle x\;=\;0$ and $T(x)=0.$ Put $4=T(0).$ on finite- The following elementary fact about linear operators dimensional vector spaces will be used: A linear operator $\scriptstyle A$ on ${\boldsymbol{R}}^{k}$ is one-to-152 REAL AND COMPLEX ANALYSIS one if and only if the range of $\scriptstyle A$ is all of $R^{k},$ R*. In that case, the inverse $4^{-1}$ of A is also linear Accordingly, we split the proof into two cases. CASE I A is one-to-one. Define $$ F(x)=A^{-1}T(x)\qquad(x\in V). $$ (2) Then $F^{\prime}(0)=A^{-1}T^{\prime}(0)=A^{-1}A=I,$ the identity operator. We shall prove that $$ \operatorname*{lim}_{r arrow0}\frac{m(F(B(0,\,r)))}{m(B(0,\,r))}=1. $$ (3) Since $T(x)=A F(x),$ we have $$ m(T(B))=m(A(F(B)))=\Delta(A)m(F(B)) $$ (4) Choose $\scriptstyle\epsilon\;>0$ Since $F(0)=0$ Hence (3) will give the desired result there exists a $\delta>0$ such that for every ball B, by $7.22(3).$ and $F(0)=I,$ $0<\vert x\vert<\delta$ implies $$ |F(x)-x|<\epsilon\left|x\right|. $$ (5) We claim that the inclusions $$ B(0,\,(1-\epsilon)r)\subset F(B(0,\,r))\subset B(0,\,(1\,+\epsilon)r) $$ (6) hold if $0<r<\delta$ The first of these follows from Lemma_ 7.23,applied to $\scriptstyle B(s,r)$ in place of $B(0,\,1),$ because $|F(x)-x|<\epsilon r$ for all $\scriptstyle{\mathcal{X}}$ with $\scriptstyle x\colon-p$ The implies second follows directly from (5) since $|F(x)|<(1+\epsilon)|x|\cdot\operatorname{II}$ is clear that (6) $$ (1-\epsilon)^{k}\leq{\frac{m(F(B(0,\,r)))}{m(B(0,\,r))}}\leq(1+\epsilon)^{k} $$ (T and this proves (3) CAsE I A is not one-to-one. In this case, $\scriptstyle A$ maps ${\boldsymbol{R}}^{k}$ into a subspace of lower dimension, i.e., into a set of measure O. Given $\scriptstyle\epsilon\;>_{0},$ there is therefore an $\scriptstyle n\,>0$ such that $m(E_{n})<\epsilon$ if $E_{n}$ is the set of all points in ${\boldsymbol{R}}^{k}$ whose distance from $A(B(0,1))$ is less than $\eta.$ Since $A=T^{\prime}(0),$ there is a $\delta>0$ such that $\scriptstyle{|x|<\delta|<\delta|}$ implies $$ |\,T(x)-4x|\leq\eta\,|\,x\,|\,. $$ (8) If $\gamma<\delta_{*}$ then $T(B(0,\,r))$ lies therefore in the set $\eta r.$ Our choice of $\textstyle\eta$ shows that $\boldsymbol{E}$ that consists of the points whose distance from Hence $4(B(0,\;r))$ is less than $m(E)<\epsilon r^{k}.$ $$ m(T(B(0,r)))<\epsilon r^{k}\qquad(0<r<\delta). $$ (9)DIFFERENTIATION 153 Since $r^{k}=m(B(0,r))/m(B(0,$ 1)), ) implies that $$ \operatorname*{lim}_{r arrow0}{\frac{m(E(0,r)))}{m(B(0,r))}}=0. $$ (10) This proves (1), since $\Delta(T^{\prime}(0))=\Delta(A)=0.$ // 7.25 Lemma Suppose $E\subset R^{k}.$ $m(E)=0,$ ${\mathbf{}}T$ maps I $\boldsymbol{E}$ into $R^{k},$ and $$ {\mathrm{im~sup}}{\frac{|T(y)-T(x)|}{|y-x|}}<\infty $$ for every $x\in E.$ as y tends to x within $\textstyle E.$ Then $m(T(E))=0$ that PRooF Fix positive integers $r_{\mathit{l}}$ a and ${\boldsymbol{p}},$ let $F=F_{n,p}$ be the set of all $\mathrm{ref}\,E\,$ such $$ |\,T(y)-\,T(x)|\leq n\,|\,y-x\,| $$ for all $y\in B(x,\,1/p)\cap E,$ , and choose $\scriptstyle x\;{\overset{\underset{\mathrm{p}}{}}{}}\;$ Since $m(F)=0,$ ${\mathbf{}}F$ can be covered into by balls $B_{i}=B(x_{i},\,r_{i}),$ where $x_{i}\in F,r_{i}<1/p,$ in such a way that E $m(B_{i})<\epsilon.$ (To do this, cover ${\mathbf{}}F$ by an open set ${\boldsymbol{W}}$ of small measure, decompose ${\boldsymbol{W}}$ disjoint boxes of small diameter, as in Sec. 2.19, and cover each of those that intersect 中 by a ball whose center lies in the box and in $F_{\bullet}$ ${\mathbf{}}F$ If xe Fo B, then |x;-X|<r < 1/p and $x_{i}\in F$ . Hence $$ |\,T(x_{i})-\,T(x)|\leq n\,|\,x_{i}-x\,|<n r_{i} $$ so that $T(F\ \cap B_{i})\subset B(T(x_{i}),n r_{i}).$ Therefore $$ T(F)\subset{\big\lfloor}\bigcup_{i}B(T(x_{i}),n r_{i}). $$ The measure of this union is at most $$ \sum_{i}m(B(T(x_{i}),n r_{i})=n^{k}\sum_{i}m(B_{i})<n^{k}\epsilon. $$ tion is measurable and $m(T(F))=0.$ Since Lebesgue measure is complete and e was arbitrary, it follows that $\scriptstyle{T(F)}$ To complete the proof, note that $\boldsymbol{E}$ is the union of the countable collec- $\{F_{n,\,p}\},$ // Here is a special case of the lemma: If ${\mathbf{}}V$ is open in ${\boldsymbol{R}}^{k}$ and T: $V{\boldsymbol{\to}}R^{k}$ is differentiable at every point of ${\mathit{V}},$ then ${\mathbf{}}T$ maps sets of measure C $\mathbf{0}$ to sets of measure O. We now come to the change-of-variables theorem 7.26 Theorem Suppose that (i) $X\subset V\subset R^{k},$ ${\mathbf{}}V$ is open, T: $V{ arrow}\,R^{k}$ is continuous;154REAL AND cOMPLEX ANALYSIS (i) $X$ is Lebesgue measurable, ${\mathbf{}}T$ is one-to-one on $X,$ and ${\mathbf{}}T$ is differentiable a every point of $X{\dot{\boldsymbol{r}}}:$ (iin $m(T(V-X))=0.$ Then, settin $Y=T(X),$ $$ \bigcap_{Y}f\,d m= \bigcap_{X}(f\circ T)|J_{T}|\,d m \rangle $$ (1) for every measurable f $R^{k}{\overset{\underset{\mathrm{a}}{}}}\to[0,\cdot]$ 00] The case $\scriptstyle x\;=\;y$ is perhaps the most interesting one. As regards condition ii), it holds, for instance, when $m(V-X)=0$ and ${\mathbf{}}T$ satisfies the hypotheses of Lemma 7.25 on $V-X.$ The proof has some elements in common with that of the implication $(b)\to(c)$ in Theorem 7.18. It will be important in this proof to distinguish between Borel sets and Lebesgue measurable sets. The o-algebra consisting of the Lebesgue measurable subsets of $R^{k}$ will be denoted by D PROOF We break the proof into the following three steps: (D If E e OR and $E\subset V,\,t h e n\,T(E)\in\mathbb{N}.$ (II) For every $\boldsymbol{E}$ E e , $$ m(T(E\cap X))=\int_{X}\chi_{E}\mid J_{T}\mid\,d m. $$ (II) For every $A\in\Re,$ $$ \bigcap_{Y}\chi_{A}~d m= \lceil_{X}(\chi_{A}\circ T)\mid J_{T}\mid d m. $$ because ${\mathbf{}}T$ $E_{0}\in\Re,\;E_{0}\subset V,$ and $m(E_{0})=0,$ is $\textstyle{\boldsymbol{\sigma}}$ r-compact, hence $F_{\sigma}$ and a set of measure O by Gii), and If $E_{1}\subset V$ is an $F_{\sigma}.$ ,then $E_{1}$ $T(E_{1})\in{\mathfrak{M}}.$ then $m(T(E_{0}-X))=0$ is ${\boldsymbol{\sigma}}\cdot$ -compact, $m(T(E_{0}\cap X))=0$ by Lemma 7.25. Thus $m(T(E_{0}))=0.$ is continuous. Thus $\scriptstyle T(E_{\circ})$ Since every $E\in{\mathfrak{D}}$ is the union of an (Theorem 2.20), D is proved To prove (ID, let r ${\mathbf{}}n$ 1 be a positive integer, and put $$ V_{n}=\{x\in V\colon|\,T(x)|<n\},\qquad X_{n}=X\,\cap\,V_{n}. $$ (2) Because of (I), we can define $$ \mu_{n}(E)=m(T(E\cap X_{n}))\qquad(E\in{\mathfrak{M}}). $$ (3) Since ${\mathbf{}}T$ is one-to-one on $X_{n}.$ the countable additivity of m shows that ${\boldsymbol{\mu}}_{n}$ is a measure on D.Also, $X_{n}),$ and $\mu_{n}\ll m,$ by another application of Lemma_ 7.25 $\textstyle X{\ ~}$ temporarily by $\mu_{n}\,$ is bounded (this was the reason for replacingDIFFERENTIATION 155 and that Theorem 7.8 tells us therefore that $(D\mu_{a})(x)$ exists a.e. [m], that $D\mu_{n}\in L^{1}(m),$ $$ \mu_{n}(E)= \{_{E}(D\mu_{n})\;d m\qquad(E\in\Re). $$ (4) We claim next that $$ ({\cal D}\mu_{n})(x)=|J_{T}(x)|\,\qquad(x\in X_{n}). $$ (5) $\scriptstyle r\gg0,$ To see this, fix $x\in X_{n},$ and note that $B(x,\,r)\subset V_{n}$ for all sufficiently small because ${\mathit{V}}_{n}$ is open. Since $V_{n}-X_{n}\subset V-X,$ hypothesis Gii) enables us to replace $X_{n}$ by ${\mathit{V}}_{n}$ in (3) without changing $\mu_{n}(E).$ Hence, for small $\scriptstyle\gamma\leq0.$ $$ \mu_{n}(B(x,\,r))=m(T(B(x,\,r))). $$ (6) ${\mathrm{If~}}$ we divide both sides of (6) by $\scriptstyle m(B(x,\,r))$ and refer to Theorem 7.24 and formula 7.22(6), we obtain(5). Since (3) implies that $\mu_{n}(E)=\mu_{n}(E\cap X_{n}),$ it follows from (3),(4), and (5) that $$ m(T(E\cap X_{n}))=\int_{X_{n}}\chi_{E}|J_{T}|\;d m\qquad(E\in\Re). $$ (7)) If we apply the monotone convergence theorem to(7),leting n→OO, we obtain (II) We begin the proof of (IID by letting $\scriptstyle A$ be a Borel set in $R^{k},$ Put $$ E=T^{-1}(A)=\{x\in V\colon T(x)\in A\}. $$ (8) Then $\chi_{E}=\chi_{A}\circ T.$ Since $\chi_{A}$ is a Borel function and ${\mathbf{}}T$ is continuous $\chi_{E}$ is a Borel function (Theorem 1.12), hence $\boldsymbol{E}$ e Dt. Also $$ T(E\cap X)=A\cap Y $$ (9) which implies, by (II), that $$ \int_{Y}\!\chi_{A}~d m=m(T(E\cap X))=\int_{X}(\chi_{A}\circ T)\,|\,J_{T}|~d m. $$ (10) Finally,f $N\in{\mathfrak{M}}$ and $m(N)=0,$ there is a_Borel set $A\supseteq N$ with $m(A)=0.$ For this $A,$ (10) shows that $(\gamma_{A}\circ T)|J_{T}|=0$ a.e.[m]. Since $0\leq$ $\chi_{N}\leq\chi_{A},$ it follows that both integrals in(10) are $\mathbf{0}$ if $\scriptstyle A$ is replaced by $N.$ Since every Lebesgue measurable set is the disjoint union of a Borel set and a set of measure O,(10) holds for every A ∈D. This proves (II). Once we have ([II),it is clear that (1) holds for every nonnegative Lebesgue measurable simple function $f.$ Another application of the monotone convergence theorem completes the proof. ///156 REAL AND CoMPLEX ANALYSIs Note that we did not prove that $f\circ T$ is Lebesgue measurable for all Lebesgue measurable f. It need not be; see Exercise 8. What the proof does estab- lish is the Lebesgue measurability of the product $(f\circ T)|J_{T}|$ Here is a special case of the theorem: Suppose p:[a,b]→[α,β] is $\mathrm{AC},$ monotonic, $\varphi(a)=x,\;\varphi(b)=\beta,$ andf≥ 0 is Lebesgue measurable. Then $$ \bigcap_{x}^{\beta}f(t)\;d t= \lceil{\overset{b}{\operatorname{a}}}f(\varphi(x))\varphi^{\prime}(x)\;d x. $$ (15) the set of all $x\in V-\Omega$ To derive this from Theorem 7.26, put where $\varphi^{\prime}(x)$ exists (and is finite) let $\Omega$ be the union be of the maximal segments on which $\varphi$ $V=(a,\,b),\,T=\varphi,$ $\textstyle X{\ ~}$ is constant (if there are any) and let Exercises 1 Show th $\operatorname{at}|f(x)|\leq(M f)(x)$ at every Lebesgue point of fiff e L(R") 2 For $\delta>0,$ let I(6) be the segment $(-\delta,\delta)\in R^{1}.$ Given α and $\beta,0\leq\alpha\leq\beta\leq1,$ construct a measur- able set $E\subset R^{1}$ so that the upper and lower limits of $$ \frac{m(E\frown I(\delta))}{2\delta} $$ are $\boldsymbol{\beta}$ and $\alpha_{s}$ respectively, as $\delta{ arrow}{\bf0}.$ (Compare this with Section 7.12.) 3 Suppose that $\bar{E}$ is a measurable set of real numbers with arbitrarily small periods. Explicitly, this means that there are positive numbers $p_{i},$ converging to $\mathbf{\sigma}_{0}$ 0 as $\dot{\bar{\imath}}\longrightarrow\aleph\;$ 00, so that $$ E+p_{i}=E\qquad(i=1,\,2,\,3,\,\ldots). $$ Prove that then either $\scriptstyle{\vec{E}}$ or its complement has measure O Hint: Pick $\scriptstyle*\in K^{\circ}.$ put $F(x)=m(E\cap[\alpha,x])$ for $x>\alpha,$ show that $$ F(x+p_{i})-F(x-p_{i})=F(y+p_{i})-F(y-p_{i}) $$ if α + $p_{i}\prec x<y.$ What does this imply about $F(x)\operatorname{if}m(E)>0^{\prime}$ 4 Call t a period of the function $\boldsymbol{\mathit{f}}$ on $R^{1}{\mathrm{~if~}}f(x+t)=f(x)$ for al $\scriptstyle{x\cdot R^{\prime}}$ Suppose fis a real Lebesgue measurable function with periods s and t whose quotient ${\mathfrak{s}}/t$ is irrational. Prove that there is a con- stant $\textstyle{\mathcal{C}}$ such ${\mathrm{fhat}}f(x)=c{\mathrm{~a.}}$ but that f need not be constant Hint: Apply Exercise 3 to the sets $\{f>\lambda\}.$ and Let small, it ollows that $\scriptstyle A\quad}$ l intersects $B_{\epsilon}\,,$ so that $A+B=\{a+b;a\in A,\,b\in B\}$ Suppose $m(A)>0,$ $m(B)>0$ Prove S If $\scriptstyle{w e s t}$ and $B\subset R^{1},$ define that $A+B$ $\scriptstyle{\bar{\mathbf{C}}}$ There are points $a_{0}$ and $b_{\mathrm{o}}$ contains a segment, by completing the following outline. have metric density 1. Choose a small $\delta>0.$ Put where $\scriptstyle A\quad}$ and $\bar{\boldsymbol{B}}$ $|b-b_{0}|<\delta.$ Then For each e, positive or negative, define ${\boldsymbol{B}}_{c}$ to be the set of all $c_{0}+\epsilon-b$ for which $b\in B$ $c_{0}=a_{0}+b_{0}\,.$ $B_{\epsilon}\subset(a_{0}+\epsilon-\delta,$ $a_{0}+\epsilon+\delta).$ If $\delta$ was well chosen and |el is sufficiently $A\ +\ B\supseteq(c_{0}-\epsilon_{0_{}_{}}.c_{0}+\epsilon_{0})$ for some $\epsilon_{0}>0.$ $m(C)=0.$ ibe Cantors "middle thirds" set and show that $\scriptstyle c+c$ is an interval, although (See also Exercise 19, Chap. 9.) 6 Suppose $\boldsymbol{\mathit{G}}$ is a subgroup of $R^{1}$ (relative to addition), $\scriptstyle G\neq\mathbb{R};$ and ${\boldsymbol{G}}$ is Lebesgue measurable. Prove that then $m(G)=0.$ Hin: Use Exercise 5DIFFERENTIATION $1{\bar{S}}{\overline{{Y}}}$ 7 Construct a continuous monotonic function $\boldsymbol{\mathit{f}}$ f on $R^{1}$ sO ${\mathrm{that}}\,f$ is not constant on any segment although $f^{\prime}(x)=0$ a.e. union $\textstyle W$ is dense in ${\mathbf{}}V$ $W_{n},\,0<\varphi_{n}(x)<2^{-n}\operatorname{in}\,W_{n}.$ Put $\varphi=\sum{\mathrm{:}}$ Choose segments $W_{n}\subset V$ in such a way that their so that 8 Let $V=(a,\ b)$ be a bounded segment in $R^{1},$ $\varphi_{n}(x)=0$ and the set $K=V-W$ has $m(K)>0.$ Choose continuous functions $\varphi_{n}$ outside p。and define $$ T(x)=\left[\stackrel{x}{\omega}(t)\ d t\right.\qquad(a<x<b). $$ Prove the following statements (a) ${\mathbf{}}T$ satisfies the hypotheses of Theorem 7.26, with $X=V.$ (b) (c) If $\boldsymbol{E}$ is $\underline{{\land}}$ is continuous $T^{\prime}(x)=0$ on $K,$ $n(T(K))=0.$ ${\mathbf{}}T$ is an infinitely differentiable homeo- is Lebesgue ${\boldsymbol{T}}^{\prime}$ (d) The functions $\varphi_{n}$ nonmeasurable subset of ${\bar{\boldsymbol{K}}}$ (see Theorem 2.22) and $A=T(E),$ then $\zeta_{A}$ measurable but $\chi_{A}\circ T$ is not can be so chosen that the resulting morphism of ${\boldsymbol{y}}^{\prime}$ onto some segment in $R^{1}$ and $\mathbf{(c)}$ still holds. 9 Suppose $\scriptstyle0\,<\,\alpha\,<\,1$ Pick $\hat{\boldsymbol{I}}$ so that $t^{a}=2.$ Then t> 2, and the construction of Example ((b) in Sec. 7.16 can be carried out with $\delta_{n}=(2/t)^{n}$ .Show that the resulting function f belongs to Lip a on $\scriptstyle{|\mathbf{y},|\rangle}$ 10 If f e Lip l on [a, b], prove that fis absolutely continuous and t $\operatorname{hat}f^{\prime}\in L^{\infty}$ 11 Assume that $1<p<\infty,f]$ is absolutely continuous on [a, $b]_{*}f^{\prime}\in D,$ and $\alpha=1/q,$ where $\scriptstyle{\mathcal{A}}$ is the exponent conjugate to p ${\mathfrak{p}}.$ Prove that fe Lip α 12 Suppose $\varphi:\left[a,b\right]\to R^{1}$ is nondecreasing $a<x\leq b$ and $\epsilon>0$ then there is a $\delta>0$ so that (a) Show that there is a left-continuous nondecreasing $\scriptstyle f$ on [a,b] so that $\{f\neq\varphi\}$ is at most countable.[Left-continuous means:if $|f(x)-f(x-t)|$ <ewhenever $\scriptstyle0\,<\,t<\delta_{1}$ on [a,万 (b) Imitate the proof of Theorem 7.18 to show that there is a positive Borel measure ${}^{\mu}$ for which $$ f(x)-f(a)=\mu([a,\,x))\qquad(a\leq x\leq b). $$ (c) Deduce from (b) $\operatorname{that}f^{\prime}(x)$ exists a.e. [m], that f'e L(m) and that $$ f(x)-f(a)= .{\overline{{\mu}}}^{x}f^{\prime}(t)\,d t+s(x)\qquad(a\leq x\leq b) $$ where s is nondecreasing and $s(x)=\mathbb{C}$ a.e. [m] a.e. [m], and that $\mu\ll m{\mathrm{~if~}}$ and only iff is ${\mathrm{AC}}$ on (d) Show that $\mu\perp m$ if and only $\operatorname{if}f^{\prime}(x)=0$ [a, b] (e) Prove that $\varphi^{\prime}(x)=f^{\prime}(x)$ a.e. [m] 13 Let ${\mathfrak{B}}V$ be the class of al $\boldsymbol{\mathit{f}}$ on [a,b] that have bounded variation on [a, b], as defined after Theorem 7.19. Prove the following statements (a) Every monotonic bounded function on [α,b] is in $B V.$ (b) If f∈ ${\mathfrak{B}}V$ is real, there exist bounded monotonic functions f $f_{1}$ i anc ${\mathfrak{I}}f_{2}\operatorname{so}\operatorname{that}f=f_{1}-f_{2}$ Hint: Imitate the proof of Theorem 7.19. (c) If fe ${\mathfrak{B}}V$ is left-continuous thenJ $f_{1}$ and $f_{2}$ can be chosen in(b) so as to be also left continuous (d) Iffe BV is left-continuous then there is a Borel measure p on [a,b] that satisfies $$ f(x)-f(a)=\mu([a,\,x))\qquad(a\leq x\leq b). $$ A < mif and only if f is ${\mathrm{AC}}$ on [a,b] $\operatorname{and}f^{\prime}\in L^{(}(m).$ (e)Every fe ${\mathfrak{B}}V$ is differentiable a.e. [m] 14 Show that the product of two absolutely continuous functions on [α,bJ is absolutely continuous Use this to derive a theorem about integration by parts.158 REAL AND CoMPLEX ANALYSIs 15 Construct a monotonic function fon ${\boldsymbol{R}}^{1}$ l so that $f^{\prime}(x)$ exists (finitely) for every $x\in R^{1},$ but f' is not a continuous function. 16 Suppose $E\subset[a,\,b],\,m(E)=0.$ $x\in E.$ Construct an absolutely continuous monotonic function f on [a, b] so that f'(x)= 0o at every sets. Hint: $E\subset\bigcap V_{n},\,V_{n}$ open, $m(V_{n})\subset2^{-n}.$ Consider the sum of the characteristic functions of these $17$ Suppose $\left\{\mu_{\eta}\right\}$ is a sequence of positive Borel measures on ${\boldsymbol{R}}^{k}$ t and $$ \mu(E)=\sum_{n=1}^{\infty}\mu_{n}(E). $$ Assume decompositions of the $\mu_{n}\,$ ,and that of $\mu{\dot{Y}}$ is a Borel measure. What is the relation between the Lebesgue $\mu(R^{k})<\alpha.$ Show that ${}_{\!\mu}$ Prove that $$ ({\cal D}\mu)(x)=\sum_{n=1}^{\infty}({\cal D}\mu_{n})(x)\quad{\mathrm{a.e.~}}[m]. $$ their sumsf=E Derive corresponding theorems for sequences $\{f_{n}\}$ of positive nondecreasing functions on $R^{1}$ and 18I et gpot) = 1 on [O,1) $\varphi_{0}(t)=-1$ on $\scriptstyle{1,23}$ extend $\varphi_{0}$ to $R^{1}$ so as to have period $2,$ and define 9m $\stackrel{\leftrightarrow}{\leftrightarrow}_{n\,t r i t}^{\infty}\stackrel{ rightarrow}{\prime}\!\!\!\!\rightarrow\,\stackrel{ rightarrow}{\rightarrow}\nu^{(s)\,\nu^{*}}\,\stackrel{ rightarrow}{\sim}1\,{\stackrel{\star\nu^{*}}{\sim}}\,{\stackrel{.}{\sim}}\,$ o and prove that the series Assume that $\textstyle\sum|c_{n}|^{2}<c$ $$ {\underset{n=1}^{\infty}{c_{n}}}\phi_{n}(t) $$ (*) converges then for almost every t (±c,) converges with probability 1. then, for $n>N,$ $L^{2}.$ If Suggestion: Probabilistic interpretation: The series $\Sigma_{}^{}$ $s_{N}=c_{1}\varphi_{1}+\cdot\cdot\cdot+c_{N}\varphi_{N},$ $\{\varphi_{n}\}$ is orthonormal on [O,1], hence(*) is the Fourier series of some fe $a=j\cdot2^{-N},b=(j+1)\cdot2^{-N},a<t<b,$ and $$ s_{N}(t)=\frac{1}{b-a}\int_{a}^{b}s_{N}\,d m=\frac{1}{b-a}\int_{a}^{b}s_{n}\,d m, $$ Lebesgue point of f. and the last integral converges to ef dm, as $n arrow\infty.$ Show that $\scriptstyle{\binom{\omega}{\bullet}}$ converges to f(t) at almost every 19 Suppose fis continuous on $R^{1},f(x)>0\,\mathrm{if}\,0<x<1,f(x)=0\,\mathrm{o}$ therwise. Define $$ h_{\epsilon}(x)=\operatorname*{sup}\;\{n^{\epsilon}f(n x);\;n=1,\,2,\,3,\,\ldots\}. $$ Prove that (b) $\quad h_{1}$ is in weak ${\boldsymbol{L}}^{1}$ $L^{1}(R^{1})\operatorname{if}_{-}0<c<1$ $L^{1}$ 1f $[c>1.$ (a) $h_{\mathrm{{c}}}$ is in but not in L(Rl), (c $h_{\mathrm{{c}}}$ is not in weak 20(a) For any set $E\subset R^{2},$ the boundary OE of $\bar{E}$ is Lebesgue measurable minus the interior whose of E. Show that radi are at least 1. Use (a) to show that 上 $\bar{E}$ is, by definition, the closure of $\bar{E}$ ${\boldsymbol{R}}^{2}$ (b) Suppose that $\bar{E}$ E is Lebesgue measurable whenever $m(\partial E)=0.$ $\bar{E}$ E isthe union of possibly uncountable collcion of closediscs i (c) Show that the conclusion of ${\mathfrak{(}}b{\mathfrak{)}}$ is true even when the radii are unrestricted. the relevant geometric property ? (d) Show that some unions of closed discs of radius l are not Borel sets. (See Sec.2.21.) ie Can discs be replaced by triangles, rectangles,arbitrary polygons, etc, all this? What isDIFFERENTIATION $159$ 21 Ifis a real function on [0,1] and $$ \gamma(t)=t+\mathrm{i}^{\prime}(t), $$ the length of the graph of fis, by definition, the total variation of y on [O, 1]. Show that this length is finite if and only $\operatorname{li}f\in B V.$ (See Exercise 13.) Show that it is equal to $$ \vert\bigcup_{0}^{1}{\sqrt{1+[f^{\prime}(t)]^{2}}}\;d t \rangle $$ if fis absolutely continuous 22(a) Assume that both $\boldsymbol{\mathit{f}}$ and its maximal function Mf are in $L^{1}(R^{k}).$ Prove that ${\mathrm{then}}\,f(x)=0$ a.e. [m] Hint: To every ${\mathrm{cother}}f\in L^{k}(R^{k})$ corresponds a constant $c=c(f)>0$ such that $$ (M f)(x)\geq c\left|x\right|^{-k} $$ whenever |xlis suffciently large $x\!\!\!/}^{-2}$ if O<x<玉f0x) = on the rest o $R^{1}.$ Then fe $L^{1}(R^{1}).$ Show that (b) Define f(x)=x-1(log $$ (M f)(x)\geq|2x\log\,(2x)|^{-1}\qquad(0<x<1/4) $$ so that JB(Mf Xx) $d x=\infty$ $(S F)(x),$ 23 The definition of Lebesgue poins, as made in Sec. 7.6,apliesto individual integrable functions, is one o these equivalence not to the equivalence classes discussed in Sec. 3.10. However, if $F\in L(R)$ classes, one may call a point $x\in R^{k}$ a Lebesgue point of ${\mathbf{}}F$ if there is a complex number, Iet us call it such that $$ \operatorname*{lim}_{r\to0}\frac{1}{m(B_{r})}\left.\right|_{B(x,r)}|f-(S F)(x)\,|\,\,d m=0\,. $$ point of Define for one (hence for every)J e ${\boldsymbol{F}}.$ Hence $\scriptstyle{\theta\cdot\varepsilon}$ that are not Lebesgue points of ${\boldsymbol{F}}.$ is also a Lebesgue $(S F)(x)$ to be O at those points $x\in R^{k}$ is a Lebesgue point of f, then $\scriptstyle{\dot{\boldsymbol{x}}}$ Prove the follwing statement: $\operatorname{If}f\in F,$ and $\textstyle{\mathcal{X}}$ ${\boldsymbol{F}},$ and $f(x)=(S F)(x).$ Thus $\boldsymbol{\mathsf{S}}$ “ selects” a member of F that has a maximal set of Lebesgue points