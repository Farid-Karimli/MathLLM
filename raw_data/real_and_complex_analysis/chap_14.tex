CHAPTER FOURTEEN CONFORMAL MAPPING Preservation of Angles 14.1 Definition Each complex number $z\neq0$ determines a direction from the origin, defined by the point $$ A[z]={\frac{z}{|z|}} $$ (1) on the unit circle_ serves angles a $\mathbb{Z}_{\mathrm{O}}$ Suppose fis a mapping of a region $\Omega$ into the plane, $z_{i}\in\Omega,$ and $\mathbb{Z}_{\mathrm{O}}$ has a if $D(z_{0};r)<\Omega$ in which $f(z)\neq f(z_{0}).$ We say that $\boldsymbol{\f}$ pre- deleted neighborhood $$ \operatorname*{lim}_{r arrow0}e^{-i\theta}A[f(z_{0}+r e^{i\theta})-f(z_{0})]\qquad(r>0) $$ (2) exists and is independent of e. In less precise language, the requirement is that for any two rays $\boldsymbol{\mathit{L}}$ and $L^{\prime},$ the same as that made by $\boldsymbol{\mathit{L}}$ the angle which their images $\scriptstyle{\mathcal{I}}(t)$ and $f(L^{\prime})$ make at $\scriptstyle\{c_{\alpha}\}$ is starting at $\mathrm{z}_{\mathrm{0}}$ and $L^{\prime},$ in size as well as in orientation The property of preserving angles at each point of a region is character- istic of holomorphic functions whose derivative has no zero in that region This is a corollary of Theorem 14.2 and is the reason for calling holomorphic functions with nonvanishing derivative conformal mappings. $f^{\prime}(z_{0})$ 14.2 Theorem Let $\boldsymbol{\mathsf{f}}$ map a region $\Omega$ into the plane. If $f^{\prime}(z_{0})$ exists at some then $\scriptstyle z_{\mathrm{ne}}\in\Omega$ and $f^{\prime}(z_{0})\neq0,$ then f preserves angles at at z $\mathbb{Z}_{0}$ 6o,and iff preserves angles at $z_{0}\,,$ tial of f exists and is different from $\mathbb{Z}_{\mathrm{O}}$ . Conversely, if the differen- $\mathbf{0}$ exists and is different from O. 278CONFORMAL MAPPING 279 a linear transformation $\mathrm{Here}f^{\prime}(z_{0})=\mathrm{lim~}[f(z)-f_{.}(z_{0})]/(z-z_{0})$ , as usual. The differential of f at $\mathrm{z}_{0}$ is ${\boldsymbol{L}}$ of $R^{2}$ into $R^{2}$ such that, writing $z_{0}=(x_{0},y_{0}),$ f $$ (x_{0}+x,\,y_{0}+y)=f(x_{0},\,y_{0})+L(x,\,y)+(x^{2}+y^{2})^{1/2}\eta(x,\,y), $$ (1) where nx, $y){\bmod{-0}}$ as $x\to0$ and $y\to0,$ as in Definition 7.22. PROOF Take $z_{0}=f(z_{0})=0,$ for simplicity. If $f^{\prime}(0)=a\neq0,$ then it is imme- diate that $$ e^{-i\theta}A[f(r e^{i\theta})]=\frac{e^{-i\theta}f(r e^{i\theta})}{|f(r e^{i\theta})|}\to\frac{a}{|a|}\quad\quad(r\to0), $$ (2) so f preserves angles at O. Conversely, if the differential of f exists at $\mathbf{0}$ and is different from O, then (1) can be rewritten in the form $$ f(z)=\alpha z+\beta\bar{z}+|z|\eta(z), $$ (3) where $\scriptstyle\eta(z) arrow0$ as $\scriptstyle z\to0,$ and $\scriptstyle{\mathcal{X}}$ and $\boldsymbol{\beta}$ are complex numbers, not both O. If also preserves angles at ${\boldsymbol{0}},$ then $$ \operatorname*{lim}_{r arrow0}e^{-i\theta}A[f(r e^{i\theta})]=\frac{\alpha+\beta e^{-2i\theta}}{|\alpha+\beta e^{-2i\theta}|} $$ (4) exists and is independent of (.We may exclude those O for which the denominator in (4) is O; there are at most two such $\theta$ in [O,2z) For all other ${\boldsymbol{\theta}},$ we conclude that $x+\beta e^{-2i\theta}$ lies on a fixed ray through O, and this is // possible only when ${\boldsymbol{\beta}}=0.$ Hence $\alpha\neq0,$ and (3) implies that f'(0) = α. Note: No holomorphic function preserves angles at any point where its derivative is O.We omit the easy proof of this. However, the differential of a transformation may be $\mathbf{0}$ at a point where angles are preserved. Example: $\scriptstyle{f(t)\;=\;t}$ $|z|z,z_{0}=0.$ Linear Fractional Transformations 14.3 If a,b, c, and $d$ are complex numbers such that $a d-b c\neq0.$ the mapping $$ z arrow{\frac{a z+b}{c z+d}} $$ (1) is called a linear fractional transformation.It is convenient to regard (1) as a mapping of the sphere ${\boldsymbol{S}}^{2}$ into ${\boldsymbol{S}}^{2},$ with the obvious conventions concerning the point ${\mathcal{D}}.$ For instance, $-d/c$ maps to ${\mathcal{O}}$ and o maps to $a/c,$ if $\scriptstyle c\neq0$ It is then easy to see that each linear fractional transformation is a one-to-one mapping of ${\boldsymbol{S}}^{2}$ onto $S^{2}.$ . Furthermore, each is obtained by a superposition of transformations of the following types: (a) Translations: $z\to z+b.$ (b)Rotations: $z\to a z,\lfloor a\rfloor=1.$280 REAL AND coMPLEX ANALYSiS (c) Homotheties: $z\to r z,\;r>0.$ (d) Inversion: $z arrow\cdot1/z.$ If $\scriptstyle c=0$ in (1), this is obvious. If $\epsilon\neq0.$ it follows from the identity $$ {\frac{a z+b}{c z+d}}={\frac{a}{c}}+{\frac{\lambda}{c z+d}},\qquad\lambda={\frac{b c-a d}{c}}. $$ (2) The first three types evidently carry lines to lines and circles to circles. This is not true of (d). But if we let $\mathcal{F}$ be the family consisting of all straight lines and all circles, then $\textstyle{\mathcal{F}}$ F is preserved by (d), and hence we have the important result that $\mathcal{F}$ is preserved by every linear fractional transformation. [It may be noted that when 多 $\mathcal{F}$ is regarded as a family of subsets of ${\boldsymbol{S}}^{2},$ then。 ${\mathcal{M}}$ consists of all circles on $S^{2},$ , via the stereographic projection 13.1(1); we shall not use this property of $\mathcal{F}$ and omit its proof.T The proof that $\mathcal{F}$ F is preserved by inversion is quite easy. Elementary analytic geometry shows that every member of $\mathcal{F}$ is the locus of an equation $$ \alpha z{\bar{z}}+\beta z+{\bar{\beta}}{\bar{z}}+\gamma=0, $$ (3) where α and If $x\neq0,$ (3) defines a circle; $\scriptstyle x\;=\;0$ gives the straight lines. Replacement $\gamma$ are real constants and $\beta$ is a complex constant, provided that $\beta{\hat{\boldsymbol{\beta}}}>\alpha\gamma.$ of z by 1/z transforms (3) into $$ x+\beta\bar{z}+\bar{\beta}z+\gamma z\bar{z}=0, $$ (4) which is an equation of the same type. Suppose a ${\mathfrak{b}},$ and $\scriptstyle{\mathcal{C}}$ are distinct complex numbers. We construct a linear fractional transformation $\varphi$ which maps the ordered triple{a,b, c} into {0,1, 0o}, namely, $$ \varphi(z)={\frac{(b-c)(z-a)}{(b-a)(z-c)}} $$ (5) ator;if There is only one such $\varphi.$ For if $\varphi(a)=0,$ we must have $z-a$ in the numer- we are $\varphi(c)=\,\infty,$ we must have $z-c$ in the denominator; and if $\varphi(b)=1,$ led to (5).If a or ${\boldsymbol{b}}$ $\scriptstyle{\mathcal{C}}$ is oo,formulas analogous to (5) can easily be written down. If we follow(5) by the inverse of a transformation of the same type, we obtain the following result : For any two ordered triples {a,b, c} and $\langle a^{\prime},b^{\prime},c^{\prime}\rangle$ in ${\boldsymbol{S}}^{2}$ there is one and only one linear fractional transformation which maps a to ${\boldsymbol{a}}^{\prime},$ ${\boldsymbol{b}}$ to $b^{\prime}.$ , and cto c and c'.) (It is of course assumed that $a\neq b,\;a\neq c,$ and $b\neq c,$ and likewise for a,b', We conclude from this that every circle can be mapped onto every circle by a linear fractional transformation. Of more interest is the fact that every circle can be mapped onto every straight line (if o is regarded as part of the line), and hence that every open disc can be conformally mapped onto every open half planecONFORMAL MAPPING 281 Let us discuss one such mapping more explicitly, namely $$ \varphi(z)={\frac{1+z}{1-z}}. $$ (6) This $\varphi$ maps {-1,0,1} to {0,1, $\operatorname{cl}\colon$ the segment $(-1,\,1)$ maps onto the positive $\varphi(0)=1,$ real axis. The unit circle ${\mathbf{}}T$ passes through $-1$ and 1; hence $\varphi(T)$ is a straight line through $\varphi(-1)=0.$ makes a right angle with the real axis at O. Thus $\varphi(T)$ rnakes a right angle with the real axis at -1,p(T) Since ${\mathbf{}}T$ is the imaginary axis. Since it follows that $\varphi$ is a conformal one-to-one mapping of the open unit disc onto the open right half plane. The role of linear fractional transformations in the theory of conformal mapping is also well illustrated by Theorem 12.6. 14,4 Linear fractional transformations make it possible to transfer theorems con- cerning the behavior of holomorphic functions near straight lines to situations where circular arcs occur instead. It will be enough to illustrate the method with an informal discussion of the reflection principle Suppose $\Omega$ is a region in ${\boldsymbol{U}},$ UJ, bounded in part by an arc ${\boldsymbol{L}}$ on the unit circle and fis continuous on ${\hat{\Omega}},$ , holomorphic in $\Omega,$ and real on $L.$ The function $$ \psi(z)={\frac{z-i}{z+i}} $$ (1) maps the upper half plane onto $U.$ If $g=f\circ\psi,$ Theorem 11.14 gives us a holo ${\mathbf{}}F$ morphic extension ${\boldsymbol{G}}$ of ${\mathfrak{g}},$ and then $F=G\circ\psi^{-1}$ gives a holomorphic extension of f which satisfies the relation $$ f(z^{\ast})={\overline{{F(z)}}}, $$ (2) where $z^{*}=1/{\bar{z}}.$ $\psi\colon$ If $w=\psi(z)$ and $w_{1}=\psi({\vec{z}}),$ The last assertion follows from a property of then $w_{1}=w^{*},$ as is easily verified by computation Exercises 2 to 5 furnish other applications of this technique Normal Families The Riemann mapping theorem will be proved by exhibiting the mapping func tion as the solution of a certain extremum problem. The existence of this solution depends on a very useful compactness property of certain families of holo- morphic functions which we now formulate. 14.5 Definition Suppose ${\mathcal{F}}\subset H(\Omega),$ for some region Q2. We call ${\mathcal{F}}$ 莎 a normal family if every sequence of members of $\textstyle{\mathcal{F}}$ contains a subsequence which con verges uniformly on compact subsets of $\Omega.$ The limit function is not required to belong to $8{\mathcal{M}}\;,$282 REAL AND cOMPLEX ANALYSIS (Sometimes a wider definition is adopted,by merely requiring that every sequence in $\mathcal{F}$ either converges or tends to oo,uniformly on compact subsets of Q. This is well adapted for dealing with meromorphic functions.) 14.6 Theorem Suppose ${\mathcal{F}}\subset H(\Omega)$ and F is uniformly bounded on each compact subset of the region S2. Then ${\mathcal{F}}$ is a normal family. PRoOOF The hypothesis means that to each compact $\kappa\in\Omega$ there corresponds a number $M(K)<\infty$ such that $|f(z)|\leq M(K)$ for $\operatorname{all}f\in{\mathcal{F}}$ and all $\scriptstyle{\varepsilon\in K}$ $K_{n}$ Let $\{K_{n}\}$ be a sequence of compact sets whose union is $\Omega,$ such that lies in the interior of $K_{n+1};$ such a sequence was constructed in Theorem 13.3. Then there exist positive numbers ${\boldsymbol{\delta}}_{n}$ such that $$ D(z;2\delta_{n})\subset K_{n+1}\qquad(z\in K_{n}). $$ (1) positively oriented circle with center at $\mathbb{Z}^{\prime}$ such that $\mid z^{\prime}-z^{\prime\prime}\mid<\delta_{n}.$ $2\delta_{n}.$ and estimate Consider two points $\mathbb{Z}^{\prime}$ and $z^{\prime\prime}$ in $K_{n},$ let $\scriptstyle\gamma$ y be the and radius $|f(z^{\prime})-f(z^{\prime\prime})|$ by the Cauchy formula. Since $$ {\frac{1}{\zeta-z^{\prime}}}-{\frac{1}{\zeta-z^{\prime}}}={\frac{z^{\prime}-z^{\prime\prime}}{(\zeta-z^{\prime})(\zeta-z^{\prime\prime})}}, $$ we have $$ f(z^{\prime})-f(z^{\prime\prime})=\frac{z^{\prime}-z^{\prime\prime}}{2\pi i} \{\frac{f(\zeta)}{(\zeta-z^{\prime})(\zeta-z^{\prime\prime})}\,d\zeta, $$ (2) ity and since $|\zeta-z^{\prime}|=2\delta_{n}$ and $|\zeta-z^{\prime\prime}|>\delta_{n}$ for all $\zeta\in\gamma^{\bullet},\,(2)$ gives the inequal- $$ \vert f(z^{\prime})-f(z^{\prime\prime})\vert<\frac{M(K_{n+1})}{\delta_{n}}\vert z^{\prime}-z^{\prime\prime}\vert\,, $$ (3) valid for all f e JF and all $\mathbb{Z}^{\prime}$ ’and $z^{\prime\prime}\in K_{n},p r o v i d e d$ that $\mid z^{\prime}\ -z^{\prime\prime}\mid<\partial_{n}.$ that This was the crucial step in the proof: We have proved, for each $K_{n},$ an infinite set $\boldsymbol{\mathsf{S}}$ the restrictions of the members of $j=1,\,2,\,3,\,\cdot\cdot.$ Theorem 11.28 implies therefore that there $\{f_{j}\}$ COn- $-{\mathcal{M}}$ to $K_{n}$ form an equicontinuous family $\mathbb{H}\ f_{s}\in{\mathcal{F}},$ for of positive integers, $S_{1}\supset S_{2}\supset S_{3}\Rightarrow\cdots,$ so that (and hence // are infinite sets ${\boldsymbol{S}}_{n}$ as $j_{ arrow} arrow\infty$ within ${\boldsymbol{S}}_{n}$ . The diagonal process yields then $K_{n}$ verges uniformly on $K_{n}$ $\{f_{j}\}$ converges uniformly on every on every compact such that as $j\to\sigma$ within ${\boldsymbol{S}}.$ $K\subset\Omega)$ The Riemann Mapping Theorem lent if there exists a 14.7 Conformal Eaquivalence We call two regions such that $\varphi$ is one-to-one in $\Omega_{2}$ conformally equiva $\Omega_{2}$ $\Omega_{1}$ and $\varphi(\Omega_{1})=\Omega_{2},\;\mathrm{i}.\alpha,$ $\varphi\in H(\Omega_{4})$ $\Omega_{1}$ and such that if there exists a conformal one-to-one mapping of $\Omega_{1}$ ontocONFORMAL MAPPING 283 Under these conditions, the inverse of $\varphi$ is holomorphic in $\Omega_{2}$ (Theorem 10.33) and hence is a conformal mapping of $\Omega_{2}$ 2,onto $\Omega_{1}$ It follows that conformally equivalent regions are homeomorphic. But there If is a much more important relation between conformally equivalent regions: If $m(n_{\mathrm{e}})$ onto $H(\Omega_{1})$ which pre $\varphi$ $\Omega_{1}$ is as above, $f\to f\circ\varphi$ is a one-to-one mapping of onto $m({\mathfrak{a}}_{i})$ serves sums and products, i.e., which is a ring isomorphism of $m({\bf a}_{3})$ has a simple structure, problems about $\scriptstyle{m(\Delta_{3})}$ can be transferred to prob- lems in mapping function $\varphi$ and the solutions can be carried back to $\scriptstyle{m(s_{3})}$ with the aid of the $m(a_{i})$ p. The most important case of this is based on the Riemann mapping theorem (where $\Omega_{2}$ is the unit disc L $U),$ which reduces the study of $\scriptstyle n(v_{i})$ to the study of $H(U),$ for any simply connected proper subregion of the plane. Of course,for explicit solutions of problems, it may be necessary to have rather precise information about the mapping function. 14.8 Theorem Every simply connected region $\Omega$ in the plane (other than the plane itself)is conformally equivalent to the open unit disc ${\boldsymbol{U}}.$ Note: The case of the plane clearly has to be excluded, by Liouville's theorem. Thus the plane is not conformally equivalent to U, although the two regions are homeomorphic The only property of simply connected regions which will be used in the proof is that every holomorphic function which has no zero in such a region has a holomorphic square root there. This will furnish the conclusion $^{\circ}(j)$ ) implies (a)’ in Theorem 13.11 and will thus complete the proof of that theorem. PRooF Suppose $\Omega$ and which map $\Omega$ into ${\boldsymbol{U}}.$ We have to prove that some $\psi\in\Sigma$ complex number, Wo生 2. Let $\Sigma$ is a simply connected region in the plane and let $\psi\in H(\Omega)$ which are one- be a $\Omega$ be the class of all $w_{0}$ to-one in maps $\Omega$ onto $U.$ We first prove that $\Sigma_{}^{}$ is not empty. Since $\Omega$ is simply connected, there of define $\psi=r/(\varphi+a),$ The next step consists in showing that if $\psi\in\Sigma,$ if $\psi(\Omega)$ $\varphi(z_{1})=\varphi(z_{2}),$ then also with $-\,\varphi(z_{2}).$ $\varphi^{2}(z_{1})=\varphi^{2}(z_{2}),$ shows that there are no two distinct points $z_{1}$ in Q.If z in $\Omega$ such that $\varphi(z_{1})=$ exists a $\rho_{\circ}\in H(\Omega)$ So that $\varphi^{2}(z)=z-w_{0}$ is one-to-one. The same argument hence $z_{1}=z_{2};$ thus $\varphi$ $0<r<|a|$ Since $\varphi$ The disc $D(-a;r)$ Z」 and $z_{2}$ $D(a;r),$ is an open mapping,q(Q2) contains a disc therefore fails to intersect p(Q2),and if we ${\boldsymbol{U}},$ , and $|f z_{0}\in\Omega,$ we see that $\psi\in\Sigma.$ with does not cover all then there exists a $\psi_{1}\in\Sigma$ $$ |\psi_{1}{}^{\prime}(z_{0})|>|\psi^{\prime}(z_{0})|\,. $$ It will be convenient to use the functions $\varphi_{\alpha}$ defined by $$ \varphi_{\alpha}(z)={\frac{z-\alpha}{1-\bar{\alpha}z}}. $$284 REAL AND CoMPLEX ANALYSIs For α∈ U $\varphi_{\alpha}$ is a one-to-one mapping of $U$ onto $U$ ;its inverse is $\varphi_{-\alpha}$ (Theorem 12.4). zero in $\Omega;$ is one-to-one (as in the proof t ${\mathrm{hat}}\,\Sigma\neq\mathbb{C}).$ Then p。。y ∈E, and $\varphi_{\alpha}\circ\psi$ has no where Suppose $\psi\in\Sigma,\,\alpha\in U,$ and $x\notin\psi(\Omega).$ such that $g^{2}=\varphi_{\alpha}\circ\psi.$ We see that g $\beta=g(z_{0}),$ hence there exists a $g\in H(\Omega)$ hence $y\in\Sigma\colon$ and if $\psi_{1}=\varphi_{\beta}\circ g,$ it follows that $\psi_{1}\in\Sigma$ With the notation $w^{2}=s(w),$ we now have $$ \psi=\varphi_{-\alpha}\circ s\circ g=\varphi_{-\alpha}\circ s\circ\varphi_{-\beta}\circ\psi_{1}. $$ Since $\psi_{1}(z_{0})=0,$ the chain rule gives $$ \psi^{\prime}(z_{0})=F^{\prime}(0)\psi_{1}^{\prime}(z_{0}), $$ where ${\cal F}=\varphi_{-\alpha}\circ s\circ\varphi_{-\beta}.$ We see that $F(U)\subset U$ $\psi$ is one-to-one in Q.] is not one-to and that ${\mathbf{}}F$ one in ${\cal U}.$ Therefore $|F^{\prime}(0)|<1,$ by the Schwarz lemma (see Sec. 12.5), so that $|\psi^{\prime}(z_{0})|<|\psi_{1}^{\prime}(z_{0})|\,.$ [Note that $\psi^{\prime}(z_{0})\neq0,$ since Fix $\scriptstyle\mathrm{r_{o}}\mathrm{en}$ and put $$ \eta=\operatorname*{sup}\;\{\;|\,\psi^{\prime}(z_{0})|\!:\psi\in\Sigma\}. $$ The foregoing makes it clear that any $\scriptstyle n\in\Sigma$ for which $\left|\,h^{\prime}(z_{0})\,\right|\,=\,\eta$ will map $\Omega$ onto ${\boldsymbol{U}},$ Hence the proof will be completed as soon as we prove the existence of such an $h_{\mathrm{.}}$ Since $|\,\psi(z)\,|\,<1$ for all $\psi\in\Sigma$ and $\varepsilon\in\Omega$ Theorem 14.6 shows that $\Sigma$ is a such that normal family. The definition of $\textstyle\eta$ shows that there is a sequence $\{\psi_{n}\}$ in $\Sigma$ $|\,\psi_{n}^{\prime}(z_{0})\,|\to\eta,$ and by normality of $\Sigma_{}^{}$ we can extract a subsequence Since compact subsets of $\Omega,$ to a limit (again denoted by{yn},for simplicity)) which converges, uniformly on By Theorem 10.28, for $n=1,$ 2,3,.. $h\in H(\Omega).$ $\left|{h^{\prime}}(z_{0})\right|=\eta.$ we haye $\Sigma\neq\mathbb{Q}.$ $\eta>0,$ SO ${\boldsymbol{h}}$ is not constant. Since $\psi_{n}(\Omega)\subset U,$ $h(\Omega)\subset U,$ but the open mapping theorem shows that actually $\overline{{D}}$ that $h(\Omega)\subset U$ and $z_{2}\in\Omega;$ put $\alpha=h(z_{1})$ and $\alpha_{n}=\psi_{n}(z_{1})$ is one-to-one. Fix distinct and such points $h-\alpha$ So all that remains to be shown is that with center at ${\tilde{D}}.$ . This is possible, since the zeros $z_{1}$ has no zero on the boundary of ${\boldsymbol{h}}$ for $\scriptstyle n\;=\;1.$ 2,3,.…. nd let be a closed circular disc in $\Omega$ ${\mathfrak{Z}}_{2}\,.$ ,such that $\scriptstyle z_{*},\phi,D$ of $h-\alpha$ have no limit point in $\Omega$ The functions $\psi_{n}-\alpha_{n}$ converge to $h-\alpha,$ uniformly on ${\overline{{D}}}\colon$ ; they have no zero in $D\!\!\!\!/$ since they are one-to-one and have a zero at $z_{1}\,;$ it now follows from Rouché's theorem that $\scriptstyle n\,=\,a$ has no zero in ${\boldsymbol{D}};$ in particular, $h(z_{2})\neq h(z_{1}).$ // Thus $*\in\Sigma,$ and the proof is complete A more constructive proof is outlined in Exercise 26.cONFORMAL MAPPINiG 285 and $\beta\neq0,$ then 14.9 Remarks The preceding proof also shows that and $h(z_{0})=0.$ For if $h(z_{0})=\beta$ $\varphi_{\beta}\circ h\in\Sigma,$ $$ |(\varphi_{\beta}\circ h\rangle^{\prime}(z_{0})|=|\varphi_{\beta}^{\prime}(\beta)h^{\prime}(z_{0})|=\frac{|h^{\prime}(z_{0})|}{1-|\beta|^{2}}>|h^{\prime}(z_{0})|\,. $$ It is interesting to observe that although $\boldsymbol{h}$ was obtained by maximizing $|\psi^{\prime}(z_{0})|$ for $\psi\in\Sigma,\,h$ also maximizes $|f^{\prime}(z_{0})|$ if f is allowed to range over the (not necessarily class consisting of all holomorphic mappings of $\Omega$ into $U$ hence one-to-one). For iff is such a function, then $g=f\circ h^{-1}$ maps $U$ into ${\boldsymbol{U}},$ $|g^{\prime}(0)|\leq1.$ , with equality holding (by the Schwarz lemma) if and only if $\scriptstyle{\mathcal{G}}$ is a rotation, so the chain rule gives the following result: If fe $H(\Omega),f(\Omega)\subset U,$ and $\scriptstyle z_{\mathrm{e}}\in\Omega_{\mathrm{i}}$ then $|f^{\prime}(z_{0})|\leq|h^{\prime}(z_{0})|\,.$ Equality holds if and only $i f f(z)=\lambda h(z),$ for some constant 入 $w\bar{n}\vert i\vert\lambda\vert=1.$ The Class ${\mathcal{P}}$ 14.10 Definition ${\mathcal{P}}$ is the class of $\operatorname{all}f\in H(U)$ which are one-to-one in $U$ and which satisfy $$ f(0)=0,\;\;\;\;\;\;f^{\prime}(0)=1. $$ (1) Thus every f∈ ${\mathcal{P}}$ has a power series expansion $$ f(z)=z+\sum_{n=2}^{\infty}a_{n}z^{n}\qquad(z\in U). $$ (2) The class ${\mathcal{G}}$ is not closed under addition or multiplication, but has many other interesting properties. We shall develop only a few of these in this section. Theorem 14.15 will be used in the proof of Mergelyan's theorem, in Chap. 20. 14.11 Example $\mathbb{F}\left|\alpha\right|\leq1$ and $$ f_{\alpha}(z)={\frac{z}{(1-\alpha z)^{2}}}=\sum_{n=1}^{\infty}n\alpha^{n-1}z^{n} $$ thenf. e ${\mathcal{P}}.$ $f_{\alpha}(z)=f_{\alpha}(w),$ then $(z-w)(1-\alpha^{2}z w)=0,$ and the second factor is For if not When $0\mathrm{~if~}|z|<1\mathrm{~and~}|w|<1$ is called a Koebe function. We leave it as an exercise to $|\alpha|=1,f_{s}$ find the regions fxUD 14.12 Theorem (a) Iff ∈ ${\mathcal{I}},|\alpha|=1,$ and $g(z)=\bar{\alpha}f(\alpha z),\,t h e n\ g\in\mathcal{P}.(b)\,I f F\in\mathcal{P}$ there exists a ${\mathfrak{g}}\in{\mathcal{Y}}$ such that $$ g^{2}(z)=f(z^{2})\qquad(z\in U). $$ (1)286 REAL AND coMPLEX ANALYSIS and $h\in H(U)$ PROOF (a) is clear. To prove (b), write $f(z)=z\varphi(z).$ Then $\varphi\in H(U),\,\varphi(0)=1,$ $\varphi$ with has no zero in U, since f has no zero in $U-\{0\}.$ Hence there exists an $h(0)=1,\,h^{2}(z)=\varphi(z).$ Put $$ g(z)=z h(z^{2})\qquad(z\in U). $$ (2) Then $g^{2}(z)=z^{2}h^{2}(z^{2})=z^{2}\varphi(z^{2})=f(z^{2}),$ and $g(z)=g(w).$ Since $\boldsymbol{\f}$ is one-to-one,(1) implies $z=-w.$ // so that (1) holds. It is clear that that and since $\scriptstyle{\mathcal{G}}$ has no zero in We have to show that $\scriptstyle{\mathcal{G}}$ is one-to-one $g(z)=g(w)=0,$ $\theta(0)=0$ and $\theta^{\prime}(0)=1.$ $U$ (which is what we want to prove) or Suppose $\widetilde{\mathbb{Z}}$ and w ∈ $z=w$ it follows that $z^{2}=w^{2}.$ So either $g(z)=\,-\,g(w);$ $z=w=0.$ In the latter case, (2) shows that , we have $U-\{0\}$ 14.13 Theorem $I f\in H(U-\{0\}),F$ is one-to-one in ${\boldsymbol{U}},$ J, and $$ F(z)=\frac{1}{z}+\sum_{n=0}^{\infty}\alpha_{n}z^{n}\qquad(z\in U), $$ (1) then $$ \sum_{n=1}^{\infty}n\mid\alpha_{n}|^{2}\leq1. $$ (2) This is usually called the area theorem, for reasons which will become apparent in the proof. PROOF The choice of $\alpha_{0}$ is clearly irrelevant. So assume $\scriptstyle x_{0}=0$ Neither the hypothesis nor the conclusion is affected if we replace $\scriptstyle{P(t)}$ by $\lambda F(\lambda z)$ Put is one-to-one. Write So we may assume that $\alpha_{1}$ is real. and $V_{r}=\{z\colon r<|z|<1\},$ for ${\mathbf{}}F$ $[|\lambda|=1).$ $0<r<1.$ Then $\scriptstyle{\vec{F}}(t_{i})$ $U_{r}=\{z\colon|z|<r\},\ \,C_{r}=\{z\colon|z|=r\},$ ${\mathcal{O}}$ (by the open mapping is a neighborhood of theorem, applied to $1/F)\colon$ ; the sets $F(U_{r}),F(C_{r}),$ and $\scriptstyle{\vec{F}}(\nu_{j})$ are disjoint, since $$ F(z)=\frac{1}{z}+\alpha_{1}z+\varphi(z)\qquad(z\in U), $$ (3) $F=u+i v,$ and $$ A={\frac{1}{r}}+\alpha_{1}r,\qquad B={\frac{1}{r}}-\alpha_{1}r. $$ (4) For $z=r e^{\theta}.$ we then obtain $$ u=A\cos\theta+\mathrm{Re}\ \ \mathrm{and}\ \ \ v=\ -B\sin\theta+\mathrm{Im}\ \varphi. $$ (5)cONFORMAL MAPPING 287 Divide Eqs.(5) by A and ${\boldsymbol{B}},$ , respectively, square, and add: $$ {\frac{u^{2}}{A^{2}}}+{\frac{v^{2}}{B^{2}}}=1+{\frac{2\,\cos\,\theta}{A}}\,\,\mathrm{Re}\,\,\varphi+\left({\frac{\mathrm{Re}\,\,\varphi}{A}}\right)^{2}-{\frac{2\,\sin\,\theta}{B}}\,\,\mathrm{Im}\,\,\varphi+\left({\frac{\mathrm{Im}\,\,\varphi}{B}}\right)^{2}. $$ By (3),p has a zero of order at least 2 at the origin. If we keep account of (4) it follows that there exists an $\eta>0$ such that, for all sufficiently small ${\boldsymbol{r}},$ $$ \frac{u^{2}}{A^{2}}+\frac{v^{2}}{B^{2}}<1+\eta r^{3}\qquad(z=r e^{i\theta}). $$ (6) This says that F(C,) is in the interior of the ellipse $E_{r}$ whose semiaxes are $A{\sqrt{1+\eta r^{3}}}$ and $B{\sqrt{1+\eta r^{3}}},$ and which therefore bounds an area $$ \pi{\cal A}B(1+\eta r^{3})=\pi\biggl(\frac{1}{r}+\alpha_{1}r\biggr)\biggl(\frac{1}{r}-\alpha_{1}r\biggr)(1+\eta r^{3})\leq\frac{\pi}{r^{2}}\,(1+\eta r^{3}). $$ (T) Since $\scriptstyle{P(\zeta)}$ is in the interior of $E_{r},$ we have $E,<F(U_{r});$ hence $\scriptstyle{\vec{F}}(\nu_{j})$ is in the interior of $E_{r},$ ,so the area of $\scriptstyle{\vec{F}}({\boldsymbol{r}}_{i})$ is no larger than (7). The Cauchy- is Riemann equations show that the Jacobian of the mapping (x $y) arrow(u,v)$ $|F^{\prime}|^{*}$ . Theorem 7.26 therefore gives the following result: $$ \begin{array}{r l}{{\frac{\pi}{2}}(1+\eta r^{3})\geq\left[{\frac{1}{r}}\right]^{1}|F|^{2}}\\ {{}=\sum_{n}^{r}d t\int_{0}^{1}\left(t^{-3}+{\frac{\pi}{2}}n^{2}|x_{n}|^{2n-1}\right)d t}\\ {{}=\pi\left[r^{-2}-1+{\frac{\pi}{1}}n|x_{n}|^{2n-1}\right]^{2}d\theta}\end{array} $$ (8) If we divide (8) by z and then subtract $r^{-2}$ from each side, we obtain $$ \sum_{n=1}^{N}n|\,\alpha_{n}|^{2}(1-r^{2n})\leq1+\eta r $$ (9)) for all sufficiently small ${\mathbf{}}r$ and for all positive integers $N.$ Let $r\to0$ in (9), then let $N\to\alpha.$ This gives (2) // Corollary Under the same hypothesis, lαil≤1 which is one-to-one in ${\boldsymbol{U}}.$ That this is in fact best possible is shown by $F(z)=(1/z)+\alpha z,\ |\alpha|=1,$288 REAL AND coMPLEX ANALYSIs 14.14 Theorem Iff∈ J9, and $$ f(z)=z+\sum_{n=2}^{\infty}a_{n}z^{n}, $$ then (a)|azl≤ 2,and $(b)f(U)\supset D(0;\ {\textstyle\frac{1}{4}}).$ The second assertion is $\mathrm{that}f(U)$ contains all w with $|\,w\,|\,<\,{\frac{1}{4}}.$ PROOF By Theorem 14.12, there exists a ge J’ so that $g^{2}(z)=f(z^{2}).$ If $G=1/g,$ then Theorem 14.13 applies to ${\cal G},$ and this will give (a). Since $$ f(z^{2})=z_{-}^{2}(1+a_{2}z^{2}+\cdots), $$ we have $$ g(z)=z(1+{\textstyle{\frac{1}{2}}}a_{2}\,z^{2}\,+\,\cdots), $$ and hence $$ G(z)=\frac{1}{z}\left(1-\frac{1}{2}a_{2}z^{2}+\cdots\right)=\frac{1}{z}-\frac{a_{2}}{2}\,z+\cdots. $$ The Corollary to Theorem 14.13 shows now that lazl≤ 2 To prove (b), suppose w $\scriptstyle{i f f(v)}$ Define $$ h(z)=\frac{f(z)}{1-f(z)/w}. $$ Then $h\in H(U),h$ is one-to-one in ${\boldsymbol{U}},$ and $$ h(z)=(z+a_{2}z^{2}+\cdots\ )\biggl(1+{\frac{z}{w}}+\cdots\biggr)=z+\biggl(a_{2}+{\frac{1}{w}}\biggr)z^{2}+\cdots, $$ so that we finally obtain $|1/w|\leq4.$ Apply (a) to h: We have $|a_{2}+(1/w)|\leq2,$ and since |azl≤ 2, // ${\mathfrak{s e}}\ S_{*}$ So|w|≥ for every w 生f(U) This completes the proof. Example 14.11 shows that both (a and (b) are best possible $f(0)=1,$ Moreover, given any $\scriptstyle x\neq0.$ one can find entire functions $f,$ with $f(0)=0,$ that omit the value α. For example, $$ f(z)=\alpha(1-e^{-z/\alpha}). $$ Of course, no such fcan be one-to-one in ${\boldsymbol{U}}$ if $|\alpha|<\frac{1}{4}.$ order 1 at 14.15 Theorem Suppose $F\in H(U-\{0\}),$ ${\mathbf{}}F$ is one-to-one in $U,F$ has a pole of $\scriptstyle z=0,$ with residue l, and neither $w_{1}$ nor ${\boldsymbol{w}}_{2}$ are in $\scriptstyle{\cal{P}}(t).$ $$ T h e n\mid w_{1}-w_{2}\mid\leq4. $$coNFORMAL MAPPING 289 PROOF ${\mathrm{If}}f=1/(F-w_{1}),$ then fe ${\mathcal{F}},$ hence $f(U)\supset D(0,\ {\frac{1}{4}}).$ so the image of $U$ under $F-w_{1}$ contains all w with $|w|>4,$ Since $w_{2}-w_{1}$ is not in this image, // we have $|\,w_{2}-w_{1}\,|\leq4.$ contain the points Note that this too is best possible:: If $F(z)=z^{-1}+z,$ then $\scriptstyle{\vec{F}}(t)$ does not $2,-2.$ In fact, the complement of $\scriptstyle{\vec{F}}(U)$ is precisely the interval [-2,2] on the real axis. Continuity at the Boundary Under certain conditions, every conformal mapping of a simply connected region $\Omega$ onto $U$ can be extended to a homeomorphism of its closure $\bar{\Omega}$ onto ${\tilde{U}}.$ . The nature of the boundary of Q plays a decisive role here. 14.16 Definition A boundary point $\beta$ of a simply connected plane region $\Omega$ will be called a simple boundary point of $\Omega$ if $\boldsymbol{\beta}$ has the following property: To curve $\gamma_{\mathrm{,\varepsilon}}$ every sequence $\{\alpha_{n}\}$ in $\Omega$ such that $x_{n}\to\beta$ as n→Oo there corresponds a $0<t_{1}<t_{2}<$ …,t。→1, such that with parameter interval [0,1], and a sequence $\{t_{n}\}.$ $0\leq t<1.$ $\gamma(t_{n})=\alpha_{n}(n=1,$ 2, 3, ... and yt) e SQ for In other words, there is a curve in S which passes through the point $\scriptstyle x_{n}$ and which ends at β. 14.17 Examples Since examples of simple boundary points are obvious,let us look at some that are not simple $\boldsymbol{\beta}$ If S $\Omega$ Q is $U-\{x;0\leq x<1\},$ then $\Omega$ is simply connected; and if $0<\beta\leq1,$ is a boundary point of $\Omega$ which is not simple. To get a more complicated example, let $\Omega_{0}$ be the interior of the square with vertices at the points O,1, 1 + i, and i. Remove the intervals $$ \left[{\frac{1}{2n}},{\frac{1}{2n}}+{\frac{n-1}{n}}\ i\right]\ \ {\mathrm{and}}\ \ \left[{\frac{1}{2n+1}}+{\frac{i}{n}},{\frac{1}{2n+1}}+i\right] $$ from $\Omega_{0}$ . The resulting region $\Omega$ is simply connected. If $0\leq y\leq1,$ then iy is a boundary point which is not simple 14.18 Theorem Let $\Omega$ be a bounded simply connected region in the plane, and let fbe a conformal mapping of Q2 onto ${\boldsymbol{U}}.$ (a)If $\beta$ is a simple boundary point of $\Omega,$ then f has a continuous extension to (b)If $\beta_{1}$ and $\beta_{2}$ . Iff is so extended,then $|f(\beta)|=1.$ Q2 $\cup\left\{\beta\right\}$ are distinc simple boundary poinsof Q and i is extended to Q2 $\cup\left\{\beta_{1}\right\}\cup\left\{\beta_{2}\right\}$ as in (a), then f $(\beta_{1})\neq f(\beta_{2})$ $\scriptstyle{\mathcal{G}}$ PRoOF Let $\mathbf{\Omega}^{g}$ be the inverse of f. Then $g\in H(U),$ by Theorem 10.33, g(U) = 2, is one-to-one, and $\scriptstyle g\,\in\,H^{\prime}$ , sinceS $\Omega$ 2 is bounded290 REAL AND coMPLEX ANALYsis $K_{r}$ and put $\Gamma(t)=f(\gamma(t)),$ for Suppose (a is false. Then there is a sequence and $w_{1}\neq w_{2}\,.$ Choose $\scriptstyle\gamma$ as in Definition 14.16, $x_{n}\to\beta,$ $\{\alpha_{n}\}$ in $\Omega$ such that $t^{*}<t<1.$ $f(\alpha_{2n}) arrow w_{1},\,f(\alpha_{2n+1}) arrow w_{2}\,,$ . Put $K_{r}=g({\tilde{D}}(0;r)),$ $t\to1,$ there exists a $t^{*}<1$ $0\leq t<1$ for $0<r<1.$ Then is a compact subset of $\Omega.$ Since $\scriptstyle y(t)\to\beta$ as Thus 「T(0|>r if and (depending on r) such that $\gamma(t)\notin K_{r}\;\;\mathrm{if}\;\;\;t^{*}<t<t<$ 1. $\Gamma(t_{2n})\to w_{1}$ This says that $|\,\Gamma(t)\,|\to1$ as t→1. Since $\Gamma(t_{2n+1})\to w_{2}\,,$ we also have $|w_{1}|=|w_{2}|=1$ $\boldsymbol{J}$ whose union is Iit now follows that one of the two open arcs point of $\boldsymbol{J}$ intersects the range of ${\Gamma}$ has the property that every radius of $U$ which ends at a ${\boldsymbol{T}}.$ $T-(\{w_{1}\}\cup\{w_{2}\})$ in a set which has a limit point on Note that $g(\Gamma(t))=\gamma(t)$ for $0\leq t<1$ and that $\scriptstyle{\mathcal{G}}$ has radial limits a.e. on ${\boldsymbol{T}},$ since $g\in H^{\omega}.$ Hence $$ \operatorname*{lim}_{r\to1}g(r e^{i t})=\beta\qquad({\mathrm{a.e.~on~}}J), $$ (1) $\scriptstyle{\mathcal{G}}$ since $g(\Gamma(t)) arrow\beta$ as $t\to1.$ By Theorem 11.32, applied to $g-\beta,(1$ ) shows that is constant. But $\scriptstyle{\mathcal{G}}$ is one-to-one in ${\boldsymbol{U}},$ and we have a contradiction. Thus $w_{1}=w_{2}$ , and $\mathbf{\tau}_{(a)}$ is proved. by a suitable constant of absolute Suppose (b) is false. If we multiply $\boldsymbol{\f}$ $\beta_{i}.$ Put value l, we then have and $\beta_{2}$ are simple boundary points of $1)\subset U,$ and $\Gamma_{1}(1)=\Gamma_{2}(1)=1.$ and $\gamma_{i}(1)=$ $g(\Gamma_{i}(t))=\gamma_{i}(t)$ on [O, $1),$ $\beta_{1}\neq\beta_{2}$ but $f(\beta_{1})=f(\beta_{2})=1.$ there are curves $\mathbf{\Omega}^{2}$ with Since $\beta_{1}$ Then F(EO, $1)\backslash\mathbf{<}\Omega$ for $\scriptstyle{i=1}$ and $\gamma_{i}$ Since $\Omega,$ parameter interval [0,1] such that y([O, $\Gamma_{i}(t)=f(\gamma_{i}(t)).$ we have $$ \operatorname*{lim}_{t arrow1}g(\Gamma_{i}(t))=\beta_{i}\qquad(i=1,\,2). $$ (2) $\beta_{2}$ Theorem 12.10 implies therefore that the radial limit of $\scriptstyle{\mathcal{G}}$ g at 1 is $\beta_{1}$ as well as / . This is impossible if $\beta_{1}\neq\beta_{2}$ 14.19 Theorem If $\Omega$ is a bounded simply connected region in the plane and if onto U every boundary point of $\Omega$ is simple, then every conformal mapping of $\Omega$ extends to a homeomorphism of $\bar{\Omega}$ Q onto ${\bar{U}}.$ $\{\alpha_{n}\}$ PRoOF Suppose $f\in H(\Omega),f(\Omega)=U_{\L_{\L\L}}$ and $\boldsymbol{\f}$ f is one-to-one. By Theorem 14.18 whenever and is a sequence in $\Omega$ to a mapping of $\bar{\Omega}$ into $\bar{U}$ such that $f(\alpha_{n})\to f(z)$ $\bar{\Omega}$ which we can extend $\boldsymbol{\mathsf{f}}$ which converges to $|\,z.\uparrow\mathbb{F}\left\{z_{n}\right\}$ is a sequence in converges to $\mathbb{Z},$ there exist points $\alpha_{n}\in\Omega$ such that $|\alpha_{n}-z_{n}|<1/n$ $|f(\alpha_{n})-f(z_{n})|<1/n.$ Thus $\alpha_{n}\to z,$ hence $f(\alpha_{n})\to f(z),$ and this shows that $f(z_{n})\to f(z).$ The compactness of $\bar{U}$ implies that $\boldsymbol{\f}$ is continuous on ${\bar{\Omega}}.$ Also We have now proved that our extension of ${\cal U}_{ arrow}f(\vec{\Omega})\subset\bar{{\cal U}}.$ $I(x)$ is compact. Hence $f(\Omega)=U$cONFORMAL MAPPING 291 Theorem 14.18(6) shows that $\boldsymbol{\mathsf{f}}$ is one-to-one on Q.Since every contin- uous one-to-one mapping of a compact set has a continuous inverse([261 Theorem 4.17), the proof is complete. // 14.20 Remarks (a)The preceding theorem has a purely topological corollary: If every bound ary point of a bounded simply connected plane region $\Omega$ is simple, then the boundary of $\Omega$ is a Jordan curve, and $\bar{\Omega}$ is homeomorphic to ${\widetilde{U}}.$ (A Jordan curve is, by definition, a homeomorphic image of the unit circle.) The converse is true, but we shall not prove it: If the boundary of $\Omega$ is a Jordan curve, then every boundary point of $\Omega$ is simple. (b)Suppose fis as in Theorem 14.19,a, ${\mathfrak{b}},$ and c are distinct boundary points of S, and $A,$ ${\boldsymbol{B}},$ B, and ${\boldsymbol{C}}$ are distinct points of ${\boldsymbol{T}}.$ There is a linear fractional onto transformation $\varphi$ which maps the triple $\{f(a),f(b),f(c)\}$ to{A, B, C; then $\bar{U}$ suppose the orientation of {A,B, C} agrees with that of $\{f(a),f(b),f(c)\}_{2}$ to pre- $\varphi_{=}U)=U,$ and the function $g=\varphi\circ f$ is a homeomorphism of 2 ${\mathfrak{b}},{\mathfrak{k}}$ which is holomorphic in Q and which maps {a, is uniquely scribed values{A, B, C} It follows from Sec. 14.3 that g $\scriptstyle{\mathcal{G}}$ determined by these requirements. (c) Theorem 14.19, as well as the above remark (b), extends without difficulty to simply connected regions $\Omega$ in the Riemann sphere $S_{2},$ all of whose boundary points are simple, provided that $\scriptstyle S^{2}\,-\,\Omega$ has a nonempty inte- rior, for then a linear fractional transformation brings us back to the case in which Q is a bounded region in the plane. Likewise, $U$ can be replaced, for instance, by a half plane. then (d) More generally, if f, and f $\ f_{2}$ is a homeomorphism of ${\tilde{\Omega}}_{1}$ onto ${\bar{\Omega}}_{2}$ , as in Theorem 14.19 fz map $\Omega_{1}$ 2,and $\Omega_{2}$ onto ${\boldsymbol{U}},$ $f=f_{2}^{-1}\circ f_{1}$ which is holo- morphic in $\Omega_{1}.$ Conformal Mapping of an Annulus 14.21 It is a consequence of the Riemann mapping theorem that any two simply connected proper subregions of the plane are conformally equivalent, since each of them is conformally equivalent to the unit disc. This is a very special property of simply connected regions. One may ask whether it extends to the next simplest situation, i.e., whether any two annuli are conformally equivalent. The answer is negative. Fo $\tau\circ>r<R.$ let $$ A(r,R)=\{z\colon r<|z|<R\} $$ (1) equivalent i be the annulus with inner radius ${}^{r}$ and outer radius ${\boldsymbol{R}}.$ If $\scriptstyle{\lambda>0}$ the mapping Z→入z maps $A(r,\,R)$ onto $4(\lambda r,\lambda R)$ Hence ${\mathit{A}}(r,\,R)$ and A(r, $R_{1})_{\circ}$ are conformally $R/r=R_{1}/r_{1}.$ . The surprising fact is that thsSuficient condition is292 REAL AND COMPLEX ANALYSIS also necessary; thus among the annuli there is a different conformal type associ- ated with each real number greater than 1. if 14.22 Theorem $A(r_{1},R_{1})$ and $A(r_{2},R_{2})$ are conformally equivalent if and only $R_{1}/r_{1}=R_{2}/r_{2}\,.$ PROOF Assume $r_{1}=r_{2}=1,$ without loss of generality. Put $$ A_{1}=A(1,\,R_{1}),\qquad A_{2}=A(1,\,R_{2}), $$ (1) Let ${\cal K}\,\,$ and assume there exists $f\in H(A_{1})$ such that f is one-to-one and f(A1)= A2 be the circle with center at ${\mathbf{0}}$ and radius $r={\sqrt{R_{2}}}\cdot\mathrm{Since}f^{-1}\!:A_{2}\!\to\!A_{1}$ is also holomorphic,f 1(K) is compact. Hence $$ A(1,1+\epsilon)\cap f^{-1}(K)=\emptyset $$ (2) for some $\scriptstyle x\,>0$ Then $V=f(A(1,\;1+\epsilon))$ is a connected subset of $A_{2}$ which is if does not intersect $R_{2}/f.$ So we can assume that V c A(1, r). If $V\subset A(r,\,R_{2})$ In the latter case, and $K,$ so that $V\subset A(1,\,r)$ or replace f by and $\scriptstyle(f(x_{3})$ has no limit point in $A_{2}$ (since ${\mathcal{F}}^{-1}$ continuous); thus $|f(z_{n})|\to1$ $1<\vert z_{n}\vert<1+\epsilon$ 1z1→1,then $f(z_{n})\in V$ In the same manner we see that $|f(z_{n})|\to R_{2}$ $|z_{n}|\to R_{1}.$ Now define $$ \alpha={\frac{\log R_{k}}{\log R_{k}}} $$ (3) and $$ u(z)=2\,\log\,|f(z)|-2\alpha\,\log\,|z|\qquad(z\in A_{1}). $$ (4) chain rule gives Let a be one of the Cauchy-Riemann operators. Since $\scriptstyle{\vec{g}}=0$ and $\partial\!\!\!/=f^{\prime},$ the $$ \partial(2\log\vert f\vert)=\partial(\log\left(\tilde{f}\right))=f^{\prime}/f, $$ (5) so that $$ (\hat{\omega}u)(z)=\frac{f^{\prime}(z)}{f(z)}-\frac{\alpha}{z}\;\;\;\;\;\;\;(z\in A_{1}). $$ (6) $A_{1}.$ Thus $\boldsymbol{u}$ is a harmonic function in $A_{1}$ which, by the first paragraph of this proof, extends to a continuous function on ${\bar{A}}_{1}$ which is O on the boundary of Since nonconstant harmonic functions have no local maxima or minima, we conclude that $u=0.$ Thus $$ {\frac{f^{\prime}(z)}{f(z)}}={\frac{\alpha}{z}}\qquad(z\in A_{1}). $$ (7)cONFORMAL MAPPING 293 Put $\gamma(t)={\sqrt{R_{1}}}\,e^{i t}$ (-n ≤t≤ T); put T =f。y. As in the proof of Theorem 10.43,(7) gives $$ x={\frac{1}{2\pi i}} \{\frac{f^{\prime}(z)}{f(z)}\,d z=\mathrm{Ind}_{\Gamma}\,(0). $$ (8) $\mathrm{Thus}f(z)=c z^{\circ}$ Thus α is an integer. By (3),α> 0.By (T), the derivative of Hence $R_{2}=R_{1}.$ is $\mathbf{0}$ in $A_{1}.$ . Since f is one-to-one in $z^{-\alpha}f(z)$ // $4_{1},\alpha=1.$ Exercises 1 Find necessary and sufficient conditions which the complex numbers a,b,c, and $z\to(a z+b)/(z+d)$ maps the uper half plane onto itsef $d$ have to satisfy so that the linear fractional transformation Im $f(z)\to0$ as $z\to L.$ 2 In Theorem 11.14 the hypotheses were, in simplified form, that $\scriptstyle1\,0\,=\,{\mathfrak{m}}^{\prime}$ ${\mathbf{}}L$ is on the real axis, and Use this theorem to establish analogous refection theorems under tefollowing hypotheses (a $\scriptstyle\mathbf{a}=\mathbf{n}^{*},L$ on rc $\mathrm{:}{\mathrm{all~axis,}}\;|f(z)|\to1\;{\mathrm{as}}$ $\to L.$ $z\to L.$ (b) $\Omega\subset U,L\subset T,|f(z)|\to1{\mathrm{~as~}}z\to L.$ show that its extension has a pole at 1/. What are the ie) e U, $L\subset T,\operatorname{Im}f(z)\to0{\mathrm{~as~}}z$ $\scriptstyle{x\cdot b}$ In case b), if f has a zero at analogues of this in cases (a) and (c)? 3 Suppos ${\boldsymbol{R}}$ is a rational function such $\operatorname{that}|R(z)|=1\operatorname{if}|z|=1.{\operatorname{Proveth}}$ at $$ R(z)=c z^{m}\prod_{n=1}^{k}\ {\frac{z-\alpha_{n}}{1-\tilde{\alpha}_{n}z}} $$ $|\alpha_{n}|\neq1.$ where c is a constant, m is an integer, and $\alpha_{1},\dots,\alpha_{k}$ are complex numbers such that $x_{n}\neq0$ and Note that each of the above factors has absolute value ${\mathfrak{k}}|z|=1.$ 4 Obtain an analogous description of those rational functions which are positive on ${\boldsymbol{T}}.$ Hint: Such a function must have the same number of zeros as poles in $\boldsymbol{\mathit{U}}$ Consider products of factors of the form $$ \frac{(z-a)(1-\tilde{\alpha}z)}{(z-\beta)(1-\tilde{\beta}z)} $$ where $\scriptstyle n\,\scriptstyle n\times\mathrm{T}$ and $|\beta|<1.$ S Suppose fis a trigonometric polynomial, $$ f(\theta)=\sum_{k=-n}^{n}a_{k}e^{i k\theta}, $$ and f() > 0 for all real e. Prove that there is a polynomial $P(z)=c_{0}+c_{1}z+\cdots+$ c,r"suchtha $$ f(\theta)=|P(e^{i\theta})|^{2}\qquad(\theta\ \mathrm{real}). $$ instead Hint: Apply Exercise 4 to the rational function $\Sigma a_{k}z^{k}.$ Is the result still valid if we assume $f(\theta)\geq0$ $o\mathbb{f}f(\theta)>0\uparrow$ itself? 6 Find the fixed points of the mappings $\varphi_{a}$ (Definition 12.3). Is there a straight line which $\varphi_{x}$ maps to 7 Find all complex numbers c for which fis one-to-one in $U,$ where $$ f_{\alpha}(z)={\frac{z}{1+\alpha z^{2}}}. $$ Describ $f_{\alpha}(U)$ for all these cases.294 REAL AND cOMPLEX ANALYSIs 8 Suppose $f(z)=z+(1/z).$ Describe the families of ellipses and hyperbolas onto which f maps circles with center at O and rays through O. 9(a) Suppose $\Omega=\{z\colon-1<\mathbb{R}\circ z<1\}.$ Find an explicit formula for the one-to-one conformal mapping f of $\scriptstyle\Omega$ onto $U_{\mathbf{\delta}}$ for which f(O $\mathbf{\tau}_{i}=0$ and $|f^{\prime}(0)>0.$ Compute f′(0). ${\mathfrak{(}}b{\mathfrak{)}}$ Note that the inverse of the function constructed in (a) has its real part bounded in $U,$ whereas its imaginary part is unbounded. Show that this implies the existence of a continuous real function $\dot{\boldsymbol{u}}$ on $\bar{U}$ which is harmonic in $\boldsymbol{\mathit{U}}$ and whose harmonic conjugate $\boldsymbol{\mathit{U}}$ p is unbounded in $\boldsymbol{\mathit{U}}$ [v is the function which makes $u+i\mathrm{i}$ v holomorphic in ${\boldsymbol{U}};$ we can determine v uniquely by the requirement $v(0)=0.3$ (c) Suppose $g\in H(U),\left|{\mathrm{Re~}}g\right|<1$ in ${\boldsymbol{U}},$ , and $g(0)=0.$ Prove that $$ |g(r e^{i\theta})|\leq\frac{2}{\pi}\log\frac{1+r}{1-r}. $$ Hint: See Exercise 10. (d) Let $\underline{{\Omega}}$ be the strip that occurs in Theorem 12.9. Fix a point $x+i\beta$ in $\Omega.$ 2. Let ${\boldsymbol{h}}$ be a conformal one-to-one mapping of $\Omega$ 2 onto $\underline{{\Omega}}$ that carries $\alpha+i\beta$ to O. Prove that $$ |k(\alpha+i\beta)|=1/\cos\beta. $$ Prove that 10 Suppose f and g are holomorphic mappings of $U$ into Q,fis one-to-one, $f(U)=\Omega,\,\mathrm{and}\,f(0)=g(0).$ $$ g(D(0;r))\subseteq f(D(0;r))\qquad(0<r<1). $$ 11 Let SQ be the upper half of the unit disc ${\boldsymbol{U}}.$ Find the conformal mapping f of Q onto $U_{\mathbf{\delta}}$ that carries $\{-1,0,1\}$ to $(-1,-i,1).$ Find $\scriptstyle{\varepsilon\circ\Omega}$ such $\operatorname{that}f(z)=0.$ Find f(/) Hint: f= 9。s。业, where qp and $\psi$ are linear fractional transformations and $s(\lambda)=\lambda^{2}.$ 12 Suppose Q is a convex region. $f\epsilon\ H(\Omega),$ and $\operatorname{Re}f^{\prime}(z)>0$ for a $||z\in\Omega,|$ Prove that fis one-to-one in $\Omega.$ Is the result changed if the hypothesis is weakened to $\mathrm{Re}\,f^{\prime}(z)\geq0^{\gamma}$ (Exclude the trivial case ${\boldsymbol{f}}=$ constant.) Show by an example that “convex" cannot be replaced by " simply connected.” 13 Suppose $\underline{{\Omega}}$ is a region,f,e H(Q) for $n=1,$ 2 $\mathbf{3},\mathbf{\boldsymbol{\cdot}}\cdot\cdot\cdot,\mathrm{each}\,f_{n}$ is one-to-one in $\Omega,$ and $f_{n}\to f$ uniformly on compact subsets of Q. Prove that fis either constant or one-to-one in Q. Show that both cases can occur 14 Suppose Q ={x+ iy: -1 <y<1},fe H(Q), J|<1,and f(x)→O as x→αo. Prove that $$ \operatorname*{lim}_{x arrow x_{0}}f(x+i y)=0\qquad(-1<y<1) $$ and that the passage to the limit is uniform if $\scriptstyle{y}$ is confined to an interval [-α,α], where α < 1. Hint: Consider the sequence $\{f_{n}\},$ where $f_{n}(z)=z+n,$ in the square $|x|<1,|y|<1$ near a boundary point of What does this theorem tell about the behavior of a function $g\in H^{\infty}$ $U_{\mathbf{\delta}}$ at which the radial limit of g exists? 15 Let ${\mathcal{Q}}^{\prime}$ be the class of al $f\in H(U)$ such that $\mathrm{Re}\,f>0$ and $f(0)=1.$ Show that $\mathcal{\sqrt{\theta\,}}$ is a normal family. Can the condition ${}^{\omega}f(0)=1^{\infty}|$ be omitted? Can it be replaced by ${}^{\circ}|f(0)|\leq$ 1? 16 Let ${\mathcal{N}}$ be the class of ${\mathrm{all}}f\in H(U)$ for which $$ \vert\int_{U}^{}\vert f(z)\vert^{2}\,d x\,d y\leq1. $$ Is this a normal family ? such $\operatorname{that}f_{n}$ is one-to-one on $\scriptstyle{\mathcal{K}}$ $f_{n}\in H(\Omega)$ for $n=1,$ 2, $3,\ldots,f_{n}\to f$ uniformly on compact subsets of $\Omega,$ , and 17 Suppose $\underline{{\Omega}}$ is a region f is onc-to-one in A. Does it follow that to each compact for all $n>N(K){\mathcal{Y}}$ Give proof or counterexample. there corresponds an integer $N(K)$ $\scriptstyle{\kappa\cdot\Omega}$CONFORMAL MAPPING 295 of $\scriptstyle\Omega$ 2 onto $U_{\mathbf{\delta}}$ which carry $\scriptstyle{Z_{0}}$ is a simply coniected region, $z_{0}\in\Omega,$ and f and g are one-to-one conformal mappings 18 Suppose $\Omega_{\mathbf{\Lambda}}$ 。to O. What relation exists between and g? Answer the same question if $f(z_{0})=g(z_{0})=a,\mathrm{for~some~}a\in{}$ U 19 Find a homeomorphism of $\boldsymbol{\mathit{U}}$ onto $\boldsymbol{\mathit{U}}$ which cannot be extended to a continuous function on ${\vec{U}}.$ g"(2) =/(r") for al 20 It fe 9 (Definition 14.10) and nis a positive inter, prove that there exis ${\textbf{a}}g\in{\mathcal{F}}$ such that $z\in U.$ 21 Find all fe J such that $(a)f(U)\supset U,(b)f(U)\supset{\bar{U}},(c)\mid a_{2}\mid=2$ Prove that f(iz)= if(z). 2z Suppose f is a one-to-one conformal mapping of prove that $c_{n}=0$ unless $n-1$ is a multiple of ${\mathfrak{4}}.$ Generalize $U_{\mathbf{\delta}}$ onto a square with center at $0,$ , and $f(0)=0.$ $\operatorname{If}f(z)=\Sigma_{c_{n}}z^{n},$ this: Replace the square by other simply connected regions with rotational symmetry 23 Let $\;\underline{{\Omega}}$ be a bounded region whose boundary consists of two nonintersecting circles. Prove that that there is a one-to-one conformal mapping of $\underline{{\Omega}}$ h onto an annulus.(This is true for every region $\;\underline{{\Omega}}$ such $\scriptstyle s^{2}=\mathbf{a}$ has exactly two components, each of which contains more than one point, but this general situation is harder to handle) sequence of $\{f_{n}\}$ 24 Complete the details in thefollowing proof of Theorem 14.22. Suppose $R_{1})$ onto A(1, $R_{2}).$ Define $f_{1}=f$ and $f_{n}=f\circ f_{n-1}.$ Then a sub- $\scriptstyle{\vec{f}}$ is a ${\mathsf{I}}<R_{2}<R_{i}$ and one-to-one conformal mapping of $A(1,$ range of $\scriptstyle{\mathcal{G}}$ converges uniformly on compact subsets of $A(1,\,R_{1})$ to a function $g.$ Show that the cannot contain any nonempty open set (by the three-circle theorem, for instance). On the other hand, show that g $\scriptstyle{\mathcal{G}}$ cannot be constant on the circle $(z:|z|^{2}=R_{1}\}$ Hence f cannot exist 2S Here is yet another proof of Theorem 14.22. If f is as in 14.22, repeated use of the reflection principle extends fto an entire function such that $|f(z)|=1$ whenever $\scriptstyle x\;|x|=1$ This implies $f(z)=\alpha z^{n},$ where $\scriptstyle n\;=\;1$ and $\scriptstyle{\vec{n}}$ is an integer. Complete the details 26 Iteration of Step 2 in the proof of Theorem 14.8 leads to a proof (due to Koebe) of the Riemann mapping theorem which is constructive in the sense that it makes no appeal to the theory of normal of the proof it is convenient to assume that $\underline{{\Omega}}$ families and so does not depend on the existence of some unspecified subsequence. For the final step has property (h) of Theorem 13.11. Then any region conformally equivalent to $\;\underline{{\Omega}}$ will satisfy (h). Recall also that (h) implies O), trivially By Step lin Theorem 14.8 we may assume, without loss of generality, that $0\in\Omega,\Omega<U$ and $\Omega\neq U$ Put $\Omega=\Omega_{0}\,.$ The proof consists in the construction of regions $\Omega_{1},$ $\mathbf{a}_{i},\mathbf{a}_{3},\ldots$ and of func- tions $f_{1},f_{2},f_{3},\ldots,$ so that f,(Q2,- $\;_{1}=\Omega_{n}$ and so that the functions f, $circ f_{n-1}\circ\cdot\cdot\cdot\circ f_{2}\circ f_{1}$ converge to a conformal mapping of $\;\underline{{\Omega}}$ onto $\boldsymbol{\mathit{U}}$ Complete the details in the following outline. a boundary point of $\Omega_{n-1}$ with is constructed, let ${\boldsymbol{r}}_{n}$ choose $\beta_{n}$ be the largest number such that $D(0;\,r_{n})\subset\Omega_{n-1},\;\mathrm{let}\;\alpha_{n}$ be (a) Suppose $\Omega_{n-1}$ $|\alpha_{n}|=r_{n},$ so that $\beta_{n}^{2}=-\alpha_{n},$ and put $$ F_{n}=\varphi_{-\alpha_{n}}\circ\,s\circ\varphi_{-\beta_{n}}. $$ region $\Omega_{n}\subset U,$ (The notation is as in the proof of Theorem and $c=G_{n}^{\prime}(0).$ (This $\textstyle\int_{n}$ is the Koebe mapping associated with $G_{n}$ in $\Omega_{n-1},$ $14,8.,$ Show that ${\boldsymbol{F}}_{n}$ has a holomorphic inverse and put $f_{n}=\lambda_{n}G_{n},$ where ${\lambda}_{n}=|c|/c$ involves only two linar fractional transformations onto a $\Omega_{n-1}.$ Note that ${f}_{n}$ is an elementary function. $\operatorname{It}$ Show that $\textstyle\left|\not p_{\scriptscriptstyle R}\right|$ is a one-to-one mapping of $\underline{{\Omega}}$ and a square root.) $\mathrm{that}f_{n}^{\prime}(0)=(1+r_{n})/2{\sqrt{r_{n}}}>1.$ (b) Compute (e Put $\psi_{0}(z)=z$ and $\psi_{n}(z)=f_{n}(\psi_{n-1}(z)).$ that {w.(O)} is bounded,that $$ \psi_{n}^{\prime}(0)=\prod_{k=1}^{n}{\frac{1+r_{k}}{2{\sqrt{r_{k}}}}}, $$ and that therefore $r_{n}\to1$ as n→OO show that $\left|\,h_{n}\,\right|\leq\,\left|\,h_{n+1}\,\right|$ .apply Harnacks theorem and Exer- $\Omega,$ and a Wrte $\psi_{n}(z)=z h_{n}(z),$ for $\scriptstyle{\varepsilon\circ\Omega}$ converges uniformily on compact subsets of cise 8 of Chap. 11 to{log $|h_{n}|\}$ to prove that $\left\{\hat{\psi}_{m}\right\}$ ${\boldsymbol{U}}.$ show that lim $\textstyle{\dot{\mathcal{Y}}}_{R}$ is a one-to-one mapping of $\underline{{\Omega}}$ onto296 REAL AND cOMPLEX ANALYSIs $27$ Prove that $\textstyle\sum_{n=1}^{\infty}(1-r_{n})^{2}<\infty,{\mathrm{where~}}\{r_{n}$ } is the sequence which occurs in Exercise 26. Hint: $$ \frac{1+r}{2\sqrt{r}}=1+\frac{(1-\sqrt{r})^{2}}{2\sqrt{r}}. $$ 28 Suppose that in Exercise 26 we choose $y_{_{n}}\in U-\left|\mu_{n-1}\right.$ without insisting that $|\alpha_{n}|=r_{n}.$ For example, insist only that $$ |\alpha_{n}|\leq{\frac{1+r_{n}}{2}}. $$ Will the resulting sequence $\{\dot{\psi}_{m}\}$ still converge to the desired mapping function? $\mathbb{Z}9$ Suppose $\scriptstyle\Omega$ is a bounded region, $a\in\Omega,f\in H(\Omega),f(\Omega)\subset\Omega,{\mathrm{~and~}}f(a)=a.$ (a) $\ P\mathrm{ut}\,f_{1}=f\,\mathrm{and}\,f_{n}=f\circ f_{n-1},$ compute fA(a), and conclude that |f"(a)l ≤ 1. (b) $\operatorname{If}f^{\prime}(a)=1,$ prove ${\mathrm{:that}}f(z)=z{\mathrm{~for~all~}}z\in\Omega,I$ Hint: If $$ f(z)=\bar{z}+\bar{c}_{n}(\bar{z}-a)^{n}+\cdots, $$ $g(\Omega)<\Omega$ compute the coeficient of $(z-a)^{m}$ in the expansion of $n_{k}\to\infty$ such that $\gamma^{n}\to1$ and $f_{n_{k}} arrow g.$ Then $g^{\prime}(a)=1,$ (c) $\operatorname{ff}\left|f^{\prime}(a)\right|=1,$ prove that fis one-to-one and t $f_{n}(z).$ $\operatorname{hat}f(\Omega)=\Omega.$ Hint:If $\gamma=f^{\prime}(a),$ there are integers part (b). Use g to draw the desired conclusions (by Exercise 2o. Chap 10), hence $g(z)=z,\,\mathbf{by}$ about f 30 Let $\Lambda$ be the set of all linear fractional transformations If $\{a,\,\beta,\,\gamma,\,\delta\}$ is an ordered quadruple of distinct complex numbers, its cross ratio is defined to be $$ [\alpha,\,\beta,\,\gamma,\,\delta]=\frac{(\alpha-\beta)(\gamma-\delta)}{(\alpha-\delta)(\gamma-\beta)}. $$ (a) $\operatorname{If}\varphi(z)=[z,\,\alpha,\,\beta,$ y], show that $\varphi\in\mathbb{A}$ If one of these numbers is o, the definition is modifed in the obvious way, by continuity. The same maps $\{\alpha,\,\beta,\,\gamma\}$ to {0,1, α} $w=\varphi(z);$ then applies if a coincides with $\beta$ or y or $\delta.$ and $\varphi$ (6) Show that the equation [w, $a,\;b,\;c\}=[z,\,\alpha,\;\beta,\;\gamma]$ can be solved in the form gp ∈ A maps {α,β, $\gamma_{j}$ to $\{a,\,b,\,c\}$ (c) If $\varphi\in\Lambda,$ show that $$ [\varphi(x),\;\varphi(\beta),\;\varphi(\gamma),\;\varphi(\delta)]=[\alpha,\;\beta,\,\gamma,\,\delta]. $$ line. (d) Show that $[\alpha,\,\beta,\,\gamma,\,\delta]$ is real if and only if the four points lie on the same circle or straight (e) Two points $\mathbb{Z}$ and $\mathbb{Z}^{\oplus}$ are said to be symmetric with respect to the circle (or straight line) C through α, ${\boldsymbol{\beta}},$ and $\gamma\,\mathrm{if}\ [z^{*},\,\alpha,\,\beta,\,\gamma]$ is the complex conjugate of [z,α,β, y]. If $\scriptstyle{\bar{C}}$ is the unit circle, find a simple geometric relation between z and $z^{\bullet}.$ Do the same if $\scriptstyle{\vec{C}}$ is a straight line $\mathbf{(f)}$ Suppose z and $\scriptstyle Z^{\bullet}$ are symmetric with respect to C. Show that op(z) and p(z*) are symmetric with respect to g(C), for every g∈ A $\varphi\in\Lambda$ and $\psi\in\Lambda,$ show that $\varphi\circ\psi\in\Lambda$ 31 (a) Show that A(see Exrcise 30) is a group, with composition as group operation. That is $\varphi^{-1}$ of $\textstyle{\varphi}$ is in A. Show that A is not $\hat{\mathbf{f}}$ and that the inverse commutative. (b)Show that each member of $\Lambda^{}$ (other than the identity mapping) has either one or two fxed points on $S^{2}.$ [A fixed point of p is a point a such that $\varphi(x)=\alpha.1$ $\varphi_{1}=\psi^{-1}\circ\varphi\circ\psi.$ (c) Call two mappings $\varphi\in\mathbb{A}$ with a unique fixed point is conjugate to the mapping $\bar{\mathcal{I}}=\mu\;\bar{\mathcal{I}}\;\downarrow\;\Big|_{\nu}$ Prove that Prove that every $\textstyle{\varphi}$ and $\varphi_{1}\in\Lambda$ conjugate if there exists a veA such that number;to what extent is α determined by every pe A with two distinct fixed points is conjugate to the mapping z→αz, where a is a complex $\varphi\,?$cONFORMAL MAPPING $297$ there corresponds a β such that (a) Let a be a complex number. Show that to every qp e A which has a for its unique fixed point $$ {\frac{1}{\varphi(z)-\alpha}}={\frac{1}{z-\alpha}}+\beta. $$ and (e) Let $\scriptstyle{\dot{\mathbf{r}}}$ and be the set of all these p, plus the identity transformation. Prove that given by $G_{a,\beta}$ be the set of all is a subgroup of $\Lambda^{}$ and Let $G_{\alpha}$ is isomorphic to the additive group of all complex numbers $G_{\alpha}$ which have α that $G_{x}$ $\boldsymbol{\beta}$ be distinct complex numbers, and let $\varphi\in\Lambda$ $\boldsymbol{\beta}$ as fixed points. Show that every $\varphi\in G_{a,\beta}{\mathrm{is}}$ $$ \frac{\varphi(z)-\alpha}{\varphi(z)-\beta}=\gamma\cdot\frac{z-\alpha}{z-\beta}, $$ where $\gamma$ y is a complex number. Show that $G_{\alpha,\beta}$ is a subgroup of $\Lambda$ which is isomorphic to the multipli- (f) If $\scriptstyle{\varphi}$ cative group of all nonzero complex numbers. or (e),for which circles $\scriptstyle{\bar{C}}$ is it true that $\varphi(C)=C^{\prime}$ The answer should be in is as in $\mathbf{(}d)$ terms of the parameters $x,\,\beta,$ and ${\mathfrak{I}}.$ $32$ For $z\in{\overline{{U}}},z^{2}\neq1,$ define $$ f(z)=\exp\left\{i\log{\frac{1+z}{1-z}}\right\}, $$ choosing the branch of log that has log $1=0.$ Describe f(E) if E is (a ${\boldsymbol{U}},$ ${\boldsymbol{T}},$ (b) the upper half of ${\boldsymbol{T}},$ (e) the lower half of (d) any circular arc (in U) from $-1{\mathrm{~to~}}1,$ (e) the radius [0,1) $\mathbf{(f)}$ any disc $\{z;|z-r|<1-r\},0<r<1.$ $\underline{{\bigcup}}_{*}$ (g) any curve in U tending to 33 If $\varphi_{a}$ is as in Definition 12.3, show that $$ \begin{array}{c}{{(a)\ \frac{1}{\pi}\displaystyle\int_{U}|\varphi_{\alpha}^{\prime}|^{2}\ d m=1,}}\\ {{(b)\ \frac{1}{\pi}\displaystyle\int_{U}|\varphi_{\alpha}^{\prime}|\ d m=\frac{1-|\alpha|^{2}}{|\alpha|^{2}}\log\frac{1}{1-|\alpha|^{2}}.}}\end{array} $$ Here m denotes Lebesgue measure in $R^{2}{\mathrm{;}}$