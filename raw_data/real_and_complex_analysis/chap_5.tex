CHAPTER FIVE EXAMPLES OF BANACH SPACE TECHNIQUES Banach Spaces 5.1 In the preceding chapter we saw how certain analytic facts about trigonomet- ric series can be made to emerge from essentially goemetric considerations about general Hilbert spaces,involving the notions of convexity, subspaces,orthog- onality, and completeness. There are many problems in analysis that can be attacked with greater ease when they are placed within a suitably chosen abstract framework. The theory of Hilbert spaces is not always suitable since orthogonality is something rather special. The class of all Banach spaces affords greater variety In this chapter we shall develop some of the basic properties of Banach spaces and illustrate them by applications to concrete problems. 5.2 Definition A complex vector space $X$ Y is said to be a normed linear space if to each ${\mathfrak{x}}\in X$ there is associated a nonnegative real number llxl|, called the norm of ${\boldsymbol{x}},$ such that (a) $\|x+y\|\leq\|x\|+\|y\|$ llx| i $\operatorname{xe}{\mathcal{X}}$ and α is a scalar, $\gamma\epsilon\,X,$ for all x and 6 $|a x||=|x|$ implies $\scriptstyle x\;=\;0$ (c $|x||=0$ By (a), the triangle inequality $$ \|x-y\|\leq\|x-z\|+\|z-y\|\qquad(x,\,y,\,z\in X) $$ holds. Combined with (b) (take $\scriptstyle x\;=\;0$ $x=-1)$ and (c) this shows that every normed linear space may be regarded as a metric space, the distance between $\scriptstyle{\mathcal{X}}$ and y being $\|x-y\|.$ A Banach space is a normed linear space which is complete in the metric defined by its norm. 9596 REAL AND coMPLEX ANALYSIs $1\leq p\leq\infty,$ and so is For instance, every Hilbert space is a Banach space,so is every $\left\|x\right\|=\left|x\right|$ $\scriptstyle T_{(i)}$ normed by $\|f\|_{p}$ $c_{d}X)$ (provided we identify functions which are equal a.e.) if with the supremum norm. The simplest Banach space is of course the complex feld itself, normed by One can equally well discuss real Banach spaces; the definition is exactly the same, except that all scalars are assumed to be real. 5.3 Definition Consider a linear transformation A from a normed linear space $X$ into a normed linear space ${\boldsymbol{Y}},$ and define its norm by $$ \|\land\|=\operatorname*{sup}:\{\|\land x\|:x\in X,\;\|x\|\leq1\}. $$ (1) If In(1), ${\big\Vert}\,x{\big\Vert}$ is the norm of $\scriptstyle{\mathcal{X}}$ , then A is called a bounded linear transformation $X,$ $\|\Delta x\|$ is the norm of Ax in ${\cal{Y}};$ it will $|\mathbf{A}|<\infty$ in frequently happen that several norms occur together, and the context will make it clear which is which. Observe that we could restrict ourselves to unit vectors x $\scriptstyle{\mathcal{X}}$ in $(1),\,\mathrm{i}{\boldsymbol{\epsilon}}.$ to x with $|x||=1,$ without changing the supremum, since $$ \|\Lambda(x x)\|=\|\alpha\Lambda x\|=|\alpha|\Vert\Lambda x\|. $$ (2) Observe also that $\|\land\ \|$ is the smallest number such that the inequality $$ \|\Lambda x\|\leq\|\Lambda\|\ \ \|x\| $$ (3) holds for etvery x ∈ $X.$ The following geometric picture is helpful: $\Lambda$ maps the closed unit ball in X, i.e., the set $$ \{x\in X\colon\|x\|\leq1\}, $$ (4) into the closed ball in ${\bf Y_{\nu}}$ with center at $\mathbf{0}$ and radius |lA An important special case is obtained by taking the complex field for Y; in that case we talk about bounded linear functionals. 5.4 Theorem For a linear transformation A $\mathrm{\A}$ of a normed linear space $X$ into a normed linear space Y, each of the following three conditions implies the other two: (a)A is bounded (b)A is continuous (c)A is continuous at one point oy $X.$ words, $|x||<\delta$ and (b) implies (c) trivially. Suppose $\Lambda$ is continuous at $\scriptstyle x_{0}$ , it is clear that (a) implies ((b) $\scriptstyle\epsilon\;>0$ one PR0OF Since $\|\Lambda\|(x_{1}-x_{2})\|\leq\|\Lambda\|\ \|x_{1}-x_{2}\|$ .To each can then find a $\scriptstyle\delta>0$ so that $\|x-x_{0}\|<\delta$ implies $\|\Lambda x-\Lambda x_{0}\|<\epsilon.$ In other implies $$ \|\Lambda(x_{0}+x)-\Lambda x_{0}\|<\epsilon. $$EXAMPLES OF BANACH SPACE TECHNIQUES $97$ implies (a). But then the linearity of $\Lambda$ shows that $|\mathbb{A}x|<\epsilon.$ Hence $\|\Lambda\|\le\epsilon/\delta,$ and(c) // Consequences of Baire's Theorem 5.5 The manner in which the completeness of Banach spaces is frequently exploited depends on the following theorem about complete metric spaces, which also has many applications in other parts of mathematics. It implies two of the three most important theorems which make Banach spaces useful tools ir analysis, the Banach-Steinhaus theorem and the open mapping theorem. The third is the Hahn-Banach extension theorem, in which completeness plays no role. 5.6 Baire's Theorem If $X$ is a complete metric space, the intersection of ever countable collection of dense open subsets of $X$ is dense in $X.$ In particular (except in the trivial case $X={\mathcal{O}},$ the intersection is not empty This is often the principal significance of the theorem. set in $X.$ PRoor Suppose $V_{\mathrm{i}}$ $V_{2},\,V_{3},\,\ldots$ $X{\dot{\cdot}}$ Y;let us write has a point in $\textstyle W$ if $W\neq{\mathcal{O}}.$ be any open are dense and open in $X.$ Let ${\boldsymbol{W}}$ Let $\boldsymbol{\rho}$ . We have to show that $\bigcap{V_{n}}$ p be the metric of $$ S(x,\,r)=\{y\in X\colon\rho(x,\,y)<r\} $$ (1) and let ${\tilde{S}}(x,$ r) be the closure of $\scriptstyle5e,\,n$ [Note: There exist situations in which ${\tilde{S}}(x,\,r)$ does not contain all $\scriptstyle{\mathcal{Y}}$ with $\rho(x,y)\leq r\uparrow\rfloor$ Since ${\mathit{V}}_{1}$ is dense, $W\cap V_{1}$ is a nonempty open set, and we can therefore find $x_{1}$ and $r_{1}$ such that $$ \tilde{S}(x_{1},\,r_{1})\subset W\cap V_{1}\quad\mathrm{and}\quad0<r_{1}<1. $$ (2) If $\scriptstyle n\geq2$ and $x_{n-1}$ and $\textstyle r_{n-1}$ are chosen, the denseness of ${\textstyle\bigwedge}_{\mathrm{{H}}}$ shows that $V_{n}\cap$ $S(x_{n-1},\,r_{n-1})$ is not empty, and we can therefore find $\scriptstyle x_{n}$ , and ${\boldsymbol{r}}_{n}$ such that $$ \bar{S}(x_{n},\,r_{n})\subset{V_{n}}\cap{S(x_{n-1},\,r_{n-1})}\quad\mathrm{and}\quad0<r_{n}<\frac{1}{n}. $$ (3) $j>n,$ plete, there is a point ${\mathfrak{x}}\in X$ By induction, this process produces a sequence ${\boldsymbol{x}}_{i}$ and $x_{j}$ both lie in $S(x_{n},\,r_{n}),$ If $\scriptstyle{i\approx n}$ and $\rho(x_{i},\,x_{j})<2r_{n}<2/n,$ and hence such that $x=\operatorname*{lim}\;x_{n}.$ $\{x_{n}\}$ in $X.$ so that the construction shows that $\left\{x_{n}\right\}$ is a Cauchy sequence. Since $X$ is com- Since x ${\boldsymbol{x}}_{i}$ x: lies in the closed set ${\tilde{S}}(x_{n},\,r_{n})$ n→0 it follows that $\scriptstyle{\mathcal{X}}$ lies in each // ${\tilde{S}}(x_{n},\,r_{n}),$ and (3) shows that i $\iota>n,$ $x\in W.$ This completes the proof. $\scriptstyle{X}$ lies in each $V_{n}.$ By (2), Corollary In a complete metric space, the intersection of any countable collec- tion of dense $G_{\delta}^{\cdot}s$ is again a dense $G_{\delta}\ .$98 REAL AND coMPLEX ANALYSIs This follows from the theorem, since every $G_{\delta}$ is the intersection of a count able collection of open sets, and since the union of countably many countable sets is countable. 5.7 Baire's theorem is sometimes called the category theorem, for the following reason Call a set $\scriptstyle{F\in{\mathcal{X}}}$ nowhere dense if its closure $\overline{{E}}$ contains no nonempty open subset of $X$ . Any countable union of nowhere dense sets is called a set of the first category; all other subsets of $X$ are of the second category (Baire's terminology) Theorem 5.6 is equivalent to the statement that no complete metric space is of the first category. To see this, just take complements in the statement of Theorem 5.6. 5.8 The Banach-Steinhaus Theorem Suppose X is a Banach space, ${\bf Y_{\nu}}$ is a normed linear space, and $\scriptstyle\{\Lambda_{3}\}$ is a collection of bounded linear transformations of $\textstyle X{\mathrm{~}}$ into Y, where α ranges over some index set A. Then either there exists an $M<\infty$ such that $$ |\Lambda_{x}|\leq M $$ (1) for every α∈ A, or $$ \operatorname*{sup}_{\alpha\circ A}|\Lambda_{\alpha}x||=\infty $$ (2) for all x belonging to some dense $G_{\delta}$ 。in $X.$ In geometric terminology, the alternatives are as follows: Either there is a ball $\boldsymbol{B}$ in ${\mathbf{}}Y$ (with radius A ${\cal{M}}$ and center at O) such that every $\Lambda_{\alpha}$ ,maps the unit ball of $X$ into ${\boldsymbol{B}},$ 3, or there exist $x\in{\mathcal{X}}$ (Gin fact, a whole dense $G_{\delta}$ of them) such that no ball in ${\bf Y_{\nu}}$ contains ${\underset{\mathrm{A}}{\operatorname{A}x}}$ x for all α. The theorem is sometimes referred to as the uniform boundedness principle PROOF Put $$ \varphi(x)=\operatorname*{sup}_{x\in A}\|\Lambda_{\alpha}x\|\qquad(x\in X) $$ (3) and let $$ V_{n}=\{x;\,\varphi(x)>n\}\qquad(n=1,\,2,\,3,\,\ldots). $$ (4) Since each $\Lambda_{\alpha}$ is continuous and since the norm of ${\mathbf{}}Y$ is a continuous function on ${\mathbf{}}Y$ (an immediate consequence of the triangle inequality, as in the proof of Theorem 4.6), each function $x\to\|\Lambda_{x}x\|$ is continuous on $X.$ Hence c g is lower semicontinuous, and each V $\varphi$ ${\mathit{V}}_{n}$ , is open.EXAMPLES OF BANACH SPACE TECHNiQUEs 99 $x_{*}\in X$ If one of these sets,say $V_{N}{\mathrm{:}}$ fails to be dense in $X,$ then there exist an and an $\scriptstyle\gamma\simeq0$ such that l|xll ≤r implies $x_{0}+x\notin V_{N}$ ; this means that $\varphi(x_{0}+x)\leq N,$ or $$ \|\Lambda_{\alpha}(x_{0}+x)\|\leq N $$ (5) for all α∈ A and all $\scriptstyle{X}$ with $\|x\|\leq r.$ Since ${\boldsymbol{x}}=(x_{0}+{\boldsymbol{x}})-x_{0}\,,$ we then have $$ \|\Lambda_{\alpha}x\|\leq\|\Lambda_{\alpha}(x_{0}+x)\|+\|\Lambda_{\alpha}x_{0}\|\leq2N, $$ (6) dense $G_{\delta}$ in $X.$ and it follows that (1) holds with $M=2N/r$ is dense in $X.$ In that case,Q V, is a ${\dot{x}}\not\in\bigcap V_{n},$ // The other possibility is that every ${\mathit{V}}_{n}$ for every by Baire's theorem; and since $\varphi(x)=\infty$ the proof is complete. 5.9 The Open Mapping Theorem Let $U$ and ${\mathbf{}}V$ be the open unit balls of the Banach spaces $X$ and ${\boldsymbol{Y}}.$ To every bounded linear transformation A of $X$ onto Y there corresponds a $\delta>0$ so that $$ \Lambda(U)\supseteq\delta V. $$ (1) Note the word“onto”in the hypothesis. The symbol $\delta V$ stands for the set $y\colon y\in V\}.$ i.e., the set of all $\scriptstyle\gamma\in\gamma$ with $|y|<\delta.$ that the image of every open ball in It follows from (1) and the linearity of $\Lambda$ , with center at $x_{0}.$ io,say, contains an open ball in ${\mathbf{}}Y$ with center at $\Lambda x_{0}$ Hence e image of every open set is open. This explains the name of the theorem with $|y|<\delta$ there corresponds Here is another way of stating (1): To every $\mathbf{\vec{y}}$ x with $\|x\|<1$ so that $\Lambda x=y.$ PRoor Given y ${\boldsymbol{W}}$ is the limit of a sequence Hence ${\mathbf{}}Y$ is the union of the sets A(kU)、for it $\boldsymbol{k}$ $k=1,\,2,\,3,\,\ldots.$ Since ${\mathbf{}}Y$ there exists an $\operatorname{xe}{\mathcal{X}}$ such that $\Lambda x=y;\;\mathrm{if}\;\left\|x\right\|<k,$ ${\mathfrak{E}}\cap{\mathfrak{L}}.$ follows that $y\in\Lambda(k U).$ is complete, the Baire theorem implies that there is a point of nonempty open set $\textstyle W$ in the closure of some $\textstyle \lfloor\operatorname{Ar}_{\mathrm{i}} \rfloor$ ,where $x_{i}\in k U;$ from now on, $\Lambda(k U),$ This means that every and any such $\mathbf{\vec{y}}$ are fixed. $y_{0}\in W,$ and choose $\scriptstyle n\,>0$ so that $y_{0}+y\in W$ if $\|y\|<\eta$ .For ${\boldsymbol{W}}$ Choose there are sequences $\{x_{i}^{n}\},\{x_{i}^{n}\}\operatorname{in}$ $\textstyle k U$ such that $$ \Lambda x_{i}^{\prime} arrow y_{0}\,,\qquad\Lambda x_{i}^{\prime\prime} arrow y_{0}+y\qquad(i arrow\infty). $$ (2) Setting $x_{i}=x_{i}^{\prime\prime}-x_{i}^{\prime}.$ we have $\left\|x_{i}\right\|<2k$ $\Lambda$ shows that the following is true, if every $\mathbf{\vec{y}}$ with $|y|<\eta,$ the linearity of and $\Lambda x_{i} arrow y_{i}$ Since this holds for $\delta=\eta/2k$ and to each $\scriptstyle\epsilon\;>0$ there corresponds an $x\in{\mathcal{X}}$ such that To each $\scriptstyle y\in Y$ $$ \|x\|\le\delta^{-1}\|y\|\quad a n d\quad\|y-\Lambda x\|<\epsilon. $$ (3) This is almost the desired conclusion, as stated just before the start of the proof, except that there we had $\scriptstyle x\;=\;0$100 REAL AND coMPLEX ANALYSis Fix y eSV, and fixe > 0. By (3) there exists an $x_{1}$ with $\left\|x_{1}\right\|<1$ and $$ \|y-\Lambda x_{1}\|<\textstyle{\frac{1}{2}}\delta\epsilon. $$ (4) Suppose $X_{1}\!_{}^{},\,\cdot\cdot\cdot\cdot\cdot\times\!_{}n$ are chosen so that $$ \|y-\Lambda x_{1}-\cdot\cdot\cdot-\Lambda x_{n}\|<2^{-n}\delta\epsilon. $$ (5) Use (3), with $\scriptstyle{y}$ replaced by the vector on the left side of (5), to obtain an $x_{n+1}$ so that (5) holds with $n+1$ in place of r ${\mathfrak{n}},$ n, and $$ \|x_{n+1}\|<2^{-n}\epsilon\qquad(n=1,\,2,\,3,\,\ldots). $$ (6) $X.$ If we set $\textstyle X$ is complete, there exists an ${\mathfrak{x}}\in X$ so that $S_{n}\to X$ . The inequality Since $s_{n}=x_{1}+\cdot\cdot\cdot+x_{n},$ (6) shows that $\left\{S_{n}\right\}$ is a Cauchy sequence in $\|x_{1}\|<1,$ together with (6), shows that Hence $\Lambda x=y$ . Since A is continuous, $\|x\|<1+\epsilon$ $\Lambda s_{n}{ arrow}$ Ax. By (5), $\Lambda s_{n}{ arrow}\;y.$ We have now proved that $$ \Lambda((1+\epsilon)U)\supset\delta V, $$ (7) or $$ \Lambda(U) arrow(1+\epsilon)^{-1}\delta V, $$ (8) for every $\mathbf{\hat{\mathbf{e}}}\approx0.$ The union of the sets on the right of (8), taken over all $\scriptstyle\epsilon\;>0,$ is BV. This proves (1). // 5.10 Theorem J $\textstyle X$ and ${\mathbf{}}Y$ are Banach spaces and if $\mathrm{\A}$ is a bounded linear transformation of $X$ onto ${\bf Y_{\nu}}$ which is also one-to-one, then there is a $\scriptstyle\delta>0$ such that $$ \|\Lambda x\|\geq\delta\|x\| $$ (1) In other words, $\Lambda^{-1}$ is a bounded linear transformation of ${\mathbf{}}Y$ onto $X.$ PRoor If T $\delta$ is chosen as in the statement of Theorem 5.9, the conclusion of that theorem, combined with the fact that $\mathrm{A}$ is now one-to-one, shows that $|\mathbf{A}x||<\delta$ implies $\|x\|<1.$ Hence $\|x\|\geq1$ implies $|\mathbf{A}x|\geq\delta,$ and(1))is proved. The transformation $\stackrel{\wedge^{-}}{\sim}$ is defined on ${\bf Y_{\nu}}$ by the requirement that $\Lambda^{-1}y=x$ if $y=\Lambda x.$ A trivial verification shows that $\scriptstyle3^{-1}$ is linear, and(1) // implies that $\|\Lambda^{-1}\|\leq1/\delta.$ Fourier Series of Continuous Functions 5.11 A Convergence Problem Is it true for everyfe C(T) that the Fourier series o f converges to f(x) at every point x?EXAMPLES OF BANACH SPACE TECHNIQUES 101 Let us recall that the nth partial sum of the Fourier series of f at the point x is given by $$ s_{n}(f;\,x)=\frac{1}{2\pi}\, [\O_{-\pi}^{\pi}f(t)D_{n}(x-t)\ d t\qquad(n=0,\,1,\,2,\,...), $$ (1) where $$ D_{n}(t)=\sum_{k=-n}^{n}e^{i k t}. $$ (2) This follows directly from formulas 4.26(1) and 4.26(3) The problem is to determine whether $$ \operatorname*{lim}_{n\to\infty}s_{n}(f;\,x)=f(x) $$ (3) for every $f\in C(T)$ and for every real x. We observed in Sec. 4.26 that the partial sums do converge to $\boldsymbol{\mathit{f}}$ in the ${\boldsymbol{L}}^{2}$ -norm, and Theorem 3.12 implies therefore that each $f\in L^{n}(T)$ [hence also each fe C(T)] is the pointwise limit a.e. of some sub- sequence of the full sequence of the partial sums. But this does not answer the present question. We shal eethat the Banach-Steinhaus theorem answers the question nega- tively. Put $$ s^{*}(f;\,x)=\operatorname*{sup}_{n}\,{\vert s_{n}(f;\,x)\vert}\,. $$ (4) To begin with, take $\scriptstyle x\;=\;0,$ and define $$ \Lambda_{n}f=s_{n}(f;\,0)\qquad(f\in C(T);\,n=1,\,2,\,3,\,\ldots). $$ (5) We know that $\operatorname{c}(n)$ is a Banach space, relative to the supremum norm $\|f\|_{\alpha}\cdot\mathbf{k}$ follows from (1) that each $\Lambda_{n}$ is a bounded linear functional on $\operatorname{c}(n)_{\mathrm{s}}$ of norm $$ \|\Lambda_{n}\|\leq{\frac{1}{2\pi}}\int_{-\pi}^{\pi}|D_{n}(t)|\ d t=\|D_{n}\|_{1}. $$ (6) We claim that $$ \|\Lambda_{n}\|\to\infty\qquad\mathrm{as~}n\to\infty. $$ (7 This will be proved by showing that equality holds in (6) and that $$ \|D_{n}\|_{1} arrow\infty\qquad\mathrm{as~}n arrow\infty. $$ (8) Multiply (2) by $\!\,e^{i t/2}$ and by $e^{-i t/2}$ and subtract one of the resulting two equa tions from the other, to obtain $$ D_{n}(t)={\frac{\sin{(n+{\frac{1}{2}})t}}{\sin{(t/2)}}}. $$ (9)102 REAL AND coMPLEX ANALYSIS Since |sin $x|\leq|x|$ for all real x,(9) shows that $$ \|D_{n}\|_{1}>{\frac{2}{\pi}}\int_{0}^{\pi}{\bigg|}\sin{\bigg(}n+{\frac{1}{2}}{\bigg)}t{\bigg|}{\frac{d t}{t}}={\frac{2}{\pi}}\int_{0}^{(n+1/2)\pi}|\sin t|\,{\frac{d t}{t}} $$ $$ \mathrm{>}{\frac{2}{\pi}}\sum_{k=1}^{n}{\frac{1}{k\pi}}\,\ |_{(k-1)\pi}^{k\pi}|\sin\,t|\ d t={\frac{4}{\pi^{2}}}\sum_{k=1}^{n}{\frac{1}{k}}\to\infty, $$ which proves (8). $f_{j}\in C(T)$ Next, fix n, and put $g(t)=1$ if $D_{n}(t)\geq0,\;g(t)=-1$ if D,At)< 0. There exist such that $-1\leq f_{j}\leq1$ and fJt)→g(t) for every t asj→00. By the domi- nated convergence theorem i一 li $$ \operatorname*{m}_{\bf\Phi}\Lambda_{n}(f_{j})=\operatorname*{lim}_{j arrow\infty}\frac{1}{2\pi}\left.\right>_{-\pi}^{\pi}f_{j}(-t)D_{n}(t)\,d t=\frac{1}{2\pi}\,\int_{-\pi}^{\pi}g(-t)D_{n}(t)\,d t=\|D_{n}\|_{1}. $$ Thus equality holds in(6) and we have proved (T) for every fin some dense $G_{\delta}{}^{\circ}$ Since (7) holds, the Banach-Steinhaus theorem asserts now that $s^{*}(f;0)=\infty$ We chose $x=0$ ;-set in $\operatorname{c}(n)$ just for convenience. It is clear that the same result holds for every other ${\mathbf{}}x\,;$ To each real number x there corresponds a set $E_{x}\in C(T)$ which is a dense $G_{\delta}$ in $\operatorname{c}(n)$ such that $s^{*}(f;x)=\circ5o r\;e v e r y f\in E_{x}.$ In particular, the Fourier series of each $f\in E_{x}$ diverges at x, and we have a negative answer to our question.(Exercise 22 shows the answer is positive if mere continuity is replaced by a somewhat stronger smoothness assumption.) It is interesting to observe that the above result can be strengthened by and let $\boldsymbol{E}$ another application of Baire's theorem. Let us take countably many points $x_{i}\,,$ be the intersection of the corresponding sets $$ E_{x_{i}}\subset C(T). $$ By Baire's theorem, $\boldsymbol{E}$ is a dense $G_{\delta}$ in C(T). Every f∈ $\boldsymbol{E}$ has $$ s^{*}(f;x_{i})=\infty $$ at every point ${\boldsymbol{x}}_{i}$ For each f, s*(f;x) is a lower semicontinuous function of x, since (4) exhibits it as the supremum of a collection of continuous functions. Hence $\{x\colon s^{\ast}(f;x)=\infty\}$ is a $G_{\mathcal{S}_{\mathcal{S}}}$ , in $R^{1},$ for each f. If the above points ${\boldsymbol{x}}_{i}$ ; are taken so that their union is dense in $(-$ 元,7), we obtain the following result: 5.12 Theorem There is a set $E\subset C(T)$ which is a dense $G_{\delta}$ in $\operatorname{c}(n)$ and which has the following property: For $r a c h f\in E,$ the set $$ Q_{f}=\{x\colon s^{*}(f;x)=\infty\} $$ is a dense $G_{\partial}$ in $R^{1},$EXAMPLES OF BANACH SPACE TECHNiQUEs 103 This gains in interest if we realize that ${\boldsymbol{E}},$ as well as each $Q_{f},$ is an uncount able set: 5.13 Theorem In a complete metric space $X$ which has no isolated points, no countable dense set is a $G_{\delta}$ a PROOF Let ${\boldsymbol{x}}_{k}$ be the points of a countable dense set ${\mathit{V}}_{n}$ is dense and open. Let in $X.$ Assume that $\boldsymbol{E}$ E is $\boldsymbol{E}$ $G_{\delta}.$ Then $E=\bigcap$ V, where each $$ W_{n}=V_{n}-\bigcup_{k=1}^{n}\{x_{k}\}. $$ Then each $W_{n}$ is still a dense open set, but $\bigcap{}$ $W_{n}={\mathcal{D}},$ in contradiction to // Baire's theorem. Note: A slight change in the proof of Baire's theorem shows actually that every dense $G_{\delta}$ contains a perfect set if $X$ is as above. Fourier Coeficients of ${\boldsymbol{L}}^{1}$ -functions 5.14 As in Sec.4.26, we associate to every f ∈ $L^{1}(T)$ a function fon Z defined by $$ {\hat{f}}(n)={\frac{1}{2\pi}}\,\int_{-\pi}^{\pi}f(t)e^{-i n t}\,d t\qquad(n\in Z). $$ (1) $\operatorname{c}(\eta)$ It is easy to prove $\operatorname{that}f(n)\to0$ as $|n|\to\infty,$ for every $\scriptstyle f\in L$ For we know that is dense in $\scriptstyle{I(T)}$ (Theorem 3.14) and that the trigonometric polynomials are dense in $\operatorname{c}(\eta)$ (Theorem 4.25).If $\scriptstyle x\,>0$ and $f\in L^{n}(T),$ this says that there is a $\|g-P\|_{\alpha}<\epsilon.$ and a trigonometric polynomial ${\mathbf{}}P$ such that $\|f-g\|_{1}<\epsilon$ and $g\in C(T)$ Since $$ \|g-P\|_{1}\leq\|g-P\|_{\alpha} $$ if follows that $\|f-P\|_{1}<2\epsilon;\mathrm{~and~if~}|n|$ is large enough (depending on $P\rangle,$ then $$ \mid\hat{f}(n)\mid=\left|\frac{1}{2\pi}\right.\left|_{-\pi}^{\pi}\{f(t)-P(t)\}e^{-i n t}\;d t\right|\leq\left\|f-P\right\|_{1}<2\epsilon. $$ (2) T ${\mathrm{hus}}{\hat{f}}(n)\to0$ as $n\to\pm\varnothing\quad$ This is known as the Riemann-Lebesgue lemma The question we wish to raise is whether the converse is true. That is to say if $\{a_{n}\}$ is a sequence of complex numbers such that such $\operatorname{that}{\hat{f}}(n)=a_{n}$ for all $*\in Z^{\prime}$ In other words, does it $a_{n}{ arrow}0$ as $n\to\pm\infty.$ follow that there is $\operatorname{an}f\in L^{1}(T)$ is something like the Riesz-Fischer theorem true in this situation? This can easily be answered (negatively) with the aid of the open mapping theorem.104 REAL AND COMPLEX ANALYSIS Let $c_{\mathrm{0}}$ be the space of all complex functions p on $\scriptstyle{\mathcal{Z}}$ such that $\varphi(n)\to0$ as n→±0, with the supremum norm $$ \|\varphi\|_{\infty}=\operatorname*{sup}{\big\{}|\varphi(n)|\colon n\in\mathbb{Z}{\big\}}. $$ (3) Then $c_{\mathrm{0}}$ is easily seen to be a Banach space. In fact, if we declare every subset of $\scriptstyle{\mathcal{Z}}$ to be open, then $\scriptstyle{\mathcal{Z}}$ is a locally compact Hausdorff space, and $c_{\mathrm{o}}$ is nothing but $C_{0}(Z).$ The following theorem contains the answer to our question: 5.15 Theorem The mapping f→fis $\bar{a}$ one-to-one bounded linear transformation of $L^{1}(T)$ into (but not onto) $c_{\mathrm{o}}$ that $\mathrm{\A}$ PRoOF Define A by $\mathbb{N}f=f,\[t$ is clear that $\Lambda$ is linear. We have just proved Then SO that that A maps $L^{1}(T)$ into $c_{\mathrm{0}}\,,$ and formula_ 5.14(1) shows that $|\hat{f}(n)|\leq\|f\|_{1},$ $\left\|\mathbf{A}\right\|\leq1.$ (Actually $\|\Lambda\|=1;$ to see this, take f = 1.) Let us now prove $n\in\mathbb{Z}.$ is one-to-one. Suppose f ∈ $L^{1}(T)$ $\operatorname{and}{\hat{f}}(n)=0$ for every $$ \bigcap_{t=\pi}^{\pi}f(t)g(t)\ d t=0 $$ (1) if $\scriptstyle{\mathcal{G}}$ is any trigonometric polynomial. By Theorem 4.25 and the dominated convergence theorem,(1) holds for every $g\in C(T).$ Apply the dominated con- vergence theorem once more, in conjunction with the Corollary to Lusin's theorem, to conclude that(1) holds if $\mathbf{\Omega}^{g}$ is the characteristic function of any of a measurable set in ${\boldsymbol{T}}.$ Now Theorem 1.39(b) shows ${\mathrm{flat}}/=0$ a,e. $\scriptstyle\delta>0$ If the range of $\Lambda$ were all of $c_{\mathrm{0}}$ , Theorem 5.10 would imply the existence such that $$ \|{\hat{f}}\|_{\alpha}\geq\delta\|f\|_{1} $$ (2) for every $f\in L(T).$ But if $D_{n}(t)$ is defined as in Sec. 5.11, then $n\to G\to$ Hence there is no $|\tilde{D}_{n}|_{\infty}=1$ for $\scriptstyle n\;=\;1.$ 2,3, …… and $\|D_{n}\|_{1}\to\varnothing$ as $D_{n}\in L^{1}(T),$ $\scriptstyle{\delta\,>\,0}$ such that the inequalities $$ \|{\vec{D}}_{n}\|_{\alpha}\geq\delta\|D_{n}\|_{1} $$ (3) hold for every n. This completes the proof. // The Hahn-Banach Theorem 5.16 Theorem If $\textstyle{M}$ is a subspace of a normed linear space $X$ and iff is a bounded linear functional on $\textstyle{M},$ then $\boldsymbol{\f}$ can be extended to a bounded linear functional ${\mathbf{}}F$ on $X$ Y so that $\|F\|=\|f\|.$ Note that $\textstyle{M}$ need not be closed.EXAMPLES OF BANACH SPACE TECHNIQUEs 105 Before we turn to the proof, some comments seem called for. First, to say Gin the most general situation) that a function ${\mathbf{}}F$ is an extension of $\boldsymbol{\f}$ means that the domain of ${\mathbf{}}F$ includes that of $\boldsymbol{\f}$ and that $F(x)=f(x)$ for all $\scriptstyle{\mathcal{X}}$ in the domain of f $f;$ Second, the norms $\|F\|$ and $\|f\|$ are computed relative to the domains of ${\mathbf{}}F$ F and explicitly, l f = sup {If(x)|: xe M,I|xll ≤ 1}, l|F"| = sup {[F(x)|:xe X $\left|x\right|\leq1 \rangle,$ The third comment concerns the field of scalars. So far everything has beer stated for complex scalars, but the complex field could have been replaced by the real field without any changes in statements or proofs. The Hahn-Banach theorem is also true in both cases; nevertheless, it appears to be essentially a “t real” theorem. The fact that the complex case was not yet proved when Banach wrote his classical book“Opérations linéaires”may be the main reason that real scalars are the only ones considered in his work It will be helpful to introduce some temporary terminology. Recall that ${\mathbf{}}V$ is a complex (real) vector space if $x+y\in V$ for $\scriptstyle{\mathcal{X}}$ and $y\in V$ /, and if $a x\in V$ for all complex (real) numbers α. It follows trivially that every complex vector space is also a real vector space. A complex function $\varphi$ on a complex vector space ${\mathbf{}}V$ is a complex-linear functional if $$ \varphi(x+y)=\varphi(x)+\varphi(y)\quad{\mathrm{~and~}}\quad\varphi(\alpha x)=\alpha\varphi(x) $$ (1) for all $\scriptstyle{\mathcal{X}}$ x and $y\in V$ and all complex α.A real-valued function $\varphi$ on a complex (real) vector space ${\mathbf{}}V$ is a real-linear functional if $x\in V,$ it is easily seen that ${\mathfrak{f}},$ i.e., if $u(x)$ is the real part $(1)$ holds for all real a. If uis the real part of a complex-linear functional of the complex number $f({\boldsymbol{x}})$ for all $\boldsymbol{u}$ is a real-linear functional. The following relations hold between f and u: 5.17 Proposition Let ${\mathbf{}}V$ be a complex vector space (a)If uis the real part of a complex-linear functional f on V, then $$ f(x)=u(x)-i u(i x)\qquad(x\in V). $$ (1) (b)If u is a real-linear functional on ${\mathbf{}}V$ and if f is defined by (1),then $\boldsymbol{\f}$ is a complex-linear functional on $V.$ (c)If ${\mathbf{}}V$ is a normed linear space and $\boldsymbol{\f}$ and u are related as in(1),then $|f\,|=||u||.$ PRooF If α and $\beta$ are real numbers and $z=\alpha+i\beta,$ , the real part of iz is $-\beta.$ This gives the identity $$ z=\mathrm{Re}\ z-i\mathrm{Re}\ (i z) $$ (2) for all complex numbers z.Since $$ \mathrm{Re}\,(i f(x))=\mathrm{Re}\,f(i x)=u(i x), $$ (3) (1) follows from $(2)$ with $z=f(x).$106 REAL AND COMPLEX ANALYSis Under the hypotheses (b), it is clear $\mathrm{that}\,f(x+y)=f(x)+f(y)$ and that $f(\alpha x)=\alpha f(x)$ for all real α. But we also have $$ f(i x)=u(i x)-i u(-x)=u(i x)+i u(x)=i f(x), $$ (4) which proves that fis complex-linear. Then Since $|u(x)|\leq|f(x)|,$ we have $\|u\|\leq\|f\|,$ On the other hand, to every ${\mathcal{I}}(x)=|f(x)|.$ $x\in V$ there corresponds a complex number $x,|\alpha|=1,$ so that $$ |f(x)|=f(\alpha x)=u(\alpha x)\leq\|u\|\cdot\|\alpha x\|=\|u\|\cdot\|x\||, $$ (5) which proves that $\|f\|\leq\|u\|,$ // 5.18 Proof of Theorem 5.16 We first assume that $X$ is a real normed linear space and, consequently, that $\boldsymbol{\mathsf{f}}$ is a real-linear bounded functional on $M.$ If $\|f\|=0,$ the desired extension is $\scriptstyle{F=0}$ Omitting this case, there is no loss of and $\lambda$ generality in assuming that $M_{1}$ consists of all vectors of the form $M_{1}$ be the vector space spanned by ,where $x\in M$ ${\cal{M}}$ $\|f\|=1.$ Choose $x_{0}\in X,\,x_{0}\notin M,$ and let and $\scriptstyle{X_{0}}$ Then $x+\lambda x_{0}$ where α is any is a real scalar. If we define $f_{1}(x+\lambda x_{0})=f(x)+\lambda x,$ to a linear fixed real number, it is trivial to verify that an extension of $\boldsymbol{\f}$ functional on $M_{1}$ is obtained. The problem is to choose α so that the extended functional still has norm 1.This will be the case provided that $$ |f(x)+\lambda\alpha|\leq\|x+\lambda x_{0}\|\qquad(x\in M,\lambda\ {\mathrm{real}}). $$ (1) Replace x by 一x and divide both sides of(1) by |入]. The requirement is then that $$ |f(x)-\alpha|\leq\|x-x_{0}\|\qquad(x\in M), $$ (2) i.e., that $A_{x}\leq\alpha\leq B_{x}$ for all $x\in M,$ where $$ A_{x}=f(x)-\|x-x_{0}\|\quad{\mathrm{and}}\quad B_{x}=f(x)+\|x-x_{0}\|. $$ (3) There exists such an $^{\alpha}$ if and only if all the intervals [A,, $B_{x}]$ have a common point,i.e., if and only f $$ A_{x}\leq B_{y} $$ (4) for all $\scriptstyle{\mathcal{X}}$ and $y\in M.$ But $$ f(x)-f(y)=f(x-y)\leq\|x-y\|\leq\|x-x_{0}\|+\|y-x_{0}\|, $$ (5) and so (4) follows from (3) We have now proved that there exists a norm-preserving extension $f_{1}$ of f on $M_{1}$ that Let $\mathcal{P}$ be the collection of all ordered pairs $f^{\prime}$ is a real-linear extension of f to is a sub- $M^{\prime},$ with $M^{\prime}\subset M^{\prime\prime}$ and $f^{\prime\prime}(x)=f^{\prime}(x)$ for all $x\in M^{\prime}$ $(M^{\prime},f^{\prime}),$ where $M^{\prime}$ to mean space of $X$ which contains $\textstyle{M}$ and where $(M^{\prime},f^{\prime})\leq(M^{\prime\prime},f^{\prime\prime})$ $\|f^{\prime}\|=1.$ Partially order $\mathcal{P}$ by declaring The axioms of a partial orderEXAMPLES OF BANACH SPACE TECHNiQUEs 107 are clearly satisfied, ${\mathcal{P}}$ is not empty since it contains $(M,f),$ and so the Haus dorff maximality theorem asserts the existence of a maximal totally ordered subcollection $\Omega$ of ${\mathcal{P}},$ Let $\mathbf{\hat{\Phi}}$ be the collection of all $M^{\prime}$ such that $(M_{\bar{z}}^{\prime}f^{\prime})\in\Omega.$ Then $\Phi$ is totally is a ordered, by set inclusion, and therefore the union M of all members of $\Phi$ subspace of $X.$ (Note that in general the union of two subspaces is not a subspace. An example is two planes through the origin in $R^{3}.$ 3.) If $x\in{\hat{M}}$ ,then $x\in M^{\prime}$ for some M' e O; define $F(x)=f^{\prime}(x),$ where $f^{\prime}$ is the function which occurs in the pair $(M^{\prime},f^{\prime})\in\Omega.$ Our definition of the partial order in $\scriptstyle{\vec{F}}(x)$ as long as Q shows $M^{\prime}$ $\Omega$ that it is immaterial which $M^{\prime}\in\Phi$ we choose to define contains $\mathbf{x}.$ If $\vec{\cal M}$ It is now easy to check that ${\mathbf{}}F$ is a linear functional on ${\widetilde{M}},$ with $\|F\|=1.$ were a proper subspace $X,$ the first part of the proof would give us a 2. Thus further extension of ${\boldsymbol{F}},$ and this would contradict the maximality of $\Omega$ ${\tilde{M}}=X.$ , and the proof is complete for the case of real scalars If now fis a complex-linear functional on the subspace ${\cal{M}}$ of the complex normed linear space $X,$ let u be the real part of f, use the real Hahn-Banach define theorem to extend uto a real-linear functional $U$ on $X,$ with $\|U\|=\|u\|,$ and $$ F(x)=U(x)-i U(i x)\qquad(x\in X). $$ (6) By Proposition 5.17, ${\mathbf{}}F$ is a complex-linear extension ${\mathfrak{o f}}{\mathfrak{f}},$ and $$ \|F\|=\|U\|=\|I\|. $$ This completes the proof. // Let us mention two important consequences of the Hahn-Banach theorem: 5.19 Theorem Let I ${\cal{M}}$ be a linear subspace of a normed linear space X, and let $x_{\mathrm{n}}\in X$ Then $x_{0}$ is in the closure $\bar{M}$ of ${\cal{M}}$ 1 if and only if there is no bounded linear functional fon $X$ such that $f(x)=0$ for al $x\in M$ but $f(x_{0})\neq0.$ PRoor If $x_{0}\in{\tilde{M}},f$ is a bounded linear functional on $\lambda$ is a scalar. Since and $f(x)=0$ for all and define $f(x+\lambda x_{0})=\lambda$ the continuity of f shows that we also have $X,$ $\delta>0$ such that ${\boldsymbol{x}}_{0}\,,$ $x\in M,$ Conversely, suppose $x_{0}\notin{\hat{M}}.$ $f(x_{0})=0.$ and Then there exists a $\|x-x_{0}\|>\delta$ for all $x\in M.$ Let $M^{\prime}$ be the subspace generated by ${\cal M}$ if $x\in M$ and $$ \delta\mid\lambda\mid\leq\mid\lambda\mid\Vert x_{0}+\lambda^{-1}x\Vert=\Vert\lambda x_{0}+x\Vert, $$ we see that $\boldsymbol{\mathsf{f}}$ is a linear functional on $M^{\prime}$ whose norm is at most $\delta^{-1}.$ Also f from $M^{\prime}$ to $X$ $M,f(x_{0})=1.$ The Hahn-Banach theorem allows us to extend this / $f(x)=0$ on 5.20 Theorem If $X$ ’is a normed linear space and i $x_{\mathrm{n}}\in X$ $x_{0}\neq0,$ there is a bounded linear functional f on $X,$ of norm 1, so that $f(x_{0})=\|x_{0}\|.$108 REAL AND CoMPLEX ANALYSIs PRoOF Let $M=\{\lambda x_{0}\}$ and define $f(\lambda x_{0})=\lambda\|x_{0}\|.$ Then f is a linear function al of norm 1 on $M.$ and the Hahn-Banach theorem can again be applied. / 5.21 Remarks If $\textstyle X$ is a normed linear space, let $X^{\bullet}$ be the collection of all bounded linear functionals on $X.$ If addition and scalar multiplication of linear functionals are defined in the obvious manner, it is easy to see that $X^{\bullet}$ is again a normed linear space. In fact, $X^{\bullet}$ is a Banach space; this follows from the fact that the field of scalars is a complete metric space. We leave the verification of these properties of $X^{\bullet}$ as an exercise. One of the consequences of Theorem 5.20 is that $X^{\bullet}$ is not the trivial vector space (i.e.,, $X^{\bullet}$ consists of more than O) if $X$ is not trivial. In fact, $X^{\bullet}$ such separates points on $X.$ This means that if $x_{1}\neq x_{2}$ in $X$ there exists ${\mathfrak{a n}}f\in X^{*}$ $\mathrm{that}\,f(x_{1})\neq f(x_{2})$ .To prove this, merely take $x_{0}=x_{2}-x_{1}$ in Theorem 5.20. Another consequence is that, for ${\mathfrak{c}}\in X.$ $$ \|x\|=\operatorname*{sup}\;\{\,|f(x)|\!:f\in X^{*},\ \|f\|=1\}. $$ on Hence, for fixed ${\mathfrak{x e}}\,X.$ the mapping and $X^{\bullet}$ (the so-called“dual space”of $X)$ $X^{*}{\mathrm{:}}$ of norm This interplay between $\textstyle X$ $f\to f(x)$ is a bounded linear functional $\|x\|.$ forms the basis of a large portion of that part of mathematics which is known as functional analysis. An Abstract Approach to the Poisson Integral 5.22 Successful applications of the Hahn-Banach theorem to concrete problems depend of course on a knowledge of the bounded linear functionals on the normed linear space under consideration. So far we have only determined the bounded linear functionals on a Hilbert space(where a much simpler proof of the Hahn-Banach theorem exists; see Exercise $6),$ and we know the positive linear functionals on C,(X). We shall now describe a general situation in which the last-mentioned func tionals occur naturally Let $\textstyle K$ be a compact Hausdorff space, let ${\boldsymbol{H}}$ be a compact subset of $K_{\mathrm{{,}}}$ K, and let $\scriptstyle A$ be a subspace of $\scriptstyle C(R)$ such that $\mid\in A\setminus(1$ denotes the function which assigns the number 1 to each $x\in K)$ and such that $$ \|f\|_{K}=\|f\|_{H}\qquad(f\in A). $$ (1) Here we used the notation $$ \|f\|_{E}=\operatorname*{sup}\;\{\,|f(x)|\!:x\in E\}. $$ (2) Because of the example discussed in Sec. 5.23, ${\boldsymbol{H}}$ is sometimes called a bound- ary of $K,$ corresponding to the space ${\boldsymbol{A}}$EXAMPLES OF BANACH SPACE TECHNIQUEs 109 If f∈ A and $x\in K,(1)$ says that $$ |f(x)|\leq\|f\|_{H}. $$ (3) and In particular, i $\mathbb{F}f(y)=0$ for every $\textstyle v\in H$ , then $f(x)=0$ for al $x\in K$ Hence i f $f=1$ $f_{2}\in A$ and $f_{1}(y)=f_{2}(y)$ for every $\nu\in H,$ then $f_{1}=f_{2};$ to see this, put $f_{1}-f_{2}$ Let ${\cal M}$ be the set of all functions on ${\boldsymbol{H}}$ that are restrictions to ${\boldsymbol{H}}$ of members of A. It is clear that $\textstyle{M}$ has a unique extension to a member of $\mathbb{C}(m)$ The preceding remark shows that Thus we have a $\textstyle{M}$ is a subspace of each member of $A.$ natural one-to-one correspondence between $\textstyle{M}$ and $A.$ ,which is also norm preserving, by (1). Hence it will cause no confusion if we use the same letter to designate a member of $\scriptstyle A$ l and its restriction to $H.$ Fix a point $x\in\kappa.$ The inequality (3) shows that the mapping $f\to f(x)$ is a bounded linear functional on $\textstyle{M},$ of norm 1 [since equality holds in (3) i $f=1\rfloor$ such that By the Hahn-Banach theorem there is a linear functional $\Lambda$ on $\mathbb{C}(H_{3}$ of norm 1, $$ \Lambda f=f(x)\qquad(f\in M). $$ (4) We claim that the properties $$ \Lambda1=1,~~~~~||\Lambda||=1 $$ (5) $|g+i r|^{2}\leq1+r^{2}$ imply that A is a positive linear functional on $\mathbb{C}(m)$ put $g=2f-1,$ and put $\Lambda g=\alpha+$ To prove this, suppose $f\in C(H),\ \ 0\leq f\leq1,$ Note that $-1\leq g\leq1,$ so that iβ,where α and $\beta$ are real. Hence (5) implies that for every real constant ${\boldsymbol{r}}.$ $$ (\beta+r)^{2}\leq|\alpha+i(\beta+r)|^{2}=|\Lambda(g+i r)|^{2}\leq1+r^{2}. $$ (6) Thus $\beta^{2}+2r\beta\leq1$ for every real ${\boldsymbol{r}}_{}$ which forces $\beta=0.$ Since $|g|_{H}\leq1,$ we have $|\alpha|\leq1;$ hence $$ \Lambda f=\textstyle{\frac{1}{2}}\Lambda(1+g)=\textstyle{\frac{1}{2}}(1+\alpha)\geq0. $$ (T) Now Theorem 2.14 can be applied. It shows that there is a regular positive Borel measure $\mu_{x}\,$ on ${\boldsymbol{H}}$ such tha $$ \Lambda f=\int_{H}f\,d\mu_{x}\qquad(f\in C(H)). $$ (8) In particular, we get the representation formula $$ f(x)=\int_{H}^{}f\,d\mu_{x}\qquad(f\in{\cal A}). $$ (9) What we have proved is that to each $x\in K$ there corresponds a positive for everyfe $A.$ on the“boundary” ${\boldsymbol{H}}$ which“represents”x in the sense that (9) holds measure $\mu_{x}\,$110 REAL AND coMPLEX ANALYSIS Note that A determines $\mu_{x}\,$ uniquely; but there is no reason to expect the Hahn-Banach extension to be unique. Hence, in general, we cannot say much about the uniqueness of the representing measures. Under special circumstances we do get uniqueness, as we shall see presently. 5.23 To see an example of the preceding situation, let $U=\{z\colon|z|<1\}$ be the open unit disc in the complex plane, put $K={\vec{U}}$ (the closed unit disc), and take for ${\boldsymbol{H}}$ the boundary ${\mathbf{}}T$ of C ${\boldsymbol{U}}.$ We claim that every polynomial f, ie., every function of the form $$ f(z)=\sum_{n=0}^{N}a_{n}z^{n}, $$ (1) where $a_{0},\,\cdot\cdot,$ $a_{N}$ are complex numbers, satisfies the relation $$ \|f\|_{U}=\|f\|_{T}. $$ (2) as that over (Note that the continuity of f shows that the supremum of $|f|$ over $U$ is the same ${\bar{U}}.$ $z\in{\bar{U}}$ . Assume $z_{0}\in U.$ is compact, there exists a $z_{0}\in{\bar{U}}$ such that $|f(z_{0})|\geq|f(z)|$ for al Since L $\bar{U}$ Then $$ f(z)=\sum_{n=0}^{N}b_{n}(z-z_{0})^{n}, $$ (3) and if $0<r<1-|z_{0}|,$ we obtain $$ \sum_{n=0}^{N}\;|b_{n}|^{2}r^{2n}=\frac{1}{2\pi}\,\int_{-\pi}^{\pi}|f(z_{0}+r e^{i\theta})|^{2}\;d\theta\leq\frac{1}{2\pi}\,\int_{-\pi}^{\pi}|f(z_{0})|^{2}\;d\theta=|b_{0}|^{2}, $$ so that $b_{1}=b_{2}=\cdot\cdot\cdot=b_{N}=0;$ i.e., f is constant. Thus $\scriptstyle{\bar{z}}_{\delta}\in T$ for every noncon stant polynomial $f,$ and this proves (2) (We have just proved a special case of the maximum modulus theorem; we shall see later that this is an important property of all holomorphic functions.) 5.24 The Poisson Integral Let $\scriptstyle A$ be any subspace of $C({\vec{U}})$ (where $\bar{U}$ is the closed unit disc, as above) such that $\scriptstyle A$ contains all polynomials and such that $$ \|f\|_{U}=\|f\|_{T} $$ (1) holds for every f∈ A.We do not exclude the possibility that ${\cal A}_{\bf d}$ consists of precisely the polynomials, but $\scriptstyle A$ might be larger. The general result obtained in Sec. 5.22 applies to $\scriptstyle A$ l and shows that to each ZE $U$ there corresponds a positive Borel measure $\mu_{z}\,$ on ${\mathbf{}}T$ such that $$ f(z)=\bigcap_{T}f\,d\mu_{z}\qquad(f\in A). $$ (2) (This also holds for $z\in T,$ but is then trivial: $\mu_{z}\,$ is simply the unit mass concen- trated at the point z.)EXAMPLES OF BANACH SPACE TECHNIQUES 111 If We now fix $z\in U$ and write $z=r e^{i\theta},0\leq r<1,\theta$ real. $u_{n}(w)=w^{n},$ then $u_{n}\in A$ for n = 0, 1, 2,...;hence (2) shows that $$ r^{n}e^{i n\theta}=\left[\frac{u}{r}u_{n}\ d\mu_{z}\right.\qquad(n=0,\ 1,2,\ldots). $$ (3) Since $u_{-n}={\bar{u}}_{n}$ on $\scriptstyle T_{\circ}(0)$ leads to $$ \Biggl[u_{n}\,d\mu_{z}=r^{|n|}e^{i n\theta}\qquad(n=0,\,\pm1,\,\pm2,\,\ldots). $$ (4) This suggests that we look at the real function $$ P_{r}(\theta-t)=\sum_{n=-\infty}^{\infty}r^{|n|}e^{i n(\theta-t)}\qquad(t{\mathrm{~real}}), $$ (5) since $$ \frac{1}{2\pi} |_{-\pi}^{\pi}P_{r}(\theta-t)e^{i m t}\,d t=r^{|n|}e^{i n\theta}\qquad(n=0,\ \pm1,\ \pm2,\ ...).\qquad\qquad\qquad(n=0,\pm1,\ \pm1,\ldots). $$ (6) Note that the series (5) is dominated by the convergent geometric series $\bar{\sum}$ r"|, so that it is legitimate to insert the series into the integral (6) and to integrate term by term, which gives (6). Comparison of (4) and (6) gives $$ \left\{_{T}f\,d\mu_{x}={\frac{1}{2\pi}}\,\right\}_{-\pi}^{\pi}f(e^{i t})P_{r}(\theta-t)\ d t $$ (7) for $f=u_{n},$ hence for every trigonometric polynomial $f,$ and Theorem 4.25 now mined by (2). Why?] implies that (7) holds for every fe C(T). [This shows that $\mu_{z}\,$ was uniquely deter- In particular,(T) holds if f∈ A, and then (2) gives the representation $$ f(z)=\frac{1}{2\pi}\int_{-\pi}^{\pi}f(e^{i t})P_{r}(\theta-t)\;d t\qquad(f\in A). $$ (8) The series (5) can be summed explicitly, since it is the real part of $$ 1\,+\,2\,\sum_{1}^{\infty}\,(z e^{-i t})^{n}={\frac{e^{i t}+z}{e^{i t}-z}}={\frac{1-r^{2}+2i r\ \sin\left(\theta-t\right)}{|1-z e^{-i t}|^{2}}}. $$ Thus $$ P_{r}(\theta-t)={\frac{1-r^{2}}{1-2r\cos\left(\theta-t\right)+r^{2}}}. $$ (9) This is the so-called “Poisson kernel.”Note that $P_{r}(\theta-t)\geq0\,\mathrm{if}\,0\leq r<1.$ We now summarize what we have proved:112 REAL AND cOMPLEX ANALYSIs 5.25 Theorem Suppose $\scriptstyle A$ is a vector space of continuous complex functions on the closed unit disc ${\bar{U}}.\,I f\,A$ contains all polynomials, and i $$ \operatorname*{sup}_{z\in U}|f(z)|=\operatorname*{sup}_{z\in T}|f(z)| $$ (1) for everyfe A(where ${\mathbf{}}T$ is the unit circle, the boundary o $U),$ then the Poisson integral representation $$ f(z)={\frac{1}{2\pi}} |_{-\pi}^{\pi}{\frac{1-r^{2}}{1-2r\,\cos\,(\theta-t)+r^{2}}}\,f(e^{i t})\,d t\qquad(z=r e^{i\theta}) $$ (2) is valid for every f ∈ $\scriptstyle A$ and every z ∈ ${\boldsymbol{U}}.$ Exercises ${\boldsymbol{D}},$ balls of $D(\mu),$ for consist of two points $\underset{^{\sim}}{a}$ and $b,$ put $\mu(\{a\})=\mu(\{b\})={\frac{1}{2}},$ and let $D(\mu)$ be the resulting real is this l Let $\scriptstyle{\mathcal{X}}$ -space. Identify each real function f on Note that they are convex if and only if $1\leq p\leq\infty.$ in the plane, and sketch the unit ${\boldsymbol{p}}$ $\scriptstyle{\mathcal{X}}$ with the point $(f(a),f(b))$ $\scriptstyle0\,<\,p\,\leq\,\infty.$ For which unit balla square? A circle? If $\mu(\{a\})\neq\mu(b),$ how does the situation dier from the preceding one? 2 Prove that the unit ball (open or closed) is convex in every normed linear space 3 If $1<p<\alpha$ D, prove that the unit ball o $D(\mu)$ is strictly convex; this means that if $$ \|f\|_{p}=\|g\|_{p}=1,\;\;\;\;\;\;f\neq g,\;\;\;\;\;\;\;h=\frac{1}{2}(f+g), $$ then $|h|_{p}<1.$ (Geometrically, the surface of the ball contains no straight lines.) Show that this fails in every $L^{1}(\mu),$ in every $L^{\infty}(\mu),$ and in every C(X).(Ignore trivialities, such as spaces consisting of only one point) 4 Let $\scriptstyle{\bar{C}}$ be the space of all continuous functions on [0,1], with the supremum norm. Let ${\cal{M}}$ consist of all f C for which $$ \bigcap_{0}^{1/2}f(t)\;d t-\prod_{1/2}^{1}f(t)\;d t=1. $$ Prove that ${\cal{M}}$ is a closed convex subset of $\scriptstyle{\vec{C}}$ which contains no element of minimal norm. S Let ${\cal{M}}$ be the set of ${\mathrm{all}}f\in L^{1}([0,$ 1]), relative to Lebesgue measure, such that $$ \bigcap_{0}^{1}f(t)\;d t=1. $$ Show that ${\cal{M}}$ is a closed convex subset of $\mathbf{\Omega}^{4}$ with Theorem 4.10. which contains infinitely many elements of minimal norm. (Compare this and Exercise $L^{1}([0,1])$ 6 Let $\boldsymbol{\mathit{f}}$ be a bounded linear functional on a subspace ${\cal{M}}$ of a Hilbert space ${\boldsymbol{H}}.$ Prove that f has a ishes on $M^{\bot}$ unique norm-preserving extension to a bounded linear functional on ${\boldsymbol{H}},$ , and that this extension van- 7 Construct a bounded linear functional on some subspace of some $L^{1}(\mu).$ which has two (hence infinitely many) distinct norm-preserving linear extensions to $L^{1}(\mu)$ 8 Let $\scriptstyle{\mathcal{X}}$ be a normed linear space, and let $X^{\bullet}$ be its dual space, as defned in Sec. 5.21, with the norm $$ \|f\|=\operatorname*{sup}\;\{\,|f(x)|\!:\|x\|\leq1\}. $$ (a) Prove that $X^{\bullet}$ tis a Banach space is, for each $x\in X,$ a bounded linear functional on $X^{*}{\mathrm{:}}$ , of (b)Prove that the mapping $f\to f(x)$ in its "second dual” $\textstyle{\cal X}^{\ast\beta\ast}\!\!$ the dual space of $X^{**}{\mathrm{:}}$ norm lxl. (This gives a natural imbedding of $\scriptstyle{\mathcal{X}}$EXAMPLES OF BANACH SPACE TECHNiQUEs 113 every f∈ (c) Prove that $\{\|x_{n}\|\}$ is bounded if $\left\{X_{\mathfrak{H}}\right\}$ is a sequence in $\scriptstyle{\mathcal{X}}$ such that $\{f(x_{n})\}$ is bounded for $X^{\bullet}{\mathrm{:}}$ 9 Let $c_{\mathrm{o}}$ $\ell^{1},$ and $\ell^{\omega}$ be the Banach spaces consisting of all complex sequences $x=\{\xi_{i}\},$ $i=1,2,3,\dots$ defined as follows: $$ \begin{array}{l}{{x\in{\mathcal{E}}^{\dag}{\mathrm{~if~and~only~if~}}\|x\|_{1}=\sum|\xi_{i}|<\alpha.}}\\ {{x\in\ell^{\infty}{\mathrm{~if~and~only~if~}}\|x\|_{\alpha}=\operatorname{sup}|\xi_{i}|<\alpha}}\end{array} $$ 0. co is the subspace of $\scriptstyle{\mathcal{L}}^{\infty}$ consisting of all $x\in{\mathcal{F}}^{\infty}$ for which $\xi_{i}\to0\,\mathrm{as}\,i\to\infty.$ Co,and (a)i Prove the following four statements and $\mathrm{A}x=\sum\,\xi_{i}\,\eta_{i}$ for every $x\in c_{0},$ then $\Lambda$ is a bounded linear functional on $y=\{\eta_{i}\}\in\ell^{1}$ Moreover, every $\Lambda\in\{c_{0}\}^{*}$ is obtained in this way. In brief $(c_{0})^{*}=f^{1}.$ $\|\Lambda\|=\|y\|_{1}.$ (More precisely, these two spaces are not equal; the preceding statement exhibits an isometric vector space isomorphism between them.) e Every (b)In the same sense $(\ell^{1})^{*}=\ell^{\infty},$ as in (a). However, this does not $y\in\ell^{1}$ induces a bounded linear functional on $\ell^{\omega},$ give all o $(\ell^{\alpha})^{\ast},$ since $(\ell^{\prime\sigma})^{\star}$ contains nontrivial functionals that vanish on all of $c_{\mathrm{o}}$ $\Sigma\left|x_{i}\right|<\alpha.$ (d $c_{\mathrm{o}}$ and $\ell^{1}$ are separable but $\ell^{\omega}$ is not. asi→0o, prove that 1o urE $\alpha_{i}\xi_{i}$ converges for every sequence $(\zeta_{i})$ such that $\xi_{i}\to0$ 11 For $\scriptstyle0\,<\,\alpha\,\leq\,1$ , et Lip a denote the space ofall complex functions fon [a,b]for which $$ M_{f}=\operatorname*{sup}_{s\neq t}{\frac{|f(s)-f(t)|}{|s-t|^{\alpha}}}<\infty. $$ Prove that Lip ais a Banach space, if $$ \|f\|=|f(a)|+M_{f};a|s o,\operatorname{if} $$ $$ \|f\|=M_{f}+\operatorname*{sup}_{x}|f(x)|\,. $$ (The members of Lip α are said to satisfy a Lipschitz condition of order α.) $\mathbf{1}\mathbf{2}$ Let $\scriptstyle{\mathcal{K}}$ be a triangle (two-dimensional figure) in the plane, let ${\boldsymbol{H}}$ l be the set consisting of the vertices of ${\boldsymbol{K}},$ and let ${\mathbf{}}A$ be the set of all real functions f on K ${\boldsymbol{K}},$ ,of the form $$ f(x,\,y)=\alpha x+\beta y+\gamma\qquad(\alpha,\,\beta,\,\mathrm{and}\,\,\gamma\,\,\mathrm{ret} $$ ea) Show that to each $(x_{0},y_{0})\in K$ there corresponds a unique measure ${}^{\mu}$ A on ${\boldsymbol{H}}$ I such that $$ f(x_{0},y_{0})= .\varepsilon^{ \lfloor x \rfloor}t\,d\mu. $$ (Compare Sec. 5.22.) Replace ${\boldsymbol{K}}$ by a square, let ${\boldsymbol{H}}$ again be the set of its vertices, and let $\scriptstyle A\quad}$ be as above. Show that to each point of ${\cal K}$ there still corresponds a measure on $\textstyle H,$ . with the above property, but thatuniquenes is now lost Can you extrapolate to a more general theorem?(Think of other figures, higher dimensional spaces.) such t (b)I $\xi>0,$ be a sequence of continuous complex functions on a (nonempty) complete metric space $\scriptstyle x\neq X.$ ${\mathbf{}}N$ such that $|f(x)-f_{n}(x)|\leq\epsilon$ $X,$ 13 Let $\{f_{n}\}$ exists (as a complex number) for every and a number M <oo such that $|f_{n}(x)|<M$ for all $\operatorname{hat}f(x)=\operatorname*{lim}f_{n}(x)$ $V\neq\varnothing$ (a) Prove that there is an open set xe V and for $n=1,2,$ 3,… and an integsr , prove that there is an open set $\scriptstyle\nu\neq\sigma$ if xe V and $n\geq N$ Hint for (b): For $N=1,2,3,\ldots,$ put $$ A_{N}=\{x:|f_{m}(x)-f_{n}(x)|\leq\epsilon\ {\mathrm{if}}\;m\geq N\;\mathrm{and}\;n\geq N\}. $$ Since $X=\textstyle\bigcup A_{N},$ some $A_{N}$ has a nonempty interior114 REAL AND coMPLEX ANALYSIis for al be the subset of $\textstyle{\bar{\mathbf{C}}}$ be the space of all real continuous functions on $\scriptstyle t\;=\;[0,\;1]$ with the supremum norm. Let $X_{n}$ 14 Let C $\scriptstyle{\vec{C}}$ $s\in I.$ consisting of those f for which there exists $\textstyle{\bar{\mathbf{C}}}$ C contains an open set which does not intersect $|f(s)-f(t)|\leq n\,|s-t|$ $\scriptstyle{\mathrm{ster}}\,$ such that Fix n and prove that each open set in X, ${\mathrm{:(Each}}f\in C$ can be uniformly approximated by a zigzag function g with very large slopes, and if in $\scriptstyle{\vec{C}}$ which consists $\|g-h\|$ is smalL, $h\not\in X_{n}.$ Show that this implies the existence of a dense ${\cal G}_{\theta}$ entirely of nowhere differentiable functions. $15$ each sequence $\{s_{j}\}$ a sequence be an infinite matrix with complex entries, where $\mathbf{i},j=0,$ 1, $\scriptstyle2\ldots r^{\prime}$ 4 associates with Let $A=(a_{i j})$ $\{\sigma_{i}\},$ defined by $$ \sigma_{i}=\sum_{j=0}^{\infty}a_{i j}s_{j}\qquad(i=1,\,2,\,3,\,\ldots), $$ provided that these series converge. Prove that $\scriptstyle A\quad}$ transforms every convergent sequence $\{s_{j}\}$ to a sequence $\{\sigma_{i}\}$ which converges to the same limit if and only if the following conditions are satisfied: (a $$ \operatorname*{lim}_{i\to\infty}a_{i j}=0\qquad{\mathrm{for~each}}\,j. $$ (6) $$ \mathrm{sup}_{i}\sum_{j=0}^{\infty}|a_{i j}|<\infty. $$ (c) $$ \operatorname*{lim}_{i arrow\infty}\ \sum_{j=0}^{\infty}a_{i j}=1. $$ The process of passing from $\{s_{j}\}$ to $\{\sigma_{i}\}$ is called a summability method. Two examples are: $$ a_{i j}= \{\frac{1}{i+1}\quad\quad\mathrm{if}\ 0\leq j\leq i, $$ and $$ a_{i j}=(1-r_{i})r_{i}^{j},\qquad0<r_{i}<1,\qquad r_{i} arrow1. $$ Prove that each of these also transforms some divergent sequences $\{s_{j}\}$ (even some unbounded ones) to convergent sequences $\{\sigma_{i}\}.$ 16 Suppose $\scriptstyle{\mathcal{X}}$ This is the so-called $\stackrel{16}{\longrightarrow}$ are Banach spaces, and suppose $\left\{X_{m}\right\}$ in $\scriptstyle{\mathcal{X}}$ for which $x=\operatorname*{lim}\;x_{n}$ and $y=\operatorname*{lim}\,\Lambda x_{n}$ into ${\boldsymbol{Y}},$ with the and ${\mathbf{}}Y$ $\Lambda$ is a linear mapping of $\scriptstyle{\mathcal{X}}$ exist, it is following property: For every sequence true that $y=\Lambda x.$ Prove that A is continuous be the set of all ordered pairs *closed graph theorem.” Hint: Let $X\oplus Y$ $(x,y),\ x\in X$ and $y\in Y,$ with addition and scalar multiplication defined componentwise. Prove that the pairs is a Banach space, if $\|(x,y)\|=\|x\|+\|y\|.$ The graph $\boldsymbol{\mathit{G}}$ of $\Lambda$ is the subset of $\boldsymbol{\mathit{G}}$ is a Banach space $X\oplus Y$ $X\oplus Y$ formed by $(x,\,\land x),$ $x\in X.$ Note that our hypothesis says that $\boldsymbol{\mathit{G}}$ is closed; hence Note that $(x_{i}$ $\Lambda x)\to x$ is continuous, one-to-one, and linear and maps $R^{1}$ onto $R^{1}{\mathrm{.}}$ ,for instance) whose graph is closed Observe that there exist nonlinear mappings (of $\boldsymbol{\mathit{G}}$ onto $X.$ although they are not continuous: $f(x)=1/x{\mathrm{~if~}}x\neq0,f(0)=0.$ 17 If $\mathcal{J}$ is a positive measure, each Prove that $\|M_{f}\|\leq\|f\|_{\infty}.$ For which measures $L^{2}(\mu)\dot{\gamma}$ ${}_{\mu}$ is it true that $L^{2}(\mu)$ into $L^{2}(\mu),$ for allf $\in L^{\infty}(\mu)Y$ For which f $\in L^{\infty}(\mu)$ does $\textstyle{M_{f}}$ map $L^{2}(\mu)$ onto defines a multiplication operator $\textstyle{M_{f}}$ on $\|M_{J}\|=\|f\|_{J}$ $\operatorname{t}f\in L^{\infty}(\mu)$ such that $M_{f}(g)=f g.$ 18 Suppose {A,} is a sequence of bounded linear transformations from a normed linear space $\scriptstyle{\mathcal{X}}$ to a $\{\Lambda_{n}x\}$ Banach space ${\boldsymbol{Y}},$ suppose $\|\Lambda_{n}\|\leq M<\infty$ for all ${\mathfrak{n}},$ and suppose there is a dense set $x\in X.$ $E\subset X$ such that converges for each $x\in E.$ Prove that $\{\Lambda_{n}x\}$ converges for eachEXAMPLES OF BANACH SPACE TECHNIQUES 115 19 If uniformly, as $n arrow\infty,$ 。 is the nth partial sum of the Fourier series of a function fe C(T), prove that $s_{a}/\log\;n\to0$ $s_{n}$ for each fe C(T). That is, prove that $$ \operatorname*{lim}_{n arrow\infty}{\frac{\left\|s_{n}\right\|_{\alpha}}{\log n}}=0. $$ On the other hand, if $\lambda_{n}/\log\;\;n arrow0,$ prove that there exists an $f\in C(T)$ such that the sequence $\{s_{n}(f;0)/\lambda_{n}\}$ is unbounded. Hint: Apply the reasoning of Exercise $18$ and that of Sec.5.11, with a better estimate of $|D_{n}|_{1}$ than was used there 20 (a) Does there exist a sequence of continuous positive functions ${\mathfrak{f}}_{n}$ 。on $R^{1}$ such that {JA(x)}is unbounded if and only if xis rational? logues of (a and $(b).$ (b) Replace"“ rational" by “irrational"in (a) and answer the resulting question ${\mathfrak{r}}_{n}(x)\to\infty$ as $n{ arrow}\infty^{+}$ and answer the resulting ana- (c) Replace ${}^{\omega}\{f_{n}(x)\}$ is unbounded” by $21$ Suppose $\scriptstyle{k\in{R^{\prime}}}$ is measurable, and $m(E)=0.$ Must there be a translate $E+x$ of $\boldsymbol{E}$ that does not intersect $E^{\gamma}$ Must there be a homeomorphism ${\boldsymbol{h}}$ i of $R^{1}$ onto $R^{1}$ so that $h(E)$ does not intersect $E^{\gamma}$ converges to $f(x),$ 22 Suppose fe C(T) and fe Lip a for some $\alpha>0.$ $s_{n}(f;0)$ (See Exrcise 11) Prove that the Fourier sere $\mathrm{of}f$ $f(0)=0.$ by completing the following outline: It is enough to consider the case $x=0,$ The difference between the partial sums and the integrals $$ {\frac{1}{\pi}}\left\vert_{-\pi}^{\pi}f(t)\,{\frac{\sin\,\,n t}{t}}\,d t\right. $$ tends to O as $n arrow\infty.$ The function $f(t)/i$ is in $L^{1}(T).$ Apply the Riemann-Lebesgue lemma. More careful reasoning shows that the convergence is actually uniform on ${\boldsymbol{T}}.$