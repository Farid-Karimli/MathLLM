CHAPTER NINE FOURIER TRANSFORMS Formal Properties 9.1 Definitions In this chapter we shall depart from the previous notation and use the letter m not for Lebesgue measure on $R^{1}$ but for Lebesgue measure divided by $\sqrt{2}\pi.$ This convention simplifies the appearance of results such as the inversion theorem and the Plancherel theorem. Accord- ingly, we shall use the notation $$ \left.\stackrel{\circ}{\to}_{-\infty}f(x)\,d m(x)=\frac{1}{\sqrt{2\pi}}\right.\stackrel{\circ}{\to}_{-\infty}^{\prime\infty}f(x)\,d x, $$ (1) where ${\boldsymbol{d}}x$ refers to ordinary Lebesgue measure, and we define and $$ \begin{array}{c}{{||f||_{p}=\displaystyle\left\{\displaystyle\ <\sim\left\{\right\}\sum_{-\infty}^{\infty}|f(x)|^{p}\,d m(x)\right\}^{1/p}~~~~~(1\leq p<\infty),}}\\ {{ .(f\ast g)(x)=\displaystyle\int_{-\infty}^{\infty}f(x-y)g(y)\,d m(y)~~~~~(x\in R^{1}),}}\end{array} $$ (2) (3) $$ \hat{f}(t)= [\O_{-\infty}^{\infty}f(x)e^{-i x t}\,d m(x)\qquad(t\in R^{1}). $$ (4) Throughout this chapter, we shall write ${\boldsymbol{D}}^{p}$ in place of $L^{p}(R^{1}),$ and $C_{0}$ will denote the space of all continuous functions on $R^{1}$ which vanish at infinity. $1f f\in L$ , the integral (4) is well defined for every real ${\bar{t}}.$ The function f is called the ${\mathbf{}}F$ ourier transform of f. Note that the term“Fourier transform ${\mathfrak{s l}}$ is also applied to the mapping which takes f to f 178FOURIER TRANSFORMs $179$ The formal properties which are listed in Theorem 9.2 depend intimately on the translation-invariance of ${\mathfrak{m}}$ and on the fact that for each real α the mapping character of $R^{1}$ if」 is a character of the additive group and if $R^{1}.$ By definition, a function $\boldsymbol{\varphi}$ is a $x\to e^{i\alpha x}$ $\varphi(t)|=1$ $$ \varphi(s+t)=\varphi(s)\varphi(t) $$ (5) for all real s and ${\mathfrak{t}}.$ In other words, $\varphi$ is to be a homomorphism of the additive group $R^{1}$ into the multiplicative group of the complex numbers of absolute value 1. We shall see later (in the proof of Theorem 9.23) that every continuous charac ter of $R^{1}$ is given by an exponential. 9.2 Theorem Suppose $f\in L^{1}.$ ,and α and $\lambda$ are real numbers. (c)Ifg $\scriptstyle{\epsilon\;\epsilon\;{\bar{L}}}$ and (a)If (x) =/(xpel", then G(t) = f(t-a) thenG( =joe-a (b)1f $g(x)=f(x-\alpha),$ $h=f*g,t h e n\ {\hat{h}}(t)={\hat{f}}(t){\hat{g}}(t).$ Thus the Fourier transform converts multiplication by a character into translation, and vice versa, and it converts convolutions to pointwise products. (d)If g(x) = f(一x), then ${\hat{\sigma}}(t)={\overline{{j(t)}}}.$ ${\hat{g}}(t)_{_{\sim}}\ d{\hat{f}}(\lambda t).$ ${\hat{\mathcal{I}}}^{\prime}(t)={\hat{\mathcal{I}}}(t).$ (e)If g(x) = f(x/2) and ${\boldsymbol{\lambda}}>0,$ then (f)If g(x)= -ixf(x) and $g\in L,$ then fis diferentiable and PRoOF (a),(b),(d),and (e) are proved by direct substitution into formula 9.1(4).The proof of $\mathbf{\Psi}_{(c)}$ is an application of Fubini's theorem (see Theorem 8.14 for the required measurability proof) $$ \begin{array}{l}{{\frac{1}{\hbar}|x|^{3}|x|}}\\ {{\frac{1}{\hbar}}}\\ {{\frac{1}{\hbar}}}\\ {{\frac{1}{\hbar}}}\\ {{\frac{1}{\hbar}}}\\ {{\frac{1}{\hbar}}}\\ {{\frac{1}{\hbar}}}\\ {{\frac{1}{\hbar}}}\\ {{\frac{1}{\hbar}}}\\ {{\frac{1}{\hbar}}}\end{array} $$ Note how the translation-invariance of m was used. To prove $(f),$ note that $$ {\frac{{\hat{f}}(s)-{\hat{f}}(t)}{s-t}}= |\sum_{-\infty}^{\infty}f(x)e^{-i x t}\varphi(x,\,s-t)\;d m(x)\qquad(s\neq t),\qquad\qquad(1) $$180 REAL AND coMPLEX ANALYSIs where qp(x, $u)=(e^{-i x u}-1)/u.$ Since $|\,\varphi(x,\,u)|\leq|\,x\,|$ for all real $u\neq0$ and since p(x,u)→-ix as $u\to0$ ,the dominated convergence theorem applies to (1), if s tends to t, and we conclude that $$ \hat{f}^{\prime}(t)=\left.-i\,\right|_{-\infty}^{\ast\alpha}x f(x)e^{-i x t}\,d m(x). $$ (2) // 9.3 Remarks (a) In the preceding proof, the appeal to the dominated convergence theorem may seem to be illegitimate since the dominated convergence theorem deals only with countable sequences of functions. However, it does enable us to conclude that $$ \operatorname*{lim}_{n\to\infty}{\frac{{\hat{f}}(s_{n})-{\hat{f}}(t)}{s_{n}-t}}=-i\left.\right.{\stackrel{\cdots}{\int}}x f(x)e^{-i x t}\,d m(t) $$ for every sequence $\{s_{n}\}$ which converges to $t_{\mathrm{,}}$ and this says exactly that $$ \operatorname*{lim}_{s\to t}{\frac{{\hat{f}}(s)-{\hat{f}}(t)}{s-t}}=-i\int_{-\infty}^{\infty}x f(x)e^{-i x t}\,d m(t). $$ We shall encounter similar situations again, and shall apply con vergence theorems to them without further comment (b) Theorem $9.2(b)$ shows that the Fourier transform of $$ \complement f(x+\alpha)-f(x)\rfloor/\alpha $$ is $$ {\hat{f}}(t){\frac{e^{i t t}-1}{\alpha}}. $$ This suggests that an analogue of Theorem 9.2(f) should be true under certain conditions, namely, that the Fourier transform of $f^{\prime}$ is itf (t).If $f\in L^{1},f^{\prime}\in L^{1},$ and if $\boldsymbol{\f}$ is the indefinite integral of $f^{\prime},$ the result is easily established by an integration by parts. We leave this, and some related results,as exercises. The fact that the Fourier transform converts differ entiation to multiplication by $t i$ makes the Fourier transform a useful tool in the study of differential equations. The Inversion Theorem 9.4 We have just seen that certain operations on functions correspond nicely to operations on their Fourier transforms. The usefulness and interest of this corre spondence will of course be enhanced if there is a way of returning from the transforms to the functions, that is to say, if there is an inversion formula.FoURIER TRANSFORMs 181 Let us see what such a formula might look like, by analogy with Fourier series. If $$ c_{n}={\frac{1}{2\pi}}\int_{-\pi}^{\pi}f(x)e^{-i n x}\;d x\qquad(n\in Z), $$ (1) then the inversion formula is $$ f(x)=\sum_{-\infty}^{\infty}c_{n}e^{i n x}. $$ (2) We know that (2) holds, in the sense of $L^{2}.$ -convergence, if f ∈ $\scriptstyle{E(T)}.$ We also know that (2) does not necessarily hold in the sense of pointwise convergence, even if f is continuous. Suppose now that f∈ $\scriptstyle{I(T)_{\mathrm{s}}}$ that $\{c_{n}\}$ is given by (1), and that $$ \sum_{0\atop m}^{\infty}|c_{n}|<\infty. $$ (3) Put $$ g(x)=\sum_{-\infty}^{\infty}c_{n}e^{i n x}. $$ (4) By (3),the series in(4) converges uniformly (hence $\scriptstyle{\mathcal{G}}$ is continuous),and the Fourier coefficients of $\scriptstyle{\mathcal{G}}$ are easily computed: $$ {\frac{1}{2\pi}}\left|\stackrel{\pi}{\ldots}^{\circ}g(x)e^{-i k x}\;d x={\frac{1}{2\pi}}\right.\int_{-\pi}^{\pi}\left\{\sum_{n=-\infty}^{\infty}c_{n}e^{i n x}\right\}e^{-i k x}\;d x $$ $$ =\sum_{n=-\infty}^{\infty}c_{n}\,{\frac{1}{2\pi}}\,\Bigg|_{-\pi}^{*\pi}\,e^{i(n-k)x}\,d x $$ = Ck· (5) Thus f and $\scriptstyle{\mathcal{G}}$ have the same Fourier coefficients. This implies f = g a.e., so the Fourier series of f converges to $\scriptstyle{\mathcal{I}}(x)$ a.e The analogous assumptions in the context of Fourier transforms are that $\scriptstyle f\in E$ and $\beta_{\mathrm{e}}\,L_{i}$ and we might then expect that a formula like $$ f(x)= .{\overline{{\bigcup_{-\infty}^{\infty}}}}{\hat{f}}(t)e^{i x}\,\,d m(t)\, $$ (6) is valid. Certainly, $\operatorname{lf}f\in L^{1},$ the right side of (6) is well defined; call it g(x); but if we want to argue as in(5), we run into the integral $$ \widetilde{\Xi}_{-\infty}^{\ast\alpha}\,e^{\widetilde{\imath}\left(t-s\right)x}\, (\mathcal{A}x, $$ (7) which is meaningless as it stands. Thus even under the strong assumption that ${\hat{J}}\in L^{1},$ a proof of (6) (which is true) has to proceed over a more devious route182 REAL AND cOMPLEX ANALYSIS [It should be mentioned that (6) may hold even if ${\hat{J}}\notin L^{1},$ if the integral over (-00, co)is interpreted as the limit,as $A\to\infty,$ of integrals over(-A, A) (Analogue: a series may converge without converging absolutely.) We shall not go into this.] off defined by 9.5 Theorem For any function f on $R^{1}$ and every $\scriptstyle\nu\in R^{\dagger}$ ,let $f_{y}$ be the translate $$ f_{y}(x)=f(x-y)\qquad(x\in R^{1}). $$ (1) If1 ≤p < and iff∈ L, the mapping $$ \scriptstyle{\nu\to\int_{\gamma}} $$ (2) is a uniformly continuous mapping of $R^{1}$ l into $L^{p}(R^{1}).$ PROOF Fixc> 0. Since $f\in L^{p},$ there exists a continuous function $\scriptstyle{\mathcal{G}}$ whose support lies in a bounded interval [-A,A], such that $$ \|f-g\|_{p}<\epsilon $$ (Theorem 3.14).The uniform continuity of g shows that there exists a $\delta\in(0,\,A)$ such that $|s-t|<\delta$ implies $$ |g(s)-g(t)|<(3A)^{-1/p}\epsilon. $$ If $|s-t|<\delta,$ it follows that $$ |\stackrel{\circ}{\to}_{\circ}|g(x-s)-g(x-t)|^{p}\;d x<(3A)^{-1}\epsilon^{p}(2A+\delta)<\epsilon^{p}, $$ so that $\|g_{s}-g_{t}\|_{p}<\epsilon.$ ${\boldsymbol{D}}^{\prime}$ -norms (relative to Lebesgue measure)are translation- Note that invariant: $\|f\|_{p}=\|f_{s}\|_{p}.$ Thus ll/. ,ll ≤ |f,- g.ll + lg, - 9l + I9, - ,I = I( - g)ll, + lg.- ,|,+ (g -J,l,、<32 whenever $|s-t|<\delta.$ This completes the proof // 9.6 Theorem $U f\in L^{1},\,t h e n\dot{f}\in C_{0}$ and $$ \|{\hat{f}}\|_{\infty}\leq\|f\|_{1}. $$ (1) PRoOF The inequality (1) is obvious from 9.1(4) $\mathbf{\partial}\cdot\mathbf{I}f~t_{n}{ arrow}t,$ then $$ .\qquad|\,{\hat{f}}(t_{n})-{\hat{f}}(t)\,|\leq\sum_{n\,\omega}^{\infty}|f(x)|\,|\,e^{-i t_{n}x}-e^{-i t x}|\,d m(x). $$ (2)FoURIER TRANSFORMs 183 Hence ${\hat{f}}(t_{n})\to{\hat{f}}(t),$ The integrand is bounded by 21J(x)l and tends to O for every x,as $\dot{\boldsymbol{f}}$ f is contin- $n\to\varnothing.$ by the dominated convergence theorem. Thus uous. Since $e^{\pi i}=\,-\,1,9.1(4)\,;$ gives $$ \hat{f}(t)=\ -\,\int_{-\infty}^{\infty}f(x)e^{-i t(x+\pi t)}\;d m(x)=\ -\,\int_{-\infty}^{\infty}f(x-\pi/t)e^{-i x}\;d m(x). $$ (3) Hence $$ 2\hat{f}(t)=\left\{\O_{-\infty}^{\infty} \{f(x)-f\Bigl(x-\frac{\pi}{t}\Bigr)\right\}e^{-i t x}\,\,d m(x), $$ (4) so that $$ 2\left|\,\hat{f}(t)\right|\le\left| |f-f_{\pi/} |\right|_{1}, $$ (5) which tends to O ast→±00, by Theorem 9.5 // 9.7 A Pair of Auxiliary Functions In the proof of the inversion theorem it will be convenient to know a positive function ${\boldsymbol{H}}$ which has a positive Fourier transform whose integral is easily calculated. Among the many possibilities we choose one which is of interest in connection with harmonic functions in a half plane. (See Exercise 25, Chap. 11.) Put $$ H(t)=e^{-\mid t\mid} $$ (1) and define $$ h_{\lambda}(x)= \{\O_{-\infty}^{\infty}H(\lambda t)e^{i t x}\;d m(t)\qquad(\lambda>0). $$ (2) A simple computation gives $$ h_{\lambda}(x)=\sqrt{\frac{2}{\pi}}\,\frac{\lambda}{\lambda^{2}+x^{2}} $$ (3) and hence $$ \vert\sum_{-\infty}^{\infty}h_{\lambda}(x)\ d m(x)=1. $$ (4) Note also that $0<H(t)\leq1$ and that $H(\lambda t) arrow1$ as .→0. 9.8 Proposition $I f\in L^{2},$ then $$ (f*h_{\lambda})(x)= [\O_{-\infty}^{\infty}H(\lambda t)\hat{f}(t)e^{i x t}\,d m(t). $$184 REAL AND cOMPLEX ANALYSIS PRooF This is a simple application of Fubini's theorem $$ \begin{array}{r l}{(f*h_{L})(x)={\overline{{\bigcup_{-\infty}^{\infty}f(x-y)\,d m(y)}}\int_{-\infty}^{\infty}H(\lambda t)e^{i y}\,d m(t)}\\ {={\overline{{\bigcup_{--\infty}^{\infty}}}}H(\lambda t)\,d m(t)\,{\overline{{\bigcup_{-\infty}^{\infty}}}}f(x-y)e^{i y}\,d m(y)}\\ {={\overline{{\bigcup_{--\infty}^{\infty}}}}H(\lambda t)\,d m(t){\overline{{\int_{-\infty}^{\infty}}}}f(y)e^{i x\,d m(t)}}\end{array} $$ // 9.9 Theorem $\;U g\in L^{n}$ and $\scriptstyle{\mathcal{G}}$ is continuous at a point ${\boldsymbol{x}},$ then $$ \operatorname*{lim}_{s\to0}\;(g\,*\,h_{s})(x)=g(x). $$ (1) PRoOF On account of $9.7(4),$ we have $$ \begin{array}{r l}{(g*h_{\lambda})(x)-g(x)={\binom{\sum_{-\infty}^{\infty}{\binom{g(x-y)}{\binom{\theta(x-y)}{\binom{\theta}{\lambda}}}}}}}\\ {={\int_{-\infty}^{\infty}}[g(x-y)-g(x)]{\lambda^{-1}}h_{1}{\binom{y}{\frac{\sqrt}{\lambda}}}\,d m(y)}\end{array} $$ $$ = \langle\sum_{-\infty}^{\infty}\left[g(x-\lambda s)-g(x)\right]h_{1}(s)\;d m(s). $$ The last integrand is dominated by $2|g||_{\infty}h_{1}(s)$ and converges to O pointwise for every ${\boldsymbol{S}}_{\mathsf{J}}$ as $\lambda{ arrow}\,0$ Hence(1))follows from the dominated convergence theorem. // 9.10 Theorem $l f1\leq p<\infty\;a n d f\in L^{p},t h e n$ $$ \operatorname*{lim}_{\lambda\to0}\;\|f*h_{\lambda}-f\|_{p}=0. $$ (1) The cases $p=1$ and $p=2$ will be the ones of interest to us, but the general case is no harder to prove. PROOF Since defined for every x.(In fact, f * $h_{\lambda}$ is the exponent conjugate to $p,\;(f*\;h_{3})(x)$ is $h_{\lambda}\in L^{\prime}.$ where $\scriptstyle{\mathcal{A}}$ is continuous; see Exercise 8.) Because of 9.7(4) we have $$ (f*h_{\lambda})(x)-f(x)=\left.\right|_{-\infty}^{\infty}[f(x-y)-f(x)]h_{\lambda}(y)\;d m(y) $$ (2)FOURIER TRANSFORMs 185 and Theorem 3.3 gives $$ |(f*~h_{\lambda})(x)-f(x)|^{p}\leq\prod_{-\infty}^{*\infty}|f(x-y)-f(x)|^{p}h_{\lambda}(y)\;d m(y). $$ (3) Integrate (3) with respect to $\scriptstyle{\mathcal{X}}$ and apply Fubini's theorem: $$ \|f*\,h_{\lambda}-f\|_{p}^{p}\leq\int_{-\infty}^{\infty}\|f_{y}-f\|_{p}^{p}h_{\lambda}(y)\ d m(y). $$ (4) If $g(y)=\|f_{y}-f\,\|_{p}^{p}.$ ,then $\scriptstyle{\mathcal{G}}$ is bounded and continuous, by Theorem 9.5, and ),by Theorem 9.9. // $g(0)=0.$ Hence the right side of (4) tends to $\mathbf{0}$ as $\lambda\mathbf{\hat{e}} arrow0$ 9.11 The Inversion Theorem I $f f\in L^{1}$ and fe L, and j $$ g(x)= \{\O_{-\infty}^{\infty}{\hat{f}}(t)e^{i x t}\,d m(t)\qquad(x\in R^{1}), $$ (1) then $g\in C_{0}$ and f(x) = g(x) a.e PRoOF By Proposition 9.8, $$ (f*h_{\lambda})(x)=\left.\right|_{-\infty}^{\infty}H(\lambda t)\hat{f}(t)e^{i x t}\,d m(t). $$ (2) $H(\lambda t) arrow1$ as ${\bar{\lambda}}\to0,$ The integrands on the right side of (2) are bounded by $|\,{\hat{f}}(t)\,|\,,$ and since by the right side of (2) converges to $g(x),$ for every $x\in R^{\dagger}$ the dominated convergence theorem If we combine Theorems 9.10 and 3.12,we see that there is a sequence $\{\lambda_{n}\}$ such that $\lambda_{n} arrow0$ and n→α lim (f * h,)Xx) =f(x) a.e (3) Hence f(x) = 9(x) a.e. That $g\in C_{k}$ follows from Theorem 9.6. // 9.12 The Uniqueness S Theorem $\;U\;f\in L^{1}$ and $f(t)=0~~f o r~~a l l~~t\in R^{1},$ then $f(x)=0$ α.e. PRoor Since ${\hat{f}}=0$ we ${\mathrm{have}}\,f\in L^{1}$ and the result follows from the inversion // theorem. The Plancherel Theorem Since the Lebesgue measure of $R^{1}$ is infinite, ${\boldsymbol{L}}^{2}$ is not a subset of ${\boldsymbol{L}}^{1}$ L!, and the ${\boldsymbol{L}}^{2}$ definition of the Fourier transform by formula $9.1(4)$ is therefore not directly and extends to an isometry of ${\boldsymbol{L}}^{2}$ The definition does apply, however, ${\bf i}f\rho\in L^{1}\cap L^{2},$ into applicable to every $f\in_{2}L^{2}.$ In fact, $\|{\hat{f}}\|_{2}=\|_{2}f\|_{2}!$ This isometry of $L^{1}\cap L$ it turns out that then ${\hat{f}}\in L^{2}.$ onto L, and this extension defines the Fourier186 REAL AND coMPLEX ANALYSIS transform(sometimes called the Plancherel transform)of every $f\in L^{2}.$ The resulting $L^{2}.$ -theory has in fact a great deal more symmetry than is the case in $L^{1}.$ In $L^{2},f$ and f play exactly the same role 9.13 Theorem One can associate to each $f\in L^{2}$ a function ${\hat{f}}\in L^{2}$ so that the following properties hold: (a)Iff e $L^{1}\cap L^{2},$ ,then fis the previously defined Fourier transform off $L^{2}$ Onto $L^{2},$ (b)For every f e $L^{2},\left\|_{\bar{z}}\right\|_{2}=\left\|_{-}f\right\|_{2}$ (c The mapping f→fis a Hilbert spce isomorphism of (d)The following symmetric relation exists between f and f: ${\cal I}{\cal f}$ $$ \varphi_{A}(t)=\left|_{-A}^{*A}f(x)e^{-i x t}\,d m(x)\quad a n d\quad\psi_{A}(x)=\right|_{-A}^{x A}\hat{f}(t)e^{i x t}\,d m(t), $$ then llp。一厂ll2→0 and lV。一/ll1→0 as A→ $f{\boldsymbol{\to}}{\hat{f}}$ Note: Since $I^{1}\cap E^{2}$ is dense in $L^{2},$ properties $\mathbf{\tau}_{(a)}$ and(b) determine the mapping "uniquely. Property (d) may be called the ${\boldsymbol{L}}^{2}$ inversion theorem. PR0OF Our first objective is the relation $$ \|{\hat{f}}\|_{2}=\|f\|_{2}\qquad(f\in L^{1}\cap L^{2}). $$ (1) We $\mathrm{fr}\,f\in L^{1}\,\cap\,L^{2},\,\mathrm{put}\,{\tilde{f}}(x)={\overline{{f(-x)}}},$ and define $g=f*{{\hat{f}}}.$ Then $$ g(x)=\left.\right|_{-\infty}^{\infty}f(x-y)\overline{{{f(-y)}}}\ d m(y)=\left.\stackrel{ (\infty\right)}{\sim}f(x+y)\overline{{{f(y)}}}\ d m(y), $$ (2) or $$ g(x)=(f_{-x},f), $$ (3) where the inner product is taken in the Hilbert space ${\boldsymbol{L}}^{2}$ and $f_{-x}$ denotes a mapping of $R^{1}$ into as in Theorem 9.5. By that theorem, $x\to f_{-x}$ is a continuous translate of $f,$ $L^{2},$ and the continuity of the inner product (Theorem 4.6) shows that therefore implies that $\scriptstyle{\mathcal{G}}$ is a continuous function. The Schwarz inequality $$ |g(x)|\leq\|f_{-x}\|_{2}\|f\|_{2}=\|f\|_{2}^{2} $$ (4) Since ${\mathrm{g}}\in L^{n}$ is bounded. Also, $\scriptstyle g\,\in\,L$ since fe ${\boldsymbol{L}}^{1}$ l and fe L so that $\scriptstyle{\mathcal{G}}$ , we may apply Proposition 9.8: $$ (g\ast h_{\lambda})(0)= .\stackrel{\sim}{\sim}\!H(\lambda t)\dot{g}(t)\,d m(t). $$ (5) Since g $\mathbf{\Omega}^{g}$ g is continuous and bounded, Theorem $9.9$ shows that $$ \operatorname*{lim}_{\lambda\to0}\;(g\ast h_{\lambda})(0)=g(0)=\left\|f\right\|_{2}^{2}. $$ (6)FOURIER TRANSFORMs 187 ${\boldsymbol{\lambda}} arrow0,$ Theorem 9.2(d shows that ${\hat{g}}=|{\hat{f}}|^{2}\geq0,$ and since $H(\lambda t)$ increases to l as the monotone convergence theorem gives $$ \operatorname*{lim}_{\lambda\to0}\,\left\{\bigcup_{-\infty}^{\infty}H(\lambda t)\dot{g}(t)\;d m(t)=\right\}_{-\infty}^{\infty}\vert\hat{f}(t)\vert^{2}\;d m(t). $$ (7) Now (5),(6), and (T) shows $\operatorname{hat}f\in L^{2}$ and that () holds. This was the crux of the proof. (1), $\scriptstyle V\in E$ We claim that ${\mathbf{}}Y$ be the space of all Fourier transforms $\hat{\boldsymbol{f}}$ of functions f ∈ $L^{1}\cap L^{2}.$ By Let ${\mathbf{}}Y$ The functions $x\to e^{i\alpha x}H(\lambda x)$ is dense in $L^{2},$ i.e., that $Y^{1}=\{0\}.$ $\scriptstyle\lambda\,>0$ are in $L^{1}\cap L^{2},$ for all real α and al Their Fourier transforms $$ h_{\lambda}(\alpha-t)=\bigcap_{-\infty}^{\infty}e^{i\alpha x}H(\lambda x)e^{-i x t}\,d m(x) $$ (8) are therefore in Y. If $w\in L^{2},w\in Y^{\perp},$ it follows that $$ (h_{\lambda}\ast\bar{w})(x)=\left\{\O_{-\infty}^{\infty}h_{\lambda}(\alpha-\imath)\bar{w}(t)\ d m(t)=0\right.\qquad\qquad\qquad\qquad\qquad\qquad\qquad\qquad $$ (9) for all α. Hence $w=0,$ by Theorem 9.10, and therefore ${\mathbf{}}Y$ is dense in $L^{2}.$ Let us introduce the temporary notation $L^{2}.$ L子 -isometry from one dense subspace of $L^{2},$ proved so far, we see that $\Phi$ $\Phi f\operatorname{for}f.$ From what has been is an namely $L^{1}\cap L^{2}.$ ,onto another, namely ${\boldsymbol{Y}}.$ Elementary Cauchy sequence argu- ments (compare with Lemma 4.16) imply therefore that D extends to an isometry $\widetilde\Phi$ of ${\boldsymbol{L}}^{2}$ onto I $L^{2}.$ If we write ffor bf, we obtain properties $\mathbf{\Psi}_{(a)}$ ) and (b) Property (c) follows from (b), as in the proof of Theorem 4.18. The $P a r.$ seval formula $$ .\int_{-\infty}^{\infty}f(x){\overline{{g(x)}}}\ d m(x)= \{_{-\infty}^{\infty}{\hat{f}}(t){\widetilde{g(t)}}\ d m(t) $$ (10) holds therefore for all f ∈ ${\boldsymbol{L}}^{2}$ and g ${\mathfrak{c}},{\mathfrak{t}}$ To prove (d),let $k_{A}$ be the characteristic function of[- A,A4] Then $k_{A}f\in L^{1}\cap L^{2}\ {\mathrm{iff}}f\in L^{2},$ and $$ \varphi_{A}=(k_{A}f)^{<}. $$ (11) Since $\|f-k_{A}f\,\|_{2}\to0$ as $A\to\mathbb{Q},\mathbb{H}$ follows from (b) that $$ \|{\hat{f}}-\varphi_{A}\|_{2}=\|(f-k_{A}f)\cap_{\|_{2}}\to0 $$ (12) as $A\to\varnothing.$ The other half of (d) is proved the same way // 9.14 Theorem $l f f\in L^{2}\;a n d\,{\hat{f}}\in L^{1},t h e n$ $$ f(x)=\sum_{-\infty}^{\infty}{\hat{f}}(t)e^{i x t}\,d m(t)\qquad a.e. $$188 REAL AND cOMPLEX ANALYSIs PR0O0F This is corollary of Theorem 9.13(4) / f ∈ $L^{2}.$ 9.15 Remark Iff∈ $L^{1}.$ the Plancherel theorem defines $\hat{f}$ formula 9.1(4) defines f(t) unambiguously for every t. If uniquely as an element of the Hilbert space $L^{2}.$ but as a point function ${\hat{f}}(t)$ is only determined almost everywhere. This is an important difference between the theory of Fourier transforms in ${\boldsymbol{L}}^{1}$ and in ${\boldsymbol{L}}^{2}$ The indeterminacy of ${\hat{f}}(t)$ t)) as a point function will cause some difficulties in the problem to which we now turn. 9.16 Translation-Invariant Subspaces of $\scriptstyle t^{2}\cdot\mathbf{A}$ subspace $\textstyle{M}$ of ${\boldsymbol{L}}^{2}$ is said to be $f(x-x).$ translation-invariant if $f\in M$ implies tha $\operatorname{tr}f_{x}\in M$ for every real α,where $\scriptstyle{i{\mathcal{A}}\ \;\equiv\;{\mathcal{X}}\;\ \;\;}$ Translations have already played an important part in our study o Fourier transforms. We now pose a problem whose solution will afford an illus- tration of how the Plancherel theorem can be used. (Other applications will occur in Chap.19.) The problem is: Describe the closed translation-invariant subspaces of $L^{2}.$ Let ${\cal{M}}$ be a closed translation-invariant subspace of $L^{2},$ and let $\hat{\cal M}$ be the image of $\textstyle{M}$ under the Fourier transform. Then $\hat{M}$ is closed (since the Fourier ${\hat{f}}e_{\alpha},$ transform is an $L^{2}.$ -isometry). $\operatorname{If}f_{x}$ 。 is a translate of ${\boldsymbol{f}},$ the Fourier transform $\tilde{M}$ is invariant is where extends to L, as can be seen from Theorem $9.13(d).$ ${\mathit{I}}t$ follows that $\operatorname{of}f_{\alpha}$ $e_{\alpha}(t)=e^{-i\alpha t};$ we proved this for $f\in L$ in Theorem 9.2; the result a.e. on $E,$ under multiplication by $e_{\alpha},f o r\;a l l\;\alpha\in R_{.}^{1}.$ $R^{1}.$ If $\hat{\cal M}$ is the set of all $\varphi\in L^{2}$ which vanish Let $\boldsymbol{E}$ be any measurable set in then $\hat{M}$ certainly is a subspace of $L^{2},$ which is invariant under multipli- cation by all ${\mathcal{C}}_{\alpha}$ (note that $|e_{s}|=1,$ so $\varphi e_{s}\in L^{2}$ if $\varphi\in L^{2})_{!}$ and $\hat{\cal M}$ is also closed Proof: $\varphi\in{\hat{M}}$ if and only if $\boldsymbol{\varphi}$ p is orthogonal to every $\psi\in L^{2}$ which vanishes a.e. on the complement of $\textstyle E.$ If $\textstyle{M}$ is the inverse image of this ${\hat{M}},$ ,under the Fourier transform, then $\textstyle{M}$ is a space with the desired properties. One may now conjecture that every one of our spaces ${\cal{M}}$ is obtained in this manner, from a set $E<R^{1}.$ To prove this, we have to show that to every closed translation-invariant $M\subset L^{2}$ there corresponds a set $E=R^{\prime}$ such that fe M if and only if ${\hat{f}}(t)=0$ a.e,on $\textstyle E.$ The obvious way of constructing $\boldsymbol{E}$ from ${\cal M}$ is to to define $\boldsymbol{E}$ as the intersection of these sets $E_{f},$ E consisting of all points at which ${\hat{f}}(t)=0,$ and associate with each fe M the set $E_{f}$ E . But this obvious attack runs into a serious diffculty: Each $E_{f}$ E is defined only up to sets of measure O.If $\langle A_{i}\rangle.$ is a countable collection of sets, each determined up to sets of measure O, then $\bigcap A_{i}$ is also determined up to sets of measure O. But there are uncountably many $f\in M,$ so we lose all control over $\bigcap{\ }_{2}E_{f.}$ This difficulty disappears entirely if we think of our functions as elements of the Hilbert space $L^{2},$ 2, and not primarily as point functions. We shall now prove the conjecture. Let $\tilde{M}$ be the image of a closed translation-invariant subspace $M<L^{2}.$ ,under the Fourier transform. Let ${\mathbf{}}P$ be theFOURIER TRANSFORMs 189 orthogonal projection of ${\boldsymbol{L}}^{2}$ onto $\tilde{M}$ (Theorem 4.11): To each $\scriptstyle f\in E^{k}$ there corre sponds a unique $P f\in{\tilde{M}}$ such $\mathrm{fat}f-P f$ is orthogonal to ${\widetilde{M}}.$ . Hence $$ f-P f\perp P g\qquad(f\operatorname{and}\ g\in L^{2}) $$ (1) and since $\hat{\cal M}$ is invariant under multiplication by $e_{\alpha\,:}$ we also have $$ f-P f\perp(P g)e_{x}\qquad(f{\mathrm{~and~}}g\in L^{2},\,\alpha\in R^{1}), $$ (2)) If we recall how the inner product is defined in $L^{2},$ we see that (2) is equivalent to $$ .\widetilde{\Delta}_{-\alpha}^{\ast\alpha}(f-P f)\cdot\overline{{{P g}}}\cdot e_{-\alpha}\,d m=0\qquad(f\mathrm{~and~}g\in L^{2},\,\alpha\in R^{1}) $$ (3) and this says that the Fourier transform of $$ (f-P f)\cdot{\overline{{P g}}} $$ (4) is O. The function (4) is the product of two $L^{2}.$ -functions, hence is in $L^{1},$ and the uniqueness theorem for Fourier transforms shows now that the function ((4) is $\mathbf{0}$ a.e. This remains true if ${\boldsymbol{P}}g$ is replaced by $P g.$ Hence $$ f\cdot P g=(P f)\cdot(P g)\qquad(f{\mathrm{~and~}}g\in L^{2}). $$ (5) Interchanging the roles of f and $\scriptstyle{\mathcal{G}}$ leads from (5) to $$ f\cdot P g=g\cdot P f\ ~\quad(f\ a n d~g\in L^{2}). $$ (6) Define Now let g be a fixed positive function in $L^{2}{\mathrm{:}}$ ; for instance, put $g(t)=e^{-\mathbf{i}t},$ $$ \varphi(t)={\frac{(P g|t)}{g(t)}}. $$ (7) (PgXt) may only be defined a.e.; choose any one determination in(7). Now (6) becomes $$ {\cal P}\!f=\varphi\cdot f\qquad(f\in{\cal L}^{2}). $$ (8) If $f\in{\widehat{M}},$ then $\scriptstyle{y=y}$ This says that $P^{2}=P,$ and it follows that $\varphi^{2}=\varphi,$ because $$ \varphi^{2}\cdot g=\varphi\cdot P g=P^{2}g=P g=\varphi\cdot g. $$ (9) Since $\varphi^{2}=\varphi,$ we have $\scriptstyle\phi=0$ or 1 a.e., and if we let $f\in L^{2}$ which are $\mathbf{0}$ a.e. on ${\boldsymbol{E}},$ where $\varphi(t)_{\frac{-}{c}}^{}\mathbf{0},$ then $\tilde{M}$ consists precisely of those $\boldsymbol{E}$ be the set of al ${\mathbf{}}t$ since $f\in{\widehat{M}}$ if and only $\operatorname{if}f=P f=\varphi\cdot f.$ we therefore obtain the following solution to our problem190 REAL AND coMPLEX ANALYsis 9.17 Theorem Associate to each measurable set $E=R^{\prime}$ the space $M_{E}$ of all $f\in L^{2}$ such that $\scriptstyle{J=0}$ a.e. on E. Then $M_{E}$ is a closed translation-invariant sub- is $M_{E}$ ; for some $E,$ space of $L^{2}.$ . EDery closed translation-inbariant subspace of ${\boldsymbol{L}}^{2}$ and $M_{A}=M_{B}$ if and only if $$ m(A-B)\cup(B-A))=0. $$ The uniqueness statement is easily proved; we leave the details to the reader The above problem can of course be posed in other function spaces. It ha been studied in great detail in ${\boldsymbol{L}}^{1}$ The known results show that the situation is infinitely more complicated there than in $L^{2}.$ The Banach Algebra ${\boldsymbol{L}}^{1}$ 9.18 Definition A Banach space $\scriptstyle A$ is said to be a Banach algebra if there is a multiplication defined in $\scriptstyle A$ which satisfies the inequality $$ \|x y\|\leq\|x\|\ \|y\|\qquad(x{\mathrm{~and~}}y\in A), $$ (1) the associative law x(yz) = (xy)z, the distributive laws $$ x(y+z)=x y+x z,\quad(y+z)x=y x+z x\quad(x,\,y,\,\mathrm{and}\,z\in A), $$ (2) and the relation $$ (x x)y=x(\alpha y)=\alpha(x y) $$ (3) where α is any scalar 9.19 Examples (a)) Let $A=C(X),$ where $X$ is a compact Hausdorff space,with the supremum norm and the usual pointwise multiplication of functions $(f g)(x)=f(x)g(x).$ This is a commutative Banach algebra $(f g=g f)$ with unit the constant function 1) (b) $C_{0}(R^{1})$ is a commutative Banach algebra without unit, ie., without ar element u such that uf = f for all f∈ C,(R')) (c)The set of all linear operators on $R^{k}$ *(or on any Banach space), with the operator norm as in Definition 5.3, and with addition and multiplication defined by $$ (A+B)(x)=A x+B x,\ \ \ (A B)x=A(B x), $$ is a Banach algebra with unit which is not commutative when $\operatorname{k>1}$ (d) ${\boldsymbol{L}}$ is a Banach algebra if we define multiplication by convolution; since $$ \|f\ast g\|_{1}\leq\|f\|_{1}\|g\|_{1}, $$ the norm inequality is satisfied. The associative law could be verified directly(an application of Fubini's theorem),but we can proceed asFOURIER TRANSFORMs 191 follows: We know that the Fourier transform of $f*\,g$ is $\textstyle\int\!s\,\ \,$ and we know that the mapping f→fis one-to-one. For $\textstyle\exp\exp\ell\in R^{1},$ $$ {\hat{f}}(t)[{\hat{g}}(t){\hat{h}}(t)]=[{\hat{f}}(t){\hat{g}}(t)]{\hat{h}}(t), $$ by the associative law for complex numbers. It follows that $$ f*(g*h)=(f*g)*h. $$ In the same way we see immediately that $f*g=g*f.$ The remaining requirements of Definition 9.18 are also easily seen to hold in ${\boldsymbol{L}}^{1}$ Thus ${\boldsymbol{L}}^{1}$ is a commutative Banach algebra. The Fourier transform is $\beta=1,$ an algebra isomorphism of ${\boldsymbol{L}}^{1}$ into $C_{\mathfrak{g}}$ . Hence there is no $\scriptstyle f\in D$ with and therefore ${\boldsymbol{L}}^{1}$ has no unit. 9.20 Complex Homomorphisms The most important complex functions on a Banach algebra $\scriptstyle A$ are the homomorphisms of $\scriptstyle A$ into the complex field. These are precisely the linear functionals which also preserve multiplication, i.e., the func- tions $\varphi$ such that $$ \varphi(\alpha x+\beta y)=\alpha\varphi(x)+\beta\varphi(y),\qquad\varphi(x y)=\varphi(x)\varphi(y) $$ for all $\scriptstyle{\mathcal{X}}$ and $\scriptstyle y\in A$ and all scalars α and ${\boldsymbol{\beta}}.$ B. Note that no boundedness assumption is made in this definition. It is a very interesting fact that this would be redundant: 9.21 Theorem If p is a complex homomorphism on a Banach algebra A, then the norm of $\varphi,$ as a linear functional, is at most ${\bf1}.$ Put Since PRoOF Assume, to get a contradiction, that $x=x_{0}/\lambda.$ Then $\left\Vert x\right\Vert<1$ and $\varphi(x)=1.$ for some $x_{0}\in A.$ $|\varphi(x_{0})|>\|x_{0}\|$ $\lambda=\varphi(x_{0}),$ and put and $\|x\|<1,$ the elements $\|x^{n}\|\leq\|x\|^{n}$ $$ s_{n}=-x-x^{2}-\cdot\cdot\cdot-x^{n} $$ (1) form a Cauchy sequence in A. Since $\scriptstyle A$ is complete, being a Banach space there exists a $y\in A$ such that $\|y-s_{n}\|\to0,$ and it is easily seen that $x+s_{n}=$ $x s_{n-1},$ so that $$ x+y=x y. $$ (2) Hence $\varphi(x)+\varphi(y)=\varphi(x)\varphi(y),$ which is impossible since $\varphi(x)=1.$ // 9.22 The Complex Homomorphisms of ${\boldsymbol{L}}^{1}$ Supposep is a complex homo- morphism of $L^{1}.$ }, i.e., a linear functional (of norm at most 1, by Theorem 9.21) which also satisfies the relation $$ \varphi(f*g)=\varphi(f)\varphi(g)\qquad(f\operatorname{and}g\in L^{1}). $$ (1)192 REAL AND coMPLEX ANALYSsis By Theorem 6.16, there exists a ${\mathfrak{s e}}\,G^{\mathfrak{t}}$ such that $$ \varphi(f)= |_{-\infty}^{\infty}f(x)\beta(x)\ d m(x)\qquad(f\in L^{1}). $$ (2) hand, We now exploit the relation (1) to see what else we can say about ${\boldsymbol{\beta}}.$ On the one $$ \begin{array}{r l}{\varphi(f*g)=\left\{\sum_{-\infty}^{\infty}(f*g)(x)\beta(x)\,d m(x)}\\ {\vdots\,=\displaystyle{\int_{-\infty}^{\infty}\beta(y)\,d m(x)}\right\}_{-\infty}^{\infty}f(x-y)\beta(y)\,d m(y)}\\ {=\displaystyle{\int_{-\infty}^{\infty}\beta(y)\,d m(x)}\displaystyle{\int_{-\infty}^{\infty}f(x)\beta(x)\,d m(y)}}\end{array} $$ (3) On the other hand $$ \varphi(f)\varphi(g)=\varphi(f)\stackrel{ harpoonup}{\displaystyle\bigcup_{-\infty}^{\infty}}\varrho(y)\beta(y)\;d m(y). $$ (4) Let us now assume that $\varphi$ is not identically O.Fix f∈ ${\boldsymbol{L}}^{1}$ so that $\varphi(f)\neq0.$ Since the last integral in (3) is equal to the right side of (4) for every $\scriptstyle g\in E,$ , the uniqueness assertion of Theorem 6.16 shows that $$ \varphi(f)\beta(y)=\varphi(f_{y}) $$ (5) for almost all $y.$ But $y{ arrow}f_{v}$ is a continuous mapping of $R^{1}$ into ${\boldsymbol{L}}^{1}$ (Theorem 9.5) and $\varphi$ does not affect (2)] that $\boldsymbol{\beta}$ is continuous on L. Hence the right side of (5) is a continuous function of y by $x+y$ and thenf byf. y, and we may assume [by changing B(y) on a set of measure $\mathbf{0}$ if necessary, which is continuous. If we replace $\mathbf{\vec{y}}$ in (5), we obtain $$ \varphi(f)\beta(x+y)=\varphi(f_{x+y})=\varphi(f_{x})\beta=\varphi(f_{x})\beta(y)=\varphi(f)\beta(x) $$ :)βOy), so that $$ \beta(x+y)=\beta(x)\beta(y)\qquad(x{\mathrm{~and~}}y\in R^{1}). $$ (6) Since $\boldsymbol{\beta}$ is not identically 0,(6))implies that $\beta(0)=1,$ and the continuity of $\beta$ shows that there is a $\scriptstyle\mathbf{\hat{\theta}}>0$ such that $$ \bigcap_{0}^{\delta}\beta(y)\;d y=c\neq0. $$ (7)FOURIER TRANSFORMs 193 Then $$ c\beta(x)=\left[\stackrel{\circ}{\delta}_{0}^{}\beta(y)\beta(x)\;d y= (\stackrel{\circ}{\delta}_{0}^{\prime}\beta(y+x)\;d y=\right]_{x}^{x+\delta}\beta(y)\;d y. $$ (8) Since $\boldsymbol{\beta}$ is continuous, the last integral is a diferentiable function of $X;$ ; hence (8) shows that $\boldsymbol{\beta}$ is differentiable. Differentiate (6) with respect to $y_{\mathrm{:}}$ , then put $y=0;$ the result is $$ \beta(x)=A\beta(x),\qquad A=\beta^{\prime}0). $$ (9) Hence the derivative of $\beta(x)e^{-\lambda x}$ is O, and since $\beta(0)=1,$ we obtain $$ \beta(x)=e^{\alpha x}. $$ (10) But $\boldsymbol{\beta}$ is bounded on $R^{1}.$ Therefore $\scriptstyle A$ must be pure imaginary, and we conclude: There exists a t $\scriptstyle{\pi R^{\prime}}$ such that $$ \beta(x)=e^{-i x}. $$ (11) We have thus arrived at the Fourier transform 9.23 Theorem To every complex homomorphism p on ${\boldsymbol{L}}^{1}$ (except to9 = 0) there corresponds a unique $\scriptstyle t\in R^{\prime}$ such that $\varphi(f)=f(t).$ The existence of t was proved above. The uniqueness follows from the obser- vation that ift ≠ s then there exists an $\scriptstyle f\in D$ such $\operatorname{that}{\hat{f}}(t)\neq{\hat{f}}(s);$ take for fx) a suitable translate of $e^{-|x|}.$ Exercises 1 Suppose f e $L^{1},f>0.$ Prove that $|\,{\hat{f}}(y)|<{\hat{f}}(0){\mathrm{~for~every~}}y\neq\mathbb{C}$ 2 Compute the Fourier transform of the characteristic function of an interval. For n = 1,2,3,……, et ${\mathfrak{g}}_{n}$ a function $f_{n}\in L^{1};$ be the characteristic function of [-n,n], let ${\boldsymbol{h}}$ be the characteristic function o $\scriptstyle{\mathrm{tot.}}$ 1], and compute ${\mathit{g_{n}}} leftharpoons h$ explicitly.(CThe graph is piecewise linear.) Show that ${\mathit{g_{n}}}\,\ast\,h$ is the Fourier transform of except for a multiplicative constant, $$ f_{*}(x)={\frac{\sin x\sin n x}{x^{2}}} $$ Show that $\|f_{n}\|_{1}\to\infty$ and conclude that the mappingJ→fmaps $L^{1}$ into a proper subset o ${\cal C}_{0}\,.$ Show, however, that the range of this mapping is dense in ${\cal C}_{0}\,.$ 3 Find $$ \operatorname*{lim}_{4\to\infty}\,\prod_{-_{A}}^{4}{\frac{\sin\,{\lambda}t}{t}}\,e^{i x}\,d t\qquad(-\,\infty<x<\infty) $$ where Ais a positive constant 4 Give examples of fe ${\boldsymbol{L}}^{2}$ such that f生 ${\boldsymbol{L}}^{1}$ but fe $L^{1}.$ Under what circumstances can this happen?194 REAL AND coMPLEX ANALYSIs 5 If $f\in L^{1}$ and $\textstyle\int|t^{j}(t)|\,d m(t)<\infty,$ prove that $\scriptstyle{\hat{f}}$ coincides a.e. with a diferentiable function whose derivative is $$ i\left|\begin{array}{c}{{\phi_{0}}}\\ {{\left|\beta\right|}}\end{array}\right|\hat{J}\left|\left<\left|\beta^{\left|1\right|}\right|\left<\hat{J}\right|\hat{J}\right|\right|\hat{J} |, $$ 6 Suppose $f\in L^{1},f_{\geq}$ differentiable almost everywhere, and $f^{\prime}\in L^{1}$ Does it follow that the Fourier transform $\operatorname{off}^{\prime}$ is $t i{\hat{f}}(t){\bar{Y}}$ 7 Let $\boldsymbol{\mathsf{S}}$ be the class of all functions $f_{\mathrm{i}}$ on $R^{1}$ which have the following property: fis infinitely differen- tiable, and there are numbers $A_{m n}(f)<\infty,$ for ${\mathfrak{m}}\,$ and $n=0,1,2,\ldots,$ such that $$ |x^{n}D^{m}f(x)|\leq A_{m n}(f)\qquad(x\in R^{1}). $$ Here ${\mathbf{}}D{\mathbf{}}$ is the ordinary differentiation operator $\boldsymbol{\mathsf{S}}$ onto ${\bar{S}}.$ Find examples of members of S Prove that the Fourier transform maps 8 If ${\boldsymbol{p}}$ and $\scriptstyle{\mathcal{A}}$ are conjugate exponents, $f\in D,\;g\in B,$ and $h=f*g,$ prove that ${\boldsymbol{h}}$ is uniformly contin- uous. If also $1<p<\infty,$ then $h\in C_{0};$ show that this fails for some f ∈ $\scriptstyle{E_{\circ}g\in E^{\circ}}$ 9 Suppose $1\leq p<\infty,f\in D,$ and $$ g(x)= .{\overline{{x}}}_{x}^{*+1}f(t)\;d t. $$ Prove that $g\in C_{0}.$ What can you say about $g\operatorname{iff}f\in L^{\infty}:$ all 10 Let $C^{\infty}$ be the class fall infnitely differentiable complex functions on $R^{1},$ and let $C_{c}^{\infty}$ consist of $g\in C^{\infty}$ whose supports are compact. Show that $C_{\epsilon}^{\alpha}$ does not consist of O alone. able and $|{\mathrm{f}}f\in L_{\mathrm{be}}^{1}$ and ge Let Ll。be the class of all f which belong to L $L^{1}$ locally; that $\mathrm{i}s,f\in L_{\mathrm{ist}}^{1}$ provided that fis measur- $\textstyle{ \lceil I\rceil<\alpha \rceil}$ for every bounded interval $I.$ $\textstyle\bigotimes_{c}^{\infty}y$ prove that f * $\{{\mathcal O}_{n}\}$ in $C_{\epsilon}^{\infty}$ such that $g\in C^{\infty}$ Prove that there are sequences $$ \|f*g_{n}-f\|_{1}\Longrightarrow0 $$ as $n arrow\infty,$ for every $f\in L^{1}$ (Compare Theorem 9.10.) Prove that $\{g_{n}\}$ the convergence occurs at every point $(f*\,g_{n})(x)\to f(x)$ a.e., for $e v e r y f\in L_{\mathrm{ist}}^{1}$ in fact, for suitable $\{g_{n}\}$ can also be so chosen that x at which fis the derivative of is indefinite integral Prove that $(f*h_{\lambda})(x)\to f(x)$ a.e. $\operatorname{if}f\in L^{1},$ as ${\boldsymbol{\lambda}} arrow0,$ and $\operatorname{that}f*h_{\lambda}\in C^{\infty},$ although $h_{\lambda}$ does not have compact support. $(h_{\lambda}$ is defined in Sec. 9.7. 11 Find conditions on fand/or fwhich ensure the correctness of the following formal argument: If $$ \varphi(t)={\frac{1}{2\pi}}\int_{-\infty}^{\infty}f(x)e^{-i t x}~d x $$ and $$ F(x)=\sum_{k=-\infty}^{\infty}f(x+2k\pi) $$ then F ${\mathbf{}}F$ is periodic, with period 2元, the nth Fourier coefficient of ${\mathbf{}}F$ is p(n) hence $F(x)=\sum$ one-. particular, $$ \sum_{k=-\alpha\sigma}^{\alpha}f(2k\pi)=\sum_{n=-\alpha\sigma}^{\alpha}\varphi(n). $$ More generally $$ \sum_{k=-\infty}^{\infty}f(k\beta)=\alpha\sum_{n=-\infty}^{\infty}\varphi(n\alpha)\qquad{\mathrm{if~}}\alpha>0,\;\beta>0,\,\alpha\beta=2\pi. $$ (*)FOURIER TRANSFORMs 195 What does() say about the limit, as $x\to0,$ of the right-hand side (for “nice" functions, of course)?? Is this in agreement with the inversion theorem ? $\textstyle[\![{}^{\circ})$ is known as the Poisson summation formula.] 12 $\Gamma_{\mathrm{ake}}f(x)=e^{-\left|x\right|}$ in Exercise ll and derive the identit $$ \frac{e^{2\pi a}+1}{e^{2\pi a}-1}=\frac{1}{\pi}\sum_{n=-\infty}^{\infty}\frac{\phi}{\alpha^{2}+n^{2}}. $$ 13 If O <c< oo, define ${\tilde{f}}_{c}$ . Hint: If g =,,an integration by parts gives $2c\varphi^{(t)}(t)+t\varphi(t)=0.$ (a)Compute $f_{c}(x)_{=}\exp\,(-c x^{2}).$ (d) Take f $\mathrm{\stackrel{r}{=}}\int_{\cal C}$ (b)Show that there is one (and only one) c for which ${\hat{f}}_{c}=f_{c}\,.$ ${\mathfrak{b}}.$ (c)Show that f $*f_{b}=y f_{c};$ ; find $\gamma$ and cexplicitly in terms of a and in Exercise 11. What is the resulting identity? 14 The Fourier transform can be defined $\mathrm{for}f\in L^{1}(R^{k})$ by $$ \hat{f}(y)=\int_{R^{3}}^{}f(x)e^{-i x\cdot y}\;d m_{k}(x)\qquad(y\in R^{k}), $$ wher ${\mathrm{e}}\ x\ \cdot\ y=\sum\ \xi_{i}\,\eta_{i}\,\mathrm{if}\ x=(\xi_{1},\ \cdot\cdot,\xi_{k}),\,y$ = (n,……,nx), and $m_{k}$ L is Lebesgue measure on $R^{k}{}_{;}$ divided by $(2\pi)^{k/2}$ for convenience. Prove the inversion theorem and the Plancherel theorem in this context, as well as the analogue of Theorem 9.23 15 $\operatorname{I\!I}f\in L^{1}(R^{k}),$ A is a linear operator on $R^{k},$ and $g(x)=f(A x),$ how is $\hat{\boldsymbol{g}}$ related tof? If is invariant same is true $\mathrm{of}{\hat{f}}.$ under rotations,ie., if f(x) depends only on the euclidean distance of $\textstyle X$ from the origin, prove that the 16 The Laplacian of a function fon ${\boldsymbol{R}}^{k}$ is $$ \Delta f=\sum_{j=1}^{k}{\frac{\partial^{2}f}{\partial x_{j}^{2}}}, $$ integrability conditions are satisfied? provided the partial derivatives exist. What is the relation between f and gif $g=\Delta f$ and all necessary $\mathbf{t}_{\mathrm{t}}$ It is clear that the Laplacian commutes with translations. Prove that it also commutes with rotations, .e., that $$ \Delta(f\circ A)=(\Delta f)\circ A $$ whenever f has continuous second derivatives and $\scriptstyle A\quad\quad A$ is a rotation of ${\boldsymbol{R}}^{k}$ (Show that it is enough to do this under the additional assumption that f has compact support.) 17 Show that every Lebesgue measurable character of $R^{1}$ is continuous. Do the same for $R^{k},$ (Adapt part of the proof of Theorem 9.23) Compare with Exercise 18 18 Show (with the aid of the Hausdorff maximality theorem) that there exist real discontinuous func- tionsfon $R^{1}$ such that $$ f(x+y)=f(x)+f(y) $$ (1) for all $\textstyle{\mathcal{X}}$ and $\scriptstyle{\epsilon\times K^{\prime}}$ Show that if (1) holds and fis Lebesgue measurable, then fis continuous Show that if (1) holds and the graph of fis not dense in the plane, then fis continuous Find all continuous functions which satisfy (1) 19 Suppose $\scriptstyle A\quad}$ and $\bar{\boldsymbol{B}}$ are measurable subsets of $R^{1},$ having finite positive measure.Show that the convolution $\chi_{A}*\chi_{B}$ is continuous and not identically O. Use this to prove that $A+B$ contains a segment (A different proof was suggested in Exercise 5, Chap. 7)