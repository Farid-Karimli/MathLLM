CHAPTER SIXTEEN ANALYTIC CONTINUATION In this chapter we shall be concerned with questions which arise because func tions which are defined and holomorphic in some region can frequently be extended to holomorphic functions in some larger region. Theorem 10.18 shows that these extensions are uniquely determined by the given functions. The exten sion process is called analytic continuation. It leads in a very natural way to the consideration of functions which are defined on Riemann surfaces rather than in plane regions. This device makes it possible to replace“multiple-valued functions”(such as the square-root function or the logarithm) by functions. A systematic treatment of Riemann surfaces would take us too far afield, however, and we shall restrict the discussion to plane regions. Regular Points and Singular Points 16.1 Definition Let $D\!\!\!\!/$ be an open circular disc, suppose f ∈ $m(p_{i})$ and let $\beta$ be a boundary point of $D.$ We call $\boldsymbol{\beta}$ a regular point of f if there exists a disc for al $z\in D\bigcap\sim$ $D_{1}$ with center at $\boldsymbol{\beta}$ and a function $g\in H(D_{1})$ such that $g(z)=f(z)$ $D_{1}.$ Any boundary point of $D\!\!\!\!/$ which is not a regular point of f is called a singular point of f It is clear from the definition that the set of all regular points of f is an open (possibly empty) subset of the boundary of ${\boldsymbol{D}}.$ $U$ for $D,$ without any In the following theorems we shall take the unit disc loss of generality 319320 REAL AND coMPLEX ANALYSIS 16.2 Theorem Suppose fe H(U), and the power series $$ f(z)=\sum_{n=0}^{\infty}a_{n}z^{n}\qquad(z\in U) $$ (1) has radius of convergence 1. Then f has at least one singular point on the unit circle ${\boldsymbol{T}}.$ The compactness of PRoor Suppose, on the contrary, that every point of ${\mathbf{}}T$ is a regular point of f and ${\mathbf{}}T$ 'implies then that there are open discs $v_{1},\ldots,\cdot$ $D_{n}$ functions $g_{j}\in H(D_{j})$ such that the center of each in $V_{i j}$ . Since $D_{i}\cap D_{j}$ is connected, it $r\in$ function ${\boldsymbol{h}}$ $D_{i}\cap D_{j}\neq\varnothing$ and $V_{i j}=D_{i}\cap D_{j}\cap U,$ then $D_{j}$ ); is on ${\boldsymbol{T}},$ such that If $D_{1}\ \cup\ \cdots\ \cup\ D_{n},$ and such that $g_{j}(z)=f(z)$ in $D_{j}\cap U,{\mathrm{for~}}j=1,\ldots,n.$ (since the centers of the $D_{j}$ follows from Theorem 10.18 that $g_{i}=g_{j}$ in $D_{i}\sim D_{j}.$ $V_{i}\neq{\mathcal{O}}$ are on T), and $g_{i}=f=g_{j}\mathrm{~i~}$ Hence we may define a h in $\Omega=U\cup D_{1}\cup\dots\ \cup\ D_{n}\lor$ by $$ h(z)=\left\{f(z)\right.\ \ \ \ (z\in U), $$ (2) Since $\Omega\Rightarrow{\bar{U}}$ and $\Omega$ is open,there exists an $\scriptstyle x\;>\;0$ such that the disc contrary $D(0;1+\epsilon)\subset\Omega.$ But he H(Q),h(z) is given by (1) in $\mathbf{(1)}$ is at least $1+\epsilon,$ ${\boldsymbol{U}},$ and now Theorem 10.16 implies that the radius of convergence of to our assumption. // then ${\mathbf{}}T$ 16.3 Definition If $f\in H(U)$ and if every point of ${\mathbf{}}T$ is a singular point of $f,$ is said to be the natural boundary of ${\boldsymbol{f}}.$ . In this case,J $\boldsymbol{\mathsf{f}}$ has no holo- morphic extension to any region which properly contains ${\boldsymbol{U}}.$ 16.4 Remark It is very easy to see that there exist f∈ $H(U)$ for which ${\mathbf{}}T$ is a natural boundary. In fact, if $\Omega$ is any region, it is easy to find an $f\in H(\Omega)$ which has no holomorphic extension to any larger region. To see this, let $\scriptstyle A$ be any countable set in $\Omega$ which has no limit point in $\Omega$ but such that every boundary point of $\Omega$ is a limit point of A. Apply Theorem 15.11 to get a function $f\in H(\Omega)$ which is O at every point of $\scriptstyle A$ 4 but is not identically O.f $g\in H(\Omega_{1}),$ where $\Omega_{1}$ is a region which properly contains $\Omega,$ , and if $g=f\ln\Omega,$ the zeros of $\scriptstyle{\mathcal{G}}$ g would have a limit point in $\Omega_{1},$ and we have a contradiction. A simple explicit example is furnished by $$ f(z)=\sum_{n=0}^{\infty}z^{2^{n}}=z+z^{2}+z^{4}+z^{8}+\cdots\qquad(z\in U). $$ (1) This f satisfies the functional equation $$ f(z^{2})=f(z)-z, $$ (2) from which it follows (we leave the details to the reader) that fis unbounded on every radius of $U$ which ends at exp $\{2\pi i k/2^{n}\}.$ where $\boldsymbol{k}$ and $\scriptstyle n$ are positiveANALYTIC cONTINUATION 321 integers. These points form a dense subset of ${\boldsymbol{T}}{\boldsymbol{;}}$ and since the set of all singular points of fis closed, f has ${\mathbf{}}T$ as its natural boundary That this example is a power series with large gaps (i.e., with many zero coefficients) is no accident. The example is merely a special case of Theorem 16.6, due to Hadamard, which we shall derive from the following theorem of Ostrowski: 16.5 Theorem Suppose 入 $p_{k},$ and $q_{k}$ are positive integers, $$ p_{1}<p_{2}<p_{3}<\cdots, $$ and $$ \lambda q_{k}>(\lambda+1)p_{k}\qquad(k=1,\,2,\,3,\,\ldots). $$ (1) Suppose $$ f(z)=\sum_{n=0}^{\infty}a_{n}z^{n} $$ (2) has radius of convergence 1, and $a_{n}=0$ whenever $p_{k}\<n<q_{k}$ for some k. I $\scriptstyle{j\neq i}$ sequence is the pth partial sum of (2), and j β is a regular point of f on ${\boldsymbol{T}},$ then the $\left\{S_{p_{k}}(z)\right\}$ converges in some neighborhood of β Note that the full sequence $\{s_{p}(z)\}$ cannot converge at any point outside ${\bar{U}}.$ The gap condition (1) ensures the existence of a subsequence which converges in a neighborhood of ${\boldsymbol{\beta}},$ hence at some points outside ${\bar{U}}.$ This phenomenon is called overconvergence PRoor If $g(z)=f(\beta z),$ then $\scriptstyle{\mathcal{G}}$ also satisfies the gap condition. Hence we may has a holomorphic assume, without loss of generality, that which contains $U\cup\{1\}.$ Put $\boldsymbol{\mathsf{f}}$ $\beta=1.$ Then extension to a region $\Omega$ $$ \varphi(w)={\textstyle{\frac{1}{2}}}(w^{\lambda}+w^{\lambda+1}) $$ (3) an and define $F(w)=f(\varphi(w))$ for all w such that $\varphi(w)\in\Omega,{\mathrm{~If~}}|w|\leq1$ but $w\neq1,$ then $|\varphi(w)|<1,$ since $|1+w|<2.$ AIso, $\varphi(1)=1.$ It follows that there exists $\scriptstyle e\;>0$ Such that $\varphi(D(0;1+\epsilon))\subset\Omega.$ Note that the region $\varphi(D(0;\ 1+\epsilon))$ contains the point 1. The series $$ F(w)=\sum_{m=0}^{\infty}b_{m}\,w^{m} $$ (4) converges $\mathbb{F}\left|w\right|<1+$ e.322 REAL AND coMPLEX ANALYSis The highest and lowest powers of w in [o(w)]" have exponents $(\lambda+1)n$ and An. Hence the highest exponent in [cp(w)]” is less than the lowest expo- nent in [p(w)]*, by (1). Since $$ F(w)=\sum_{n=0}^{\infty}a_{n}[\varphi(w)]^{n}\qquad(|w|<1), $$ (5) the gap condition satisfied by $\{a_{n}\}$ now implies that $$ \sum_{n=0}^{p_{k}}a_{n}[\varphi(w)]^{n}=\sum_{m=0}^{(\lambda+1)p_{k}}b_{m}\,w^{m}\qquad(k=1,\,2,\,3,\,..\,.). $$ (6) The right side of ((6) converges,as $k\to\infty.$ whenever $|\mathbf{w}|<1+\epsilon.$ Hence $\left\{S_{p_{i}}(z)\right\}$ converges for all $z\in\varphi(D(0;1+\epsilon)).$ This is the desired conclusion. // Note: Actually, {s,(z)} converges uniformly in some neighborhood of ${\boldsymbol{\beta}}.$ . We leave it to the reader to verify this by a more careful examination of the preceding proof. 16.6 Theorem Suppose Ais a positive integer, $\{p_{k}\}$ is a sequence of positive integers such that $$ p_{k+1}>\left(1+{\frac{1}{\lambda}}\right)p_{k}\qquad(k=1,\,2,\,3,\ldots), $$ (1) and the power series $$ f(z)=\sum_{k=1}^{\infty}c_{k}\,z^{p_{k}} $$ (2) has radius of convergence 1. Then f has T.as its natural boundary PRooF The subsequence $\{s_{p_{k}}\}$ of Theorem 16.5 is now the same (except for repetitions) as the full sequence of partial sums of (). The latter cannot con verge at any point outside $\bar{U}$ ; hence Theorem 16.5 implies that no point of ${\mathbf{}}T$ can be a regular point of f // exp 16.7 Example Put $\scriptstyle u_{n}=1$ i $\scriptstyle n$ is a power of $2,$ put $a_{n}=0$ otherwise, pu $\eta_{n}=$ $(-{\sqrt{n}}),$ and define $$ f(z)=\sum_{n=0}^{\infty}a_{n}\eta_{n}z^{n}. $$ (1) Since l $$ \operatorname*{lim}_{n arrow\infty}\h|\:a_{n}\eta_{n}|^{1/n}=1, $$ (2)ANALYrIc cONTINUATION 323 the radius of convergence of $\mathbf{(1)}$ )is 1. By Hadamard's theorem, f has ${\mathbf{}}T$ as its natural boundary. Nevertheless, the power series of each derivative of f $$ f^{(k)}(z)=\sum_{n=k}^{\infty}n(n-1)\cdot\cdot\cdot\cdot(n-k+1)a_{n}\,\eta_{n}\,z^{n-k}, $$ (3) converges uniformly on the closed unit disc. Each $f^{(k)}$ ) is therefore uniformly continuous on ${\tilde{U}},$ and the restriction of f to ${\mathbf{}}T$ is infinitely differentiable, as a function of e, in spite of the fact that ${\mathbf{}}T$ is the natural boundary of f The example demonstrates rather strikingly that the presence of singularities in the sense of Definition 16.1, does not imply the presence of discontinuities or (stated less precisely) of any lack of smoothness. This seems to be the natural place to insert a theorem in which continuity does preclude the existence of singularities: 16.8 Theorem Suppose Q is a region, ${\boldsymbol{L}}$ L is a straight line or a circular arc $\scriptstyle\Omega\,-\,L$ holomorphic in $\Omega_{1}$ is the union of two regions $\Omega_{1}$ 2. Then f is holomorphic in $\Omega.$ is continuous in $\Omega,$ and $\boldsymbol{\mathsf{f}}$ is and $\mathbf{a}_{2},f$ and in S $\Omega_{2}$ PRoOF The use of linear fractional transformations shows that the general case follows if we prove the theorem for straight lines $L.$ By Morera's theorem, it is enough to show that the integral of f over the boundary aA is $\mathbf{0}$ for every triangle $\Delta$ in Q.The Cauchy theorem implies that the integral of J $\boldsymbol{\f}$ vanishes over every closed path $\scriptstyle{\mathcal{Y}}$ in A $\mathbb{1}\cap\Omega_{1}$ or in $\Delta\cap\Omega_{2}$ . The continuity of shows that this is still true if part of $\gamma$ is in $L,$ and the integral over A is the sum of at most two terms of this sort. /// Continuation along Curves open circular disc and fe 16.9 Definitions A function element is an ordered pair (, Two function elements $(f_{0},D_{0})$ and $(f_{1},D_{1})$ $D),$ where $D\!\!\!\!/$ is an $m(p)$ are direct continuations of each other if two conditions hold : $D_{0}\cap D_{1}\neq\varnothing,$ and fo(z) =/,(z) for all $z\in D_{0}\cap D_{1}.$ In this case we write $$ (f_{0},\,D_{0})\sim(f_{1},\,D_{1}). $$ (1) A chain is a finite sequence $\mathcal{C}$ of discs, say ${\mathcal{C}}=\{D_{0},\;D_{1},\ldots,D_{n}\},$ such that $D_{i-1}\cap D_{i}\neq\varnothing$ fo ${\textsf{r}}i=1,\ldots,n.{\mathbb{F}}(f_{0},D_{0})$ is given and if there exist elements $(f_{i},\,D_{i})$ such that $\scriptstyle(y_{c},$ $,\,D_{i-1}\rangle\sim(f_{i},\,D_{i})$ for $i=1,\ldots,n,$ then $(f_{n},\,D_{n})$ is said to mined by Be the analytic coninuation o C6 $D_{0})$ along G. Note Then $g_{1}=f_{0}=f_{1}$ in $D_{0}\sim D_{1};$ now $f_{0}$ $\mathrm{that}f_{n}$ is uniquely deter- and Güf it exists at aln) To se this, suppose (1) holds, and and since suppose(1) also holds with ${\mathfrak{g}}_{1}$ in place $\mathfrak{o l}_{i,\cdot}$ in $D_{1}.$ The uniqueness ${\mathfrak{o f}}f_{n}$ $D_{1}$ is connected, we haye $y_{1}=f_{1}$ ${\mathcal{C}}.$ follows by induction on the number of terms in324 REAL AND coMPLEX ANALYSIs If $(f_{n},\,D_{n})$ is the continuation of $(f_{0},D_{0})$ along G, and if $D_{n}\cap D_{0}\neq\varnothing,{\mathrm{it}}$ is not need not be true that $(f_{0},\,D_{0})\sim(f_{n},\,D_{n});$ in other words, the relation ${}^{\prime}\hookrightarrow$ y transitive. The simplest example of this is furnished by the square-root func- center of $D_{0},\gamma(1)$ is the center of $D_{n},\operatorname{anc}$ d be discs of radius 1, with centers 1, 0, and and so that $(f_{0},D_{0})\sim$ $\omega^{2},$ tion: Let $D_{0},D_{1},$ and $D_{2}$ where $\omega^{3}=1.$ choose $f_{j}\in H(D_{j})$ so that $f_{j}^{2}(z)=z$ with parameter inter A chain $(f_{1},\,D_{1}),\,(f_{1},\,D_{1})\sim(f_{2},\,D_{2}).\,\mathrm{In}\,\,i$ D。o D, we $\mathrm{have}\,f_{2}=-f_{0}\neq f_{0}\,.$ $\scriptstyle{\mathcal{Y}}$ such that y(O) is the ${\mathcal{C}}=\{D_{0},\ldots,D_{n}\}$ is said to cover a curve val [0,1] if there are numbers $0=s_{0}<s_{1}<\cdot\cdot\cdot<s_{n}=1$ $$ \gamma(\left[s_{i},\,s_{i+1}\right])\subset D_{i}\qquad(i=0,\,\ldots,\,n-1). $$ (2) If $(f_{0},\ D_{0})$ can be continued along this $\textstyle{\mathcal{C}}$ to $(f_{n},D_{n}),$ we call $(f_{n},\,D_{n})$ an analytic continuation of $D_{0})$ is then said to admit an analytic continuation along Theorem 16.11); (fo, $(f_{0},D_{0})$ along y(uniqueness will be proved in P. Although the relation (1) is not transitive,a restricted form of transitivity does hold. It supplies the key to the proof of Theorem 16.11. 16.10 Proposition Suppose that $D_{0}\cap D_{1}\cap D_{2}\neq{\mathcal{D}},\;(D_{0},f_{0})\sim(D_{1},f_{1}),$ and (D, f)~(D2,f2). Then (Do,Jfo)~(D2,fz) $D_{0}\cap D_{2}$ PROoF By assumption, $\scriptstyle f_{i}=f_{i}$ in $D_{0}\cap D_{1}$ and f $\scriptstyle-j_{2}$ in $D_{1}\cap D_{2}$ Hence in $\scriptstyle J_{0}=J_{2}$ in the nonempty open set $D_{0}\cap D_{1}\cap D_{2}.$ Since $f_{0}$ and $\ f_{2}$ are holo- // morphic in $D_{\mathrm{o}}\cap D_{2}$ and $D_{\mathrm{o}}\cap D_{2}$ is connected, it follows that $f_{0}=f_{2}$ 16.11 Theorem If (f, ${\boldsymbol{D}})$ is a function element and ${\mathfrak{f}}\gamma$ is a curve which starts at $\scriptstyle{\gamma}.$ the center of ${\boldsymbol{D}},$ then $\scriptstyle(J,D)$ admits at most one analytic continuation along Here is a more explicit statement of what the theorem asserts: $\operatorname{If}\,\gamma$ is covered if by chains ${\mathcal{C}}_{1}=\{A_{0},\,A_{1},\,...,\,A_{m}\}$ and ${\mathcal{C}}_{2}=\{B_{0},\,B_{1},\,\dots\,B_{n}\},$ where $A_{0}=B_{0}=D,$ in and if $\scriptstyle(y,\,b)$ can be analytically continued along ${\mathcal{C}}_{2}$ to a function element then $g_{m}=h_{n}$ $\scriptstyle(\scriptstyle I,\,D\rangle$ can be analytically continued along ${\mathcal{C}}_{1}$ to $(b_{n},\,B_{n}),$ $(g_{m},\;A_{m}),$ that $A_{m}\cap B_{n}.$ and $B_{n}$ are, by assumption, discs with the same center y(1),it follows and we may as well Since $A_{m}$ have the same expansion in powers of $z-\gamma(1),$ ${\mathcal{G}}_{m}$ and $h_{n}$ by whichever is the larger one of the two. With this agreement replace $A_{m}$ and $B_{n}$ $g_{m}=h_{n}$ the conclusion is that PR0OF Let ${\mathcal{C}}_{1}$ and ${\mathcal{C}}_{2}$ be as above. There are numbers $$ 0=s_{0}<s_{1}<\cdot\cdot\cdot<s_{m}=1=s_{m+1} $$ and 0 = 0。<01 <0。=1 = 0.+ such that $$ \gamma([s_{i},s_{i+1}])\subset A_{i},\qquad\gamma([\sigma_{j},\sigma_{j+1}])\subset B_{j}\qquad(0\leq i\leq m,\,0\leq j\leq n).\quad(1) $$ANALYTic coNTiNUAToN 325 [o,, $\sigma_{j+}$ There are function elements and $0\leq j\leq n-1.$ ${\mathrm{It}}$ is clear that then and $(h_{j},\,B_{j})\sim(h_{j+1},\,B_{j+1}),$ intersects for Theni≥1, and since [s;, $\scriptstyle\mathbb{N}_{\mathrm{stal}}$ $(g_{i},\,A_{i})\sim(g_{i+}$ Here 1, $\scriptstyle A_{i+3}$ and if $[s_{i},s_{i+1}]$ $s_{i}\geq\sigma_{j}.$ $0\leq i\leq m-1$ and $0\leq j\leq n,$ $i+j>0.$ Suppose one for which $\scriptstyle{i+j}$ is minimal. intersects $g_{0}=h_{0}=f.$ ], we see that We claim that if $0\leq i\leq m$ for which this is wrong. Among them there is 1], then $(g_{i},\;A_{i})\sim(h_{j},\;B_{j}),$ $\scriptstyle(\log\theta$ Assume there are pairs $[\sigma_{j},\sigma_{j+}$ $$ \gamma(s_{i})\in A_{i-1}\cap A_{i}\cap B_{j}. $$ (2) The minimality ofi+j shows that $(g_{i-1},\,A_{i-1})\sim(h_{j},\,B_{j});$ and since This $(g_{i-1},\;A_{i-1})\sim(g_{i},\;A_{i}),$ Proposition 16.10 implies that $(g_{i},\;A_{i})\sim(h_{j},\;B_{j}).$ contradicts our assumption. The possibility $s_{i}\leq\sigma_{j}$ is ruled out in the same way. So our claim is established. In particular, it holds for the pair (m, n), and this is what we had to prove. // 16.12 Definition Suppose $\scriptstyle{\dot{\alpha}}$ and $\boldsymbol{\beta}$ are points in a topological space $X$ and $\varphi$ $X$ is a continuous mapping of the unit square and p(1, $\scriptstyle{\theta=\beta}$ for all $\textstyle{\epsilon\,L.}$ The curves y, defined by into $I^{2}=I\times I$ (where $I=[0,1])$ such that p(O, $t)=\alpha$ $$ \gamma_{t}(s)=\varphi(s,\,t)\qquad(s\in I,\,t\in I) $$ (1) are then said to form a one-parameter family $\{\gamma_{t}\}$ of curves from α to $\boldsymbol{\beta}$ in X We now come to a very important property of analytic continuation: 16.13 Theorem Suppose $\{\gamma_{t}\}$ $|0\leq t\leq1,$ )is a one-parameter family of curves from α to $\boldsymbol{\beta}$ in the plane, $D\!\!\!\!/$ is a disc with center at α,and the function element (f, D) admits analytic continuation along each $\gamma_{t},$ 。 to an element $(g_{t},\,D_{t})$ .Then $g_{1}=g_{0}$ The last equality is to be interpreted as in Theorem 16.11: $$ (g_{1},\,D_{1})\sim(g_{0},\,D_{0}), $$ and $D_{0}$ and $D_{1}$ are discs with the same center, namely ${\boldsymbol{\beta}}.$ PROOF Fix t∈ ${\boldsymbol{\jmath}}.$ There is a chain ${\mathcal{C}}=\{A_{0},\ldots,\,A_{n}\}$ which covers $\gamma_{t},$ with $A_{0}=D,$ such that $(g_{t},\,D_{t})$ is obtained by continuation of $\scriptstyle(y,\,0)$ along G. There are numbers $0=s_{0}<\cdot\cdot<s_{n}=1$ such that $$ E_{i}=\gamma_{i}([s_{i},s_{i+1}])\subset A_{i}\qquad(i=0,1,\ldots,n-1). $$ (1) There exists an $\scriptstyle e\,>0$ which is less than the distance from any of the compact sets continuity of $\varphi$ to the complement of the corresponding open disc $A_{i}.$ The uniform $\delta>0$ $\textstyle E_{i}$ (see Definition 16.12) shows that there exists a on ${\boldsymbol{J}}^{2}$ such that $$ |\gamma_{i}(s)-\gamma_{u}(s)|<\epsilon\qquad\mathrm{if~}s\in I,\,u\in I,\,|u-t|<\delta. $$ (2)326 REAL AND COMPLEX ANALYSIS Suppose u satisfes these conditions. Then (2) shows that G covers and ${\mathfrak{g}}_{u}$ , are obtained by continua- $\gamma_{u}\,,$ and therefore Theorem 16.11 shows that both ${\mathfrak{g}}_{t}$ $I\subset J_{t}$ .Since ${\mathbf I}$ tion of (f, D) along this same chain B. Hence ${\mathbf I}$ is covered by finitely many ${\mathbf{}}J_{t}$ , such that $g_{u}=g_{t}$ for all ue is $g_{t}=g_{u}.$ Thus each $\scriptstyle{:e\,I}$ T is covered by a segment is compact, ${\mathit{J_{t}}}\,;$ and since ${\mathbf I}$ connected, we see in a finite number of steps that $g_{1}=g_{0}$ // Our next item is an intuitively obvious topological fact 16.14 Theorem Suppose ${\Gamma}_{\mathrm{0}}$ and ${\boldsymbol{\Gamma}}_{1}$ are curves in a topological space $X,$ with common initial point α and common end point β. I $\textstyle X$ is simply connected, then there exists a one-parameter family $\{\gamma_{t}\}$ $(0\leq t\leq1)$ of curves from α to $\beta$ in $X,$ such that $\gamma_{0}=\Gamma_{0}$ and $\gamma_{1}=\Gamma_{1}.$ PRoOF Let [O,z] be the parameter interval of ${\Gamma}_{\mathrm{0}}$ and T. Then $$ \Gamma(s)=\left\{\Gamma_{0}(s)\qquad\begin{array}{l l}{{(0\leq s\leq\pi)}}\\ {{\Gamma_{1}(2\pi-s)\qquad(\pi\leq s\leq2\pi)}}\end{array}\right. $$ (1) defines a closed curve in X. Since $\textstyle X$ is simply connected, T is null-homotopic in $X.$ Hence there is a continuous $H\colon[0,\,2\pi]\times[0,\,1]\to X$ such that $$ H(s,0)=\Gamma(s),\qquad H(s,1)=c\in X,\qquad H(0,\,t)=H(2\pi,\,t). $$ (2) If $\Phi\colon{\bar{U}}\to X$ is defined by $$ \Phi(r e^{i\theta})=H(\theta,1-r)\qquad(0\leq r\leq1,\,0\leq\theta\leq2\pi), $$ (2) implies that $\Phi$ is continuous. Put $$ \gamma_{t}(\theta)=\Phi[(1-t)e^{i\theta}+t e^{-i\theta}]\qquad(0\le\theta\le\pi,\,0\le t\le1). $$ Since D(ei)= H(,0) = T(O), it follows that $$ \gamma_{t}(0)=\Phi(1)=\Gamma(0)=\alpha\qquad(0\leq t\leq1), $$ Po(0) l·) $$ \begin{array}{r l}{\gamma_{t}(\pi)=\Phi(-1)=\mathrm{\large~P~}}&{{}\quad(0\leq t\leq1),}\\ {\circ~t\Omega-\mathrm{\large~n}.i\theta1-\mathrm{\boldmath~r~}\circ r\neq\mathrm{\large~-~r~}\prime\Omega\quad}&{{}\quad\alpha<a<\neg)}\end{array} $$ 1 0(0) T and Y,(0) = 0(e**)= 0(e"i7-0)= T(Zm - 0) = F,(0) (0 ≤0≤ T) This completes the proof. // The Monodromy Theorem The preceding considerations have essentially proved the following important theorem.ANALYTIC coNTINUATION 327 element, 16.15 Theorem Suppose $\Omega$ is a simply connected region,(f, ${\boldsymbol{D}})$ is a function $D\in\Omega$ and (f, D) can be analytically continued along every curve in Q for all $\varepsilon\in D$ that starts at the center of ${\boldsymbol{D}}.$ Then there exists $g\in H(\Omega)$ such that $g(z)=f(z)$ $\beta_{1}.$ PROOF Let $\Gamma_{0}$ and ${\Gamma_{1}}$ be two curves in ${\Gamma}_{1}$ lead to the same element , then $(g_{\beta_{1}},\ D_{\beta_{1}})$ $D\!\!\!\!/$ to some to point ${\mathrm{,~}}\beta\in\Omega$ is a disc with center at ${\boldsymbol{\beta}}.$ If $D_{\beta_{1}}$ intersects $D_{\beta}.$ from the center α of where $\Omega$ It follows from Theorems 16.13 and 16.14 that the analytic con- $D_{\beta}$ tinuations of (, D) along ${\Gamma}_{0}$ and to ${\boldsymbol{\beta}}.$ then along the straight line from $\boldsymbol{\beta}$ $(g_{\beta},D_{\beta}),$ can be obtained by frst continuing $\scriptstyle(y,\,D)$ This shows that $g_{\beta_{1}}=g_{\beta}\ln D_{\beta_{1}}\cap D_{\beta}.$ The definition $$ g(z)=g_{\beta}(z)\qquad(z\in D_{\beta}) $$ is therefore consistent and gives the desired holomorphic extension of f. /// $D\!\!\!\!/$ path $D(\beta;r)<\Omega,$ if $f^{\prime}(z)=(z-w)^{-1}$ in be a plane region, fix $w\notin\Omega,$ let $D\!\!\!\!/$ be a disc in Q. Since if $\scriptstyle p_{p}\equiv$ 16.16 Remark Let $\Omega$ Note that is simply connected,there exists $f\in H(D)$ such that exp $[f(z)]=z-w.$ ${\boldsymbol{D}},$ and that the latter function is holomorphic in all of $\scriptstyle\gamma$ in $\Omega$ that starts at the center $\scriptstyle(y,\,b)$ can be analytically continued along every to ${\boldsymbol{\beta}},$ $\Omega$ . This implies that of $D\!\!\!\!/$ ): If $\scriptstyle{\mathcal{Y}}$ goes from $\scriptstyle{\dot{\mathbf{x}}}$ $\scriptstyle{\dot{\alpha}}$ $$ \Gamma_{z}=\gamma\div[\beta,z]\qquad(z\in D_{\beta}) $$ (1) and if $$ g_{\beta}(z)=\int_{{\Gamma}_{z}}(\zeta-w)^{-1}~d\zeta+f(\alpha)\qquad(z\in D_{\beta}), $$ (2) then $(g_{\beta},D_{\rho})$ is the continuation $\scriptstyle1^{1}(J,D)$ along y in $D.$ Then Note tha $g_{\beta}^{\prime}(z)=(z-w)^{-1}$ in $D_{\beta}\,.$ such that $g(z)=f(z)$ $g^{\prime}(z)=(z-w)^{-1}$ Assume now that there exists $g\in H(\Omega)$ it follows that for all z ∈ Q. If $\boldsymbol{\Gamma}$ is a closed path in $\Omega,$ $$ \mathrm{Ind_{F}\left(w\right)=}\frac{1}{2\pi i}\left.\right\}_{\mathcal{C}}^{\prime}(z)\ d z=0. $$ (3) We conclude (with the aid of Theorem 13.11) that the monodromy theorem fails in every plane region that is not simply connected.328 REAL AND coMPLEX ANALYSIs Construction of a Modular Function 16.17 The Modular Group This is the set ${\boldsymbol{G}}$ of all linear fractional transform- ations of the form $$ \varphi(z)={\frac{a z+b}{c z+d}} $$ (1) where a, ${\mathfrak{b}},$ c, and $d$ are integers and $a d-b c=1.$ Since ${\boldsymbol{a}},$ ${\mathfrak{b}},$ c, and $d$ l are real, each $\scriptstyle{\sigma\in G}$ maps the real axis onto itself (except for o). The imaginary part of p(i) is $(c^{2}+d^{2})^{-1}>0.$ Hence $$ \varphi(\Pi^{+})=\Pi^{+}\qquad(\varphi\in G), $$ (2) where ${\boldsymbol{\Pi}}^{+}$ is the open upper half plane. If $\varphi$ is given by (1), then $$ \varphi^{-1}(w)={\frac{d w-b}{-c w+a}} $$ (3) so that $\varphi^{-1}$ e G. Also $\varphi\circ\psi\in G$ if $\sigma\in G$ and $\psi\in G.$ Thus ${\boldsymbol{G}}$ is a group, with composition as group operation. In view of (Q) it is customary to regard $\boldsymbol{\mathit{G}}$ as a group of transformations on $\Pi^{+}$ $(a=d=0$ The transformations $z\to z+1$ $(a=b=d=1,\;\;\;c=0)$ and Z→-1/2 $b=-1,\,c=1)$ belong to ${\cal G}.$ In fact, they generate G (i.e., there is no proper subgroup of G which contains these two transformations). This can be proved by the same method which will be used in Theorem 16.19(c) A modular function is a holomorphic (or meromorphic) function $\boldsymbol{\mathsf{f}}$ 'on $\Pi^{+}$ which is invariant under ${\boldsymbol{G}}$ or at least under some nontrivial subgroup T of ${\cal G}.$ This means that f。 = f for every p e T. 16.18 A Subgroup We shall take for T the group generated by o and t, where $$ \sigma(z)={\frac{z}{2z+1}},\qquad\tau(z)=z+2. $$ (1) One of our objectives is the construction of a certain function A which is invari- ant under $\Gamma$ and which leads to a quick proof of the Picard theorem. Actually, it is the mapping properties of $\lambda$ which are important in this proof, not its invari- ance, and a quicker construction (using just the Riemann mapping theorem and the reflection principle) can be given. But it is instructive to study the action of ${\Gamma}$ on $\Pi^{+}$ in geometric terms, and we shall proceed along this route Let ${\boldsymbol{Q}}$ be the set of all z which satisfy the following four conditions, where $z=x+i y\cdot$ $$ y>0,\qquad-1\leq x<1,\qquad|2z+1|\geq1,\qquad|2z-1|>1. $$ (2) ${\boldsymbol{Q}}$ is bounded by the vertical lines $\frac{1}{2}$ , with centers at $-{\frac{1}{2}}$ and at $\frac{1}{2}$ ${\boldsymbol{Q}}$ contains those of $x=-1$ and $\scriptstyle x\;=\;1$ and is bounded below by two semicircles of radius its boundary points which lie in the left half of $\Pi^{*}$ $\textstyle{\mathcal{Q}}$ contains no point of the real axis.ANALYTIc coNrINUATION 329 We claim that ${\boldsymbol{Q}}$ is a fundamental domain of T. This means that statements (a and (b) of the following theorem are true 16.19 Theorem Let ${\Gamma}$ T and ${\boldsymbol{Q}}$ 2 be as above. (b) (a) If p, and $\scriptstyle\sigma_{\mathrm{p}}\in\Gamma$ an $d\ {\varphi}_{1}\neq{\varphi}_{2},t h e n\ {\varphi}_{1}(Q)\cap\ {\varphi}_{2}(Q)={\mathcal{L}}$ X. 了0 $\varphi(Q)=\Pi^{*}$ of the form (c) T contains all transformations $\scriptstyle\varphi\in G$ $$ \varphi(z)={\frac{a z+b}{c z+d}} $$ (1) for which a and $d$ are odd integers, ${\boldsymbol{b}}$ and $\scriptstyle{\mathcal{C}}$ c are even. if $\left(a^{\prime}\right)$ PROOF Let ${\Gamma_{1}}$ be the set of all $\scriptstyle\phi\;\in\;G$ described in (c). It is easily verified that $\Gamma\in\Gamma_{1}.$ ${\Gamma_{1}}.$ For ${\Gamma_{1}}$ is a subgroup of G. Since $\sigma\in\Gamma_{i}$ and $\tau\in\Gamma_{1},$ it follows that by To show that $\Gamma=\Gamma_{1},$ i.e., to prove $(c),$ it is enough to prove that (a) and ((b) ${\boldsymbol{\Gamma}}$ hold, where (a') is the statement obtained from (a) by replacing We shall need the relation and (b) hold, it is clear that T cannot be a proper subset of ${\Gamma_{1}}$ $$ \mathrm{Im~}\varphi(z)={\frac{\mathrm{Im~}z}{|c z+d|^{2}}} $$ (2) which is valid for every ${\mathfrak{o}}\in G$ given by (1). The proof of (2) is a matter of straightforward computation, and depends on the relation $a d-b c=1.$ $\varphi_{1}^{-1}\circ\varphi_{2}$ If We now prove (a) Suppose $\varphi_{1}$ and $\varphi_{2}\in\Gamma_{1},\;\varphi_{1}\neq\varphi_{2},$ and define $\varphi=$ enough to show that $z\in\varphi_{1}(Q)\cap\varphi_{2}(Q),$ then $\varphi_{1}^{-1}(z)\in{\cal Q}\cap\varphi({\cal Q}).$ It is therefore $$ Q\,\cap\,\varphi(Q)=\mathcal{D} $$ (3) if $\varphi\in\Gamma_{1}$ and $\varphi$ is not the identity transformation The proof of (3) splits into three cases If c= 0 in(1),then $\scriptstyle a/=1.$ and since $\bar{a}$ and $d$ are integers,we have of O ${\boldsymbol{Q}}$ $a=d=\pm1.$ Hence $\varphi(z)=z+2n$ for some integer $n\neq0,$ , and the description makes it evident that (3) holds If c = 2d,then $c=\pm2$ and $d=\pm1$ (since $a d-b c=1)$ Therefore If & for these w, cw $\ \!+\!a$ 2m, where m is an integer. Since ${\boldsymbol{Q}}.$ The description of ${\boldsymbol{Q}}$ shows that i $\textstyle x\neq-{\frac{1}{2}}$ points -1, 0, 1 lies in $D(\alpha;\,r).$ Hence |cv $\forall+d|<1$ $\sigma(Q)\subset{\tilde{D}}(_{2}^{1};$ ),(3) holds. Otherwise, $\varphi(z)=\sigma(z)+$ and $c\neq2d,$ we claim that $|c z+d|>1$ for all z e ${\cal Q}.$ or 1. But the disc $\scriptstyle{s\neq0}$ would intersect ${\tilde{D}}(\alpha;\,r)$ intersects ${\cal Q},$ then at least one of the $\mathbf{0}$ ${\bar{D}}(-d/c;\ 1/|c|)$ for $w=-1$ or is a real number and if is an odd integer whose absolute value cannot be less than 1. So $|c z+d|>1,$ and it now follows from (2) that Im o(z) < Im z for330 REAL AND cOMPLEX ANALYS1S every z ∈Q. If it were true for some z $\in{\mathcal{Q}}$ that $\varphi(z)\in G,$ the same argument would apply to $\varphi^{-1}$ and would show that $$ \mathrm{Im~}z=\mathrm{Im~}\varphi^{-1}(\varphi(z))\leq\mathrm{Im~}\varphi(z). $$ (4) This contradiction shows that (3) holds. Hence (a') is proved $\Sigma\subset\Pi^{*}$ To prove $(b),$ let $\Sigma$ be the union of the sets $\scriptstyle{\varphi(0)}$ for $\varphi\in\mathbb{T}\cdot\mathbb{H}$ is clear that Also, $\Sigma_{}^{}$ Since o maps contains the sets r"(Q),for $n=0,~\pm1,~\pm2,\ldots,$ where $\tau^{n}(z)=z+2n$ the circle $|2z+1|=1$ onto the circle ities $|2z-1|=1,$ we see that $\Sigma_{}^{}$ contains every $z\in\Pi^{+}$ which satisfies all inequal- $$ |2z-(2m+1)|\geq1\qquad(m=0,\pm1,\pm2,\ldots). $$ (5) Fix $w\in\Pi^{*}$ Since Im $w>0,$ there are only finitely many pairs of inte- gers C $\scriptstyle{\mathcal{C}}$ and $d$ such that |cw + d| lies below any given bound, and we can choose $\varphi_{0}\in\Gamma$ so that lcw + dlis minimized. By (2), this means that $$ \mathrm{Im}\ \varphi(w)\leq\mathrm{Im}\ \varphi_{0}(w)\qquad(\varphi\in\Gamma). $$ (6) Put $z=\varphi_{0}(w).$ Then (6) becomes $$ \mathrm{Im~}\varphi(z)\leq\mathrm{Im~}z\qquad(\varphi\in\Gamma). $$ (7) Apply (7) to $\varphi=\sigma\tau^{-n}$ and to $\varphi=\sigma^{-1}\tau^{-n}.$ Since $$ (\sigma\tau^{-n})(z)=\frac{z-2n}{2z-4n+1},\qquad(\sigma^{-1}\tau^{-n})(z)=\frac{z-2n}{-2z+4n+1}, $$ (8) it follows from (2) and (T) that $$ |2z-4n+1|\geq1,\qquad|2z-4n-1|\geq1\qquad(n=0,\pm1,\pm2,\ldots). $$ (9) Thus z satisfies (), hence z ∈ E; and since $w=\varphi_{0}^{-1}(z)$ and $\varphi_{0}^{-1}\in\Gamma,$ we have w ∈ $\Sigma_{\circ}$ This completes the proof. // The following theorem summarizes some of the properties of the modular function Awhich was mentioned in Sec. 16.18 and which will be used in Theorem 16.22. 16.20 Theorem If T and Q are as described in Sec. 16.18, there exists a func tion Ae H(TI') such that (a)入。p = . for every gp ∈T. (b)入is one-to-one on ${\cal Q}.$ (c) The range $\Omega$ of入[which is the same as A(Q), by (a)], is the region consist ing of all complex numbers different from O and 1 (d)入 has the real axis as its natural boundaryANALYTIC coNTINUATION 331 PROOF Let $Q_{0}$ be the right half of ${\cal Q}.$ More precisely, $Q_{0}$ consists of all z ∈ $\Pi^{+}$ such that $$ \begin{array}{r l}{0<\mathrm{{Re}}:z<1,}&{{}\quad|2z-1|>1.}\end{array} $$ (1) By Theorem 14.19 (and Remarks 14.20) there is a continuous function $\boldsymbol{h}$ on ${\tilde{Q}}_{0}$ which is one-to-one on and $h(\infty)=\infty.$ The reflection principle (Theorem 11.14) $h(Q_{o})=\Pi^{*}$ ${\tilde{Q}}_{0}$ and holomorphic in $Q_{0}.$ such that $h(0)=0,\ h(1)=1,$ shows that the formula $$ h(-x+i y)={\overline{{h(x+i y)}}} $$ (2) extends ${\boldsymbol{h}}$ to a continuous function on the closure $\bar{Q}$ of ${\boldsymbol{Q}}$ which is a confor mal mapping of the interior of ${\boldsymbol{Q}}$ onto the complex plane minus the non negative real axis. We also see that ${\boldsymbol{h}}$ h is one-to-one on ${\cal Q},$ that $\scriptstyle a(0)$ is the region S $\Omega$ described in (c), that $$ h(-1+i y)=h(1+i y)=h(\tau(-1+i y))\qquad(0<y<\varpi), $$ (3 and that $$ h(-\textstyle{\frac{1}{2}}+\textstyle{\frac{1}{2}}e^{i\theta})=h(\textstyle{\frac{1}{2}}+\textstyle{\frac{1}{2}}e^{i\alpha-\theta})=h(\textstyle{\sigma}(-\textstyle{\frac{1}{2}}+{\textstyle{\frac{1}{2}}}e^{i\theta}))\qquad(0<\theta<\pi). $$ (4) Since $\boldsymbol{h}$ is real on the bogndary of ${\cal Q}.$ ,(3) and(4) follow from(2) and the definitions of o and t We now define the function 入: $$ \lambda(z)=h(\varphi^{-1}(z))\qquad(z\in\varphi(Q),\,\varphi\in\Gamma). $$ (5) By Theorem 16.19,each z e $\Pi^{+}$ lies in $\scriptstyle{\rho(0)}$ for one and only one g e T Thus (5) defines $\lambda(z) \{\mathbf{or}\,z\in\Pi^{+},$ and we see immediately that $\lambda$ has properties (a) to (c) and that $\lambda$ is holomorphic in the interior of each of the sets $\scriptstyle{\rho(0)}$ It follows from (3) and (4) that $\lambda$ is continuous on $$ Q\cup\tau^{-1}(Q)\cup\sigma^{-1}(Q), $$ hence on an open set ${\mathbf{}}V$ which contains ${\cal Q}.$ . Theorem 16.8 now shows that $\lambda$ is holomorphic in V. Since $\Pi^{+}$ is covered by the union of the sets p(V), $\varphi\in\Gamma,$ and since $\lambda\circ\ \varphi=\lambda,$ we conclude that $\lambda\in H(\Pi^{*})$ is dense on the real axis. If A , the Finally, the set of all numbers $\varphi(0)=b/d$ could be analytically continued to a region which properly contains $\Pi^{+}\!,$ zeros of A would have a limit point in this region, which is impossible since 入 is not constant. // The Picard Theorem 16.21 The so-called “little Picard theorem”asserts that every nonconstant entire function attains each value, with one possible exception. This is the theorem which is proved below. There is a stronger version: Every entire function which is $f(z)=e^{z},$ which omits the not a polynomial atins each value infitely many times, again with one possb exception. That one exception can occur is shown by332 REAL AND coMPLEX ANALYs value O.The latter theorem is actually true in a local situation: If f has an isolated $\mathbb{Z}_{0}$ singularity at a point $\mathbb{Z}_{0}$ Zo and if f omits two values in some neighborhood of $z_{0:}$ fo,then is is a removable singularity or a pole of f. This so-called “big Picard theorem $\mathfrak{y}$ a remarkable strengthening of the theorem of Weierstrass (Theorem 10.21) which merely asserts that the image of every neighborhood of $\mathbb{Z}_{0}\,.$ We shall not prove it here. $\mathbb{Z}_{0}$ is dense in the plane if f has an essential singularity at 16.22 Theorem Iff is an entire function and if there are two distinct complex numbers α and $\boldsymbol{\beta}$ which are not in the range of f, then fis constant. replace $\boldsymbol{\f}$ PROOF Without loss of generality we assume that $\scriptstyle x\;=\;0$ and $\beta=1;$ if not, $\Omega$ by $(f-\alpha)/(\beta-\alpha).$ Then $\boldsymbol{\mathsf{f}}$ maps the plane into the region described in Theorem 16.20. one on $V_{\mathrm{i}}$ With each disc and $\lambda(V_{\mathrm{1}})=D_{\mathrm{1}};$ each such $V_{\mathrm{i}}$ intersects at most two of the $\lambda$ is one-to- there are infinitely many such $V_{1},$ there is associated a region $V_{1}\in\Pi^{*}$ (in fact, $D_{1}\subset\Omega$ one for each $\varphi\in\Gamma$ such that domains p(Q). Corresponding to each choice of $V_{\mathrm{i}}$ there is a function ${\boldsymbol{\psi}}_{1}$ ∈ $\scriptstyle{m(s_{1})}$ such that $\psi_{1}(\lambda(z))=z$ for all $z\in V_{1}.$ we can choose a corre- If $D_{2}$ is another disc in $\Omega$ and if $D_{1}\cap D_{2}\neq\varnothing.$ sponding ${\mathit{V}}_{2}$ so that $V_{1}\cap V_{2}\neq\varnothing.$ The function elements $(\psi_{1},D_{1})$ and $(\psi_{2},\,D_{2})$ will then be direct analytic continuations of each other. Note that $\psi_{i}(D_{j})\subset\Pi^{*}$ is in $\Omega,$ there is a disc $A_{\mathrm{0}}$ with center at O so that Since the range of $\boldsymbol{\mathit{f}}$ for $A_{0},\ldots,A_{n},$ so that each $\scriptstyle I(A_{3})$ lies in a disc $D_{i}$ in as above, put $g(z)=\psi_{0}(f(z))$ for $\scriptstyle{f(A_{a})}$ lies in a disc $D_{0}$ in Q. Choose $\psi_{0}\in H(D_{0}),$ $z\in A_{0}\,,$ f。y is a compact subset of $\Omega.$ Hence $\scriptstyle\gamma$ , be any curve in the plane which starts at O. The range of and let $\scriptstyle{\mathcal{Y}}$ can be covered by a chain of discs, $\Omega,$ and we can choose $\psi_{i}\in H(D_{i})$ so that $(\psi_{i},\,D_{i})$ is a direct analytic continuation o $\mathbf{f}\left(\psi_{i-1},\,D_{i-1}\right),$ $i=1,\ \ldots n.$ This gives an analytic continuation of the function element (g, $A_{0})$ along the chain $\{A_{0},\ldots,A_{n}\};$ ; note that yn。f has positive imaginary part. Since (g, $A_{0})$ can be analytically continued along every curve in the plane and since the plane is simply connected, the monodromy theorem implies that $\scriptstyle{\mathcal{G}}$ extends to an entire function. Also,the range of $\scriptstyle{\mathcal{G}}$ is in $\Pi^{+}\!,$ , hence that $\scriptstyle{\mathcal{G}}$ is constant, and since $\psi_{0}$ is bounded, hence constant, by Liouville's theorem. This shows was a non- // $(g-i)/(g+i)$ empty open set, we conclude that fis constant. was one-to-one on f(Ao) and $A_{0}$ Exercises 1 Suppose $f(z)=\Sigma a_{n}z^{i}$ $a_{n}\geq0,$ and the radius of convergence of the series is 1. Prove that $\boldsymbol{\mathit{f}}$ has a would converge at some $x>1.$ Hint: Expand fin powers of $\scriptstyle{t-1}$ .Ifl were a regular point of f, the new series singularity at $z=1.$ What would this imply about the orignal series ? in ${\boldsymbol{D}}.$ Suppos $(f,D)$ and (g. ${\boldsymbol{D}})$ are function elements, P ${\mathbf{}}P$ Pis a polynomial in two variables, and $f_{1}$ and ${\mathit{g_{1}}}.$ Prove that Suppose $\scriptstyle{\mathcal{f}}$ and $\scriptstyle{\mathcal{G}}$ can be analytically continued along a curve $\gamma_{\mathrm{{J}}}$ $P(f,$ ${\mathfrak{g}})=0$ 。toANALYTic coNTINUATION 333 $P(f_{1},\,g_{1})=0$ . Extend this to more than two functions. Is there such a theorem for some class of functions ${\mathbf{}}P$ which is larger than the polynomials? 3 Suppose $\underline{{\Omega}}$ is a simply connected region, and uis a real harmonic function in Q. Prove that there exists an $f\in H(\Omega)$ such that $u=\mathbf{Re}f.$ Show that this fails in every region which is not simply con- nected. 4 Suppose $\scriptstyle{X}$ is the closed unit square in the plane, fis a continuous complex function on $\scriptstyle{\mathcal{G}}$ g on $\scriptstyle{\mathcal{X}}$ such that $f=e^{\theta}$ . For what class of no zero in $X.$ Prove that there is a continuous function $X,$ and f has spaces $\scriptstyle{\mathcal{X}}$ (other than the above square) is this also true? S Prove that the transformations $z\to z+1$ and $z\to-1/z$ generate the full modular group G. Let ${\boldsymbol{R}}$ $x\leq0.$ consist of all $z=x+i y$ such that $|x|<{\frac{1}{2}},$ $y>0,$ and $|z|>1,$ plus those limit points which have Prove that ${\boldsymbol{R}}$ is a fundamental domain of $G.$ 6 Prove that $\scriptstyle{G}$ is also generated by the transformations $\textstyle{\varphi}$ and $\psi,$ , where $$ \varphi(z)=-{\frac{1}{z}},~~~~~\psi(z)={\frac{z-1}{z}}. $$ Show that qp has period $2,\psi$ has period 3 7 Find the relation between composition of linear fractional transformations and matrix multiplica tion. Try to use this to construct an algebraic proof of Theorem 16.19(c) or of the first part of Exercise S5 8 Let $\bar{E}$ be a compact set on the real axis, of positive Lebesgue measure, let $\underline{{\Omega}}$ be the complement of ${\boldsymbol{E}},$ relative to the plane, and define $$ f(z)= \{\frac{d t}{t-z}\quad\quad(z\in\Omega). $$ Answer the following questions: (a) Isf constant? (b) Can f be extended to an entire function? (c) Does lim zf(z) exist as z→00? If so, what is it? (d) Does f have a holomorphic square root in $\Omega{\dot{r}}$ (e) Is the real part of f bounded in 92? $\mathbf{(f)}$ Is the imaginary part of f bounded in Q? [I“" yes"in (e) or (C). give a bound.] (g) What is ${\mathfrak{f}}_{\gamma}$ f(z) dz if y is a positively oriented circle which has $\underline{{\Omega}}$ p which is not constant? (h) Does there exist a bounded holomorphic function $\bar{E}$ E in its interior? $\mathcal{\boldsymbol{\varphi}}$ rp in S2 9 Check your answers in Exercise 8 against the special case $$ E=[-1,1]. $$ 10 Call a compact set $\boldsymbol{E}$ in the plane removable if there are no nonconstant bounded holomorphic functions in the complement of ${\boldsymbol{E}}.$ (a) Prove that every countable compact set is removable. (b) If $\bar{E}$ is a compact subset of the real axis, and $m(E)=0,$ prove that $\bar{E}$ is removable. Hint: $\bar{E}$ can be surrounded by curves of arbitrarilysmall total length. Apply Cauchy's formula, as in Exercise 25, Chap 10 (c) Suppose $\boldsymbol{E}$ is removable,S $\underline{{\Omega}}$ is a region, $E\subset\Omega,f\in H(\Omega-E),{\mathrm{~and~}}f{\mathrm{~is}}$ bounded. Prove that f can be extended to a holomorphic function in $\Omega.$ (d) Formulate and prove an analogue of part (b) for sets $\boldsymbol{E}$ which are not necessarily on the real axis (e) Prove that no compact conncted subset of the plane (with more than one poin is remov- able.334 REAL AND COMPLEX ANALYSIS 11 For each positive number $\alpha,$ let $\Gamma_{\alpha}$ 。be the path with parameter interval (-0, o) defined by $$ \Gamma_{\alpha}(t)=\left\{\begin{array}{c c}{{-t-\pi i\ }}&{{(-\,\infty\,<t\leq-\alpha)}}\\ {{\alpha+\displaystyle\frac{\pi i t}{\alpha}}}&{{(-\,\alpha\leq t\leq\alpha),}}\\ {{\ \ \ \ \ (\alpha\leq t<\infty).}}\end{array}\right. $$ Let E $\Omega_{\alpha}$ be the component of the complement of $\textstyle\Gamma_{a}^{*}$ t which contains the origin, and define $$ f_{\alpha}(z)={\frac{1}{2\pi i}} [\frac{\exp{(e^{\Phi})}}{{w}-z}\,d{w}\qquad(z\in\Omega_{\alpha}). $$ Prove whose restriction to $\Omega_{x}$ is an analytic continuation $\operatorname{off}_{s}$ if α $<\beta.$ Prove that therefore there is an entire function J $\operatorname{that}f_{\beta}$ is . Prove tha $$ \operatorname*{lim}f(r e^{i\theta})=0 $$ for every $e^{i\theta}\neq1.$ (Here r is positive and $\theta\nonumber$ is real, as usual.) Prove that fis not constant. [Hint: Look at fGr)] 1 $$ g=f\exp\,(-f), $$ prove that $$ \operatorname*{lim}_{r arrow c}\varrho(r^{i\delta})=0 $$ for ever $e^{i\theta},$ such that Show that there exists an entire function ${\boldsymbol{h}}$ $$ \operatorname*{lim}_{n arrow\infty}h(n z)= \{{1\atop0\atop\mathrm{if~z=0,}}\quad\mathrm{~if~}z=0, $$ 12 Suppose $$ f(z)=\sum_{k=1}^{\infty}\left({\frac{z-z^{2}}{2}}\right)^{3}=\sum_{n=1}^{\infty}a_{n}z^{n}. $$ Find the regions in which the two series converge. Show that this illustrates Theorem 16.5. Find the singular point of f which is nearest to the origin. $13$ Let $\Omega=\{z:{\frac{1}{2}}<|z|<2\}.$ For $n=1,$ 2, $3,\,\ldots$ let $X_{n}$ be the set of all fe H(Q) that are nth deriv with domain H(2).] atives of some $g\in H(\Omega).$ [In other words, $X_{n}$ is the range of the differential operator ${\boldsymbol{D}}^{n}$ (a) Show that f∈ $X_{1}$ if and only if, f(z) $d z=0,$ where yis the positively oriented unit circle $D(0;\cdot$ 2) (b)Show that f e $X_{n}$ for every nif and ony if extends to a holomorphic function in 14 Suppose $\Omega_{\mathbf{\Lambda}}$ is a region, pe QL $\,\kappa<\alpha$ Let $\textstyle{\mathcal{F}}$ F be the class of all f∈ H(Q) such that $|f(y)|\leq R$ and f(Q) contains neither O nor 1. Prove that ${\mathcal{Q}}$ is a normal family. 1s5 Show that Theorem 16.2 leads to a very simple proof of the special case of the monodromy theorem (16.15) in which $\underline{{\Omega}}$ Q and $D\!\!\!\!/$ are concentric discs. Combine this special case with the Riemann mapping theorem to prove Theorem 16.15 in the generality in which it is stated