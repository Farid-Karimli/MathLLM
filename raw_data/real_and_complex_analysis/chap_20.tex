CHAPTER TWENTY UNIFORM APROXIMATION BY POLYNOMIALS Introduction 20.1 Let $K^{0}$ be the interior of a compact set $\textstyle K$ in the complex plane.(By defini tion, $K^{0}$ is the union of all open discs which are subsets of $K{\mathrm{:}}$ ; of course, $K^{0}$ may be empty even if $ .K$ is not.) Let $P(K)$ denote the set of all functions on ${\cal K}\,\,\,\,\,\,\,\,\,\,{\cal K}$ which are uniform limits of polynomials in z. Which functions belong to $P(K){\dot{Y}}$ Two necessary conditions come to mind immediately:If fe P(K),then $f\in C(K)$ and $f\in H(K^{0}).$ The question arises whether these necessary conditions are also sufficient The answer is negative whenever ${\boldsymbol{K}}$ separates the plane (i.e., when the com plement of ${\cal K}\,\,\,\,$ is not connected). We saw this in Sec. 13.8. On the other hand, if ${\cal K}\,\,\,\,$ is an interval on the real axis Gin which case $K^{0}={\emptyset}),$ the Weierstrass approx imation theorem asserts that $$ P(K)=C(K). $$ So the answer is positive if $\displaystyle K$ is an interval. Runge's theorem also points in this direction, since it states, for compact sets K $\textstyle K$ which do not separate the plane, that $P(K)$ contains at least all those f∈ C(K) which have holomorphic extensions to some open set $\scriptstyle\mathbf{a}\gg\kappa$ In this chapter we shall prove the theorem of Mergelyan which states without any superfluous hypotheses, that the above-mentioned necessary condi- tions are also sufficient if ${\cal K}\,\,\,\,$ does not separate the plane The principal ingredients of the proof are: Tietze's extension theorem,a smoothing process invoving convolutions,Runge's theorem, and Lemma 20.2. whose proof depends on properties of the class ${\mathcal{P}}$ which was introduced in Chap 14. 386UNIFORM APPROXIMATION BY POLYNOMIALS 387 Some Lemmas 20.2 Lemma Suppose $D\!\!\!\!/$ is an open disc of radius r > 0, $\scriptstyle{E\cdot D}$ ${\boldsymbol{E}}$ is compact and connected, $\scriptstyle\mathbf{a}={\mathcal{S}}^{2}-E$ is connected, and the diameter of $\boldsymbol{E}$ is at least ${\boldsymbol{r}}.$ Then there is a function $g\in H(\Omega)$ and a constant b, with the following property: ${\cal I}{\cal f}$ $$ Q(\zeta,z)=g(z)+(\zeta-b)g^{2}(z), $$ (1) the inequalities $$ \begin{array}{c}{{|Q(\zeta,z)|<{\frac{100}{r}}}}\\ {{\Big|Q(\zeta,z)-{\frac{1}{z-\zeta}}\Big|<{\frac{1,000r^{2}}{1z-\zeta!}}}}\end{array} $$ (2) (3) hold for all $\varepsilon\in\Omega$ and for $a l l\,\zeta\in D,$ We recall that ${\boldsymbol{S}}^{2}$ is the Riemann sphere and that the diameter of $\boldsymbol{E}$ is the supremum of the numbers $|z_{1}-z_{2}|.$ where $z_{1}\in E{\mathrm{~and~}}z_{2}\in E.$ PROOF We assume, without loss of generality, that the center of $D\!\!\!\!/$ is at the origin. So $D=D(0;r)$ The implication $(d)\to(b)$ of Theorem 13.11 shows that such that $F(0)=\infty$ ${\mathbf{}}F$ has an fore a conformal mapping ${\mathbf{}}F$ of $U$ onto $\Omega$ $\Omega$ is simply con- nected. (Note that $\infty\in\Omega\}$ By the Riemann mapping theorem there is there- expansion of the form $$ F(w)={\frac{a}{w}}+\sum_{n=0}^{\infty}c_{n}w^{n}\qquad(w\in U). $$ (4) We define $$ g(z)=\frac{1}{a}\,F^{-1}(z)\qquad(z\in\Omega), $$ (5) where ${\boldsymbol{F}}^{\prime\prime}$ is the mapping of $\Omega$ onto $U$ which inverts ${\boldsymbol{F}},$ and we put $$ b={\frac{1}{2\pi i}}\ {\biggr\}}_{\Gamma}^{\circ}g(z)\;d z, $$ (6) where T is the positively oriented circle with center O and radius r diam of the complement of By (4), Theorem 14.15 can be applied to $F/a.$ It asserts that the diameter Since $(F/a)(U)$ is at most 4. Therefore diam $E\leq4|a|.$ $m\geq r.$ it follows that $$ |a|\geq_{4}^{r}. $$ (7)388 REAL AND coMPLEX ANALYSIS Since $\scriptstyle{\mathcal{G}}$ is a conformal mapping of Q onto $\mathrm{po};$ 1/|al),(7) shows that $$ |g(z)|<{\frac{4}{r}}\qquad(z\in\Omega) $$ (8) and since ${\Gamma}$ T is a path in $\Omega,$ of length 2mr, (6) gives $$ |b|<4r. $$ (9) If e D, then $|\zeta|<r,$ so (1),(8), and (9) imply $$ |Q|\leq{\frac{4}{r}}+5r\biggl({\frac{16}{r^{2}}}\biggr)<{\frac{100}{r}}. $$ (10) f This proves (Q2) , then $z g(z)=w F(w)/a;$ and since $w F(w)\to a$ as $w\to0,$ we have Fix<e D $z=F(w)$ zg(z)→1 as $z\to c o.$ Hence $\scriptstyle{\mathcal{G}}$ g has an expansion of the form $$ g(z)=\frac{1}{z-\zeta}+\frac{\lambda_{2}(\zeta)}{(z-\zeta)^{2}}+\frac{\lambda_{3}(\zeta)}{(z-\zeta)^{3}}+\cdot\cdot\cdot\;\;\;\;\;\;(|z-\zeta|>2r). $$ (11) Let ${\Gamma}_{\mathrm{0}}$ be a large circle with center at $0;(11)$ gives (by Cauchy's theorem) that $$ \lambda_{2}(\zeta)=\frac1{2\pi i}\left(\right)_{\Gamma_{0}}(z-\zeta)g(z)\;d z=b-\zeta. $$ (12) Substitute this value of 入,(G) into (11). Then (1) shows that the function $$ \varphi(z)=\left[Q(\zeta,z)-{\frac{1}{z-\zeta}}\right]\!(z-\zeta)^{3} $$ (13) is bounded as z→0. Hence $\varphi$ has a removable singularity at o.If $z\in\Omega\frown D.$ then $|z-\zeta|<2r,$ so (2) and (13) give $$ |\,\varphi(z)\,|\,<8r^{3}\,|\,Q(\zeta,\,z)\,|\,+\,4r^{2}\,<\,1,000r^{2}. $$ (14) By the maximum modulus theorem,(14) holds for all z e Q. This proves (3) /// 20.3 Lemma Suppose $f\in C_{c}^{\prime}(R^{2}),$ the space of all continuously differentiable functions in the plane, with compact support. Put $$ {\bar{\partial}}={\frac{1}{2}}\left({\frac{\partial}{\partial x}}+i{\frac{\partial}{\partial y}}\right). $$ (1) Then the following“Cauchy formula”holds: $$ f(z)=\,-\,{\frac{1}{\pi}}\left|\right|\!\!\int_{R^{2}}{\frac{(\bar{\partial}f)(\zeta)}{\zeta-z}}\,d\xi\,\,d\eta\qquad(\zeta=\xi+i\eta). $$ (2)UNIFORM APPROXIMATION BY POLYNOMIALS 389 PRoOF This may be deduced from Green's theorem. However, here is a simple direct proof: Put p(r, 0) = f(z + re"), r > 0,8 real. I $[\zeta=z+r e^{i\theta},$ the chain rule gives $$ (\bar{\partial}f)(\zeta)={\frac{1}{2}}\,e^{i\theta}\left[{\frac{\partial}{\bar{\partial}r}}+{\frac{i}{r}}\,{\frac{\partial}{\bar{\partial}\theta}}\right]\varphi(r,\theta). $$ (3) The right side of (2) is therefore equal to the limit, as $\scriptstyle\epsilon\to0.$ of $$ -\,\frac{1}{2\pi}\,\biggr)_{\epsilon}^{\;\infty}\,\displaystyle{\int_{0}^{2\pi}\left(\frac{\partial\varphi}{\partial r}+\frac{i}{r}\,\frac{\partial\varphi}{\partial\theta}\right)}\,d\theta\,d r. $$ (4) For each $\ r>0,$ $\varphi$ is periodic in (, with period 2z. The integral of Op/00 is therefore O, and (4) becomes $$ -{\frac{1}{2\pi}}\left[\right]_{0}^{2\pi}d\theta\,\left[\right]_{\epsilon}^{\left(\infty\right)}{\frac{\hat{\omega}\varphi}{\hat{\partial}r}}\,d r={\frac{1}{2\pi}}\int_{0}^{2\pi}\!\varphi(\epsilon,\,\theta)\,d\theta. $$ (5) Ase→0,p(e,0)→f(z) uniformly. This gives (2) // We shall establish Tietze's extension theorem in the same setting in which we proved Urysohn's lemma, since it is a fairly direct consequence of that lemma. 20.4 Tietze's Extension Theorem Suppose ${\cal K}\,\,$ is a compact subset of a locally compact Hausdorff space $X,$ and fe C(K). Then there exists an $F\in C_{c}(X)$ such that F(x) =f(x) for all $x\in K.$ (As in Lusin's theorem, we can also arrange it so that $\|F\|_{X}=\|f\|_{K}.$ PROOF Assume $\boldsymbol{\f}$ C is real, -1≤f≤1. Let $\mathbf{}W$ be an open set with compact closure so that $K\subset W.$ Put $$ K^{*}=\{x\in K:f(x)\geq{}\}\},\ \ \ \ \ K^{-}=\{x\in K:f(x)\leq-{}\}\} $$ (1) Then $K^{+}$ and $K^{-}$ are disjoint compact subsets of $\textstyle W.$ As a consequence of $\textstyle W.$ Urysohn's lemma there is a function $f_{1}\in C_{c}(X)$ such that $f_{1}(x)={\frac{1}{3}}$ on $K^{+}\!\!$ $f_{1}(x)=-{\frac{1}{3}}$ on $K^{-},\,-{\textstyle\frac{1}{3}}\leq f_{1}(x)\leq{\textstyle\frac{1}{3}}$ for al $x\in X_{\circ}$ and the support ${\mathfrak{o l}}/{\mathfrak{i}}$ lies in Thus $$ |f-f_{1}|\leq_{3}\mathrm{on}\ K,\qquad|f_{1}|\leq_{3}\mathrm{on}\ X. $$ (2) Repeat this construction with $f-f_{1}$ in place of $f\colon$ :: There exists an $\ f_{2}$ ∈ $C_{c}(X),$ with support in $W,$ so that $$ |f-f_{1}-f_{2}|\leq(\O_{3})^{2}{\mathrm{~on~}}K,\qquad|f_{2}|\leq\underline{{{3}}}\cdot{\frac{2}{3}}{\mathrm{~on~}}X. $$ (3) In this way we obtain functions f,∈ C,(X), with supports in W, such that $$ |f-f_{1}-\cdot\cdot\cdot-f_{n}|\leq(\O_{3}^{2})^{n}\ \mathrm{on}\ K,\qquad|f_{n}|\leq\textstyle\frac{1}{3}\cdot(\textstyle{\frac{2}{3}})^{n-1}\ \mathrm{ol}\quad $$ n X. (4)390 REAL AND coMPLEX ANALYSIS Put $F=f_{1}+f_{2}+f_{3}+\cdots.$ $X.$ Hence ${\mathbf{}}F$ is continuous. Also, the support of on $K,$ and it converges uniformly on By (4), the series converges to $\boldsymbol{\f}$ lies ${\mathbf{}}F$ in ${\overline{{W}}}.$ // Mergelyan's Theorem 20.5 Theorem If $\textstyle K$ is a compact set in the plane whose complement is con- which is holomorphic in the nected, if is a contimuous complex function on $\textstyle K$ interior of K,andi $\scriptstyle\epsilon\;>_{0},$ then there exists a polynomial ${\mathbf{}}P$ such tha |f(z)- P(z)|<e for all z e K. If the interior of $\textstyle K$ is empty, then part of the hypothesis is vacuously satis fied, and the conclusion holds for every fe C(K). Note that ${\cal K}\,\,$ need not be con nected. PROOF By Tietze's theorem, $\boldsymbol{\f}$ can be extended to a continuous function in the plane, with compact support. We fix one such extension, and denote it again byJ let $\scriptstyle{\omega(0)}$ be the supremum of the numbers For any $\delta>{\mathfrak{o}}.$ $$ |f(z_{2})-f(z_{1})| $$ where $z_{1}$ and $z_{2}$ are subject to the condition $|z_{2}-z_{1}|\leq\delta$ Since $\boldsymbol{\mathsf{f}}$ is uni- formly continuous, we have $$ \operatorname*{lim}_{\delta\lnot\Phi}\omega(\delta)=0. $$ (1) From now on, $\delta$ will be fixed. We shall prove that there is a polynomial ${\mathbf{}}P$ such that $$ |f(z)-P(z)|<10.000\omega(\delta)\qquad(z\in K). $$ (2) By (1), this proves the theorem for all z Our first objective is the construction of a function $\Phi\in C_{c}^{\prime}(R^{2}),$ such that l f(z)- 0(z)| ≤ 0(6), (3) $$ |(\hat{\sigma}\Phi)(z)|<\frac{2\omega(\delta)}{\delta}, $$ (4) and $$ \Phi(z)=\,-\,\frac{1}{\pi}\left[\right]\! [\frac{(\bar{d}\Phi)(\zeta)}{\zeta-z}\,d\zeta\,\,d\eta\qquad(\zeta=\zeta+i\eta), $$ (5)UNIFORM APPROxIMATION BY POLYNOMIALS 391 where $X$ is the set of all points in the support of p whose distance from the complement of $\textstyle K$ does not exceed 6.(Thus $X$ contains no point which is“ far within”K.) We construct $\Phi$ as the convolution of f with a smoothing function A. Put $a(r)=0$ if $\ r>\delta_{*}$ put $$ a(r)=\frac{3}{\pi\delta^{2}}\left(1-\frac{r^{2}}{\delta^{2}}\right)^{2}\,\qquad(0\leq r\leq\delta), $$ (6) and define $$ A(z)=a(|z|) $$ (7) for all complex z. It is clear that $A\in C_{c}^{\prime}(R^{2})$ We claim that $$ \begin{array}{c}{{\left.\displaystyle\int_{R^{2}}\right|A=1,}}\\ {{\left\{\displaystyle\int_{R^{2}}^{} \{\displaystyle\frac{1}{\delta A}=0,}}\\ {{ \{\displaystyle\frac{1}{\delta A}\right\}=\frac{24}{15\delta}<;}}\end{array} $$ 天 (10) (8) (9) The constants are so adjusted in (6) that (8) holds. (Compute the integral in polar coordinates.)(9) holds simply because A has compact support. To compute (10), express ${\bar{\partial}}A$ in polar coordinates, as in the proof of Lemma 20.3, and note that $\partial A/\partial\theta=0,\left|\partial A/\partial r\right|=-a^{\prime}(r).$ Now define $$ \Phi(z)=\prod_{R^{2}}^{\cdot}f(z-\zeta)A(\zeta)\;d\zeta\;d\eta=\prod_{R^{2}}^{\cdot}A(z-\zeta)f(\zeta)\;d\zeta\;d\eta. $$ (11 Since f and A have compact support, so does O. Since $$ \Phi(z)-f(z)=\prod_{R^{2}}^{r}\left[f(z-\zeta)-f(z)\right]A(\zeta)\;d\xi\;d\eta $$ (12) and $A(\zeta)=0$ if $|\zeta|>\delta,$ (3) follows from (8). The difference quotients of $\scriptstyle A$ converge boundedly to the corresponding partial derivatives of $A,$ since392 REAL AND coMPLEX ANALYSIs $A\in C_{c}^{\prime}(R^{2}).$ Hence the last expression in(11) may be differentiated under the integral sign, and we obtain ( $$ \begin{array}{c}{{\mathrm{j}\Phi(z)=\displaystyle\int_{a^{\prime}}^{\left[ (\bar{\partial}A\right)(z-\sqrt{\langle\bar{A}A\rangle(\bar{\partial}\rangle\,d\bar{\gamma}\,d\eta}}}}\\ {{\mathrm{~}=\displaystyle\int_{a^{\prime}}^{ [\int(z-\sqrt{\langle\bar{A}A\rangle(\bar{\partial})\,d\bar{\gamma}\,d\bar{\psi}}}}}\end{array} $$ ds dr. (13) The last equality depends on (9). Now (10) and(13) give(4). If we write (13) with $\Phi_{x}$ and $\Phi_{y}$ in place of ,we see that $\Phi_{\mathrm{\scriptscriptstyleI}}$ and $\mathbf{\nabla}(5)$ will follow if we can $\Phi$ has continuous partial derivatives. Hence Lemma 20.3 applies to show that ${\hat{\sigma}}\Phi=0$ in ${\cal G},$ where ${\boldsymbol{G}}$ is the set of all $\scriptstyle{\varepsilon\in K}$ whose distance from the complement of $ .K$ exceeds 6. We shall do this by showing that $$ \Phi(z)=f(z)\qquad(z\in G); $$ (14) note that ${\bar{\partial}}f=0$ in ${\cal G},$ since $\boldsymbol{\mathsf{f}}$ is holomorphic there. (We recall that $\tilde{\partial}$ is the Cauchy-Riemann operator defined in Sec. 11.1.) Now if $\scriptstyle{\varepsilon\in G}$ j, then $z-\zeta$ is in the interior of ${\cal K}\,\,\,\,\,\,\,\,$ for all $\textstyle\left\{{\begin{array}{l l}{\left\{\begin{array}{l l}{\end{array}}\right.}\end{array}}\right.$ with $|\zeta|<\delta.$ The mean value property for harmo- nic functions therefore gives, by the first equation in (11), $$ \begin{array}{c}{{\Phi(z)=\displaystyle\int_{0}^{\delta}a(r)r~d r\ \displaystyle\int_{0}^{2\pi}f(z-r e^{i\theta})~d\theta}}\\ {{\ }}\\ {{=2\pi f(z)\displaystyle\int_{0}^{\delta}a(r)r~d r=f(z)\displaystyle\int_{R^{2}}^{\delta}A=f(z)}}\end{array} $$ (15) for all $\varepsilon\in G$ We have now proved (3), (4), and (5). The definition of $\textstyle X$ X shows that $\textstyle X$ is compact and that $X$ can be covered $K.$ by finitely many open discs $S^{2}-K.$ It follows that each D contains a compact con- whose centers are not in Since $D_{1},\ldots,D_{n},$ of radius $2\delta,$ can be joined to oo by a $g^{2}-k$ is connected, the center of each $\underline{{D}}_{i}$ polygonal path in that nected set $E_{j},$ of diameter at least 26, so that $S^{2}-E_{j}$ is connected and so $K\cap E_{j}=\mathcal{D}.$ $r=2\delta.$ There exist functions We now apply Lemma_20.2,withUNIFORM APPROxIMATION BY PoLYNOMIALS 393 $g_{j}\in H(S^{2}-E_{j})$ and constants $b_{j}$ so that the inequalities $$ |Q_{i}\xi,z\}|<\frac{30}{\delta}, $$ (16) $$ \left|Q_{j}(\zeta,z)-\frac{1}{z-\zeta}\right|<\frac{4,000\delta^{2}}{|z-\zeta|^{3}} $$ (17) hold for z味 $E_{j}$ and $\zeta\in D_{j},$ if $$ Q_{j}(\zeta,z)=g_{j}(z)+(\zeta-b_{j})g_{j}^{2}(z). $$ (18) Let Q be the complement of $E_{1}\cup\cdot\cdot\sim\ \cup E_{n},$ Then $\Omega$ is an open set which contains K. 2 $:i\leq n.$ Define $$ {\begin{array}{l l}{X_{1}=X\cap D_{1}\quad{\mathrm{~and~}}\quad X_{j}=(X\cap D_{j})-(X_{1}\setminus\cdots\setminus X_{j-1}),}\\ {-1}&{}\end{array}}\, $$ for Put $$ R(\zeta,z)=Q_{j}(\zeta,z)\qquad(\zeta\in X_{j},z\in\Omega) $$ (19) and $$ F(z)=\frac{1}{\pi}\left|\!\right>^{\!\left(\widetilde{\theta}\Phi\right)\!(\zeta)R(\zeta,\,z)\ d\xi\ d\eta\qquad(z\in\Omega). $$ (20) Since $$ F(z)=\sum_{j=1}^{n}{\frac{1}{\pi}}\prod_{x_{j}}^{\left(\right)}(\tilde{\phi}\Phi)(\zeta)Q_{j}(\zeta,z)\;d\zeta\;d\eta, $$ (21) Hence (18) shows that ${\mathbf{}}F$ is a finite linear combination of the functions ${\mathfrak{g}}_{j}$ and $g_{j}^{2}.$ $F\in H(\Omega).$ By (20),(4), and (5) we have $$ |F(z)-\Phi(z)|<{\frac{2\omega(\delta)}{\pi\delta}}\prod_{\chi}\left|\right.\left|\frac{}{}R(\zeta,z)-{\frac{1}{z-\zeta}}\right|d\zeta\;d\eta\qquad(z\in\Omega). $$ (22 Q(C, Observe that the inequalities (16) and $\left(\bigsqcup{}^{\mu}{\mathcal{J}}\right)$ are valid with ${\boldsymbol{R}}$ in place of $Q_{j}$ if $\zeta\in X$ and z e Q. For if $\zeta\in X$ then $\zeta\in X_{j}$ for some ${\dot{j}},$ and then R(, z)= $z_{\mathrm{{\scriptsize~Z}}}$ for all z e Q2. if Now $\mathbb{H}x:\in\Omega,$ put $\zeta=z+\rho e^{i\theta},$ and estimate the integrand in (22) by (16) $\rho<4\delta,$ by (17) if $4\delta\leq\rho.$ The integral in (22) is then seen to be less than the sum of $$ 2\pi\mid_{\delta}^{*4\delta}{\left({\frac{50}{\delta}}+{\frac{1}{\rho}}\right)\rho~d\rho}=808\pi\delta $$ (23)394 REAL AND coMPLEX ANALYSIs and $$ 2\pi\, |_{4\delta}^{*\circ}\frac{4.000\delta^{2}}{\rho^{3}}\,\rho\,\,d\rho=2.000\pi\delta. $$ (24) Hence (22) yields $$ |F(z)-\Phi(z)|<6,000\omega(\delta)\qquad(z\in\Omega). $$ (25) Since F∈ H(Q) $\kappa\in\Omega,$ and $\scriptstyle{g^{2}-K}$ is connected, Runge's theorem shows that ${\mathbf{}}F$ can be uniformly approximated on ${\cal K}\,\,\,$ by polynomials. Hence $({\mathbf{3}})$ and (25) show that (2) can be satisfied This completes the proof // One unusual feature of this proof should be pointed out. We had to prove that the given function $\boldsymbol{\f}$ is in the closed subspace P(K) of C(K).(We use the by O. But terminology of Sec.20.1.) Our frst step consisted in approximating $\boldsymbol{\mathsf{f}}$ this step took us outside $\scriptstyle P(K),$ since p was so constructed that in general $\ \Phi$ will not be holomorphic in the whole interior of $K.$ Hence $\Phi$ is at some positive distance from $\scriptstyle P(k).$ However, (25) shows that this distance is less than a constant multiple of $\scriptstyle4(0)$ [In fact, having proved the theorem, we know that this distance is at most D(6), by (3) rather than 6,000 o(6).] The proof of (25) depends on the inequality (4) and on the fact that ${\hat{\sigma}}\Phi=0$ in ${\cal G}.$ Since holomorphic functions $\varphi$ are characterized by ${\bar{\partial}}\varphi=0,(4)$ may be regarded as saying that $\Phi$ is not too far from being holomorphic, and this interpretation is confirmed by (25) Exercises 1 Extend Mergelyan's theorem to the case in which $\scriptstyle{\vec{g}}\,-\,{\vec{kappa}}$ has finitely many components: Prove that then every fe C(K) which is holomorphic in the interior of ${\cal K}$ can be uniformly approximated on ${\mathbf{}}K$ by rational functions. discs in $\boldsymbol{\mathit{U}}$ whose union ${\boldsymbol{y}}$ is dense in $U_{\mathrm{,}}$ J, such that $\Sigma r_{n}<\infty$ Put $K={\bar{U}}-V.$ Let ${\Gamma}$ and in the plane, by be the 2 Show that the result of Exercise 1 does not extend to arbitrary compact sets $\scriptstyle{\mathcal{K}}$ verifying the details of the following example. For $n=1,2.3$ ,.…, let $D_{n}{}_{-}\,D(\alpha_{n};\,r_{n})$ be disjoint open paths $\gamma_{n}\,$ $$ \Gamma(t)=e^{i t},\quad\quad\gamma_{n}(t)=\emptyset_{n}+r_{n}e^{i t},\quad\quad0\leq t\leq2\pi, $$ and define $$ L(f)=\int_{\Gamma}f(z)\,d z-\,\sum_{n=1}^{\infty}\int_{\gamma_{n}}^{t}f(z)\,d z\qquad(f\in C(K)). $$ Prove that $\boldsymbol{\mathit{L}}$ is a bounded linear functional on $C(K),$ prove that $L(R)=0$ for every rational function R whose poles are outside $K,$ and prove that there exists $\mathbf{an}f\in C(K)$ for which $L(f)\neq0.$ among 3 Show that the function $\scriptstyle{\mathcal{G}}$ constructed in the proof of Lemma 20.2 has the smallest supremum norm can therefore be replaced by $\mathbf{a}|I\in H(\Omega)$ such that $z f(z)\to1$ as $z arrow\infty.$ (This motivates the proof of the lemma)) Show also that $b=c_{\mathrm{0}}$ in that proof and that the inequality ${\boldsymbol{E}}.$ $|b|<4r$ 1b|<r. In fact, b lies in the convex hull of the setAPPENDIX HAUSDORFF'S MAXIMALITY THEOREM We shall first prove a lemma which, when combined with the axiom of choice, leads to an almost instantaneous proof of Theorem 4.21. that If 沙 is a collection of sets and $\Phi\subset{\mathcal{F}}.$ , we call $\Phi$ a subchain of $\mathcal{F}$ provided and $\mathcal{F}$ $\mathbf{\Phi}\Phi$ is totally ordered by set inclusion. Explicitly, this means that if $\scriptstyle4\;\epsilon\oplus$ $B\in\Phi_{s}$ then either $\scriptstyle A\leq B$ or ${\mathfrak{g}}\in{\mathfrak{A}}$ The union of all members of $\Phi$ will simply be referred to as the union of $\Phi.$ Lemma Suppose . ${\widehat{\mathcal{M}}}$ is a nonempty collection of subsets of a set $X$ such that the associates to each $A\in{\mathcal{F}}$ union of every subchain of F belongs to 多.Suppose $\scriptstyle{\mathcal{G}}$ is a function which $g(A)-A$ a set g(A) ∈ F such that $A\in g(A)$ and consists of at most one element. Then there exists an A∈F for which $g(A)=A.$ PRO0F Fix $A_{0}$ e 多. Call a subcollection ${\mathcal{F}}^{\prime}$ of $\mathcal{F}$ a tower if ${\mathcal{F}}^{\prime}$ has the fol- lowing three properties: (a) $A_{0}\in{\mathcal{F}}^{\prime}.$ belongs to ${\mathcal{F}}^{\prime}.$ (b) The union of every subchain of ${\mathcal{F}}^{\prime}$ (c) If A∈ F', then also $g(A)\in{\mathcal{F}}^{\prime}.$ $A\in{\mathcal{F}}$ such that $4_{0}\subset A,$ The family of all towers is nonempty. For if ${\mathcal{F}}_{1}$ is a tower. Let ${\mathcal{F}}_{0}$ be the intersection of $\widetilde{\mathcal{P}^{\prime}}\[$ is the collection of all then all towers. Then ${\mathcal{F}}_{0}$ is a tower the verification is trivial),but no proper subcollection of ${\mathcal{P}}_{0}$ is a tower. Also, $A_{0}\subset A\ \ {\mathrm{if}}\ A\in{\mathcal{F}}_{0}$ The idea of the proof is to show that ${\mathcal{F}}_{0}$ is a subchain of 多 395