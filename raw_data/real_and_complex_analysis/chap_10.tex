CHAPTER TEN ELEMENTARY PROPERTIES OF HOLOMORPHIC FUNCTIONS Complex Differentiation We shall now study complex functions defined in subsets of the complex plane. It will be convenient to adopt some standard notations which will be used through out the rest of this book. 10.1 Definitions $|t r>0$ and $\bar{a}$ is a complex number, $$ D(a;r)=\{z\colon|z-a|<r\} $$ (1) is the open circular disc with center at $\bar{a}$ and radius $r.{\tilde{D}}(a;r)$ is the closure of $\scriptstyle D(n,\,n)$ and $$ D^{\prime}(a;r)=\{z:0<|z-a|<r\} $$ (2) is the punctured disc with center at $\bar{a}$ and radius ${\boldsymbol{r}}.$ A set $\boldsymbol{E}$ in a topological space $X$ is said to be not connected if $\boldsymbol{E}$ is the union of two nonempty sets $\scriptstyle A$ and $\boldsymbol{B}$ such that $$ \vec{A}\,\cap\,B=\mathcal{D}=A\,\cap\,\vec{B}. $$ (3) If $\scriptstyle A$ and $\boldsymbol{B}$ are as above, and if ${\mathbf{}}V$ and ${\boldsymbol{W}}$ are the complements of $\bar{A}$ and ${\tilde{B}},$ respectively, it follows that $A\subset W$ and $\scriptstyle B\in V_{\circ}$ Hence E c V o W, E o V ≠ O, E o W ≠ 8, EoVo W = 0.(4) Conversely, if open sets ${\mathbf{}}V$ and $\textstyle W$ exist such that (4) holds, it is easy to see that $\boldsymbol{E}$ is not connected, by taking $A=E\cap W,B=E\cap V.$ is the union of If $\boldsymbol{E}$ is closed and not connected, then (3) shows that $\boldsymbol{E}$ two disjoint nonempty closed sets; for if $A\<A\cup B$ and $A\cap B=\varnothing,$ then $\bar{A}=A.$ 196ELEMENTARY PROPER TIES OF HOLOMORPHIC FUNCTIONs 197 If $\boldsymbol{E}$ is open and not connected, then (4) shows that $\boldsymbol{E}$ is the union of two disjoint nonempty open sets, namely $E\,\cap\,V$ and $E\frown W.$ family $\Phi_{x}$ maximal connected subset of $\textstyle E.$ Each set consisting of a single point is obviously connected. If that contain $\scriptstyle{\mathcal{X}}$ is therefore not empty. $\scriptstyle x\in E_{s}$ the of all connected subsets of ${\boldsymbol{E}}$ The union of all members of G $\Phi_{x}$ is easily seen to be connected, and to be a $\textstyle E.$ Any . These sets are called the components of two components of $\boldsymbol{E}$ are thus disjoint, and $\boldsymbol{E}$ is the union of its components By a region we shall mean a nonempty connected open subset of the complex plane. Since each open set S $\Omega$ in the plane is a union of discs, and since all discs are connected, each component of $\Omega$ is open. Every plane open set is thus a union of disjoint regions. The letter $\Omega$ will from now on denote a plane open set 10.2 Definition Suppose fis a complex function defined in Q2. If zo∈ Qand if $$ \operatorname*{lim}_{z\to z_{0}}{\frac{f(z)-f(z_{0})}{z-z_{0}}} $$ (1) exists, we denote this limit $z_{i_{0}}\in\Omega_{i}$ we say that $\boldsymbol{\f}$ is holomorphic (or analytic) in $\scriptstyle m(0)$ $\mathbb{Z}_{\mathrm{O}}$ $15f^{\prime}(z_{0})$ exists for every class of all holomorphic functions in $\Omega$ and call it the derivative off at $\Omega.$ The $\mathfrak{s y}f(z_{0})$ will be denoted by To be quite explicit, $f^{\prime}(z_{0})$ exists $\mathrm{if}$ to every $\scriptstyle x\;y$ there corresponds a $\scriptstyle\delta>0$ such that $$ \left|{\frac{f(z)-f(z_{0})}{z-z_{0}}}-f^{\prime}(z_{0})\right|<\epsilon\qquad{\mathrm{for~all~}}z\in D^{\prime}(z_{0};\,\delta). $$ (2) Thus $f^{\prime}(z_{0})$ is a complex number, obtained as a limit of quotients of complex numbers. Note that fis a mapping of S $\Omega$ into $R^{2}$ and that Definition 7.22 associates with such mappings another kind of derivative, namely,a linear operator on ${\boldsymbol{R}}^{2}$ . In our present situation, if (2) is satisfied, this linear operator turns out to be multiplication by f(zo) (regarding ${\boldsymbol{R}}^{2}$ as the complex field). We leave it to the reader to verify this. 10.3 Remarks If $f\in H(\Omega)$ and $g\in H(\Omega),$ then also $f+g\in H(\Omega)$ and $f g\in H(\Omega),$ so that $\scriptstyle{m_{(0)}}$ is a ring; the usual differentiation rules apply More interesting is the fact that superpositions of holomorphic functions are holomorphic: If fe H(Q2), if f(Q) c Q,i $g\in H(\Omega_{1}),$ and if $h=g\circ f,$ then $h\in H(\Omega),$ and h ’can be computed $b y$ the chain rule $h^{\prime}$ $$ h\left(z_{0}\right)=g^{\prime}(f(z_{0}))f^{\prime}(z_{0})\qquad(z_{0}\in\Omega). $$ (1) To prove this $\mathbb{H}\times z_{0}\in\Omega,$ and put w。=f(zo). Then $$ \begin{array}{l c r}{{f(z)-f(z_{0})=[f^{\prime}(z_{0})+\epsilon(z)](z-z_{0}),}}\\ {{g(w)-g(w_{0})=[g^{\prime}(w_{0})+\eta(w)](w-w_{0}),}}\end{array} $$ (2) (3)198 REAL AND coMPLEX ANALYSIS where $\epsilon(z)\to0$ as $z\to z_{0}$ and n(w)→0 as w→Wo. Put $w=f(z),$ and substitute (2) into (3): If $z\neq z_{0}\,,$ $$ {\frac{h(z)-h(z_{0})}{z-z_{0}}}=[g^{\prime}(f(z_{0}))+\eta(f(z))][f^{\prime}(z_{0})+\epsilon(z)]. $$ (4) from (4). The differentiability of $\boldsymbol{\mathsf{f}}$ forces f to be continuous at $Z_{\mathrm{O}}\,.$ o. Hence(1) follows 10.4 Examples For $n=0,1,2,\ldots,z^{n}$ is holomorphic in the whole plane, and is holomorphic in the same is true of every polynomial in z. One easily verifies directly that $1/z$ $\{z:z\neq0\}.$ Hence, taking $g(w)=1/w$ in the chain rule, we see that $\mathrm{if}f_{1}$ and fz are in $\scriptstyle{H(x)}$ and $\Omega_{0}$ is an open subset of $\Omega$ in whichf has no zero, then $f_{1}/f_{2}\in H(\Omega_{0}).$ Another example of a function which is holomorphic in the whole plane (such functions are called entire) is the exponential function defined in the Prologue. In fact, we saw there that exp is diferentiable everywhere, in the sense of Definition 10.2, and that exp’(z) = exp (z) for every complex z. 10.5 Power Series From the theory of power series we shall assume only one fact as known, namely, that to each power series $$ \sum_{n=0}^{\infty}c_{n}(z-a)^{n} $$ (1) there corresponds a number $R\in[0,\,\infty]$ such that the series converges absolutely and uniformly in $\scriptstyle{B(\omega)}$ ${\boldsymbol{r}}),$ for every $\gamma<R_{*}$ and diverges if $z\not\in{\bar{D}}(a;R)$ The"“radius of convergence” ${\boldsymbol{R}}$ is given by the root test: $$ {\frac{1}{R}}=\operatorname*{lim}_{n\to\infty}{\operatorname*{sup}}\,|\,c_{n}|^{1/n}. $$ (2) Let us say that a function f defined in $\Omega$ is representable by power series in $\Omega$ for all if to every disc $D(a;r)<\Omega$ there corresponds a series (1) which converges to f(z) $z\in D(a;r).$ 10.6 Theorem Iff is representable by power series in Q, then fe H(Q) and $f^{\prime}$ is also representable by power series in Q. In fact,i $$ f(z)=\sum_{n=0}^{\infty}c_{n}(z-a)^{n} $$ (1) for $z\in D(a;r),$ then for these $\widetilde{\mathbb{Z}}$ we also have $$ f^{\prime}(z)=\sum_{n=1}^{\infty}n c_{n}(z-a)^{n-1}. $$ (2)ELEMENTARY PROPERTIES OF HOLoMORPHIC FUNCTIONs 199 PRoOF If the series (1) converges in $\scriptstyle D(a,\,r)$ the root test shows that the series (2) also converges there. Take $\scriptstyle a=0,$ without loss of generality, denote the If $z\neq w,$ sum of the series (2) by g(z),frx $w\in D(a;r),$ and choose $\boldsymbol{\rho}$ so that $|w|<\rho<r.$ we have $$ {\frac{f(z)-f(w)}{z-w}}-g(w)=\sum_{n=1}^{\infty}c_{n}{\biggl[}{\frac{z^{n}-w^{n}}{z-w}}-n w^{n-1}{\biggr]}. $$ (3) The expression in brackets is $\mathbf{0}$ if $\textstyle n=1$ , and is $$ (z-w){\!}_{k=1}^{n-1}k w^{k-1}z^{n-k-1} $$ (4) if $n\geq2.{\mathbb{F}}\,|z|<\rho,$ the absolute value of the sum in (4) is less than $$ {\frac{n(n-1)}{2}}\,\rho^{n-2} $$ (5) so $$ \lfloor{\frac{f(z)-f(w)}{z-w}}-g(w) \rfloor\leq\mid z-w\mid_{n=2}^{\infty}n^{2}\mid c_{n}\mid\rho^{n-2}. $$ (6) Since $\rho<r,$ the last series converges. Hence the left side of (6) tends to $\mathbf{0}$ as Z→w. This says that f"(w) = g(w) and completes the proof. // Corollary Since $f^{\prime}$ " ’is seen to satisfy the same hypothesis as f does, the theorem can be applied to f'.It follows that $\boldsymbol{\mathsf{f}}$ has derivatives of all orders, that each derivative is representable by power series in $\Omega,$ Q2, and that $$ f^{(k)}(z)=\sum_{n=k}^{\infty}n(n-1)\cdot\cdot\cdot\cdot(n-k+1)c_{n}(z-a)^{n-k} $$ (7) if(1) holds. Hence(1) implies that $$ k!c_{k}=f^{(k)}(a)\qquad(k=0,\;1,\;2,\,\cdot..), $$ (8) so that for each ae Q there is a unique sequence {c} for which(1) holds We now describe a process which manufactures functions that are represent able by power series. Special cases will be of importance later 10.7 Theorem Suppose p is a complex (finite) measure on a measurable space $X,$ $\varphi$ is a complex measurable function on X, $\Omega$ is an open set in the plane which does not intersect $\varphi(X),$ and $$ f(z)=\int_{x}{\frac{d\mu(\zeta)}{\varphi(\zeta)-z}}\qquad(z\in\Omega). $$ (1) Then f is representable by power series in Q200 REAL AND coMPLEX ANALYSIS PRooF Suppose $D(a;r)<\Omega.$ Since $$ \left|\frac{z-a}{\varphi(\zeta)-a}\right|\leq\frac{|z-a|}{r}<1 $$ (2) for every z e $D(a;r)$ and every $\zeta\in X,$ the geometric series $$ \sum_{n=0}^{\infty}\,\frac{(z-a)^{n}}{(\varphi(\zeta)-a)^{n+1}}=\frac{1}{\varphi(\zeta)-z} $$ (3) converges uniformly on X $X,$ for every fixed $z\in D(a;r).$ Hence the series(3) may be substituted into (1), and f(z) may be computed by interchanging sum- mation and integration. It follows that $$ f(z)=\sum_{0}^{\infty}\,c_{n}(z-a)^{n}\qquad(z\in D(a;\,r)) $$ (4) where $$ c_{n}=\int_{X}\frac{d\mu(\zeta)}{(\varphi(\zeta)-a)^{n+1}}\qquad(n=0,1,\,2,\,\ldots). $$ (5) // Note: The convergence of the series (4) in $D(a;r)$ is a consequence of the proof. We can also derive it from (5), since (5) shows that $$ |c_{n}|\leq{\frac{|\mu|(X)}{r^{n+1}}}\qquad(n=0,1,2,\ldots). $$ (6) Integration over Paths Our first major objective in this chapter is the converse of Theorem 10.6: Every f e H(Q2) is representable by power series in Q. The quickest route to this is via Cauchy's theorem which leads to an important integral representation of holo- morphic functions. In this section the required integration theory will be devel oped; we shall keep it as simple as possible, and shall regard it merely as a useful tool in the investigation of properties of holomorphic functions. 10.8 Definitions If $X$ is a topological space, a curve in $X$ is a continuous mapping yof a compact interval $[x,\,\beta]\subset R^{1}$ into $X{\dot{\boldsymbol{x}}}:$ ; here α<β.We call [α,β] the parameter interval of $\scriptstyle{\mathcal{Y}}$ and denote the range of y by p*. Thus y is a mapping, and yt is the set of all points yt), for α ≤t ≤ β. If the initial point y(c) of $\scriptstyle{\mathcal{Y}}$ coincides with its end point y(), we call , a closed curve. A path is a piecewise continuously differentiable curve in the plane. More explicitly, a path with parameter interval [α, $\beta]$ is a continuous complex function y on [α,β], such that the following holds: There are finitely manyELEMENTARY PROPER TIES OF HOLOMORPHIC FUNCTIONS 201 points $s_{j},$ $\alpha=s_{0}<s_{1}<\cdots<s_{n}=\beta,$ and the restriction of $\scriptstyle{\mathcal{Y}}$ may differ to each interval $S_{1},\,\cdot\cdot\cdot,\,S_{n-1}$ A closed path is a closed curve which is also a path. $\scriptstyle{\mathcal{Y}}$ $\left[s_{j-1},\ s_{j}\right]$ has a continuous derivative on $\textstyle\left[S_{j-1},\,S_{j}\right];$ however, at the points the left- and right-hand derivatives of Now suppose $\scriptstyle\gamma$ is a path, and $\boldsymbol{\mathsf{f}}$ is a continuous function on $\mathfrak{p}^{\mathrm{2}\#},$ The integral of f over $\scriptstyle{\mathcal{V}}$ is defined as an integral over the parameter interval [ ${\mathfrak{t}},{\mathfrak{p}}{\mathfrak{t}}$ of $\gamma\colon$ $$ \int_{\gamma}f(z)\ d{z}=\int_{a}^{\varepsilon\beta}f(\gamma(t))\gamma^{\prime}(t)\ d{t}. $$ (1) $\gamma_{1}$ Let $\varphi$ be a continuously differentiable one-to-one mapping of an interval and put $\gamma_{1}=\gamma\circ\varphi.$ is Then $[\alpha_{1},\beta_{1}]$ onto [α,β], such that $\varphi(\alpha_{1})=\alpha,\,\varphi(\beta_{1})=\beta,$ ,]; the integral of f over $\gamma_{1}$ is a path with parameter interval $[\alpha_{1},\,\beta$ $$ \bigcap_{x_{1}}^{\beta_{1}}f(\gamma_{1}(t))\gamma_{1}^{\prime}(t)\;d t= \bigcap_{a_{1}}^{\beta_{1}}f(\gamma(\varphi(t)) \rangle\gamma^{\prime}(\varphi(t))\varphi^{\prime}(t)\;d t= \bigcap_{\alpha}^{\beta}f(\gamma(s))\gamma^{\prime}(s)\;d s, $$ so that our“ reparametrization”has not changed the integral: $$ (\int_{\gamma_{1}}f(z)\ d z= (\frac\phisharp f(z)\ d z. $$ (2) $\gamma$ $\gamma_{1}$ Whenever (2) holds for a pair of paths $\scriptstyle\gamma$ and ${\mathfrak{Y}}_{1}$ 1 (and for $\operatorname{al}f_{\mathrm{{J}}}$ we shall regard y and ', as equivalent It is convenient to be able to replace a path by an equivalent one, i.e., to cides with the initial point of $\gamma_{2}$ choose parameter intervals at will. For instance, if the end point of $\gamma_{1}$ coin- , we may locate their parameter intervals so that $\gamma_{1}$ and $\gamma_{2}$ 'z join to form one path y, with the property that $$ \bigcup_{J}^{*}f=\bigcup_{y_{1}}f+\bigcup_{\gamma_{2}}^{}f $$ (3) for every continuous f on $\gamma^{*}=\gamma_{1}^{*}\cup\gamma_{2}^{*}$ y, and However, suppose that [o, i] is he parameter interval of a path $\gamma,$ $\gamma_{1}(t)=\gamma(1-t),\,0\leq t\leq1.$ We call $\gamma_{1}$ ,the path opposite to y, for the following reason: For any f continuous on $\gamma_{1}^{*}=\gamma^{*},$ we have $$ \bigcap_{0}^{1}f(\gamma_{1}(t))\gamma_{1}^{\prime}(t)\;d t=-\int_{0}^{1}f(\gamma(1-t))\gamma^{\prime}(1-t)\;d t=-\;\int_{0}^{t_{1}}f(\gamma(s))\gamma^{\prime}(s)\;d s, $$ so that $$ \bigcap_{p}f=- \{f_{r} $$ (4)202 REAL AND coMPLEX ANALYSIS From (1) we obtain the inequality $$ |\bigcap_{\gamma}f(z)\;d z\bigcup\leq\|f\|_{\cdots}\int_{a}^{\beta}|\gamma^{\prime}(t)|\;d t, $$ (5) where $\|f\parallel_{\infty}$ is the maximum of $|f|$ on $\gamma^{\bullet}$ and the last integral in (5) is (by definition) the length of $\textstyle\iint_{*}$ 10.9 Special Cases (a)If a is a complex number and $\scriptstyle\gamma\geq0,$ the path defined by $$ \gamma(t)=a+r e^{t t}\qquad(0\leq t\leq2\pi) $$ (1) is called the positively oriented circle with center at a and radius ${\boldsymbol{r}}{\hat{\boldsymbol{r}}}$ we have $$ \stackrel{\bullet}{\mapsto}f(z)\ d z=i r\stackrel{\circ}{\sim}^{i\pi}f(a+r e^{i\theta})e^{i\theta}\ d\theta, $$ (2) and the length of $\scriptstyle{\mathcal{Y}}$ is $2\pi r,$ as expected (b)If a and $\boldsymbol{\ b}$ are complex numbers, the path $\scriptstyle{\mathcal{Y}}$ given by $$ \gamma(t)=a+(b-a)t\qquad(0\leq t\leq1) $$ (3) is the oriented interval [a,b]; its length is|b -a|, and $$ \bigcap_{[a,b]}f(z)\;d z=(b-a)\bigcap_{0}^{1}f[a+(b-a)t]\;d t. $$ (4) If $$ \gamma_{1}(t)=\frac{a(\beta-t)+b(t-\alpha)}{\beta-\alpha}\qquad(\alpha\leq t\leq\beta), $$ (5) we obtain an equivalent path, which we still denote by [a, 万].The path opposite to [a,b] is [b,a]. (c) Let {a, b, c} be an ordered triple of complex numbers, let $$ \Delta=\Delta(a,\,b,\,c) $$ be the triangle with vertices at a,b, and c (A is the smallest convex set which contains a, ${\mathfrak{b}},$ and c), and define $$ \bigcap_{b\Delta}f=\left\{\bigcup_{[a,b]}f+\ \right\}_{[b,c]}f+\ \{\bigcup_{[c,a]}f_{i} $$ (6) for any f continuous on the boundary of $\Delta.$ We may regard (6) as the defini tion of its left side. Or we may regard $\delta\Delta$ as a path obtained by joining [a, b] to [b, c] to [c, a], as outlined in Definition 10.8,in which case (6) is easily proved to be true.ELEMENTARY PROPER TIES OF HOLOMORPHIC FUNCTIONS 203 If $\{a,b,c\}$ is permuted cyclically, we see from (6) that the left side of (6) is unaffected. If {a,b, c} is replaced by {a, c, b}, then the left side of (6) change sign. We now come to a theorem which plays a very important role in function theory. 10.10 Theorem Let y be a closed path, let $\Omega$ be the complement of y* (relativ to the plane), and define $$ \operatorname{Ind}_{\gamma}(z)={\frac{1}{2\pi i}}\left|\!\!\begin{array}{c\right|_{\gamma}{\frac{d\zeta}{\zeta-z}}\qquad(z\in\Omega). $$ (1) Then Ind, is an integer-valued function on $\Omega$ which is constant in each com ponent of Q and which is $\mathbf{0}$ in the unbounded component of Q2. We call Ind, (z) the index of z with respect to ${\mathfrak{Y}}.$ Note that $\gamma^{\star}$ is compact hence $\gamma^{\bullet}$ lies in a bounded disc $D\!\!\!\!/$ whose complement i ${\boldsymbol{D}}^{c}$ is connected; thus ${\boldsymbol{D}}^{c}$ e lies in some component of $\Omega.$ L . This shows that $\Omega$ has precisely one unbounded com ponent. PRoOF Let [α, β] be the parameter interval of $\gamma_{\mathrm{,\varepsilon}}$ fix z e Q, then $$ \operatorname{Ind}_{\gamma}\left(z\right)={\frac{1}{2\pi i}}\int_{x}^{\beta}{\frac{\gamma^{\prime}(s)}{\gamma(s)-z}}\,d s. $$ (2) Since w/2ri is an integer if and only if $e^{w}=1,$ , the first assertion of the theorem, namely, that Ind,(z) is an integer,is equivalent to the assertion that $\varphi({\boldsymbol{\beta}})=1,$ where $$ \varphi(t)=\exp\left\{\int_{\pi}^{t}{\frac{\gamma^{\prime}(s)}{\gamma(s)-z}}\;d s\right\}\qquad(\alpha\leq t\leq\beta). $$ (3) Differentiation of (3) shows that $$ {\frac{\varphi^{\prime}(t)}{\varphi(t)}}={\frac{\gamma^{\prime}(t)}{\gamma(t)-z}} $$ (4) except possibly on a finite set $\boldsymbol{\mathsf{S}}$ where $\scriptstyle{\mathcal{Y}}$ is not differentiable. Therefore $[x,\beta]-S.$ Since $\boldsymbol{\mathsf{S}}$ is a continuous function on [α, $\varphi/(\gamma-z)$ is constant on [α, $\beta];$ ; and since $\varphi/(\gamma-z)$ $\beta]_{}^{}$ whose derivative is zero in is finite, $\varphi(\alpha)=1,$ we obtain $$ \varphi(t)={\frac{\gamma(t)-z}{\gamma(x)-z}}\qquad(\alpha\leq t\leq\beta). $$ (5) (5) shows that We now use the assumption that $\scriptstyle\gamma$ is a closed path, i.e., tha $\gamma(\beta)=\gamma(\alpha);$ $\varphi({\boldsymbol{\beta}})=1,$ and this, as we observed above, implies that Ind,(z) is an integer.204 REAL AND coMPLEX ANALYSIS By Theorem 10.7,(1) shows that Ind,e H(Q). The image of a connected set under a continuous mapping is connected ([26], Theorem 4.22), and since Ind, is an integer-valued function, Ind, must be constant on each component of $\Omega.$ Finally,(2) shows that |Ind,(z)|<1 if |z| is sufficiently large. This implies that Ind,(z) = 0 in the unbounded component of Q. // Remark: If ${\dot{\lambda}}(t)$ denotes the integral in (3), the preceding proof shows that 2m Ind, (z) is the net increase in the imaginary part of A( ${\dot{\lambda}}(t),$ ),as ${\mathbf{}}t$ runs from α to ${\boldsymbol{\beta}},$ and this is the same as the net increase of the argument of $\scriptstyle{i(t)-z}.$ (We have not defined“argument”and will have no need for it.) If we divide this increase by 2元, we obtain“ the number of times that $\scriptstyle{\mathcal{Y}}$ winds around z,”and this explains why the term“winding number”is frequently used for the index. One virtue of the preceding proof is that it establishes the main properties of the index without any reference to the (multiple-valued) argu- ment of a complex number. 10.11 Theorem If y is the positively oriented circle with center at $\bar{a}$ and radius r, then $$ {\mathrm{Ind}}_{\gamma}\left(z\right)={\Bigg\{}{\frac{1}{0}}\quad\mathrm{~if~}\left|z-a\right|<r, $$ PROOF We take $\scriptstyle{\mathcal{V}}$ ,as in Sec. 10.9(a).、 By Theorem 10.10,it is enough to compute Ind,a), and 10.9(2) shows that this equals $$ \frac{1}{2\pi i}\left\{\right\}_{\gamma}\frac{d z}{z-a}=\frac{r}{2\pi}\left.\right\}_{0}^{z_{-}}(r e^{i t})^{-1}e^{i t}\,d t=1. $$ // The Local Cauchy Theorem path or cycle in S $\Omega,$ There are several forms of Cauchy's theorem. They all assert that if $\scriptstyle{\mathcal{Y}}$ is a closed and if y and $\Omega$ satisfy certain topological conditions, then the integral of every $f\in H(\Omega)$ over $\scriptstyle{\mathcal{Y}}$ is O. We shall frst derive a simple local version of this (Theorem 10.14) which is quite sufficient for many applications. A more general global form will be established later 10.12 Theorem Suppose $F\in H(\Omega)$ and ${\boldsymbol{F}}^{\prime}$ 4 is continuous in S.Then $$ \bigcup_{\gamma}^{*}F^{\prime}(z)\ d z=0 $$ for every closed path $\scriptstyle{\mathcal{Y}}$ in $\Omega.$ELEMENTARY PROPER TIES OF HOLOMORPHIC FUNCTIONS 205 PROOF If $\scriptstyle[x,\,\beta]$ is the parameter interval of $\gamma,$ the fundamental theorem of calculus shows that $$ \int_{\gamma}^{t}\!F^{\prime}(z)\ d z=\int_{\alpha}^{\beta}\!F^{\prime}(\gamma(t))\gamma^{\prime}(t)\ d t=F(\gamma(\beta))-F(\gamma(\alpha))=0, $$ since $\gamma(\beta)=\gamma(x).$ / have Corollary Since ${\mathrm{z}}^{n}$ is the derivative of $z^{n+1}/(n+1)$ for all itegers $n\neq-1,$ we $$ \binom{\bar{z}}{z}^{n}\ d{z}=0 $$ for every closed path y i $n=0,1,2,\ldots,$ and for those closed paths y for which 0 ≠ y* f $n=-2,-3,-4,\ldots$ The case $n=-1$ was dealt with in Theorem 10.10 10.13 Cauchy's Theorem for a Triangle SupposeL is a closed triangle in a $a n d f\in H(\Omega-\{p\}).$ Then plane open set Q, p∈ Q,f is continuous on Q2 $$ \bigcap_{b\Delta}f(z)\ d z=0. $$ (1) For the definition of OA we refer to Sec. 10.9(c). We shall se later that our hypothesis actually implies that $f\in H(\Omega),$ i.e., that the exceptional point ${\boldsymbol{p}}$ is no really exceptional. However, the above formulation of the theorem will be usefu in the proof of the Cauchy formula. PROOF We assume first that p生A. Let a,b, and $\scriptstyle{\mathcal{C}}$ be the vertices of $\Delta,$ let $a^{\prime},$ $b^{\prime},$ and ${\boldsymbol{c}}^{\prime}$ C’ be the midpoints of [b, c],[c, a], and [a, b], respectively, and con sider the four triangles $\Delta^{j}$ formed by the ordered triples $$ \{a,\,c^{\prime},\,b^{\prime}\},\qquad\{b,\,a^{\prime},\,c^{\prime}\},\qquad\{c,\,b^{\prime},\,a^{\prime}\},\qquad\{a^{\prime},\,b^{\prime},\,c^{\prime}\}. $$ (2) If J is the value of the integral (1),it follows from 10.9(6) that $$ J=\sum_{j=1}^{4}\; [\sum_{b\omega^{\prime}}f(z)\;d z. $$ (3) The absolute value of at least one of the integrals on the right of (3) is there fore at least |J/4|. Call the corresponding triangle $\Delta_{1}.$ repeat the argumen with $\Delta_{1}$ in place of $\Delta,$ A、 and so forth. This generates a sequence of triangles $\Delta_{n}$206 REAL AND cOMPLEX ANALYSIs such that $\Delta\to\Delta_{1}\to\Delta_{2}\to\cdots,$ such that the length of $\delta\Delta_{n}$ is $2^{-n}L,$ if ${\boldsymbol{L}}$ is the length of OA, and such that $$ \left|J\right|\leq4^{n}\left|\ \right|\int_{\mathrm{{bd}}_{n}}f(z)\ d z |\qquad(n=1,\,2,\,3,\,\ldots). $$ (4) There is a (unique) point $z_{\mathrm{O}}$ which the triangles $\Delta_{n}$ have in common. Since $\Delta$ is compact, $z_{0}\in\Delta,$ so fis differentiable at $z_{\mathrm{0}}$ such that Let $\scriptstyle x\,>0$ be given. There exists an $\scriptstyle\gamma\simeq0$ $$ |f(z)-f(z_{0})-f^{\prime}(z_{0})(z-z_{0})|\leq\epsilon|z-z_{0}| $$ (5) whenever $|z-z_{0}|<r,$ and there exists an $\;n$ such that $|z-z_{0}|<r$ for all z ∈ A.. For this n we also have $|z-z_{0}|\leq2^{-n}L$ for all $z\in\Delta_{n}.$ $\mathbf{By}$ the Corol- lary to Theorem 10.12, $$ \left|\bigcup_{b\Delta_{n}}^{\circ}f(z)\ d z= .\right|_{\mathbb{s l}_{n}}[f(z)-f(z_{0})-f^{\prime}(z_{0})(z-z_{0})]\ d z, $$ (6) so that (5) implies $$ \left|\;\right|_{\delta\Delta_{n}}f(z)\;d z\mid\leq\epsilon(2^{-n}L)^{2}, $$ (7) and now (4) shows that $|J|\leq\epsilon L^{2}.$ Hence ${\mathbf{}}J=0$ if $p\notin\Delta.$ are collinear Assume next that $~~~~~~~~~~~~~~~~~~~~~~~~~~~~~~~~~~~~~~~~~~~~~~~~~~~~~~~~~~~~~~~~~~~~~~~~~~~~~~~~~~~~~~~~~~~~~~~~~~~~~~~~~~~~~~~~~~~~~~~~~~~~~~~~~~~~~~~~~~~~~~~~~~~~~~~~~~~~~~~~~~~~~~~~~~~~~$ is a vertex of $\Delta,$ say $p=a.$ If a,b, and $\scriptstyle{\mathcal{C}}$ then (1) is trivial, for any continuous ${\boldsymbol{a}},$ and observe that the integral of f over OL is the $x\in[a,b]$ and $f.$ If not, choose points y e [a, c], both close to sum of the integrals over the boundaries of the triangles {α,×,y},{x, b,y} ${\boldsymbol{p}}.$ and {b,c,y}. The last two of these are ${\boldsymbol{0}},$ since these triangles do not contain Hence the integral over o is the sum of the integrals over [a, x],[x, yJ bounded on A, we again obtain (1). and [y, a], and since these intervals can be made arbitrarily short and $\boldsymbol{\f}$ f is Finally, if $\boldsymbol{\mathit{P}}$ is an arbitrary point of $\Delta,$ apply the preceding result to {α,b,p},{b,c, p}, and {c,α, p} to complete the proof. / 10.14 Cauchy's Theorem in a Convex Set Suppose $\Omega$ is a convex open set, $F\in H(\Omega).$ p e 9,fis continuous on $\Omega,$ and $f\in H(\Omega-\{p\}).$ Then $f=F^{\prime}$ for some Hence $$ \dagger_{\gamma}f(z)\ d z=0 $$ (1) for every closed path $\scriptstyle{\mathcal{Y}}$ in $\Omega.$ELEMENTARY PROPER TIES OF HOLOMORPHIC FUNCTIONs 207 PR0OF Fix a∈ Q. Since $\Omega$ is convex, $\Omega$ contains the straight line interval from a to z for every $\varepsilon\in\Omega$ so we can define $$ F(z)=\bigcap_{|a,z|}f(\xi)\;d\xi\qquad(z\in\Omega). $$ (2) $z_{0}\,.$ For any z and $z_{a}\in\Omega,$ the triangle with vertices at a, $z_{0}\,,$ and $\mathbb{Z}$ lies in Q; hence $F(z)-F(z_{0})$ is the integral of f over [zo, z], by Theorem 10.13. Fixing , we thus obtain $$ \frac{{\cal F}(z)-{\cal F}(z_{0})}{z-z_{0}}-f(z_{0})=\frac{1}{z-z_{0}}\left\vert_{t z_{0},z\bar{z}} [f(\zeta)-f(z_{0})\right]\,d\zeta $$ (3) if $z\neq z_{0}$ . Given $\scriptstyle\epsilon\;>_{0},$ the continuity of $\boldsymbol{\mathsf{f}}$ at $\mathbb{Z}_{0}$ shows that there is a $\delta>0$ such that $|f(\xi)-f(z_{0})|<\epsilon$ if $|\xi-z_{0}|<\delta;$ hence the absolute value of the $f=F^{\prime}.$ In particular, left side of (3) is less than eas soon Now $\mathbf{(1)}$ follows from Theorem 10.12. This proves that // ${\bf a}{\bf\nabla}|z-z_{0}|<\delta.$ $F\in H(\Omega).$ 10.15 Cauchy's Formula in a Convex Set Suppose y is a closed path in a convex open set $\Omega,$ and fe H(Q2) $\;|J z\in\Omega$ and z生 *, then $$ f(z)\cdot\operatorname{Ind}_{\gamma}(z)={\frac{1}{2\pi i}} \{\frac{f(\xi)}{\xi-z}\,d\xi. $$ (1) The case of greatest interest is, of course, Ind $(z)=1.$ PROOF Fix $\mathbb{Z}$ so that the above conditions hold, and define $$ g(\xi)=\Bigg\{\frac{f(\xi)-f(z)}{\xi-z}\qquad\mathrm{if}\ \xi\in\Omega,\,\xi\neq z, $$ (2) Then g satisfies the hypotheses of Theorem 10.14. Hence $$ \frac{1}{2\pi i} \vert_{y}(\xi)\;d\xi=0. $$ (3) If we substitute (2) into (3) we obtain (1) // The theorem concerning the representability of holomorphic functions by power series is an easy consequence of Theorem 10.15, if we take a circle for y 10.16 Theorem For every open set $\Omega$ in the plane, every fe H(Q) is represent- able by power series in SQ.208 REAL AND COMPLEX ANALYSIS PROOF Suppose $f\in H(\Omega)$ and $D(a;R)\subset\Omega$ If $\gamma$ is a positively oriented circle with center at $\bar{a}$ and radius $\gamma<R,$ the convexity of $D(a;R)$ allows us to apply Theorem 10.15; by Theorem 10.11, we obtain $$ f(z)={\frac{1}{2\pi i}} \{{\frac{f(\xi)}{\zeta-z}}\;d\xi\qquad(z\in D(a;r)). $$ (1) But now we can apply Theorem 10.7,with $X=[0,2\pi],$ $\varphi=\gamma,$ and dut) =f(r(t)y'() dt, and we conclude that there is a sequence $\{c_{n}\}$ such that $$ f(z)=\sum_{n=0}^{\infty}c_{n}(z-a)^{n}\qquad(z\in D(a;r)). $$ (2) The uniqueness of $\{c_{n}\}$ (see the Corollary to Theorem 10.6) shows that the same power series is obtained for every $\scriptstyle\gamma<R$ (as long as a is fixed) Hence the representation (2) is valid for every z ∈ D(a; R), and the proof is complete // Corollary Iff ∈ H(Q), then f’e H(Q). PROOF Combine Theorems 10.6 and 10.16. // The Cauchy theorem has a useful converse open set S 10.17 Morera's Theorem Suppose $\boldsymbol{\mathsf{f}}$ is a continuous complex function in an $\Omega$ S2 such that $$ \bigcap_{b\Delta}f(z)\ d z=0 $$ for every closed triangle △ c Q.Then fe H(Q) PROOF Let ${\mathbf{}}V$ be a convex open set in $\Omega.$ 2 As in the proof of Theorem 10.14 we can construct $F\in H(V)$ such that $\scriptstyle{P=f}$ Since derivatives of holomorphic for every // functions are holomorphic (Theorem 10.16),we have $f\in H(V),$ convex open $V\subset\Omega,{\mathrm{hence}}f\in H(\Omega)$ The Power Series Representation The fact that every holomorphic function is locally the sum of a convergen power series has a large number of interesting consequences. A few of these are developed in this section. 10.18 Theorem Suppose $\Omega$ is a region,fe H(Q), and $$ Z(f)=\{a\in\Omega\colon f(a)=0\}. $$ (1)ELEMENTARY PROPER TIES OF HOLoMORPHiC FUNCrioNs 209 Then either $Z(f)=\Omega,$ or Z(f) has no limit point in Q. In the latter case there such that corresponds to each $a\in Z(f)$ a unique positive integer $m=m(a)$ $$ f(z)=(z-a)^{m}g(z)\qquad(z\in\Omega), $$ (2) where $g\in H(\Omega)$ and g(a) ≠ 0; furthermore, $\scriptstyle2(p)$ is at most countable. (We recall that regions are connected open sets.) Clearly, The integer m is called the order of the zero which $\Omega.$ We call has at the point a $f^{\circ}$ $Z(f)=\Omega$ if and only if f is identically O in $\scriptstyle{2(J)}$ the zero set of $f.$ Analogous results hold of course for the set of α-points of ${\mathfrak{f}},$ i.e., the zero set of $f-\alpha,$ where $\scriptstyle{\vec{\alpha}}$ is any complex number. PRO0F Let $\scriptstyle A$ be the set of all limit points of $\scriptstyle{2t/1}$ in $\Omega.$ Sincefis continuous, $A\in Z(f).$ and choose $\scriptstyle r\gg0$ so that $D(a;r)<\Omega.$ By Theorem 10.16, $\operatorname{Fix}a\in Z(f),$ $$ f(z)=\sum_{n=0}^{\infty}c_{n}(z-a)^{n}\qquad(z\in D(a;r)). $$ (3) and $\bar{a}$ There are now two possibilities. Either all $A,$ or there is a smallest integer m [necessarily $D(a;r)\subset A$ is an interior point of $\textstyle c_{n}$ are ${\boldsymbol{0}},$ in which case positive, $\mathrm{since}f(a)=0]$ such that $c_{m}\neq0.$ In that case, define $$ g(z)= \{(z-a)^{-m}f(z)\qquad(z\in\Omega-\{a\}),\qquad $$ (4) Then (2) holds. It is clear that $g\in H(\Omega-\{a\}).$ But (3) implies $$ g(z)=\sum_{k=0}^{\infty}c_{m+k}(z-a)^{k}\qquad(z\in D(a;r)). $$ (5) Hence $g\in H(D(a;r)),$ so actuall $g\in H(\Omega).$ shows that there is a neigh- Moreover, $g(a)\neq0,$ and the continuity of $\scriptstyle{\mathcal{G}}$ borhood of $\bar{a}$ in which $\scriptstyle{\mathcal{G}}$ has no zero. Thus $\bar{a}$ is an isolated point of Z(f), by (2). If $a\in A,$ it is clear from the definition of ${\boldsymbol{A}}$ the first case must therefore occur. So A is open. If $B=\Omega-A,$ as a set of limit points that $\boldsymbol{B}$ is open. Thus $\Omega$ is the union of the disjoint open sets $\scriptstyle A$ and ${\boldsymbol{B}}.$ Since $\Omega$ is connected, we $\Omega$ is o-compact, has at most finitely many points in each compact subset of $\Omega,$ In the latter case // $\scriptstyle2(p)$ have either $A=\Omega,$ in which case $Z(f)=\Omega,$ or $A={\mathcal{O}}.$ and since $\mathrm{z}(p)$ is at most countable. Corollary Iff and g are holomorphic functions in a region $\Omega$ and i $f(z)=g(z)$ for allz in some set which has a limit poin in Q, hen Z) = (62)forall z Q.210 REAL AND coMPLEX ANALYSIs In other words, a holomorphic function in a region $\Omega$ is determined by its values on any set which has a limit point in $\Omega.$ This is an important uniqueness theorem. Note: The theorem fails if we drop the assumption that $\Omega$ is connected::If $\Omega=\Omega_{0}\cup\Omega_{1},$ and $\Omega_{0}$ o and $\Omega_{1}$ are disjoint open sets, put $f=0$ in $\Omega_{0}$ and $f=1$ in $\Omega_{1}.$ 10.19 Definition If $\overset{a\in\Omega}{\hookrightarrow}$ and $f\in H(\Omega-\{a\}),$ then $\boldsymbol{\f}$ is said to have an iso lated singularity at the point a.Iff can be so defined at $\bar{a}$ 7 that the extended function is holomorphic in $\Omega,$ the singularity is said to be removable. $\scriptstyle\gamma\leq0,$ 10.20 Theorem Suppose $f\in H(\Omega-\{a\})$ and fis bounded in $y_{\mathrm{to}}$ r), for some Then f has a removable singularity a ${\boldsymbol{a}}.$ Recall that $D^{\prime}(a;r)=\{z\colon0<|z-a|<r\}.$ PR0OF Define $h(a)=0,$ and $h(z)=(z-a)^{2}f(z)$ ${\boldsymbol{h}}$ is evidently differentiable at every other point of QL, we have $h\in H(\Omega),$ so in $\Omega-\{a\}.$ Our boundednes assumption shows that $h(a)=0.$ Since $$ h(z)=\sum_{n=2}^{\infty}c_{n}(z-a)^{n}\qquad(z\in D(a;r)). $$ then We obtain the desired holomorphic extension of f by settin $f(a)=c_{2}\,,$ for $$ f(z)=\sum_{n=0}^{\infty}c_{n+2}(z-a)^{n}\qquad(z\in D(a;\,r)). $$ // 10.21 Theorem $I f\ a\in\Omega$ and $f\in H(\Omega-\{a\}),$ then one of the following three cases must ocCur: (a)f has a removable singularity at a (b)There are complex numbers $C_{1\}}\,\cdot\,\cdot\,\cdot\,\cdot\,\cdot\,\cdot\,\cdot\,\textstyle\,C_{m}$ ,where m is a positive integer and $c_{m}\neq0,$ such that $$ f(z)-\sum_{k=1}^{m}\,{\frac{c_{k}}{(z-a)^{k}}} $$ has a removable singularity at a (c)If $r>0$ and $D(a;r)<\Omega,$ then f(D'(a; r) is dense in the plane In case (b),f is said to have a pole of order m at a. The function $$ \sum_{k=1}^{m}c_{k}(z-a)^{-k}, $$ELEMENTARY PROPER TIES OF HOLOMORPHIC FUNCTIONS 211 a polynomial in $(z-a)^{-1},$ is called the principal part of f at a. It is clear in this situation that $|f(z)|\to\infty$ aS $z\to a.$ In case ${\mathfrak{q}}_{3}f$ is said to have an essential singularity at a. A statement equiva lent to (c) is that to each complex number w there corresponds a sequence {z, such that $z_{n}\to a$ and $f(z_{n})\to$ w as n→OO. ${\boldsymbol{D}}^{\prime}$ for $D(a;r).$ PRoOF Suppose (c) fails. Then there exist in $D(a;r).$ Let us write $D\!\!\!\!/$ for $D(a;\,r)$ and $r>0,\ \delta>0,$ and a complex number w such that $|f(z)-w|>\delta$ Define $$ g(z)=\frac{1}{f(z)-w}\;\;\;\;\;\;\;\;(z\in D^{\prime}). $$ (1) Then $g\in H(D)$ and $|g|<1/\delta.$ By Theorem 10.20, $\mathbf{\Omega}^{g}$ extends to a holo- morphic function in $D\!\!\!\!/$ If $g(a)\neq0,$ (1) shows that fis bounded in $D(a;\rho)$ for some $\rho>0.$ Hence (a) holds, by Theorem 10.20. If g $\scriptstyle{\mathcal{G}}$ has a zero of order $\scriptstyle{m\geq1}$ at a, Theorem 10.18 shows that $$ g(z)=(z-a)^{m}g_{1}(z)\qquad(z\in D), $$ (2)) where $g_{1}\in H(D)$ and $g_{1}(a)\neq0$ Also, ${\mathfrak{g}}_{1}$ has no zero in ${\boldsymbol{D}}^{\prime}.$ 、by (1). Put $h=$ $\Big|\frac{\,}{\sqrt{\ }}O_{\bf1}$ in $D.$ Then $h\in H(D),h$ has no zero in ${\boldsymbol{D}},$ and $$ f(z)-w=(z-a)^{-m}h(z)\qquad(z\in D^{\prime}). $$ (3) But ${\boldsymbol{h}}$ has an expansion of the form $$ h(z)=\sum_{n=0}^{\infty}b_{n}(z-a)^{n}\qquad(z\in D), $$ (4) with $b_{0}\neq0.$ Now (3) shows that (b) holds, with $c_{k}=b_{m-k},k=1,\,\ldots,$ m. // This completes the proof. We shall now exploit the fact that the restriction of a power series Z c,(z -a" to a circle with center at $\bar{a}$ is a trigonometric series. 10.22 Theorem ${\boldsymbol{J}}{\boldsymbol{J}}$ $$ f(z)=\sum_{n=0}^{\infty}c_{n}(z-a)^{n}\qquad(z\in D(a;R)) $$ (1) and if O $<r<R$ , then $$ \sum_{n=0}^{\infty}\;|c_{n}|^{2}r^{2n}={\frac{1}{2\pi}}\int_{-\pi}^{\pi}|f(a+r e^{i\theta})|^{2}\;d\theta. $$ (2)$212$ REAL AND COMPLEX ANALYSIS PRoOF We have $$ f(a+r e^{i\theta})=\sum_{n=0}^{\infty}c_{n}r^{n}e^{i n\theta}. $$ (3) For $\gamma<R,$ the series (3) converges uniformly on [一兀,元]. Hence $$ c_{n}\,r^{n}=\frac{1}{2\pi} |\sp\pi_{\pi}\sp2\,f(a+r e\sp{i\theta})e\sp{-i n\theta}\,d\theta\qquad\big(n=0,\,1,\,2,\,...\big), $$ (4)) and (2) is seen to be a special case of Parseval's formula // Here are some consequences: 10.23 Liouville's Theorem Every bounded entire function is constant. Recall that a function is entire if it is holomorphic in the whole plane PRoOF Suppose $\boldsymbol{\mathit{f}}$ is entire, $|f(z)|<M$ for all $\mathbb{Z},$ and $f(z)=\sum c_{n}z^{n}$ for all $\mathbf{z}.$ By Theorem 10.22, $$ \sum_{n=0}^{\infty}|c_{n}|^{2}r^{2n}<M^{2} $$ for all ${\boldsymbol{r}},$ which is possible only if $c_{n}=0$ for all $n\geq1.$ // 10.24 The Maximum Modulus Theorem Suppose $\Omega$ is a region, f∈ H(Q), and ${\tilde{D}}(a;r)<\Omega$ Then $$ |f(a)|\leq\operatorname*{max}_{a}|f(a+r e^{i\theta})|\,. $$ (1) Equality occurs in (1) if and only if is constant in Q2 Consequently, If| has no local maximum at any point of $\Omega,$ unless f is con- stant. PROOF Assume that $|f(a+r e^{i\theta})|\leq|f(a)|$ for all real e. In the notation of Theorem 10.22 it follows then that $$ \sum_{n=0}^{\infty}|c_{n}|^{2}r^{2n}\leq|f(a)|^{2}=|c_{0}|^{2}. $$ $\Omega$ Hence $c_{1}=c_{2}=c_{3}=\cdot\cdot\cdot=0,$ , which implies tha $\mathbf{\tau}|f(z)=j(a)$ in $\scriptstyle D(\omega_{c})$ 八以.Sinc // is connected, Theorem 10.18 shows that fis constant in $\Omega.$ Corollary Under the same hypotheses, $$ |f(a)|\geq\operatorname*{min}_{\theta}|f(a+r e^{i\theta})| $$ (2) if f has no zero in D(a; r)ELEMENTARY PROPER TIES OF HOLOMORPHIC FUNCTIONs 213 PRooF If $f(a+r e^{i\theta})=0$ for some $\theta$ then (2) is obvious. In the other case, there is a region $\Omega_{0}$ that contains ${\tilde{D}}(a;\,r)$ and in which $\boldsymbol{\f}$ has no zero; hence // (1) can be applied to ${1}/{5}$ and (2) follows 10.25 Theorem If n is a positive integer and $$ P(z)=z^{n}+a_{n-1}z^{n-1}+\cdots+a_{1}z+a_{0}, $$ where $a_{0},\ \sim\cdot\cdot n_{n-1}$ are complex numbers, then ${\mathbf{}}P$ has precisely n zeros in the plane. Of course, these zeros are counted according to their multiplicities: A zero of order m, say, is counted as m zeros. This theorem contains the fact that the complex field is algebraically closed, i.e., that every nonconstant polynomial with complex coefficients has at least one complex zero. PROOF Choose $r>1+2 |\,a_{0}\,|+|\,a_{1}|+\cdots+|\,a_{n-1}|\,.$ Then $$ |P(r e^{i\theta})|>|P(0)|\;\;\;\;\;\;\;(0\leq\theta\leq2\pi). $$ If ${\mathbf{}}P$ had no zeros,then the function $f=1/P$ would be entire and would satisfy $|f(0)|>|f(r e^{i\theta})|$ for all ${\boldsymbol{\theta}},$ which contradicts the maximum modulus ${\cal Q},$ theorem. Thus $P(z_{1})=0$ for some $z_{1}$ Consequently, there is a polynomial of degree $n-1,$ such that $P(z)=(z-z_{1})Q(z).$ The proof is completed by // induction on n $z\in D(a;R),$ 10.26 Theorem (Cauchy's Estimates) $I f\in H(D(a;R))$ and |f(z)|≤ M for al then $$ |f^{(n)}(a)|\leq{\frac{n!\,M}{R^{n}}}\qquad(n=1,\,2,\,3,\,\ldots). $$ (1) PRoOF For each $\gamma<R.$ each term of the series 10.22(2) is bounded above by $M^{2}.$ // If we take a= 0, $\scriptstyle R\,=\,1.$ and $f(z)=z^{s};$ , then $M=1,f^{(n)}(0)=n!,$ and we see that (1) cannot be improved. $\scriptstyle e\;>0$ 10.27 Definition A sequence $\{J_{\partial}\}$ of functions in $\Omega$ is said to converge to $\boldsymbol{\mathsf{f}}$ uniformly on compact subsets $\scriptstyle{9\;\Omega}$ if to every compac $\kappa\in\Omega$ and to every for all there corresponds an $N=N(K,\epsilon)$ such that $|f_{j}(z)-f(z)|<\epsilon$ $z\in K\ {\mathrm{if}}j>N.$ For instance, the sequence {z"} converges to O uniformly on compact subsets of D(0;1), but the convergence is not uniform in $\overline{{\mathfrak{p}}}(s)$ 1)214 REAL AND coMPLEX ANALYsIs It is uniform convergence on compact subsets which arises most natu- rally in connection with limit operations on holomorphic functions. The term “ almost uniform convergence”is sometimes used for this concept. 10.28 Theorem Suppose f;e H(Q), for $j=1,$ 2,3,.…, and f;→f uniformly on compact subsets of Q.Then fe H(Q), and f{→f’uniformly on compact subsets of Q2. PRoOF Since the convergence is uniform on each compact disc in Q, $\boldsymbol{\f}$ is continuous. Let A be a triangle in Q. Then is compact, so $$ \bigcap_{\delta\Delta}f(z)\ d z=\operatorname*{lim}_{j arrow\infty}\ [\bigcup_{\delta\Delta}f_{j}(z)\ d z=0, $$ by Cauchy's theorem. Hence Morera's theorem implies that fe H(Q2) Let ${\cal K}$ be compact, $\kappa\in\Omega$ There exists an $\scriptstyle\nu\simeq0$ such that the union $\boldsymbol{E}$ of the closed discs $\scriptstyle{B e s}$ r), for all $\varepsilon\in K_{\circ}$ is a compact subset of $\Omega.$ Applying Theorem 10.26 to $f-f_{j},$ we have $$ |f^{\prime}(z)-f_{j}^{\prime}(z)|\leq r^{-1}\|f-f_{j}\|_{E}\qquad(z\in K), $$ where $\|f\|_{E}$ denotes the supremum of $K.$ $|f|$ on $E.$ Since $f_{j}{ arrow}f$ uniformly on $\scriptstyle E_{\mathrm{su}}$ follows that f;→f’uniformly on // Corollary Under the same hypothesis, f"→f”) uniformly, a $j\to\varnothing.$ , on every compact set $K\in\Omega.$ and for every positive integer n Compare this with the situation on the real line, where sequences of infinitely differentiable functions can converge uniformly to nowhere differentiable func tions! The Open Mapping Theorem 1f $\Omega$ is a region and fe H(), then f(Q) is either a region or a point This important property of holomorphic functions will be proved, in more detailed form, in Theorem 10.32. 10.29 Lemma $I f\in H(\Omega)$ and g is defined in $\scriptstyle\mathbf{a}\times\mathbf{a}$ by $$ g(z,\,w)=\sqrt{\frac{f(z)-f(w)}{z-w}}\qquad{\mathrm{if~w\neqz,}} $$ then $\scriptstyle{\mathcal{G}}$ is continuous in ${\mathfrak{a}}\times\Omega$ELEMENTARY PROPER TIES OF HOLOMORPHIC FUNCTIONS 215 PRoOF The only points $(z,\,w)\in\Omega\times\Omega$ at which the continuity of g is poss- ibly in doubt have $z=w$ Fix ae Q. Fixe> 0. There exists $\zeta\in D(a;r).$ If z and w are in $D(a;r)$ and if $\scriptstyle\eta\in\Omega$ and $\scriptstyle\gamma\simeq0$ such that D(a; lf() -f(a)|<e for al $$ \zeta(t)=(1-t)z+t w, $$ then K(t)∈ D(a; r) for ${\mathfrak{o}}\leq t\leq1,$ and $$ g(z,\,w)-g(a,\,a)=\prod_{0}^{1}[f^{\prime}(\zeta(t))-f^{\prime}(a)]\,\,d t. $$ The absolute value of the integrand is <6, for every t, Thus l g(z,w)- g(a,a|<e.This proves that $\scriptstyle{\mathcal{G}}$ is continuous at (a,a) // 10.30 Theorem Suppose $\varphi\in H(\Omega),$ $z_{n}\in\Omega,$ and $\varphi(z_{0})\neq0$ .Then Q contains a neighborhood ${\mathbf{}}V$ of $\mathbb{Z}_{0}$ such tha (a)p is one-to-one in ${\mathit{V}},$ (b) $W=\varphi(V)$ is an open set, and (c)ü业: $W\to V$ is defined by $\psi(\varphi(z))=z,t h e n\;\psi\in H(W).$ Thus $\varphi\colon V\to W$ has a holomorphic inverse neighborhood ${\mathbf{}}V$ PRoOF Lemma 10.29,applied to $\varphi$ in place of ${\boldsymbol{f}},$ shows that $\Omega$ contains a of $\mathbf{z_{0}}$ such that $$ |\,\varphi(z_{1})-\varphi(z_{2})|\,\geq{\textstyle{\frac{1}{2}}}\,|\,\varphi^{\prime}(z_{0})\,|\,|\,z_{1}-z_{2}\,| $$ (1) if $z_{1}$ e ${\mathbf{}}V$ and $\mathbb{Z}_{2}$ e V. Thus (a) holds, and also $$ \varphi^{\prime}(z)\neq0\qquad(z\in V). $$ (2) there exists To prove (b), pick a eV and choose $\scriptstyle\gamma\simeq0$ so that ${\tilde{D}}(a,r)<V.$ By (1) $\scriptstyle c\;s\;\backslash$ such that $$ |\,\varphi(a+r e^{i\theta})-\varphi(a)\,|>2c\qquad(-\,\pi\leq\theta\leq\pi). $$ (3) If Ae D(qp(a); c), then $|\lambda-\varphi(a)|<c,$ hence (3) implies $$ \operatorname*{min}_{\ln}|\,\lambda-\varphi(a+r e^{i\theta})|>c. $$ (4) By the corollary to Theorem 10.24 $\lambda-\varphi$ must therefore have a zero in $D(a;\,r)$ Thus $\lambda=\varphi(z)$ for some $z\in D(a;\,r)\subset V.$ Since $\bar{a}$ was an arbitrary point of ${\mathit{V}},$ $\varphi(V)$ This proves that $D(\varphi(a);\,c)\subset\varphi(V),$ for a unique $z_{1}\in V.$ If we W is open. $w_{1}\in W.$ Then $\varphi(z_{1})=w_{1}$ To prove (c), fix and $\psi(w)=z\in V,$ we have $$ \frac{\psi(w)-\psi(w_{1})}{w-w_{1}}=\frac{z-z_{1}}{\varphi(z)-\varphi(z_{1})}. $$ (5)216 REAL AND cOMPLEX ANALYSIS ${\mathrm{By~}}\;(1),\;z\to z_{1}$ when w→W. Hence (2) implies that $\psi^{\prime}(w_{1})=1/\varphi^{\prime}(z_{1}).$ Thus $\psi\in H(W).$ // 10.31 Definition For $m=1,$ 2,3,.…, we denote the“ $m^{\mathrm{th}}$ power function” then $z\longrightarrow z^{m}\left|99\right.\pi_{m}$ is $\pi_{m}(z)$ for precisely ${\mathfrak{m}}\,$ distinct values of z: 1 ${\textsf{f}}w=r e^{i\theta},\,r>0,$ Each $w\neq0$ $\pi_{m}(z)=w$ if and only if $z\,=\,r^{1/m}e^{i(\theta+2k\pi)/m}$ $k=1,\dots,m.$ is open and does not Note also that each ${\mathcal{T}}_{\mathfrak{m}}$ is an open mapping: If ${\mathbf{}}V$ contain O,then $\pi_{m}(V)$ is open by Theorem 10.30.On the other hand, $\pi_{m}(D(0;\,r))=D(0;\,r^{m}$ is open, by Theorem 10.30, if ${\boldsymbol{\varphi}}^{\prime}$ Compositions of open mappings are clearly open. In particular $\pi_{m}\circ\varphi$ has no zero. The following theorem(which contains the more detailed version of the open mapping theorem that was mentioned prior to Lemma 10.29) states a converse: Every nonconstant holo- additive constant morphic function in a region is locally of the form $\pi_{m}\ \circ\ \varphi,$ except for an $w_{0}=f(z_{0}).$ Then there exists a neighborhood ${\mathbf{}}V$ is a region, fe H(Q2) f is not constant $z_{0},\ V<\Omega,$ and there exists $\mathbb{Z}_{\mathrm{O}}$ and 10.32 Theorem Suppose $\Omega$ $\scriptstyle z_{\mathrm{e}}\in\Omega,$ Let m be the order of the zero which the function $f-w_{0}$ has at of $varphi\in H(V),$ such that (a) $f(z)=w_{0}+\lceil\varphi(z)\rceil^{n}$ n for all $z\in V,$ onto a disc ${\mathfrak{p}}(t)$ ; r). (b)p’ has no zero in ${\mathbf{}}V$ and $\boldsymbol{\varphi}$ is an invertible mapping of ${\mathbf{}}V$ Thus $f-w_{0_{-}}=\pi_{m}$ 。p in V. It follows that $\boldsymbol{\f}$ is an exactly m-to-1 mapping of $V-\{z_{0}\}$ onto $D^{\prime}(w_{0};\,r^{m}),$ and that each wo∈ f(Q) is an interior point of f(Q) Hence f(Q) is open. borhood of $\mathrm{z}_{\mathrm{0}}$ PRoOF Without loss of generality we may assume that $\mathrm{that}\,f(z)\neq w_{0}$ if $z\in\Omega-\{z_{0}\}.$ is a convex neigh- which is so small $\Omega$ Then $$ f(z)-w_{0}=(z-z_{0})^{m}g(z)\qquad(z\in\Omega) $$ (1) ${\boldsymbol{h}}$ for some $g\in H(\Omega)$ which has no zero in Q. Hence $g/g\in H(\Omega).$ By Theorem 10.14, $g^{\prime}/g=h^{\prime}$ for some $h\in H(\Omega).$ The derivative of g·exp $\scriptstyle(-b)$ is ( $\mathbf{0}$ in Q.If is modified by the addition of a suitable constant, it follows that $g=\exp\,(h).$ Define $$ \varphi(z)=(z-z_{0})\exp{\frac{h(z)}{m}}\qquad(z\in\Omega). $$ (2) Then (a) holds, for al $\varepsilon\in\Omega$ Also, $\sigma(z_{0})=0$ and $\varphi^{\prime}(z_{0})\neq0$ The existence of an open set ${\mathbf{}}V$ that satisfies (b) follows now from Theorem 10.30. This completes the proof. /ELEMENTARY PROPER TIES OF HOLoMORPHIC FUNCrioNs 217 The next theorem is really contained in the preceding results, but it seems advisable to state it explicitly. 10.33 Theorem Suppose $\Omega$ is a region,fe H(Q), and fis one-to-one in Q. Then f′(z) ≠ 0 for every z ∈ Q, and the inverse off is holomorphic. PRooF If $f^{\prime}(z_{0})$ were O for some $z_{i}\in\Omega_{i}$ the hypotheses of Theorem 10.32 would hold with some $m>1.$ so that J $\boldsymbol{\f}$ would be m-to-1 in some deleted neighborhood of z $Z_{0}\,.$ o, Now apply part (c) of Theorem 10.30 // Note that the converse of Theorem 10.33 is false: If $f(z)=e^{z},$ then $f^{\prime}(z)\neq0$ for every z, but fis not one-to-one in the whole complex plane The Global Cauchy Theorem Before we state and prove this theorem, which will remove the restriction to convex regions that was imposed in Theorem 10.14, it will be convenient to add a little to the integration apparatus which was sufficient up to now. Essentially, it is a matter of no longer restricting ourselves to integrals over single paths, but to consider finite“sums”of paths instead. A simple instance of this occurred already in Sec. 10.9(c) the formula 10.34 Chains and Cycles Suppose $\gamma_{1},\,\cdot\cdot\cdot,\,\gamma_{n}$ are paths in the plane, and put $\scriptstyle{\bar{\kappa}}=$ yf U U y*. Each $\gamma_{i}$ induces a linear functional ${\tilde{\gamma}}_{i}$ ; on the vector space $C(K),$ by $$ \tilde{\gamma}_{i}(f)= \{_{\gamma_{i}}^{}f(z)\,d z. $$ (1) Define $$ \widetilde\Gamma=\widetilde\gamma_{1}+\cdot\cdot\cdot+\widetilde\gamma_{n}. $$ (2) Explicitly $\tilde{\Gamma}(f)=\tilde{\gamma}_{1}(f)+\cdot\cdot\cdot+\tilde{\gamma}_{n}(f)$ for all $\boldsymbol{\f}$ e C(K). The relation (2) suggests that we introduce a “formal sum $$ \Gamma=\gamma_{1}\div\cdots*\gamma_{n} $$ (3) and define $$ \T[F(z)~d z=\hat{\Gamma}(f). $$ (4) Then (3) is merely an abbreviation for the statement $$ \left\vert_{\Gamma}f(z)\;d z=\sum_{i=1}^{n}\; \vert_{\eta^{\prime}}f(z)\;d z\qquad(f\in C(K)).\right\vert\; $$ (5) Note that (5) serves as the definition of its left side.218 REAL AND cOMPLEX ANALYSIS The objects T so defined are called chains. If each $\gamma_{j}$ in (3) is a closed path then $\boldsymbol{\Gamma}$ is called a cycle. If each $\gamma_{j}$ in (3) is a path in some open set Q, we say that T is a chain in $\Omega.$ If (3) holds, we define $$ \Gamma^{*}=\gamma_{1}^{*}\ 0\ ^{\mathrm{~}\cdot\cdot\cdot}\cup\gamma_{n}^{*}. $$ (6) If ${\boldsymbol{\Gamma}}$ is a cycle and α生T*, we define the index of α with respect to T by $$ \operatorname{Ind}_{\Gamma}\,(\alpha)={\frac{1}{2\pi i}}\prod_{\Gamma}{\frac{d z}{z-\alpha}}, $$ (7) just as in Theorem 10.10. Obviously,(3) implies $$ \mathrm{Ind}_{\mathrm{F}}\left(x\right)=\sum_{i=1}^{n}\mathrm{Ind}_{\gamma_{i}}\left(x\right). $$ (8) If each chain will be denoted by $\mathrm{T}{\mathrm{;}}$ Then in (3) is replaced by its opposite path (see Sec. 10.8) the resulting $\gamma_{i}$ $$ \bigcap_{-\Gamma}f(z)\;d z=-\left\lceil_{\Gamma}f(z)\;d z\qquad(f\in C(\Gamma^{\ast})).\qquad\right. $$ (9) In particular ${\mathrm{Ind}}_{--}\left({\boldsymbol{x}}\right)=-\operatorname{Ind}_{\Gamma}\left({\mathsf{x}}\right)$ αc) if ${\Gamma}$ is a cycle and α生T* Chains can be added and subtracted in the obvious way, by adding or sub tracting the corresponding functionals: The statement $\Gamma=\Gamma_{1}+\Gamma_{2}$ means $$ \left|\L_{T}^{\star}f(z)\ d z=\right|_{\Gamma_{1}}f(z)\ d z+\bigg|_{\Gamma_{2}}^{\ast}f(z)\ d z $$ (10) for every f∈ C(Tf U T艺) Finally, note that a chain may be represented as a sum of paths in many ways. To say that $$ \gamma_{1}\stackrel{\cdot}{+}\cdot\cdot\cdot\cdot\stackrel\cdot\gamma_{n}=\delta_{1}\stackrel{\cdot}{+}\cdot\cdot\cdot\stackrel\cdot4}\end{array}\delta_{k} $$ means simply that $$ \sum_{i}\left(\sum_{j\atop i}f(z)\ d z=\sum_{j}\right)_{\delta j}f(z)\ d z $$ for every f hat is continuous on yr U y; U 6* U… 6t. In particular, a cycle may very well be represented as a sum of paths that are not closed. 10.35 Cauchy's Theorem Suppose fe H(Q2), where $\Omega$ is an arbitrary open set in the complex plane. If T is a cycle in $\Omega$ that satisfies $$ \mathrm{Ind_{F}\left(x\right)=0}\;\;\;\;\;\;\;f o r\;e v e r y\propto n o t\;i n\;5 $$ 2, (1)ELEMENTARY PROPER TIES OF HOLOMORPHIIC FUNCTIONs 219 then $$ f(z)\cdot\mathrm{Ind}_{\Gamma}\left(z\right)=\frac{1}{2\pi i}\int_{\Gamma}\frac{f(w)}{w-z}\,d w\qquad f o r\ z\in\Omega-\Gamma^{*} $$ (2) and $$ \{f(z)\ d z=0. $$ (3) If Toand ${\Gamma_{1}}$ are cycles in $\Omega$ such that $$ \mathrm{Ind_{T_{0}}\left(x\right)=I n d_{F_{1}}\left(x\right)}\ \ \ \ \,f o r\ e v e r y\ \alpha\ n o t\ i n\ ! $$ 92, (4) then $$ \bigcap_{\Gamma_{0}}f(z)\ d z=\bigcap_{\Gamma_{1}}f(z)\ d z. $$ (5) PROOF The function $\mathbf{\Omega}^{g}$ defined in $\scriptstyle\mathbf{a}\times\mathbf{a}$ by $$ g(z,\,w)=\sqrt{\frac{f(w)-f(z)}{w-z}}\qquad\mathrm{if}\,w\neq z, $$ (6) is continuous in $\mathbf{a}\times\mathbf{a}$ (Lemma 10.29). Hence we can define $$ h(z)={\frac{1}{2\pi i}}\left.\right|_{\mathrm{{F}}}^{\circ}g(z,\,w)\,d w\qquad(z\in\Omega). $$ (7) For $\varepsilon\in\Omega-\Gamma^{*}$ the Cauchy formula (2) is clearly equivalent to the assertion that $$ h(z)=0. $$ (8) To prove (8),let us first prove that $h\in H(\Omega).$ Note that $\mathbf{\Omega}^{g}$ is uniformly it continuous on every compact subset of $\Omega\times\Omega,\;\Pi,z\in\Omega,z_{n}\in\Omega,$ and $z_{n} arrow z,$ Hence $h(z_{n})\to h(z).$ This proves that ${\boldsymbol{h}}$ h is continuous in $\Omega.$ Let (a compact subset of Q) be a closed follows that $g(z_{n},\,w) arrow g(z,\,w)$ uniformly for $w\in\Gamma^{*}$ $\underline{{\land}}$ triangle in Q2. Then $$ \left(\begin{array}{c}{{\Rightarrow}}\\ {{a}}\\ {{a}}\end{array}\right)_{\k}=\frac{1}{2\pi i}\left\{\frac{1}{\textstyle (\int_{\delta\Delta}}g(z,\,w)\ d z\right)d w. $$ (9)) For each $w\in\Omega,\;z\to g(z)$ w) is holomorphic in $\Omega.$ (The singularity at $\mathbf{0}$ for every is $z=w$ removable.) The inner integral on the right side of (9) is therefore w ∈ $\Gamma^{\mathbb{N}}.$ Morera's theorem shows now that $h\in H(\Omega).$220 REAL AND coMPLEX ANALYSis Next, we let $\Omega_{1}$ be the set of all complex numbers z for which Ind, $\scriptstyle x\;=\;x$ 0, and we define $$ h_{1}(z)={\frac{1}{2\pi i}} [\frac{f(w)}{w-z}\,d w\qquad(z\in\Omega_{1}). $$ (10) If there is a function $\varphi\in H(\Omega\cup\Omega_{1})$ whose restriction to $\Omega$ $h_{1}(z)=h(z).$ Hence $\varepsilon\in\Omega\cap\Omega_{\cdots}$ the definition of $\Omega_{1}$ makes it clear that is h and whose restriction to $\Omega_{1}$ is $h_{1}.$ $\mathbf{(1)}$ shows that $\Omega_{1}$ contains the complement of 2. Thus qp Our hypothesis is an entire function. $\Omega_{1}$ also contains the unbounded component of the com plement of T*, since Indp $(z)\ {\mathrm{is~}}0$ there. Hence $$ \operatorname*{lim}_{\vert z\vert arrow\infty}\varphi(z)=\operatorname*{lim}_{\vert z\vert arrow\infty}h_{1}(z)=0. $$ (11) Liouville's theorem implies now that $\varphi(z)=0$ for every z. This proves (8), and hence (2) Then To deduce (3) from (2), pick $a\in\Omega-\Gamma^{*}$ and define $F(z)=(z-a)f(z).$ $$ {\frac{1}{2\pi i}}\prod_{\Gamma}^{}f(z)\ d z={\frac{1}{2\pi i}}\left[{\frac{F(z)}{z-a}}\ d z=F(a)\cdot{\mathrm{Ind}}_{\Gamma} (a\right)=0, $$ (12)) because $F(a)=0.$ $\Gamma=\Gamma_{1}-\Gamma_{0}.$ // Finally,,(5)) follows from (4) if (3) is applied to the cycle This completes the proof. 10.36 Remarks (a) If y is a closed path in a convex region $\Omega$ 2 and if $\scriptstyle*\epsilon\equiv\Omega.$ an application of Theorem 10.14 to $f(z)=(z-\alpha)^{-1}$ shows that Ind, $\scriptstyle(n)\;=\;0$ Hypothesis $\Omega$ is (1) of Theorem 10.35 is therefore satisfied by every cycle in $\Omega$ if convex. This shows that Theorem 10.35 generalizes Theorems 10.14 and 10.15. (b)The last part of Theorem 10.35 shows under what circumstances integra- tion over one cycle can be replaced by integration over another, without rounds $D_{i}$ changing the value of the integral. For example, let removed. If T, $\gamma_{1},$ Y2,ys are postivel ; sur- $\Omega$ be the plane with three disjoint closed discs $D_{i}$ surrounds $D_{1}\sim D_{2}\sim D_{3}$ and $\gamma_{i}$ oriented circles in $\Omega$ such that ${\Gamma}$ but not $D_{j}\mathrm{for}\,j\neq i,$ then $$ \left\vert_{T}^{*}f(z)\;d z=\sum_{i=1}^{3}\;\right\vert_{\gamma_{i}}^{*}f(z)\;d z $$ for every fe H(Q2) (c)In order to apply Theorem 10.35, it is desirable to have a reasonably effcient method of finding the index of a point with respect to a closed path. The following theorem does this for all paths that occur in practiceELEMENTARY PROPERTIES OF HOLoMORPHIC FUNCTIONs 221 It says, essentially, that the index increases by l when the path is crossed “ from right to left.”If we recall that Ind, $\scriptstyle(n)\;=\;0$ if $\scriptstyle{\dot{\mathbf{x}}}$ is in the unbounded has only component of the complement $\textstyle W$ of y $\textstyle{\int}^{\frac{1\beta\hbar}{\hbar}},$ y*, we can then successively deter- mine Ind,(α) in the other components of $W,$ provided that $\textstyle W$ finitely many components and that y traverses no arc more than once. 10.37 Theorem Suppose $\scriptstyle{\mathcal{Y}}$ is a closed path in the plane,with parameter interva [α,β]. Suppose d $x<u<v<\beta,a$ and $\boldsymbol{\ b}$ are complex numbers, $|b|=r>0,$ and (ii) IP(S) (i))v(u) = a-b, $\gamma(v)=a+b,$ $\mathbf{\nabla}u<s<v,$ or $s=v.$ (i) Ir(S) $-\,a|<r$ if and only i $s=u$ $-\ a|=r$ if and only i ${\mathfrak{p}}_{\sim}$ Assume furthermore that ${\cal D}(a;r)-\gamma^{*}$ is the union of two regions, $D_{+}$ and labeled so that $a+b i\in{\bar{D}}_{+}$ and $a-b i\in{\bar{D}}_{-}$ .Then $$ {\mathrm{Ind}}_{\gamma}\left(z\right)=1+{\mathrm{Ind}}_{\gamma}\left(w\right) $$ if $x\in D_{*}$ and $w\in D_{-}$ As $\gamma(t)$ traverses $D(a;\,r)$ from $a-b$ to a + b, $D_{-}$ is “on the right”and $D_{+}$ is“on the left”of the path. PRoOF To simplify the writing, reparametrize y so that $u=0$ and $v=\pi.$ Define Since not intersect which does not contain $\left.\zeta\right.$ $$ \begin{array}{l l}{{C(s)=a-b e^{i s}\ \ \ }}&{{(0\leq s\leq2\pi)}}\\ {{f(s)=\left\{C(s)\ \ \ \ \ \ (0\leq s\leq\pi)}}&{{(\pi\leq s\leq\pi)}}\\ {{g(s)= \{\gamma(s)\ \ \ \ \ \ \ (\pi\leq s\leq2\pi)}}&{{(\pi\leq s\leq\pi)}}\\ {{\langle0\leq s\leq\pi\right\}}}&{{(0\leq s\leq\pi).}}\end{array} $$ and 1 ${\boldsymbol{h}}$ are closed paths s ≤D) 2r) If $\gamma(0)=_{-}C(0)$ and $\gamma(\pi)=C(\pi),f,g,$ and $\zeta\notin E,$ then $\boldsymbol{E}$ lies in the disc $D(2a-\zeta;$ does ${\mathfrak{g}}^{\mathfrak{s}},$ $E\subset D(a;\,r),\,|\zeta-a|=r,$ Apply this to E $={g^{\star},\ \zeta=a-b i,$ to see [from Remark 10.36(a)] tha ${\mathrm{Ind}}_{g_{c}}(a-b i)=0.$ Since ${\tilde{D}}_{-}$ is connected and $D_{-}$ it follows that $$ \mathrm{Ind}_{g}\left(w\right)=0\qquad i f\,w\in D_{-}. $$ (1) The same reasoning shows that $$ [\mathrm{Ind}_{f}\left(z\right)=0\qquad i f\,z\in D_{+}\,. $$ (2)$222$ REAL AND coMPLEX ANALYSIS We conclude that $$ \begin{array}{l l}{{\mathrm{Ind}_{\gamma}\left(z\right)=\mathrm{Ind}_{h}\left(x\right)=\mathrm{Ind}_{h}\left(w\right)}}\\ {{}}&{{=\mathrm{Ind}_{c}\left(w\right)+\mathrm{Ind}_{\gamma}\left(w\right)=1+\mathrm{Ind}_{\gamma}\left(w_{.}\right.}}\end{array} $$ J. The first of these equalities follows from (2) since $h=\gamma\div f.$ The second holds because $\mathbb{Z}$ z and w lie in $D(a;r),$ a connected set which does not intersect $h^{\oplus}.$ The third follows from(1), since $h\div g=C\div\gamma,$ and the fourth is a conse- // quence of Theorem 10.11. This completes the proof. We now turn to a brief discussion of another topological concept that is relevant to Cauchy's theorem. 10.38 Homotopy Suppose $\gamma_{0}$ and $\gamma_{1}$ are closed curves in a topological space are $X\cdotp$ X- -homotopic if $X,$ both with parameter interval $I=[0,1\}$ We say that $\gamma_{0}$ , and $\gamma_{1}$ into $X$ such that there is a continuous mapping ${\boldsymbol{H}}$ of the unit square $I^{2}=I\times I$ $$ H(s,\,0)=\gamma_{0}(s),\qquad H(s,\,1)=\gamma_{1}(s),\qquad H(0,\,t)=H(1,\,t)\,. $$ 女) (1) for all s ∈ ${\mathbf I}$ and $\epsilon,t.$ Put $\gamma_{t}(s)=H(s,$ $t\rangle,$ Then (1) defines a one-parameter famil $\gamma_{0}$ If of closed curves is $X\cdotp$ -homotopic to a constant mapping $\gamma_{1},$ within $\gamma_{0}$ and $\gamma_{1}.$ Y. Intuitively, this means that $\gamma_{t}\;i n\;X,$ which connects can be continuously deformed to $X.$ $\gamma_{0}$ point), we say that $\gamma_{0}$ is null-homotopic in $X.$ If $\gamma_{1}$ (i.e., if yft consists of just one $X$ is connected and if every closed curve in $X$ is null-homotopic, $X$ is said to be simply connected. a closed curve in $\Omega.$ For example, every convex region $\Omega$ is simply connected. To see this, let $\gamma_{\mathrm{O}}$ be fix $\scriptstyle z_{1}\in\Omega,$ and define $$ H(s,\,t)=(1-t)\gamma_{0}(s)+t z_{1}\qquad(0\leq s\leq1,\quad0\leq t\leq1). $$ (2) Theorem 10.40 will show that condition (4) of Cauchy's theorem 10.35 holds whenever $\Gamma_{0}$ and ${\Gamma_{1}}$ i are $\Omega\cdot$ -homotopic closed paths. As a special case of this, not in $\Omega$ if Q is simply that condition (1) of Theorem 10.35 holds for every closed path $\Gamma$ connected. 10.39 Lemma If yo and $\gamma_{1}$ Y are closed paths with parameter interval [0,1], if α is a complex number, and i $$ |\gamma_{1}(s)-\gamma_{0}(s)|<|\alpha-\gamma_{0}(s)|\;\;\;\;\;\;\;(0\leq s\leq1) $$ (1) then Ind,, $\mathbf{(x)}=\operatorname{Ind}_{\gamma0}$ (α). define $\gamma=(\gamma_{1}-\alpha)/(\gamma_{0}-\alpha).$ Then PRoor Note first that(1) implies that α生y8 and α≠ yf. Hence one can $$ {\frac{\gamma^{\prime}}{\gamma}}={\frac{\gamma_{1}^{\prime}}{\gamma_{1}-\alpha}}-{\frac{\gamma_{0}^{\prime}}{\gamma_{0}-\alpha}} $$ (2)ELEMENTARY PROPER TIES OF HOLOMORPHIC FUNCTIONS 223 and $|1-\gamma|<1,$ by (1). Hence $\nu^{*}<D(1);$ 1), which implies that Ind, $\scriptstyle0\,n\;=\;0$ // Integration of (2) over [0,1] now gives the desired result. 10.40 Theorem $I f\Gamma_{0}$ and ${\Gamma}_{1}$ are $\Omega{\mathrm{:}}$ -homotopic closed paths in a region $\Omega,$ and $i{\boldsymbol{f}}$ α 生 Q, then $$ \mathrm{Ind}_{\Gamma_{1}}\left(x\right)=\mathrm{Ind}_{\Gamma_{0}}\left(x\right)}, $$ (1) PRoOF By definition, there is a continuous $H:I^{2}\to\Omega$ such that $$ H(s,0)=\Gamma_{0}(s),\qquad H(s,1)=\Gamma_{1}(s),\qquad H(0,\,t)=H(1,\,t). $$ (2) Since ${\boldsymbol{J}}^{2}$ is compact, so is $\scriptstyle{H(t^{2})}$ Hence there exists $\scriptstyle x\;{\overset{\underset{\mathrm{a}}{}}}\;0$ such that $$ |\alpha-H(s,t)|>2\epsilon\qquad{\mathrm{if}}\qquad(s,t)\in I^{2}. $$ (3 Since ${\boldsymbol{H}}$ is uniformly continuous, there is a positive integer ${\mathbf{}}n$ such tha $$ \begin{array}{c c c}{{|H(s,t)-H(s^{\prime},t^{\prime})|<\epsilon}}&{{\mathrm{~if~}}}&{{\mathrm{~is-s^{\prime}|+|t-t^{\prime}|\leq1/n.}}}\end{array} $$ (4) Define polygonal closed paths $\gamma_{0},\cdot\cdot\cdot,\gamma_{n}$ by $$ \gamma_{k}(s)=H{\biggl(}{\frac{i}{n}},{\frac{k}{n}}{\biggr)}(n s+1-i)+H{\biggl(}{\frac{i-1}{n}},{\frac{k}{n}}{\biggr)}(i-n s) $$ (5) if i-1≤ ns ≤iand $i=1,\dots,n,\mathrm{{By}}\left(4\right)$ and (5) $$ |\gamma_{k}(s)-H(s,\,k/n)|<\epsilon\qquad(k=0,\,\ldots,\,n;\,0\leq s\leq1). $$ (6 In particular, taking $\scriptstyle k\;=\;0$ and $k=n,$ $$ |\gamma_{0}(s)-\Gamma_{0}(s)|<\epsilon,\qquad|\gamma_{n}(s)-\Gamma_{1}(s)|<\epsilon. $$ (7T By (6) and (3) $$ |\alpha-\gamma_{k}(s)|>\epsilon~~~~~(k=0,\ldots,n;0\leq s\leq1). $$ (8) On the other hand,(4) and (5) also give $$ |\gamma_{k-1}(s)-\gamma_{k}(s)|<\epsilon\qquad(k=1,\dots,n;0\le s\le1). $$ (9) Now it follows from (7),(8),(9), and $n+2$ applications of Lemma 10.39 // T. This proves the theorem. that has te sameinexwi Tret t c tepatsTF。.To,… Note: If $\Gamma_{i}(s)=H(s,t)$ in the preceding proof, then each $\textstyle\Gamma_{t}$ is a closed curve, but not necessarily a path, since ${\boldsymbol{H}}$ is not assumed to be differentiable. The paths $\gamma_{k}$ were introduced for this reason. Another (and perhaps more satisfactory) way to circumvent this difficulty is to extend the definition of index to closed curves. This is sketched in Exercise 28.224 REAL AND coMPLEX ANALYSIs The Calculus of Residues 10.41 Definition A function f is said to be meromorphic in an open set $\Omega$ if there is a set $\scriptstyle4\,=\,\Omega$ such that (a)A has no limit point in $\Omega,$ (b)fe H(Q2 - A) $A.$ (c)f has a pole at each point of Note that the possibility $A={\mathcal{O}}$ is not excluded. Thus every $f\in H(\Omega)$ is meromorphic in $\Omega.$ Note also that (a) implies that no compact subset of $\Omega$ contains infinitely many points of $A,$ and that $\scriptstyle A$ is therefore at most countable. If f and $\scriptstyle A$ are as above, if $a\in A,$ and if $$ Q(z)=\sum_{k=1}^{m}c_{k}(z-a)^{-k} $$ (1) is the principal part of f at a, as defined in Theorem 10.21 (.e., if f - Q has a removable singularity at a), then the number $c_{\mathrm{I}}$ is called the residue off at ${\boldsymbol{a}}\colon$ $$ c_{1}=\mathrm{Res}\ (f;\,a). $$ (2) If ${\Gamma}$ is a cycle and a生 $\Gamma^{*},(1)$ implies $$ {\frac{1}{2\pi i}}\prod_{\Gamma}^{}Q(z)\ d z=c_{1}\mathrm{~Ind_{F}~}(a)=\mathrm{Res~}(Q;\,a)\mathrm{~Ind_{F}~}(a). $$ (3) This very special case of the following theorem will be used in its proof 10.42 The Residue Theorem Suppose f is a meromorphic function in $\Omega.$ Let A be the set of points in $\Omega$ at whichf has poles. ${\boldsymbol{U}}{\boldsymbol{T}}{\boldsymbol{\Gamma}}$ is a cycle in $\alpha-A$ such that $$ \mathrm{Ind}_{\mathrm{F}}\left(x\right)=0\qquad f o r\ a l l\qquad\alpha\not\equiv\Omega, $$ (1) then $$ {\frac{1}{2\pi i}}\left|_{\mathrm{{T}}}f(z)\,\,d z=\sum_{a\,\in A}\mathrm{{Res}}\,\,(f;\,a)\,\,{\mathrm{In}}\atop{\mathrm{d}}c\,{\mathrm{(}}a\right.\mathrm{(}a\mathrm{)}. $$ (2) PR0OF Let $B=\{a\in A\colon\operatorname{Ind}_{\Gamma}(a)\neq0\}.$ Let ${\boldsymbol{W}}$ be the complement of Since $\scriptstyle A$ has no ${\mathbf{}}V$ $\Gamma^{\ast\ast}$ . Then intersects $\Omega^{\mathrm{c}}.$ Ind- (z) is constant in each component ${\mathbf{}}V$ of ${\boldsymbol{W}}\cdot\mathbf{I}{\boldsymbol{V}}$ is unbounded, or if 2,(1) implies that Ind, $\scriptstyle(t_{i})\;=\;0$ for every $z\in V.$ limit point in $\Omega.$ we conclude that $\boldsymbol{B}$ is a finite set. The sum in (2), though formally infinite, is therefore actually finite at Let $a_{1},\,\ldots,\,a_{n}$ be the points of Put $\Omega_{0}=\Omega-(A-B)$ (If $B=\mathbb{Q},$ a possibility which $a_{1},\,\ldots,\,a_{n},$ and put $g=f-(Q_{1}+\cdots+Q_{n}).$ Since $\scriptstyle{\mathcal{G}}$ be the principal parts of f ${\boldsymbol{B}},$ let $Q_{1},\ldots,Q_{n}$ is not excluded, then $\scriptstyle g=f_{j}$ has removableELEMENTARY PROPERTIEs OF HOLoMORPHIC FUNCTIONs 225 singularities at $a_{1},\,\cdot\cdot\cdot,\,a_{n},$ Theorem 10.35,applied to the function $\mathbf{\Omega}^{g}$ g and the open set $\Omega_{0}\,,$ shows that $$ [g(z)~d z=0. $$ (3) Hence $$ {\frac{1}{2\pi i}}\prod_{\Gamma}^{*}\!f(z)\,d z=\sum_{i=1}^{n}{\frac{1}{2\pi i}}\int_{\Gamma}\!Q_{k}(z)\,d z=\sum_{k=1}^{n}\,\operatorname{Res}\,{(Q_{k};\,a_{k})}\,\operatorname{Ind}_{\Gamma}{(a_{k})}, $$ and sincef and $Q_{k}$ have the same residue at ${\boldsymbol{a}}_{k}\,,$ we obtain (2) // We conclude this chapter with two typical applications of the residue theorem. The first one concerns zeros of holomorphic functions, the second is the evaluation of a certain integral. 10.43 Theorem Suppose $\scriptstyle{\mathcal{Y}}$ is a closed path in a region $\Omega,$ 2, such that Ind, $\scriptstyle(v)\;=\;0$ and let C for every α not in $\Omega.$ Suppose also that Ind ${\mathfrak{t}}_{\nu}(x)=0$ or 1 for every $x\in\Omega-\gamma^{*}$ $\Omega_{1}$ be the set of all with let I $N_{f}$ be the number of zeros of f in $\Omega_{1},$ counted accord- For any $\operatorname{Ind}_{\gamma}(x)=1.$ $f\in H(\Omega)$ ing to their multiplicities. (a)ff e H(Q2) and f has no zeros on $\gamma^{\ast}$ then $$ N_{f}={\frac{1}{2\pi i}}\left(\frac{f^{\prime}(z)}{f(z)}\,d z=\mathrm{Ind_{F}} (0\right) $$ (1) where $\Gamma=f\circ$ P· and (b)f also $g\in H(\Omega)$ $$ |f(z)-g(z)|<|f(z)|~~~~~~f o r~a l l~z\in\gamma^{*} $$ (2) then N。 = N Part $\mathbf{(}b\mathbf{)}$ is usually called Rouché's theorem. It says that two holomorphic if they are close together on the functions have the same number of zeros in $\Omega_{1}$ boundary of $\Omega_{1},$ as specified by (2) of order morphic in some neighborhood ${\mathbf{}}V$ a meromorphic function in $\Omega.$ If $a\in\Omega$ and $\boldsymbol{\f}$ has a zero PROOF Put $\varphi=f^{\prime}/f,$ at a, then $f(z)=(z-a)^{n}h(z),$ where ${\boldsymbol{h}}$ and $1/h$ are holo- $m=m(a)$ of ${\boldsymbol{a}}.$ In $V-\{a\},$ $$ \varphi(z)={\frac{f^{\prime}(z)}{f(z)}}={\frac{m}{z-a}}+{\frac{h^{\prime}(z)}{h(z)}}. $$ (3) Thus $$ {\mathrm{Res~}}(\varphi\,;\,a)=m(a). $$ (4)226 REAL AND CoMPLEX ANALYSIs Let $A=\{a\in\Omega_{1}:f(a)=0\}$ If our assumptions about the index of $\scriptstyle{\mathcal{Y}}$ y are combined with the residue theorem one obtains $$ {\frac{1}{2\pi i}} |\frac{f^{\prime}(z)}{f(z)}\,d z=\sum_{a\,\in\,A}\mathrm{Res}\ (\varphi\,;\,a)=\sum_{a\,\in\,A}m(a)=N_{f}\,. $$ This proves one half of (1). The other half is a matter of direct computation: $$ \mathrm{Ind_{F}\left(0\right)=}\frac{1}{2\pi i}\left|\!\begin{array}{c}{{d z}}\\ {{z}}\\ {{\bar{z}}}\end{array}\!\!=}&{{\!\!2\pi i}}\end{array}\!\right|_{0}^{z\pi}\frac{\Gamma^{\prime}(s)}{\Gamma\!\left(s\right)}\,d s $$ $$ ={\frac{1}{2\pi i}}\int_{0}^{2\pi}{\frac{f^{\prime}(\gamma(s))}{f(\gamma(s))}}\;\gamma^{\prime}(s)\;d s={\frac{1}{2\pi i}}\int_{\gamma}{\frac{f^{\prime}(z)}{f(z)}}\;d z. $$ The parameter interval of $\scriptstyle{\mathcal{Y}}$ was here taken to be [0,2元] of f. Put Next, (2) shows that $\scriptstyle{\mathcal{G}}$ has no zero on $\gamma^{\star}.$ Hence (1) holds with $\scriptstyle{\mathcal{G}}$ in place $\Gamma_{0}=g\circ\gamma.$ Then it follows from (1),(2), and Lemma 10.39 that $$ N_{g}=\mathrm{Ind}_{\Gamma_{0}}\left(0\right)=\mathrm{Ind}_{\Gamma}\left(0\right)=N_{f}. $$ // 10.44 Problem For real t, find the limit, as $A\to\varnothing.$ of $$ \bigcap_{-A}^{A}{\frac{\sin\,x}{x}}\,e^{i x t}\,d x. $$ (1) SOLUTION Since $z^{-1}$ · sin $z\cdot e^{i t z}$ is entire, its integral over [-A, A] equals that over the path $\Gamma_{A}$ obtained by going from -A to $-1$ along the real axis, along from -1 to l along the lower half of the unit circle, and from 1 to $\scriptstyle A$ the real axis. This follows from Cauchy's theorem. $\Gamma_{A}$ avoids the origin, and we may therefore use the identity $$ 2i\,\sin\,z=e^{i z}-e^{-i z} $$ to see that (1)equals $\varphi_{A}(t+1)-\varphi_{A}(t-1),$ where $$ \frac{1}{\pi}\;\varphi_{A}(s)=\frac{1}{2\pi i}\left.\right|_{\Gamma_{A}}\frac{e^{i s z}}{z}\;d z. $$ (2) Complete $\Gamma_{A}$ to a closed path in two ways: First, by the semicircle from A to - Ai to $-A\,;$ secondly, by the semicircle from A to $A i$ to -A. The function $e^{i s z}/z$ has a single pole, at $\scriptstyle z\ =0,$ where its residue is 1. It follows that $$ \frac{1}{\pi}\;\varphi_{A}(s)=\frac{1}{2\pi}\left.\right|_{-\pi}^{0}\exp\left(i s A e^{i\theta}\right)\,d\theta $$ (3) and $$ \frac{1}{\pi}\,\varphi_{A}(s)=1-\frac{1}{2\pi}\int_{0}^{\pi}\!\exp\,\left(i s A e^{i\theta}\right)\,d\theta. $$ (4)ELEMENTARY PROPER TIES OF HOLOMORPHIC FUNCTIONS $227$ Note that $$ |\exp\,(i s4e^{i\theta})|=\exp\,(-\,A s\,\sin\,\theta), $$ (5) and that this is $<1$ and tends to $\mathbf{0}$ ) as $A\to\infty$ if s and sin $\theta$ have the same in (3) tends to C $\mathbf{0}$ if $s<0,$ sign. The dominated convergence theorem shows therefore that the integral ) if $s>0.$ Thus and the one in (4) tends to $\mathbf{0}$ $$ \operatorname*{lim}_{A arrow\infty}\varphi_{A}(s)={\binom{\pi}{0}}\quad{\textrm{~i f~}}s>0, $$ (6) and if we apply (6) to $s=t+1$ and to $s=t-1,$ we get $$ \operatorname*{lim}_{A\to\infty}\bigg)_{-A}^{A}\ {\frac{\sin\,x}{x}}\,e^{i t x}\,d x={\left\{\pi\atop0}\,\,\,\,\,\,\,\,\,{\mathrm{~if~}}x\right\}t\vdash1<t<1, $$ (7) Since $\varphi_{A}(0)=\pi/2,$ the limit in (7) is $\pi/2$ when $t=\pm1$ // Note that (T) gives the Fourier transform of (sin x)/x. We leave it as an exer- cise to check the result against the inversion theorem Exercises 1 The following fact was tacitly used in this chapter: If $\scriptstyle A\quad\quad A$ and $\bar{\boldsymbol{B}}$ are disjoint subsets of the plane, if A is compact, and if $\bar{\boldsymbol{B}}$ is closed, then there exists a $\delta>0$ such that $|\alpha-\beta|\geq\delta$ for all $\alpha\in A$ and $\scriptstyle{\theta\cdot b}$ Prove this, with an arbitrary metric space in place of the plane. 2 Suppose that fis an entire function, and that in every power series $$ f(z)=\sum_{n=0}^{\infty}c_{n}(z-a)^{n} $$ at least one coefficient is O. Prove that fis a polynomial Hint: $n!c_{n}=f^{(n)}(a).$ 3 Suppose $\scriptstyle{\hat{f}}$ and $\scriptstyle{\mathcal{G}}$ are entire functions, and $|f(z)|\leq|g(z)|$ for every z. What conclusion can you draw? 4 Suppose fis an entire function, and $$ |f(z)|\leq A+B|z|^{k} $$ for all z, where $^{4,B}$ and $\boldsymbol{k}$ are positive numbers. Prove that f must be a polynomial converges for every $\scriptstyle{\varepsilon\circ\mathbf{u}}$ is a uniformly bounded sequence of holomorphic functions in $\underline{{\Omega}}$ such that {JAD) 5 Suppose $\{f_{n}\}$ Prove that the convergence is uniform on every compact subset of Q. Hint: Apply the dominated convergence theorem to the Cauchy formula for $f_{n}-f_{m}.$ log' 6 There is a region $\underline{{\Omega}}$ Find the coefficients $\scriptstyle a_{n}$ $\Omega)=D(1;$ $\mathbf{I}.$ Show that exp is one-to-one in $\Omega,$ but that there are many such $\Omega.$ that exp ,for $|z-1|<1,$ to be that $\scriptstyle{v\cdot u}$ for which $e^{w}=z.$ Prove that Fix one, and define log $\mathbb{Z}_{3}$ $(z)=1/z.$ , in $$ {\frac{1}{z}}=\sum_{n=0}^{\infty}a_{n}(z-1)^{n} $$228 REAL AND cOMPLEX ANALYSIS and hence find thecoefficients $c_{n}$ in the expansion $$ \log z=\sum_{n=0}^{\infty}c_{n}(z-1)^{n}. $$ In what other discs can this be done? 7 Iffe H(Q), the Cauchy formula for the derivatives $\operatorname{of}f,$ $$ f^{(n)}(z)=\frac{n!}{2\pi i}\int_{\Gamma}\frac{f(\zeta)}{(\zeta-z)^{n+1}}\,d\zeta\qquad(n=1,\,2,\,3,\,\ldots) $$ is valid under certain conditions on z and T. State these, and prove the formula 8 Suppose ${\mathbf{}}P$ and $\scriptstyle{\mathcal{Q}}$ are polynomials, the degree of Q $Q_{\mathbf{\delta}}Q$ Q exceeds that of ${\mathbf{}}P$ P by at least 2, and the rational is Zri times function $R=P/Q$ has no pole on the real axis Prove that the integral o ${\boldsymbol{R}}$ over $(-\infty,\,\infty)$ by one over a the sum of the residues of ${\boldsymbol{R}}$ in the upper half plane. [Replace the integral over $(-A,\,A)$ suitable semicircle, and apply the residue theorem.] What is the analogous statement for the lower half plane? Use this method to compute $$ |{\overline{{x}}}_{-\alpha}{\frac{x^{2}}{1+x^{4}}}\,d x. $$ 9 Compute $\textstyle\int_{-\infty}^{\infty}e^{i x}/(1+x^{2})\;d x$ for real $t_{\mathrm{,}}$ by the method described in Exercise 8.Check your answer against the inversion theorem for Fourier transforms. $10$ Let y be the positively oriented unit circle, and compute $$ {\frac{1}{2\pi i}} (\begin{array}{l l}{{e^{z}-e^{-z}}}\\ {{z^{4}}}\end{array}d z. $$ 1 Suppose cis a complex number, $|\alpha|\neq1,$ and compute $$ \left|\bigcup_{0}^{2\pi}{\frac{d\theta}{1-2\alpha\cos\theta+\alpha^{2}}}\right|=\left|\operatorname*{P}_{0}\right|^{2} $$ by integrating $(z-\alpha)^{-1}(z-1/\alpha)^{-1}$ over the unit circle $\scriptstyle12$ Compute $$ \left.\left(\cdots_{\alpha}^{*\alpha} (\frac{\sin\,x}x\right)^{2}e^{i k x}\;d x\qquad (\mathrm{for~real~}t\right), $$ 13 Compute $$ \=_{0}^{\infty}{\frac{d x}{1+x^{n}}}\qquad(n=2,\,3,\,4,\,\dots). $$ [For even ${\boldsymbol{n}}_{s}$ the method of Exercise 8 can be used. However, a different path can be chosen, which simplifies the computation and which also works for odd ${\boldsymbol{n}}\colon$ from O to ${\boldsymbol{R}}$ to R exp $(2\pi i/n)$ to 0.] Answer:(r/n)/sin (α/n) and $\Omega_{2}{\mathrm{:}}$ What if we know that $\scriptstyle{\mathcal{G}}$ g and ${\boldsymbol{h}}$ are plane regions, $\boldsymbol{\mathit{f}}$ and $\scriptstyle{\mathcal{G}}$ 门 are nonconstant complex functions defined in $\Omega_{1}$ 14 Suppose $\Omega_{1}$ and $\Omega_{2}$ $\scriptstyle f(\Omega_{1})\in\Omega_{2}$ Put $h=g\circ f.$ If f and $\scriptstyle{\mathcal{G}}$ are holomorphic, we know that his $g^{\prime{2}}$ respectively, and holomorphic. Suppose we know that $\boldsymbol{\mathit{f}}$ and $\boldsymbol{\mathit{h}}$ are holomorphic. Can we conclude anything about are holomorphic? 15 Suppose $\underline{{\Omega}}$ is a region, $\varphi\in H(\Omega),$ at $z_{0}{}^{*}$ at $w_{0\,{\mathrm{-}}}$ then $\scriptstyle{\mathcal{G}}$ also has a zero of order m a and $w_{0}=$ modified if ${\boldsymbol{\varphi}}^{\prime}$ p(Zo). Prove that if has a zero of order g’has no zero in $\Omega,f\in H(\varphi(\Omega)),$ 9 $=f\circ\varphi,\,z_{0}\in\Omega,$ ${\mathfrak{A}}\colon Z_{0}$ How is this ${\mathfrak{m}}\,$ ’ has a zero of order $\displaystyle{\boldsymbol{k}}$ELEMENTARY PROPER TIEs OF HOLoMORPHIc FUNCTIoNs 229 16 Suppose ${}^{\mu}$ is a complex measure on a measure space X, $\;\underline{{\Omega}}$ is an open set in the plane, gp is a bounded function on $\scriptstyle\mathbf{a}\cdot{\mathcal{X}}$ such that qp(z, ${\mathbf{}}t{\mathrm{}}$ is a measurable function of t,for each $\scriptstyle{\varepsilon\circ\Omega}$ and p(z, t) is holomorphic in $\Omega,$ for each $*\epsilon\,X.$ Define $$ f(z)=\int_{x}\varphi(z,t)\;d\mu(t) $$ $\scriptstyle{\mathrm{or~}}z\in\Omega$ Prove that f∈ $\scriptstyle m_{\mathrm{th}}$ Hint: Show that to every compact $\scriptstyle{K\equiv\Omega}$ there corresponds a constant $M<\infty$ such that $$ \left|\frac{\varphi(z,t)-\varphi(z_{0},t)}{z-z_{0}}\right|<M\qquad(z{\mathrm{~and~}}z_{0}\in K,t\in X). $$ 17 Determine the regions in which thefollowing functions are defined and holomorphic: $$ f(z)=\int_{0}^{1}{\frac{d t}{1+t z}},\qquad g(z)=\int_{0}^{\infty}{\frac{e^{t z}}{1+t^{2}}}\,d t,\qquad h(z)=\int_{-1}^{1}{\frac{e^{t z}}{1+t^{2}}}\,d t. $$ Hint: Either use Exercise 16, or combine Morera's theorem with Fubini's 18 Suppose fe H(Q), ${\tilde{D}}(a;r)<\Omega,$ y is the positively oriented circle with center at a and radius r, and f has no zero on $\gamma^{*}$ *. For $p=0,$ the integral $$ {\frac{1}{2\pi i}}\equiv{\frac{\int_{r}^{r\left(z\right)}{f\left(z\right)}}}\,z^{p}~d z $$ of f) for is equal to the number of zeros of f in $D(a;r).$ What is the value of this integral in terms of the zeros $p=1,2,3,\ldots^{\gamma}$ What is the answer if $\mathbb{Z}^{\mathsf{P}}$ is replaced by any $\varphi\in H(\Omega)^{\gamma}$ 19 Suppose fE $\hat{\tau}\,{\cal H}(U),\,g\in{\cal H}(U),$ and neither f nor $\scriptstyle{\mathcal{G}}$ has a zero in U. If $$ {\frac{f^{\prime}}{f}}\left({\frac{1}{n}}\right)={\frac{g^{\prime}}{g}}\left({\frac{1}{n}}\right){\pmod{1,2,3,\dots}} $$ find another simple relation between f and $g.$ 20 Suppose $\scriptstyle\Omega$ is a region, fe H(Q) for $n=1,$ 2,3,.…, none of the functions ${f}_{n}$ has $\underline{{\land}}$ zero in $\Omega,$ and for all $\scriptstyle{\varepsilon\circ\Omega}$ converges tof uniformly on compact subsets of $\Omega.$ Prove that either f has no zero in $\underline{{\Omega}}$ .2 o $\tau f(z)=0$ $\{f_{n}\}$ If $\Omega^{\prime}$ is a region that contains every f,(Q), and if fis not constant, prove that f $(\Omega)\subset\Omega^{\prime}$ 21 Suppose fe H(Q), $\scriptstyle\Omega$ contains the closed unit disc, and $|f(z)|<1$ if $|z|=1$ How many fixed points must f have in the disc? That is, how many solutions does the equation $f(z)=z$ have there ? zero in the unit disc? 22 Suppose fe H(Q), Q contains the closed unit disc,| f(z)|> 2 f $|z|=1,\operatorname{and}f(0)=1.$ Must f have a 23 Suppose $P_{n}(z)=1+z/1!+\cdots+z^{n}/n!,$ $Q_{n}(z)=P_{n}(z)-1,$ $n{\dot{\gamma}}$ Be as specific as you can 2, ${\mathfrak{I}},\,\ldots.$ What can you say where $n=1,$ about the location of the zeros of ${\mathbf{}}P_{n}$ and O, for large 24 Prove the following general form of Rouche's theorem: Let Q be the interior of a compact set ${\cal K}$ in all the plane. Suppose f and g are continuous on ${\bar{\boldsymbol{K}}}$ and holomorphic in $\Omega.$ and $|f(z)-g(z)|<|f(z)|$ for $\Omega,$ $\kappa\in K-\Omega$ Then fand $\scriptstyle{\mathcal{G}}$ have the same number of zeros in 25 Let A be the annulus $\{z\colon r_{1}<|z|<r_{2}\},$ where $r_{1}$ , and ${\boldsymbol{r}}_{2}$ - are given positive numbers (a) Show that the Cauchy formula $$ f(z)={\frac{1}{2\pi i}}\left(\right)_{r_{1}}+\left\{\right\}_{r_{2}} ){\frac{f(\zeta)}{\zeta-z}}\,d\zeta $$ is valid under the following conditions: fe H(A), $$ r_{1}+\epsilon<|z|<r_{2}-\epsilon, $$230 REAL AND coMPLEX ANALYSIs and $$ \gamma_{1}(t)=(r_{1}+\epsilon)e^{-i t},\qquad\gamma_{2}(t)=(r_{2}-\epsilon)e^{i t}\qquad(0\le t\le2\pi). $$ is holomorphic outside (b) Show by means of (a) that everyfe and $f_{2}\in H(D(0;\,r_{2}));$ the decomposition is unique if we require that where f $H(A)$ can be decomposed into a $1\operatorname{sum}f=f_{1}+f_{2},$ $D(0;r_{1})$ $f_{1}(z)\to0$ as $|z|\to\infty.$ (c) Use this decomposition to associate with each fe H(A) its so-called “Laurent series $$ \sum_{-\infty}^{\infty}c_{n}z^{n} $$ which converges to f in $\scriptstyle A\quad}$ Show that there is only one such series for each f. Show that it converges to f uniformly on compact subsets of $\scriptstyle A\quad\quad A$ (d) $\scriptstyle1F/\in R(A)$ and fis bounded in $A,$ 4, show that the components $f_{1}$ and $f_{2}$ f are also bounded (e How much ot the foregoing can you extend to the case $r_{1}=0\left(\mathrm{or}\ r_{2}=\infty,\mathrm{or}\right.$ both)? $(f)$ How much of the foregoing can you extend to regions bounded by finitely many (more than two) circles? $26$ It is required to expand the function $$ {\frac{1}{1-z^{2}}}+{\frac{1}{3-z}} $$ in a series of the form $\sum_{-\alpha}^{\infty}c_{n}z^{n}.$ How many such expansions are there?? In which region is each of them valid? Find the coeff- cients $c_{n}$ 。explicitly for each of these expansions. $27$ and f(z) $=f(z+1)$ for al $\scriptstyle{\varepsilon\circ\mathbf{a}}$ Suppose Q is a horizontal strip, determined by the inequalities $a<y<b,$ say. Suppose fe H(Q) Prove that f has a Fourier expansion in SQ, $$ f(z)=\sum_{-\alpha_{0}}^{\infty}c_{n}e^{2\pi i n z}, $$ which converges uniformly in $\{z\colon a+\epsilon\leq y\leq b-\epsilon\},$ for every $\epsilon>0.$ Hint: The map $z\to e^{2\pi i z}$ con- verts f to afunction in an annulus. Find the integral formulas by means of which the coefficients $c_{n}$ can be computed from f imate 28 Suppose ${\mit\Gamma}$ is a closed curve in the plane, with parameter interval [O,2z]. Take Show that Ind- ${\bf\nabla}_{*}(x)=\mathrm{Ind}_{\Gamma_{*}}(x):$ if m and $\scriptstyle n$ are ${\Gamma}$ uniformly by trigonometric polynomials $x\notin\Gamma^{*}.$ Approx- $\Gamma_{n}$ sufficiently large. Define this common value to be Ind- (Ga). Prove that the result does not depend on the choice of $\{\Gamma_{n}\}_{i}$ prove that Lemma 10.39 is now true for closed curves, and use this to give a different proof of Theorem 10.40. $29$ Define $$ f(z)={\frac{1}{\pi}}\left|_{0}^{*1}r~d r\right.\left.\right|_{-\pi}^{\pi}{\frac{d\theta}{r e^{i\theta}+z}}. $$ Show $\operatorname{that}f(z)={\bar{z}}\operatorname{if}\left|z\right|<1$ and $\operatorname{that}f(z)=1/z\,\mathrm{if}\left|z\,\right|\geq1$ and Exercise 16 on the other. Thus is not holomorphic in the unit disc, alihough the integrand is a holomorphic function of $10.7$ z. Note the contrast between this, on the one hand, and Theorem Suggestion: Compute the inner integral separately for $\scriptstyle{\epsilon\in[x]}$ and for $r>|z|.$ 30 Let $\underline{{\Omega}}$ be the plane minus two points, and show that some closed paths T in Q satisfy assumption (1) of Theorem 10.35 without being null-homotopic in $\Omega.$