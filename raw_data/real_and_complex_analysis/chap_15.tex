CHAPTER FIF'TEEN ZEROSOF HOLOMORPHIC FUNCTIONS Infinite Products 15.1 So far we have met only one result concerning the zero set $\Omega,$ , namely, $\mathbb{Z}(f)$ has no limit point in of a non- constant holomorphic function f in a region $\scriptstyle2(J)$ S. We shall see presently that this is all that can be said about $\angle(J),$ if no other $f\in H(\Omega).$ If 15.11) which asserts that every $\scriptstyle4\,=\Omega$ f, because of the theorem of Weierstrass (Theorem $\ \f$ is to choose functions for some conditions are imposed on $f,$ $A=\{x_{n}\},$ without limit point in $\Omega$ is $\mathbb{Z}(f)$ , a natural way to construct such an $f_{n}\in H(\Omega)$ so that ${\mathfrak{f}}_{n}$ has only one zero, at $\alpha_{n}\,,$ and to consider the limit of the products $$ p_{n}=f_{1}f_{2}\cdot\cdot\cdot f_{n}, $$ as $n\to G o.$ One has to arrange it so that the sequence $\{p_{n}\}$ converges to some $f\in H(\Omega)$ and so that the limit function $\boldsymbol{\f}$ is not O except at the prescribed points $\alpha_{n}\cdot\operatorname{It}$ is therefore advisable to begin by studying some general properties of infin- ite products 15.2 Definition Suppose $\left\{u_{n}\right\}$ is a sequence of complex numbers, $$ p_{n}=(1+u_{1})(1\,+\,u_{2})\cdot\cdot\cdot(1\,+\,u_{n}), $$ (1) and $p=\operatorname*{lim}_{n\to\infty}\,p_{n}$ exists. Then we write $$ p=\prod_{n=1}^{\infty}\;(1+u_{n}). $$ (2) The $\boldsymbol{p_{n}}$ are the partial products of the infinite product (2). We shall say that the $\{p_{n}\}$ converges. infinite product (2) converges if the sequence 298ZEROs oF HOLoMORPHiC FUNCrioNs 299 In the study of infinite series $\Sigma a_{n}$ it is of significance whether the a, approach O rapidly. Analogously, in the study of infinite products it is of interest whether the factors are or are not close to 1. This accounts for the above notation: $1+u_{n}$ is close to 1 if $u_{n}$ is close to O 15.3 Lemma $I f u_{1},\ldots,u_{N}\,a_{i}$ re complex numbers, and $i f$ $$ p_{N}=\prod_{n=1}^{N}\;(1\,+\,u_{n}),\qquad p_{N}^{*}=\prod_{n=1}^{N}\;(1\,+\,|\,u_{n}|\,), $$ (1) then $$ p_{N}^{\ast}\leq\exp\,\left(|\,u_{1}\,|\,+\,\cdot\cdot\cdot\,+\,|u_{N}\,|\right) $$ (2) and $$ |\,p_{N}-1|\leq p_{N}^{*}-1. $$ (3) PR00F Fo $\scriptstyle x\geq0.$ the inequality $1+x\leq e^{x}$ is an immediate consequence of and multiply the expansion of ${\mathcal{E}}^{\times}$ in powers of $x.$ Replace $\scriptstyle{\mathcal{X}}$ by $|u_{1}|\,,\,\ldots,\,|u_{N}|$ the resulting inequalities. This gives (2). For $N=1,(3)$ is trivial. The general case follows by induction ${\mathrm{For~}}k=1,\ldots,N-1,$ $$ p_{k+1}-1=p_{k}(1+u_{k+1})-1=(p_{k}-1)(1+u_{k+1})+u_{k+1}, $$ so that if(3) holds with $\boldsymbol{k}$ in place of $N,$ then also l P2+1-1|≤ (pt -1)(1 +「ux+11) +「ux+11 = P*+1- 1. // set ${\boldsymbol{S}},$ such that E|u,(s)| converges uniformly on $\boldsymbol{\mathsf{S}}$ is a sequence of bounded complex functions on a 15.4 Theorem Suppose $\{u_{n}\}$ Then the product $$ f(s)=\prod_{n=1}^{\infty}\;(1+u_{n}(s)) $$ (1) converges uniformly on ${\boldsymbol{S}},$ and $f(s_{0})=0$ at some $\scriptstyle\iota_{0}\in S$ if and only i $u_{n}(s_{0})=$ -1 for some n also have Furthermore,if $\{n_{1},\,n_{2},\,n_{3},\,\ldots\}$ is any permutation of {1, 2, 3,…..}, then we $$ f(s)=\prod_{k=1}^{\infty}\;(1+u_{n_{k}}(s))\qquad(s\in S). $$ (2) there is a constant $c<\alpha$ such that $|p_{N}(s)|\leq C$ for all is bounded on ${\boldsymbol{S}},$ and if $p_{N}$ PRoOF The hypothesis implies that $\Sigma\left|u_{a}(s)\right|$ and all s denotes the Nth partial product of (1), we Conclude from Lemma 15.3 that ${\mathbf{}}N$300 REAL AND coMPLEX ANALYSIs Choose $\epsilon,0<\epsilon<\textstyle{\frac{1}{2}}.$ There exists an $N_{\mathrm{0}}$ such that $$ \displaystyle\sum_{n=N_{0}}^{\infty}|u_{n}(s)|<\epsilon\qquad(s\in S). $$ (3) Let $\{n_{1},\,n_{2},\,n_{3},\,\ldots\}$ be a permutation of $\{1,2,3,\ldots\}$ If $N\geq N_{0}$ , if Λ $\textstyle{M}$ l is so large that $$ \{1,\,2,\,\ldots,\,N\}\subset\{n_{1},\,n_{2},\,\ldots,\,n_{M}\}, $$ (4) and if $q_{M}(s)$ denotes the Mth partial product of (2), then $$ q_{M}-p_{N}=p_{N}\{\prod(1+u_{m})-1\}. $$ (5) The $\scriptstyle n_{k}$ which occur in (5) are all distinct and are larger than $N_{\mathrm{0}}$ Therefore $({\mathbf{3}})$ and Lemma 15.3 show that $$ |q_{M}-p_{N}|\leq|p_{N}|(e^{e}-1)\leq2|p_{N}|\epsilon\leq2C\epsilon. $$ (6) If $n_{k}=k$ $(k=1,2,3,\ldots),$ then $q_{M}=p_{M}$ and $\mathbf{\tau}(6)$ shows that $\scriptstyle(p_{\mathrm{sl}})$ con- verges uniformly to a limit function f. Also,(6) shows that $$ |\,p_{M}-p_{N_{0}}|\leq2\,|\,p_{N_{0}}|\epsilon\qquad(M>N_{0}), $$ (7) so that $|p_{M}|\geq(1-2\epsilon)|p_{N_{0}}|.$ Hence $$ \vert f(s)\vert\geq(1-2\epsilon)\vert p_{N_{0}}(s)\vert\qquad(s\in S), $$ (8) which shows $\operatorname{that}f(s)=0$ if and only if $p_{N_{0}}(s)=0.$ Finally,(6) also shows that $\langle v_{w}\rangle$ converges to the same limit as {pw}./ 15.5 Theorem Suppose $0\leq u_{n}<1$ .Then $$ \prod_{n=1}^{\infty}\;(1-u_{n})>0\;\;\;\;\;\;\;\;i f\,a n d\;o n l y\;i f\quad\;\;\;\sum_{n=1}^{\infty}u_{n}<\infty. $$ lim PRoor If $p_{N}=(1-u_{1})\cdot\cdot\cdot(1-u_{N}),$ then $p_{1}\geq p_{2}\geq\cdots$ $p_{x}>0,$ hence ${\boldsymbol{p}}=$ $p_{N}$ exists. 1 $\Sigma u_{n}<\infty,$ Theorem 15.4 implies $\scriptstyle p\gg0$ On the other hand, $$ p\leq p_{N}=\prod_{1}^{N}\,\left(1-u_{n}\right)\leq\exp\,\left\{-\,u_{1}-u_{2}-\cdots-u_{N}\right\}, $$ and the last expression tends to O as $N arrow\infty,{\mathrm{if~}}\Sigma u_{n}=\infty$ // We shall frequently use the following consequence of Theorem 15.4: 15.6 Theorem Suppose $f_{s}\in H(\Omega)$ for n = 1,2,3,..., no ${\mathfrak{f}}_{n}$ is identically O in any component of Q, and $$ \sum_{n=1}^{\infty}\mid1-f_{n}(z)\mid $$ (1)ZEROS OF HOLOMORPHIC FUNCTIONS 301 converges uniformly on compact subsets of Q2. Then the product $$ f(z)=\prod_{n=1}^{\infty}f_{n}(z) $$ (2) converges uniformly on compact subsets of Q. Hence fe H(Q2) Furthermore, we have $$ m(f;z)=\sum_{n=1}^{\infty}m(f_{n};z)\qquad(z\in\Omega), $$ (3) then m(f; where m(f; z) is defined to be the multiplicity of the zero of f at $:z,\left[\mathrm{If}f(z)\neq0,\right]$ $z)=0.$ PRooF The first part follows immediately from Theorem 15.4. For the second part, observe that each $\scriptstyle*\in\Omega$ has a neighborhood V ${\mathbf{}}V$ in which at most finitely many of $\mathrm{the}\,f_{n}$ have a zero, by (1). Take these factors first. The product of the remaining ones has no zero in V, by Theorem 15.4,and this gives (3). Inci dentally, we see also that at most finitely many terms in the series (3) can be positive for any given z ${\mathrm{en}}$ /// The Weierstrass Factorization Theorem 15.7 Definition Put $E_{0}(z)=1-z,\operatorname{and}\,\operatorname{for}p=1,2,$ 3, $$ E_{p}(z)=(1-z)\exp\left\{z+{\frac{z^{2}}{2}}+\cdots+{\frac{z^{p}}{p}}\right\}. $$ These functions, introduced by Weierstrass, are sometimes called elementary factors. Their only zero is at $z=1.$ Their utility depends on the fact that they are close to 1 if $|z|<1$ and ${\boldsymbol{p}}$ is large, although $E_{p}(1)=0.$ 15.8 Lemma $F o r|z|\leq1$ and $p=0,1,2,\ldots,$ $$ |1-E_{p}(z)|\leq|z|^{p+1}. $$ PRoOF For $p=0,$ this is obvious. For $\scriptstyle p\geq1,$ direct computation shows that $$ -\;E_{p}^{\prime}(z)=z^{p}\;\mathrm{exp}\;\bigg\{z+\frac{z^{2}}{2}+\cdots+\frac{z^{p}}{p}\bigg\}\;. $$ So $-E_{p}^{\prime}$ has a zero of order $\boldsymbol{\mathit{P}}$ p at $\scriptstyle z\qquad\qquad z\qquad\qquad z\qquad\qquad z\qquad z\qquad z\qquad z\qquad z\qquad z\qquad z\qquad z\qquad z\qquad z\qquad z\qquad z\qquad z\qquad z\qquad z\in0,z$ and the expansion of $-E_{p}^{\prime}$ in powers of z has nonnegative real coefficients. Since $$ 1-E_{p}(z)=\ -\ \Rightarrow_{[0,\,z]}E_{p}^{\prime}(w)\ d w, $$302 REAL AND coMPLEX ANALYSis $1-E_{p}$ has a zero of order $p+1\;\mathrm{at.}z=0,$ and if $$ \varphi(z)=\frac{1-E_{p}(z)}{z^{p+1}}, $$ then $\varphi(z)=\Sigma a_{n}z^{n},$ with all $a_{n}\geq0.$ Hence $|\varphi(z)|\leq\varphi(i)=1$ if $|z|\leq1,$ and this // gives the assertion of the lemma_ 15.9 Theorem Let $\left\{z_{n}\right\}$ be $\bar{a}$ sequence of complex numbers such that $z_{n}\neq0$ and lzmI→OO as n→00. ${\boldsymbol{\jmath}}$ $\{p_{n}\}$ is a sequence of nonnegative integers such that $$ \sum_{n=1}^{\infty}\,\left({\frac{r}{r_{n}}}\right)^{1+p_{n}}<\infty $$ (1) for every positive r(where $r_{n}=|z_{n}|\,\rangle,$ then the infinite product $$ P(z)=\prod_{n=1}^{\infty}\ E_{p_{n}}\!\left(\frac{z}{z_{n}}\right) $$ (2) defines an entire function ${\mathbf{}}P$ D which has a zero at each point $\mathbb{Z}_{n}$ ,and which has no other zeros in the plane. More precisely, ifα occurs m times in the sequence $\{z_{n}\}.$ ,then ${\mathbf{}}P$ has a zero of order m at α. Condition (1) is always satisfied i ${p}_{n}=n-1,f o r\;i n s t a n c e.$ PRoor For every r $r_{n}>2r$ for all but finitely many n, hence $r/r_{n}<{\frac{1}{2}}$ for these n, so (1) holds with $1+p_{n}=n.$ Now fix $r.\operatorname{If}|z|\leq r,$ Lemma 15.8 shows that $$ \left|\mid1-E_{p n}^{\quad\quad}\! |\leq |\frac{z}{z_{n}^{\quad\mid}^{\d1+p_{n}}}\leq (\frac{r}{r_{n}^{\quad\mid}^{\d+p_{n}}}\right)^{\mid+p_{n}} $$ if $r_{n}\geq r,$ which holds for all but finitely many n. It now follows from (1) that the series $$ \sum_{n=1}^{\infty}\Big|1-E_{p n}\Big(\frac{z}{z_{n}}\Big)\Big| $$ the desired conclusion. converges uniformly on compact sets in the plane, and Theorem 15.6 gives / Note: For certain sequences $\left\{r_{n}\right\}$ ,(1) holds for a constant sequence $\{p_{n}\}.$ It is of interest to take this constant as small as possible; the resulting function (2) is $\Sigma1/r_{n}<\infty$ , we can take then called the canonical product corresponding to $\{z_{n}\}.$ For instance,if $p_{n}=0,$ and the canonical product is simply $$ \prod_{n=1}^{\infty}\left(1-{\frac{z}{z_{n}}}\right)\!. $$ZEROS OF HOLoMORPHIC FUNCTIONs 303 If $\Sigma1/r_{n}=\infty$ but $\Sigma1/r_{n}^{2}<\infty,$ the canonical product is $$ \prod_{n=1}^{\infty}\left(1-{\frac{z}{z_{n}}}\right)e^{z/z_{n}}. $$ Canonical products are of great interest in the study of entire functions of finite order. (See Exercise 2 for the definition.) We now state the Weierstrass factorization theorem. 15.10 Theorem Let f be an entire function, suppose $f(0)\neq0,$ and let $z_{1},$ $z_{2}$ ${\mathrm{z}}_{3}$ .. be the zeros of f, listed according to their multiplicities. Then there exist an entire function g and a sequence $\{p_{n}\}$ of nonnegative integers, such tha $$ f(z)=e^{\theta(z)}\ \prod_{n=1}^{\infty}\ E_{p_{n}}\biggl(\frac{z}{z_{n}}\biggr). $$ (1) Note: (a) Iff has a zero of order $\boldsymbol{k}$ at $z=0,$ the preceding applies $\mathrm{to}f(z)/z^{k}.(b)$ The factorization(1) is not unique; a unique factorization can be associated with those f whose zeros satisfy the condition required for the convergence of a canon- ical product. PROOF Let ${\mathbf{}}P$ be the product in Theorem 15.9,formed with the zeros of $f.$ Then $f/P$ has only removable singularities in the plane, hence is (or can be extended to) an entire function. Also, f $f/P$ P has no zero, and since the plane is simply connected, f/P = e for some entire function ${\mathfrak{g}}.$ // The proof of Theorem 15.9 is easily adapted to any open set: 15.11 Theorem Let $\Omega$ be an open set in $S^{2},\Omega\neq S^{2}$ Suppose $\scriptstyle4\,\equiv\Omega$ and $\scriptstyle A$ l has no limit point in Q2. With each $\scriptstyle x\in A$ associate a positive integer m(α).Then there exists an fe H(Q2) all of whose zeros are in A, and such that f has a zero of order m(α) at each αe A. PxOOF It simplifies the argument, and causes no loss of generality, to assume that $\textstyle\mathbf{\nabla}_{\alpha}\operatorname{e}\mathbf{a}$ but o生 A.(If this is not so, a linear fractional transformation will make it so.) Then $\scriptstyle S^{2}\,{\mathrm{-a}}$ is a nonempty compact subset of the plane and ${\mathcal{D}}$ is not a limit point of $A.$ If $\scriptstyle A$ is finite, we can take a rational function for f If A is infinite, then A is countable (otherwise there would be a limit $\scriptstyle S^{2}\,-\,\Omega$ point in Q). Let $\{\alpha_{n}\}$ be a sequence whose terms are in for all $\beta\in S^{2}-\Omega;$ and in which each $\scriptstyle b_{x}\epsilon$ $\scriptstyle A$ $\scriptstyle x\in A$ such that is listed precisely m(α)times. Associate with each $\textstyle x_{n}$ a point $|\beta_{n}-\alpha_{n}|\leq|\beta-a_{n}|$ this is possible since $\scriptstyle S^{2}\,-\,\Omega$ is compact. Then $$ |\beta_{n}-\alpha_{n}|\to0 $$304 REAL AND coMPLEX ANALYSIs as $n\to\infty\,;$ otherwise $\scriptstyle A$ would have a limit point in Q2. We claim that $$ f(z)=\prod_{n=1}^{\infty}\,E_{n}\!\left(\frac{\alpha_{n}-\beta_{n}}{z-\beta_{n}}\right) $$ has the desired properties. Put $r_{n}=2\mid\alpha_{n}-\beta_{n}|\,.$ Let ${\cal K}\,\,\,\,\,$ be a compact subset of Q. Since $r_{n} arrow0,$ there exists an ${\mathbf{}}N$ such that $|z-\beta_{n}|>r_{n}$ for all $\scriptstyle{\epsilon\;\epsilon\;K$ and all $n\geq N.$ Hence $$ \left|{\frac{\alpha_{n}-\beta_{n}}{z-\beta_{n}}}\right|\leq{\frac{1}{2}}, $$ which implies,by Lemma 15.8, that $$ \left|1-E_{n}\!\left(\frac{\alpha_{n}-\beta_{n}}{z-\beta_{n}}\right)\right|\leq\!\left(\frac{1}{2}\right)^{n+1}\qquad(z\in K,\,n\geq N), $$ and this again completes the proof, by Theorem 15.6. / As a consequence,we can now obtain a characterization of meromorphic functions (see Definition 10.41): 15.12 Theorem Every meromorphic function in an open set $\Omega$ is a quotient o two functions which are holomorphic in SQL The converse is obvious: Ifge H(2), h e H(Q), and h is not identically O in any component of $\Omega,$ , then g/h is meromorphic in Q PRoOF Suppose fis meromorphic in Q; let $\scriptstyle A$ be the set of all poles of f in Q: and for each α∈ A, let $\scriptstyle m(x)$ be the order of the pole of fat α.By Theorem f = 9/h in 15.11 there exists an $h\in H(\Omega)$ such that ${\boldsymbol{h}}$ has a zero of multiplicity m(c) at Clearly // each αe A, and $\boldsymbol{h}$ has no other zeros. Put $\scriptstyle g=\beta h.$ The singularities of g at the points of A are removable, hence we can extend g so that $g\in H(\Omega).$ $\Omega-A.$ An Interpolation Problem The Mittag-Leffler theorem may be combined with the Weierstrass theorem 15.11 to give a solution of the following problem: Can we take an arbitrary set $A\in\Omega.$ without limit point in Q, and find a function fe H(Q) which has pre scribed values at every point of $A\,{\dot{\gamma}}$ The answer is affirmative. In fact, we can do even better, and also prescribe finitely many derivatives at each point of $\scriptstyle A$ point in 15.13 Theorem Suppose SQ is an open set in the plane, $4<\Omega.$ A has no limi $\Omega,$ and to each αe A there are associated a nonnegative integer m(a)ZEROs OF HOLOMORPHIC FUNCTIONS 305 and complex numbers v $\nu_{n,\,\alpha},\,0\leq n\leq m(\alpha).$ Then there exists an fe H(Q2) such that $$ f^{(n)}(\alpha)=n!\;w_{n,\,\alpha}\;\;\;\;\;\;\;\;(\alpha\in A,\,0\leq n\leq m(\alpha)). $$ (1) associate to each PRoOF By Theorem 15.11, there exists a a function $P_{\alpha}$ of the form at each $\alpha\in A.$ We claim we can $\scriptstyle A$ $g\in H(\Omega)$ whose only zeros are in and such that $\scriptstyle{\mathcal{G}}$ has a zero of order $m(x)+1$ $\alpha\in A$ $$ P_{\alpha}(z)=\stackrel{1+m(\alpha)}{\displaystyle{j=1}}c_{j,\alpha}(z-\alpha)^{-j} $$ (2) such that $g P_{\alpha}$ has the power series expansion $$ g(z)P_{x}(z)=w_{0,\,\alpha}+w_{1,\,\alpha}(z-\alpha)+\cdot\cdot\cdot+w_{m(\alpha),\,\alpha}(z-\alpha)^{m(\alpha)}+\cdot\cdot\cdot $$ (3) in some disc with center at α To simplify the writing, take $\scriptstyle x\;=\;0$ and $m(x)=m,$ and omit the subscripts α.For z near O, we have $$ g(z)=b_{1}z^{m+1}+b_{2}z^{m+2}+\cdots, $$ (4) where $b_{1}\neq0.1$ If $$ P(z)=c_{1}z^{-1}+\cdot\cdot\cdot+c_{m+1}z^{-m-1}, $$ (5) then $$ g(z)P(z)=(c_{m+1}+c_{m}z+\cdot\cdot\cdot+c_{1}z^{m})\quad(b_{1}+b_{2}\,z+b_{3}\,z^{2}+\cdot\cdot\cdot). $$ (6) The b's are given, and we want to choose the c's so that $$ g(z)P(z)=w_{0}+w_{1}z+\cdot\cdot\cdot+w_{m}z^{m}+\cdot\cdot\cdot. $$ (7) If we compare the coefficients of $\setminus_{3}\ Z_{>}\ \circ\ \circ\ \times^{m}$ in (6) and $({\bar{7}}),$ we can solve the resulting equations successively for cm+1, Cm, …,Ci, since $b_{1}\neq0.$ In this way we obtain the desired in $\Omega$ whose principal parts are these $P_{\alpha}^{*}$ a's, and if we // $P_{\alpha}^{\circ}$ 's. The Mittag-Leffler theorem now gives us a meromorphic ${\boldsymbol{h}}$ putf= gh we obtain a function with the desired properties The solution of this interpolation problem can be used to determine the structure of all finitely generated ideals in the rings $\scriptstyle{m_{(0)}}$ $m(f;\alpha)=0,$ 15.14 Definition The ideal $[g_{1},\dots,g_{n}]$ generated by the functions where $f_{i}\in H(\Omega)$ for ∈ If $f\in H(\Omega),$ H(Q) is the set of all functions of the form $\scriptstyle{1/\theta_{1}}$ $g_{1},\,\cdot\cdot\cdot,\,g_{n}$ then $i=1,\dots,n\ A$ principal ideal is one that is generated by a single function. Note that $[1]=H(\Omega)$ α∈ Q2, and f is not identically O in a neighborhood of α,the multiplicity of the zero of f at α will be denoted by $m(f;\ x).\;\mathrm{If}f(x)\neq0,$ as in Theorem 15.6.306 REAL AND COMPLEX ANALYSIS 15.15 Theorem Every finitely generated ideal in H(Q2) is principal. More explicitly: If $g_{1},\,...\,,\,g_{n}\in H(\Omega),$ then there exist functions g,f, $h_{i}$ e H(Q2) such that $$ g=\sum_{i=1}^{n}f_{i}g_{i}\quad{\mathrm{and}}\quad g_{i}=h_{i}g\qquad(1\leq i\leq n). $$ PROOF We shall assume that $\Omega$ is a region. This is done to avoid problems posed by functions that are identically O in some components of $\Omega$ but not in all. Once the theorem is proved for regions, that case can be applied to each component of an arbitrary open set $\Omega.$ 2, and the full theorem can be deduced We leave the details of this as an exercise Let $\scriptstyle{p_{(n)}}$ be the following proposition: every $\scriptstyle P_{10}$ $I f g_{1},\ldots,g_{n}\in H(\Omega),$ if no $g_{i}$ is identically ${\boldsymbol{0}},$ and if no point of S is a zero of $g_{1},\,\ldots,$ ${g}_{i},$ then [g,. $g_{n}!=[1].$ $\scriptstyle n\gg1$ and that $P(n-1)$ is true. Take is trivial. Assume that exists $\varphi\in H(\Omega)$ such that without common zero. By the Weierstrass theorem 15.11 there $g_{n}\in H(\Omega),$ $$ m(\varphi\,;\,\alpha)=\operatorname*{min}\;\{m(g_{i};\,\alpha)\!:\,1\leq i\leq n-1\}\qquad(\alpha\in\Omega). $$ (1) The functions $f_{i}=g_{i}/\varphi\left(1\leq i\leq n-1\right)$ are in $\scriptstyle n(0)$ and have no common zero in SQ. Since $P(n-1)$ is true, $[f_{1},\dots,f_{n-1}]=[1].\mathrm{Rer}$ ice $$ [g_{1},\ast,\,g_{n-1},\,g_{n}]=[\varphi,\,g_{n}]. $$ (2) Moreover, our choice of $\varphi$ shows that $g_{a}(x)\neq0$ at every point of the set $A=\{\alpha\in\Omega\colon\varphi(\alpha)=0\}$ . Hence it follows from Theorem 15.13 that there exists $h\in H(\Omega)$ such that $$ m(1-h g_{n};\,\alpha)\geq m(\varphi\,;\,\alpha)\qquad(x\in\Omega). $$ (3) Such an $\boldsymbol{\mathit{h}}$ is obtained by a suitable choice of the prescribed values of $h^{(k)}(\alpha)$ for α∈ A and for $0\leq k\leq m(\varphi;\,\alpha).$ By (3) $(1-h g_{n})/\varphi$ has removable singularities. Thus $$ 1=h g_{n}+f\varphi $$ (4) for some $f\in H(\Omega).$ By (2) and (4), $1\in[{\mathcal{G}}_{1},\,\dots,\,{\mathcal{G}}_{n}].$ Hence $\scriptstyle{p_{(n)}}$ is true for all ${\mathfrak{n}}.$ We have shown that $P(n-1)$ implies $\scriptstyle{P_{\mathrm{eff}}}$ $G_{i}$ is identically O.(This Finally, suppose $G_{1},\,\ldots,\,G_{n}\in H(\Omega),$ and no involves no loss of generality.) Another application of Theorem 15.11 yields $\varphi\in H(\Omega)$ with $m(\varphi;\alpha)=\operatorname*{min}\,m(G_{i};\,\alpha)$ for all $\scriptstyle x\in\Omega$ Put $g_{i}=G_{i}/\varphi$ Then $\ U_{i}\ \in$ H(Q),and the functions $g_{1},\,\ldots,\,g_{n}$ have no common zeros in Q. By $\scriptstyle{p_{\mathrm{th}}}$ $[g_{1},\dots,g_{n}]=[1].$ Hence $[G_{1},\ldots,$ $G_{n}]=[\varphi].$ This completes the proof. ///ZEROS OF HOLoMORPHIC FUNCTIONs 307 Jensen's Formula 15.16 As we see from Theorem 15.11, the location of the zeros of a holomorphic function in a region $\Omega$ 2 is subject to no restriction except the obvious one concern- ing the absence of limit points in Q2. The situation is quite different if we replace H(Q2) by certain subclasses which are defined by certain growth conditions. In those situations the distribution of the zeros has to satisfy certain quantitative conditions. The basis of most of these theorems is Jensen's formula(Theorem 15.18). We shall apply it to certain classes of entire functions and to certain sub classes of $m(v).$ The following lemma affords an opportunity to apply Cauchy's theorem to the evaluation of a definite integral. 15.17 Lemma 1 (2 log |l - ei"|d0 = 0 Z元 PxoOF Let $\Omega=\{z\colon\mathbf{R}\circ z<1\}$ .Since $1-z\neq0$ in $\Omega$ and $\Omega$ is simply con nected, there exists an $h\in H(\Omega)$ such that $$ \exp\;\{h(z)\}=1-z $$ in $\Omega,$ and this ${\boldsymbol{h}}$ is uniquely determined if we require that $h(0)=0.$ Since Re $(1-z)>0$ in Q, we then have $$ \mathrm{Re}~h(z)=\log\vert1-z\vert,\qquad\vert\mathrm{Im}~h(z)\vert<{\frac{\pi}{2}}\qquad(z\in\Omega). $$ (1) For small $\delta>0$ , let ${\Gamma}$ be the path $$ \Gamma(t)=e^{i t}\qquad(\delta\leq t\leq2\pi-\delta), $$ (2) and let $\scriptstyle{\mathcal{Y}}$ , be the circular arc whose center is at l and which passes from $e^{i\delta}$ to $e^{-i\delta}$ within ${\cal U}.$ Then $$ {\frac{1}{2\pi}}\int_{\delta}^{2\pi-\partial}\log\left|1-e^{i\theta}\right|\,d\theta=\operatorname{Re}\left[{\frac{1}{2\pi i}}\int_{\Gamma}h(z)\,{\frac{d z}{z}}\right]=\operatorname{Re}\left[{\frac{1}{2\pi i}}\right]_{\gamma}\left(z\right){\frac{d z}{z}} ]. $$ (3) The last equality depended on Cauchy's theorem; note that $h(0)=0.$ The length of $\scriptstyle{\mathcal{V}}$ is less than To,so (1) shows that the absolute value of the is a constant. This gives // the result if last integral in (3) is less than in (3). $(1/\delta),$ where ${\boldsymbol{C}}$ ${\boldsymbol{C}}\delta$ log $\delta\to0$ 15.18 Theorem Suppose Q = D(0;R), fe H(Q) $f(0)\neq0,$ 0<r< R,and $\alpha_{1},$ ··· $x_{N}$ are the zeros off in ${\bar{D}}(0;r),$ listed according to their multiplicities. Then $$ |f(0)|\prod_{n=1}^{N}\frac{r}{|\alpha_{n}|}=\exp\left\{\frac{1}{2\pi}\int_{-\pi}^{\pi}\log|f(r^{i\theta})|\ d\theta\right\}. $$ (1)308 REAL AND coMPLEX ANALYSIs This is known as Jensen's formula. The hypothesis $f(0)\neq0$ causes no harm in applications, for if $\boldsymbol{\f}$ has a zero of order $\boldsymbol{k}$ at O, the formula can be applied to $f(z)/z^{k}.$ PROOF Order the points $\alpha_{j}$ α; So that $\alpha_{1},\,\cdot\cdot\cdot,\,\alpha_{m}$ are in $D(0;r)$ and |αm+1|= $\mathbf{\partial}\cdot\mathbf{\partial}=\left|\,\alpha_{N}\right|=r.\mathbf{\partial}(\mathbf{Of}$ course, we may hav $\mathbf{\hat{\Pi}}m=N$ or $m=0.$ ) Put $$ g(z)=f(z)\prod_{n=1}^{m}\frac{r^{2}-\vec{\alpha}_{n}z}{r(\alpha_{n}-z)}\prod_{n=m+1}^{N}\frac{\alpha_{n}}{\alpha_{n}-z}. $$ (2) Then $g\in H(D),$ where $D=D(0;\,r+\epsilon)$ for some $\scriptstyle\epsilon>0,$ g has no zero in ${\boldsymbol{D}},$ hence log lglis harmonic in $D\!\!\!\!/$ (Theorem 13.12), and so $$ \log\mid g(0)\mid=\frac{1}{2\pi}\int_{-\pi}^{\pi}\log\mid g(r e^{i\theta})\mid d\theta. $$ (3) By (2) $$ |g(0)|=|f(0)|\prod_{n=1}^{m}{\frac{r}{|\alpha_{n}|}}. $$ (4) For $1\leq n\leq n$ n, the factors in (2) have absolute value 1 $\mathbb{F}\mid z\mid=r.$ If $x_{n}=r e^{i\theta_{n}}$ for $m<n\leq N,{\mathrm{it}}$ follows that $$ \mathrm{log}\,\,\vert\,g(r e^{i\theta})\,\vert=\mathrm{log}\,\,\vert\,f(r e^{i\theta})\,\vert\,-\sum_{n=m+1}^{N}\log\,\vert\,1-e^{i\theta-\theta_{n}}\vert\,. $$ (5) Lemma 15.17 shows therefore that the integral in(3) is unchanged if $\scriptstyle{\mathcal{G}}$ is replaced by f. Comparison with (4) now gives (1) // Jensen's formula gives rise to an inequality which involves the boundary values of bounded holomorphic functions in $U$ (we recall that the class of these functions has been denoted by $H^{\mathrm{{c}}}$ "): 15.19 Theorem Iffe H",f not identically O, define $$ \mu_{r}(f)={\frac{1}{2\pi}}\prod_{-\pi}^{\pi}\log\left|f(r e^{i\theta})\right|\,d\theta\qquad(0<r<1) $$ (1) and $$ \mu^{\ast}(f)=\frac{1}{2\pi}\int_{-\pi}^{\pi}\log\mid f^{\ast}(e^{i\theta})\mid d\theta $$ (2) where f is the radial limit function off, as in Theorem 11.32.Then $$ \begin{array}{c c c}{{\mu_{r}(f)\leq\mu_{s}(f)}}&{{i f\quad0<r<s<1,}}\\ {{\mu_{r}(f)\to\log\vert f(0)\vert}}&{{a s\quad r\to0,}}\end{array} $$ (3) (4)ZEROs OF HOLOMORPHIC FUNCTiONs 309 and $$ \mu_{r}(f)\leq\mu^{*}(f)\quad i f\quad0<r<1. $$ (5) Note the following consequence: One can choose r so $\operatorname{that}f(z)\neq0$ if | z|=r: then p,f) is finite, and so is $\mu^{*}(f),$ by (5). Thus log $|f^{*}|\in L^{1}(T),$ and $f^{*}(e^{i\theta})\neq0$ at almost every point of ${\boldsymbol{T}}.$ PRoOF There is an integer $\scriptstyle m\geq0$ such $\mathrm{that}\,f(z)=z^{m}g(z),\,g\in H^{\infty},$ and $\scriptstyle{\theta(1)\neq}$ 0. Apply Jensen's formula _15.18(1) to $\scriptstyle{\mathcal{G}}$ in place of ${\boldsymbol{f}}.$ Its left side obviously cannot decrease if r increases. Thus $\mu_{*}(g)\leq\mu_{*}(g)\mathbf{i}$ ifr < s. Since $$ \mu_{r}(f)=\mu_{r}(g)+m\,\log r, $$ we have proved (3) Let us now assume, without loss of generality, that $|I|\leq1$ Write $f_{r}(e^{i\theta})$ log in place of $f(r e^{i\theta}).$ Then $f_{r} arrow f(0)$ as $\scriptstyle\gamma\to0.$ and $f_{r} arrow f^{\star}$ a.e. as $r\to1.$ Since $(1/|f_{r}|)\geq0,$ two applications of Fatou's lemma, combined with (3), give (4) and (5) // 15.20 Zeros of Entire Functions Suppose fis an entire function $$ M(r)=\operatorname*{sup}_{\theta}\left|\,f(r e^{i\theta})\right|\qquad(0<r<\infty), $$ (1) and ${\boldsymbol{n}}({\boldsymbol{r}})$ is the number of zeros of $\boldsymbol{\mathsf{f}}$ in ${\tilde{D}}(0;\,r).$ Assume $f(0)=1,$ for simplicity. Jensen's formula gives $$ M(2r)\geq\exp\left\{{\frac{1}{2\pi}}\int_{-\pi}^{\pi}\log\mid f(2r e^{i\theta})\mid d\theta\right\}=\prod_{n=1}^{n(2r)}{\frac{2r}{\mid\alpha_{n}\mid}}\geq\prod_{n=1}^{n(r)}{\frac{2r}{\mid\alpha_{n}\mid}}\geq2^{n(r)}. $$ if $\{\alpha_{n}\}$ is the sequence of zeros ${\mathrm{of}}f,$ arranged so that $\operatorname{tr}_{1}\leq|\alpha_{2}|\leq\cdots.$ Hence $$ n(r)\ \mathrm{log}\ 2\leq\log\,M(2r). $$ (2) $f)$ Thus the rapidity with which ${\boldsymbol{n}}({\boldsymbol{r}})$ can increase (i.e., the density of the zeros of is controlled by the rate of growth of $\scriptstyle M(r)$ Suppose, to look at a more specific situation, that for large r $$ M(r)<\exp\:\{A r^{k}\} $$ (3) where $\scriptstyle A$ and k are given positive numbers. Then (2) leads to $$ \operatorname*{lim}_{r\to\infty}\operatorname*{sup}_{U\otimes\ell\in T}{\leq k}. $$ (4) For example, if $\boldsymbol{k}$ is a positive integer and $$ f(z)=1-e^{z^{k}}, $$ (5)310 REAL AND coMPLEX ANALYsis then n(r) is about $\pi^{-1}k r^{k},$ so that $$ \operatorname*{lim}_{r\to\infty}{\frac{\log{n(r)}}{\log r}}=k. $$ (6) This shows that the estimate (4) cannot be improved Blaschke Products Jensen's formula makes it possible to determine the precise conditions which the zeros of a nonconstant f∈ $H^{\infty}$ must satisfy. 15.21 Theorem If $\{\alpha_{n}\}$ is a sequence in $U$ such tha $x_{n}\neq0$ and $$ \sum_{n=1}^{\infty}(1-|\alpha_{n}|)<\infty, $$ (1) if k is a nonnegative integer, and $\dot{f}$ $$ B(z)=z^{k}\prod_{n=1}^{\infty}{\frac{\alpha_{n}-z}{1-{\bar{\alpha}}_{n}z}}\,{\frac{|\alpha_{n}|}{\alpha_{n}}}\qquad(z\in U), $$ (2) then $B\in H^{\circ}.$ ,and $\boldsymbol{B}$ has no zeros except at the points $\propto_{n}$ (and at the origin, i $k>0$ 0) We call this function $\boldsymbol{B}$ a Blaschke product. Note that some of the $\textstyle x_{n}$ , may be repeated, in which case $\boldsymbol{B}$ has multiple zeros at those points. Note also that each factor in (2) has absolute value l on ${\boldsymbol{T}}.$ The term“Blaschke product”will also be used if there are only finitely many factors, and even if there are none, in which case $B(z)=1.$ PROOF The nth term in the series $$ \sum_{n=1}^{\infty}\left|1-{\frac{\alpha_{n}-z}{1-{\bar{\alpha}}_{n}z}}\cdot{\frac{|\alpha_{n}|}{\alpha_{n}}}\right| $$ is $$ \lfloor{\frac{\alpha_{n}+\lfloor\alpha_{n}\rfloor z}{(1-\bar{\alpha}_{n}z)\alpha_{n}}} \rfloor(1-\rfloor\alpha_{n}\rfloor)\leq{\frac{1+r}{1-r}}\left(1-\vert\alpha_{n}\vert\right) $$ if $|z|\leq r,$ Hence Theorem 15.6 shows that $B\in H(U)$ and that $\boldsymbol{B}$ B has only the ${\boldsymbol{U}},$ prescribed zeros. Since each factor in (2) has absolute value less than 1 in it follows that $|B(z)|<1,$ and the proof is complete. // 15.22 The preceding theorem shows that $$ \sum_{n=1}^{\infty}\left(1\,-\,\left|\,\alpha_{n}\,\right|\,\right)\,<\,\infty $$ (1)ZEROS OF HOLOMORPHIC FUNCTIONS 311 is a sufficient condition for the existence of an f∈ $H^{\infty}$ which has only the presc- ribed zeros {α,}. This condition also turns out to be necessary: $I f\in H^{\infty}$ and fis not identically zero, the zeros of f must satisfy (1). This is a special case of Theorem 15.23. It is interesting that (1) is a necessary condition in a much larger class of functions, which we now describe We let ${\mathbf{}}N$ For any real number t, define $\log^{+}t=\log t{\mathrm{~if~}}t\geq1$ and logt $\scriptstyle t\,=\,0$ if $t<1.$ (for Nevanlinna) be the class of all f ∈ $H(U)$ for which $$ \operatorname*{sup}_{0<r<1}\frac{1}{2\pi} |\O_{-\pi}^{\pi}\log^{+}\ |f(r^{i\delta})|\ d\theta<\infty. $$ (2) It is clear that $H^{\omega}\in N.$ Note that (2) imposes a restriction on the rate of growth of lf(z)| as |z|→1, whereas the boundedness of the integrals $$ {\frac{1}{2\pi}}\int_{-\pi}^{\pi}\log\left|\,f(r e^{i\theta})\right|\,d\theta $$ (3) imposes no such restriction. For instance,(3) is independent of r if $f=e^{g}$ for any $g\in H(U),$ The point is that (3) can stay small because log $|f|$ assumes large nega- will tive values as well as large positive ones, whereas log t $|f|\geq0.$ The class ${\mathbf{}}N$ be discussed further in Chap. 17. 15.23 Theorem Suppose fe $\scriptstyle\mathrm{V}_{S}$ is not identically O in $U,$ and α $1, .x_{2}, .x_{3},\ldots a r e$ the zeros off, listed according to their multiplicities. Then $$ \sum_{n=1}^{\infty}\,(1-|\alpha_{n}|)<\infty. $$ (1) (We tacitly assume that $\boldsymbol{\mathsf{f}}$ has infinitely many zeros in ${\cal U}.$ If there are only finitely many, the above sum has only finitely many terms, and there is nothing to prove. Also, $\left|\alpha_{n}\right|\leq\left|\alpha_{n+1} |.\right|.$ in and $\scriptstyle{\mathcal{G}}$ ${\tilde{D}}(0;r),\,{\mathrm{fix}}\,k,$ and take PROOF Iff has a zero of order m at the origin, and $g(z)=z^{-m}f(z),$ then $g\in N,$ without loss of generality $\scriptstyle r\times1$ so that $\,n(r)>k.$ except at the origin. Hence we may assume has the same zeros as ${\mathfrak{f}},$ ${\mathrm{that}}\,f(0)\neq0.$ Let n(r) be the number of zeros of f Then Jensen's formula $$ |f(0)|\prod_{n=1}^{n(r)}\frac{r}{|\alpha_{n}|}=\exp\left. \lbrace\frac{1}{2\pi}\int_{-\pi}^{\pi}\log |f(r e^{i\theta})\right|\,d\theta \rbrace $$ (2) implies that $$ |f(0)|\prod_{n=1}^{k}{\frac{r}{|\alpha_{n}|}}\leq\exp\left\{{\frac{1}{2\pi}}\int_{-\pi}^{\pi}\log^{+}\,|f(r^{i\theta})|\,d\theta\right\}. $$ (3)312 REAL AND coMPLEX ANALYSIs Our assumption that $f\in N$ is equivalent to the existence of a constant C < oo which exceeds the right side of (3) for all $r,0<r<1.$ It follows that $$ \prod_{n=1}^{k}|\alpha_{n}|\geq C^{-1}|f(0)|r^{k}. $$ (4) The inequality persists, for every $k_{\mathrm{{,}}}$ as $r\to1.$ Hence $$ \prod_{n=1}^{\infty}|\alpha_{n}|\geq C^{-1}|f(0)|>0. $$ (5) By Theorem 15.5,(5) implies (1) // Corollary Iff e $H^{\infty}$ (or even iff∈ N),f α,αz,α3,,…….are the zeros of f in U and jf E $\therefore(1-|\alpha_{n}|)=\circ,t h e n f(z)=0f o r\;a l l\;z\in U.$ For instance, no nonconstant bounded holomorphic function in $U$ can have a zero at each of the points $(n-1)/n\,(n=1,\,2,\,3,\,\ldots).$ We conclude this section with a theorem which describes the behavior of a radial limits Blaschke product near the boundary of ${\boldsymbol{U}}.$ Recall that as a member of $H^{\infty}.$ $\boldsymbol{B}$ B has $B^{*}(e^{i\theta})$ at almost all points of ${\boldsymbol{T}}.$ 15.24 Theorem If $\boldsymbol{B}$ is a Blaschke product, ther $\left|B^{*}(e^{i\theta})\right|=1$ a.e. and $$ \operatorname*{lim}_{r arrow1}\frac{1}{2\pi}\int_{-\pi}^{\pi}\log\left|{\cal B}(r e^{i\theta})\right|\,d\theta=0. $$ (1) PRoo The existence of the limit is a consequence of the fact that the integral is a monotonic function of r. Suppose $\scriptstyle B(t)$ is as in Theorem 15.21, and put $$ B_{N}(z)=\prod_{n=N}^{\infty}{\frac{\alpha_{n}-z}{1-\bar{\alpha}_{n}z}}\cdot{\frac{|\alpha_{n}|}{\alpha_{n}}} $$ (2) Since log $(\mid B/B_{N}\mid)$ is continuous in an open set containing ${\boldsymbol{T}},$ the limit (1) is $B_{N}$ we unchanged if $\boldsymbol{B}$ is replaced by $B_{N}.$ .If we apply Theorem 15.19 to therefore obtain 1c $$ \mathrm{og}\mid B_{N}(0)|\leq\operatorname*{lim}_{r\to1}{\frac{1}{2\pi}}\prod_{-\pi}^{\pi}\log\mid B(r^{i\theta})|\ d\theta\leq{\frac{1}{2\pi}}\int_{-\pi}^{\pi}\log\mid B^{*}(e^{i\theta})|\ d\theta\leq0. $$ (3) As $N\to\alpha.$ the first term in(3) tends to 0. This gives (1), and shows that // io $|\log|B^{*}|=0$ Since log $1.B^{n}1\leq0$ a.e., Theorem 1.39(a) now implies tha $\mathbf{g}\,|\,B^{*}\,|=0$ a.e. The Miintz-Szasz Theorem 15.25 A classical theorem of Weierstrass ([26],Theorem 7.26) states that the polynomials are dense in C(I), the space of all continuous complex functions onZEROs OF HOLOMORPHIC FUNCTIONs 313 the closed interval $\scriptstyle I=10$ 1], with the supremum norm. In other words, the set of all finite linear combinations of the functions $$ 1,t,t^{2},t^{3},\ldots $$ (1) is dense in $\mathrm{c}(t)$ This is sometimes expressed by saying that the functions (1) span C(D. This suggests a question: $$ \operatorname{P}0<\lambda_{1}<\lambda_{2}<\lambda_{3}<\cdots, $$ under what conditions is it true that the functions $$ 1,t^{43},t^{42},t^{43},\ldots $$ (2) span $\mathbb{C}[b)^{2}$ It turns out that this problem has a very natural connection with the problem of the distribution of the zeros of a bounded holomorphic function in a half plane (or in a disc; the two are conformally equivalent). The surprisingly neat answer is that the functions (2) span C(D) if and only if $\Sigma1/\lambda_{n}=\infty.$ Actually, the proof gives an even more precise conclusion: 15.26 Theorem Suppose $0<\lambda_{1}<\lambda_{2}<\lambda_{3}<\cdots$ and let $X$ be the closure in C(I) of the set of all fnite linear combinations of the functions $$ 1,\,t^{k_{1}},\,t^{3_{2}},\,t^{43},\,.... $$ (a) If E1/入,= 00, then $X=C(I).$ $\scriptstyle\lambda\neq^{n}0.$ then $X$ does not contain the function (の if E1/X.<oo, and i扩入生{2} $t^{\lambda}.$ PROOF ${\mathrm{It}}$ is a consequence of the Hahn-Banach theorem (Theorem 5.19) that $\varphi\in C(I)$ but 生 $X$ if and only if there is a bounded linear functional on $\operatorname{c}(t)$ which does not vanish at $\varphi$ but which vanishes on all of $X.$ Since every bounded linear functional on ${\boldsymbol{I}},$ 1。 (a) will be a consequence of the following pro- $\mathbb{C}(t)$ is given by integration with respect to a complex Borel measure on position: $I f\Sigma{1}/\lambda_{n}=\infty$ and if pu is a complex Borel measure on ${\mathbf I}$ such that $$ \{t^{\lambda_{n}}\,d\mu(t)=0\qquad(n=1,\,2,\,3,\,\ldots), $$ (1) then also $$ \{t^{k}\,d\mu(t)=0\qquad(k=1,\,2,\,3,\,\ldots).\qquad\qquad\qquad\qquad\qquad\qquad(k=1,\,2,\,3,\,\ldots). $$ (2) For if this is proved, the preceding remark shows that $\textstyle X{\ ~}$ contains all functions t; since ${\mathfrak{k}}\notin X.$ all polynomials are then in $X,$ , and the Weierstrass theorem therefore implies that $X=C(I).$314 REAL AND coMPLEX ANALYsis So assume that (1) holds. Since the integrands in (1) and (2) vanish at ${\boldsymbol{0}},$ we may as well assume that ${\boldsymbol{\mu}}$ u is concentrated on $(\mathbf{0},$ 1]. We associate with ${\boldsymbol{\mu}}$ the function $$ f(z)=\left\{t^{z}\;d\mu(t).\right. $$ (3) For $t>0,\ t^{z}=\exp (z$ log t). by definition. We claim that $\boldsymbol{\f}$ is holo- morphic in the right half plane. The continuity of f is easily checked, and we (1) says $\mathrm{that}f(\lambda_{n})=0.$ can then apply Morera's theorem. Furthermore, if Thus $\boldsymbol{\f}$ is bounded in the right half plane, and $\scriptstyle x\,>0.$ and if $z=x+i y,$ if $0<t\leq1$ then $|t^{x}|=t^{x}\leq1.$ 2,3 …Dfine , for $n=1,$ $$ g(z)=f{\biggl(}{\frac{1+z}{1-z}}{\biggr)}\qquad(z\in U). $$ (4) $f(k)=0$ for $k=1,$ 2, $\Sigma(1-|\alpha_{n}|)=\infty$ i $x_{n}=(\lambda_{n}-1)/(\lambda_{n}+1).$ A simple computa- Then ${\mathfrak{g}}\in H^{\star}$ and $g(x_{n})=0,$ where $\Sigma1/\lambda_{n}=\infty.$ The Corollary to Theorem tion shows that 15.23 therefore tells us that $\theta(z)=0$ for all $z\in U.$ Hence $f=0.$ In particular, $\mathbb{s...}$ and this is (2). We have thus proved part (a) of the theorem To prove (b) it will be enough to construct a measure p ${\boldsymbol{\mu}}$ u on ${\mathbf I}$ such that (3) defines a function $\boldsymbol{\f}$ which is holomorphic in the half plane ${\boldsymbol{0}},$ $\lambda_{1},\,\lambda_{2},\,\lambda_{3},\,\ldots$ and which $\operatorname{Re}z>-1$ (anything negative would do here), which is O at has no other zeros in this half plane. For the functional induced by this measure ${\boldsymbol{\mu}}$ will then vanish on $X$ but will not vanish at any function $t^{\lambda}$ if $\mathrm{~ |~\cdot~}\neq0\;\mathrm{and}\;\lambda\;\phi\;\mathrm{ \{~}\lambda_{n} \rangle$ which has these prescribed zeros, We begin by constructing a function $\boldsymbol{\f}$ and we shall then show that this f can be represented in the form (3). Define $$ f(z)={\frac{z}{(2+z)^{3}}}\;\prod_{n=1}^{\infty}{\frac{\lambda_{n}-z}{2+\lambda_{n}+z}}. $$ (5) Since $$ 1-\frac{\lambda_{n}-z}{2+\lambda_{n}+z}=\frac{2z+2}{2+\lambda_{n}+z}, $$ the infinite product in((5) converges uniformly on every compact set which 0,入,人z,入3, tion in the whole plane, with poles at $-2$ and $-\lambda_{n}-2,$ follows that fis a meromorphic func and with zeros at contains none of the points $-\lambda_{n}-2.$ .. Also, each factor in the infinite product (S) is less than 1 in $(2+z)^{3}$ absolute value if R ${\mathfrak{C}}\ z>-1.$ Thus $|f(z)|\leq1$ if Re $z\geq-1.$ The factor $\scriptstyle{f(t)}$ ensures that the restriction of f to the line $\operatorname{Re}z=-1$ is in $L^{1}.$ Fix z so that Re $z>-1.$ and consider the Cauchy formula for where the path of integration consists of the semicircle with center at $-1,$ radius $R>1+|z|,$ from $-1-i R$ to $-1+R$ to ${\mathrm{i}}+i R,$ followed by theZEROS OF HOLOMORPHIC FUNCTIONs 315 interval from $-1+i R\ {\mathrm{to}}\ -1-i R.$ The integral over the semicircle tends to $\mathbf{0}$ as $R\to\infty,$ so we are left with $$ f(z)=-\frac{1}{2\pi} |_{-\infty}^{\infty}\frac{f(-1+i s)}{-1+i s-z}\,d s\qquad(\mathrm{Re}\;z>-1). $$ (6) But $$ {\frac{1}{1+z-i s}}=\bigcap_{0}^{1}t^{z-i s}\,d t\qquad(\mathrm{Re}\,z>-1). $$ (7) Hence (6) can be rewritten in the form $$ f(z)=\int_{0}^{1}t^{z}{\sqrt{\frac{1}{2\pi}}}\int_{-\infty}^{\infty}f(-1+i s)e^{-i s\log t}\,d s{\Biggr\}}\,d t. $$ (8) The interchange in the order of integration was legitimate: If the integrand in (8) is replaced by its absolute value, a finite integral results Put $g(s)=f(-1+i s).$ Then the inner integral in(8) is. Gdlog t), where $\hat{\mathcal{G}}$ G 1S the Fourier transform of ${\mathfrak{g}}.$ . This is a bounded continuous function on (0,1], and if we set $d\mu(t)={\dot{g}}(\log\,t)$ dt we obtain a measure which represents f in the desired form (3). This completes the proof. // 15.27 Remark The theorem implies that whenever $\left\{1,\,t^{\lambda_{1}},\,t^{\lambda_{2}},\,\ldots\right\}$ spans C(I), then some infinite subcollection of the $t^{\lambda_{i}}$ M can be removed without altering the span. In particular, C(I) contains no minimal spanning sets of this type. This is in marked contrast to the behavior of orthonormal sets in a Hilbert space: if any element is removed from an orthonormal set, its span is dimin- ished. Likewise,if {1 $t^{\lambda_{1}},$ $t^{\lambda_{2}}\cdot\cdot\cdot\ \}$ does not span $\mathbb{C}(t)_{i}$ removal of any of its elements will diminish the span; this follows from Theorem 15.26(b). Exercises I Suppose $\{{\cal Q}_{n}\}$ and $\left\{D_{n}\right\}$ are sequences of complex numbers such that $\Sigma|a_{n}-b_{n}|<\infty$ On what sets will the roduct $$ \prod_{n=1}^{\infty}{\frac{z-a_{n}}{z-b_{n}}} $$ converge uniformly? Where willit define a holomorphic function ? 2 Suppose fis entire,Ais a positive number, and the inequality $$ |f(z)|<\exp{(|z|^{4})} $$ holds for all large enough |z|.(Such functions $\boldsymbol{\mathit{f}}$ are said to be of finite order. The greatest lower prove that the bound of all 入 for which the above condition holds is the order of f.) $\operatorname{If}f(z)=\Sigma a_{n}z^{n}.$ inequality $$ \left|a_{n}\right|\leq\left({\frac{e\lambda}{n}}\right)^{m/4} $$316 REAL AND COMPLEX ANALYSIS holds for all arge enough n. Consider the functions exp $(z^{k}),\,k=1,$ 2,3, .…,to determine whether the above bound on $|a_{n}|\cdot$ is close to best possible. 3 Find all complex z for which exp (exp $(z)=1.$ Sketch them as points in the plane. Show that there $\;f\equiv0,$ is no entire function of finite order which has a zero at each of these points (except, of course 4 Show that the function $$ \pi_{\mathrm{\scriptsize~COI~}}\pi z=\pi i\,\frac{e^{\pi i z}+e^{-\pi i z}}{e^{\pi i z}-e^{-\pi i z}} $$ has a simple pole with residue l at each integer. The same is true of the function $$ f(z)={\frac{1}{z}}+\sum_{n=1}^{\infty}{\frac{2z}{z^{2}-n^{2}}}=\operatorname*{lim}_{N\to\infty}{\sum_{n=-N}^{N}{\frac{1}{z-n}}}. $$ function, hence a constant, and that this constant is actually ${\mathfrak{O}},$ , that their diference is a bounded entire Show that both functions are periodic $[f(z+1)=f(z)]$ ), since $$ \operatorname*{lim}_{y\to\infty}f(i y)=-2i\int_{0}^{\infty}\frac{d t}{1+t^{2}}=-\pi\mathrm{i}. $$ This gives the partial fractions decomposition $$ \pi_{\mathrm{\scriptsize~COI~}}\pi z=\frac{1}{z}+\sum_{1}^{\infty}\frac{2z}{z^{2}-n^{2}}. $$ (Compare with Exercise 12, Chap. 9.) Note that $\scriptstyle\pi$ cot $\scriptstyle\pi z$ is (g'/gXz)if $g(z)=\sin\pi$ z. Deduce the product representation $$ {\frac{\sin\pi z}{\pi z}}=\prod_{n=1}^{\infty}\left(1-{\frac{z^{2}}{n^{2}}}\right) $$ and S Suppose $\boldsymbol{\hat{\Pi}}$ k is a positive integer, $\left\{z_{n}\right\}$ is a sequence of complex numbers such that $\Sigma\left|\,z_{n}\right|^{-k-1}<\infty,$ $$ f(z)=\prod_{n=1}^{\infty}E_{k}{\bigg(}{\frac{z}{z_{n}}}{\bigg)}. $$ (See Definition 15.7.) What can you say about the rate of growth of $$ M(r)=\operatorname*{max}_{\theta}|f(r e^{i\theta})|\ ? $$ Sec. 15.20.) 6 Suppose f is entire, f(0) ≠ 0 $|\cdot|f(z)|<\exp{(|z|^{p})}$ for large |z|, and $\{z_{n}\}$ is the sequence of zeros of , counted according to their multiplicities. Prove that $\textstyle\sum|z_{n}|^{-p-\epsilon}<\infty$ ifor every $\epsilon>0.$ (Compare with that T Suppose is an entire function $f({\sqrt{n}})=0$ for n $\mathbf{\tau}_{1}=1,2,3,\ldots$ and there is a positive constant α such for all $z^{\gamma}$ $|f(z)|<\exp_{-}|z|^{x},$ " for all large enough |zl. For which $\scriptstyle{\mathcal{X}}$ does it follow ${\mathrm{that}}\,f(z)=0$ [Consider sin (zz).] $^{\mathrm{S}}$ Let $\{z_{n}\}$ be a sequence of distinct complex numbers, $\mathbb{Z}_{\mathbb{H}}$ and which has no other poles. Ifz唯{z,}, let (z) be any path as $n\to\infty,$ and let $\{m_{n}\}$ simple pole with residue from 0to z which passes through none of the points $\mathbb{Z}_{\mathbb{H}}\,,$ $z_{n}\neq0,$ such that $\mathbb{Z}_{n}\longrightarrow\mathbb{Q}0$ be a sequence of positive integers. Let g be a meromorphic function in the plane, which has a $m_{n}$ at each and define $$ f(z)=\exp\left\{\int_{\gamma(z)}g(\zeta)\ d\zeta\right\}. $$ZEROS OF HOLOMORPHIC FUNCTIONs 317 theorem. and that the extension of f has a zero of order $m_{_{n}}\,\mathrm{{d}}\Gamma\ z_{_{n}}\,.$ is independent of the choice of yz) (although the integral itself is not), that fis $\mathbb{Z}_{\mathbb{R}}\,,$ Prove $\operatorname{that}f(z)$ $\left\{{\mathcal{Z}}_{n}\right\}_{1}$ that f has a removable singularity at each of the points holomorphic in the complement of The existence theorem contained in Theorem 15.9 can thus be deduced from the Mittag-Lefle 9 Suppose $0<\alpha<1,\,0<\beta<1,f\in H(U),f(U)\subset U,$ $(a)\propto={\frac{1}{2}},$ $\beta={\frac{1}{2}};\left(b\right)\alpha={\frac{1}{4}}$ $\beta={\frac{1}{2}};$ (c)α =系,β= };(d)α= 1/1,000 disc D(0; $\beta\gamma$ What is the answer if and f(0) = . How many zeros can fhave in the $\beta=1/10^{\gamma}$ 10 For $N=1,2,3,\dots,$ define $$ g_{N}(z)=\prod_{n=N}^{\infty}\left(1-{\frac{z^{2}}{n^{2}}}\right) $$ Prove that the ideal generated by $\{g_{N}\}$ in the ring of entire functions is not a principal ideal ${\mathfrak{I}}{\mathfrak{I}}$ Under what conditions on a sequence of real numbers $y_{n}$ does there exist a bounded holomorphic $1+i y_{n};$ In particular, can this happen if (a) function in the open right half plane which is not identically zero but which has a zero at each point $y_{n}=\log n,\left({\right)}\,y_{n}={\sqrt{n}},\left(c\right)y_{n}=n,\left(d\right)\,y_{n}=n^{2} )$ Let $\boldsymbol{E}$ a pole at each point of ${\boldsymbol{E}}.$ $0<|\alpha_{n}|<1,\Sigma(1-|\alpha_{n}|)<\infty,$ and $\bar{\boldsymbol{B}}$ is the Blaschke product with zeros at the points ${\mathcal{Q}}_{\mathfrak{R}}\,.$ 12 Suppose be the set of all points $1/{\tilde{\alpha}}_{n}$ and let $\underline{{\Omega}}$ be the complement of the closure of 2,so that $B\in H(\Omega),$ Prove that the $\bar{\boldsymbol{B}}$ B has ${\boldsymbol{E}}.$ product actually converges uniformly on every compact subset of $\Omega,$ and that I (This is of particular interest in those cases in which Q is connected.) $13$ Put $\alpha_{n}=1-n^{-2},$ for $n=1,2,3,\ldots.$ let $\bar{\boldsymbol{B}}$ be the Blaschke product with zeros at these points α and prove t ${\mathrm{hat~lim}}_{r arrow1}B(r)=0.({\mathrm{It~is~wr}}$ iderstood that $0<r<1.)$ More precisely, show that the estimate $$ |B(r)|<\prod_{1}^{N-1}\frac{r-\alpha_{n}}{1-\alpha_{n}r}<\prod_{1}^{N-1}\frac{\alpha_{N}-\alpha_{n}}{1-\alpha_{n}}<2e^{-N/3} $$ is valid if $\alpha_{N-1}<r<\alpha_{N}$ 14 Prove that there is a sequence $\left\{x_{n}\right\}$ with $0<\alpha_{n}<1,$ which tends to 1 so rapidly that the Blaschke product with zeros at the points $\alpha_{n}$ satisfies the condition $$ \operatorname*{lim}_{r\to1}\operatorname*{sup}_{P\to1}\mid B(r)|=1. $$ Hence this $\bar{\boldsymbol{B}}$ has no radial limit at $\mathbf{z}=1.$ $15$ Let $\scriptstyle{\varphi}$ g-orbit of z to be the set $\{\varphi_{n}(z)\},$ be a linear fractional transformation which maps $\boldsymbol{\mathit{U}}$ onto ${\boldsymbol{U}}.$ For any $z\in U$ define the where $\varphi_{0}(z)=z,\;\varphi_{n}(z)=\varphi(\varphi_{n-1}(z)),\,n=1,$ 2, $3,\ldots.$ Ignore the case $\varphi(z)=2$ (a) For which $\textstyle{\varphi}$ is it true that the ${\mathcal{\phi}}^{.}$ -orbits satisfy the Blaschke condition $\Sigma(1-|\varphi_{n}(z)|)<\infty^{\gamma}$ [The answer depends in part on the location of the fixed points of p. There may be one fixed point in ${\boldsymbol{U}},$ or one fixed point on ${\boldsymbol{T}},$ or two fixed points on ${\boldsymbol{T}}.$ In the last two cases it is advantageous to transfer the problem to (say) the upper half plane, and to consider transformations on it which either leave only oo fixed or leave $\mathbf{\partial}$ and o fixed.] which are invariant under op, ie. (b) For which $\mathcal{\varphi}$ do there exist nonconstant functions fe $H^{\omega}$ which satisfy the relation $f(\varphi(z))=f(z)$ for all $z\in U^{\gamma}$ 16 Suppose $|\alpha_{1}|\leq|\alpha_{2}|\leq|\alpha_{3}|\leq\cdots<1$ , and let Ir) be the number oft terms in the sequence $\{\alpha_{j}\}$ such that lα,|≤r. Prove that $$ \bigcap_{0}^{1}n(r)\ d r=\sum_{j=1}^{\infty}(1-|\alpha_{j}|). $$ 17 If $B(z)=\Sigma c_{k}z^{k}$ is a Blaschke product with at least one zero off the origin, is it possible to have $c_{k}\geq0$ for $k=0,1,2,\ldots\gamma$318 REAL AND cOMPLEX ANALYSIs 18 Suppose Bis a Blaschke product all of whose zeros lie on the segment (O, 1) and $$ f(z)=(z-1)^{2}B(z). $$ Prove that the derivative of fis bounded in ${\boldsymbol{U}},$ 19 Putf(Gz) = exp [-(1 + z)/1 - 2)]. Using the notation of Theorem 15.19, show that $$ \operatorname*{lim}_{r\to1}\mu_{r}(f)<\mu^{*}(f), $$ although e $H^{\infty}.$ Note the contrast with Theorem 15.24 in the Miintz-Szasz theorem. What is the conclusion of the 20 Suppose $\lambda_{1}>\lambda_{2}>\cdots.$ and $\lambda_{n} arrow0$ theorem, under these conditions? 21 Prove an analogue of the Mintz-Szasz theorem, with $L^{2}(I)$ in place of $C(I).$ 22 PL of the functions ${\mathbf{}}f_{n}$ is dense in L(O, o) Hint: If g e $L^{2}(0,$ and prove that the set ofall fnite linear combinations f, and i ${}^{\mathrm{t}}f_{n}(t)=t^{n}e^{-t}\left(0\leq t<\infty,n=0,1,2,\ldots\right)$ o) is orthogonal to each $f_{n}$ $$ F(z)= \{\O_{0}^{\infty}e^{-\imath z}\overline{{{g(t)}}}\ d t\qquad(\mathrm{Re}~z>0), $$ then all derivatives of F are O at z = 1. Consider $F(1+i y).$ 23 Suppose $\Omega\supset\bar{U},f\in H(\Omega),\,|\,f(e^{i\theta})\,|\geq3$ for all real ${}^{n},f(0)=0,$ and $\lambda_{1},$ , $,{\lambda}_{N}$ are the zeros of ${1-f i n}\,U,$ counted according to their multiplicities. Prove that $$ \left|\lambda_{1}\lambda_{2}\cdots\lambda_{N}\right|<{\frac{1}{2}}. $$ Suggestion: Look at $B/(1-f),$ where $\bar{\boldsymbol{B}}$ is a certain Blaschke product.