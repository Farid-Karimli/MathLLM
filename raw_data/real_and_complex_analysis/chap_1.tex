CHAPTER ONE ABSTRACT INTEGRATION Toward the end of the nineteenth century it became clear to many mathemati- cians that the Riemann integral (about which one learns in calculus courses) should be replaced by some other type of integral, more general and more flex- ible, better suited for dealing with limit processes. Among the attempts made in this direction, the most notable ones were due to Jordan, Borel w H. Young. and Lebesgue. It was Lebesgue's construction which turned out to be the most successful In brief outline, here is the main idea: The Riemann integral of a function $\boldsymbol{\mathsf{f}}$ over an interval [a, b] can be approximated by sums of the form $$ \sum_{i=1}^{n}f(t_{i})m(E_{i}) $$ length of $E_{i},$ and $t_{i}\in E_{i}$ for $i=1,$ E, are disjoint intervals whose union is [a,b], $\scriptstyle m(z_{3})$ denotes the where $E_{1},\,\ldots,\,F$ .…,n. Lebesgue discovered that a completely satisfactory theory of integration results if the sets $\textstyle E_{i}$ in the above sum are allowed to belong to a larger class of subsets of the line,the so-called “measurable sets,”and if the class of functions under consideration is enlarged to what he called“measurable functions.”The crucial set-theoretic properties involved are the following: The union and the intersection of any countable family of measurable sets are measurable; so is the complement of every measur able set; and, most important, the notion of“length”(now called“measure”) can be extended to them in such a way that $$ m(E_{1}\cup E_{2}\cup E_{3}\cup\dots)=m(E_{1})+m(E_{2})+m(E_{3})+\cdots $$ 56 REAL AND COMPLEX ANALYSiS for every countable collection {E;} of pairwise disjoint measurable sets. This pro perty of m is called countable additivity The passage from Riemann's theory of integration to that of Lebesque is a process of completion (in a sense which will appear more precisely later). It is of the same fundamental importance in analysis as is the construction of the real number system from the rationals. The above-mentioned measure m is of course intimately related to the geometry of the real line. In this chapter we shall present an abstract (axiomatic) version of the Lebesgue integral, relative to any countably additive measure on any set. (The precise definitions follow.) This abstract theory is not in any way more difficult than the special case of the real line; it shows that a large part o integration theory is independent of any geometry (or topology) of the underlying space; and, of course, it gives us a tool of much wider applicability. The existence of a large class of measures, among them that of Lebesgue, will be established in Chap. 2. Set-Theoretic Notations and Terminology 1.1 Some sets can be described by listing their members. Thus $\{x_{1},\ldots,x_{n}\}$ is the set whose members are $X_{\mathrm{1}},\ \ \circ\ ,\ \ -\ ,\ \ \ X_{\mu}\sp{\circ},$ and $\left\{x\right\}$ is the set whose only member is $\scriptstyle{X.}$ More often, sets are described by properties. We write $$ \scriptstyle(x,\,P_{1} $$ for the set of all elements x which have the property ${\boldsymbol{P}}.$ The symbol O denotes the empty set. The words collection, family, and class will be used synonymously with set. We write $x\in A$ if $\scriptstyle{\mathcal{X}}$ is a member of the set A; otherwise x生 A.If $\boldsymbol{B}$ B is a subset of A, i.e., if If $\scriptstyle B\in{\mathcal{A}}$ and $A\neq B,$ B is a proper subset of A. Note that If $\scriptstyle B\in{\mathcal{A}}$ and $4\in B,$ then $\scriptstyle A=B$ $x\in B$ implies $x\in A.$ we write $\scriptstyle B\in{\mathcal{A}}$ for every $\mathbb{Q}\subset A$ set A. $A\cup B$ and A o B are the union and intersection of $\scriptstyle A$ and ${\boldsymbol{B}},$ respectively. If {A}is a collection of sets, where αruns through some index set I, we write $$ \begin{array}{c c c}{{\bigcup_{\alpha\,\in\,I}A_{\alpha}}}&{{\mathrm{~and~}}}&{{\bigcap A_{\alpha}}}\\ {{\bigcup_{\alpha\,\in\,I}A_{\alpha}}}&{{}}\end{array} $$ for the union and intersection of $\{4_{2}\};$ $$ \begin{array}{c}{{\bigcup_{\alpha:I_{\alpha}}\ {\cal A}_{\alpha}=\{x;\,x\in\ {\cal A}_{\alpha}{\mathrm{~for~at~one~}}\alpha\in I\}}}\\ {{\ \{\prod_{\alpha}{\cal A}_{\alpha}=\{x;\,x\in\ {\cal A}_{\alpha}{\mathrm{~for~every~}}\alpha\in I\}.}}\end{array} $$ If I is the set of all positive integers, the customary notations are $$ \bigcup_{n=1}^{\infty}A_{n}\quad{\mathrm{and}}\quad\bigcap_{n=1}^{\infty}A_{n}\,. $$ABSTRACT INTEGRATION 7 If no two members of{A} have an element in common, then{A} is a disjoint collection of sets: We write $A-B=\{x\colon x\in A,\,x\notin B\},$ , and denote the complement of $\scriptstyle A$ by $A^{\mathfrak{c}}$ whenever it is clear from the context with respect to which larger set the com plement is taken. The cartesian product $A_{1}\times\cdots\times A_{n}$ of the sets $A_{1},\ \ldots,\ A_{n}$ is the set of all ordered n-tuples $(a_{1},\ldots,a_{n})$ where $a_{i}\in A_{i}\mathrm{for}\,i_{.}=1,\ldots,n.$ , and The real line (or real number system) is $\textstyle R^{1},$ $$ R^{k}=R^{1}\times\cdots\times R^{1}\qquad(k\;\mathrm{factors}). $$ The extended real number system is $R^{1}$ with two symbols, O and -00, adjoined and with the obvious ordering. I $\mathrm{f-co}\leq a\leq b\leq\infty,$ the interval [a,b] and the segment (a, b) are defined to be $$ [a,\,b]=\{x;\,a\leq x\leq b\},\qquad(a,\,b)=\{x:a<x<b\}. $$ We also write $$ [a,b)=\{x;a\leq x<b\},\qquad(a,b\}=\{x;a<x\leq b\}. $$ If $E\<\,\lfloor-\infty,$ oo] and $E\neq{\mathcal{O}},$ the least upper bound (supremum) and great est lower bound (infimum) of $\boldsymbol{E}$ 中 exist in[-00,oo] and are denoted by sup $\boldsymbol{E}$ and inf ${\boldsymbol{E}}.$ Sometimes (but only when sup $E\in E)$ we write max $\boldsymbol{E}$ for sup $\textstyle E.$ The symbol $$ f\colon X\to Y $$ ${\cal{Y}};$ means that $\boldsymbol{\mathit{f}}$ assigns to each $\operatorname{xe}{\mathcal{X}}$ is a function (or mapping or transformation) of the set $X$ into the set the i.e., $\boldsymbol{\mathit{f}}$ an element $f(x)\in Y.$ If $4\in{\mathcal{X}}$ and $\scriptstyle B\in Y.$ image of $\scriptstyle A$ and the inverse image (or pre-image) of $\boldsymbol{B}$ are $$ f(A)=\{y\colon y=f(x){\mathrm{~for~some~}}x\in A\}, $$ $$ f^{-1}(B)=\{x;f(x)\in B\}. $$ Note that $f^{-1}(B)$ may be empty even when $\textstyle X$ onto ${\boldsymbol{Y}}.$ $B\neq\mathbb{Q}.$ $f^{-1}(y)$ consists of at The domain of fis $X.$ The range of f is f(X) for every y e Y. If $\operatorname{If}f(X)=Y,f\operatorname{is}$ said to map $f^{-1}(\{y\}),$ We write $f^{-1}(y),$ instead of s said to be one-to-one. If fis one-to-one, then most one point, for each $y\in Y,f\mathbb{I}$ $f^{\prime\dagger}$ is a function with domain $\scriptstyle I(X)$ and range $X.$ If f: X→[-, o] and $E\in X.$ it is customary to wite sup-egf(x) rather than sup f(E) and g $\mathbf{r} arrow\mathbf{Z},$ the composite function $g\circ f\colon X arrow Z$ is defined by irf 广 $x{\xrightarrow{}}Y$ the formula ( $$ g\circ f)(x)=g(f(x))\qquad(x\in X). $$3 REAL AND COMPLEX ANALYSIs If the range of f lies in the real line (or in the complex plane), then fis said to be a real function (or a complex function). For a complex function ${\boldsymbol{f}},$ the statement “f≥0”means that all values f(x) of f are nonnegative real numbers. The Concept of Measurability The class of measurable functions plays a fundamental role in integration theory It has some basic properties in common with another most important class of functions, namely, the continuous ones. It is helpful to keep these similarities in mind. Our presentation is therefore organized in such a way that the analogies between the concepts topological space, open set, and continuous function, on the one hand, and measurable space, measurable set, and measurable function, on the other, are strongly emphasized. It seems that the relations between these concepts emerge most clearly when the setting is quite abstract, and this (rather than a desire for mere generality) motivates our approach to the subject 1.2 Definition (a)A collection t of subsets of a set $\textstyle X$ is said to be a topology in $X$ if t has the following three properties: (i)O et and X e t. (i)If $V_{i}\in\tau\mathrm{~for~}i=1,\ldots,n,$ then $V_{1}\ \cap\ V_{2}\ \cap\ \cdots\ \cap\ V_{n}\in\tau.$ ii ir $\scriptstyle V_{\circ}\quad$ is an arbitrary collection of members of t (finite, countable, or uncountable), then $\bigcup_{\alpha}V_{\alpha}\in\tau.$ (b)If t is a topology in $X$ K, then $\textstyle X{\ ~}$ is called a topological space, and the members of r are called the open sets in $\textstyle X$ (c)If $X$ X and ${\mathbf{}}Y$ are topological spaces and if fis a mapping of $X$ into ${\boldsymbol{Y}},$ then $\boldsymbol{\f}$ is said to be continuous provided that $f^{-1}(V)$ is an open set in $X$ for every open set ${\mathbf{}}V$ in ${\boldsymbol{Y}}.$ 1.3 Definition (a)A collection の of subsets of a set $X$ is said to be a o-algebra in $\textstyle X$ if p has the following properties (i) X ∈0. (i) If A ∈ D, then $A^{\mathrm{c}}$ ∈ D, where $A^{\mathfrak{c}}$ is the complement of A relative to $X.$ ii)If A =UJ8=1 A, and if A, ∈ Dt for n = 1,2,3, .…., then A e D (b)If is a ${\boldsymbol{\sigma}}\cdot$ -algebra in $X,$ then $X$ X is called a measurable space, and the members of の are called the measurable sets in $X.$ (c) If $X$ is a measurable space, $\boldsymbol{\f}$ is said to be measurable provided that ${\boldsymbol{Y}}.$ $\boldsymbol{\f}$ is a mapping is a of $X$ measurable set in $X$ for every open set ${\mathbf{}}V$ in is a topological space, and $f^{-1}(V)$ into Y, then ${\mathbf{}}Y$ABSTRACT INTEGRATION 9 It would perhaps be more satisfactory to apply the term“measurable space” to the ordered pair (X, 90), rather than to $X.$ After all, $X$ is a set, and $X$ Y has not been changed in any way by the fact that we now also have a o-algebra of its subsets in mind. Similarly, a topological space is an ordered pair (X, rJ. But if this sort of thing were systematically done in all mathematics, the terminology would become awfully cumbersome. We shall discuss this again at somewhat greater length in Sec. 1.21. 1.4 Comments on Definition 1.2 The most familiar topological spaces are the metric spaces. We shall assume some familiarity with metric spaces but shall give the basic definitions, for the sake of completeness A metric space is a set $X$ in which a distance function (or metric) $\quad{\boldsymbol{\rho}}$ is defined with the following properties: (c) (d) p(x, (a)0≤ p(x, y)< oo for all $\scriptstyle{\mathcal{X}}$ and $\scriptstyle v\,\in\,X$ $z\in X$ (b)p(x, $y)=0$ if and only if $\scriptstyle x\;=\;y.$ $\scriptstyle v\,\in\,X$ $\rho(x,\,y)=\rho(y,\,x)$ for all $\scriptstyle{\mathcal{X}}$ and r all x, y, and $y)\leq\rho(x,z)+\rho(z,y)\,{\mathrm{fo}}$ If ${\mathfrak{x}}\in X$ and Property (d is called the triangle inequalit y and radius $\scriptstyle{\mathcal{I}}$ is the set $\scriptstyle\gamma\leq0.$ the open ball with center at $\scriptstyle{\mathcal{X}}$ Tf $X$ $\{y\in X\colon\rho(x,\,y)<r\}.$ is the collection of al sets $\scriptstyle{E\left(x\right)}$ which are is a metric space and if $\mathbb{Z}$ arbitrary unions of open balls,then t is a topology in $X.$ K. This is not hard to $B_{1}$ and $B_{2}$ verify; the intersection property depends on the fact that if $x\in B_{1}\cap B_{2}$ ,where are open balls, then $\scriptstyle{\mathcal{X}}$ is the center of an open bal $B\subset B_{1}\cap B_{2}$ We leave this as an exercise. For instance, in the real line $R^{1}$ a set is open if and only if it is a union of open segments (a, b). In the plane $R^{2}.$ the open sets are those which are unions of open circular discs. Another topological space,which we shall encounter frequently,is the extended real line [-00, oo]; its topology is defined by declaring the following sets to be open:(a,b),[-00,a), (a, oo], and any union of segments of this type The definition of continuity given in Sec. 1.2(c) is a global one. Frequently it is desirable to define continuity locally:A mapping f of $\textstyle X$ into ${\mathbf{}}Y$ is said to be continuous at the point $\scriptstyle c_{\mathrm{n}}\in X$ if to every neighborhood ${\mathbf{}}V$ ’ of $\scriptstyle f(x_{n})$ there corre- sponds a neighborhood $\textstyle W$ of $x_{0}$ such that f(W) c V. (A neighborhood of a point $\scriptstyle{\mathcal{X}}$ is, by definition, an open set which contains x.) When $\textstyle X{\mathrm{~}}$ and ${\mathbf{}}Y$ are metric spaces, this local definition is of course the same as the usual epsilon-delta definition, and is equivalent to the requirement that lim $f(x_{n})=f(x_{0})$ in ${\mathbf{}}Y$ whenever lim $x_{n}=x_{0}$ in $X.$ The following easy proposition relates the local and global definitions of con tinuity in the expected manner: 1.5 Proposition Let $\textstyle X{\mathrm{~}}$ and ${\bf Y_{\nu}}$ be topological spaces. A mapping f of $X$ Y into ${\mathbf{}}Y$ is continuous if and only iff is continuous at every point of $X.$10 REAL AND COMPLEX ANALYSiS PR0OF If $\boldsymbol{\f}$ is continuous and $x_{\mathrm{r}}\in X$ then $f^{-1}(V)$ is a neighborhood of $x_{0}\,\mathrm{:}$ is $f^{-1}(V).$ for every neighborhood ${\mathbf{}}V$ of f(xo). Since $f(f^{-1}(V))\subset V,$ it follows that $\boldsymbol{\f}$ is continuous at $\scriptstyle x_{0}$ $f^{-1}(V)$ is the union of the open sets $f(W_{\circ})\in V.$ Therefore $W_{x}\subset$ $x\in_{\mathcal{S}}^{-1}(V)$ If f is continuous at every point of $\textstyle X$ and if ${\mathbf{}}V$ is open in Y, every poin has a neighborhood $W_{x}$ such that It follows that $W_{x}.$ ,so $f^{-1}(V)$ // itself open. Thus fis continuous. 1.6 Comments on Definition 1.3 Let DR be a c-algebra in a set $X.$ Referring to Properties G to (i) of Definition 1.3(a), we immdiately derive thefllowing facts. (a) Since ${\mathcal{D}}=X^{c},({\mathrm{i}})$ and Gin imply that O e (c) (b) Taking $A_{n+1}=A_{n+2}=\mathbf{\cdot}\mathbf{\cdot}\mathbf{\cdot}\mathbf{\cdot}$ in (ii), we see that $A_{1}\cup\ A_{2}\cup\cdots\setminus A,$ e DR if $A_{i}\in\mathfrak{M}$ for $i=1,\dots,n.$ Since $$ \bigcap_{n=1}^{\infty}A_{n}={\biggl(}\bigcup_{n=1}^{\infty}A_{n}^{c}{\biggr)}^{c}, $$ DR is closed under the formation of countable (and also finite) intersec tions. and $B\in{\mathfrak{M}}$ (d)Since $A-B=B^{c}\cap\,,$ 4, we have $A-B\in{\mathfrak{M i f}}\,A\in{\mathfrak{M}}$ The prefix o refers to the fact that fii is required to hold for all countabl unions of members of R. If fii is required for finite unions only, then DR is called an algebra of sets. 1.7 Theorem Let Y and Z be topological spaces, and let g:Y $\scriptstyle\tau\to Z$ be contin- uous. (a)f $X$ is a topological space,ff $X\to Y$ is continuous, and j h = 9。f, then h: $x{\overset{\underset{\mathrm{a}}{}}{\to}}Z$ is continuous (b) If $X$ is a measurable space,iff: X→Y is measurable, and if h = g。f, then h: $x{\xrightarrow{}}\,Z$ is measurable Stated informally, continuous functions of continuous functions are contin- uous; continuous functions of measurable functions are measurable PRoOF If ${\mathbf{}}V$ is open in $\mathbb{Z},$ Z, then $g^{-1}(V)$ is open in ${\boldsymbol{Y}},$ and $$ h^{-1}(V)=f^{-1}(g^{-1}(V)). $$ If f is continuous, it follows that $h^{-1}{\sb(}V{)}$ is open, proving (a) // If f is measurable, it follows that $h^{-1}(V)$ is measurable, proving (b).ABSTRACT INTEGRATION 11 1.8 Theorem Let u and v be real measurable functions on a measurable space $X,$ let $\Phi$ be a continuous mapping of the plane into a topological space ${\boldsymbol{Y}},$ and define $$ h(x)=\Phi(\omega(x),\,v(x)) $$ for $\operatorname{xe}{\mathcal{X}}$ Then h: $X\to Y$ is measurable. PR00F ${\mathrm{Put}}\,f(x)=(u(x),\,v$ (x). Then $\boldsymbol{\mathsf{f}}$ maps $X$ into the plane. Since $h=\Phi\circ f,$ Theorem 1.7 shows that it is enough to prove the measurability of f If ${\boldsymbol{R}}$ is any open rectangle in the plane, with sides parallel to the axes then ${\boldsymbol{R}}$ is the cartesian product of two segments ${\mathit{I}}_{\downarrow}$ and ${\mathit{I}}_{2}\,,$ and $$ f^{-1}(R)=u^{-1}(I_{1})\cap v^{-1}(I_{2}), $$ which is measurable, by our assumption on $\boldsymbol{\ u}$ and ${\mathcal{U}}.$ Every open set ${\mathbf{}}V$ in the plane is a countable union of such rectangles $R_{i},$ and since $$ f^{-1}(V)=f^{-1}\!\left(\bigcup_{i=1}^{\infty}R_{i}\right)=\bigcup_{i=1}^{\infty}f^{-1}(R_{i}), $$ $f^{-1}(V)$ is measurable / 1.9 Let $\textstyle X$ be a measurable space. The following propositions are corollaries of Theorems 1.7 and 1.8: (a) $l f f=u+i v,$ where u and v are real measurable functions on $X,$ then $\boldsymbol{\f}$ is a complex measurable function on $X.$ (b) $\;U f=u+i v$ This follows from Theorem 1.8, with $X.$ $\Phi(z)=z.$ are real measurable functions on is a complex measurable function on $X,$ then u、D, and $|f|$ For real f and $\scriptstyle{\mathcal{G}}$ This follows from Theorem 1.7, with $g(z)=\operatorname{Re}(z),$ Im (z), and |z| g and fg (c)Iff and g are complex measurable functions on $X_{\cdot}$ , then so are $f{\boldsymbol{+\ g}}$ this follows from Theorem 1.8, with $$ \Phi(s,\,t)=s+t $$ and D(s, $t)=s t.$ The complex case then follows from (a) and (b) (d)If $\boldsymbol{E}$ is a measurable set in 】 $\textstyle X$ K and i $$ \gamma_{E}(x)={\biggl\{}{\bf l}\qquad\mathrm{if~}x\in E $$ $\chi_{E}$ is a measurable function the characteristic function of the set E. The then X This is obvious. We call $\chi_{E}$ (e) letter $\scriptstyle{\mathcal{X}}$ will be reserved for characteristic functions throughout this book. there is a complex measurable If f is a complex measurable function on $X,$ function α on $\textstyle X$ such that |α|= $1\ a n d f=\alpha|f|$12 REAL AND COMPLEX ANALYSIs removed, define $\varphi(z)=z/|z|\operatorname{for}z\in Y,$ and put be the complex plane with the origin PROOF Let $E=\{x:f(x)=0\},$ let ${\mathbf{}}Y$ $$ x(x)=\varphi(f(x)+\chi_{E}(x))\qquad(x\in X). $$ If $x\in E_{z}$ $x(x)=1;$ if $x\notin E,$ $x(x)=f(x)/|f(x)|$ Since $\varphi$ is continuous on ${\bf Y_{\nu}}$ and since ${\boldsymbol{E}}$ is measurable (why ?), the measurability of c follows from (c),(aA and Theorem 1.7. // We now show that ${\boldsymbol{\sigma}}.$ -algebras exist in great profusion. 1.10 Theorem If $\mathcal{F}$ F is any collection of subsets of $X,$ there exists a smallest o-algebra Dt in $X$ such that ${\mathcal{F}}\subset{\mathfrak{M}}^{*}.$ This Ot* is sometimes called the ${\boldsymbol{\sigma}}\cdot$ -algebra generated by ${\mathcal{F}},$ PROOF Let $\Omega$ be the family of all ${\boldsymbol{\sigma}}\cdot$ -algebras oD in $X$ which contain ${\mathcal{F}}.$ . Since the collection of all subsets of $X$ is such a o-algebra, $\Omega$ is not empty. Let Ot* be the intersection of all D e Q. It is clear that ${\mathcal{F}}\subset{\mathfrak{M}}^{*}$ and that ODt* lies in every o-algebra in $X$ which contains ${\mathcal{F}}.$ To complete the proof, we have to show that Ot* is itself a o-algebra. If $A_{n}\in\mathfrak{M}^{*}$ for $n=1,2,3,\ldots.$ and if D e Q, then A,∈ D, so $\bigcup A_{n}\in\mathfrak{M},$ since D is a o-algebra.Since $\bigcup{}_{A_{n}}\in\mathbb{N}$ for every Ot e Q,we conclude tha U A,∈ 0*. The other two defining properties of a o-algebra are verifed in the same manner. /// 1.11 Borel Sets Let $X$ be a topological space.By Theorem 1.10, there exists a smallest ${\boldsymbol{\sigma}}.$ -algebra ${\mathcal{A}}$ in $X$ such that every open set in $X$ belongs to 8. The members of ${\mathcal{A}}$ B are called the Borel sets of $\textstyle X$ In particular, closed sets are Borel sets (being, by definition, the complements of open sets)and so are all countable unions of closed sets and all countable intersections of open sets. These last two are called $F_{\sigma}{}^{\circ}$ o's and Gs's, respectively, and play a considerable role. The notation is due to Hausdorff. The letters ${\mathbf{}}F$ and ${\boldsymbol{G}}$ were used for closed and open sets, respectively, and o refers to union (Summe), $\delta$ to intersection (Durchschnitt). For example, every half-open interval [a,b) is a $G_{\delta}$ and an $F_{\sigma}$ in $R^{1}.$ Since $\mathcal{B}$ is a o-algebra, we may now regard $X$ as a measurable space, with the Borel sets playing the role of the measurable sets; more concisely, we consider the measurable space (X,9). If $f\colon X\to Y$ is a continuous mapping of X $X,$ , where ${\bf Y_{\nu}}$ Yis any topological space, then it is evident from the definitions that $f^{-1}(V)\in{\mathcal{B}}$ for every open set ${\mathbf{}}V$ in Y. In other words, every continuous mapping of X is Borel tions. measurable. Borel measurable mappings are often called Borel mappings, or Borel func-ABSTRACT INTEGRATION 13 1.12 Theorem Suppose D is a o-algebra in X, and ${\mathbf{}}Y$ is a topological space. Let f map $X$ into ${\boldsymbol{Y}}.$ (a)If Q is the collection of all sets $\scriptstyle{E\in Y}$ such that f-1(E) ∈ D, then Q is a G-algebra in ${\boldsymbol{Y}}.$ is a Borel set in Y,then f-1(E) e b)Iff is measurable and $\boldsymbol{E}$ (c)If $Y=[-\infty,$ o] and f-1((α,oo]) ∈ t for every real α, then f is measur able. (d)Iffis measurable,ü $\mathbb{Z}$ is a topological space,ig:Y→Z is a Borel mapping, and $i f h=g\circ f,$ then h: $x{\mathrel{ arrow}}Z$ is measurable. Part (c) is a frequently used criterion for the measurability of real-valued functions. (See also Exercise 3.) Note that (d generalizes Theorem 1.7(6) PROOF (a) follows from the relations $$ f^{-1}(Y)=X, $$ $$ f^{-1}(Y-A)=X-f^{-1}(A), $$ and $$ f^{-1}(A_{1}\ \cup\ A_{2}\ \cup\ \cdot\cdot\cdot)=f^{-1}(A_{1})\ \cup f^{-1}(A_{2})\ \cup\ \cdot\cdot\cdot. $$ To prove(b), let $\Omega$ be as in (a);the measurability of $\Omega$ contains all Borel $\Omega$ contains all open sets in ${\boldsymbol{Y}},$ and since $\Omega$ 2 is a o-algebra, $\boldsymbol{\mathit{f}}$ implies that sets in ${\boldsymbol{\zeta}}.$ To prove (c)let $\Omega$ be the collection of all $E<\operatorname{L}-\alpha,$ oo] such that $f^{-1}(E)\,\epsilon$ 0. Choose a real number $\alpha,$ and choose $\alpha_{n}<\alpha$ so that $\alpha_{n} arrow\alpha$ as $n\to c o.$ Since (α。 $\infty\}\in\Omega$ for each n, since $$ [-\infty,\,\alpha)=\bigcup_{n=1}^{\infty}\left[-\infty,\,\alpha_{n}\right]=\bigcup_{n=1}^{\infty}(\alpha_{n},\,\infty)^{c}, $$ and since (a) shows that $\Omega$ is a o-algebra, we see that $\mathbb{C}-\infty,\,\alpha)\in\Omega.$ The same is then true o $$ (\alpha,\,\beta)=[-\infty,\,\beta)\cap(\alpha,\,\alpha). $$ Since every open set in[-o0,o] is a countable union of segments of the above types, $\Omega$ contains every open set. Thus fis measurable. since To prove (d),let $\scriptstyle V\in Z$ be open. Then $g^{-1}(V)$ is a Borel set of ${\boldsymbol{Y}},$ and $$ h^{-1}(V)=f^{-1}(g^{-1}(V)), $$ (b) shows that $h^{-1}(V)\in{\mathfrak{M}}$ // 1.13 Definition Let $\{a_{n}\}$ be a sequence in[-00,o], and put $$ b_{k}=\operatorname*{sup}\;\{a_{k},\;a_{k+1},\;a_{k+2},\ldots\}\qquad\bigl(k=1,\,2,\,3,\,\ldots\bigr) $$ (1)14 REAL AND cOMPLEX ANALYSis and $$ \beta=\operatorname*{inf}\left\{b_{1},b_{2},b_{3},\ldots\right\}. $$ (2) We call $\beta$ the upper limit of $\{a_{n}\},$ and write $$ \beta=\operatorname*{lim}_{n\to\infty}\operatorname*{sup}_{a}a_{n}. $$ (3) The following properties are easily verified: First, $b_{1}\cong b_{2}\geq b_{3}\geq\cdots,$ so that $b_{1}\to\beta$ as $k\div\alpha;$ secondly, there is a subsequence $\{a_{n_{i}}\}$ of $\{a_{n}\}$ such that $a_{n_{i}} arrow\beta$ as $i\div\alpha_{*}$ and $\boldsymbol{\beta}$ is the largest number with this property The lower limit is defined analogously: simply interchange sup and inf in (1) and (2). Note that $$ \operatorname*{lim}_{n\to\infty}\operatorname{in}_{n}(\mathbf{a}_{n}=\mathbf{-}\operatorname*{lim}_{n\to\infty}\operatorname{sup}_{n\to\infty}\left(-a_{n}\right) $$ (4) If $\{a_{n}\}$ converges, then evidently $$ \operatorname*{lim}_{n\to\infty}\operatorname*{sup}_{a_{n}}=\operatorname*{lim}_{n\to\infty}\operatorname*{inf}_{a_{n}}=\operatorname*{lim}_{n\to\infty}a_{n}. $$ (5) Suppose $\{f_{n}\}$ is a sequence of extended-real functions on a set $X.$ Then sup ${\mathfrak{f}}_{n}$ and lim sup f, are the functions defined on $X$ by n n+ $$ {\biggl(}{\operatorname*{sup}}\,f_{n}{\biggr)}(x)=\operatorname*{sup}_{n}\,(f_{n}(x)), $$ (6) $$ {\biggl(}\operatorname*{lim}_{n\to\infty}\operatorname{sup}f_{n}{\biggr)}(x)=\operatorname*{lim}_{n\to\infty}\operatorname{sup}\,(f_{n}(x)). $$ (7) If $$ f(x)=\operatorname*{lim}_{n\to\infty}f_{n}(x), $$ (8) the limit being assumed to exist at every ${\mathfrak{x e}}\,X.$ then we call f the pointwise limit of the sequence $\{f_{n}\}.$ 1.14 Theorem $$ \begin{array}{c}{{I f f_{n}\colon X\to[-\infty,\;\infty]\ i s\;m e a s u r a b l e,f o r\ n=1,\;2,\;3,\ldots,a n d}}\\ {{}}&{{g=\operatorname*{sup}f_{n},\qquad h=\operatorname*{lim}_{n\to\infty}f_{n},}}\end{array} $$ then $\scriptstyle{\mathcal{G}}$ and $\boldsymbol{h}$ are measurable. $\scriptstyle{\mathcal{G}}$ PRo0F $g^{-1}((\alpha,\,\alpha0))=\bigcup_{n=1}^{\infty}\,f_{n}^{-1}((\alpha,\,\alpha0)).$ Hence Theorem 1.12(c) implies that is measurable. The same result holds of course with inf in place of sup, and since $$ h=\operatorname*{inf}_{k\geq1}{\binom{}{}}_{i\geq k}f_{i}^{}, $$ it follows that $\boldsymbol{h}$ is measurable //ABSTRACT INTEGRATION 15 Corollaries (a)The limit of every pointwise convergent sequence of complex measurable functions is measurable (b)Iff and $\scriptstyle{\mathcal{G}}$ are measurable (with range in[-00,o]), then so are max $\{J,\phi\}$ and min {f, g}. In particular, this is true of the functions $$ f^{+}=\operatorname*{max}\:\{f,0\}\quad\mathrm{and}\quad f^{-}=-\operatorname*{min}\:\{f,0\}. $$ We have 1.15 The above functions $f^{+}$ and f" are called the positive and negative parts of f as a $|f|=f^{+}+f^{-}$ and $f=f^{+}-f^{-},$ a standard representation of $\boldsymbol{\f}$ difference of two nonnegative functions, with a certain minimality property: Proposition $I f=g-h,g\geq0,$ and h≥ 0, then ft ≤g and f"≤ h PRooF $\scriptstyle{f\leq\theta}$ and $\scriptstyle0\,\leq\,q$ clearly implies max $\{f,0\}\leq g.$ // Simple Functions 1.16 Definition A complex function s on a measurable space $X$ whose range consists of only finitely many points will be called a simple function. Among these are the nonnegative simple functions, whose range is a finite subset of [O,oo). Note that we explicitly exclude o from the values of a simple func- tion. If $\alpha_{1},\,\ldots,\,\alpha_{n}$ are the distinct values of a simple function $S_{\mathrm{,}}$ and if we set $A_{i}=\{x\colon s(x)=\alpha_{i}\},$ then clearly $$ s=\sum_{i=1}^{n}\alpha_{i}\chi_{A_{i}}, $$ where $\zeta_{A_{i}}$ is the characteristic function of $A_{i},$ as defined in Sec. 1.9(d). is measurable It is also clear that s is measurable if and only if each of the sets $A_{i}$ 1.17 Theorem Let f: $X{ arrow}[0,$ oo] be measurable. There exist simple measur- able functions s ${\boldsymbol{S}}_{n}$ inon $X$ 【 such that (a) $0\leq_{\circ}s_{1}\leq_{\circ}s_{2}\leq\cdots\leq f_{\circ}~.\qquad\qquad\qquad-.$ ${\mathfrak{x}}\in X;$ (b)s,(x)→f(x) as n→O, for every PxOOF Put $\delta_{n}=2^{-n}.$ To each positive integer $\;n$ and each real number t cor- Define responds a unique integer $k=k_{\mathrm{s}}(t)$ that satisfies $k\delta_{n}\leq t<(k+1)\delta_{n}\,.$ $$ \varphi_{n}(t)={\binom{k_{n}(t)\delta_{n}}{n}}\quad{\textrm{i f}}\underbrace{0\leq t<n}_{n}. $$ (1)16 REAL AND coMPLEX ANALYSIs Each $\varphi_{n}$ , is then a Borel function on [O, oo] $$ t-\delta_{n}<\varphi_{n}(t)\leq t\qquad{\mathrm{if~}}0\leq t\leq n, $$ (2) $0\leq\varphi_{1}\leq\varphi_{2}\leq\cdots\leq t,$ and $\varphi_{n}(t)\to t$ as n→0,for every te[O,oo]. It follows that the functions $$ s_{n}=\varphi_{n}\circ f $$ (3) satisfy (a) and (b); they are measurable, by Theorem $1.12(d).$ // Elementary Properties of Measures 1.18 Definition (a)A positive measure is a function L, defined on a o-algebra D, whose range is in [0,oo] and which is countably additive. This means that if $\scriptstyle\{A_{i}\}$ is a disjoint countable collection of members of O, then $$ \mu{\biggl(}\bigcup_{i=1}^{\infty}A_{i}{\biggr)}=\sum_{i=1}^{\infty}\mu(A_{i}). $$ (1) Ae亚 To avoid trivialities, we shall also assume that $\mu(A)<\infty$ for at least one (b)A measure space is a measurable space which has a positive measure defined on the o-algebra of its measurable sets. (c)A complex measure is a complex-valued countably additive function defined on a o-algebra. Note:What we have called a positive measure is frequently just called a then ${\boldsymbol{\mu}}$ measure; we add the word“positive”for emphasis. If $\mu(E)=0$ for every $\boldsymbol{E}$ ∈ 领 u is a positive measure, by our definition. The value o is admissible for a positive measure; but when we talk of a complex measure ${\boldsymbol{\mu}}_{\l}$ it is understood that $\scriptstyle{\theta(t)}$ is a complex number, for every Ee の.The real measures form a subclass of the complex ones, of course. 1.19 Theorem Let p be a positive measure on a o-algebra D. Then (a)u(O) = 0 (b)从A U…U A,) = (A,) +… + A(A,)i A,.…, A,are pairwise disjoint members of Oe. (c)A e B implies u(A)≤ A(B)f A∈ D, Be D (d)u(A,)→A(A) as n→Ooü A = U8=1A,, A,∈观, and $$ A_{1}\subset A_{2}\subset A_{3}\subset\cdots. $$ (e)u(A,)→A(A) asn→oo i $$ A=\bigcap_{n=1}^{\infty}\,A_{n},\,A_{n}\in\mathfrak{D} $$ $$ A_{1}\supset A_{2}\to A_{3}\to\cdots, $$ and p(A)is finiteABSTRACT INTEGRATION 17 As the proof will show, these properties, with the exception of (Cc), also hold for complex measures; (b) is called finite additivity; (c is called monotonicity. PROOF (a) Take $A\in\mathfrak{O}t$ so that $\mu(A)<\alpha,$ and take $A_{1}=A$ and $A_{2}=A_{3}=\cdot\cdot\cdot=$ O in 1.18(1). (b)Take $A_{n+1}=A_{n+2}=\cdot\cdot\cdot\cdot=\mathcal{D}\,\mathrm{in}\,1.18(1).$ c Since B= A "(B- A) and A o(B - 4)= 0, we see tha b implie A $(B)=\mu(A)+\mu(B-A)\geq\mu(A).$ for $\textstyle n=2$ ,3, 4,.…… Then $B_{n}$ ∈ D, (a Put $B_{1}=A_{1},$ and put $B_{n}=A_{n}-A_{n-1}$ . Hence $B_{i}\cap B_{j}=\varnothing$ if i≠j, 4,= B," "U B,,and $A=\bigcup_{i=1}^{\infty}B_{i}$ $$ \mu(A_{n})=\sum_{i=1}^{n}\mu(B_{i})\quad{\mathrm{and}}\quad\mu(A)=\sum_{i=1}^{\infty}\mu(B_{i}). $$ (e) Put $C_{n}=A_{1}-A_{n}.$ Now (d) follows, by the definition of the sum of an infinite series. Then $C_{1}\subset C_{2}\subset C_{3}\subset\cdots,$ $$ \mu(C_{n})=\mu(A_{1})-\mu(A_{n}), $$ $A_{1}-A=\bigcup C_{n},$ and so (d) shows that $$ \mu(A_{1})-\mu(A)=\mu(A_{1}-A)=\operatorname*{lim}_{n\to\infty}\mu(C_{n})=\mu(A_{1})-\operatorname*{lim}_{n\to\infty}\mu(A_{n}). $$ This implies (e) // 1.20 Examples The construction of interesting measure spaces requires some labor, as we shall see. However, a few simple-minded examples can be given immediately: (a) For any $E\in X.$ where $\textstyle X$ is any set, define $\mu(E)=\infty$ if $\boldsymbol{E}$ is an infinite set and let ${\boldsymbol{\mu}}$ be the counting measure on the set {1,2,3,.…}, let if $\boldsymbol{E}$ is finite. This p if $x_{0}\neq E_{1}$ for any counting measure on $X.$ be the number of points in $\boldsymbol{E}$ ${\boldsymbol{\mu}}$ u is called the $\scriptstyle{\theta(t)}$ $E=X.$ This $\boldsymbol{\mu}$ define $\mu(E)=1$ if $x_{\mathrm{e}}\in E$ and $\mu(E)=0$ $x_{0}\,.$ (b)Fix $x_{\mathrm{e}}\in X_{\mathrm{f}}$ may be called the unit mass concentrated at (c)Let Then $\bigcap A_{n}={\mathcal{D}}$ but $\mu(A_{n})=\infty$ for n = 1,2,3,.…. This $A_{n}=\{n,\,n+1,$ $n+2,\ldots\}.$ shows that the hypothesis $$ \mu(A_{1})<\infty $$ is not superfluous in Theorem 1.19(e 1.21 A Comment on Terminology One frequently sees measure spaces referred to as“ordered triples”(X,0, $\boldsymbol{\mu}$ where $X$ is a set, DR is a o-algebra in X, and ${\boldsymbol{\mu}}$ L is a measure defined on D. Similarly, measurable spaces are “ordered. pairs,..LX, 9D).18 REAL AND coMPLEX ANALYSIs This is logically all right, and often convenient, though somewhat redundant. For instance, in (X, J) the set $X$ is merely the largest member of O, so if we know o we also know X. Similarly,every measure has a o-algebra for its domain,by definition, so if we know a measure ${\boldsymbol{\mu}}$ we also know the o-algebra OR on which $\boldsymbol{\mu}$ is defined and we know the set $X$ in which oDR is a o-algebra. It is therefore perfectly legitimate to use expressions like“Let ${\boldsymbol{\sigma}}\cdot$ algebra or the set in question, to say $\boldsymbol{\mu}$ be a measure”or, if we wish to emphasize the “ Let ${\boldsymbol{\mu}}$ be a measure on の”or“Let ${\boldsymbol{\mu}}$ be a measure on $X.$ What is logically rather meaningless but customary (and we shall often follow mathematical custom rather than logic) is to say“Let $X$ be a measure space”; the emphasis should not be on the set, but on the measure.Of course, when this wording is used, it is tacitly understood that there is a measure defined on some o-algebra in $X$ and that it is this measure which is really under dis- cussion. in the set Similarly, a topological space is an ordered pair (X,t), where t is a topology but “the $X,$ and the significant data are contained in t, not in $X,$ topological space $X^{\prime}$ ”is what one talks about. This sort of tacit convention is used throughout mathematics. Most mathe- matical systems are sets with some class of distinguished subsets or some binary operations or some relations (which are required to have certain properties), and one can list these and then describe the system as an ordered pair, triple, etc. depending on what is needed. For instance, the real line may be described as a quadruple (R1,+,·, <), where +,,and < satisfy the axioms of a complete archimedean ordered field. But it is a safe bet that very few mathematicians think of the real field as an ordered quadruple Arithmetic in [O,oo] 1.22 Throughout integration theory, one inevitably encounters ${\mathcal{O}}.$ One reason is that one wants to be able to integrate over sets of infinite measure; after all, the real line has infinite length. Another reason is that even if one is primarily inter ested in real-valued functions, the lim sup of a sequence of positive real functions or the sum of a sequence of positive real functions may well be o at some points and much of the elegance of theorems like 1.26 and 1.27 would be lost if one had to make some special provisions whenever this occurs. Let us define a + 0 = 00 + a= o if O ≤a≤ O0, and $$ a\cdot\sigma=\infty\cdot a={\binom{\infty}{0}}\qquad{\mathrm{if~}}a\leq a\leq\infty $$ It may seem strange to define sums and products of real numbers are of course defined in the usual way However, one verifes without dif $0\cdot\infty=0.$ culty that with this definition the commutative, associative, and distributive laws hold in [O, o] without any restrictionABSTRACT INTEGRATION 19 The cancellation laws have to be treated with some care $a+b=a+c$ implies $\scriptstyle{B=c}$ only when $a<G.$ , and $a b=a c$ implies $b=c$ only when $0<a<\infty.$ Observe that the following useful proposition holds: If O ≤a ≤a,≤.,0≤b, ≤b,≤…,a。→α, and b,→b,then a,b→ ab If we combine this with Theorems 1.17 and 1.14, we see that sums and pro- ducts of measurable functions into [O, oo] are measurable. Integration of Positive Functions on D. In this section, DR will be a o-algebra in a set X $X$ and ${\boldsymbol{\mu}}$ will be a positive measure 1.23 Definition If $\scriptstyle:X\,{\overset{\frown}{\to}}$ [0, o)is a measurable simple function, of the form $$ s=\sum_{i=1}^{n}\alpha_{i}\chi_{A_{i}}, $$ (1) where $\alpha_{1},\,\dots,\,\alpha_{n}$ are the distinct values of s (compare Definition 1.16), and if $E\in\mathbb{N},$ we define $$ \bigcup_{E}^{*}d\mu=\sum_{i=1}^{n}\alpha_{i}\,\mu(A_{i}\cap E). $$ (2) The convention $0\cdot\cdot\infty=0$ is used here; it may happen that $\scriptstyle x_{i}=0$ for some i and that $\mu(A_{i}\cap E)=\infty.$ If f: X→[0, o] is measurable, and $E\in\mathbb{N}.$ we define $$ \bigcap_{E}f\,d\mu=\operatorname*{sup}\, \lceil s\,d\mu, $$ (3) the supremum being taken over all simple measurable functions s such that $0\leq s\leq f.$ over $\textstyle E,$ with The left member of (3) is called the Lebesgue integral of $\boldsymbol{\mathsf{f}}$ respect to the measure ${\boldsymbol{\mu}},$ It is a number in [O,]. f dp if f is simple Observe that we apparently have two definitions for $\bigcap_{E}$ namely,(2) and(3).However,these assign the same value to the integral since f is, in this case, the largest of the functions s which occur on the right of (3). 1.24 The following propositions are immediate consequences of the definitions The functions and sets occurring in them are assumed to be measurable: (a) $I/0\leq f\leq g,$ then Gef du ≤ Jeg du. (b)f $A\subset B$ and f≥0, then Ja f du ≤Jg f dp20 REAL AND COMPLEX ANALYSis (c) $I f\geq0$ and $\scriptstyle{\mathcal{C}}$ is $\bar{a}$ constant, $0\leq c<\infty.$ then $$ \bigcap_{E}c f\,d\mu=c\mid_{E}f\,d\mu. $$ (d) $I/f(x)=0$ for all xe E, then $\textstyle{\int_{E}f\,d\mu=0},$ even if u(E) = 0O. $\textstyle E.$ (e) $I f\mu(E)=0.$ then( $\L{\,\,\int{d\mu_{\L=}}\,0,}$ even iff(x) = 0o for every xe () $\;U f\geq0,$ then Je fdu =Jx xf du This last result shows that we could have restricted our definition of integra- tion to integrals over all of $X,$ without losing any generality. If we wanted to integrate over subsets, we could then use(f) as the definition. It is purely a matter of taste which definition is preferred. One may also remark here that every measurable subset $\boldsymbol{E}$ of a measure space $\textstyle X{\ ~}$ is again a measure space, in a perfectly natural way: The new measur able sets are simply those measurable subsets of $X$ X which lie in ${\boldsymbol{E}},$ and the measure is unchanged, except that its domain is restricted. This shows again that as soon as we have integration defined over every measure space, we automati- cally have it defined over every measurable subset of every measure space. 1.25 Proposition Let s and t be nonnegative measurable simple functions on $X.$ For $\boldsymbol{E}$ ∈ D, define $$ \varphi(E)=\left.\right|_{E}\,d\mu. $$ (1) Then $\varphi$ is a measure on JD. Also $$ \bigcap_{x}(s+t)\;d\mu=\left.{\right|}_{x}^{\circ}\;d\mu+\bigcup_{x}t\;d\mu. $$ (2) (This proposition contains provisional forms of Theorems 1.27 and 1.29. PR0OF If $\boldsymbol{\mathsf{S}}$ is as in Definition 1.23,and if $E_{1},$ $E_{2},$ .are disjoint members of ot whose union is ${\boldsymbol{E}},$ the countable additivity of ${\boldsymbol{\mu}}$ shows that $$ \varphi(E)=\sum_{i=1}^{n}{\alpha_{i}}\mu(A_{i}\cap E)=\sum_{i=1}^{n}{\alpha_{i}}\sum_{r=1}^{\infty}\mu(A_{i}\cap E_{r}) $$ $$ =\sum_{r=1}^{\infty}\;\sum_{i=1}^{n}\alpha_{i}\,\mu(A_{i}\,\cap\,E_{r})=\sum_{r=1}^{\infty}\varphi(E_{r}). $$ AIso, $\varphi({\mathcal{D}})=0,$ so that $\varphi$ is not identically ${\mathcal{D}}.$ABSTRACT INTEGRATION 21 Next, let s be as before, let If $~E_{i j}=A_{i}\cap B_{j},$ then be the distinct values of ${\mathfrak{t}}_{\mathfrak{t}}$ and let $\beta_{1},\,\cdot\cdot\cdot,\,\beta_{m}$ $B_{j}=\{x;t(x)=\beta_{j}\}.$ $$ \bigcap_{E_{i j}}(s+t)\,d\mu=(\alpha_{i}+\beta_{j})\mu(E_{i j}) $$ and $$ \{\O_{E i j}\,d\mu+\bigcap_{i,j}\,d\mu=\alpha_{i}\,\mu(E_{i j})+\beta_{j}\,\mu(E_{i j}). $$ holds Thus (2) holds with $E_{i j}$ in place of $\textstyle X{\mathrm{~}}$ Y.Since $X$ is the disjoint union of the sets ${\it j}/{\it j}/{\it j}$ $E_{i j}$ (1 $1\leq i\leq n,\ 1\leq j\leq m)$ , the first half of our proposition implies that (2) We now come to the interesting part of the theory. One of its most remark able features is the ease with which it handles limit operations 1.26 Lebesgue's Monotone Convergence Theorem Let $\{f_{n}\}$ be a sequence of measurable functions on $X.$ and suppose that (a)0 $\leq f_{1}(x)\leq f_{2}(x)\leq\cdot\cdot\leq$ oo for every xe $X,$ (b)f.(x)→f(x) as $\textstyle n\!\to$ 0, for every $\scriptstyle x\in X$ Then f is measurable, and $$ \bigcap_{x}f_{n}\,d\mu\to\bigcup_{x}^{*}f\,d\mu\qquad a s\ n\to\infty. $$ PROOF Since $\textstyle{\int}f_{n}\leq\int f_{n+1},$ there exists an αe [O,o] such that $$ \bigcap_{x}f_{n}\,d\mu\to\alpha\qquad a s\ n\to\infty. $$ (1) By Theorem $1.14,f$ is measurable. Since $f_{n}\leq f,$ we have $\{f_{n}\leq\}f$ for every n, so (1) implies $$ x\leq\int_{x}f\,d\mu. $$ (2) Let $\boldsymbol{\mathsf{S}}$ be any simple measurable function such that $0\leq s\leq f,$ let c be a constant, $0<c<1$ , and define $$ E_{n}=\{x;f_{n}(x)\geq c s(x)\}\qquad(n=1,\,2,\,3,\,\ldots). $$ (3) Each $E_{n}$ is measurable, $E_{1}\subset E_{2}\subset E_{3}\subset\cdots,$ $f(x)=0,$ then $x\in E_{1};$ if $f(x)>0,$ then equality, consider some ${\mathfrak{c}}\in X$ If 、and $X=\bigcup E_{n}$ To see this $c s(x)<f(x),$ since c <1; hence $x\in E_{n}$ for some ${\mathfrak{n}}.$ Also $$ \left.\left(\frac\L\right)_{X}f_{n}\,d\mu\geq\left.\right. .\int_{E_{n}}^{\ast}f_{n}\,d\mu\,\geq c~\right|_{E_{n}}^{\ast}s\,d\mu\qquad(n=1,\,2,\,3,\,\ldots). $$ (4)$\mathbf{2}\mathbf{2}$ REAL AND coMPLEX ANALYSIS Let n→O0, applying Proposition 1.25 and Theorem 1.19(d) to the last inte- gral in (4). The result is $$ x\geq c\int_{x}s\,d\mu. $$ () Since (5) holds for every $c\triangleleft,$ we have $$ x\geq\int_{X}s\,d\mu $$ (6) for every simple measurable s satisfying $0\leq s\leq f,$ so that $$ x\geq\int_{X}f\,d\mu. $$ (7) The theorem follows from (1), 2), and (7) // 1.27 Theorem $I f_{z}\colon X\to[0,$ oo] is measurable, for n= 1,2,3, …,and $$ f(x)=\sum_{n=1}^{\infty}f_{n}(x)\qquad(x\in X), $$ (1) then $$ \bigcap_{X}f\,d\mu=\sum_{n=1}^{\infty}\,\bigcap_{X}f_{n}\,d\mu. $$ (2) PRoOF First, there are sequences {s},{SY}of simple measurable functions such that $s_{i}^{\prime}\to f_{1}$ and $s_{i}^{\prime\prime}\to f_{2}\,.$ ,as in Theorem 1.17. If $s_{i}=s_{i}+s_{i}^{\prime},$ then $s_{i}\to f_{1}+f_{2}\,,$ and the monotone convergence theorem, combined with Propo- sition 1.25, shows that $$ \bigcap_{x}(f_{1}+f_{2})\,d\mu=\bigcap_{x}^{}f_{1}\,d\mu+\bigcap_{x}f_{2}\,d\mu. $$ (3) Next, put $g_{N}=f_{1}+\cdots+f_{N}.$ The sequence $\scriptstyle(y_{\alpha}!$ converges monotoni- cally tof, and if we apply induction to (3) we see that $$ \bigcap_{X}^{}g_{N}\,d\mu=\sum_{n=1}^{N}\, [\sum_{X}^{}f_{n}\,d\mu. $$ (4) Applying the monotone convergence theorem once more, we obtain (2), and the proof is complete. // If we let ${\boldsymbol{\mu}}$ be the counting measure on a countable set, Theorem 1.27 is a statement about double series of nonnegative real numbers (which can of course be proved by more elementary means):ABSTRACT INTEGRATION $23$ Corollary $I f\,a_{i j}\geq0\,f o r\ i\ a n d\,j=1,2,3,\ldots,t h e n$ $$ \sum_{i=1}^{\infty}\ \sum_{j=1}^{\infty}a_{i j}=\sum_{j=1}^{\infty}\ \sum_{i=1}^{\infty}a_{i j}. $$ n, then 1.28 Fatou's Lemma $I f_{z}\colon X\to$ [O, oo] is measurable, for each positive integer $$ \bigcap_{X}{\underset{\mathrm{im}}{\underbrace{\land}{\operatorname{nm}\circ{\mathsf{n f}}f_{n}}}}\rfloor\,d\mu\leq\operatorname*{lim}_{n\to\infty}{\operatorname{in}\operatorname{inf}\bigcup_{X}}f_{n}\,d\mu. $$ (1) Strict inequality can occur in (1); see Exercise 8 PROOF Put $$ g_{k}(x)={\underset{i\geq k}{\operatorname*{inf}}}\,f_{i}(x)\qquad(k=1,\,2,\,3,\,\ldots;\,x\in X). $$ (2) Then $g_{k}\leq f_{k},$ so that $$ \bigcap_{x}^{\bullet}g_{k}\,d\mu\leq\left\lceil_{x}^{}f_{k}\,d\mu\right.\qquad(k=1,\,2,\,3,\,...). $$ (3) Also, $0\leq g_{1}\leq g_{2}\leq\cdots.$ each ${\mathfrak{g}}_{k}$ is measurable,by Theorem 1.14,、and gAx)→lim inf fKx) as $k\to\varnothing,$ by Definition 1.13. The monotone convergence theorem shows therefore that the left side of (3) tends to the left side of((1), a k→0. Hence (1) follows from (3) // 1.29 Theorem Suppose f: X->[0,αo] is measurable, and $$ \varphi(E)= (E^{\prime}\,d\mu\qquad(E\in\mathfrak{N}). $$ (1) Then $\varphi$ is a measure on DR, and $$ \ D_{x}^{\bullet}\,d\varphi=\bigcup_{X}^{\bullet}\!g f\,d\mu $$ (2) for every measurable $\scriptstyle{\mathcal{G}}$ g on $X$ with range in [O,oo]. PROOF Let $E_{1},$ $E_{2}\,,$ $E_{3},$ .…. be disjoint members of の whose union is $E.$ Observe that $$ \chi_{E}f=\sum_{j=1}^{\infty}\chi_{E_{j}}f $$ (3) and that $$ \varphi(E)=\left[\chi_{E}^{\prime}f\,d\mu,\qquad\varphi(E_{j})=\right]\chi_{E j}f\,d\mu. $$ (4)24 REAL AND COMPLEX ANALYSIS It now follows from Theorem 1.27 that $$ \varphi(E)=\sum_{j=1}^{\infty}\varphi(E_{j}). $$ (5) Since $\varphi({\mathcal{D}})=0,(5)$ proves that $\varphi$ is a measure for some $E\in{\mathfrak{M}}.$ Hence Next,(1) shows that (2) holds whenever $\scriptstyle g=\pi\,n$ (2) holds for every simple measurable function ${\mathfrak{g}},$ and the general case follows from the monotone convergence theorem. // Remark The second assertion of Theorem 1.29 is sometimes written in the form $$ d\varphi=f\,d\mu. $$ (6) We assign no independent meaning to the symbols $d\varphi$ and du;(6) merely means that (2) holds for every measurable $\scriptstyle g\leq0$ Theorem 1.29 has a very important converse,the Radon-Nikodym theorem, which will be proved in Chap. 6. Integration of Complex Functions As before, ${\boldsymbol{\mu}}$ will in this section be a positive measure on an arbitrary measurable space $X.$ 1.30 Definition We define $L^{1}(\mu)$ to be the collection of all complex measur able functions fon $\textstyle X$ for which $$ \bigcap_{x}|f|\ d\mu<\varnothing. $$ Note that the measurability of f implies that of |f|, as we saw in Propo- sition 1.9(b); hence the above integral is defined. The members of $L^{1}(\mu)$ are called Lebesgue integrable functions(with respect to p) or summable functions. The significance of the exponent lwil become clear in Chap. 3. 1.31 Definition If $\displaystyle f=u+i v,$ where u $\boldsymbol{\ u}$ u and ${\boldsymbol{\ U}}$ are real measurable functions on $X,$ and $\mathbb{F}f\in L^{n}(\mu),$ we define $$ \left[\stackrel{\circ}{E}f\,d\mu= (\right)_{E}u^{+}\ d\mu- (\stackrel{\circ}{\int}_{E}u+i \{\stackrel{\circ}{\partial}v^{+}\ d\mu-i\stackrel{\circ}{\partial}v^{-}\ d\mu $$ (1) Sec Here for every measurable set and $u^{-}$ are the positive and negative parts of $u,$ , as defined in $\textstyle E.$ $\boldsymbol{u}^{+}$ $1.15\colon v^{*}$ and $v^{{\overset{}{\to}}}$ are similarly obtained from . These four functions are measurable, real, and nonnegative; hence the four integrals on the right of(1) exist, by Definition 1.23. Furthermore, we have $u^{+}\leq|u|<|f|\,.$ etc., SO thatABSTRACT INTEGRATION 25 each of these four integrals is finite. Thus $\mathbf{(1)}$ defines the integral on the left as a complex number Occasionally it is desirable to define the integral of a measurable func tion f with range in[-00,oo] to be $$ \bigcap_{E}f\,d\mu=\bigcap_{n}f^{+}\ d\mu- \{_{E}f^{-}\ d\mu, $$ (2) provided that at least one of the integrals on the right of (2) is finite. The left side of(Q) is then a number in [-o,0o] αf + βg e 1.32 Theorem Suppose f and and $g\in L^{1}(\mu)$ and α and $\boldsymbol{\beta}$ are complex numbers. Then $\scriptstyle T_{(i,i)}$ $$ [\stackrel{\circ}{x}(\circ\!{\mathcal{I}}+\beta g)\;d\mu=\alpha\int_{x}^{}f\,d\mu+\beta\int_{x}g\;d\mu. $$ (1) PxooF The measurability of df + βg follows from Proposition 1.9(c). By Sec 1.24 and Theorem 1.27, $$ \begin{array}{r}{\left|_{x}|\,\alpha f+\beta g\,|\,d\mu\leq\int_{x}(|\,\alpha|\,|f|\,+\,|\,\beta\,|\,|\,g\,|\,)\,d\mu}\\ {\ }&{{}}\\ {\vdots\,=|\,\alpha\,|\,\displaystyle\int_{X}|\,f\,|\,d\mu\,+\,|\,\beta\,|\,\displaystyle|\,g\,|\,d\mu<\,\infty.\right. $$ Thus $\alpha f+\beta g\in L^{1}(\mu)$ it is clearly sufficient to prove To prove $(1),$ and $$ \bigcap_{x}(f+g)\;d\mu=\bigcap_{x}f\,d\mu+\bigcap_{x}g\;d\mu $$ (2) $$ \dagger_{x}(\alpha f)\;d\mu=\alpha\;\Bigg|_{X}f\;d\mu, $$ (3) Assuming this, and setting $h=f+_{\mathrm{f}}$ 2) will fllow if we prove (2) for real f and $\mathbf{\Omega}^{g}$ in $\scriptstyle{T_{(i,j)}}$ and the general case of $(2)$ g, we have $$ h^{+}-h^{-}=f^{+}-f^{-}+g^{+}-g^{-} $$ or $$ h^{+}+f^{-}+g^{-}=f^{+}+g^{+}+h^{-}. $$ (4) By Theorem 1.27, $$ \int h^{+}\;+\;\int f^{-}\;+\;\int g^{-}=\int f^{+}\;+\;\int g^{+}\;+\;\int h^{-}, $$ (5) and since each of these integrals is finite, we may transpose and obtain (2)$26$ REAL AND COMPLEX ANALYSIs That (3) holds if $\scriptstyle x\,\geq\,0$ follows from Proposition 1.24(c). It is easy to $\scriptstyle x\,=\,{\frac{1}{x\,=\,i}}$ verify that (3) holds if $\alpha=-1.$ using relations like $(-u)^{+}=u^{-}.$ The case is also easy: $\operatorname{Pf}=u+i v,$ then $$ \begin{array}{c}{{\left[\left(i f\right)=\int\left(i u-v\right)=\int\left(-v\right)+i\int u=-\,\int v+i\int u=i (\int u+i\int v\right)}}\\ {{\ }}\\ {{=i}}\end{array} $$ Combining these cases with (2), we obtain (3) for any complex α / 1.33 Theorem $I f\in L^{1}(\mu),$ then $$ \left|\ D_{X}f\,d\mu\right|\leq\int_{X}|f|\ d\mu. $$ PROOF Put $\quad z=\int_{X}f\,d\mu.$ Since $\widetilde{\mathbb{Z}}$ is a complex number, there is a complex number α, with $|\alpha|=1,$ such that $\alpha z=|z|.$ Let u be the real part of cf. Then $u\leq|\alpha f|=|f|\,.$ Hence $$ \bigg|\;\int_{x}f\,d\mu\bigg|=\alpha\;\int_{x}^{\ }f\,d\mu=\left[\mathbf{\sigma}_{x}^{\alpha}/\,d\mu=\left(\begin{array}{c}{{u\;d\mu\leq\displaystyle\int_{x}|f|\ d\mu}}\\ {{\vdots}}\end{array}\right)\right|\;d\mu. $$ The third of the above equalities holds since the preceding ones show tha cf du is real. /// We conclude this section with another important convergence theorem 1.34 Lebesgue's Dominated Convergence Theorem Suppose 1 $\textstyle X$ such tha $\{f_{n}\}$ is a sequence of complex measurable functions on $$ f(x)=\operatorname*{lim}_{n\to\infty}f_{n}(x) $$ (1) exists for every x∈ $X.$ ${\mathit{l}}{\hat{\boldsymbol{J}}}$ there is a function g e $L^{1}(\mu)$ such that $$ |f_{n}(x)|\leq g(x)\qquad(n=1,\,2,\,3,\,\ldots;\,x\in X), $$ (2)) then f ∈e $L^{1}(\mu),$ $$ \operatorname*{lim}_{n arrow\infty}\ {\ ^{\ }\prod_{x}|f_{n}-f|\ d\mu=0,} $$ (3) and $$ \operatorname*{lim}_{n arrow\infty}\,\bigg|_{X}f_{n}\,d\mu=\bigg(\int_{X}f\,d\mu. $$ (4)ABSTRACT INTEGRATION $27$ PROOF Since $|f|\leq g$ and $\boldsymbol{\mathsf{f}}$ is measurable,f ∈ $L^{1}(\mu).$ Since $|f_{n}-f\,|\leq2g,$ Fatou's lemma applies to the functions $2g-|f_{n}-f|$ and yields $$ \begin{array}{r l}{\left.\left(1_{x}^{\ast}-g\ d\mu\leq\operatorname*{lim}_{n\to\infty}\right)_{x}^{\ast}(2g-\mid f_{n}-f\mid)\ d\mu}\\ {\quad\quad=\left(\sum_{n\to\infty}^{\ast}d\mu+\operatorname*{lim}_{n\to\infty}\operatorname*{sup}_{\rightarrow\infty}\right)_{x}\left(-\begin{array}{l}{{\vdots}}\\ {{\vdots}}\\ {{\theta_{n}-f\mid d\mu.}}\end{array}\right)}\end{array} $$ Since(2g dp is finite, we may subtract it and obtain $$ \operatorname*{lim}_{n\to\infty}\operatorname*{sup}_{\sim\infty} \vert_{X}\vert f_{n}-f\mid d\mu\leq0. $$ (5) If a sequence of nonnegative real numbers fails to converge to O, then its upper limit is positive. Thus () implies (3). By Theorem 133,applied to $f_{n}-f,(3)$ implies (4) // The Role Played by Sets of Measure Zero 1.35 Definition Let ${\mathbf{}}P$ P be a property which a point $\scriptstyle{\mathcal{X}}$ may or may not have If For instance, ${\mathbf{}}P$ might be the property $\{f_{n}\}$ is a given sequence of functions. is a given function, or it holds might be ${}^{*}\{f_{n}(x)\}$ converges”if ${}^{\ast}f(x)>0^{\ast}\,\mathrm{if}f\,\mathrm{i}$ ,the statement “ ${\mathbf{}}P$ $\boldsymbol{\mu}$ is a measure on a c-algebra and if $\scriptstyle{E\left(\scriptstyle{\frac{1}{2\pi}}\right)}$ almost everywhere on $E^{\gamma}$ (abbreviated to“ ${\mathbf{}}P$ holds a.e. on ${\boldsymbol{E}}^{\prime}$ ”)means that of there exists an $N\in{\mathfrak{M}}$ such that $\mu(N)=0,$ $N\subset E,$ and ${\mathbf{}}P$ holds at every point $E-N.$ This concept of a.e. depends of course very strongly on the given measure, and we shall write “a.[u]”whenever clarity requires that the measure be indicated. For example, iff and $\mathbf{\Omega}^{g}$ g are measurable functions and if $$ \mu(\{x;f(x)\neq g(x)\})=0, $$ (1) we say that $f=g$ a.e.[u] on X $X,$ X。 , and we may $w\Pi\vert\alpha f\sim g.$ This is easily seen to $f\sim h)$ be an equivalence relation. The transitivity $(f\sim g$ and $g\sim h$ implies measure O. is a consequence of the fact that the union of two sets of measure $\mathbf{0}$ has Note that ${\mathrm{if}}f\sim g,$ then, for every $E\in{\mathfrak{D}}\ ,$ $$ \bigcap_{E}f\,d\mu= [_{E}g\,d\mu. $$ (2) To see this, let ${\cal N}$ be the set which appears in (1); then $\boldsymbol{E}$ is the union of the disjoint sets $E-N$ and $E\,\cap\,N;$ on $E-N,f=g,$ and $\mu(E\cap N)=0.$28 REAL AND CoMPLEX ANALYSis Thus, generally speaking, sets of measure O are negligible in integration It ought to be true that every subset of a negligible set is negligible. But it may happen that some set $N\in{\mathfrak{M}}$ with $\mu(N)=0$ has a subset 上 $\boldsymbol{E}$ which is not a member of DR. Of course we can define $\mu(E)=0$ in this case. But will this extension of ${\boldsymbol{\mu}}$ still be a measure, ie., will it still be defined on a c-algebra? It is a pleasant fact that the answer is affirmative: 1.36 Theorem Let(X,D,u) be a measure space, let ODt* be the collection of all $\scriptstyle{E\in X}$ for which there exist sets $\scriptstyle A$ and B ∈の such that $A\<E<B$ and $\mu(B-A)=0.$ and define $\mu(E)=\mu(A)$ in this situation. Then Dt is a o-algebra, and ${\boldsymbol{\mu}}$ is a measure on ${\mathfrak{M}}/{\mathfrak{\ast}}$ This extended measure ${\boldsymbol{\mu}}$ is called complete, since all subsets of sets of measure O are now measurable; the o-algebra のt* is called the ${\boldsymbol{\mu}}\cdot$ -completion of The theorem says that every measure can be completed, so, whenever it is conve nient, we may assume that any given measure is complete; this just gives us more measurable sets, hence more measurable functions. Most measures that one meets in the ordinary course of events are already complete, but there are excep tions; one of these will occur in the proof of Fubini's theorem in Chap. 8 letters $\scriptstyle A$ 4 and $\boldsymbol{B}$ PROOF We begin by checking that $\boldsymbol{\mu}$ is well defined for every $\boldsymbol{E}$ e 0t*. Suppose $A\subset E\subset B,\ \ A_{1}\subset E\subset B_{1},$ and $\mu(B-A)=\mu(B_{1}-A_{1})=0.$ (The will denote members of O throughout this proof) Since $$ A-A_{1}\subset E-A_{1}\subset B_{1}-A_{1} $$ we have $\mu(A-A_{1})=0,$ hence $\mu(A)=\mu(A\cap A_{1}).$ For the same reason, $\mu(A_{1})=\mu(A_{1}\cap A).$ We conclude that indeed $\mu(A_{1})=\mu(A).$ Next, Iet us verify that O* has the three defining properties of a o algebra. ij if (ü) X ∈ 0*, because $\scriptstyle X\in{\mathfrak{M}}$ and ${\mathfrak{M}}\in{\mathfrak{M}}^{*}.$ implies $E^{*}\in{\mathfrak{M}}^{*}$ , because $A\prec E\subset B$ then $B^{c}\subset E^{c}\subset A^{c}.$ Thus $E\in{\mathfrak{D}}^{*}$ $A^{c}-B^{c}=A^{c}\cap B=B-A.$ (ii) If A;C E c E $3_{i},E=\left(\right)\,E_{i},\,A= (\frac{}{}\,A_{i},\,B=\bigcup B_{i},$ then $A\in E\subset B$ and $$ B-A=\bigcup_{1}^{\infty}(B_{i}-A)\subset\bigcup_{1}(B_{i}-A_{i}). $$ Since countable unions of sets of measure zero have measure zero, it follows that $E\in{\mathfrak{M}}^{*}$ if $E_{i}\in{\mathfrak{M}}^{*}$ for $i=1,2,3,$ Finally, if the sets $\textstyle E_{i}$ are disjoint in step Gii) the same is true of the sets $A_{i},$ and we conclude that $$ \mu({\cal E})=\mu({\cal A})=\sum_{1}^{\infty}\,\mu({\cal A}_{i})=\sum_{1}^{\infty}\,\mu({\cal E}_{i}). $$ This proves that ${\boldsymbol{\mu}}$ is countably additive on ${\mathfrak{M}}^{*}.$ //ABSTRACT INTEGRATION 29 1.37 The fact that functions which are equal a.e. are indistinguishable as far as integration is concerned suggests that our definition of measurable function measurable on $\textstyle X{\ ~}$ if might profitably be enlarged. Let us call a function ${\mathfrak{f}}f^{-1}(V)\cap E$ is measurable for every open set $\boldsymbol{\mathsf{f}}$ defined on a set Ee D $\mu(E^{\circ})=0$ and $V.$ If we define $f(x)=0$ for $x\in E^{c},$ we obtain a measurable function on $X,$ in the old sense.If our measure happens to be complete, we can define $\boldsymbol{\mathit{f}}$ on $E^{c}$ in a perfectly arbitrary manner, and we still get a measurable function. The integral of $\boldsymbol{\f}$ over any set $A\in{\mathfrak{M}}$ is independent of the definition of $\boldsymbol{\f}$ on $E^{\mathrm{c}};$ therefore this definition need not even be specified at all There are many situations where this occurs naturally. For instance, a func- tion f on the real line may be differentiable only almost everywhere (with respect to Lebesgue measure),but under certain conditions it is still true that $\boldsymbol{\f}$ is the measurable functions on $X$ integral of its derivative; this will be discussed in Chap. 7.Or a sequence $\{f_{n}\}$ of Y may converge only almost everywhere; with our new definition of measurability, the limit is still a measurable function on $X,$ , and we do not have to cut down to the set on which convergence actually occurs. To illustrate, let us state a corollary of Lebesgue's dominated convergence theorem in a form in which exceptional sets of measure zero are admitted: 1.38 Theorem Suppose $\{f_{n}\}$ is a sequence of complex measurable functions defined a.e. on $\textstyle X$ such that $$ \sum_{n=1}^{\infty}\,\bigcup_{\textstyle n}^{\circ}|f_{n}|\,d\mu<\infty. $$ (1) Then the series $$ f(x)=\sum_{n=1}^{\infty}f_{n}(x) $$ (2) converges for almost al $x,f\in L^{1}(\mu),$ and $$ \bigcup_{X}^{\circ}f\,d\mu=\sum_{n=1}^{\infty}\left.\right\}_{X}f_{n}\,d\mu. $$ (3) $\textstyle\sum$ PR0OF Let ${\boldsymbol{S}}_{n}$ be the set on which $\textstyle f_{n}$ is defined, so that $\mu(S_{n}^{c})=0.$ Put $\scriptstyle\phi(x)=$ l f,(x)|,for $x\in S=\bigcap$ S,Then $\mu(S^{\circ})=0.$ By (1) and Theorem 1.27, $$ \int_{s}\varphi~d\mu<\infty. $$ (4) If place of $X$ $|g_{n}|\leq\varphi,\,g_{n}(x)\to f_{*}(x)$ for all it follows from(4)that $\mu(E^{c})=0.$ The series(2) $x\in E,$ in then $E=\{x\in S;$ $\varphi(x)<\infty\}$ $E,$ so that $f\in L^{l}(\mu)$ on ${\boldsymbol{E}},$ by (4).1 $g_{n}=f_{1}+\cdot\cdot\cdot+f_{n},$ $\boldsymbol{E}$ converges absolutely for every $\operatorname{rer}E.$ and if $\scriptstyle{\mathcal{I}}(x)$ is defined by (2) for then $|f(x)|\leq\varphi(x)$ on and Theorem 1.34 gives (3) with $x\in E,$ . This is equivalent to (3), since $\mu(E^{\circ})=0.$ $J/I$30 REAL AND COMPLEX ANALYSIs Note that even if t $\operatorname{he}f_{n}$ were defined at every point of X,(1) would only imply that (2) converges almost everywhere. Here are some other situations in which we can draw conclusions only almost everywhere: 1.39 Theorem a.e. on E (a)Suppose f: X→[0, o] is measurable, Ee D, and $\bigcap_{E}$ f du = 0. Then f= 0 (b)Suppose f∈ $\scriptstyle T(t)\quad$ and Jz f du = 0 for every E∈ D. Then f = 0 a.e. on $X$ (c)Suppose f∈ $\scriptstyle I_{(i)}$ and $$ \left|\ _{\chi}f\,d\mu\right|= |_{\chi}|f|\ d\mu. $$ Then there is a constant α such that cf =|fla.e. on $X.$ Note that (c) describes the condition under which equality holds in Theorem 1.33. PROOF (a)If $A_{n}=\{x\in E;f(x)>1/n\},n=1,2,3,\ldots,t{\mathrm{het}}$ n $$ {\frac{1}{n}}\,\mu(A_{n})\leq\int_{A_{n}}f\,d\mu\leq\int_{E}f\,d\mu=0, $$ $\textstyle{\bigcap_{E}}$ so that $\mu(A_{n})=0.$ Since $\left\{x\in E\colon f(x)>0\right\}=\bigcup A_{n},(a)$ follows. is then (b)Put $\displaystyle f=u+i v,$ let $E=\{x\!:\,u(x)\geq0\}$ The real part of $\textstyle{\int_{E}f\,d\mu}$ ut du. Hence $\textstyle{ [_{E}u^{+}\,d\mu=0,}$ and (a) implies that $u^{+}=0$ a.e. We con- clude similarly that $$ u^{-}=v^{+}=v^{-}=0\qquad\mathrm{a.e.} $$ (c)Examine the proof of Theorem 1.33. Our present assumption implies that the last inequality in the proof of Theorem 1.33 must actually be an $|f|=u$ equality. Hence $\left(\left|f\right|-u\right)d\mu=0.$ Since $|f|-u\geq0,$ (a) shows that a.e. This says that the real part of $\mathbf{x}{\boldsymbol{f}}$ is equal to $\mid{\cal Q}\!\oint$ a.e., hence $x f=|\varnothing f|=|f|\ a.\mathrm{e},$ which is the desired conclusion / 1.40 Theorem Suppose $\mu(X)<\infty,f\in L^{1}(\mu),$ $\boldsymbol{\mathsf{S}}$ is a closed set in the complex plane, and the averages $$ A_{E}(f)={\frac{1}{\mu(E)}}\prod_{E}f\,d\mu $$ lie in S for every E ∈ D with $\mu(E)>0.$ Then f(x) e S for almost al $x\in X.$ABSTRACT INTEGRATION 31 PR0OF Let $\underline{{\land}}$ be a closed circular disc (with center at α and radius $\scriptstyle\gamma\geq0,$ say) in the complement of ${\boldsymbol{S}}.$ Since ${\boldsymbol{S}}^{\mathrm{c}}$ is the union of countably many such discs, it is enough to prove that $\mu(E)=0,$ where $E=f^{-1}(\Delta).$ If we had $\mu(E)>0,$ then $$ \left|\left.A_{E}(f)-\alpha\right|=\frac{1}{\mu(E)}\left|\right.\left[\right]_{E}(f-\alpha)~d\mu\right|\leq\frac{1}{\mu(E)} ]_{E}^{\left[ .\int-\alpha\right|\,d\mu\leq r, $$ which is impossible, since $A_{k}(f)\in S.$ Hence $\mu(E)=0$ // 1.41 Theorem Let $|E_{n}|$ be a sequence of measurable sets in X, such tha $$ \sum_{k=1}^{\infty}\mu(E_{k})<\infty. $$ (1) Then almost al $\scriptstyle x\in{\mathcal{X}}$ lie in at most finitely many of the sets $E_{k}$ PROOF If $\scriptstyle A$ is the set of all $\scriptstyle{\mathcal{X}}$ which lie in infnitely many $E_{k},$ we have to prove that $\mu(A)=0.$ Put $$ g(x)=\sum_{k=1}^{\infty}\chi_{E_{k}}(x)\qquad(x\in X). $$ (2) For each x, each term in this series is either O or 1. Hence xe A if and only if (1). Thus $g\in L^{n}(\mu),$ and so By Theorem 1.27, the integral of a.e. is equal to the sum in // $g(x)=\infty.$ $\scriptstyle{\mathcal{G}}$ over $\textstyle X$ $\theta(x)<\infty$ Exercises 1 Does there exist an infinite o-algebra which has only countably many members? 2 Prove an analogue of Theorem 1.8 for nfunctions every rational ${\boldsymbol{r}}_{\!_{\mathrm{J}}}$ 3 Prove that if fis a real function on a measurable space $X$ such that $\{x:f(x)\geq r\}$ is measurable for then fis measurable. 4 Let $\left\{a_{n}\right\}$ and $\{b_{n}\}$ be sequences i $[-\infty,$ oo], and prove the following assertions: (6) $$ \begin{array}{c}{{\operatorname*{lim}_{n arrow\infty}\mathrm{sup}\left(-a_{n}\right)=\,-\operatorname*{lim}_{n arrow\infty}\mathrm{inf}\ a_{n}.}}\\ {{\operatorname*{lim}_{n arrow\infty}}}\\ {{\operatorname*{lim}_{n arrow\infty}}}\end{array} $$ (a) provided none of the sums is of the form $\scriptstyle a\sp{a}+\alpha$ (e I $a_{n}\leq b_{n}$ for all n, then $$ \operatorname*{lim}_{n arrow\infty}\mathrm{i}\mathrm{ln}\Bigr\lbrack\operatorname*{lim}_{n arrow\infty}\mathrm{j}_{n}. $$ Show by an example that strict inequality can hold in (b)32 REAL AND COMPLEX ANALYSis 5 (a) Suppose f: $X\to[-\infty,$ o] and $g\colon X arrow\Gamma-\infty$ , o] are measurable. Prove that the sets $$ \{x:f(x)<g(x)\},\,\{x;f(x)=g(x)\} $$ are measurable b) Prove that the set of points at which a sequence of measurable real-valued functions con- verges (to a fnite limit) is measurable 6 Let X $\scriptstyle{\mathcal{X}}$ be an uncountable set,let OR be the collection of all set $E\subset X$ such that either $\boldsymbol{E}$ Aor ${\boldsymbol{E}}^{\mathrm{c}}$ is at o-algebra in most countable, and define and that ${}_{\!\mu}$ is a measure on OD. Describe the corresponding measurable functions and in the second. Prove that の is a $\mu(E)=0$ in the first case, $\mu(E)=1$ $\scriptstyle{\mathcal{X}}$ their integrals 7T Suppose $f_{n};$ $\scriptstyle x\to[0,\,\infty]$ is measurable for $n=1,$ 2,3 $\cdot\cdot\cdot f_{1}\geq f_{2}\geq f_{3}\geq\cdot\cdot\cdot\geq0,f_{n}(x)\to f(x)$ as n→Oo, for every $x\in X,$ and f,e L(p). Prove that then $$ \operatorname*{lim}_{n arrow\infty}\ {\ ^{\prime}}_{x}\,d\mu= \lceil_{x}f\,d\mu \rceil $$ and show that this conclusion does not followif the condition ${}^{\circ}f_{1}\in L^{\backslash}(\mu)^{\circ}{\mathrm{i}}$ is omitted. 8 Put $f_{n}=\chi_{E}$ if n is odd $-f_{n}=1-\chi_{E}$ if n is even. What is the relevance of this example to Fatou's lemma? 9 Suppose ${}_{\mu}$ is a positive measure on $X,f\colon X\to[0,\,\infty.$ ]is measurable, $\textstyle{ [x\ f\,d\mu=c,}$ where $0<c<\infty,$ and αis a constant. Prove that $$ \operatorname*{lim}_{n arrow\infty}\left\{_{x}n\,\mathrm{log}\,\,[1+(f/n)^{x}]\,\,d\mu={\frac{\left|^{\infty}\right.\mathbf{if}\,0<\alpha<1,}{\left|c\right.\right. . .}_{\mathrm{if}}\,\left|(\frac{\alpha}{\alpha})\right|^{2}\left|(x)\right|^{2} \}}_{t=\alpha<1.} $$ Hint: I $\scriptstyle x\,\geq\,0.$ the integrands are dominated by af.Ifa <1, Fatou's lemma can be applied. uniformly on $X.$ $\mu(X)<\infty,\;\{f_{n}\}$ is a sequence of bounded complex measurable functions on $X.$ , and f,→力 10 Suppose Prove that $$ \operatorname*{lim}_{n=\infty}\,\bigcap_{X}f_{n}\,d\mu= \vert_{X}f\,d\mu, $$ and show that the hypothesis $\ ^{\ast}\mu(X)<\infty^{\ast}$ cannot be omitte 11 Show that $$ A=\bigcap_{n=1}^{\infty}\bigcup_{k=n}^{\infty}E_{k} $$ in Theorem 1.41, and hence prove the theorem without any reference to integration 12 Suppose f∈ $L^{1}(\mu).$ Prove that to each $\epsilon>0$ there exists a $\delta>0$ such that JAIfl du <é whenever $\mu(E)<\delta.$ 13 Show that proposition 1.24(c) is also true when $\scriptstyle{\varepsilon^{\scriptstyle{s+\alpha}}}$
