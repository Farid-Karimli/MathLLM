CHAPTER FIVE
EXAMPLES OF BANACH SPACE TECHNIQUES

Banach Spaces
5.1 In the preceding chapter we saw how certain analytic facts about trigonometric series can be made to emerge from essentially goemetric considerations about general Hilbert spaces, involving the notions of convexity, subspaces, orthogonality, and completeness. There are many problems in analysis that can be attacked with greater ease when they are placed within a suitably chosen abstract framework. The theory of Hilbert spaces is not always suitable since orthogonality is something rather special. The class of all Banach spaces affords greater variety. In this chapter we shall develop some of the basic properties of Banach spaces and illustrate them by applications to concrete problems.
5.2 Definition A complex vector space $X$ is said to be a normed linear space if to each $x \in X$ there is associated a nonnegative real number $\|x\|$, called the norm of $x$, such that
(a) $\|x+y\| \leq\|x\|+\|y\|$ for all $x$ and $y \in X$,
(b) $\|a x\|=|\alpha|\|x\|$ if $x \in X$ and $\alpha$ is a scalar,
(c) $\|x\|=0$ implies $x=0$.
By $(a)$, the triangle inequality
$$
\|x-y\| \leq\|x-z\|+\|z-y\| \quad(x, y, z \in X)
$$
holds. Combined with (b) (take $\alpha=0, \alpha=-1$ ) and (c) this shows that every normed linear space may be regarded as a metric space, the distance between $x$ and $y$ being $\|x-y\|$.

A Banach space is a normed linear space which is complete in the metric defined by its norm.

For instance, every Hilbert space is a Banach space, so is every $L^p(\mu)$ normed by $\|f\|_p$ (provided we identify functions which are equal a.e.) if $1 \leq p \leq \infty$, and so is $C_0(X)$ with the supremum norm. The simplest Banach space is of course the complex field itself, normed by $\|x\|=|x|$.
One can equally well discuss real Banach spaces; the definition is exactly the same, except that all scalars are assumed to be real.
5.3 Definition Consider a linear transformation $\Lambda$ from a normed linear space $X$ into a normed linear space $Y$, and define its norm by
$$
\|\Lambda\|=\sup \{\|\Lambda x\|: x \in X,\|x\| \leq 1\} \text {. }
$$

If $\|\Lambda\|<\infty$, then $\Lambda$ is called a bounded linear transformation.
In (1), $\|x\|$ is the norm of $x$ in $X,\|\Lambda x\|$ is the norm of $\Lambda x$ in $Y$; it will frequently happen that several norms occur together, and the context will make it clear which is which.

Observe that we could restrict ourselves to unit vectors $x$ in (1), i.e., to $x$ with $\|x\|=1$, without changing the supremum, since
$$
\|\Lambda(\alpha x)\|=\|\alpha \Lambda x\|=|\alpha|\|\Lambda x\| .
$$

Observe also that $\|\Lambda\|$ is the smallest number such that the inequality
$$
\|\Lambda x\| \leq\|\Lambda\|\|x\|
$$
holds for every $x \in X$.
The following geometric picture is helpful: $\Lambda$ maps the closed unit ball in $X$, i.e., the set
$$
\{x \in X:\|x\| \leq 1\},
$$
into the closed ball in $Y$ with center at 0 and radius $\|\Lambda\|$.
An important special case is obtained by taking the complex field for $Y$; in that case we talk about bounded linear functionals.
5.4 Theorem For a linear transformation $\Lambda$ of a normed linear space $X$ into a normed linear space $Y$, each of the following three conditions implies the other two:
(a) $\Lambda$ is bounded.
(b) $\Lambda$ is continuous.
(c) $\Lambda$ is continuous at one point of $X$.

Proof Since $\left\|\Lambda\left(x_1-x_2\right)\right\| \leq\|\Lambda\|\left\|x_1-x_2\right\|$, it is clear that (a) implies (b), and $(b)$ implies (c) trivially. Suppose $\Lambda$ is continuous at $x_0$. To each $\epsilon>0$ one can then find a $\delta>0$ so that $\left\|x-x_0\right\|<\delta$ implies $\left\|\Lambda x-\Lambda x_0\right\|<\epsilon$. In other words, $\|x\|<\delta$ implies
$$
\left\|\Lambda\left(x_0+x\right)-\Lambda x_0\right\|<\epsilon
$$

EXAMPLES OF BANACH SPACE TECHNIQUES

But then the linearity of $\Lambda$ shows that $\|\Lambda x\|<\epsilon$. Hence $\|\Lambda\| \leq \epsilon / \delta$, and (c) implies $(a)$.
$/ / /$

Consequences of Baire's Theorem
5.5 The manner in which the completeness of Banach spaces is frequently exploited depends on the following theorem about complete metric spaces, which also has many applications in other parts of mathematics. It implies two of the three most important theorems which make Banach spaces useful tools in analysis, the Banach-Steinhaus theorem and the open mapping theorem. The third is the Hahn-Banach extension theorem, in which completeness plays no role.
5.6 Baire's Theorem If $X$ is a complete metric space, the intersection of every countable collection of dense open subsets of $X$ is dense in $X$.
In particular (except in the trivial case $X=\varnothing$ ), the intersection is not empty. This is often the principal significance of the theorem.
Proof Suppose $V_1, V_2, V_3, \ldots$ are dense and open in $X$. Let $W$ be any open set in $X$. We have to show that $\bigcap V_n$ has a point in $W$ if $W \neq \varnothing$.
Let $\rho$ be the metric of $X$; let us write
$$
S(x, r)=\{y \in X: \rho(x, y)<r\}
$$
and let $\bar{S}(x, r)$ be the closure of $S(x, r)$. [Note: There exist situations in which $\bar{S}(x, r)$ does not contain all $y$ with $\rho(x, y) \leq r !]$

Since $V_1$ is dense, $W \cap V_1$ is a nonempty open set, and we can therefore find $x_1$ and $r_1$ such that
$$
\bar{S}\left(x_1, r_1\right) \subset W \cap V_1 \text { and } 0<r_1<1 .
$$

If $n \geq 2$ and $x_{n-1}$ and $r_{n-1}$ are chosen, the denseness of $V_n$ shows that $V_n \cap$ $S\left(x_{n-1}, r_{n-1}\right)$ is not empty, and we can therefore find $x_n$ and $r_n$ such that
$$
\bar{S}\left(x_n, r_n\right) \subset V_n \cap S\left(x_{n-1}, r_{n-1}\right) \quad \text { and } 0<r_n<\frac{1}{n} .
$$

By induction, this process produces a sequence $\left\{x_n\right\}$ in $X$. If $i>n$ and $j>n$, the construction shows that $x_i$ and $x_j$ both lie in $S\left(x_n, r_n\right)$, so that $\rho\left(x_i, x_j\right)<2 r_n<2 / n$, and hence $\left\{x_n\right\}$ is a Cauchy sequence. Since $X$ is complete, there is a point $x \in X$ such that $x=\lim _{n \rightarrow \infty} x_n$.

Since $x_i$ lies in the closed set $\bar{S}\left(x_n, r_n\right)$ if $i>n$, it follows that $x$ lies in each $\bar{S}\left(x_n, r_n\right)$, and (3) shows that $x$ lies in each $V_n$. By (2), $x \in W$. This completes the proof.
$/ / /$

Corollary In a complete metric space, the intersection of any countable collection of dense $G_\delta$ 's is again a dense $G_\delta$.
This follows from the theorem, since every $G_\delta$ is the intersection of a countable collection of open sets, and since the union of countably many countable sets is countable.
5.7 Baire's theorem is sometimes called the category theorem, for the following reason.

Call a set $E \subset X$ nowhere dense if its closure $\bar{E}$ contains no nonempty open subset of $X$. Any countable union of nowhere dense sets is called a set of the first category; all other subsets of $X$ are of the second category (Baire's terminology). Theorem 5.6 is equivalent to the statement that no complete metric space is of the first category. To see this, just take complements in the statement of Theorem 5.6.
5.8 The Banach-Steinhaus Theorem Suppose $X$ is a Banach space, $Y$ is a normed linear space, and $\left\{\Lambda_\alpha\right\}$ is a collection of bounded linear transformations of $X$ into $Y$, where a ranges over some index set $A$. Then either there exists an $M<\infty$ such that
$$
\left\|\Lambda_\alpha\right\| \leq M
$$
for every $\alpha \in A$, or
$$
\sup _{\alpha \in A}\left\|\Lambda_\alpha x\right\|=\infty
$$
for all $x$ belonging to some dense $G_\delta$ in $X$.

In geometric terminology, the alternatives are as follows: Either there is a ball $B$ in $Y$ (with radius $M$ and center at 0 ) such that every $\Lambda_\alpha$ maps the unit ball of $X$ into $B$, or there exist $x \in X$ (in fact, a whole dense $G_\delta$ of them) such that no ball in $Y$ contains $\Lambda_\alpha x$ for all $\alpha$.
The theorem is sometimes referred to as the uniform boundedness principle.
Proof Put
$$
\varphi(x)=\sup _{\alpha \in A}\left\|\Lambda_\alpha x\right\| \quad(x \in X)
$$
and let
$$
V_n=\{x: \varphi(x)>n\} \quad(n=1,2,3, \ldots) .
$$

Since each $\Lambda_\alpha$ is continuous and since the norm of $Y$ is a continuous function on $Y$ (an immediate consequence of the triangle inequality, as in the proof of Theorem 4.6), each function $x \rightarrow\left\|\Lambda_\alpha x\right\|$ is continuous on $X$. Hence $\varphi$ is lower semicontinuous, and each $V_n$ is open.
If one of these sets, say $V_N$, fails to be dense in $X$, then there exist an $x_0 \in X$ and an $r>0$ such that $\|x\| \leq r$ implies $x_0+x \notin V_N$; this means that $\varphi\left(x_0+x\right) \leq N$, or
$$
\left\|\Lambda_\alpha\left(x_0+x\right)\right\| \leq N
$$
for all $\alpha \in A$ and all $x$ with $\|x\| \leq r$. Since $x=\left(x_0+x\right)-x_0$, we then have
$$
\left\|\Lambda_\alpha x\right\| \leq\left\|\Lambda_\alpha\left(x_0+x\right)\right\|+\left\|\Lambda_\alpha x_0\right\| \leq 2 N,
$$
and it follows that (1) holds with $M=2 N / r$.
The other possibility is that every $V_n$ is dense in $X$. In that case, $\bigcap V_n$ is a dense $G_\delta$ in $X$, by Baire's theorem; and since $\varphi(x)=\infty$ for every $x \in \bigcap V_n$, the proof is complete.
$/ / /$
5.9 The Open Mapping Theorem Let $U$ and $V$ be the open unit balls of the Banach spaces $X$ and $Y$. To every bounded linear transformation $\Lambda$ of $X$ onto $Y$ there corresponds a $\delta>0$ so that
$$
\Lambda(U) \supset \delta V .
$$

Note the word "onto" in the hypothesis. The symbol $\delta V$ stands for the set $y: y \in V\}$, i.e., the set of all $y \in Y$ with $\|y\|<\delta$.

It follows from (1) and the linearity of $\Lambda$ that the image of every open ball in , with center at $x_0$, say, contains an open ball in $Y$ with center at $\Lambda x_0$. Hence e image of every open set is open. This explains the name of the theorem.

Here is another way of stating (1): To every $y$ with $\|y\|<\delta$ there corresponds $x$ with $\|x\|<1$ so that $\Lambda x=y$.

Proof Given $y \in Y$, there exists an $x \in X$ such that $\Lambda x=y$; if $\|x\|<k$, it follows that $y \in \Lambda(k U)$. Hence $Y$ is the union of the sets $\Lambda(k U)$, for $k=1,2,3, \ldots$. Since $Y$ is complete, the Baire theorem implies that there is a nonempty open set $W$ in the closure of some $\Lambda(k U)$. This means that every point of $W$ is the limit of a sequence $\left\{\Lambda x_i\right\}$, where $x_i \in k U$; from now on, $k$ and $W$ are fixed.

Choose $y_0 \in W$, and choose $\eta>0$ so that $y_0+y \in W$ if $\|y\|<\eta$. For any such $y$ there are sequences $\left\{x_i^{\prime}\right\},\left\{x_i^{\prime \prime}\right\}$ in $k U$ such that
$$
\Lambda x_i^{\prime} \rightarrow y_0, \quad \Lambda x_i^{\prime \prime} \rightarrow y_0+y \quad(i \rightarrow \infty) .
$$

Setting $x_i=x_i^{\prime \prime}-x_i^{\prime}$, we have $\left\|x_i\right\|<2 k$ and $\Lambda x_i \rightarrow y$. Since this holds for every $y$ with $\|y\|<\eta$, the linearity of $\Lambda$ shows that the following is true, if $\delta=\eta / 2 k$ :
To each $y \in Y$ and to each $\epsilon>0$ there corresponds an $x \in X$ such that
$$
\|x\| \leq \delta^{-1}\|y\| \text { and }\|y-\Lambda x\|<\epsilon .
$$

This is almost the desired conclusion, as stated just before the start of the proof, except that there we had $\epsilon=0$.
Fix $y \in \delta V$, and fix $\epsilon>0$. By (3) there exists an $x_1$ with $\left\|x_1\right\|<1$ and
$$
\left\|y-\Lambda x_1\right\|<\frac{1}{2} \delta \epsilon .
$$

Suppose $x_1, \ldots, x_n$ are chosen so that
$$
\left\|y-\Lambda x_1-\cdots-\Lambda x_n\right\|<2^{-n} \delta \epsilon .
$$

Use (3), with $y$ replaced by the vector on the left side of (5), to obtain an $x_{n+1}$ so that (5) holds with $n+1$ in place of $n$, and
$$
\left\|x_{n+1}\right\|<2^{-n} \epsilon \quad(n=1,2,3, \ldots) .
$$

If we set $s_n=x_1+\cdots+x_n$,(6) shows that $\left\{s_n\right\}$ is a Cauchy sequence in $X$. Since $X$ is complete, there exists an $x \in X$ so that $s_n \rightarrow x$. The inequality $\left\|x_1\right\|<1$, together with (6), shows that $\|x\|<1+\epsilon$. Since $\Lambda$ is continuous, $\Lambda s_n \rightarrow \Lambda x$. By (5), $\Lambda s_n \rightarrow y$. Hence $\Lambda x=y$.
We have now proved that
$$
\Lambda((1+\epsilon) U) \supset \delta V
$$
or
$$
\Lambda(U) \supset(1+\epsilon)^{-1} \delta V,
$$
for every $\epsilon>0$. The union of the sets on the right of (8), taken over all $\epsilon>0$, is $\delta V$. This proves (1).
IIII
5.10 Theorem If $X$ and $Y$ are Banach spaces and if $\Lambda$ is a bounded linear transformation of $X$ onto $Y$ which is also one-to-one, then there is a $\delta>0$ such that
$$
\|\Lambda x\| \geq \delta\|x\| \quad(x \in X) .
$$

In other words, $\Lambda^{-1}$ is a bounded linear transformation of $Y$ onto $X$.
Proof If $\delta$ is chosen as in the statement of Theorem 5.9, the conclusion of that theorem, combined with the fact that $\Lambda$ is now one-to-one, shows that $\|\Lambda x\|<\delta$ implies $\|x\|<1$. Hence $\|x\| \geq 1$ implies $\|\Lambda x\| \geq \delta$, and (1) is proved.

The transformation $\Lambda^{-1}$ is defined on $Y$ by the requirement that $\Lambda^{-1} y=x$ if $y=\Lambda x$. A trivial verification shows that $\Lambda^{-1}$ is linear, and (1) implies that $\left\|\Lambda^{-1}\right\| \leq 1 / \delta$.
//II

Fourier Series of Continuous Functions
5.11 A Convergence Problem Is it true for every $f \in C(T)$ that the Fourier series of $f$ converges to $f(x)$ at every point $x$ ?
Let us recall that the $n$th partial sum of the Fourier series of $f$ at the point $x$ is given by
$$
s_n(f ; x)=\frac{1}{2 \pi} \int_{-\pi}^\pi f(t) D_n(x-t) d t \quad(n=0,1,2, \ldots),
$$
where
$$
D_n(t)=\sum_{k=-n}^n e^{i k t}
$$

This follows directly from formulas 4.26(1) and 4.26(3).
The problem is to determine whether
$$
\lim _{n \rightarrow \infty} s_n(f ; x)=f(x)
$$
for every $f \in C(T)$ and for every real $x$. We observed in Sec. 4.26 that the partial sums do converge to $f$ in the $L^2$-norm, and Theorem 3.12 implies therefore that each $f \in L^2(T)$ [hence also each $f \in C(T)$ ] is the pointwise limit a.e. of some subsequence of the full sequence of the partial sums. But this does not answer the present question.

We shall see that the Banach-Steinhaus theorem answers the question negatively. Put
$$
s^*(f ; x)=\sup _n\left|s_n(f ; x)\right| .
$$

To begin with, take $x=0$, and define
$$
\Lambda_n f=s_n(f ; 0) \quad(f \in C(T) ; n=1,2,3, \ldots) .
$$

We know that $C(T)$ is a Banach space, relative to the supremum norm $\|f\|_{\infty}$. It follows from (1) that each $\Lambda_n$ is a bounded linear functional on $C(T)$, of norm
$$
\left\|\Lambda_n\right\| \leq \frac{1}{2 \pi} \int_{-\pi}^\pi\left|D_n(t)\right| d t=\left\|D_n\right\|_1 .
$$

We claim that
$$
\left\|\Lambda_n\right\| \rightarrow \infty \quad \text { as } n \rightarrow \infty .
$$

This will be proved by showing that equality holds in (6) and that
$$
\left\|D_n\right\|_1 \rightarrow \infty \quad \text { as } n \rightarrow \infty .
$$

Multiply (2) by $e^{i t / 2}$ and by $e^{-i t / 2}$ and subtract one of the resulting two equations from the other, to obtain
$$
D_n(t)=\frac{\sin \left(n+\frac{1}{2}\right) t}{\sin (t / 2)}
$$
Since $|\sin x| \leq|x|$ for all real $x$, (9) shows that
$$
\begin{aligned}
\left\|D_n\right\|_1 & >\frac{2}{\pi} \int_0^\pi\left|\sin \left(n+\frac{1}{2}\right) t\right| \frac{d t}{t}=\frac{2}{\pi} \int_0^{(n+1 / 2) \pi}|\sin t| \frac{d t}{t} \\
& >\frac{2}{\pi} \sum_{k=1}^n \frac{1}{k \pi} \int_{(k-1) \pi}^{k \pi}|\sin t| d t=\frac{4}{\pi^2} \sum_{k=1}^n \frac{1}{k} \rightarrow \infty,
\end{aligned}
$$
which proves (8).
Next, fix $n$, and put $g(t)=1$ if $D_n(t) \geq 0, g(t)=-1$ if $D_n(t)<0$. There exist $f_j \in C(T)$ such that $-1 \leq f_j \leq 1$ and $f_j(t) \rightarrow g(t)$ for every $t$, as $j \rightarrow \infty$. By the dominated convergence theorem,
$$
\lim _{j \rightarrow \infty} \Lambda_n\left(f_j\right)=\lim _{j \rightarrow \infty} \frac{1}{2 \pi} \int_{-\pi}^\pi f_j(-t) D_n(t) d t=\frac{1}{2 \pi} \int_{-\pi}^\pi g(-t) D_n(t) d t=\left\|D_n\right\|_1 .
$$

Thus equality holds in (6), and we have proved (7).
Since (7) holds, the Banach-Steinhaus theorem asserts now that $s^*(f ; 0)=\infty$ for every $f$ in some dense $G_\delta$-set in $C(T)$.

We chose $x=0$ just for convenience. It is clear that the same result holds for every other $x$ :

To each real number $x$ there corresponds a set $E_x \subset C(T)$ which is a dense $G_\delta$ in $C(T)$, such that $s^*(f ; x)=\infty$ for every $f \in E_x$.

In particular, the Fourier series of each $f \in E_x$ diverges at $x$, and we have a negative answer to our question. (Exercise 22 shows the answer is positive if mere continuity is replaced by a somewhat stronger smoothness assumption.)

It is interesting to observe that the above result can be strengthened by another application of Baire's theorem. Let us take countably many points $x_i$, and let $E$ be the intersection of the corresponding sets
$$
E_{x_i} \subset C(T) .
$$

By Baire's theorem, $E$ is a dense $G_\delta$ in $C(T)$. Every $f \in E$ has
$$
s^*\left(f ; x_i\right)=\infty
$$
at every point $x_i$.
For each $f, s^*(f ; x)$ is a lower semicontinuous function of $x$, since (4) exhibits it as the supremum of a collection of continuous functions. Hence $\left\{x: s^*(f ; x)=\infty\right\}$ is a $G_\delta$ in $R^1$, for each $f$. If the above points $x_i$ are taken so that their union is dense in $(-\pi, \pi)$, we obtain the following result :
5.12 Theorem There is a set $E \subset C(T)$ which is a dense $G_\delta$ in $C(T)$ and which has the following property: For each $f \in E$, the set
$$
Q_f=\left\{x: s^*(f ; x)=\infty\right\}
$$
is a dense $G_\delta$ in $R^1$.

This gains in interest if we realize that $E$, as well as each $Q_f$, is an uncountable set:
5.13 Theorem In a complete metric space $X$ which has no isolated points, no countable dense set is $a G_\delta$.
ProOF Let $x_k$ be the points of a countable dense set $E$ in $X$. Assume that $E$ is a $G_\delta$. Then $E=\bigcap V_n$, where each $V_n$ is dense and open. Let
$$
W_n=V_n-\bigcup_{k=1}^n\left\{x_k\right\}
$$

Then each $W_n$ is still a dense open set, but $\bigcap W_n=\varnothing$, in contradiction to Baire's theorem.
$/ / /$

Note: A slight change in the proof of Baire's theorem shows actually that every dense $G_\delta$ contains a perfect set if $X$ is as above.

Fourier Coefficients of $L^1$-functions
5.14 As in Sec. 4.26, we associate to every $f \in L^1(T)$ a function $\hat{f}$ on $Z$ defined by
$$
\hat{f}(n)=\frac{1}{2 \pi} \int_{-\pi}^\pi f(t) e^{-i n t} d t \quad(n \in Z) .
$$

It is easy to prove that $\hat{f}(n) \rightarrow 0$ as $|n| \rightarrow \infty$, for every $f \in L^1$. For we know that $C(T)$ is dense in $L^1(T)$ (Theorem 3.14) and that the trigonometric polynomials are dense in $C(T)$ (Theorem 4.25). If $\epsilon>0$ and $f \in L^1(T)$, this says that there is a $g \in C(T)$ and a trigonometric polynomial $P$ such that $\|f-g\|_1<\epsilon$ and $\|g-P\|_{\infty}<\epsilon$. Since
$$
\|g-P\|_1 \leq\|g-P\|_{\infty}
$$
if follows that $\|f-P\|_1<2 \epsilon$; and if $|n|$ is large enough (depending on $P$ ), then
$$
|\hat{f}(n)|=\left|\frac{1}{2 \pi} \int_{-\pi}^\pi\{f(t)-P(t)\} e^{-i n t} d t\right| \leq\|f-P\|_1<2 \epsilon .
$$

Thus $\hat{f}(n) \rightarrow 0$ as $n \rightarrow \pm \infty$. This is known as the Riemann-Lebesgue lemma.
The question we wish to raise is whether the converse is true. That is to say, if $\left\{a_n\right\}$ is a sequence of complex numbers such that $a_n \rightarrow 0$ as $n \rightarrow \pm \infty$, does it follow that there is an $f \in L^1(T)$ such that $\hat{f}(n)=a_n$ for all $n \in Z$ ? In other words, is something like the Riesz-Fischer theorem true in this situation?

This can easily be answered (negatively) with the aid of the open mapping theorem.
Let $c_0$ be the space of all complex functions $\varphi$ on $Z$ such that $\varphi(n) \rightarrow 0$ as $n \rightarrow \pm \infty$, with the supremum norm
$$
\|\varphi\|_{\infty}=\sup \{|\varphi(n)|: n \in Z\} .
$$

Then $c_0$ is easily seen to be a Banach space. In fact, if we declare every subset of $Z$ to be open, then $Z$ is a locally compact Hausdorff space, and $c_0$ is nothing but $C_0(Z)$.
The following theorem contains the answer to our question:
5.15 Theorem The mapping $f \rightarrow \hat{f}$ is a one-to-one bounded linear transformation of $L^1(T)$ into (but not onto) $c_0$.
Proof Define $\Lambda$ by $\Lambda f=\hat{f}$. It is clear that $\Lambda$ is linear. We have just proved that $\Lambda$ maps $L^1(T)$ into $c_0$, and formula $5.14(1)$ shows that $|\hat{f}(n)| \leq\|f\|_1$, so that $\|\Lambda\| \leq 1$. (Actually, $\|\Lambda\|=1$; to see this, take $f=1$.) Let us now prove that $\Lambda$ is one-to-one. Suppose $f \in L^1(T)$ and $\hat{f}(n)=0$ for every $n \in Z$. Then
$$
\int_{-\pi}^\pi f(t) g(t) d t=0
$$
if $g$ is any trigonometric polynomial. By Theorem 4.25 and the dominated convergence theorem, (1) holds for every $g \in C(T)$. Apply the dominated convergence theorem once more, in conjunction with the Corollary to Lusin's theorem, to conclude that (1) holds if $g$ is the characteristic function of any measurable set in $T$. Now Theorem $1.39(b)$ shows that $f=0$ a.e.

If the range of $\Lambda$ were all of $c_0$, Theorem 5.10 would imply the existence of a $\delta>0$ such that
$$
\|\hat{f}\|_{\infty} \geq \delta\|f\|_1
$$
for every $f \in L^1(T)$. But if $D_n(t)$ is defined as in Sec. 5.11, then $D_n \in L^1(T)$, $\left\|\hat{D}_n\right\|_{\infty}=1$ for $n=1,2,3, \ldots$, and $\left\|D_n\right\|_1 \rightarrow \infty$ as $n \rightarrow \infty$. Hence there is no $\delta>0$ such that the inequalities
$$
\left\|\hat{D}_n\right\|_{\infty} \geq \delta\left\|D_n\right\|_1
$$
hold for every $n$.
This completes the proof.

The Hahn-Banach Theorem
5.16 Theorem If $M$ is a subspace of a normed linear space $X$ and if $f$ is a bounded linear functional on $M$, then $f$ can be extended to a bounded linear functional $F$ on $X$ so that $\|F\|=\|f\|$.
Note that $M$ need not be closed.

Before we turn to the proof, some comments seem called for. First, to say (in the most general situation) that a function $F$ is an extension of $f$ means that the domain of $F$ includes that of $f$ and that $F(x)=f(x)$ for all $x$ in the domain of $f$. Second, the norms $\|F\|$ and $\|f\|$ are computed relative to the domains of $F$ and $f$; explicitly,
$$
\|f\|=\sup \{|f(x)|: x \in M,\|x\| \leq 1\}, \quad\|F\|=\sup \{|F(x)|: x \in X,\|x\| \leq 1\},
$$

The third comment concerns the field of scalars. So far everything has been stated for complex scalars, but the complex field could have been replaced by the real field without any changes in statements or proofs. The Hahn-Banach theorem is also true in both cases; nevertheless, it appears to be essentially a "real" theorem. The fact that the complex case was not yet proved when Banach wrote his classical book "Opérations linéaires" may be the main reason that real scalars are the only ones considered in his work.

It will be helpful to introduce some temporary terminology. Recall that $V$ is a complex (real) vector space if $x+y \in V$ for $x$ and $y \in V$, and if $a x \in V$ for all complex (real) numbers $\alpha$. It follows trivially that every complex vector space is also a real vector space. A complex function $\varphi$ on a complex vector space $V$ is a complex-linear functional if
$$
\varphi(x+y)=\varphi(x)+\varphi(y) \quad \text { and } \quad \varphi(\alpha x)=\alpha \varphi(x)
$$
for all $x$ and $y \in V$ and all complex $\alpha$. A real-valued function $\varphi$ on a complex (real) vector space $V$ is a real-linear functional if (1) holds for all real $\alpha$.

If $u$ is the real part of a complex-linear functional $f$, i.e., if $u(x)$ is the real part of the complex number $f(x)$ for all $x \in V$, it is easily seen that $u$ is a real-linear functional. The following relations hold between $f$ and $u$ :
5.17 Proposition Let $V$ be a complex vector space.
(a) If $u$ is the real part of a complex-linear functional $f$ on $V$, then
$$
f(x)=u(x)-i u(i x) \quad(x \in V) .
$$
(b) If $u$ is a real-linear functional on $V$ and if $f$ is defined by (1), then $f$ is a complex-linear functional on $V$.
(c) If $V$ is a normed linear space and $f$ and $u$ are related as in (1), then $\|f\|=\|u\|$.

Proof If $\alpha$ and $\beta$ are real numbers and $z=\alpha+i \beta$, the real part of $i z$ is $-\beta$. This gives the identity
$$
z=\operatorname{Re} z-i \operatorname{Re}(i z)
$$
for all complex numbers $z$. Since
$$
\operatorname{Re}(i f(x))=\operatorname{Re} f(i x)=u(i x),
$$
(1) follows from (2) with $z=f(x)$.

Under the hypotheses $(b)$, it is clear that $f(x+y)=f(x)+f(y)$ and that $f(\alpha x)=\alpha f(x)$ for all real $\alpha$. But we also have
$$
f(i x)=u(i x)-i u(-x)=u(i x)+i u(x)=i f(x),
$$
which proves that $f$ is complex-linear.
Since $|u(x)| \leq|f(x)|$, we have $\|u\| \leq\|f\|$. On the other hand, to every $x \in V$ there corresponds a complex number $\alpha,|\alpha|=1$, so that $\alpha f(x)=|f(x)|$. Then
$$
|f(x)|=f(\alpha x)=u(\alpha x) \leq\|u\| \cdot\|\alpha x\|=\|u\| \cdot\|x\|,
$$
which proves that $\|f\| \leq\|u\|$.

5.18 Proof of Theorem 5.16 We first assume that $X$ is a real normed linear space and, consequently, that $f$ is a real-linear bounded functional on $M$. If $\|f\|=0$, the desired extension is $F=0$. Omitting this case, there is no loss of generality in assuming that $\|f\|=1$.

Choose $x_0 \in X, x_0 \notin M$, and let $M_1$ be the vector space spanned by $M$ and $x_0$. Then $M_1$ consists of all vectors of the form $x+\lambda x_0$, where $x \in M$ and $\lambda$ is a real scalar. If we define $f_1\left(x+\lambda x_0\right)=f(x)+\lambda \alpha$, where $\alpha$ is any fixed real number, it is trivial to verify that an extension of $f$ to a linear functional on $M_1$ is obtained. The problem is to choose $\alpha$ so that the extended functional still has norm 1 . This will be the case provided that
$$
|f(x)+\lambda \alpha| \leq\left\|x+\lambda x_0\right\| \quad(x \in M, \lambda \text { real }) .
$$

Replace $x$ by $-\lambda x$ and divide both sides of (1) by $|\lambda|$. The requirement is then that
$$
|f(x)-\alpha| \leq\left\|x-x_0\right\| \quad(x \in M),
$$
i.e., that $A_x \leq \alpha \leq B_x$ for all $x \in M$, where
$$
A_x=f(x)-\left\|x-x_0\right\| \quad \text { and } B_x=f(x)+\left\|x-x_0\right\| .
$$

There exists such an $\alpha$ if and only if all the intervals $\left[A_x, B_x\right]$ have a common point, i.e., if and only if
$$
A_x \leq B_y
$$
for all $x$ and $y \in M$. But
$$
f(x)-f(y)=f(x-y) \leq\|x-y\| \leq\left\|x-x_0\right\|+\left\|y-x_0\right\|,
$$
and so (4) follows from (3).
We have now proved that there exists a norm-preserving extension $f_1$ of $f$ on $M_1$.

Let $\mathscr{P}$ be the collection of all ordered pairs $\left(M^{\prime}, f^{\prime}\right)$, where $M^{\prime}$ is a subspace of $X$ which contains $M$ and where $f^{\prime}$ is a real-linear extension of $f$ to $M^{\prime}$, with $\left\|f^{\prime}\right\|=1$. Partially order $\mathscr{P}$ by declaring $\left(M^{\prime}, f^{\prime}\right) \leq\left(M^{\prime \prime}, f^{\prime \prime}\right)$ to mean that $M^{\prime} \subset M^{\prime \prime}$ and $f^{\prime \prime}(x)=f^{\prime}(x)$ for all $x \in M^{\prime}$. The axioms of a partial order are clearly satisfied, $\mathscr{P}$ is not empty since it contains $(M, f)$, and so the Hausdorff maximality theorem asserts the existence of a maximal totally ordered subcollection $\Omega$ of $\mathscr{P}$.

Let $\Phi$ be the collection of all $M^{\prime}$ such that $\left(M^{\prime}, f^{\prime}\right) \in \Omega$. Then $\Phi$ is totally ordered, by set inclusion, and therefore the union $\tilde{M}$ of all members of $\Phi$ is a subspace of $X$. (Note that in general the union of two subspaces is not a subspace. An example is two planes through the origin in $R^3$.) If $x \in \tilde{M}$, then $x \in M^{\prime}$ for some $M^{\prime} \in \Phi$; define $F(x)=f^{\prime}(x)$, where $f^{\prime}$ is the function which occurs in the pair $\left(M^{\prime}, f^{\prime}\right) \in \Omega$. Our definition of the partial order in $\Omega$ shows that it is immaterial which $M^{\prime} \in \Phi$ we choose to define $F(x)$, as long as $M^{\prime}$ contains $x$.

It is now easy to check that $F$ is a linear functional on $\tilde{M}$, with $\|F\|=1$. If $\tilde{M}$ were a proper subspace $X$, the first part of the proof would give us a further extension of $F$, and this would contradict the maximality of $\Omega$. Thus $\tilde{M}=X$, and the proof is complete for the case of real scalars.

If now $f$ is a complex-linear functional on the subspace $M$ of the complex normed linear space $X$, let $u$ be the real part of $f$, use the real Hahn-Banach theorem to extend $u$ to a real-linear functional $U$ on $X$, with $\|U\|=\|u\|$, and define
$$
F(x)=U(x)-i U(i x) \quad(x \in X) .
$$

By Proposition 5.17, $F$ is a complex-linear extension of $f$, and
$$
\|F\|=\|U\|=\|u\|=\|f\| .
$$

This completes the proof.


Let us mention two important consequences of the Hahn-Banach theorem:
5.19 Theorem Let $M$ be a linear subspace of a normed linear space $X$, and let $x_0 \in X$. Then $x_0$ is in the closure $\bar{M}$ of $M$ if and only if there is no bounded linear functional $f$ on $X$ such that $f(x)=0$ for all $x \in M$ but $f\left(x_0\right) \neq 0$.

Proof If $x_0 \in \bar{M}, f$ is a bounded linear functional on $X$, and $f(x)=0$ for all $x \in M$, the continuity of $f$ shows that we also have $f\left(x_0\right)=0$.

Conversely, suppose $x_0 \notin \bar{M}$. Then there exists a $\delta>0$ such that $\left\|x-x_0\right\|>\delta$ for all $x \in M$. Let $M^{\prime}$ be the subspace generated by $M$ and $x_0$, and define $f\left(x+\lambda x_0\right)=\lambda$ if $x \in M$ and $\lambda$ is a scalar. Since
$$
\delta|\lambda| \leq|\lambda|\left\|x_0+\lambda^{-1} x\right\|=\left\|\lambda x_0+x\right\|,
$$
we see that $f$ is a linear functional on $M^{\prime}$ whose norm is at most $\delta^{-1}$. Also $f(x)=0$ on $M, f\left(x_0\right)=1$. The Hahn-Banach theorem allows us to extend this $f$ from $M^{\prime}$ to $X$.

5.20 Theorem If $X$ is a normed linear space and if $x_0 \in X, x_0 \neq 0$, there is a bounded linear functional $f$ on $X$, of norm 1 , so that $f\left(x_0\right)=\left\|x_0\right\|$.

Proof Let $M=\left\{\lambda x_0\right\}$, and define $f\left(\lambda x_0\right)=\lambda\left\|x_0\right\|$. Then $f$ is a linear functional of norm 1 on $M$, and the Hahn-Banach theorem can again be applied. ////
5.21 Remarks If $X$ is a normed linear space, let $X^*$ be the collection of all bounded linear functionals on $X$. If addition and scalar multiplication of linear functionals are defined in the obvious manner, it is easy to see that $X^*$ is again a normed linear space. In fact, $X^*$ is a Banach space; this follows from the fact that the field of scalars is a complete metric space. We leave the verification of these properties of $X^*$ as an exercise.
One of the consequences of Theorem 5.20 is that $X^*$ is not the trivial vector space (i.e., $X^*$ consists of more than 0 ) if $X$ is not trivial. In fact, $X^*$ separates points on $X$. This means that if $x_1 \neq x_2$ in $X$ there exists an $f \in X^*$ such that $f\left(x_1\right) \neq f\left(x_2\right)$. To prove this, merely take $x_0=x_2-x_1$ in Theorem 5.20 .
Another consequence is that, for $x \in X$,
$$
\|x\|=\sup \left\{|f(x)|: f \in X^*,\|f\|=1\right\} .
$$

Hence, for fixed $x \in X$, the mapping $f \rightarrow f(x)$ is a bounded linear functional on $X^*$, of norm $\|x\|$.

This interplay between $X$ and $X^*$ (the so-called "dual space" of $X$ ) forms the basis of a large portion of that part of mathematics which is known as functional analysis.

An Abstract Approach to the Poisson Integral
5.22 Successful applications of the Hahn-Banach theorem to concrete problems depend of course on a knowledge of the bounded linear functionals on the normed linear space under consideration. So far we have only determined the bounded linear functionals on a Hilbert space (where a much simpler proof of the Hahn-Banach theorem exists; see Exercise 6), and we know the positive linear functionals on $C_c(X)$.

We shall now describe a general situation in which the last-mentioned functionals occur naturally.

Let $K$ be a compact Hausdorff space, let $H$ be a compact subset of $K$, and let $A$ be a subspace of $C(K)$ such that $1 \in A$ (1 denotes the function which assigns the number 1 to each $x \in K$ ) and such that
$$
\|f\|_K=\|f\|_H \quad(f \in A) .
$$

Here we used the notation
$$
\|f\|_E=\sup \{|f(x)|: x \in E\} .
$$

Because of the example discussed in Sec. 5.23, $\mathrm{H}$ is sometimes called a boundary of $K$, corresponding to the space $A$.

If $f \in A$ and $x \in K$,(1) says that
$$
|f(x)| \leq\|f\|_H .
$$

In particular, if $f(y)=0$ for every $y \in H$, then $f(x)=0$ for all $x \in K$. Hence if $f_1$ and $f_2 \in A$ and $f_1(y)=f_2(y)$ for every $y \in H$, then $f_1=f_2$; to see this, put $f=$ $f_1-f_2$.

Let $M$ be the set of all functions on $H$ that are restrictions to $H$ of members of $A$. It is clear that $M$ is a subspace of $C(H)$. The preceding remark shows that each member of $M$ has a unique extension to a member of $A$. Thus we have a natural one-to-one correspondence between $M$ and $A$, which is also normpreserving, by (1). Hence it will cause no confusion if we use the same letter to designate a member of $A$ and its restriction to $H$.

Fix a point $x \in K$. The inequality (3) shows that the mapping $f \rightarrow f(x)$ is a bounded linear functional on $M$, of norm 1 [since equality holds in (3) if $f=1$ ]. By the Hahn-Banach theorem there is a linear functional $\Lambda$ on $C(H)$, of norm 1, such that
$$
\Lambda f=f(x) \quad(f \in M) .
$$

We claim that the properties
$$
\Lambda 1=1, \quad\|\Lambda\|=1
$$
imply that $\Lambda$ is a positive linear functional on $C(H)$.
To prove this, suppose $f \in C(H), 0 \leq f \leq 1$, put $g=2 f-1$, and put $\Lambda g=\alpha+i \beta$, where $\alpha$ and $\beta$ are real. Note that $-1 \leq g \leq 1$, so that $|g+i r|^2 \leq 1+r^2$ for every real constant $r$. Hence (5) implies that
$$
(\beta+r)^2 \leq|\alpha+i(\beta+r)|^2=|\Lambda(g+i r)|^2 \leq 1+r^2 .
$$

Thus $\beta^2+2 r \beta \leq 1$ for every real $r$, which forces $\beta=0$. Since $\|g\|_H \leq 1$, we have $|\alpha| \leq 1$; hence
$$
\Lambda f=\frac{1}{2} \Lambda(1+g)=\frac{1}{2}(1+\alpha) \geq 0 .
$$

Now Theorem 2.14 can be applied. It shows that there is a regular positive Borel measure $\mu_x$ on $H$ such that
$$
\Lambda f=\int_H f d \mu_x \quad(f \in C(H)) .
$$

In particular, we get the representation formula
$$
f(x)=\int_H f d \mu_x \quad(f \in A) .
$$

What we have proved is that to each $x \in K$ there corresponds a positive measure $\mu_x$ on the "boundary" $H$ which "represents" $x$ in the sense that (9) holds for every $f \in A$.

Note that $\Lambda$ determines $\mu_x$ uniquely; but there is no reason to expect the Hahn-Banach extension to be unique. Hence, in general, we cannot say much about the uniqueness of the representing measures. Under special circumstances we do get uniqueness, as we shall see presently.
5.23 To see an example of the preceding situation, let $U=\{z:|z|<1\}$ be the open unit disc in the complex plane, put $K=\bar{U}$ (the closed unit disc), and take for $H$ the boundary $T$ of $U$. We claim that every polynomial $f$, i.e., every function of the form
$$
f(z)=\sum_{n=0}^N a_n z^n,
$$
where $a_0, \ldots, a_N$ are complex numbers, satisfies the relation
$$
\|f\|_U=\|f\|_T .
$$
(Note that the continuity of $f$ shows that the supremum of $|f|$ over $U$ is the same as that over $\bar{U}$.)

Since $\bar{U}$ is compact, there exists a $z_0 \in \bar{U}$ such that $\left|f\left(z_0\right)\right| \geq|f(z)|$ for all $z \in \bar{U}$. Assume $z_0 \in U$. Then
$$
f(z)=\sum_{n=0}^N b_n\left(z-z_0\right)^n
$$
and if $0<r<1-\left|z_0\right|$, we obtain
$$
\sum_{n=0}^N\left|b_n\right|^2 r^{2 n}=\frac{1}{2 \pi} \int_{-\pi}^\pi\left|f\left(z_0+r e^{i \theta}\right)\right|^2 d \theta \leq \frac{1}{2 \pi} \int_{-\pi}^\pi\left|f\left(z_0\right)\right|^2 d \theta=\left|b_0\right|^2,
$$
so that $b_1=b_2=\cdots=b_N=0$; i.e., $f$ is constant. Thus $z_0 \in T$ for every nonconstant polynomial $f$, and this proves (2).
(We have just proved a special case of the maximum modulus theorem; we shall see later that this is an important property of all holomorphic functions.)
5.24 The Poisson Integral Let $A$ be any subspace of $C(\bar{U})$ (where $\bar{U}$ is the closed unit disc, as above) such that $A$ contains all polynomials and such that
$$
\|f\|_U=\|f\|_T
$$
holds for every $f \in A$. We do not exclude the possibility that $A$ consists of precisely the polynomials, but $\boldsymbol{A}$ might be larger.

The general result obtained in Sec. 5.22 applies to $A$ and shows that to each $z \in U$ there corresponds a positive Borel measure $\mu_z$ on $T$ such that
$$
f(z)=\int_T f d \mu_z \quad(f \in A) .
$$
(This also holds for $z \in T$, but is then trivial: $\mu_z$ is simply the unit mass concentrated at the point $z$.)

We now fix $z \in U$ and write $z=r e^{i \theta}, 0 \leq r<1, \theta$ real.
If $u_n(w)=w^n$, then $u_n \in A$ for $n=0,1,2, \ldots$; hence (2) shows that
$$
r^n e^{i n \theta}=\int_T u_n d \mu_z \quad(n=0,1,2, \ldots) .
$$

Since $u_{-n}=\bar{u}_n$ on $T,(3)$ leads to
$$
\int_T u_n d \mu_z=r^{|n|} e^{i n \theta} \quad(n=0, \pm 1, \pm 2, \ldots) .
$$

This suggests that we look at the real function
$$
P_r(\theta-t)=\sum_{n=-\infty}^{\infty} r^{|n|} e^{i n(\theta-t)} \quad(t \text { real })
$$
since
$$
\frac{1}{2 \pi} \int_{-\pi}^\pi P_r(\theta-t) e^{i n t} d t=r^{|n|} e^{i n \theta} \quad(n=0, \pm 1, \pm 2, \ldots) .
$$

Note that the series (5) is dominated by the convergent geometric series $\sum r^{|n|}$, so that it is legitimate to insert the series into the integral (6) and to integrate term by term, which gives (6). Comparison of (4) and (6) gives
$$
\int_T f d \mu_z=\frac{1}{2 \pi} \int_{-\pi}^\pi f\left(e^{i t}\right) P_r(\theta-t) d t
$$
for $f=u_n$, hence for every trigonometric polynomial $f$, and Theorem 4.25 now implies that (7) holds for every $f \in C(T)$. [This shows that $\mu_z$ was uniquely determined by (2). Why?]
In particular, (7) holds if $f \in A$, and then (2) gives the representation
$$
f(z)=\frac{1}{2 \pi} \int_{-\pi}^\pi f\left(e^{i t}\right) P_r(\theta-t) d t \quad(f \in A) .
$$

The series (5) can be summed explicitly, since it is the real part of
$$
1+2 \sum_1^{\infty}\left(z e^{-i t}\right)^n=\frac{e^{i t}+z}{e^{i t}-z}=\frac{1-r^2+2 i r \sin (\theta-t)}{\left|1-z e^{-i t}\right|^2} .
$$

Thus
$$
P_r(\theta-t)=\frac{1-r^2}{1-2 r \cos (\theta-t)+r^2} .
$$

This is the so-called "Poisson kernel." Note that $P_r(\theta-t) \geq 0$ if $0 \leq r<1$. We now summarize what we have proved:
112 REAL AND COMPLEX ANALYSIS
5.25 Theorem Suppose $A$ is a vector space of continuous complex functions on the closed unit disc $\bar{U}$. If $A$ contains all polynomials, and if
$$
\sup _{z \in U}|f(z)|=\sup _{z \in T}|f(z)|
$$
for every $f \in A$ (where $T$ is the unit circle, the boundary of $U$ ), then the Poisson integral representation
$$
f(z)=\frac{1}{2 \pi} \int_{-\pi}^\pi \frac{1-r^2}{1-2 r \cos (\theta-t)+r^2} f\left(e^{i t}\right) d t \quad\left(z=r e^{i \theta}\right)
$$
is valid for every $f \in A$ and every $z \in U$.