CHAPTER THREE LP-SPACES Convex Functions and Inequalities Many of the most common inequalities in analysis have their origin in the notion of convexity 3.1 Definition A real functiongdefined on a segment((α,b),where $-\infty\leq a<b\leq\infty,$ is called convex if the inequality $$ \varphi((1-\lambda)x+\lambda y)\leq(1-\lambda)\varphi(x)+\lambda\varphi(y) $$ (1) holds whenever $a<x<b,a<y<b,$ and O $|\leq\lambda\leq1.$ then the point (t,qp(t) p(y)) in Graphically the condition is that if $x<t<y,$ $(y,$ should lie below or on the line connecting the points (x, q(x) and the plane. Also,(1) is equivalent to the requirement that $$ \frac{\varphi(t)-\varphi(s)}{t-s}\leq\frac{\varphi(u)-\varphi(t)}{u-t} $$ (2) whenever $a<s<t<u<b,$ The mean value theorem for differentiation, combined with (2), shows imme $a<s<t<b$ implies diately that a real differentiable function $\varphi$ is convex in (a,b) if and only if is a mono $\varphi(s)\leq\varphi^{\prime}(t),$ i.e., if and only if the derivative ${\boldsymbol{\varphi}}^{\prime}$ tonically increasing function. For example, the exponential function is convex on $\left(-\right)$ 00,0O) 3.2 Theorem If p is convex on (α,b) then gp is continuous on (a,b) 6162 REAL AND cOMPLEX ANALYSIS PROOF The idea of the proof is most easily conveyed in geometric language Those who may worry that this is not“rigorous”are invited to transcribe it in terms of epsilons and deltas. Suppose $a<s<x<y<t<b.$ Write S for the point ((s,p(S) in the plane, and deal similarly with x, y, and t. Then $\textstyle X{\ ~}$ is on or below the line ${\cal S}Y,$ hence ${\mathbf{}}Y$ is on or above the line through $\boldsymbol{\mathsf{S}}$ and X $X$ Y; also, ${\bf Y_{\nu}}$ is on or below $X T$ As y→x, it follows that $\scriptstyle Y\to X.$ i.e. $\varphi(y)\to\varphi(x).$ Left-hand limits are handled / in the same manner, and the continuity of $\varphi$ follows. Note that this theorem depends on the fact that we are working on an open segment. For instance, if $\varphi(x)=0$ on [O,1) and $\varphi(1)=1,$ then $\varphi$ satisfies 3.1(1) on [0,1] without being continuous. 3.3 Theorem (Jensen's Inequality) Let p be a positive measure on a o-algebra oR in $\bar{a}$ set $\Omega,$ so that $\mu(\Omega)=1$ If f is a real function in $\scriptstyle I_{\mathrm{(gh}}$ if a $<f(x)<b$ for all $\mathrm{*e.a.}$ and if p is convex on (a, $b\mathrm{/}\mathrm{/}$ , then $$ \varphi{\Biggl(}\bigcap_{\Omega}f\,d\mu{\Biggr)}\leq\int_{\Omega}(\varphi\circ f)\ d\mu. $$ (1) Note: The cases $a=-\infty$ and $b=x$ are not excluded. It may happen that $\scriptstyle{\phi\circ f}$ is not in $\scriptstyle I_{(0)};$ in that case,as the proof will show, the integral of $\varphi\circ f$ exists in the extended sense described in Sec. 1.31, and its value is $\scriptstyle+\,\infty$ PR0OF Put $t=f_{\Omega}\ f\ d\mu.$ Then $a<t<b.$ If $\beta$ is the supremum of the quotients on the left of 3.1(2), where $a<s<t,$ then $\boldsymbol{\beta}$ is no larger than any of the quotients on the right of 3.1(2), for any $u\in(t,b).$ It follows that $$ \varphi(s)\geq\varphi(t)+\beta(s-t)\qquad(a<s<b). $$ (2) Hence $$ \varphi(f(x))-\varphi(t)-\beta(f(x)-t)\geq0 $$ (3) for every $\mathrm{*e}\,\Omega$ Since $\boldsymbol{\varphi}$ is continuous $\varphi\circ f$ is measurable. If we integrate and the both sides of ((3) with respect to L,(1) follows from our choice of ${\mathbf{}}t$ assumption $\mu(\Omega)=1.$ // To give an example, take $\varphi(x)=e^{x}$ Then(1) becomes $$ \exp\left\{\right\}_{\Omega}f\,d\mu \}\leq \{_{\Omega}e^{f}\,d\mu. $$ (4) If Q is a finite set, consisting of points $p_{1},\cdot\cdot\cdot,\;p_{n},\mathrm{Say},$ and if $$ \mu(\{p_{i}\})=1/n,\qquad f(p_{i})=x_{i}, $$LP-SPACES 63 (4) becomes $$ \exp\left\{\frac{1}{n}\left(x_{1}+\cdots+x_{n}\right)\right\}\leq\frac{1}{n}\left(e^{x_{1}}+\cdots+e^{x_{n}}\right), $$ (5) for real $x_{i}\,.$ Putting $y_{i}=e^{n},$ we obtain the familiar inequality between the arith- metic and geometric means of n positive numbers: $$ (y_{1}y_{2}\cdot\cdot\cdot y_{n})^{1/n}\leq{\frac{1}{n}}\left(y_{1}+y_{2}+\cdot\cdot\cdot+y_{n}\right). $$ (6) Going back from this to (4), it should become clear why the left and right sides of $$ \exp\left\{\right\}_{\Omega}\log\,g\ d\mu \}\leq \{_{\Omega}g\ d\mu $$ (7) are often called the geometric and arithmetic means, respectively, of the positive function g. where E ${}_{x_{i}}=1,$ then we obtain If we take $\mu(\{p_{i}\})=\alpha_{i}>0,$ $$ y_{1}^{\alpha_{1}}y_{2}^{\alpha_{2}}\cdot\cdot\cdot\cdot y_{n}^{\alpha_{n}}\leq\alpha_{1}y_{1}+\alpha_{2}y_{2}+\cdots+\alpha_{n}y_{n} $$ (8) in place of (6). These are just a few samples of what is contained in Theorem 3.3. For a converse,see Exercise 20. 3.4 Definition If $\boldsymbol{\mathit{P}}$ p and $\boldsymbol{\mathit{q}}$ are positive real numbers such that $p+q=p q,$ or equivalently $$ {\frac{1}{p}}+{\frac{1}{q}}=1, $$ (1) then we call $\boldsymbol{\mathit{P}}$ and $\scriptstyle{\mathcal{A}}$ a pair of conjugate exponents. It is clear that (1) implies $p=q=2$ $1<p<\cdots$ and $1<q<\infty.$ An important special case is As $p\to1,(1)$ forces $q{ arrow}\circ\infty.$ Consequently l and oo are also regarded as a pair of conjugate exponents. Many analysts denote the exponent conjugate to $\boldsymbol{\mathit{P}}$ by ${\boldsymbol{p}}^{\prime},$ often without saying so explicitly. 3.5 Theorem Let p and qbe conjugate exponents, $1<p<\infty$ Let $X$ be a measure space, with measure p. Let f and $\scriptstyle{\mathcal{G}}$ be measurable functions on $X,$ with range in [O,oo]. Then $$ \bigcap_{X}f g\ d\mu\leq\left\{\bigcup_{X}f^{p}\ d\mu\right\}^{1/p}\bigcap_{X}^{g^{q}\ d\mu} \}^{1/q} $$ (1) and $$ \left\{\bigcup_{X}(f+g)^{p}\;d\mu\right\}^{1/p}\leq\left\{\left\{_{X}f^{p}\;d\mu\right\}^{1/p}+ \{\int_{X}g^{p}\;d\mu\right\}^{1/p}. $$ (2) as the Schwarz inequality The inequality (1) is HOlder's;(2) is Minkowski's. If $p=q=2,(1)$ is known64 REAL AND COMPLEX ANALYSIS PROOF Let $\scriptstyle A$ and $\boldsymbol{B}$ be the two factors on the right of (1). 1 $\scriptstyle A=0,$ then f = 0 a.e.(by Theorem 1.39); hence $\scriptstyle y_{\theta}=0$ a.e., so (1) holds. If $\scriptstyle4>0$ and $B=\infty,(1)$ is again trivial. So we need consider only the case $0<A<\infty.$ $0<B<\infty.$ Put $$ F={\frac{f}{A}},\qquad G={\frac{g}{B}}. $$ (3) This gives $$ \bigcap_{X}F^{p}~d\mu= \lceil_{X}G^{q}~d\mu=1. $$ (4) If ${\mathfrak{x}}\in X$ is such that $0<F(x)<\infty$ and $0<G(x)<\infty,$ there are real the numbers s and $\hat{\boldsymbol{L}}$ such that $F(x)=e^{s/p},$ $G(x)=e^{i q_{\bar{x}}}.$ Since $1/p+1/q=1,$ convexity of the exponential function implies that $$ e^{s/p+t/q}\leq p^{-1}e^{s}+q^{-1}e^{t}. $$ (5) It follows that $$ F(x)G(x)\leq p^{-1}F(x)^{p}+q^{-1}G(x)^{q} $$ (6) for every ${\mathfrak{x}}\in X$ Integration of (6) yields $$ \{_{x}F G\ d\mu\leq p^{-1}+q^{-1}=1, $$ (7) by (4); inserting (3) into (7), we obtain (1) Note that (6) could also have been obtained as a special case of the inequality 3.3(8). To prove (2), we write $$ (f+g)^{p}=f\cdot(f+g)^{p-1}+g\cdot(f+g)^{p-1}. $$ (8) Holder's inequality gives $$ \left\{f\cdot(f+g)^{p-1}\leq\left\{\left\{\right\}f^{p}\right\}^{1/p}\left\{ \{f+g\right\}^{(p-1)q}\right\}^{1/q}. $$ (9) Let (9') be the inequality (9) with $\left(\vartheta^{\prime}\right)$ gives and $\scriptstyle{\mathcal{G}}$ interchanged. Since $(p-1)q=p,$ $\boldsymbol{\f}$ addition of (9) and $$ \left\{\,\!\begin{array}{c}{{(f+g)^{p}\leq\left\{\,\displaystyle{\left\{\,\right\}}\left(f+g\right)^{p}\right\}^{1/q}\right\}\left\{\,\!\left\{\,\!\int\!\!\int\!\!f^{p}\!\!\int^{1/p}+ \{\,\!\begin{array}{c}{{g^{p}}}\\ {{\displaystyle>}}\end{array}\right.\right\}^{-1/p}\,\!,}}\end{array} $$ (10) than $\mathbf{0}$ Clearly, it is enough to prove (2) in the case that the left side is greater for and the right side is less than α. The convexity of the function ${\boldsymbol{t}}^{p}$ $0<t<\alpha$ shows that $$ \left(\frac{f+g}{2}\right)^{p}\leq\frac{1}{2}\left(J^{p}+g^{p}\right). $$LP-SPACEs 65 Hence the left side of (2) is less than OO, and (2) follows from (10) if we divide by the first factor on the right of (10),bearing in mind that $1-1/q=1/p.$ This completes the proof. // It is sometimes useful to know the conditions under which equality can hold in an inequality. In many cases this information may be obtained by examining the proof of the inequality For instance,equality holds in(T) if and only if equality holds in (6) for almost every x. In (5),equality holds if and only if $s=t.$ Hence $P^{p}=G^{q}$ a.e.” is a necessary and sufficient condition for equality in (7),if (4) is assumed. In terms of the original functions f and g, the following result is then obtained: Assuming A<o and $\scriptstyle B\,<\,\alpha$ equality holds in(I) if and only if there are constants α and ${\boldsymbol{\beta}},$ B, not both O, such that $\alpha f^{p}=\beta g^{q}$ a.e. We leave the analogous discussion of equality in(2) as an exercise The $L^{p}.$ -spaces In this section, $\textstyle X{\mathrm{~}}$ will be an arbitrary measure space with a positive measure p define 3.6 Definition If $0<p<\cdots$ and iff is a complex measurable function on $X.$ $$ \|f\|_{p}=\left\{\right\}_{X}|f|^{p}\;d\mu \}^{1/p} $$ (1) and let $\scriptstyle T_{(i,i)}$ consist of all f for which $$ \|f\|_{p}<\infty. $$ (2) We call If $\boldsymbol{\mu}$ is Lebesgue measure on $R^{k},$ we write $L^{p}(R^{k})$ instead of $\scriptstyle R_{00}.$ as in $\|f\|_{p}$ the $L^{p}.$ norm of f Sec. 2.21. If ${\boldsymbol{\mu}}$ is the counting measure on a set $A,$ it is customary to denote the corresponding $L^{p}.$ -space by $\ell^{p}(A),$ or simply by $\ell^{p},$ if $\scriptstyle A$ is countable. An element of $\ell^{p}$ may be regarded as a complex sequence $x=\{\xi_{n}\},$ and $$ \|x\|_{p}=\left\{\sum_{n=1}^{\infty}|\xi_{n}|^{p}\right\}^{1/p}. $$ 3.7 Definition Suppose $g\colon$ $X\to[0,\infty]$ is measurable. Let $\boldsymbol{\mathsf{S}}$ be the set of all real α such that $$ \mu(g^{-1}((\alpha,\,\infty)))=0. $$ (1) If $S=\mathbb{Q},$ , put $\beta=\infty$ If $S\neq{\mathcal{D}},\operatorname{put}\beta=\operatorname{inf}S$ i. Since $$ g^{-1}((\beta,\,\infty{1})=\operatorname*{su}_{n=1}^{\infty}\,g^{-1}\Bigl(\Bigl(\beta+{\frac{1}{n}},\,\infty\Bigr]\Bigr), $$ (2)66 REAL AND COMPLEX ANALYSIS and since the union of a countable collection of sets of measure O has measure O, we see that ${\mathfrak{g}}\in S_{*}$ We call $\boldsymbol{\beta}$ the essential supremum of g If f is a complex measurable function on X, we define $\scriptstyle T(n)$ consist of all $\boldsymbol{\mathsf{f}}$ for which essential supremum of |f|, and we let $\|f\|_{\infty}$ to be the $\|f\|_{\alpha}<\infty.$ The members of $\scriptstyle T_{(i,j)}$ are sometimes called essentially bounded measurable functions on $X.$ It follows from this definition that the inequality $|f(x)|\leq\lambda$ holds for almost all xif and only i $\lambda\geq\left\|f\right\|_{\infty}.$ As in Definition 3.6, $L^{\infty}(R^{k})$ denotes the class of all essentially bounded (with respect to Lebesgue measure) functions on $R^{k}{}_{,}$ and $\ell^{\infty}(A)$ is the class of all bounded functions on A.(Here bounded means the same as essentially bounded, since every nonempty set has positive measure!) and $\mathbf{\Omega}^{g}$ e L(p)then fg e $\scriptstyle I_{\mathrm{(gh}}$ and are conjugate exponents, $1\leq p\leq\varnothing,$ and $|f f\in B(\mu)$ 3.8 Theorem If p and $\boldsymbol{\mathit{q}}$ $$ \|f g\|_{1}\leq\|f\|_{p}\|g\|_{q}. $$ (1) PROOF For $1<p<\alpha,$ (1) is simply Holder's inequality, applied to $|f|$ and $|g|\cdot\mathbb{F}\,p=\infty.$ note that $$ |f(x)g(x)|\leq\|f\|_{\infty}\|g(x)\| $$ (2) for almost all $x\colon$ integrating (2), we obtain (1). If $p=1,$ then $q=\infty,$ and the same argument applies. // and 3.9 Theorem Suppose $1\leq p\leq\infty,$ and fe LP(p)g ∈ LE(p) Then f+ g ∈ LP(p) $$ \|f+g\|_{p}\leq\|f\|_{p}+\|g\|_{p}. $$ (1) PROOF For $1<p<\alpha,$ this follows from Minkowski's inequality, since $$ \bigcap_{X}|f+g|^{p}\,d\mu\leq\int_{X}(|f|+|g|)^{p}\,d\mu. $$ For $p=1$ or $p=\,\infty,$ (1)is a trivial consequence of the inequality $|f+g|\leq|f|+|g|$ /// 3.10 Remarks Fix $p,1\leq p\leq\infty.$ If f e $\scriptstyle R(s)$ and $\scriptstyle{\dot{\alpha}}$ is a complex number, it is clear that $x/\in D(\mu).$ in fact, $$ \left\|\alpha f\right\|_{p}=\left|\alpha\left| |f\right|\right|_{p}. $$ (1) In conjunction with Theorem 3.9, this shows that $\scriptstyle{H_{(i)}}$ is a complex vector space.LP-SPACES 67 Suppose $f,$ g, and ${\boldsymbol{h}}$ are in $\scriptstyle{H_{(a)}}$ Replacing $\boldsymbol{\mathsf{f}}$ by f- g and $\scriptstyle{\mathcal{G}}$ by $\scriptstyle{g-h}$ in Theorem 3.9, we obtain $$ \|f-h\|_{p}\leq\|f-g\|_{p}+\|g-h\|_{p}. $$ (2) This suggests that a metric may be introduced in $\scriptstyle T_{(i)}$ by defining the dis tance between $\boldsymbol{\f}$ and g to be $\|f_{\sim-\ g}\|_{p}.$ Call this distance d(f,g) for the and (2) shows moment. Then $0\leq d(f,\,g)<\infty,$ $d(f,f)=0$ $d(f,\ g)=d(g,f),$ is satisfied. The only that the triangle inequality $d(f,\;h)\leq d(f,\;g)+d(g,\;h):$ other property which d should have to define a metric space is that $d(f,g)=0$ should imply that $f=g.$ In our present situation this need not be $\scriptstyle X.$ so; we have $d(f,g)=0$ precisely when $f(x)=g(x).$ for almost al Let us write $\scriptstyle{f\times\theta}$ if and only if $d(f,~g)=0.$ It is clear that this is an equivalence relation in $\scriptstyle T_{(i)}$ which partitions $\scriptstyle T_{(i)}$ into equivalence classes; each class consists of all functions which are equivalent to a given one. If ${\mathbf{}}F$ $d(F,\,G)=d(f,\,g);$ and G are two equivalence classes,choose and $g\sim g_{1}$ implies and g∈ G,and define note that J $\gamma_{\mathrm{e}}\,F$ $\textstyle{\mathbf{\hat{e}}}\sim j_{i}$ $$ d(f,g)=d(f_{1},g_{1}), $$ so that $d(F,$ G) is well defined. With this definition, the set of equivalence classes is now a metric space. Note that it is also a vector space, since $f\sim f_{1}$ and $\scriptstyle g\sim\theta_{1}$ implies $f+g\sim$ $f_{1}+g_{1}$ and $x y\sim x y_{1}$ When $\scriptstyle{H_{(i)}}$ is regarded as a metric space, then the space which is really under consideration is therefore not a space whose elements are functions, but a space whose elements are equivalence classes of functions. For the sake of simplicity of language, it is, however, customary to relegate this distinction to order ${\boldsymbol{p}},$ sponds an integer ${\mathbf{}}N$ a Cauchy sequence in the status of a tacit understanding and to continue to speak of $P(n)$ These definitions are exactly as in any $\scriptstyle x\;{\ _{3}}$ there corre- as a say that $\{f_{n}\}$ or that $\{f_{n}\}$ is such that $\|f_{n}-f_{m}\|_{p}<\epsilon$ as soon as $n>N$ $\scriptstyle R(y)$ we If $\{f_{n}\}$ space of functions. We shall follow this custom ire ECOA adi lim $\|f_{n_{-}}-f\|_{p}=0,$ is a sequence in $\scriptstyle\eta_{0,0}$ converges to f in $L^{p}.$ -convergent to $f{\mathfrak{h}}.$ f to every $\scriptstyle{H_{(i)}}$ (or that $\{f_{n}\}$ converges to f in the mean of and $m>N,$ we call $\{f_{n}\}$ metric space. It is a very important fact that $\scriptstyle T(n)$ is a complete metric space, ie., that every Cauchy sequence in $\scriptstyle T_{(i)}$ converges to an element of $\scriptstyle{H_{(i)}}$ 3.11 Theorem LEu) is a complete metric space, for $1\leq p\leq\infty$ and for every positive measure p. PxoOF Assume first that $1\leq p<\varnothing.$ Let $\{f_{n}\}$ be a Cauchy sequence in $\scriptstyle\eta_{(i)}.$ There is a subsequence $\{f_{n}\},\,n_{1}<n_{2}<\cdots,$ such that $$ \|f_{n i+1}-f_{n}\parallel_{p}<2^{-i}\qquad(i=1,2,\,3,\,\ldots). $$ (1)$68$ REAL AND COMPLEX ANALYSIS Put $$ g_{k}=\sum_{i=1}^{k}|f_{n_{i+1}}-f_{n_{i}}|\,,\qquad g=\sum_{i=1}^{\infty}|f_{n_{i+1}}-f_{n_{i}}|\,. $$ (2) particular, $g(x)<\infty$ Since (1) holds, the Minkowski inequality shows that $\|g_{k}\|_{p}<1{\mathrm{~for~}}k=1,$ . 1n 2,3,.…Hence an application of Fatou's lemma to {gf} gives $|g|_{p}\leq1$ a., so that the series $$ f_{n_{1}}(x)+\sum_{i=1}^{\infty}(f_{n_{i+1}}(x)-f_{n_{i}}(x)) $$ (3) converges absolutely for almost every ${\mathfrak{x}}\in X;$ Denote the sum of (3) by f(x) for those x at which ((3) converges; put $f(x)=0$ on the remaining set of measure zero. Since $$ f_{n_{1}}+\sum_{i=1}^{k-1}(f_{n_{i+1}}-f_{n})=f_{n_{k}}, $$ (4) we see that $$ f(x)=\operatorname*{lim}_{i\to\infty}f_{n}(x) $$ a.e. (5) exists an ${\mathbf{}}N$ Having found a function $\boldsymbol{\mathsf{f}}$ f which is the pointwise limit a.e. of $\{f_{n}\}$ Choose $\scriptstyle\epsilon\;>0$ There we such that $\parallel_{J_{n}}-f_{m}\parallel_{P_{n}}<\epsilon$ if $\quad n\geqslant N$ and $m>N.$ For every $\{f_{n i}\},$ now have to prove that this fis the $L^{p}.$ -limit of ${\boldsymbol{m}}\geq N.$ Fatou's lemma shows therefore that $$ \left|\sum_{X}^{*}f-f_{m}\left|^{p}\,d\mu\leq\operatorname*{lim}_{i\rightarrow\infty}\operatorname*{inf}_{\lambda_{X}}\right.\int_{X}\d y_{n i}-f_{m} |^{p}\,\,d\mu\leq\epsilon^{p}. $$ (6) $\scriptstyle n\;=\;1.$ 2,3, We conclude from(6) that $f-f_{m}\in L^{p}(\mu),$ ${\boldsymbol{E}}$ be the union of these sets, for [since $f=1$ $\scriptstyle T_{(i)b}$ let $A_{k}$ .Then converges uniformly to a bounded function hence that $f\in D(\mu)$ for $x\in E$ i. Then m, In $\scriptstyle{\vec{F}}(t_{i j})$ and $\mu(E)=0,$ and on the complement of $\boldsymbol{E}$ This completes the $k_{\mathrm{{,}}}$ $\{f_{n}\}$ $(f-f_{m})+f_{m}],$ and finally that $\|f-f_{m}\|_{p}\to0$ as $m\to\infty,$ and where proof for the case $1\leq p<\varnothing.$ be the sets where $|f_{k}(x)|>||f_{k}\,||_{\infty}$ the proof is much easier. Suppose $\langle J_{\mathrm{sl}}\rangle$ is a Cauchy sequence in $B_{m,n}$ and let $f_{n}(x)-f_{m}(x)|>\|\,f_{n}-f_{m}\|_{\alpha}\,,$ the sequence ${\boldsymbol{f}}.$ Define $f(x)=0$ f ∈ $L^{\infty}(\mu),$ and $\|f_{n}-f\|_{\alpha}\to0$ as $n\to\varnothing.$ ${\it j}/j{\it j}$ The preceding proof contains a result which is interesting enough to be stated separately: limit 3.12 Theorem $I\ f\ 1\leq p\leq\infty$ and i $\{f_{n}\}$ is a Cauchy sequence in L(p),with $f,$ then $\{f_{n}\}$ has a subsequence wich coverges pointwise almost ever- where to f(x)IP-SPACES 69 The simple functions play an interesting role in $\scriptstyle{H_{(i)}}$ 3.13 Theorem Let $\boldsymbol{\mathsf{S}}$ be the class of all complex, measurable, simple functions on $X$ Y such that $$ \mu(\{x;s(x)\neq0\})<\infty. $$ (1) $I f\vdash\leq p<\infty,t h e n\,S$ is dense in LP(). PROOF First, it is clear that $S\subset D(\mu).$ Suppose $f\geq0,f\in D(\mu),$ and let $\{s_{n}\}$ be as in Theorem 1.17. Since $0\leq s_{n}\leq f,$ we have $s_{n}\in D(\mu),$ hence $s_{n}\in S.$ Since $|f-s_{n}|^{p}\leq f^{p},$ the dominated convergence theorem ${\boldsymbol{S}}.$ The general case that shows $\|f-s_{n}\|_{p}\to0$ as n→OO. Thus $\boldsymbol{\f}$ is in the $L^{p}.$ -closure of // (f complex) follows from this. Approximation by Continuous Functions So far we have considered $\scriptstyle{H_{(i)}}$ on any measure space. Now let $X$ be a locall compact Hausdorff space, and let ${\boldsymbol{\mu}}$ be a measure on a o-algebra Dl in might be $R^{k},$ and ${\boldsymbol{\mu}}$ might $X,$ with the properties stated in Theorem 2.14. For example, $X$ be Lebesgue measure on $R^{k}.$ Under these circumstances, we have the following analogue of Theorem 3.13: 3.14 Theorem For 1≤p<00, $C_{c}(X)$ is dense in $\scriptstyle T_{(i)}$ PRoOF Define S as in Theorem 3.13. If s e $\boldsymbol{\mathsf{S}}$ and $\scriptstyle\epsilon\;>0,$ there exists a g e $C_{c}(X)$ such that $g(x)=s(x)$ except on a set of measure <e, and $|g|\leq\|s\|$ c (Lusin's theorem). Hence $$ \left\|g-s\right\|_{p}\leq2\epsilon^{1/p}\|s\|_{\ldots}. $$ (1) Since $\boldsymbol{\mathsf{S}}$ is dense in $\scriptstyle P(n){\mathrm{,~}}$ this completes the proof. // 3.15 Remarks Let us discuss the relations between the spaces $L^{p}(R_{.}^{k})$ (the $L^{p}.$ spaces in which the underlying measure is Lebesgue measure on $R^{k}{}_{j}$ and the space $C_{c}(R^{k})$ in some detail. We consider a fixed dimension $C_{c}(R^{k});$ the distance between $\boldsymbol{\mathit{f}}$ $\boldsymbol{k}$ For every pe[1, oo] we have a metric on $\scriptstyle{\mathcal{G}}$ $\|f-g\|_{p}.$ Note that this is a genuine metric, and that we do not and g is have to pass to equivalence classes. The point is that if two continuous func- tions on $R^{k}$ are not identical, then they differ on some nonempty open set V, and $m(V)>0,$ since ${\mathbf{}}V$ V contains a k-cell. Thus if two members of $C_{c}(R_{s}^{k})$ are equal a.e., they are equal. It is also of interest to note that in $C_{c}(R^{k})$ the essential supremum is the same as the actual supremum: $\mathrm{for}\,f\in C_{c}(R^{n})$ $$ \|f\|_{\infty}=\operatorname*{sup}_{x\in R^{k}}|f(x)|. $$ (1)$70$ REAL AND COMPLEX ANALYSI If $1\leq p<\infty,$ Theorem 3.14 says that $C_{c}(R^{k})$ is dense in IR(R*),and Theorem 3.11 shows that $L^{p}(R^{k})$ is complete. Thus ${\cal D}(R^{k})$ is the completion of the metric space which is obtained by endowing C.(R*) with the LEP-metric The cases $\scriptstyle{p=1}$ and $\scriptstyle{p=2}$ are the ones of greatest interest. Let us state $k=1;$ once more, in different words, what the preceding result says if $\scriptstyle{p=1}$ and the statement shows that the Lebesgue integral is inded the “right generalization of the Riemann integral: supports in $R^{1},$ If the distance between two continuous functions f and ${\mathfrak{g}},$ with compact is defined to be $$ \vert\sum_{-\infty}^{\infty}\vert f(t)-g(t)\vert\ d t, $$ (2) the completion of the resulting metric space consists precisely of the Lebesgue integrable functions on $R^{1},$ provided we identify any two that are equal almost everywhere. Of course, every metric space $\boldsymbol{\mathsf{S}}$ has a completion S $S^{\bullet}$ whose elements may be viewed abstractly as equivalence classes of Cauchy sequences in $\boldsymbol{\mathsf{S}}$ (see「26 p. 82). The important point in the present situation is that the various ${\boldsymbol{D}}_{-}$ completions of C,(R') again turn out to be spaces of functions on $R^{k},$ not $L^{\infty}(R^{k}),$ but is $\scriptstyle C e l R^{\dagger}$ the space of all continuous functions on ${\boldsymbol{R}}^{k}$ -completion of C,(R') i The case $p=\alpha$ differs from the cases $p<\,\infty.$ The $L^{\infty}.$ which“vanish at infnity,” a concept which will be defined in Sec. 3.16. Since(1) shows that the $L^{\infty}.$ -norm coincides with the supremum norm on C,(R),the above assertion about $C_{0}(R^{k})$ is a special case of Theorem 3.17. 3.16 Definition A complex function $\boldsymbol{\mathsf{f}}$ on a locally compact Hausdorff space $X$ is said to vanish at infinity if to every $\scriptstyle\epsilon\;>0$ there exists a compact set $\kappa\in X$ such that |f(x)|<efor all $\scriptstyle{\mathcal{X}}$ not in $K.$ The class of all continuous $\boldsymbol{\mathsf{f}}$ on $X$ which vanish at infinity is called $c_{d}(X)$ It is clear that $C_{c}(X)\subset C_{0}(X),$ and that the two classes coincide if $X$ is compact. In that case we write C(X) for either of them. 3.17 Theorem If $X$ is a locally compact Hausdorff space,then $\scriptstyle c_{a}(X)$ is the completion of C.(X), relative to the metric defined by the supremum norm $$ \|f\|=\operatorname*{sup}_{x\in X}|f(x)|\,. $$ (1) space. PxoOF An elementary verification shows that is dense in $\scriptstyle c_{d}|X|$ and (b) $c_{a}(X)$ is a complete metric We have metric space if the distance between $\boldsymbol{\mathsf{f}}$ and $\scriptstyle C_{d}|X\rangle$ satisfies the axioms of a $\scriptstyle{\mathcal{G}}$ is taken to be $\|f-g\|.$ to show that (a $\scriptstyle C_{i}(X)$LP-SPACES 71 outside K. Given fe C,(X) and e> 0,there is a compact set gives us a function $g\in C_{c}(X)$ such that Urysohn's lemma $\textstyle K$ so that $|f(x)|<\epsilon$ $0\leq g\leq1$ and $g(x)=1$ on $K.$ Put $\hbar=j g,$ Then $h\in C_{c}(X)$ and $\|f-h\|<\epsilon.$ This proves (a) set Given $\mathbf{e\!\,\!>\!0}.$ To prove (b), let $\{f_{n}\}$ be a Cauchy sequence in $c_{d}(X)$ i.e., assume that and we // $\{f_{n}\}$ ${\cal K}\,\,$ so that there exists an $r_{\mathit{l}}$ converges uniformly. Then its pointwise limit function $\|f_{n}-f\|<\epsilon/2$ and there is a compact $K,$ $\boldsymbol{\f}$ is continuous. so that $|f_{n}(x)|<\epsilon/2$ outside K. Hence $|f(x)|<\epsilon$ outside have proved that f vanishes at infinity. Thus $\scriptstyle C_{d}(X)$ is complete. Exercises 1 Prove that the supremum of any collection of convex functions on (a, b) is convex on (a,b) (if it is finite) and that pointwise limits of sequences of convex functions are convex. What can you say about upper and lower limits o sequences of convex functions? 2 If $\textstyle\mathcal{\boldsymbol{\varphi}}$ is convex on $(a,\,b)$ and if $\psi$ is convex and nondecreasing on the range of $\mathcal{\varphi}$ implies the convexity of ${\mathcal{O}}_{3}$ prove that $\psi\circ\varphi$ is convex on (a,b). For $\varphi>0,$ show that the convexity of log $\varphi_{\mathrm{i}}$ but not vice versa 3 Assume that $\mathcal{\varphi}$ is a continuous real function on (a, ${\mathfrak{b}}\}$ ) such that $$ \varphi\!\left({\frac{x+y}{2}}\right)\leq{\frac{1}{2}}\;\varphi(x)+{\frac{1}{2}}\;\varphi(y) $$ for al $\scriptstyle{\dot{\boldsymbol{x}}}$ and ye(α,b). Prove that $\textstyle{\varphi}$ is convex.(The conclusion does not follow if continuity is omitted from the hypotheses.) 4 Suppose fis a complex measurable function on $X,\,I\qquad X,\,I$ u is a positive measure on $X,$ and $$ \varphi(p)=\int_{X}|f|^{p}\;d\mu=\|f\|_{p}^{p}\qquad(0<p<\infty). $$ Let $E=\{p;$ $\varphi(p)<\infty\}.$ Assume $\|f\|_{\infty}>0.$ prove that $p\in E$ E and that $\textstyle\mathcal{\boldsymbol{\varphi}}$ is continuous on ${\boldsymbol{E}}.$ $\scriptstyle{\vec{E}}$ (a) If $\textstyle<p<S,r\in i$ E, and $s\in E,$ $\boldsymbol{E}$ 9 $\|f\|_{s}).$ Show that this implies the inclusion (d) If (6 Prove that log $\textstyle{\varphi}$ is convex in the interior of $(\|f\|_{r}$ (c)By $r<p<S,$ prove that $\|f\|_{p}\leq$ max nessarily open? Closedr Can E consist of a single point? Can $(a),\,E$ is connected. Is $\bar{E}$ be any connected subset of (0, co)? $L(\mu)\cap E(\mu)\subset D(\mu).$ < oo for some $r<\alpha_{0}$ and prove that (e) Assume that $\|f\|_{r}$ $$ \|f\|_{p}\to\|f\|_{\alpha}\qquad{\mathrm{as~}}p\to\varnothing. $$ S Assume,in addition to the hypotheses of Exercise $4_{\mathrm{,}}$ that $$ \mu(X)=1. $$ (a) Prove that $\|f\|_{r}\leq\|f\|_{s}{\mathrm{if}}\ 0<r<s\leq\infty$ 0 and $\|f\|_{r}=\|f\|_{s}<\infty\,?$ (c) Prove that b Under what conditions does it happen that ) if $\textstyle0<r<s.$ Under what conditions do these two spaces contain the $0<r<s\leq\infty$ $T(\mu)\Rightarrow E(\mu)$ same functions? (d) Assume that $\|f\|_{r}<\infty$ for some $r>0,$ and prove that $$ \operatorname*{lim}_{p\to0}\|f\|_{p}=\exp\left\{\left\{\right\}_{X}\log\mid f\mid\,d\mu\right\} $$ if exp $\scriptstyle-\,\omega_{1}$ is defined to be O.$\overline{{\nu2}}$ REAL AND COMPLEX ANALYSIs 6 Let ${\mathfrak{m}}\,$ be Lebesgue measure on [0,1], and define $\|f\|_{p}$ with respect to ${\mathfrak{m}}.$ Find all functions $\ \Phi$ on [O, co) such that the relation $$ \Phi(\operatorname*{lim}_{p\to0}\|f\|_{p})= |\bigcup_{0}^{1}(\Phi\circ f)\ d m $$ holds for every bounded, measurable, positve f Show frst that $$ c\Phi(x)+(1-c)\Phi(1)=\Phi(x^{c})\qquad(x>0,\,0\leq c\leq1). $$ Compare with Exercise S(d) there are some for which $T(\mu)$ does not contain ${\mathcal{L}}(\mu)$ $L(\mu)<L(\mu);$ for others,th inclusion is reversed; and T For some measures, the relation $r<s$ implies if r ≠ s. Give examples of these situations, and find conditions on $\mathcal{J}$ under which these situations will occur (0, 8 If g is a positive function on $h\leq g$ and $h(x)\to\infty$ as x→0. True or false? Is the problem changed if(O,1) is $(0,\,1)$ such tha $\operatorname{tr}\,g(x)\to\infty$ as $x arrow0,$ then there is a convex function h on ${\mathfrak{I}})$ such that replaced by (Q,00) and $x\to0$ is replaced by $x\to\infty^{\circ}$ P→0o, but $\|f\|_{p}\leq\Phi(p)$ for all suffciently large p? is Lebesgue measurable on(0,1), and not essentially bounded. By Exercise 4(e) $\|f\|_{p}\to$ o as 9 Suppose $\scriptstyle{\vec{f}}$ CanIJl, tend to co arbitrarily slowly? More precisely, is t true that to every $\|f\|_{p}\to\alpha$ p as $p\to\mathbb{Q}.$ on (0, co such that $\Phi(p)\to\infty$ as $p\to$ co one can find an f such that positive function $\Phi$ 10 Suppose $f_{n}\in D(\mu),$ for $n=1,$ 2,3, .…,and l。一JlI,→0 and f,→g a.e., as n→0o. What relation exists between f and $g^{\mathcal{V}}$ 11 Suppose $\mu(\Omega)=1,$ and suppose f and g are positive measurable functions on $\underline{{\Omega}}$ such $\operatorname{that}f g\geq1.$ Prove that $$ \bigcap_{\Omega}f\,d\mu\cdot\bigcup_{\Omega}^{}g\,d\mu\geq1. $$ 12 Suppoe $\mu(\Omega)=1$ and $h\colon\Omega\to[0,\infty]$ is measurable.I1 $$ A= \vert_{\Omega}^{\circ}h\,d\mu, $$ prove tha $$ \sqrt{1+A^{2}}\leq\int_{\Omega}\sqrt{1+h^{2}}\;d\mu\leq1+A. $$ If pis Lebesgue measure on [0, 1] and if ${\boldsymbol{h}}$ is continuous, $h=f^{\prime},$ the above inequalities have a simple equality geometric interpretation. From this, conjecture (for general $\Omega\mathrm{)}$ under what conditions on ${\boldsymbol{h}}$ can hold in either of the above inequalities, and prove your conjecture 13 Under what conditions on f and $\scriptstyle{\mathcal{G}}$ does equality hold in the conclusions of Theorems $3.8$ and $3.9\gamma$ You may have to treat the cases $p=1$ and $p=\infty$ separately 14 Suppose $1<p<\infty,f\in L^{p}=L^{p}((0,\infty)),$ relative to Lebesgue measure, and $$ F(x)={\frac{1}{x}}\int_{0}^{x}\!f(t)\,d t\qquad(0<x<\infty). $$ (a) Prove Hardy's inequality $$ \|F\|_{p}\leq{\frac{p}{p-1}}\parallel f\|_{p} $$ which shows that the mappingf→ F carries ${\boldsymbol{D}}$ into $\bar{\cal M}$LP-SPACEs 73 (b) Prove that equality holds only ${\mathrm{iff}}=0$ a.e (c) Prove that the constant $p/(p-1)$ cannot be replaced by a smaller one. (d) ${\mathrm{If}}f>0{\mathrm{~and}}f\in L^{1},$ prove that $\scriptstyle r:l$ . Integration by parts gives Suggestions: (a) Assume first thatf≥0and fe $C_{c}((0,\infty))$ $$ \bigcap_{0}^{\infty}F^{p}(x)\;d x=-p\;\prod_{0}^{\infty}F^{p-1}(x)x F^{\prime}(x)\;d x. $$ Note that $x F^{\prime}\ {=}f-F,$ and apply H6lder's inequality to $\{\,F^{p-1}f.$ Then derive the general case.Ce) $\operatorname{Take}f(x)=x^{-1/p}$ on [1, A],Jf(x) = 0 elsewhere, for large ${\boldsymbol{A}}.$ See also Exercise 14, Chap. 8 15 Suppos $\left\{Q_{m}\right\}$ is a sequence of positive numbers. Prove that $$ \sum_{N=1}^{\infty}\,\left({\frac{1}{N}}\,\sum_{n=1}^{N}\,a_{n}\right)^{p}\leq\left({\frac{p}{p-1}}\right)^{p}\,\sum_{n=1}^{\infty}\,a_{n}^{p} $$ f $\mathbb{!}\times p<\infty$ Hint: If $n_{n}\geq a_{n+1},$ the result can be made to follow from Exercise 14. This special case implies the general one 16 Prove Egorofr's theorem: If $\mu(X)<\infty,$ if $\{f_{n}\}$ is a sequence of complex measurable functions , with which converges pointwise at every point of $X,$ and if $\epsilon>0,$ there is a measurable set $E\subset X_{!}$ $\mu(X-E)<\epsilon,$ such that $\{f_{n}\}$ converges uniformly on ${\boldsymbol{E}}.$ (The conclusion is that by redefining $\operatorname{the}f_{n}$ on $\hat{\boldsymbol{A}}$ set of arbitrarily small measure we can convert a pointwise convergent sequence to a uniformly convergent one; note the similarity with Lusin's theorem.) Hint: Put $$ S(n,k)=\bigcap_{i,j>n}\left\{x:|f_{i}(x)-f_{j}(x)|<{\frac{1}{k}}\right\}, $$ show that $\mu(S(n,\ k))\to\mu(X)$ as $n arrow\infty,$ for each $k_{\mathrm{{s}}}$ and hence that there is a suitably increasing sequence $\{n_{k}\}$ such that Show that the theorem does not extend to ${\sigma}.$ k has the desired property. $E=\bigcap$ $S(n_{k},$ fnite spaces. Show that the theorem does extend,with essetially he same proof, to the situation in which the sequence now that,for all x e $\scriptstyle{\mathcal{X}}$ is replaced by a family $\{f_{t}\},$ where t ranges over the positive reals; the assumptions are $\{f_{n}\}$ (ü) lin ${\textrm{n}}f_{t}(x)=f(x)$ and Gi $t\to f_{i}(x)$ is continuous. 17 0 $\operatorname{lf}0<p<\infty,\operatorname{put}\gamma_{p}=\operatorname*{max}\left(1,2^{p-1}\right),$ and show that $$ |\alpha-\beta|^{p}\leq\gamma_{p}(|\alpha|^{p}+|\beta|^{p}) $$ sketched below. for arbitrary complex numbers α and ${\boldsymbol{\beta}}.$ $,_{.}^{\bf0}\ll p<\alpha_{.}^{\bf0},f\in D(\mu),f_{.n}^{\bf}\in D(\mu),f_{n}^{\bf}(x) arrow f(x)\mathrm{~a.}.$ and (b) Suppose as $n\to\varnothing.$ Show that then lim $\|f-f_{s}\|_{p}=0,$ by completing the two proofs that are ${}_{\mu}$ is a positive measure on $X,$ $\left\|\int_{m}\left\|_{P_{-}}\right.\right\|_{\frac{}{\epsilon}} \|_{\frac{}{}} \|_{\frac{}{}}$ (ü) By Egoroff's theorem, $X=A\cup B$ in such a way that $\textstyle{\int}_{A}\vert f\vert^{p}<\epsilon,~\mu(B)<\infty,$ and $f_{n}\to f$ uniformly on B. Fatou's lemma, applied to JgIJ,IP,leads to $$ |\mathrm{liD~Sup} |\bigcup_{A}|\int_{H}|^{p}\,d\mu\leq\epsilon. $$ 1.34. $h_{n}=\gamma_{p}(|f|^{p}+|f_{n}|^{p})-|f-f_{n}|^{p},$ and use Fatou's lemma as in the proof of Theorem is omitted, even if (i) Put (c) Show that the conclusion of ${\mathfrak{(}}b{\mathfrak{)}}$ is false if the hypothesis $\left\|{\mathcal{f}}_{n}\right\|_{p}\to\left\|{\mathcal{f}}\right\|_{p}$ $\mu(X)<\infty.$74 REAL AND CoMPLEX ANALYSIs 18 Let ${}_{\mu}$ to comverge in measure to the measurable function fif to every $\epsilon>0$ of complex measurable functions on $\scriptstyle{\mathcal{X}}$ is said be a positive measure on $X.$ A sequence $\{f_{n}\}$ there corresponds an ${\mathbf{}}N$ such that $$ \mu(\{x;|f_{n}(x)-f(x)|>\epsilon))<\epsilon $$ for al $n>N.$ (This notion is of importance in probability theoryJ Assume $\mu(X)<\infty$ and prove the following statements: (a) $\operatorname{If}f_{x}(x)\to f(x)$ a.e., then $f_{n}\to f$ in measure $1\leq p\leq\varnothing.$ (6) $\operatorname{If}f_{n}\in D(\mu)$ and $\|f_{n}-f\|_{p} arrow_{\sim}0,$ $\{f_{n}\}$ has a subsequence which converges to f a.e thenf,→f in measure; here (c) $\operatorname{If}f_{n}\to f$ in measure, then instance, if p is Lebesgue measure on Investigate the converses of (a) and (b) What happens to (a),b),and (c) if $\mu(X)=\infty,$ ,for $R^{1\,\gamma}$ numbers w such that 19 Define the essential range of a function $f\in L^{\alpha}(\mu)$ to be the set ${\boldsymbol{R}}_{f}$ consisting of all complex $$ \mu(\{x z\mid f(x)-w\mid<\epsilon\})>0 $$ for every $\epsilon>0.$ Prove that ${\boldsymbol{R}}_{f}$ is compact. What relation exists between the set ${\boldsymbol{R}}_{f}$ and the number $\|f\|_{\alpha_{-}^{*}}$ Let $A_{f}$ be the setofal averages $$ {\frac{1}{\mu(E)}}\prod_{E}^{\circ}f\,d\mu $$ where $E\in\mathfrak{M}$ and $\mu(E)>0.$ What relations exist between $\mathbf{y}f\in L^{\infty}(\mu)Y$ Are there measures ${}_{\mu}$ u such that $A_{f}$ fails to be measures ${}^{\mu}$ such that $A_{f}$ is convex for ever $A_{f}$ and $R_{f}Y$ Is $A_{f}$ always closed? Are there convex for some fe $L^{\infty}(\mu)^{\gamma}$ $L^{\infty}(\mu)$ is replaced by L(p), for instance? How are these results affected if 20 Suppose $\textstyle\varnothing\quad$ is a real function on $R^{1}$ such that $$ \varphi{\Biggl(}\prod_{0}^{1}f(x)\,\,d x\Biggr)\leq\int_{0}^{t}\varphi(f)\,\,d x $$ for every real bounded measurable f. Prove that $\varphi$ is then convex. 21 Call a metric space Ya completion of a metric space X $\scriptstyle{X}$ if $\scriptstyle{\mathcal{X}}$ X is dense in ${\mathbf{}}Y$ and ${}^{Y}$ is complete. In Sec. 3.15 reference was made to“the" completion of a metric space. State and prove a uniqueness theorem which justifies this terminology. 22 Suppose $\scriptstyle{\mathcal{X}}$ is a metric space in which every Cauchy sequence has a convergent subsequence. Does it follow that $\scriptstyle{\mathcal{X}}$ is complete? (See the proof of Theorem 3.11.) $23$ Ssuppose ${}_{\!\mu}$ is a positive measure on X $Y,\,\mu(X)<\alpha o,f\in L^{\infty}(\mu),\;\|f\|_{\alpha}>0,\mathrm{an}$ nd $$ \alpha_{n}= (\frac{}{}_{x}\vert f\vert^{n}\,d\mu\qquad(n=1,\,2,\,3,\,\ldots).\qquad\qquad\qquad\qquad(n=1,\,2,\,3,\,\ldots). $$ Prove that $$ \left|\operatorname*{lim}_{n\rightarrow\infty}\frac{\tilde{G}_{n+1}}{\tilde{\alpha}_{n}}=\left\|\int\right\|_{\infty}. $$ 24 Suppose $\mathcal{H}$ is a positive measure,f∈ $D(\mu),\,g\in D(\mu).$ (a) If $0<p<1,$ prove tha $$ \left|\mid\mid f\mid^{p}-\mid g\mid^{p}\mid d\mu\leq\right|^{*}\mid f-g\mid^{p}d\mu $$ that $\Delta(f,g)=\int\mid f-g\mid^{p}$ dp defines a metric on ${\mathcal{D}}(\mu),$ and that the resulting metric space is completeLP-SPACES 75 (b) If1 $\leq p<\alpha.$ and $\|f\|_{p}\leq R,\|g\|_{p}\leq R,1$ prove that $$ \bigcap|\bigstar\left|p\right|^{p}-|g|^{p}\mid d\mu\leq2p R^{p-1}\|f-g\|_{p}. $$ Hint: Prove first, for $x\geq0,y\geq0,$ that $$ |x^{p}-y^{p}|\leq{\sqrt{|x-y|^{p}}}\qquad{{\mathrm{if~}}0<p<1,} $$ Note that (a) and ${\mathfrak{(}}b{\mathfrak{)}}$ estabish he continuit of themapping $f\to|f|^{p}$ that carries ${\mathcal{D}}(\mu)$ into $L^{1}(\mu).$ 25 Suppose ${}_{\!\mu}$ is a positive measure on $\scriptstyle{\mathcal{X}}$ and $f\colon X\to(0,\,\infty)$ satisfies $\textstyle{\int_{X}f\,d\mu=1.}$ Prove, for every $E\subset X$ with $0<\mu(E)<\infty,$ that $$ \left\vert\L_{E}^{(\log f)\,d\mu\leq\mu(E)\,\log\frac{1}{\mu(E)}}\right\vert $$ and, when $0<p<1,$ $$ \bigcap_{E}f^{p}\,d\mu\leq\mu(E)^{1-p}. $$ 26 Ifis a positive measurable function on [O, 1], which is larger, $$ \left|\bigcup_{0}^{n}f(x)\log\int(x)\ d x\qquad{\mathrm{of}}\quad\bigcup_{\vartheta}^{n}\operatorname{cr}\quad\bigcup_{\vartheta}^{n}\operatorname{cl}\bigcup_{\vartheta}\\:{}^{\bullet_{1}}(t)\,d t\right| $$