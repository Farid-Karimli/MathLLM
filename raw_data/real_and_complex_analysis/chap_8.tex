CHAPTER EIGHT INTEGRATION ON PRODUCTSPACES This chapter is devoted to the proof and discussion of the theorem of Fubini concerning integration of functions of two variables. We first present the theorem in its abstract form. Measurability on Cartesian Products follows that set of all ordered pairs (x, y), with ${\mathfrak{x}}\in X$ Y are two sets, their cartesian product $x\times Y$ is the 8.1 Definitions If $\textstyle X$ X and ${\mathbf{}}Y$ $4\times B\subset X\times Y.$ and $\gamma\epsilon_{\mathrm{\tiny~Y}}$ If ${\boldsymbol{4}}\in{\mathcal{X}}$ and $B\prec Y,{\mathrm{id}}$ We call any set of the form $\scriptstyle4\times B$ a rectangle in $x\times Y$ Suppose now that $\langle x,y\rangle$ and $(Y,{\mathcal{F}})$ are measurable spaces. Recall that this simply means that ${\mathcal{F}}$ is a o-algebra in $X$ and ${\mathcal{T}}$ is a o-algebra in ${\boldsymbol{Y}}.$ A measurable rectangle is any set of the form $4\times B_{1}$ where A ∈ ${\mathcal{G}}$ and ${\mathfrak{g}}\in{\mathcal{F}}$ If $Q=R_{1}\cup\dotsb\cup R_{n};$ where each ${\boldsymbol{R}}_{i}$ is a measurable rectangle and $R_{i},\gamma$ $R_{j}=\mathcal{D}$ for $i\neq j,$ is defined to be the smallest ${\boldsymbol{\sigma}}\cdot$ the class of all elementary sets which contains J × we say that $(o\in{\mathcal{E}},$ -algebra in $x\times Y$ ${\mathcal{T}}$ every measurable rectangle. A monotone class D is a collection of sets with the following properties: If A,∈ D, B,e D, $A_{i}\ll A_{i+1},B_{i}\supset B_{i+1},1$ for $i=1,$ 2, 3,.…., and if $$ A=\bigcup_{i=1}^{\infty}A_{i},\qquad B=\bigcap_{i=1}^{\infty}B_{i}, $$ (1) then A ∈ D and $B\in{\mathfrak{M}}$ 160INTEGRATION ON PRODUCT SPACES 161 If $E\subset X\times Y,x\in X,y\in Y$ we define $$ E_{x}=\{y\colon(x,y)\in E\},\qquad E^{y}=\{x\colon(x,y)\in E\}. $$ (2) We call $E_{x}$ and $E^{\mathfrak{g}}$ the x-section and y-section, respectively, of $\textstyle E.$ Note that $E_{x}\in Y,B^{\prime}\in X.$ $\scriptstyle{y}$ e ${\boldsymbol{Y}}.$ 8.2 Theorem If $E\in{\mathcal{F}}\times{\mathcal{T}},$ then $E_{x}\in{\mathcal{T}}$ and $E^{\prime}\in{\mathcal{F}},$ for every $\mathrm{x\!\,}$ and PRoOF Let $\Omega$ be the class of al $E\in{\mathcal{F}}\times{\mathcal{F}}$ $\textstyle{\mathcal{T}}$ is a o-algebra, the following for every xe X. If $E=A\times B.$ then $E_{x}=B\ {\mathrm{if~}}x\in A,E_{x}=\varnothing$ such that $E_{x}\in{\mathcal{T}}$ if ${\mathfrak{x}}{\mathfrak{x}}$ Therefore every measurable rectangle belongs to Q. Since three statements are true. They prove that $\Omega$ is a o-algebra and hence tha $\Omega={\mathcal{F}}\times{\mathcal{F}}$ (a) $X\times Y\in\Omega.$ then $(E^{\circ})_{x}=(E_{x})^{\circ},$ hence $\mathbb{F}\in\Omega$ then $E_{x}=\bigcup(E_{i})_{x},$ hence $\ E\exp$ b)f $\textstyle E\in\Omega$ and $E=\bigcup E_{i},$ ie If $E_{i}\in\Omega\left(i=1,2,3,\ldots\right)$ The proof is the same for $E^{y}.$ // 8.3 Theorem ${\mathcal{I}}\times{\mathcal{I}}$ is the smallest monotone class which contains all elemen- tary sets. PROOF Let の be the smallest monotone class which contains ${\mathcal{B}}\,;$ the proof that this class exists is exactly like that of Theorem 1.10. Since ${\mathcal{I}}\times{\mathcal{I}}$ is a monotone class, we have WD $<{\mathcal{P}}\times{\mathcal{P}}.$ The identities $$ \begin{array}{l}{{(A_{1}\times B_{1})\cap(A_{2}\times B_{2})=(A_{1}\Large\sim\A_{2})\times(B_{1}\cap B_{2}),}}\\ {{(A_{1}\times B_{1})-(A_{2}\times B_{2})=[(A_{1}-A_{2})\times B_{1}]\cup[(A_{1}\cap\strut A_{2})\times(B_{1}-B_{2})]}}\end{array} $$ show that the intersection of two measurable rectangles is a measurable rec tangle and that their difference is the union of two disjoint measurable rec- tangles, hence is an elementary set. If $P\in{\mathcal{E}}$ and $\;\;G\in{\mathcal{E}},$ it follows easily that $P\cap Q\in{\mathcal{C}}$ and $P-Q\in{\mathcal{E}}.$ Since $$ P\cup Q=(P-Q)\cup Q $$ and $(P-Q)\cap Q=\emptyset,$ we also have $P\cup Q\in{\mathcal{E}}.$ such that For any set $P\subset X\times Y.$ define $\scriptstyle\Omega(p)$ to be the class of all $Q\subset X\times Y$ obvious: $P-Q\in\Re,\;Q-P\in\Re,$ and $P\cup Q\in\mathfrak{M}.$ The following properties are (a) $Q\in\Omega(P)$ if and only if P e Q(Q) (b)Since OR is a monotone class, so is each Q(P)162 REAL AND coMPLEX ANALYSIs Fix P∈ 8.Our preceding remarks about $\textstyle{\mathcal{E}}$ show that Q∈ Q(P) for all Q ∈8,hence ${\mathcal{E}}\subset\Omega(P),$ and now (b) implies that OR c Q(P). if $P\in{\mathcal{E}}.$ By (a), P∈ Q(Q), Next, fix Q e OR. We just saw that $Q\in\Omega(P)$ hence Summing up: $\scriptstyle{W P}$ and if we refer to (b) once more we obtain c .Q(Q). ∈ D and $P\cup Q$ e D ${\mathcal{E}}\subset\Omega(Q),$ and Q e , then $\scriptstyle{P-Q}$ It now follows that OR is a o-algebra in $x\times Y$ (i) $X\times Y\in{\mathcal{C}}.$ Hence $X\times Y\in\Re.$ (i) If ${\boldsymbol{Q}}$ ∈Dt, then $Q^{\mathrm{c}}$ ∈领, since the difference of any two members of OR is in (ii If $\scriptstyle p_{i}\,\epsilon$ oR for i= 1,2,3, ..., and $P=\bigcup P_{i},\operatorname{pat}$ $$ Q_{n}=P_{1}\cup\cdots\circ\cup P_{n}. $$ ${\mathbf{}}P$ e D $Q_{n}\in Q_{n+1}$ and Since DR is closed under the formation of finite unions, $Q_{n}$ e 领 Since $P=\bigcup Q_{n}.$ the monotonicity of 领 shows that Thus OR is a o-algebra_ $\vartheta\subset\vartheta\Omega\subset\mathcal{P}\times\mathcal{P},$ and (by definition) ${\mathcal{I}}\times{\mathcal{I}}$ is the smallest c-algebra which contains ${\mathcal{\vartheta}}\,.$ Hence O $={\mathcal{P}}\times{\mathcal{F}}.$ // 8.4 Definition With each function $\boldsymbol{\f}$ on $x\times Y$ and with each ${\mathfrak{x}}\in X$ we Similarly, if associate a function f, defined on Y ${\mathfrak{b y}}f_{x}(y)=f(x,$ y) $X$ $y f^{y}(x)=f(x,$ y) we $y\in Y,f$ is the function defined on ${\mathcal{I}},{\mathcal{I}}.$ and ${\mathcal{I}}\times{\mathcal{I}},$ Since we are now dealing with three o-algebras, shall, for the sake of clarity, indicate in the sequel to which of these three o-algebras the word“measurable”refers. 8.5 Theorem Let f be an (JP × 9 )-measurable function on $x\times Y$ Then (a)For each xe $x,t_{x}$ is a ${\mathcal{F}}.$ T-measurable function. (b)For each $y\in Y,f^{\gamma}$ is an ${\mathcal{F}}.$ measurable function PRooF For any open set ${\mathit{V}},$ put $$ Q=\{(x,y)\colon f(x,y)\in V\}. $$ Then $Q\in{\mathcal{F}}\times{\mathcal{F}},$ and $$ Q_{x}=\{y\colon f_{x}(y)\in V\}. $$ similar Theorem 8.2 shows that $Q_{x}\in{\mathcal{F}}$ This proves (a);the proof of(b)is //INTEGRATION ON PRODUcr SPACEs 163 Product Measures 8.6 Theorem Let (X, 9, $\mu )\,$ and $(Y,{\mathcal{F}},\lambda)$ be o-finite measure spaces. Suppose $Q\in{\mathcal{F}}\times{\mathcal{F}}.I f$ $$ \varphi(x)=\lambda(Q_{x}),\qquad\psi(y)=\mu(Q^{y}) $$ (1) for every ${\mathfrak{r}}\in X$ and y e Y,then gp is ${\mathcal{F}}.$ measurable, $\psi$ is J-measurable, and $$ \bigcap_{x}\varphi\;d\mu= \Vert_{Y}\cup\;d\lambda, $$ (2) are positive measures on ${\mathcal{P}}$ P and ${\mathcal{T}}.$ Notes: The assumptions on the measure spaces are, more explicitly, that $\textstyle X$ is the union of countably $\boldsymbol{\mu}$ and 入 ", respectively, that many disjoint sets $X_{n}$ with $\mu(X_{n})<\infty,$ and that ${\mathbf{}}Y$ is the union of countably many disjoint sets ${\mathit{Y}}_{m}$ , with $\lambda(Y_{m})<\infty.$ Theorem 8.2 shows that the definitions (1) make sense. Since $$ \lambda(Q_{x})=\int_{Y}\chi_{Q}(x,y)\;d\lambda(y)\qquad(x\in X), $$ (3) with a similar statement for $\mu(Q^{y}),$ the conclusion (2) can be written in the form $$ \left[\begin{array}{c}{{\star}}\\ {{x}}\end{array}\right]_{X}^{\prime}{x_{Q}(x,\,y)\ d{\lambda}(y)}= .\bigwedge_{X}^{}{\lambda}(y)\ \\ {{\sum_{X}^{}}^{\prime}Q^{(x,\,y)\ d{\mu}(x)}. $$ (4) PR0OF Let Q be the class of al $Q\in{\mathcal{F}}\times{\mathcal{F}}$ for which the conclusion of the theorem holds. We claim that $\Omega$ has the following four properties: ie)If $|\mathbf{q}_{\mathrm{i}}$ (a)Every measurable rectangle belongs to if each $\scriptstyle0,\,\in\Omega$ , and if $Q=\bigcup$ Q,then $\sigma\operatorname{ta}$ and f $\Omega.$ (b)If $Q_{1}\subset Q_{2}\subset Q_{3}\subset\cdots,$ is a disjoint countable collection of members of $\Omega,$ i $Q=\bigcup Q_{i},$ then $\sigma\notin\Omega$ if A × B5 Q1= Q:= Q, =…., if $Q=\bigcap Q_{i}$ $\mu(A)<\infty$ and $\lambda(B)<\infty,$ and $\scriptstyle0,\,\in\Omega$ for i= 1, 2, 3,..., then Q e Q If $\ \mathbf{f}\,Q=A\,\times B,$ where A ∈ J, B∈ J,then $$ \lambda(Q_{x})=\lambda(B)\chi_{A}(x)\quad{\mathrm{and}}\quad\mu(Q^{y})=\mu(A)\chi_{B}(y) $$ (5) and therefore each of the integrals in (2) is equal to $\mu(A)\lambda(B)$ This gives (a_ To prove (b), let $\varphi_{i}$ and $\psi_{i}$ be associated with $Q_{i}$ in the way in which(1) associates p and $\psi$ with ${\cal Q},$ The countable additivity of ${\boldsymbol{\mu}}$ and $\lambda$ shows that $$ \varphi_{i}(x)\to\varphi(x),\qquad\psi_{i}(y)\to\psi(y)\qquad(i\to\infty), $$ (6) the convergence being monotone increasing at every point. Since $\varphi_{i}$ and $\psi_{i}$ are assumed to satisfy the conclusion of the theorem,(b)) follows from the monotone convergence theorem.164 REAL AND COMPLEX ANALYSIS For finite unions of disjoint sets, (c) is clear, because the characteristic function of a union of disjoint sets is the sum of their characteristic functions. The general case of (c) now follows from (b) The proof of (d) is like that of(b),except that we use the dominated convergence theorem in place of the monotone convergence theorem. This is legitimate, since $\mu(A)<\infty$ and $\lambda(B)<\infty,$ Now define $$ Q_{m n}=Q\frown(X_{n}\times Y_{m})\qquad(m,\,n=1,\,2,\,3,\,\ldots) $$ (7) ${\mathfrak{m}}\,$ and ${\mathfrak{n}}.$ and let ODR be the class of all $Q\in{\mathcal{F}}\times{\mathcal{F}}$ such that $Q_{m n}\in\Omega$ for all choices of Then((b) and (d) show that OR is a monotone class; (a) and (c) show that ${\mathcal{E}}\subset{\mathfrak{M}};$ and since $9R\subset{\mathcal{F}}\times{\mathcal{F}},$ Theorem 8.3 implies that ${\mathfrak{M}}={\mathcal{F}}\times{\mathcal{F}}.$ ${\boldsymbol{Q}}$ Thus ${\mathcal{O}}_{n n}\in\Omega$ for every $Q\in{\mathcal{F}}\times{\mathcal{F}}$ and for all choices of ${\mathfrak{m}}\,$ and ${\mathfrak{n}}.$ n, Since from (c) that is the union of the sets $Q_{m n}$ and since these sets are disjoint, we conclude ${\it j}/j{\it j}$ $\sigma\operatorname{cn}$ This completes the proof. 8.7 Definition If $(X,{\mathcal{F}},\mu)$ and $(Y,{\mathcal{F}},\lambda)$ are as in Theorem 8.6, and if $Q\in{\mathcal{F}}\times{\mathcal{F}},$ we define $$ (\mu\times\lambda)(Q)=\int_{x}\lambda(Q_{x})\,d\mu(x)=\int_{Y}\!\mu(Q^{\gamma})\,d\lambda(y). $$ (1) The equality of the integrals in (1) is the content of Theorem 8.6. We call $\mu\times\lambda$ the product of the measures ${\boldsymbol{\mu}}$ and $\lambda.$ That $\mu\times\lambda$ is really a measure (i.e., that $\mu\times\lambda$ is countably additive on ${\mathcal{F}}\times{\mathcal{F}})$ follows immediately from Theorem 1.27. Observe also tha $\mu\times\lambda$ is o-finite The Fubini Theorem be an 8.8 Theorem Let $(X,{\mathcal{F}},\mu)$ and $(Y,{\mathcal{F}},\lambda)$ be o-finite measure spaces, and let f $({\mathcal{I}}\times{\mathcal{I}})$ -measurable function on $x\times Y$ (a) r0 $\leq f\leq\infty,$ and if $$ \varphi(x)=\left[y_{x}^{\phantom{*}}d\lambda,\qquad\psi(y)=\right]_{x}^{\phantom{*}}f^{y}~d\mu\qquad(x\in X,\,y\in Y), $$ (1) then $\varphi$ is ${\mathcal{P}}.$ -measurable, $\psi$ is ${\mathcal{F}}.$ -measurable, and $$ \bigcap_{x}\!\varphi\;d\mu= \bigcap_{X\times Y}f\;d(\mu\times\lambda)= \bigcap_{Y}\psi\;d\lambda. $$ (2) (b)Iff is complex and i $$ \varphi^{*}(x)=\int_{Y}|\,f|_{x}\,d\lambda\quad{\mathrm{~and~}}\quad\int_{X}^{*}\varphi^{*}\,\,d\mu<\infty, $$ (3) then fe $L(\mu\times\lambda).$INTEGRATTON ON PRODUcT SPACEs 165 (c)Iff ∈ $L(\mu\times\lambda)$ $t h e n f_{x}\in L^{1}(\lambda).$ for almost al $x\in X,f^{y}\in L^{1}(\mu)_{J}$ for almost all $\scriptstyle I=\Omega_{\parallel3}$ y e Y; the functions $\varphi$ and w, defined by (1) a.e.,are in $\scriptstyle T(\phi)$ and respectively, and (2) holds. Notes: The first and last integrals in (2) can also be written in the more usua form $$ \bigcap_{X}^{}d\mu(x)\,\bigcap_{y^{\prime}}f(x,\,y)\;d\lambda(y)=\bigcap_{y^{\prime}}d\lambda(y)\,\bigcap_{X}^{}f(x,\,y)\;d\mu(x). $$ (4) These are the so-called“iterated integrals” of f. The middle integral in (2) is often referred to as a double integral. The combination of(b)) and (c) gives the following useful result::If $\boldsymbol{\mathit{f}}$ is $({\mathcal{I}}\times{\mathcal{I}})$ -measurable and j $$ \bigcap_{x}d\mu(x) \langle_{Y}|f(x,y)|\,d\lambda(y)<\infty, $$ (5) then the two iterated integrals (4) are finite and equal In other words,“the order of integration may be reversed”for $({\mathcal{I}}\times{\mathcal{I}})$ measurable functions $\boldsymbol{\f}$ whenever $\scriptstyle{\mathcal{I}}\ \;{\mathcal{I}}\;{\mathcal{I}}$ and also whenever one of the iterated integrals of |flis finite PRoOF We first consider (a). By Theorem 8.5, the definitions of p and simple $({\mathcal{I}}\times{\mathcal{F}})$ exactly the conclusion of Theorem 8.6. Hence $\mathbf{\tau}_{(a)}$ By Definition 8.7,(2) is then make sense.Suppose $Q\in{\mathcal{F}}\times{\mathcal{F}}$ and $f=\gamma_{Q}.$ holds for all nonnegative -measurable functions s. In the general case,there is a sequence of such functions $S_{n\,,}$ If $\varphi_{n}$ such that $0\leq s_{1}\leq s_{2}\leq\cdots$ and $s_{s}(x,y)\to$ f(x, y) at every point of $x\times Y.$ is associated with ${\boldsymbol{S}}_{n}$ in the same way in which p was associated $\mathrm{to}\,f_{},$ we have $$ \bigcap_{X}^{*}\varphi_{n}\,d\mu=\left\vert_{X\times Y}^{*}\,s_{n}\,d(\mu\times\lambda)\qquad(n=1,\,2,\,3,\,...).\qquad\qquad\qquad(n=1,\,2,\,3,\,...\right). $$ (6) The monotone convergence theorem, applied on ${\mathfrak{x}}\in X,$ as n→OO. Hence the monotone con- shows that $\varphi_{n}(x)$ $(Y,{\mathcal{F}},\lambda),$ increases top(x),for every vergence theorem applies again, to the two integrals in(6),and the first equality (2) is obtained. The second half of ((2) follows by interchanging the roles of $\scriptstyle{\mathcal{X}}$ and y. This completes (a) spond to $f^{+}$ and $f^{-}$ as If we apply (a) to|f|, we see that (b) is true corresponds to f $\boldsymbol{\mathsf{f}}$ f in (1). Since $f\in L^{1}(\mu\times\lambda)$ $\varphi_{2}$ corre- then follows. If $\boldsymbol{\mathsf{f}}$ $\varphi$ is real,(a) applies to ft and tof". Let $\varphi_{1}$ and the complex case and Obviously, it is enough to prove (c) for real $L(\mu\times\lambda);$166 REAL AND coMPLEX ANALYSIs $f^{+}\leq|f|,$ and since (a) holds for ft, we see that $\varphi_{1}\in L^{n}(\mu).$ Similarly, p2 ∈ $\scriptstyle T_{(0)}$ Since $$ f_{x}=(f^{+})_{x}-(f^{-})_{x} $$ (7)) we $\operatorname{have}f_{x}\in L_{\geq}^{1}$ for every $\scriptstyle{\mathcal{X}}$ K for which $\varphi_{i}(x)<\infty$ $x\,;$ and at any such x, we have $f^{+}$ and and $\varphi_{2}(x)<\infty$ ; since $\varphi_{1}$ and $\varphi_{2}.$ are in $c_{\mathrm{tot}}$ this happens for almost all Now(2) holds with $\varphi_{1}$ and $f^{y}$ with $\varphi(x)=\varphi_{1}(x)-\varphi_{2}(x).$ Hence $\varphi\in L^{L}(\mu).$ we subtract the resulting equations, we $\varphi_{2}$ and f", in place of $\varphi\operatorname{and}f;{\mathfrak{k}}$ obtain one half of (C). The other half is proved in the same manner, with and 1 in place of f, and ${\boldsymbol{\varphi}}.$ // $\psi$ 8.9 Counterexamples The following three examples will show that the various hypotheses in Theorems 8.6 and 8.8 cannot be dispensed with (a)Let $X=Y=[0,$ 1], $\mu=\lambda=1$ Lebesgue measure on [0,1]. Choose $\{\delta_{n}\}$ so that with support in $(\delta_{n},$ $\delta_{n+1}\gamma,$ such that $\textstyle{\bigcap_{0}^{1}{g_{n}(t)}\;d t}=1,$ for and let g,be a real continuous fuinction Define $0=\delta_{1}<\delta_{2}<\delta_{3}<\cdots,\;\delta_{n}\to1,$ $n=1,2,3,\ldots.$ $$ f(x,\,y)=\sum_{n=1}^{\infty}\left[g_{n}(x)-g_{n+1}(x)\right]g_{n}(y). $$ Note that at each point (x, $y\quad$ ) at most one term in this sum is different from O Thus no convergence problem arises in the definition of f. An easy computa tion shows that $$ |\O_{0}^{1}d x\,\prod_{\vartheta_{0}}^{1}f(x,\,y)\;d y=1\neq0=\prod_{0}^{1}\!d y\,\prod_{0}^{1}f(x,\,y)\;d x, $$ so that the conclusion of the Fubini theorem fails, although both iterated integrals exist. Note that $\boldsymbol{\mathit{f}}$ is continuous in this example, except at the point (1, 1), but that $$ \bigcap_{0}^{1}d x\bigcap_{y}^{1}|f(x,y)|\;d y=\infty. $$ (b)Let on ${\boldsymbol{Y}},$ and put f(x, $\scriptstyle y=1$ if $x=y,f(x,y)=0$ if $x\neq y.$ Then $X=Y=[0,$ 1], p = Lebesgue measure on [0, 1],入= counting measure $$ \bigcap_{X}f(x,y)\;d\mu(x)=0,\qquad\bigcap_{Y}f(x,y)\;d\lambda(y)=1 $$ for all x and $\scriptstyle{y}$ in [0, 1], so that $$ \left.\int_{Y}^{}d\lambda(y)\,\left[_{X}f(x,\,y)\,d\mu(x)=0\,\neq\,1=\right.\right._{X}^{}d\mu(x)\, ]_{Y}^{}f(x,\,y)\,d\lambda(y). $$ t $\boldsymbol{\lambda}$ is not oc-finite. This time the failure is due to the fact that Observe that our function fis $({\mathcal{I}}\times{\mathcal{F}}).$ measurable, if ${\mathcal{G}}$ is the class of all Lebesgue measurable sets in [0, 1] and ${\mathcal{T}}$ consists of all subsets of [O,1].INTEGRATION ON PRODUcT SPACEs 167 To see this, note that $f=\chi_{D},$ where ${\boldsymbol{D}}$ is the diagonal of the unit square Given n, put $$ I_{j}=\left[{\frac{j-1}{n}},{\frac{j}{n}}\right] $$ and put $$ Q_{n}=(I_{1}\times I_{1})\cup(I_{2}\times I_{2})\cup\dots\cup(I_{n}\times I_{n}). $$ Then $Q_{n}$ is a finite union of measurable rectangles, and $D=\bigcap Q_{n}.$ (c) In examples (a) and (b), the failure of the Fubini theorem was due to the fact that either the function or the space was“too big”We now turn to the role played by the requirement that f be measurable with respect to the o-algebra ${\mathcal{I}}\times{\mathcal{I}}$ To pose the question more precisely, suppose $\mu(X)=\lambda(Y)=1,0\leq f\leq1$ is (so that “bigness”is certainly avoided); assume $f_{x}$ is ${\mathcal{T}}$ r-measurable and $f^{y}$ ${\mathcal{P}}.$ measurable, for all x and $y;$ $\psi$ are defined as in 8.8(1). Then P-measurable and y is $0\leq\varphi\leq1,$ and assume p is ${\mathcal{F}}.$ ${\mathcal{T}}.$ -measurable, where p and $0\leq\psi\leq1,$ and both iterated integraks are finite. (Note that no reference to product measures is needed to define iterated integrals.) Does it follow that the two iterated integrals of f are equal? The (perhaps surprising) answer is no. In the following example (due to Sierpinski), we take $$ (X,\,{\mathcal{F}},\,\mu)=(Y,\,{\mathcal{F}},\,\lambda)=[0,\,1] $$ with Lebesgue measure. The construction depends on the continuum hypoth- esis. It is a consequence of this hypothesis that there is a one-to-one mapping $\dot{\boldsymbol{\jmath}}$ of the unit interval [0,1] onto a well-ordered set $\textstyle W$ such that j(x) has at most countably many predecessors in $W,$ for each xe[0, 1]. Taking this for granted, let $Q_{\mathbf{\delta}}Q$ be the set of all (x, y) in the unit square such that jx) precedes ${\dot{j}}(y)$ in $W.$ For each xe [0,1], $Q_{x}$ contains all but countably many points of [0,1]; for each $\nu\in\mathbb{D},$ 1], $Q^{\y}$ contains at most countably many points of [0,1]. Iff = $\gamma_{Q}\,.$ it follows that f, and f' are Borel measurable and that $$ \varphi(x)=\left|^{1}f(x,y)\,d y=1,\qquad\psi(y)=\right|_{0}^{1}f(x,y)\,d x=0 $$ for all $\scriptstyle{\mathcal{X}}$ and y. Hence $$ \left|\O_{0}^{1}d x\,\prod_{0}^{1}f(x,\,y)\,d y=1\neq0= |\O_{0}^{1}\!d y\,\right|_{0}^{1}f(x,\,y)\,d x $$ Completion of Product Measures that 8.10 If (X, J9,p) and (Y, 9 × 9, $\mu\times\lambda\mathbf{)}$ is complete. There is nothing pathological about ${\mathcal{T}},\,\lambda\rangle$ are complete measure spaces, it need not be true $(X\times Y,$168 REAL AND COMPLEX ANALYSIS and suppose that there exists a this phenomenon: Suppose that there exists an A ∈ J,A ≠G,with so that $B\notin{\mathcal{T}}.$ Then $A\times B\subset A\times Y,$ $\mu(A)=0;$ $\scriptstyle{B\,\in\,Y}$ $(\mu\times\lambda)(A\times Y)=0,$ but $A\times B\not\in{\mathcal{F}}\times{\mathcal{F}}.$ (The last assertion follows from Theorem 8.2.) For instance, if $\mu=\lambda=m_{1}$ (Lebesgue measure on $R^{1})$ let $\scriptstyle A$ consist of any by its construction. However, $m_{2}$ be any nonmeasurable set in $R^{1}.$ Thus $m_{1}\times m_{1}$ is not a one point, and let $\boldsymbol{B}$ complete measure; in particular $m_{1}\times m_{1}$ is not $m_{2}\,.$ , since the latter is complete, is the completion of $m_{1}\times m_{1}$ This result gener alizes to arbitrary dimensions: 8.11 Theorem Let $m_{k}$ denote Lebesgue measure on $R^{k}.$ 1f $k=r+s,\;r\geq1,$ s≥1, then $m_{k}$ is the completion of the product measure $m_{r}\times m_{s}$ PR0OF Let ${\mathcal{B}}_{k}$ and ${\mathfrak{M}}_{k}$ be the o-algebras of all Borel sets and of all Lebesgue measurable sets in $R^{k},$ , respectively. We shall first show that $$ \mathcal{B}_{k}\subset\mathfrak{M}_{r}\times\mathfrak{M}_{s}\subset\Re_{k}\,. $$ (1) ${\mathcal{R}}_{k}$ such that Every k-cell belongs to ${\mathfrak{M}}_{r}\times{\mathfrak{M}}_{r}.$ The ${\boldsymbol{\sigma}}\cdot$ -algebra generated by the $F\in\mathfrak{M}_{n}.$ It is easy to The $k{\mathrm{~}}$ cells is Hence ${\mathcal{R}}_{k}\subset{\mathfrak{M}}_{r}\times{\mathfrak{M}}_{s}.$ Next, suppose $E\in{\mathfrak{M}},$ and belong to ${\mathfrak{M}}_{k}$ see, by Theorem 2.20(b), that both $\scriptstyle{E\times R^{\prime}}$ and $\scriptstyle\mathbf{F}\times F$ Choose same is true of their intersection $E\times F.\mathbf{T}$ follows that ${\mathfrak{M}}_{r}\times{\mathfrak{M}}_{s}\subset{\mathfrak{M}}_{k}$ and $P_{2}\in{\mathcal{B}}_{k}$ $Q\in\mathfrak{M}_{r}\times\mathbb{N}_{s}.$ Then $Q\in{\mathfrak{M}}_{k}$ ,so there are sets $P_{1}$ $P_{1}\subset Q\subset P_{2}$ and $m_{k}(P_{2}-P_{1})=0.$ Both $m_{k}$ and $m_{r}\times m_{s}$ are trans- lation invariant Borel measures on $R^{k}.$ They assign the same value to each $k{\mathrm{:}}$ -cell. Hence they agree on ${\mathcal{A}}_{k}\,,$ by Theorem 2.20(d). In particular, $$ (m_{r}\times m_{s})(Q-P_{1})\leq(m_{r}\times m_{s})(P_{2}-P_{1})=m_{s}(P_{2}-P_{1})=0 $$ and therefore $$ (m_{r}\times m_{s})(Q)=(m_{r}\times m_{s})(P_{1})=m_{k}(P_{1})=m_{k}(Q). $$ So $m_{r}\times m_{s}$ agrees with $m_{k}$ on ${\mathfrak{M}}_{n}\times{\mathfrak{M}}_{n}.$ -completion of ${\mathfrak{M}}_{r}\times{\mathfrak{M}}_{s},$ and this // It now follows that ${\mathfrak{M}}_{k}$ is the $(m_{r}\times m_{s})$ is what the theorem asserts. We conclude this section with an alternative statement of Fubini's theorem which is of special interest in view of Theorem 8.11. Let 8.12 Theorem Let $(X,{\mathcal{F}},\mu)$ and(Y, 9,2) be complete o-finite measure spaces. $\mu\times\lambda.$ Let $({\mathcal{F}}\times{\mathcal{T}})^{\star}$ be the completion of ${\mathcal{I}}\times{\mathcal{I}}.$ relatipe to the measure $\boldsymbol{\f}$ be an $({\mathcal{I}}\times{\mathcal{I}})^{*}$ -measurable function on $x\times Y$ Then all conclusions of Theorem 8.8 hold, the only difference being as follows: The J -measurabilit y of $f_{x}$ x can be asserted only for almost all xe X so that qp(x) is only defined a.e.[u]by 8.8(1); a similar statement holds for f and ${\boldsymbol{\psi}}.$INTEGRATION ON PROpUCT SPACEs 169 The proof depends on the following two lemmas: Lemma 1 Suppose v is a positive measure on a o-algebra D, 0t* is the com- pletion of W relative to v, and $\boldsymbol{\f}$ is an ${\mathfrak{M}}^{\ast}$ -measurable function. Then there exists an JDR-measurable function g such that f = g a.e. [v]. (An interesting special case of this arises when ${\boldsymbol{\mathit{U}}}$ is Lebesgue measure on ${\boldsymbol{R}}^{k}$ and R is the class of all Borel sets in F $R^{k}.$ $h=0$ Lemma2 Let h be an $({\mathcal{F}}\times{\mathcal{T}})^{*}.$ -measurable function on $x\times Y$ such that for a.e.[u ×刀. Then for almost all $\operatorname{xe}{\mathcal{X}}$ it is true that h(x $\scriptstyle y\;=\;0$ almost all $y\in Y;$ in particular, $h_{x}$ is J -measurable for almost all $\operatorname{xe}{\mathcal{X}}$ .A similar statement holds for $h^{\nu}.$ If we assume the lemmas, the proof of the theorem is immediate: If f is as in the theorem, Lemma 1(with $v=\mu\times\lambda)$ shows that $f=g+h,$ where $h=0$ a.e that $f_{x}=g_{x}$ Hence the two iterated integrals ${\mathfrak{o f}}\,f,$ measurable. Theorem 8.8 applies to $f^{\prime}=g^{\prime}$ a.e.[u] for almost al $y.$ $\lceil\mu\times\lambda\rceil$ and $\scriptstyle{\mathcal{G}}$ a.e.[] for almost all $\scriptstyle{\mathcal{X}}$ ${\mathfrak{g}}.$ Lemma 2 shows is $({\mathcal{I}}\times{\mathcal{I}})$ and that as well as the double integral, are the same as those of g, and the theorem follows. PRoOF OF LEMMA 1 Suppose $\boldsymbol{\f}$ is $\scriptstyle y x:$ measurable and $f\geq0.$ There exist ODt* each $x\in X_{\circ}$ measurable simple functions $0=s_{0}\leq s_{1}\leq s_{2}\leq\cdots$ such that $s_{n}(x)\to f(x)$ for as $n\to\infty$ Hence $f_{.}=\sum_{*}(s_{n+1}-s_{n}).$ Since $N_{n+1}-S_{n}$ is a finite linear combination of characteristic functions, it follows that there are con- stants $\scriptstyle{\varepsilon_{k}}\quad\quad$ and sets $E_{i}\in{\mathfrak{M}}^{*}$ such that $$ f(x)=\sum_{i=1}^{\infty}c_{i}\,\gamma_{E_{i}}(x)\qquad(x\in X). $$ The definition of ${\mathfrak{M}}^{*}$ (see Theorem 1.36) shows that there are sets $\scriptstyle A_{i}\in C$ 0, $B_{i}\in{\mathfrak{M}},$ such that $A_{i}\subset E_{i}\subset B_{i}$ and $v(B_{i}-A_{i})=0.$ Define $$ g(x)=\sum_{i=1}^{\infty}c_{i}\chi_{A_{i}}(x)\qquad(x\in X). $$ that Then the function $\mathbf{\Omega}^{g}$ is ODR-measurable, and $g(x)=f(x),$ except possibly when $g=f\mathbf{a}.\mathbf{c}.$ ${x}_{\ast}\in\bigcup\,(E_{i}-A_{\frac{i}{-}}-\underline{{{\bigcup}}}\,(B_{i}-A_{i}).$ Since $v(B_{i}-A_{i})=0$ for each i, we conclude [v]. The general case (f real or complex) follows from this. / PRoOF OF LEMMA2 Let ${\mathbf{}}P$ be the set of all points in $x\times Y$ at which $h(x,\,y)\neq0.$ Then $P\in({\mathcal{P}}\times{\mathcal{P}})^{*}$ and $(\mu\times\lambda)(P)=0.$ Hence there exists a $Q\in{\mathcal{I}}\times{\mathcal{I}}$ such that ${\boldsymbol{P}}\in{\mathcal{Q}}$ and $(\mu\times\lambda)(Q)=0.$ By Theorem 8.6, $$ \bigcup_{X}^{}\lambda(Q_{x})\;d\mu(x)=0. $$ (1)170 REAL AND coMPLEX ANALYSIS Let ${\mathbf{}}N$ be the set of all ${\mathfrak{x}}\in X$ at which $\delta(Q_{a})>0.$ It follows from(1) that ${\mu(N)=0}.$ For every x生N,A(Q,)= 0. Since $P_{x}\in Q_{x}$ and $(Y,{\mathcal{F}},\lambda)$ is a com- /// $h_{x}(y)=0$ plete measure space, every subset of $P_{x}$ belongs to $\textstyle{\mathcal{F}}$ if $x\notin N.$ If yf $P_{x},$ then $h_{x}(y)=0.$ Thus we see, for every $x\notin N.$ , that $h_{x}$ is ${\mathcal{F}}.$ -measurable and that a.e. [23 Convolutions 8.13 It happens occasionally that one can prove that a certain set is not empty by proving that it is actually large. The word“large”may of course refer to various properties. One of these (a rather crude one) is cardinality. An example is furnished by the familiar proof that there exist transcendental numbers: There are only countably many algebraic numbers but uncountably many real numbers, hence the set of transcendental real numbers is not empty. Applications of Baire's theorem are based on a topological notion of largeness: The dense $G_{\delta}{}^{\circ}{\mathfrak{S}}$ are “large”subsets of a complete metric space.A third type of largeness is measure-theoretic: One can try to show that a certain set in a measure space is not empty by showing that it has positive measure or, better still, by showing that its complement has measure zero. Fubini's theorem often occurs in this type of argument. For example, let $\boldsymbol{\f}$ and $g\in L^{*}(R^{1}),$ assume $f\geq0$ and $\scriptstyle g\leq0$ for the moment, and consider the integral $$ h(x)=\int_{-\infty}^{\infty}f(x-t)g(t)\,d t\qquad(-\infty-x<\infty). $$ (1) For any fixed $\scriptstyle X,$ the integrand in $\mathbf{(1)}$ is a measurable function with range in [O, oo], so that $h(x)$ is certainly well defined by (1), and $0\leq h(x)\leq\infty.$ But is there any $\scriptstyle{\mathcal{X}}$ for which $h(x)<\alpha\prime$ Note that the integrand in(1) is, for each fixed x, the product of two members of $L^{1}\!\cdot\!$ ,and such a product is not always in $L^{1}.$ !. 「Example: $f(x)=g(x)=1/{\sqrt{x}}$ if $0<x<1$ 0 otherwise.]The Fubini $h\in L^{*}(R^{1}),$ theorem will give an affirmative answer. In fact, it will show that hence that $h(x)<\infty$ a.e. 8.14 Theorem Suppose fe ${\cal L}^{1}(R^{1}),\,g\in{\cal L}^{1}(R^{1}).\,T h e n$ $$ \bigcap_{-\infty}^{\infty}|f(x-y)g(y)|\,d y<\infty $$ (1) for almost all x. For these x, define $$ h(x)=\stackrel{ harpoonup\infty}{\sim}f(x-y)g(y)\;d y. $$ (2) Then $h\in L^{n}(R^{1}),$ and $$ \|h\|_{1}\leq\|f\|_{1}\|g\|_{1}, $$ (3)INTEGRATION ON PRODUCT SPACES $171$ where $$ \|f\|_{1}=\bigcap_{-\infty}^{\infty}|f(x)|\,d x. $$ (4) We call $\boldsymbol{h}$ the convolution of f and ${\mathfrak{g}},$ and write $h=f*\,g.$ functions. PROOF There exist Borel functions $f_{0}$ foand ${\mathfrak{g}}_{0}$ such that $f_{0}=f$ a.e. and $g_{0}=g$ and g by ${\mathfrak{g}}_{0}$ . Hence we may assume, to begin with, that $\boldsymbol{\f}$ and if we replace by f are Borel a.e. The integrals (1) and (2) are unchanged, for every $\scriptstyle X_{\mathrm{,}}$ $\scriptstyle{\mathcal{G}}$ defined by To apply Fubini's theorem,we shall frst prove that the function ${\mathbf{}}F$ $$ F(x,\,y)=f(x-\,y)g(y) $$ (5) is a Borel function on $R^{2}.$ by Define $\varphi\colon R^{2}\to R^{1}$ and $\textstyle\psi\colon R^{2}\to R^{1}$ $$ \varphi(x,\,y)=x-y,\qquad\psi(x,\,y)=y. $$ (6) Then $f(x-y)=(f\circ\varphi)(x,\,y)$ and $g(y)=(g\circ\psi)(x,\,y).$ and $g\circ\psi$ are Borel functions on are Borel $R^{2}.$ functions, Theorem 1.12(d) shows $\mathrm{that}f\circ\varphi$ Since $\varphi$ and $\psi$ Hence so is their product Next we observe that $$ \left|\stackrel{\infty}{\ldots}d y\ \\ {\sim}_{-\infty}^{*\infty}|F(x,\,y)|\,d x= |\O_{-\infty}^{\infty}|\,g(y)\,|\,d y .\right>_{-\infty}^{\infty}|f(x-y)\,|\,d x=\|f\|_{1}\|g\|_{1},\,\,{\mathrm{~1}} $$ (7) since $$ \bigcap_{-\infty}^{\infty}|f(x-y)|\,d x=\|f\|_{1} $$ (8) for every $\scriptstyle\gamma\in R^{\dagger}$ by the translation-invariance of Lebesgue measure Thus $F\in L^{1}(R^{2}),$ and Fubini's theorem implies that the integral (2) exists for almost all $x\in R^{1}$ and that $h\in L^{1}(R^{1}).$ Finaiy, $$ ||h||_{1}=\left\{\O_{-\infty}^{\infty}\mid h(x)|\ d x\leq \{\O_{-\infty}^{\infty}d x\ \right\}_{-\infty}^{\infty}|F(x,\ y)|\ d y $$ $$ = \langle\mathbf{\Lambda}_{-\infty}^{\infty}d y\ \rangle_{-\infty}^{\langle\infty}|F(x,\,y)|\,d x=\|f\|_{1}\|g\|_{1}, $$ by (7). This gives (3), and completes the proof // Convolutions will play an important role in Chap. 9.172 REAL AND coMPLEX ANALYSIs Distribution Functions 8.15 Definition Let ${\boldsymbol{\mu}}$ be a o-finite positive measure on some o-algebra in some set $X.$ Let f: $X{ arrow}[0,\,\infty$ ] be measurable. The function that assigns to each $t\in[0,\;\infty)$ the number $$ \mu\{f>t\}=\mu(\{x\in X:f(x)>t\}) $$ (1) is called the distribution function of f. It is clearly a monotonic (nonincreasing function of t and is therefore Borel measurable. One reason for introducing distribution functions is that they make it possible to replace integrals over $\textstyle X$ by integrals over [0, 0o); the formula $$ \bigcap_{X}f\,d\mu= \vert{\overset{\circ}{\operatorname{o}}}\mu\{f>t\}\ d t $$ (2) is the special case $L^{p}.$ -property of the maximal functions that were introduced in Chap derive an $\varphi(t)=t$ of our next theorem. This will then be used to 7. 8.16 Theorem Suppose that f and ${\boldsymbol{\mu}}$ are as above, that p:[O, oo]→[0,oo]is and $\varphi(t)\to\varphi(\infty)$ as monotonic, absolutely continuous on [0,T] for every Then $T<\infty.$ and that $\varphi(0)=0$ $t\to\infty.$ $$ \bigcup_{X}(\varphi\circ f)\,d\mu=\bigcup_{0}^{*\omega}\mu\{f>t\}\varphi^{\prime}(t)\,d t. $$ (1) PROOF Let $\boldsymbol{E}$ be the set of all $(x,t)\in X\times[0,\;\infty)$ where $f(x)>t.$ When $\boldsymbol{\mathsf{f}}$ is simple, then $\boldsymbol{E}$ 1. is a union of finitely many measurable rectangles, and is there- fore measurable. In the general case, the measurability of $\boldsymbol{E}$ follows via the standard approximation of f by simple functions (Theorem 1.17). As in Sec. 8.1, put $$ E^{t}=\{x\in X\colon(x,\,t)\in E\}\qquad(0\leq t<\infty). $$ (2) The distribution function of fis then $$ \mu(E^{\prime})=\int_{x}\chi_{E}(x,\,t)\ d\mu(x). $$ (3) The right side of (1) is therefore $$ \left|\bigcup_{0}^{\infty}\mu(E^{t})\varphi^{\prime}(t)\;d t= |\bigcup_{x}^{}d\mu(x)\setminus\right|^{\infty}\chi_{E}(x,\,t)\varphi^{\prime}(t)\;d t, $$ (4) by Fubini's theorem.INTEGRA TION ON PRODUCr SPACES 173 For each xe X, xgx $t)=1$ if t<f(x) and is O if t≥f(x)The inner integral in (4) is therefore $$ \int_{0}^{t f(x)}\varphi^{\prime}(t)\;d t=\varphi(f(x)) $$ (5) by Theorem 7.20. Now (1) follows from (4) and (5) // 8.17 Recall now that the maximal function Mf lies in weak ${\boldsymbol{L}}^{1}$ when $f\in L^{l}(R^{l})$ (Theorem 7.4). We also have the trivial estimate $$ \|M f\|_{\infty}\leq\|f\|_{\infty} $$ (1) valid for ${\mathrm{all}}\,f\in L^{\alpha}(R^{k}).$ A technique invented by Marcinkiewicz makes it possible to “ interpolate” between these two extremes and to prove the following theorem of Hardy and Littlewood (which fails when $p=1;$ see Exercise 22, Chap. 7). 8.18 Theorem If1 <p< Oo and f∈ LE(R') then Mf ∈ LP(R) PR00F Since $M f=M(\mid f\mid)$ we may assume, without loss of generality, that f≥ 0. Theorem 7.4 shows that there is a constant A, depending only on the dimension $k_{\mathrm{{,}}}$ such that $$ m\{M g>t\}\leq\frac{A}{t}\Vert g\Vert_{1} $$ (1) for every $g\in L^{*}(R^{k}).$ Here, and in the rest of this proof, $m=m_{\mathrm{k}},$ the Lebesgue measure on $R^{k}.$ Pick a constant $c,\,0<c<1,$ which will be specified later so as to mini- mize a certain upper bound. For each t∈ (Q,o0), split finto a sum $$ f=g_{t}+h_{t} $$ (2) where $$ g_{t}(x)={\binom{f(x)}{0}}\quad\quad{\mathrm{if}}f(x)>c t. $$ (3) Then $0\leq h_{t}(x)\leq c t$ for every xe $R^{k}.$ Hence $h_{t}\in L^{\infty}$ $M h_{t}\leq c t,$ and $$ M f\leq M g_{t}+M h_{t}\leq M g_{t}+c t. $$ (4) If (Mf Xx) >t for some ${\mathfrak{X}},$ it follows that $$ (M g_{i})(x)>(1-c)t. $$ (5) Setting $E_{t}=\{f>c t\},$ (5), (1), and (3) imply that $$ m\{M f>t\}\leq m\{M g_{t}>(1-c)t\}\leq{\frac{A}{(1-c)t}}\;\|g_{t}\|_{1}={\frac{A}{(1-c)t}}\;\int_{E_{t}}f\,d m. $$174 REAL AND coMPLEX ANALYsIs We now use Theorem 8.16, with $X=R^{k},\,\mu=m,\,\varphi(t)=t^{p}.$ to calculate $$ \left[\begin{array}{c}{{\vphantom{\int}_{R^{k}}(M f)^{p}~d m=p~\displaystyle{\int_{0}^{\infty}m\{M f>t\}t^{p-1}~d t\le{\frac{A p}{1-c}}~\displaystyle{\int_{0}^{t^{p-2}}~d t}~\displaystyle{\int_{E_{t}}^{\ldots}t}}~d m}}\end{array}\right]~, $$ $$ ={\frac{A p}{1-c}}\left.{\right\}_{R}}f\,d m\,\left.{\atop{\scriptstyle b o}}^{f f c}t^{p-2}\,\,d t=\frac{A p c^{1-p}}{(1-c)(p-1)}\,\right/_{R^{3}}f^{p}\,\,d m. $$ This proves the theorem. However, to get a good constant, let us choose $\scriptstyle{\mathcal{C}}$ C sO as to minimize that last expression. This happens when $c=(p-1)/p=1/q,$ where $\boldsymbol{\mathit{q}}$ is the exponent conjugate to ${\boldsymbol{p}}.$ For this $c_{\mathrm{,}}$ $$ c^{1-p}=\left(1+{\frac{1}{p-1}}\right)^{p-1}<e, $$ and the preceding computation yields $$ \|M f\|_{p}\leq C_{p}\|f\|_{p} $$ (6) where $C_{p}=(A e p q)^{1/p}.$ // Note that $c_{r}\lnot1$ as $p\to\varnothing,$ which agrees with formula_ 8.17(1),and that $C_{p}{ arrow}\infty$ as $p\to1$ Exercises 1 Find a monotone class W in $R^{1}$ which is not a o-algebra,even though $R^{1}\in\mathbb{N}$ and $R^{1}-A\in\mathfrak{N}$ for every $A\in\mathfrak{M}.$ 2 Suppose fis a Lebesgue measurable nonnegative real function on $R^{1}$ and $A(f)$ is the ordinate set of $f.$ This is the set of all points (x, $y)\in R^{2}$ for which $0<y<f(x).$ (a) Is it true that (c) Is the graph of fa measurable subset o $R^{2}\gamma$ )is Lebesgue measurable,in the two-dimensional sense? ${\boldsymbol{R}}^{1}$ equal to the measure of $A(f)^{\gamma}$ $A(f)$ (b) If the answer to (a) is affirmative, is the integral of f over (d If the answer to (c) is affirmative, is the measure of the graph equal to zero? 3 Find an example of a positive continuous function fin the open unit square in $R^{2},$ whose integral (relative to Lebesgue measure) is finite but such that p(x) in the notation of Theorem 8.8 is infinite for some $x\in(0,1)$ 4 Suppose $1\leq p\leq\infty,f\in L^{1}(R^{1}),$ and $g\in L^{p}(R^{1}).$ gXx exis f (a) Imitate the proof of Theorem 8.14 to show that the integral defining $(f*)$ almost all x, that f * $g\in L^{p}(R^{1}),$ and that $$ \|f*\,g\|_{p}\leq\|f\|_{1}\|g\|_{p}. $$ (b) Show that equality can hold in $$ (a)\*\operatorname{if}p=1\ a n d\operatorname{if}p=\varnothing, $$ and find the conditions under which this happens. (d) Assume (e) Assume 1<p< 0,and equality holds in (a) Show that then either and $g\in L^{p}(R^{1})$ such that a.e. $\textstyle\zeta=0$ a.e. or $g=0$ $1\leq p\leq\infty,\epsilon>0,$ and show that there exist f∈ $L^{1}(R^{1})$ $$ \|f*g\|_{p}>(1-\epsilon)\|f\|_{1}\|g\|_{p}. $$nNTEGRATION ON PRODUCT SPACEs 175 S Let ${\cal{M}}$ Associate to each Borel set be the Banach spaceof all complex Borel measures on $R^{1}.$ The norm in ${\cal{M}}$ is lpl $\implies$ lpl(R") $\scriptstyle k\in K^{\prime}$ the set $$ E_{2}=\{(x,y)\colon x+y\in E\}\subset R^{2}. $$ If p and $\lambda\in M,$ define their convolution $\mu*\lambda$ to be the set function given by $$ (\mu\ast\lambda)(E)=(\mu\times\lambda)(E_{2}) $$ for every Borel set $E\subset R^{1}:\mu\times\lambda$ is as in Definition $8.7.$ (a) Prove that $\mu*\lambda$ e M and that $\|{\boldsymbol{\mu}}\ast\lambda\|\leq\|{\boldsymbol{\mu}}\|$ 目入眼. (b) Prove that $\mu*{\lambda}{\mathrm{i}}{\mathrm{s}}$ the unique $v\in M$ such that $$ {\Bigg[}f\,d v= |{\Bigg]}{\Bigg\}}f(x+y)\,d\mu(x)\,\,d\lambda(y) $$ for every fe $C_{0}(R^{1}).$ (All integrals extend over R1.) is commutative, associative, and distributive with respect to (c) Prove that convolution in ${\cal{M}}$ addition (d) Prove the formula $$ (\mu\ast\lambda)(E)=\int d(E-t)~d\lambda(t) $$ for every $\mathcal{M}$ u and $\lambda\in M$ and every Borel set ${\boldsymbol{E}}.$ . Here $$ E-t=\{x-t;x\in E\}. $$ (e) Define $\mathcal{J}$ to be discrete if $\mathcal{J}$ is concentrated on a countable set; define ${\boldsymbol{R}}^{1}$ (note that $m\notin M).$ Prove that $\mu(\{x\})_{_{\sim}}=0$ for every point $x\in R^{1};$ let m be Lebesgue measure on ${}_{\mu}$ to be continuous if $\mu*{\dot{\lambda}}$ is discrete if both ${}_{\mu}$ and $\dot{\boldsymbol{x}}$ are discrete, that $\mu\ *\lambda$ is continuous if ${}_{\!\mu}$ is continuous and $\lambda\in M,$ and that $\mu*\lambda\ll m\ {\mathrm{if~}}\mu\ll$ m dA = g dm, fe L(Rl), and g e L(Rl) and prove that $(f)$ Assume dp =f dm, $d(\mu\ast\lambda)=(f\ast g)$ dm. (g) Properties (a) and (c) show that the Banach space ${\cal{M}}$ is what one calls a commutative Banach algebra. Show that (e) and $(f)$ imply that the set of all discrete measures in ${\cal{M}}$ is a subalgebra of ${\cal M},$ that the continuous measures form an ideal in $M.$ and that the absolutely continuous measures (relative to m) form an ideal in ${\cal{M}}$ which is isomorphic (as an algebra) to $L^{1}(R^{1}).$ for all (h) Show that ${\cal{M}}$ has a unit, ie.,,show that there exists a $\delta\in M$ such that $\delta*\mu=\mu$ $\in{\mathcal{M}}$ cally $\mathbf{\partial}$ (under addition), and there exists a translation invariant Borel measure ${\mathfrak{m}}\,$ on $R^{1}$ is a commutative group $R^{1}$ is (i Only two properties of $R^{1}$ have been used in this discussion: ${\boldsymbol{R}}^{1}$ which is not identi and which is finite on all compact subsets of ${\boldsymbol{R}}^{1}$ Show that the same results hold if replaced by ${\boldsymbol{R}}^{k}$ or by $T_{\mathbf{\delta}}$ (the unit circle) or by ${\boldsymbol{T}}^{k}$ (the k-dimensional torus, the cartesian product of $\displaystyle{\boldsymbol{k}}$ copies of $T),$ as soon as the definitions are properly formulated. and $\scriptstyle A\quad\quad A$ 6(Polar coordinates in $R^{k}{}_{;})$ Let $S_{k-1}$ be the unit sphere in $R^{k}{}_{i}$ , i.e., the set of all u ∈ ${\boldsymbol{R}}^{k}$ whose distance form from the origin $m_{k}$ be Lebesgue measure on $S_{k-1}.$ $R^{k},$ and define a measure $\sigma_{k-1}$ on $S_{k-1}$ as follows: If $4\prec S_{k-1}$ cartesian product $\mathbf{(0,}$ o) × is I. Show that every $x\in R^{k},$ except for $x=0,$ has a unique representation of the $\mathbf{\partial}^{0}$ $x=r u,$ where ${\mathbf{}}T$ is a positive real number and $u\in S_{k-1}.$ Thus $R^{k}-\{0\}$ may be regarded as the Let l is a Borel set, let $\vec{A}$ I be the set of all points $r u_{i}$ where $\ 0<r<1$ and u∈ ${\textsf{}}\,\colon A,$ and define $$ \sigma_{k-1}(A)=k\cdot m_{k}(\tilde{A}). $$176 REAL AND coMPLEX ANALYSIS Prove that the formula $$ \bigcap_{R^{k}}f\,d m_{k}=\left.\right\}_{0}^{\infty}r^{k-1}\,d r\, |_{S_{k-1}}f(r u)\,d\sigma_{k-1}(u) $$ is valid for every nonnegative Borel function $\boldsymbol{\mathit{f}}$ on $R^{k},$ Check that this coincides with familiar results when $k=2$ and when $k=3.$ and if $\scriptstyle A\quad}$ is an open subset of $S_{k-1},$ let $\boldsymbol{E}$ be the set of all ${\boldsymbol{E}}.$ Pass from with Sugestion: If $0<r_{1}<r_{2}$ $R^{k}.$ $r u$ $r_{1}<r<r_{2},u\in A,$ and verify that the formula holds for the characteristic function of these to characteristic functions of Borel sets in 7 Suppose $\displaystyle(X,{\mathcal{I}},\mu)$ and $(Y,{\mathcal{F}},\lambda)$ are ${\sigma}.$ -finite measure spaces,and suppose $\psi$ is a measure on ${\mathcal{F}}\times{\mathcal{F}}$ such that $$ \psi(A\times B)=\mu(A)\lambda(B) $$ whenever $A\in{\mathcal{I}}$ and $B\in{\mathcal{T}}.$ Prove that then $\psi(E)=(\mu\times\lambda)(E)$ for ever $\boldsymbol{E}$ ∈ ${\mathcal{F}}\times{\mathcal{F}}.$ $\boldsymbol{\mathsf{S}}$ (a) Suppose f is a real function on ${\boldsymbol{R}}^{2}$ such that each section ${\boldsymbol{R}}^{2}$ . Note the contrast between this and Example $f^{\gamma}$ is continuous. Prove that fis Borel measurable on ${\boldsymbol{f}}_{x}$ is Borel measurable and each section 8.9(c) (b)Suppose g $\scriptstyle{\mathcal{G}}$ is a real function on ${\boldsymbol{R}}^{k}$ which is continuous in each of the $\displaystyle{\boldsymbol{k}}$ variables separately. Prove that More explicitly, for every choice of ${\boldsymbol{x}}_{2}\,,\,\dots\,,\,{\boldsymbol{x}}_{k}\,,$ the mapping $x_{1}\to g(x_{1},\,x_{2},\,\ldots,\,x_{k})$ is continuous, etc $\scriptstyle{\mathcal{G}}$ is a Borel function. ut Hint: If $(i-1)/n=a_{i-1}\leq x\leq a_{i}=i/n,{\mathrm{P}}$ $$ f_{n}(x,y)=\frac{a_{i}-x}{a_{i}-a_{i-1}}\,f(a_{i-1},y)+\frac{x-a_{i-1}}{a_{i}-a_{i-1}}\,f(a_{i},y). $$ 9 Suppose $\bar{E}$ is a dense set in $R^{1}$ and fis a real function on ${\boldsymbol{R}}^{2}$ such that (a)f, is Lebesgue measurable for each $x\in E$ and $(b)f^{\gamma}$ is continuous for almost all $\scriptstyle{\gamma\cdot K^{\mathrm{F}}}$ Prove that fis Lebesgue measurable on ${\boldsymbol{R}}^{2}$ 10 Suppose fis a real function on $\scriptstyle{\mathbb{R}^{1}f}$ is Lebesgue measurable for each $x,$ and $f^{\gamma}$ is continuous for each $y_{\circ}$ Suppose $\scriptstyle y\cdot R^{\dagger}\to R^{\dagger}$ $R^{1}$ $h(y)=f(g(y),\;\rfloor$ 八小Prove that ${\boldsymbol{h}}$ is is Lebesgue measurable, and put Lebesgue measurable on as in Exercise 8, put $h_{n}(y)=f_{n}(g(y),\,y),$ show that each ${}\quad\int_{\overrightarrow{I}}_{\scriptscriptstyle B}$ is measurable, and that Hint: Define $\textstyle\int_{m}$ $h_{n}(y)\to h(y).$ 11 Let ${\mathcal{B}}_{k}$ be the ${\sigma}^{*}$ -algebra of all Borel sets in $R^{k}.$ Prove that $\mathcal{A}_{m+n}=\mathcal{B}_{m}\times\mathcal{B}_{n}$ This is relevant in Theorem 8.14. 12 Use Fubini's theorem and the relation $$ \frac{1}{x}= [\stackrel{\infty}{\infty}e^{-x t}\,d t\qquad(x>0)\qquad(x>0) $$ to prove that $$ \operatorname*{lim}_{A\to\infty}\ {\sqrt{1}}^{A}{\frac{\sin\,\,x}{x}}\,d x={\frac{\pi}{2}}. $$ 13. Ifpis a complex measure on a o-algebra D, show that every set $E\in{\mathfrak{M}}$ has a subset ${\mathbf{}}A$ for which $$ |\,\mu({\cal A})|\geq\frac{1}{\pi}\,|\,\mu\,|({\cal E}). $$INTEGRATION ON PRODUCT SPACES 177 Suggestion: There is a measurable real function $\theta\L_{\theta}$ so that $d\mu=e^{i\theta}~d|\mu|$ Let $A_{\alpha}$ be the subset of $\scriptstyle{\vec{E}}$ where cos $(\theta-\alpha)>0,$ show that $$ \mathrm{Re}\,\left[e^{-i a}\mu(A_{a})\right]=\left\{_{E}^{}{\cos}^{+}\, (\theta-\alpha\right)\,d\,\vert\,\mu\vert\,, $$ and integrate with respect to α(as in Lemma 6.3) Show, by an example, that 1/r is the best constant in this inequality 14 Complete the following proof of Hardy's inequality (Chap. 3, Exercise 14): Suppose $f\geq0$ on $0,\;\infty),f\in D,\;1<p<\infty,\;\mathrm{and}$ $$ F(x)={\frac{1}{x}}\int_{0}^{x}f(t)\;d t. $$ Write $x F(x)=\int_{0}^{x}f(t)t^{x}t^{-a}\,d t,$ where $0<\alpha<1/q,$ use Holder's inequality to get an upper bound for $F(x)^{p},$ and integrate to obtain $$ \bigcap_{0}^{\infty}F^{p}(x)\;d x\leq(1-\alpha q)^{1-p}(\alpha p)^{-1}\;\int_{0}^{\infty}f^{p}(t)\;d t. $$ Show that the best choice of α yields $$ \bigcap_{0}^{\infty}F^{p}(x)\;d x\leq\left({\frac{p}{p-1}}\right)^{p}\; \{_{0}^{\infty}f^{p}(t)\;d t. $$ 15 Put $:\varphi(t)=1-\cos t\operatorname{if}\ 0\leq t\leq2\pi,\,\varphi(t)=0$ for all other real t. For -0 <x<α, define $$ f(x)=1,\qquad g(x)=\varphi^{\prime}(x),\qquad h(x)=\stackrel{x}{\displaystyle\bigwedge_{-\infty}^{x}}\varphi(t)\;d t. $$ Verify the following statements about convolutions of these functions: (6) $(f*\,g)(x)=0\,{\mathrm{for~all~}}x.$ (Gi) $(g*h)(x)=(\varphi*\varphi)(x)>0\ 0n\,(0,4\pi).$ whereasf *(g * h)is a positive constant (ii) Therefore $(f*g)*h=0,$ But convolution is supposedly associative, by Fubinis theorem (Exercise S(C). What went wrong? 16 Prove the following analogue of Minkowski's inequality for f ≥ 0: $$ \left\{\left\{\frac{}{}\right\}\left\{\frac{}{}f(x,\textstyle{\ y})\;d\lambda(\mathsf{y})\right\}^{p}\;d\mu(x)\right\}^{1/p}\;\leq\; \}\left[\frac{}{}f^{\prime\prime}(x,\mathsf{y})\;d\mu(x)\right]^{1/p}d\lambda(y). $$ Supply the required hypotheses. (Many further developments of this theme may be found in [9].)