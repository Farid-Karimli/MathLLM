\documentclass[10pt]{article}
\usepackage[utf8]{inputenc}
\usepackage[T1]{fontenc}
\usepackage{amsmath}
\usepackage{amsfonts}
\usepackage{amssymb}
\usepackage[version=4]{mhchem}
\usepackage{stmaryrd}
\usepackage{mathrsfs}
\usepackage{bbold}

\begin{document}
\section{APPLICATIONS TO DIFFERENTIAL EQUATIONS}
\section{Fundamental Solutions}
8.1 Introduction We shall be concerned with linear partial differential equations with constant coefficients. These are equations of the form

$$
P(D) u=v
$$

where $P$ is a nonconstant polynomial in $n$ variables (with complex coefficients), $P(D)$ is the corresponding differential operator (see Section 7.1), $v$ is a given function or distribution, and the function (or distribution) $u$ is a solution of (1).

A distribution $E \in \mathscr{D}^{\prime}\left(R^{n}\right)$ is said to be a fundamental solution of the operator $P(D)$ if it satisfies (1) with $v=\delta$, the Dirac measure:

$$
P(D) E=\delta
$$

The basic result (Theorem 8.5, due to Malgrange and Ehrenpreis) that will be proved here is that such fundamental solutions always exist.

Suppose we have an $E$ that satisfies (2), suppose $v$ has compact support, and put

$$
u=E * v
$$

Then $u$ is a solution of (1), because

$$
P(D)(E * v)=(P(D) E) * v=\delta * v=v
$$

by Theorems 6.35 and 6.37 .

The existence of a fundamental solution thus leads to a general existence theorem for the equation (1); note also that every solution of (1) differs from $E * v$ by a solution of the homogeneous equation $P(D) u=0$. Moreover, (3) gives some additional information about $u$. For instance, if $v \in \mathscr{D}\left(R^{n}\right)$, then $u \in C^{\infty}\left(R^{n}\right)$.

It may of course happen that the convolution $E * v$ exists for certain $v$ whose support is not compact. This raises the problem of finding $E$ so that its behavior at infinity is well under control. The best possible result would of course be to find an $E$ with compact support. But this can never be done. If it could, $\hat{E}$ would be an entire function, and (2) would imply $P \hat{E}=1$. But the product of an entire function and a polynomial cannot be 1 unless both are constant.

However, the equation $P \hat{E}=1$ can sometimes be used to find $E$, namely, when $1 / P$ is a tempered distribution; in this case, the Fourier transform of $1 / P$ furnishes a fundamental solution which is a tempered distribution. For examples of this, see
Exercises 5 to 9 .

Another related question concerns the existence of solutions of (1) with compact support if the support of $v$ is compact. The answer (given in Theorem 8.4) shows very clearly that it is not enough to study $P$ on $R^{n}$ in problems of this sort but that the behavior of $P$ in the complex space $\mathbb{C}^{n}$ is highly significant.

8.2 Notations $T^{n}$ is the torus that consists of all points

$$
w=\left(e^{i \theta_{1}}, \ldots, e^{i \theta_{n}}\right)
$$

in $\mathbb{C}^{n}$, where $\theta_{1}, \ldots, \theta_{n}$ are real; $\sigma_{n}$ is the Haar measure of $T^{n}$, that is, Lebesgue meas-
ure divided by $(2 \pi)^{n}$.

A polynomial in $\mathbb{C}^{n}$, of degree $N$, is a function

$$
P(z)=\sum_{|\alpha| \leq N} c(\alpha) z^{\alpha} \quad\left(z \in \mathbb{C}^{n}\right)
$$

where $\alpha$ ranges over multi-indices and $c(\alpha) \in \mathbb{C}$. If (2) holds and if $c(\alpha) \neq 0$ for at least one $\alpha$ with $|\alpha|=N, P$ is said to have exact degree $N$.

8.3 Lemma If $P$ is a polynomial in $\mathscr{C}^{n}$, of exact degree $N$, then there is a constant $A<\infty$, depending only on $P$, such that

$$
|f(z)| \leq A r^{-N} \int_{T^{n}}|(f P)(z+r w)| d \sigma_{n}(w)
$$

for every entire function $f$ in $\mathscr{C}^{n}$, for every $z \in \mathbb{C}^{n}$, and for every $r>0$.

PROOF Assume first that $F$ is an entire function of one complex variable and that

$$
Q(\lambda)=c \prod_{i=1}^{N}\left(\lambda+a_{i}\right) \quad(\lambda \in \mathbb{C})
$$

Put $Q_{0}(\lambda)=c \prod\left(1+\bar{a}_{i} \lambda\right)$. Then $c F(0)=\left(F Q_{0}\right)(0)$. Since $\left|Q_{0}\right|=|Q|$ on the unit circle, it follows that

$$
|c F(0)| \leq \frac{1}{2 \pi} \int_{-\pi}^{\pi}\left|(F Q)\left(e^{i \theta}\right)\right| d \theta
$$

The given polynomial $P$ can be written in the form $P=P_{0}+P_{1}+\cdots+P_{N}$, where each $P_{j}$ is a homogeneous polynomial of degree $j$. Define $A$ by

$$
\frac{1}{A}=\int_{T^{n}}\left|P_{N}\right| d \sigma_{n}
$$

This integral is positive, since $P$ has exact degree $N$. [See part $(b)$ of Exercise 1]. If $z \in \mathbb{C}^{n}$ and $w \in T^{n}$, define

$$
F(\lambda)=f(z+r \lambda w), \quad Q(\lambda)=P(z+r \lambda w) \quad(\lambda \in \mathbb{C})
$$

The leading coefficient of $Q$ is $r^{N} P_{N}(w)$. Hence (3) implies

$$
r^{\tilde{N}}\left|P_{N}(w)\right||f(z)| \leq \frac{1}{2 \pi} \int_{-\pi}^{\pi}\left|(f P)\left(z+r e^{i \hat{\theta}} w\right)\right| d \dot{\theta} .
$$

If we integrate (6) with respect to $\sigma_{n}$, we get

$$
|f(z)| \leq A r^{-N} \cdot \frac{1}{2 \pi} \int_{-\pi}^{\pi} d \theta \int_{T^{n}}\left|(f P)\left(z+r e^{i \theta} w\right)\right| d \sigma_{n}(w)
$$

The measure $\sigma_{n}$ is invariant under the change of variables $w \rightarrow e^{i \theta} w$. The inner integral in (7) is therefore independent of $\theta$. This gives (1). I//I

8.4 Theorem Suppose $P$ is a polynomial in $n$ variables, $v \in \mathscr{D}^{\prime}\left(R^{n}\right)$, and $v$ has compact support. Then the equation

$$
P(D) u=v
$$

has a solution with compact support if and only if there is an entire function $g$ in $\mathbb{C}^{n}$ such that

$$
P g=\hat{v} .
$$

When this condition is satisfied, (1) has a unique solution $u$ with compact support; the support of this $u$ lies in the convex hull of the support of $v$.

PROOF If (1) has a solution $u$ with compact support, $(a)$ of Theorem 7.23 shows that (2) holds with $g=\hat{u}$.

Conversely, suppose (2) holds for some entire $g$. Choose $r>0$ so that $v$ has its support in $r B=\left\{x \in R^{n}:|x| \leq r\right\}$. By Lemma 8.3, (2) implies

$$
|g(z)| \leq A \int_{T^{n}}|\hat{v}(z+w)| d \sigma_{n}(w) \quad\left(z \in \mathbb{C}^{n}\right)
$$

By (a) of Theorem 7.23, there exist $N$ and $\gamma$ such that

$$
|\hat{v}(z+w)| \leq \gamma(1+|z+w|)^{N} \exp \{r|\operatorname{Im}(z+w)|\}
$$

There are constants $c_{1}$ and $c_{2}$ that satisfy

$$
1+|z+w| \leq c_{1}(1+|z|)
$$

and

$$
|\operatorname{Im}(z+w)| \leq c_{2}+|\operatorname{Im} z|
$$

for all $z \in \mathbb{C}^{n}$ and all $w \in T^{n}$. It follows from these inequalities that

$$
|g(z)| \leq B(1+|z|)^{N} \exp \{r|\operatorname{Im} z|\} \quad\left(z \in \mathbb{C}^{n}\right)
$$

where $B$ is another constant (depending on $\gamma, A, N, c_{1}, c_{2}$, and $r$ ). By (7) and (b) of Theorem 7.23, $g=\hat{u}$ for some distribution $u$ with support in $r B$. Hence (2) becomes $P \hat{u}=\hat{v}$, which is equivalent to (1).

The uniqueness of $u$ is obvious, since there is at most one entire function $\hat{u}$ that satisfies $P \hat{u}=\hat{v}$.

The preceding argument showed that the support $S_{u}$ of $u$ lies in every closed ball centered at the origin that contains the support $S_{v}$ of $v$. Since (1) implies

$$
P(D)\left(\tau_{x} u\right)=\tau_{x} v \quad\left(x \in R^{n}\right)
$$

the same statement is true of $x+S_{u}$ and $x+S_{v}$. Consequently, $S_{u}$ lies in the intersection of all closed balls (centered anywhere in $R^{n}$ ) that contain $S_{v}$. Since this intersection is the convex hull of $S_{v}$, the proof is complete.

8.5 Theorem If $P$ is a polynomial in $\mathbb{C}^{n}$, of exact degree $N$, then the differential operator $P(D)$ has a fundamental solution $E$ that satisfies

$$
|E(\psi)| \leq A r^{-N} \int_{T^{n}} d \sigma_{n}(w) \int_{R^{n}}|\hat{\psi}(t+r w)| d m_{n}(t)
$$

for every $\psi \in \mathscr{D}\left(R^{n}\right)$ and for every $r>0$.

Here $A$ is the constant that appears in Lemma 8.3. The main point of the theorem is the existence of a fundamental solution, rather than the estimate (1) which arises from the proof.

PROOF Fix $r>0$, and define

$$
\|\psi\|=\int_{T^{n}} d \sigma_{n}(w) \int_{R^{n}}|\hat{\psi}(t+r w)| d m_{n}(t)
$$

In preparation for the main part of the proof, let us first show that

$$
\lim _{j \rightarrow \infty}\left\|\psi_{j}\right\|=0 \quad \text { if } \psi_{j} \rightarrow 0 \text { in } \mathscr{D}\left(R^{n}\right)
$$

Note that $\hat{\psi}(t+w)=\left(e_{-w} \psi\right)^{\wedge}(t)$ if $t \in R^{n}$ and $w \in \mathbb{C}^{n}$. Hence

$$
\|\psi\|=\int_{T^{n}} d \sigma_{n}(w) \int_{R^{n}}\left|\left(e_{-r w} \psi\right)^{\wedge}\right| d m_{n}
$$

If $\psi_{j} \rightarrow 0$ in $\mathscr{D}\left(R^{n}\right)$, all $\psi_{j}$ have their supports in some compact set $K$. The functions $e_{r w}\left(w \in T^{n}\right)$ are uniformly bounded on $K$. It follows from the Leibniz formula that

$$
\left\|D^{\alpha}\left(e_{-r w} \psi_{j}\right)\right\|_{\infty} \leq C(K, \alpha) \max _{\beta \leq \alpha}\left\|D^{r} \psi_{j}\right\|_{\infty}
$$

The right side of (5) tends to 0 , for every $\alpha$. Hence, given $\varepsilon>0$, there exists $j_{0}$ such that

$$
\left\|(I-\Delta)^{n}\left(e_{-r w} \psi_{j}\right)\right\|_{2}<\varepsilon \quad\left(j>j_{0}, w \in T^{n}\right)
$$

where $\Delta=D_{1}^{2}+\cdots+D_{n}^{2}$ is the Laplacian. By the Plancherel theorem, (6) is the same as

$$
\int_{R^{n}}\left|\left(1+|t|^{2}\right)^{n} \hat{\psi}_{j}(t+r w)\right|^{2} d m_{n}(t)<\varepsilon^{2}
$$

from which it follows, by the Schwarz inequality and (2), that $\left\|\psi_{j}\right\|<C \varepsilon$ for all $j>j_{0}$, where

$$
C^{2}=\int_{R^{n}}\left(1+|t|^{2}\right)^{-2 n} d m_{n}(t)<\infty
$$

This proves (3).

Suppose now that $\phi \in \mathscr{D}\left(R^{n}\right)$ and that

$$
\psi=P(D) \phi
$$

Then $\hat{\psi}=P \hat{\phi}, \hat{\phi}$ and $\hat{\psi}$ are entire, hence $\psi$ determines $\phi$. In particular, $\phi(0)$ is a linear functional of $\psi$, defined on the range of $P(D)$. The crux of the proof
consists in showing that this functional is continuous, i.e., that there is a distribution $u \in \mathscr{D}^{\prime}\left(R^{n}\right)$ that satisfies

$$
u(P(D) \phi)=\phi(0) \quad\left(\phi \in \mathscr{D}\left(R^{n}\right)\right)
$$

because then the distribution $E=\check{u}$ satisfies

$$
\begin{aligned}
(P(D) E)(\phi) & =E(P(-D) \phi)=u\left((P(-D) \phi)^{\vee}\right) \\
& =u(P(D) \check{\phi})=\check{\phi}(0)=\phi(0)=\delta(\phi),
\end{aligned}
$$

so that $P(D) E=\delta$, as desired.

Lemma 8.3, applied to $P \hat{\phi}=\hat{\psi}$, yields

$$
|\hat{\phi}(t)| \leq A r^{-N} \int_{T^{n}}|\hat{\psi}(t+r w)| d \sigma_{n}(w) \quad\left(t \in R^{n}\right)
$$

By the inversion theorem, $\phi(0)=\int_{R n} \hat{\phi} d m_{n}$. Thus (11), (2), and (9) give

$$
|\phi(0)| \leq A r^{-N}\|P(D) \phi\| \quad\left(\dot{\phi} \in \mathscr{D}\left(R^{n}\right)\right)
$$

Let $Y$ be the subspace of $\mathscr{D}\left(R^{n}\right)$ that consists of the functions $P(D) \phi$, $\phi \in \mathscr{D}\left(R^{n}\right)$. By (12), the Hahn-Banach theorem 3.3 shows that the linear functional that is defined on $Y$ by $P(D) \phi \rightarrow \phi(0)$ extends to a linear functional $u$ on $\mathscr{D}\left(R^{n}\right)$ that satisfies $(10)$ as well as

$$
|u(\psi)| \leq A r^{-N}\|\psi\| \quad\left(\psi \in \mathscr{D}\left(R^{n}\right)\right) .
$$

By (3), $u \in \mathscr{D}^{\prime}\left(R^{n}\right)$. This completes the proof.

\section{Elliptic Equations}
8.6 Introduction If $u$ is a twice continuously differentiable function in some open set $\Omega \subset R^{2}$ that satisfies the Laplace equation

$$
\frac{\partial^{2} u}{\partial x^{2}}+\frac{\partial^{2} u}{\partial y^{2}}=0
$$

then it is very well known that $u$ is actually in $C^{\infty}(\Omega)$, simply because every real harmonic function in $\Omega$ is (locally) the real part of a holomorphic function. Any theorem of this type-one which asserts that every solution of a certain differential equation has stronger smoothness properties than is a priori evident-is called a regularity theorem.

We shall give a proof of a rather general regularity theorem for elliptic partial differential equations. The term "elliptic" will be defined presently. It may be of interest to see, first of all, that the equation

$$
\frac{\partial^{2} u}{\partial x \partial y}=0
$$

behaves quite differently from (1), since it is satisfied by every function $u$ of the form $u(x, y)=f(y)$, where $f$ is any differentiable function. In fact, if (2) is interpreted to mean

$$
\frac{\partial}{\partial y}\left(\frac{\partial u}{\partial x}\right)=0
$$

then $f$ can be a perfectly arbitrary function.

8.7 Definitions Suppose $\Omega$ is open in $R^{n}, N$ is a positive integer, $f_{\alpha} \in C^{\infty}(\Omega)$ for every multi-index $\alpha$ with $|\alpha| \leq N$, and at least one $f_{\alpha}$ with $|\alpha|=N$ is not identically 0 . These data determine a linear differential operator

$$
L=\sum_{|\alpha| \leq N} f_{\alpha} D_{\alpha}
$$

which acts on distributions $u \in \mathscr{D}^{\prime}(\Omega)$ by

$$
L u=\sum_{|\alpha| \leq N} f_{\alpha} D_{\alpha} u
$$

The order of $L$ is $N$. The operator

$$
\sum_{|\alpha|=N} f_{\alpha} D_{\alpha}
$$

is the principal part of $L$. The characteristic polynomial of $L$ is

$$
p(x, y)=\sum_{|\alpha|=N} f_{\alpha}(x) y^{\alpha} \quad\left(x \in \Omega, y \in R^{n}\right)
$$

This is a homogeneous polynomial of degree $N$ in the variables $y=\left(y_{1}, \ldots, y_{n}\right)$, with coefficients in $C^{\infty}(\Omega)$.

The operator $L$ is said to be elliptic if $p(x, y) \neq 0$ for every $x \in \Omega$ and for every $y \in R^{n}$, except, of course, for $y=0$. Note that ellipticity is defined in terms of the principal part of $L$; the lower-order terms that appear in (1) play no role.

For example, the characteristic polynomial of the Laplacian

$$
\Delta=\frac{\partial^{2}}{\partial x_{1}^{2}}+\cdots+\frac{\partial^{2}}{\partial x_{n}^{2}}
$$

is $p(x, y)=-\left(y_{1}^{2}+\cdots+y_{n}^{2}\right)$, so that $\Delta$ is elliptic. elliptic.

On the other hand, if $L=\partial^{2} / \partial x_{1} \partial x_{2}$, then $p(x, y)=-y_{1} y_{2}$, and $L$ is not

The main resuit that we are aiming at (Theorem 8.12) involves some special spaces of tempered distributions, which we now describe.

8.8 Sobolev spaces Associate to each real number $s$ a positive measure $\mu_{s}$ on $R^{n}$ by setting

$$
d \mu_{s}(y)=\left(1+|y|^{2}\right)^{s} d m_{n}(y)
$$

If $f \in L^{2}\left(\mu_{s}\right)$, that is, if $j|f|^{2} d \mu_{s}<\infty$, then $f$ is a tempered distribution [Example (c) of 7.12]; hence $f$ is the Fourier transform of a tempered distribution $u$. The vector space of all $u$ so obtained will be denoted by $H^{s}$; equipped with the norm

$$
\|u\|_{s}=\left(\int_{R^{n}}|\hat{u}|^{2} d \mu_{s}\right)^{1 / 2}
$$

$H^{s}$ is clearly isometrically isomorphic to $L^{2}\left(\mu_{s}\right)$.

These spaces $H^{s}$ are called Sobolev spaces. The dimension $n$ will be fixed throughout, and no reference to it will be made in the notation.

By the Plancherel theorem, $H^{0}=L^{2}$.

It is obvious that $H^{s} \subset H^{t}$ if $t<s$. The union $X$ of all spaces $H^{s}$ is therefore a vector space. A linear operator $\Lambda: X \rightarrow X$ is said to have order $t$ if the restriction of $\Lambda$ to each $H^{s}$ is a continuous mapping of $H^{s}$ into $H^{s-t}$; note that $t$ need not be an integer.

Here are the properties of the Sobolev spaces that will be needed.

\subsection{Theorem}
(a) Every distribution with compact support lies in some $\boldsymbol{H}^{\text {s. }}$.

(b) If $-\infty<t<\infty$, the mapping $u \rightarrow v$ given by

$$
\hat{v}(y)=\left(1+|y|^{2}\right)^{t / 2} \hat{u}(y) \quad\left(y \in R^{n}\right)
$$

is a linear isometry of $H^{s}$ onto $H^{s-i}$ and is therefore an operator of order $t$ whose inverse has order $-t$.

(c) If $b \in L^{\infty}\left(R^{n}\right)$, the mapping $u \rightarrow v$ given by $\hat{v}=b \hat{u}$ is an operator of order 0 .

(d) For every multi-index $\alpha, D_{\alpha}$ is an operator of order $|\alpha|$.

(e) If $f \in \mathscr{S}_{n}$, then $u \rightarrow f u$ is an operator of order 0 .

PROOF If $u \in \mathscr{D}^{\prime}\left(R^{n}\right)$ has compact support, $(a)$ of Theorem 7.23 shows that

$$
|\hat{u}(y)| \leq C(1+|y|)^{N} \quad\left(y \in R^{n}\right)
$$

for some constants $C$ and $N$. Hence $u \in H^{s}$ if $s<-N-n / 2$. This proves part $(a) ;(b)$ and $(c)$ are obvious. The relation

implies

$$
\left|\left(D_{\alpha} u\right)^{\wedge}(y)\right|=\left|y^{\alpha}\right||\hat{u}(y)| \leq\left(1+|y|^{2}\right)^{|\alpha| / 2}|\hat{u}(y)|
$$

$$
\left\|D_{\alpha} u\right\|_{s-|\alpha|} \leq\|u\|_{s}
$$

so that $(d)$ holds.

The proof of $(e)$ depends on the inequality

$$
\left(1+|x+y|^{2}\right)^{5} \leq 2^{|s|}\left(1+|x|^{2}\right)^{s}\left(1+|y|^{2}\right)^{|s|}
$$

valid for $x \in R^{n}, y \in R^{n},-\infty<s<\infty$. The case $s=1$ of (3) is obvious. From it the case $s=-1$ is obtained by replacing $x$ by $x-y$ and then $y$ by $-y$. The general case of (3) is obtained from these two by raising everything to the power $|s|$. It follows from (3) that

$$
\int_{R^{n}}|h(x-y)|^{2} d \mu_{s}(x) \leq 2^{|s|}\left(1+|y|^{2}\right)^{|s|} \int_{R^{n}}|h|^{2} d \mu_{s}
$$

for every measurable function $h$ on $R^{n}$.

Now suppose $u \in H^{s}, f \in \mathscr{S}_{n}, t>|s|+n / 2$. Since $\hat{f} \in \mathscr{S}_{n},\|f\|_{t}<\infty$. Put $\gamma=\mu_{|s|-t}\left(R^{n}\right)$. Then $\gamma<\infty$. Define $F=|\hat{u}| *|\hat{f}|$. By Theorem 7.19,

$$
\left|(f u)^{\wedge}\right|=|\hat{u} * \hat{f}| \leq|\hat{u}| *|\hat{f}|=F .
$$

By the Schwarz inequality,

$$
|F(x)|^{2} \leq \int_{R^{n}}|\hat{f}(y)|^{2} d \mu_{t}(y) \int_{R^{n}}|\hat{u}(x-y)|^{2} d \mu_{-t}(y)
$$

for every $x \in R^{n}$. Integrate (6) over $R^{n}$, with respect to $\mu_{s}$. By (4), the result is

$$
\int_{R^{n}}|F|^{2} d \mu_{s} \leq 2^{|s|} \gamma\|f\|_{t}^{2}\|u\|_{s}^{2}
$$

It follows from (5) and (7) that

$$
\|f u\|_{s} \leq\left(2^{|s|} \gamma\right)^{1 / 2}\|f\|_{t}\|u\|_{s} .
$$

This proves $(e)$.

8.10 Definition Let $\Omega$ be open in $R^{n}$. A distribution $u \in \mathscr{D}^{\prime}(\Omega)$ is said to be locally $H^{s}$ if there corresponds to each point $x \in \Omega$ a distribution $v \in H^{s}$ such that $u=v$ in some neighborhood $\omega$ of $x$. (See Section 6.19.)

8.11 Theorem If $u \in \mathscr{D}^{\prime}(\Omega)$ and $-\infty<s<\infty$, the following two statements are equivalent:

(a) $u$ is locally $H^{s}$.

(b) $\psi u \in H^{s}$ for every $\psi \in \mathscr{D}(\Omega)$.

Moreover, if $s$ is a nonnegative integer, (a) and (b) are equivalent to

(c) $D_{\alpha} u$ is locally $L^{2}$ for every $\alpha$ with $|\alpha| \leq s$.

Statement (b) may need some clarification, since $u$ acts only on test functions whose supports lie in $\Omega$. However, $\psi u$ is the functional that assigns to each $\phi \in \mathscr{D}\left(R^{n}\right)$ the number

$$
(\psi u)(\phi)=u(\psi \phi)
$$

Note that $\psi \phi \in \mathscr{D}(\Omega)$, so that $u(\psi \phi)$ is defined.

PROOF Assume $u$ is locally $H^{s}$. Let $K$ be the support of some $\psi \in \mathscr{D}(\Omega)$. Since $K$ is compact, there are finitely many open sets $\omega_{i} \subset \Omega$, whose union covers $K$, and in which $u$ coincides with some $v_{i} \in H^{s}$. There exist functions $\psi_{i} \in \mathscr{D}\left(\omega_{i}\right)$ such that $\sum \psi_{i}=1$ on $K$. If $\phi \in \mathscr{D}\left(R^{n}\right)$ it follows that

$$
u(\psi \phi \phi)=\sum u\left(\psi_{i} \psi \phi\right)=\sum v_{i}\left(\psi_{i} \psi \phi\right)
$$

since $\psi_{i} \psi \phi \in \mathscr{D}\left(\omega_{i}\right)$. Thus $\psi u=\sum \psi_{i} \psi v_{i}$. By $(e)$ of Theorem 8.9, $\psi_{i} \psi v_{i} \in H^{s}$ for each $i$. Thus $\psi u \in H^{s}$, and (a) implies (b).

If $(b)$ holds, if $x \in \Omega$, and if $\psi \in \mathscr{D}(\Omega)$ is 1 in a neighborhood $\omega$ of $x$, then $u=\psi u$ in $\omega$, and $\psi u \in H^{s}$ by assumption. Thus $(b)$ implies $(a)$.

Assume again that $(b)$ holds. If $\psi \in \mathscr{D}(\Omega)$, then $\psi u \in H^{s}$, hence $D_{\alpha}(\psi u) \in$ $H^{s-|\alpha|}$, by $(d)$ of Theorem 8.9. If $|\alpha| \leq s$, then

$$
H^{s-|\alpha|} \subset H^{0}=L^{2}\left(R^{n}\right) .
$$

Thus $D_{\alpha}(\psi u) \in L^{2}\left(R^{n}\right)$. Taking $\psi=1$ in some neighborhood of a point $x \in \Omega$ shows that $D_{\alpha} u$ is locally $L^{2}$ in $\Omega$. Thus $(b)$ implies (c).

Finally, assume $D_{\alpha} u$ is locally $L^{2}$ for every $\alpha$ with $|\alpha| \leq s$. Fix $\psi \in \mathscr{D}(\Omega)$. The Leibniz formula shows that $D_{\alpha}(\psi u) \in L^{2}\left(R^{n}\right)$ if $|\alpha| \leq s$. Hence

$$
\int_{R^{n}}\left|y^{\alpha}\right|^{2}\left|(\psi u)^{\wedge}(y)\right|^{2} d m_{n}(y)<\infty \quad(|\alpha| \leq s)
$$

If $s$ is a nonnegative integer, (1) holds with the monomials $y_{1}^{s}, \ldots, y_{n}^{s}$ in place of $y^{\alpha}$. It follows, as in the proof of Theorem 7.25, that

$$
\int_{R^{n}}\left(1+|y|^{2}\right)^{s}\left|(\psi u)^{\wedge}(y)\right|^{2} d m_{n}(y)<\infty
$$

Thus $\psi u \in H^{s},(c)$ implies (b), and the proof is complete.

8.12 Theorem Assume $\Omega$ is an open set in $R^{n}$, and

(a) $L=\sum f_{\alpha} D_{\alpha}$ is a linear elliptic differential operator in $\Omega$, of order $N \geq 1$, with coefficients $f_{\alpha} \in C^{\infty}(\Omega)$,

(b) for each $\alpha$ with $|\alpha|=N, f_{\alpha}$ is a constant,

(c) $u$ and $v$ are distributions in $\Omega$ that satisfy

$$
L u=v
$$

and $v$ is locally $H^{s}$.

Then $u$ is locally $H^{s+N}$.

Corollary If $L$ satisfies $(a)$ and $(b)$ and if $v \in C^{\infty}(\Omega)$, then every solution $u$ of (1) belongs to $C^{\infty}(\Omega)$. In particular, every solution of the homogeneous equation $L u=0$ is in $C^{\infty}(\Omega)$.

For if $v \in C^{\infty}(\Omega)$, then $\psi v \in \mathscr{D}\left(R^{n}\right)$ for every $\psi \in \mathscr{D}(\Omega)$; hence $v$ is locally $H^{s}$ for every $s$, and the theorem implies that $u$ is locally $H^{s}$ for every $s$; it follows from Theorems 8.11 and 7.25 that $u \in C^{\infty}(\Omega)$.

Assumption (b) can be dropped from the theorem, but its presence makes the proof considerably easier.

PROOF Fix a point $x \in \Omega$, let $B_{0} \subset \Omega$ be a closed ball with center at $x$, and let $\phi_{0} \in \mathscr{D}(\Omega)$ be 1 on some open set containing $B_{0}$. By $(a)$ of Theorem 8.9, $\phi_{0} u \in H^{t}$ for some $t$. Since $H^{t}$ becomes larger as $t$ decreases, we may assume that $t=$ $s+N-k$, where $k$ is a positive integer. Choose closed balls

$$
\bar{B}_{0} \supset \bar{B}_{1} \supset \cdots \supset \bar{B}_{k}
$$

each centered at $x$, and each properly contained in the preceding one. Choose $\phi_{1}, \ldots, \phi_{k} \in \mathscr{D}(\Omega)$ so that $\phi_{i}=1$ on some open set containing $B_{i}$, and $\phi_{i}=0$ off $B_{i-1}$. Since $\phi_{0} u \in H^{t}$, the following "bootstrap" proposition implies that

$$
\phi_{1} u \in H^{i+1}, \ldots, \phi_{k} u \in H^{i+k}
$$

It therefore leads to the conclusion that $u$ is locally $H^{s+N}$, because $t+k=s+N$ and $\phi_{k}=1$ on $B_{k}$.

Proposition If, in addition to the hypotheses of Theorem 8.12, $\psi u \in H^{t}$ for some $t \leq s+N-1$ and for some $\psi \in \mathscr{D}(\Omega)$ which is 1 on an open set containing the support of a function $\phi \in \mathscr{\mathscr { D }}(\Omega)$, then $\phi u \in \bar{H}^{\hat{i}+1}$.

PROOF We begin by showing that

$$
L(\phi u) \in H^{t-N+1}
$$

Consider the distribution

$$
\Lambda=L(\phi u)-\phi L u=L(\phi u)-\phi v .
$$

Since its support lies in the support of $\phi, u$ can be replaced by $\psi u$ in (3), without changing $\Lambda$ :

$$
\Lambda=L(\phi \psi u)-\phi L(\psi u)=\sum_{|\alpha| \leq N} f_{\alpha} \cdot\left[D_{\alpha}(\phi \psi u)-\phi D_{\alpha}(\psi u)\right]
$$

If the Leibniz formula is applied to $D_{\alpha}(\phi \cdot \psi u)$, one sees that the derivatives of order $N$ of $\psi u$ cancel in (4). Therefore $\Lambda$ is a linear combination [with coefficients in $\mathscr{D}\left(R^{n}\right)$ ] of derivatives of $\psi u$, of orders at most $N-1$. Since $\psi u \in \boldsymbol{H}^{\boldsymbol{t}}$, parts $(d)$ and $(e)$ of Theorems 8.9 imply that $\Lambda \in H^{t-N+1}$. By Theorem 8.11, $\phi v \in H^{s}$, and since $t-N+1 \leq s$, we have $\phi v \in H^{t-N+1}$. Now (2) follows from (3).

Since $L$ is elliptic, its characteristic polynomial

$$
p(y)=\sum_{|\alpha|=N} f_{\alpha} y^{\alpha} \quad\left(y \in R^{n}\right)
$$

has no zero in $R^{n}$, except at $y=0$. Define functions

$$
q(y)=|y|^{-N} p(y), \quad r(y)=\left(1+|y|^{N}\right) q(y)
$$

for $y \in R^{n}, y \neq 0$; and define operators $Q, R, S$ on the union of the Sobolev spaces by

$$
(Q w)^{\wedge}=q \hat{w}, \quad(R w)^{\wedge}=r \hat{w}
$$

and

$$
S=\sum_{|\alpha|<N} \psi f_{\alpha} D_{\alpha}
$$

Since $p$ is a homogeneous polynomial of degree $N, q(\lambda y)=q(y)$ if $\lambda>0$, and since $p$ vanishes only at the origin, the compactness of the unit sphere in $R^{n}$ implies that both $q$ and $1 / q$ are bounded functions. It follows from $(c)$ of Theorem 8.9 that boin $Q$ and $Q^{-1}$ are operators of order 0 .

$\therefore \quad$ Since both $\left(1+|y|^{2}\right)^{-N / 2}\left(1+|y|^{N}\right)$ and its reciprocal are bounded functions on $R^{n}$, it follows from the preceding paragraph, combined with $(b)$ and (c) of Theorem 8.9, that $R$ is an operator of order $N$ whose inverse $R^{-1}$ has order $-N$.

Since $\psi f_{\alpha} \in \mathscr{D}\left(R^{n}\right)$ it follows from (d) and (e) of Theorem 8.9 that $S$ is an operator of order $N-1$.

have

Since $p=r-q$, and since $p$ is assumed to have constant coefficients $f_{\alpha}$, we

$$
\left(\sum_{|\alpha|=N} f_{\alpha} D_{\alpha} w\right) \wedge=p \hat{w}=(r-q) \hat{w}=(R w-Q w)^{\wedge}
$$

if $w$ lies in some Sobolev space. Hence

$$
(R-Q+S)(\phi u)=L(\phi u)
$$

By (2), $L(\phi u) \in H^{t-N+1}$. Hence

Since $\psi u \in H^{t}$ and $\phi \psi=\phi,(e)$ of Theorem 8.9 implies that $\phi u=\phi \psi u \in H^{t}$.

$$
(Q-S)(\phi u) \in H^{t-N+1}
$$

because $Q$ has order 0 and $S$ has order $N-1 \geq 0$. It now follows from (10) that

$$
R(\phi u) \in H^{t-N+1}
$$

and since $R^{-1}$ has order $-N$, we finally conclude that $\phi u \in H^{t+1}$.

8.13 Example Suppose $L$ is an elliptic differential operator in $R^{n}$, with constant coefficients, and $E$ is a fundamental solution of $L$. In the complement of the origin, the equation $L E=\delta$ reduces to $L E=0$. Theorem 8.12 implies therefore that, except at the origin, $E$ is an infinitely differentiable function. The nature of the singularity of $E$ at the origin depends, of course, on $L$.

8.14 Example The origin in $R^{2}$ is the only zero of the polynomial $p(y)=y_{1}+i y_{2}$. If $\Omega$ is open in $R^{2}$, and if $u \in \mathscr{D}^{\prime}(\Omega)$ is a distribution solution of the Cauchy-Riemann equation

$$
\left(\frac{\partial}{\partial x_{1}}+i \frac{\partial}{\partial x_{2}}\right) u=0
$$

Theorem 8.12 implies that $u \in C^{\infty}(\Omega)$. It follows that $u$ is a holomorphic function of $z=x_{1}+i x_{2}$ in $\Omega$. In other words, every holomorphic distribution is a holomorphic function.


\end{document}